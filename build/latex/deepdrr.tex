%% Generated by Sphinx.
\def\sphinxdocclass{report}
\documentclass[letterpaper,10pt,english]{sphinxmanual}
\ifdefined\pdfpxdimen
   \let\sphinxpxdimen\pdfpxdimen\else\newdimen\sphinxpxdimen
\fi \sphinxpxdimen=.75bp\relax
\ifdefined\pdfimageresolution
    \pdfimageresolution= \numexpr \dimexpr1in\relax/\sphinxpxdimen\relax
\fi
%% let collapsible pdf bookmarks panel have high depth per default
\PassOptionsToPackage{bookmarksdepth=5}{hyperref}

\PassOptionsToPackage{booktabs}{sphinx}
\PassOptionsToPackage{colorrows}{sphinx}

\PassOptionsToPackage{warn}{textcomp}
\usepackage[utf8]{inputenc}
\ifdefined\DeclareUnicodeCharacter
% support both utf8 and utf8x syntaxes
  \ifdefined\DeclareUnicodeCharacterAsOptional
    \def\sphinxDUC#1{\DeclareUnicodeCharacter{"#1}}
  \else
    \let\sphinxDUC\DeclareUnicodeCharacter
  \fi
  \sphinxDUC{00A0}{\nobreakspace}
  \sphinxDUC{2500}{\sphinxunichar{2500}}
  \sphinxDUC{2502}{\sphinxunichar{2502}}
  \sphinxDUC{2514}{\sphinxunichar{2514}}
  \sphinxDUC{251C}{\sphinxunichar{251C}}
  \sphinxDUC{2572}{\textbackslash}
\fi
\usepackage{cmap}
\usepackage[T1]{fontenc}
\usepackage{amsmath,amssymb,amstext}
\usepackage{babel}



\usepackage{tgtermes}
\usepackage{tgheros}
\renewcommand{\ttdefault}{txtt}



\usepackage[Bjarne]{fncychap}
\usepackage{sphinx}

\fvset{fontsize=auto}
\usepackage{geometry}


% Include hyperref last.
\usepackage{hyperref}
% Fix anchor placement for figures with captions.
\usepackage{hypcap}% it must be loaded after hyperref.
% Set up styles of URL: it should be placed after hyperref.
\urlstyle{same}

\addto\captionsenglish{\renewcommand{\contentsname}{Contents:}}

\usepackage{sphinxmessages}
\setcounter{tocdepth}{2}



\title{DeepDRR函数调用文档}
\date{Oct 08, 2023}
\release{1.1.0a3}
\author{无}
\newcommand{\sphinxlogo}{\vbox{}}
\renewcommand{\releasename}{Release}
\makeindex
\begin{document}

\ifdefined\shorthandoff
  \ifnum\catcode`\=\string=\active\shorthandoff{=}\fi
  \ifnum\catcode`\"=\active\shorthandoff{"}\fi
\fi

\pagestyle{empty}
\sphinxmaketitle
\pagestyle{plain}
\sphinxtableofcontents
\pagestyle{normal}
\phantomsection\label{\detokenize{index::doc}}


\sphinxstepscope

\sphinxAtStartPar
../../README.md

\sphinxstepscope


\chapter{deepdrr package}
\label{\detokenize{deepdrr:deepdrr-package}}\label{\detokenize{deepdrr::doc}}
\sphinxstepscope


\section{deepdrr.annotations package}
\label{\detokenize{deepdrr.annotations:deepdrr-annotations-package}}\label{\detokenize{deepdrr.annotations::doc}}

\subsection{deepdrr.annotations.fiducials}
\label{\detokenize{deepdrr.annotations:module-deepdrr.annotations.fiducials}}\label{\detokenize{deepdrr.annotations:deepdrr-annotations-fiducials}}\index{module@\spxentry{module}!deepdrr.annotations.fiducials@\spxentry{deepdrr.annotations.fiducials}}\index{deepdrr.annotations.fiducials@\spxentry{deepdrr.annotations.fiducials}!module@\spxentry{module}}\index{Fiducial (class in deepdrr.annotations.fiducials)@\spxentry{Fiducial}\spxextra{class in deepdrr.annotations.fiducials}}

\begin{fulllineitems}
\phantomsection\label{\detokenize{deepdrr.annotations:deepdrr.annotations.fiducials.Fiducial}}
\pysigstartsignatures
\pysiglinewithargsret{\sphinxbfcode{\sphinxupquote{class\DUrole{w,w}{  }}}\sphinxcode{\sphinxupquote{deepdrr.annotations.fiducials.}}\sphinxbfcode{\sphinxupquote{Fiducial}}}{\sphinxparam{\DUrole{n,n}{data}\DUrole{p,p}{:}\DUrole{w,w}{  }\DUrole{n,n}{ndarray}}}{}
\pysigstopsignatures
\sphinxAtStartPar
Bases: {\hyperref[\detokenize{deepdrr.geo:deepdrr.geo.core.Point3D}]{\sphinxcrossref{\sphinxcode{\sphinxupquote{Point3D}}}}}
\index{data (deepdrr.annotations.fiducials.Fiducial attribute)@\spxentry{data}\spxextra{deepdrr.annotations.fiducials.Fiducial attribute}}

\begin{fulllineitems}
\phantomsection\label{\detokenize{deepdrr.annotations:deepdrr.annotations.fiducials.Fiducial.data}}
\pysigstartsignatures
\pysigline{\sphinxbfcode{\sphinxupquote{data}}\sphinxbfcode{\sphinxupquote{\DUrole{p,p}{:}\DUrole{w,w}{  }ndarray}}}
\pysigstopsignatures
\end{fulllineitems}

\index{from\_fcsv() (deepdrr.annotations.fiducials.Fiducial class method)@\spxentry{from\_fcsv()}\spxextra{deepdrr.annotations.fiducials.Fiducial class method}}

\begin{fulllineitems}
\phantomsection\label{\detokenize{deepdrr.annotations:deepdrr.annotations.fiducials.Fiducial.from_fcsv}}
\pysigstartsignatures
\pysiglinewithargsret{\sphinxbfcode{\sphinxupquote{classmethod\DUrole{w,w}{  }}}\sphinxbfcode{\sphinxupquote{from\_fcsv}}}{\sphinxparam{\DUrole{n,n}{path}\DUrole{p,p}{:}\DUrole{w,w}{  }\DUrole{n,n}{Path}}\sphinxparamcomma \sphinxparam{\DUrole{n,n}{world\_from\_anatomical}\DUrole{p,p}{:}\DUrole{w,w}{  }\DUrole{n,n}{{\hyperref[\detokenize{deepdrr.geo:deepdrr.geo.core.FrameTransform}]{\sphinxcrossref{FrameTransform}}}\DUrole{w,w}{  }\DUrole{p,p}{|}\DUrole{w,w}{  }None}\DUrole{w,w}{  }\DUrole{o,o}{=}\DUrole{w,w}{  }\DUrole{default_value}{None}}}{}
\pysigstopsignatures
\end{fulllineitems}

\index{from\_json() (deepdrr.annotations.fiducials.Fiducial class method)@\spxentry{from\_json()}\spxextra{deepdrr.annotations.fiducials.Fiducial class method}}

\begin{fulllineitems}
\phantomsection\label{\detokenize{deepdrr.annotations:deepdrr.annotations.fiducials.Fiducial.from_json}}
\pysigstartsignatures
\pysiglinewithargsret{\sphinxbfcode{\sphinxupquote{classmethod\DUrole{w,w}{  }}}\sphinxbfcode{\sphinxupquote{from\_json}}}{\sphinxparam{\DUrole{n,n}{path}\DUrole{p,p}{:}\DUrole{w,w}{  }\DUrole{n,n}{Path}}\sphinxparamcomma \sphinxparam{\DUrole{n,n}{world\_from\_anatomical}\DUrole{p,p}{:}\DUrole{w,w}{  }\DUrole{n,n}{{\hyperref[\detokenize{deepdrr.geo:deepdrr.geo.core.FrameTransform}]{\sphinxcrossref{FrameTransform}}}\DUrole{w,w}{  }\DUrole{p,p}{|}\DUrole{w,w}{  }None}\DUrole{w,w}{  }\DUrole{o,o}{=}\DUrole{w,w}{  }\DUrole{default_value}{None}}}{}
\pysigstopsignatures
\end{fulllineitems}

\index{save() (deepdrr.annotations.fiducials.Fiducial method)@\spxentry{save()}\spxextra{deepdrr.annotations.fiducials.Fiducial method}}

\begin{fulllineitems}
\phantomsection\label{\detokenize{deepdrr.annotations:deepdrr.annotations.fiducials.Fiducial.save}}
\pysigstartsignatures
\pysiglinewithargsret{\sphinxbfcode{\sphinxupquote{save}}}{\sphinxparam{\DUrole{n,n}{path}\DUrole{p,p}{:}\DUrole{w,w}{  }\DUrole{n,n}{Path}}}{}
\pysigstopsignatures
\end{fulllineitems}


\end{fulllineitems}

\index{FiducialList (class in deepdrr.annotations.fiducials)@\spxentry{FiducialList}\spxextra{class in deepdrr.annotations.fiducials}}

\begin{fulllineitems}
\phantomsection\label{\detokenize{deepdrr.annotations:deepdrr.annotations.fiducials.FiducialList}}
\pysigstartsignatures
\pysiglinewithargsret{\sphinxbfcode{\sphinxupquote{class\DUrole{w,w}{  }}}\sphinxcode{\sphinxupquote{deepdrr.annotations.fiducials.}}\sphinxbfcode{\sphinxupquote{FiducialList}}}{\sphinxparam{\DUrole{n,n}{points}\DUrole{p,p}{:}\DUrole{w,w}{  }\DUrole{n,n}{List\DUrole{p,p}{{[}}{\hyperref[\detokenize{deepdrr.geo:deepdrr.geo.core.Point3D}]{\sphinxcrossref{Point3D}}}\DUrole{p,p}{{]}}}}\sphinxparamcomma \sphinxparam{\DUrole{n,n}{world\_from\_anatomical}\DUrole{p,p}{:}\DUrole{w,w}{  }\DUrole{n,n}{{\hyperref[\detokenize{deepdrr.geo:deepdrr.geo.core.FrameTransform}]{\sphinxcrossref{FrameTransform}}}\DUrole{w,w}{  }\DUrole{p,p}{|}\DUrole{w,w}{  }None}\DUrole{w,w}{  }\DUrole{o,o}{=}\DUrole{w,w}{  }\DUrole{default_value}{None}}\sphinxparamcomma \sphinxparam{\DUrole{n,n}{anatomical\_coordinate\_system}\DUrole{p,p}{:}\DUrole{w,w}{  }\DUrole{n,n}{Literal\DUrole{p,p}{{[}}\DUrole{s,s}{\textquotesingle{}RAS\textquotesingle{}}\DUrole{p,p}{,}\DUrole{w,w}{  }\DUrole{s,s}{\textquotesingle{}LPS\textquotesingle{}}\DUrole{p,p}{{]}}}\DUrole{w,w}{  }\DUrole{o,o}{=}\DUrole{w,w}{  }\DUrole{default_value}{\textquotesingle{}RAS\textquotesingle{}}}}{}
\pysigstopsignatures
\sphinxAtStartPar
Bases: \sphinxcode{\sphinxupquote{object}}
\index{from\_fcsv() (deepdrr.annotations.fiducials.FiducialList class method)@\spxentry{from\_fcsv()}\spxextra{deepdrr.annotations.fiducials.FiducialList class method}}

\begin{fulllineitems}
\phantomsection\label{\detokenize{deepdrr.annotations:deepdrr.annotations.fiducials.FiducialList.from_fcsv}}
\pysigstartsignatures
\pysiglinewithargsret{\sphinxbfcode{\sphinxupquote{classmethod\DUrole{w,w}{  }}}\sphinxbfcode{\sphinxupquote{from\_fcsv}}}{\sphinxparam{\DUrole{n,n}{path}\DUrole{p,p}{:}\DUrole{w,w}{  }\DUrole{n,n}{Path}}\sphinxparamcomma \sphinxparam{\DUrole{n,n}{world\_from\_anatomical}\DUrole{p,p}{:}\DUrole{w,w}{  }\DUrole{n,n}{{\hyperref[\detokenize{deepdrr.geo:deepdrr.geo.core.FrameTransform}]{\sphinxcrossref{FrameTransform}}}\DUrole{w,w}{  }\DUrole{p,p}{|}\DUrole{w,w}{  }None}\DUrole{w,w}{  }\DUrole{o,o}{=}\DUrole{w,w}{  }\DUrole{default_value}{None}}}{{ $\rightarrow$ {\hyperref[\detokenize{deepdrr.annotations:deepdrr.annotations.fiducials.FiducialList}]{\sphinxcrossref{FiducialList}}}}}
\pysigstopsignatures
\sphinxAtStartPar
Load a FCSV file from Slicer3D
\begin{quote}\begin{description}
\sphinxlineitem{Parameters}
\sphinxAtStartPar
\sphinxstyleliteralstrong{\sphinxupquote{path}} (\sphinxstyleliteralemphasis{\sphinxupquote{Path}}) \textendash{} Path to the FCSV file

\sphinxlineitem{Returns}
\sphinxAtStartPar
Array of 3D points

\sphinxlineitem{Return type}
\sphinxAtStartPar
np.ndarray

\end{description}\end{quote}

\end{fulllineitems}

\index{from\_json() (deepdrr.annotations.fiducials.FiducialList class method)@\spxentry{from\_json()}\spxextra{deepdrr.annotations.fiducials.FiducialList class method}}

\begin{fulllineitems}
\phantomsection\label{\detokenize{deepdrr.annotations:deepdrr.annotations.fiducials.FiducialList.from_json}}
\pysigstartsignatures
\pysiglinewithargsret{\sphinxbfcode{\sphinxupquote{classmethod\DUrole{w,w}{  }}}\sphinxbfcode{\sphinxupquote{from\_json}}}{\sphinxparam{\DUrole{n,n}{path}\DUrole{p,p}{:}\DUrole{w,w}{  }\DUrole{n,n}{Path}}\sphinxparamcomma \sphinxparam{\DUrole{n,n}{world\_from\_anatomical}\DUrole{p,p}{:}\DUrole{w,w}{  }\DUrole{n,n}{{\hyperref[\detokenize{deepdrr.geo:deepdrr.geo.core.FrameTransform}]{\sphinxcrossref{FrameTransform}}}\DUrole{w,w}{  }\DUrole{p,p}{|}\DUrole{w,w}{  }None}\DUrole{w,w}{  }\DUrole{o,o}{=}\DUrole{w,w}{  }\DUrole{default_value}{None}}}{}
\pysigstopsignatures
\end{fulllineitems}

\index{save() (deepdrr.annotations.fiducials.FiducialList method)@\spxentry{save()}\spxextra{deepdrr.annotations.fiducials.FiducialList method}}

\begin{fulllineitems}
\phantomsection\label{\detokenize{deepdrr.annotations:deepdrr.annotations.fiducials.FiducialList.save}}
\pysigstartsignatures
\pysiglinewithargsret{\sphinxbfcode{\sphinxupquote{save}}}{\sphinxparam{\DUrole{n,n}{path}\DUrole{p,p}{:}\DUrole{w,w}{  }\DUrole{n,n}{Path}}}{}
\pysigstopsignatures
\end{fulllineitems}

\index{to\_LPS() (deepdrr.annotations.fiducials.FiducialList method)@\spxentry{to\_LPS()}\spxextra{deepdrr.annotations.fiducials.FiducialList method}}

\begin{fulllineitems}
\phantomsection\label{\detokenize{deepdrr.annotations:deepdrr.annotations.fiducials.FiducialList.to_LPS}}
\pysigstartsignatures
\pysiglinewithargsret{\sphinxbfcode{\sphinxupquote{to\_LPS}}}{}{{ $\rightarrow$ {\hyperref[\detokenize{deepdrr.annotations:deepdrr.annotations.fiducials.FiducialList}]{\sphinxcrossref{FiducialList}}}}}
\pysigstopsignatures
\end{fulllineitems}

\index{to\_RAS() (deepdrr.annotations.fiducials.FiducialList method)@\spxentry{to\_RAS()}\spxextra{deepdrr.annotations.fiducials.FiducialList method}}

\begin{fulllineitems}
\phantomsection\label{\detokenize{deepdrr.annotations:deepdrr.annotations.fiducials.FiducialList.to_RAS}}
\pysigstartsignatures
\pysiglinewithargsret{\sphinxbfcode{\sphinxupquote{to\_RAS}}}{}{{ $\rightarrow$ {\hyperref[\detokenize{deepdrr.annotations:deepdrr.annotations.fiducials.FiducialList}]{\sphinxcrossref{FiducialList}}}}}
\pysigstopsignatures
\end{fulllineitems}


\end{fulllineitems}



\subsection{deepdrr.annotations.line\_annotation}
\label{\detokenize{deepdrr.annotations:module-deepdrr.annotations.line_annotation}}\label{\detokenize{deepdrr.annotations:deepdrr-annotations-line-annotation}}\index{module@\spxentry{module}!deepdrr.annotations.line\_annotation@\spxentry{deepdrr.annotations.line\_annotation}}\index{deepdrr.annotations.line\_annotation@\spxentry{deepdrr.annotations.line\_annotation}!module@\spxentry{module}}\index{LineAnnotation (class in deepdrr.annotations.line\_annotation)@\spxentry{LineAnnotation}\spxextra{class in deepdrr.annotations.line\_annotation}}

\begin{fulllineitems}
\phantomsection\label{\detokenize{deepdrr.annotations:deepdrr.annotations.line_annotation.LineAnnotation}}
\pysigstartsignatures
\pysiglinewithargsret{\sphinxbfcode{\sphinxupquote{class\DUrole{w,w}{  }}}\sphinxcode{\sphinxupquote{deepdrr.annotations.line\_annotation.}}\sphinxbfcode{\sphinxupquote{LineAnnotation}}}{\sphinxparam{\DUrole{n,n}{startpoint}\DUrole{p,p}{:}\DUrole{w,w}{  }\DUrole{n,n}{{\hyperref[\detokenize{deepdrr.geo:deepdrr.geo.core.Point3D}]{\sphinxcrossref{Point3D}}}}}\sphinxparamcomma \sphinxparam{\DUrole{n,n}{endpoint}\DUrole{p,p}{:}\DUrole{w,w}{  }\DUrole{n,n}{{\hyperref[\detokenize{deepdrr.geo:deepdrr.geo.core.Point3D}]{\sphinxcrossref{Point3D}}}}}\sphinxparamcomma \sphinxparam{\DUrole{n,n}{volume}\DUrole{p,p}{:}\DUrole{w,w}{  }\DUrole{n,n}{{\hyperref[\detokenize{deepdrr.vol:deepdrr.vol.volume.Volume}]{\sphinxcrossref{Volume}}}\DUrole{w,w}{  }\DUrole{p,p}{|}\DUrole{w,w}{  }None}\DUrole{w,w}{  }\DUrole{o,o}{=}\DUrole{w,w}{  }\DUrole{default_value}{None}}\sphinxparamcomma \sphinxparam{\DUrole{n,n}{world\_from\_anatomical}\DUrole{p,p}{:}\DUrole{w,w}{  }\DUrole{n,n}{{\hyperref[\detokenize{deepdrr.geo:deepdrr.geo.core.FrameTransform}]{\sphinxcrossref{FrameTransform}}}\DUrole{w,w}{  }\DUrole{p,p}{|}\DUrole{w,w}{  }None}\DUrole{w,w}{  }\DUrole{o,o}{=}\DUrole{w,w}{  }\DUrole{default_value}{None}}\sphinxparamcomma \sphinxparam{\DUrole{n,n}{anatomical\_coordinate\_system}\DUrole{p,p}{:}\DUrole{w,w}{  }\DUrole{n,n}{str\DUrole{w,w}{  }\DUrole{p,p}{|}\DUrole{w,w}{  }None}\DUrole{w,w}{  }\DUrole{o,o}{=}\DUrole{w,w}{  }\DUrole{default_value}{None}}}{}
\pysigstopsignatures
\sphinxAtStartPar
Bases: \sphinxcode{\sphinxupquote{object}}

\sphinxAtStartPar
Really a “segment annotation”, but Slicer calls it a line.
\index{startpoint (deepdrr.annotations.line\_annotation.LineAnnotation attribute)@\spxentry{startpoint}\spxextra{deepdrr.annotations.line\_annotation.LineAnnotation attribute}}

\begin{fulllineitems}
\phantomsection\label{\detokenize{deepdrr.annotations:deepdrr.annotations.line_annotation.LineAnnotation.startpoint}}
\pysigstartsignatures
\pysigline{\sphinxbfcode{\sphinxupquote{startpoint}}}
\pysigstopsignatures
\sphinxAtStartPar
The startpoint in anatomical coordinates.
\begin{quote}\begin{description}
\sphinxlineitem{Type}
\sphinxAtStartPar
{\hyperref[\detokenize{deepdrr.geo:deepdrr.geo.Point}]{\sphinxcrossref{geo.Point}}}

\end{description}\end{quote}

\end{fulllineitems}

\index{endpoint (deepdrr.annotations.line\_annotation.LineAnnotation attribute)@\spxentry{endpoint}\spxextra{deepdrr.annotations.line\_annotation.LineAnnotation attribute}}

\begin{fulllineitems}
\phantomsection\label{\detokenize{deepdrr.annotations:deepdrr.annotations.line_annotation.LineAnnotation.endpoint}}
\pysigstartsignatures
\pysigline{\sphinxbfcode{\sphinxupquote{endpoint}}}
\pysigstopsignatures
\sphinxAtStartPar
The endpoint in anatomical coordinates.
\begin{quote}\begin{description}
\sphinxlineitem{Type}
\sphinxAtStartPar
{\hyperref[\detokenize{deepdrr.geo:deepdrr.geo.Point}]{\sphinxcrossref{geo.Point}}}

\end{description}\end{quote}

\end{fulllineitems}

\index{anatomical\_coordinate\_system (deepdrr.annotations.line\_annotation.LineAnnotation property)@\spxentry{anatomical\_coordinate\_system}\spxextra{deepdrr.annotations.line\_annotation.LineAnnotation property}}

\begin{fulllineitems}
\phantomsection\label{\detokenize{deepdrr.annotations:deepdrr.annotations.line_annotation.LineAnnotation.anatomical_coordinate_system}}
\pysigstartsignatures
\pysigline{\sphinxbfcode{\sphinxupquote{property\DUrole{w,w}{  }}}\sphinxbfcode{\sphinxupquote{anatomical\_coordinate\_system}}\sphinxbfcode{\sphinxupquote{\DUrole{p,p}{:}\DUrole{w,w}{  }str}}}
\pysigstopsignatures
\end{fulllineitems}

\index{anatomical\_from\_world (deepdrr.annotations.line\_annotation.LineAnnotation property)@\spxentry{anatomical\_from\_world}\spxextra{deepdrr.annotations.line\_annotation.LineAnnotation property}}

\begin{fulllineitems}
\phantomsection\label{\detokenize{deepdrr.annotations:deepdrr.annotations.line_annotation.LineAnnotation.anatomical_from_world}}
\pysigstartsignatures
\pysigline{\sphinxbfcode{\sphinxupquote{property\DUrole{w,w}{  }}}\sphinxbfcode{\sphinxupquote{anatomical\_from\_world}}\sphinxbfcode{\sphinxupquote{\DUrole{p,p}{:}\DUrole{w,w}{  }{\hyperref[\detokenize{deepdrr.geo:deepdrr.geo.core.FrameTransform}]{\sphinxcrossref{FrameTransform}}}}}}
\pysigstopsignatures
\end{fulllineitems}

\index{direction\_in\_world (deepdrr.annotations.line\_annotation.LineAnnotation property)@\spxentry{direction\_in\_world}\spxextra{deepdrr.annotations.line\_annotation.LineAnnotation property}}

\begin{fulllineitems}
\phantomsection\label{\detokenize{deepdrr.annotations:deepdrr.annotations.line_annotation.LineAnnotation.direction_in_world}}
\pysigstartsignatures
\pysigline{\sphinxbfcode{\sphinxupquote{property\DUrole{w,w}{  }}}\sphinxbfcode{\sphinxupquote{direction\_in\_world}}\sphinxbfcode{\sphinxupquote{\DUrole{p,p}{:}\DUrole{w,w}{  }{\hyperref[\detokenize{deepdrr.geo:deepdrr.geo.core.Vector3D}]{\sphinxcrossref{Vector3D}}}}}}
\pysigstopsignatures
\end{fulllineitems}

\index{endpoint\_in\_world (deepdrr.annotations.line\_annotation.LineAnnotation property)@\spxentry{endpoint\_in\_world}\spxextra{deepdrr.annotations.line\_annotation.LineAnnotation property}}

\begin{fulllineitems}
\phantomsection\label{\detokenize{deepdrr.annotations:deepdrr.annotations.line_annotation.LineAnnotation.endpoint_in_world}}
\pysigstartsignatures
\pysigline{\sphinxbfcode{\sphinxupquote{property\DUrole{w,w}{  }}}\sphinxbfcode{\sphinxupquote{endpoint\_in\_world}}\sphinxbfcode{\sphinxupquote{\DUrole{p,p}{:}\DUrole{w,w}{  }{\hyperref[\detokenize{deepdrr.geo:deepdrr.geo.core.Point3D}]{\sphinxcrossref{Point3D}}}}}}
\pysigstopsignatures
\end{fulllineitems}

\index{from\_fcsv() (deepdrr.annotations.line\_annotation.LineAnnotation class method)@\spxentry{from\_fcsv()}\spxextra{deepdrr.annotations.line\_annotation.LineAnnotation class method}}

\begin{fulllineitems}
\phantomsection\label{\detokenize{deepdrr.annotations:deepdrr.annotations.line_annotation.LineAnnotation.from_fcsv}}
\pysigstartsignatures
\pysiglinewithargsret{\sphinxbfcode{\sphinxupquote{classmethod\DUrole{w,w}{  }}}\sphinxbfcode{\sphinxupquote{from\_fcsv}}}{\sphinxparam{\DUrole{n,n}{path}\DUrole{p,p}{:}\DUrole{w,w}{  }\DUrole{n,n}{Path}}\sphinxparamcomma \sphinxparam{\DUrole{o,o}{**}\DUrole{n,n}{kwargs}}}{{ $\rightarrow$ {\hyperref[\detokenize{deepdrr.annotations:deepdrr.annotations.line_annotation.LineAnnotation}]{\sphinxcrossref{LineAnnotation}}}}}
\pysigstopsignatures
\sphinxAtStartPar
Load a LineAnnotation from a .fcsv file.
\begin{quote}\begin{description}
\sphinxlineitem{Parameters}\begin{itemize}
\item {} 
\sphinxAtStartPar
\sphinxstyleliteralstrong{\sphinxupquote{path}} (\sphinxstyleliteralemphasis{\sphinxupquote{Path}}) \textendash{} Path to the .fcsv file.

\item {} 
\sphinxAtStartPar
\sphinxstyleliteralstrong{\sphinxupquote{world\_from\_anatomical}} (\sphinxstyleliteralemphasis{\sphinxupquote{Optional}}\sphinxstyleliteralemphasis{\sphinxupquote{{[}}}{\hyperref[\detokenize{deepdrr.geo:deepdrr.geo.FrameTransform}]{\sphinxcrossref{\sphinxstyleliteralemphasis{\sphinxupquote{geo.FrameTransform}}}}}\sphinxstyleliteralemphasis{\sphinxupquote{{]}}}\sphinxstyleliteralemphasis{\sphinxupquote{, }}\sphinxstyleliteralemphasis{\sphinxupquote{optional}}) \textendash{} The pose of the volume in world coordinates. Defaults to None.

\end{itemize}

\sphinxlineitem{Returns}
\sphinxAtStartPar
The loaded annotation.

\sphinxlineitem{Return type}
\sphinxAtStartPar
{\hyperref[\detokenize{deepdrr.annotations:deepdrr.annotations.line_annotation.LineAnnotation}]{\sphinxcrossref{LineAnnotation}}}

\end{description}\end{quote}

\end{fulllineitems}

\index{from\_json() (deepdrr.annotations.line\_annotation.LineAnnotation class method)@\spxentry{from\_json()}\spxextra{deepdrr.annotations.line\_annotation.LineAnnotation class method}}

\begin{fulllineitems}
\phantomsection\label{\detokenize{deepdrr.annotations:deepdrr.annotations.line_annotation.LineAnnotation.from_json}}
\pysigstartsignatures
\pysiglinewithargsret{\sphinxbfcode{\sphinxupquote{classmethod\DUrole{w,w}{  }}}\sphinxbfcode{\sphinxupquote{from\_json}}}{\sphinxparam{\DUrole{n,n}{path}\DUrole{p,p}{:}\DUrole{w,w}{  }\DUrole{n,n}{str}}\sphinxparamcomma \sphinxparam{\DUrole{n,n}{volume}\DUrole{p,p}{:}\DUrole{w,w}{  }\DUrole{n,n}{{\hyperref[\detokenize{deepdrr.vol:deepdrr.vol.volume.Volume}]{\sphinxcrossref{Volume}}}\DUrole{w,w}{  }\DUrole{p,p}{|}\DUrole{w,w}{  }None}\DUrole{w,w}{  }\DUrole{o,o}{=}\DUrole{w,w}{  }\DUrole{default_value}{None}}\sphinxparamcomma \sphinxparam{\DUrole{n,n}{world\_from\_anatomical}\DUrole{p,p}{:}\DUrole{w,w}{  }\DUrole{n,n}{{\hyperref[\detokenize{deepdrr.geo:deepdrr.geo.core.FrameTransform}]{\sphinxcrossref{FrameTransform}}}\DUrole{w,w}{  }\DUrole{p,p}{|}\DUrole{w,w}{  }None}\DUrole{w,w}{  }\DUrole{o,o}{=}\DUrole{w,w}{  }\DUrole{default_value}{None}}\sphinxparamcomma \sphinxparam{\DUrole{n,n}{anatomical\_coordinate\_system}\DUrole{p,p}{:}\DUrole{w,w}{  }\DUrole{n,n}{str\DUrole{w,w}{  }\DUrole{p,p}{|}\DUrole{w,w}{  }None}\DUrole{w,w}{  }\DUrole{o,o}{=}\DUrole{w,w}{  }\DUrole{default_value}{None}}}{{ $\rightarrow$ {\hyperref[\detokenize{deepdrr.annotations:deepdrr.annotations.line_annotation.LineAnnotation}]{\sphinxcrossref{LineAnnotation}}}}}
\pysigstopsignatures
\end{fulllineitems}

\index{from\_markup() (deepdrr.annotations.line\_annotation.LineAnnotation class method)@\spxentry{from\_markup()}\spxextra{deepdrr.annotations.line\_annotation.LineAnnotation class method}}

\begin{fulllineitems}
\phantomsection\label{\detokenize{deepdrr.annotations:deepdrr.annotations.line_annotation.LineAnnotation.from_markup}}
\pysigstartsignatures
\pysiglinewithargsret{\sphinxbfcode{\sphinxupquote{classmethod\DUrole{w,w}{  }}}\sphinxbfcode{\sphinxupquote{from\_markup}}}{\sphinxparam{\DUrole{o,o}{*}\DUrole{n,n}{args}}\sphinxparamcomma \sphinxparam{\DUrole{o,o}{**}\DUrole{n,n}{kwargs}}}{}
\pysigstopsignatures
\end{fulllineitems}

\index{get\_direction() (deepdrr.annotations.line\_annotation.LineAnnotation method)@\spxentry{get\_direction()}\spxextra{deepdrr.annotations.line\_annotation.LineAnnotation method}}

\begin{fulllineitems}
\phantomsection\label{\detokenize{deepdrr.annotations:deepdrr.annotations.line_annotation.LineAnnotation.get_direction}}
\pysigstartsignatures
\pysiglinewithargsret{\sphinxbfcode{\sphinxupquote{get\_direction}}}{}{{ $\rightarrow$ {\hyperref[\detokenize{deepdrr.geo:deepdrr.geo.core.Vector3D}]{\sphinxcrossref{Vector3D}}}}}
\pysigstopsignatures
\end{fulllineitems}

\index{get\_mesh() (deepdrr.annotations.line\_annotation.LineAnnotation method)@\spxentry{get\_mesh()}\spxextra{deepdrr.annotations.line\_annotation.LineAnnotation method}}

\begin{fulllineitems}
\phantomsection\label{\detokenize{deepdrr.annotations:deepdrr.annotations.line_annotation.LineAnnotation.get_mesh}}
\pysigstartsignatures
\pysiglinewithargsret{\sphinxbfcode{\sphinxupquote{get\_mesh}}}{}{}
\pysigstopsignatures
\sphinxAtStartPar
Get the mesh in anatomical coordinates.

\end{fulllineitems}

\index{get\_mesh\_in\_world() (deepdrr.annotations.line\_annotation.LineAnnotation method)@\spxentry{get\_mesh\_in\_world()}\spxextra{deepdrr.annotations.line\_annotation.LineAnnotation method}}

\begin{fulllineitems}
\phantomsection\label{\detokenize{deepdrr.annotations:deepdrr.annotations.line_annotation.LineAnnotation.get_mesh_in_world}}
\pysigstartsignatures
\pysiglinewithargsret{\sphinxbfcode{\sphinxupquote{get\_mesh\_in\_world}}}{\sphinxparam{\DUrole{n,n}{full}\DUrole{p,p}{:}\DUrole{w,w}{  }\DUrole{n,n}{bool}\DUrole{w,w}{  }\DUrole{o,o}{=}\DUrole{w,w}{  }\DUrole{default_value}{True}}\sphinxparamcomma \sphinxparam{\DUrole{n,n}{use\_cached}\DUrole{p,p}{:}\DUrole{w,w}{  }\DUrole{n,n}{bool}\DUrole{w,w}{  }\DUrole{o,o}{=}\DUrole{w,w}{  }\DUrole{default_value}{False}}}{{ $\rightarrow$ PolyData}}
\pysigstopsignatures
\end{fulllineitems}

\index{midpoint\_in\_world (deepdrr.annotations.line\_annotation.LineAnnotation property)@\spxentry{midpoint\_in\_world}\spxextra{deepdrr.annotations.line\_annotation.LineAnnotation property}}

\begin{fulllineitems}
\phantomsection\label{\detokenize{deepdrr.annotations:deepdrr.annotations.line_annotation.LineAnnotation.midpoint_in_world}}
\pysigstartsignatures
\pysigline{\sphinxbfcode{\sphinxupquote{property\DUrole{w,w}{  }}}\sphinxbfcode{\sphinxupquote{midpoint\_in\_world}}\sphinxbfcode{\sphinxupquote{\DUrole{p,p}{:}\DUrole{w,w}{  }{\hyperref[\detokenize{deepdrr.geo:deepdrr.geo.core.Point3D}]{\sphinxcrossref{Point3D}}}}}}
\pysigstopsignatures
\end{fulllineitems}

\index{save() (deepdrr.annotations.line\_annotation.LineAnnotation method)@\spxentry{save()}\spxextra{deepdrr.annotations.line\_annotation.LineAnnotation method}}

\begin{fulllineitems}
\phantomsection\label{\detokenize{deepdrr.annotations:deepdrr.annotations.line_annotation.LineAnnotation.save}}
\pysigstartsignatures
\pysiglinewithargsret{\sphinxbfcode{\sphinxupquote{save}}}{\sphinxparam{\DUrole{n,n}{path}\DUrole{p,p}{:}\DUrole{w,w}{  }\DUrole{n,n}{str}}\sphinxparamcomma \sphinxparam{\DUrole{n,n}{color}\DUrole{p,p}{:}\DUrole{w,w}{  }\DUrole{n,n}{List\DUrole{p,p}{{[}}float\DUrole{p,p}{{]}}}\DUrole{w,w}{  }\DUrole{o,o}{=}\DUrole{w,w}{  }\DUrole{default_value}{{[}1.0, 0.5000076295109483, 0.5000076295109483{]}}}}{}
\pysigstopsignatures
\sphinxAtStartPar
Save the Line annotation to a mrk.json file, which can be opened by 3D Slicer.
\begin{quote}\begin{description}
\sphinxlineitem{Parameters}\begin{itemize}
\item {} 
\sphinxAtStartPar
\sphinxstyleliteralstrong{\sphinxupquote{path}} (\sphinxstyleliteralemphasis{\sphinxupquote{str}}) \textendash{} Output path to the file.

\item {} 
\sphinxAtStartPar
\sphinxstyleliteralstrong{\sphinxupquote{color}} (\sphinxstyleliteralemphasis{\sphinxupquote{List}}\sphinxstyleliteralemphasis{\sphinxupquote{{[}}}\sphinxstyleliteralemphasis{\sphinxupquote{int}}\sphinxstyleliteralemphasis{\sphinxupquote{{]}}}\sphinxstyleliteralemphasis{\sphinxupquote{, }}\sphinxstyleliteralemphasis{\sphinxupquote{optional}}) \textendash{} The color of the saved annotation.

\end{itemize}

\end{description}\end{quote}

\end{fulllineitems}

\index{startpoint\_in\_world (deepdrr.annotations.line\_annotation.LineAnnotation property)@\spxentry{startpoint\_in\_world}\spxextra{deepdrr.annotations.line\_annotation.LineAnnotation property}}

\begin{fulllineitems}
\phantomsection\label{\detokenize{deepdrr.annotations:deepdrr.annotations.line_annotation.LineAnnotation.startpoint_in_world}}
\pysigstartsignatures
\pysigline{\sphinxbfcode{\sphinxupquote{property\DUrole{w,w}{  }}}\sphinxbfcode{\sphinxupquote{startpoint\_in\_world}}\sphinxbfcode{\sphinxupquote{\DUrole{p,p}{:}\DUrole{w,w}{  }{\hyperref[\detokenize{deepdrr.geo:deepdrr.geo.core.Point3D}]{\sphinxcrossref{Point3D}}}}}}
\pysigstopsignatures
\end{fulllineitems}

\index{trajectory\_in\_world (deepdrr.annotations.line\_annotation.LineAnnotation property)@\spxentry{trajectory\_in\_world}\spxextra{deepdrr.annotations.line\_annotation.LineAnnotation property}}

\begin{fulllineitems}
\phantomsection\label{\detokenize{deepdrr.annotations:deepdrr.annotations.line_annotation.LineAnnotation.trajectory_in_world}}
\pysigstartsignatures
\pysigline{\sphinxbfcode{\sphinxupquote{property\DUrole{w,w}{  }}}\sphinxbfcode{\sphinxupquote{trajectory\_in\_world}}\sphinxbfcode{\sphinxupquote{\DUrole{p,p}{:}\DUrole{w,w}{  }{\hyperref[\detokenize{deepdrr.geo:deepdrr.geo.core.Vector3D}]{\sphinxcrossref{Vector3D}}}}}}
\pysigstopsignatures
\end{fulllineitems}

\index{world\_from\_anatomical (deepdrr.annotations.line\_annotation.LineAnnotation property)@\spxentry{world\_from\_anatomical}\spxextra{deepdrr.annotations.line\_annotation.LineAnnotation property}}

\begin{fulllineitems}
\phantomsection\label{\detokenize{deepdrr.annotations:deepdrr.annotations.line_annotation.LineAnnotation.world_from_anatomical}}
\pysigstartsignatures
\pysigline{\sphinxbfcode{\sphinxupquote{property\DUrole{w,w}{  }}}\sphinxbfcode{\sphinxupquote{world\_from\_anatomical}}\sphinxbfcode{\sphinxupquote{\DUrole{p,p}{:}\DUrole{w,w}{  }{\hyperref[\detokenize{deepdrr.geo:deepdrr.geo.core.FrameTransform}]{\sphinxcrossref{FrameTransform}}}}}}
\pysigstopsignatures
\end{fulllineitems}


\end{fulllineitems}



\subsection{Module contents}
\label{\detokenize{deepdrr.annotations:module-deepdrr.annotations}}\label{\detokenize{deepdrr.annotations:module-contents}}\index{module@\spxentry{module}!deepdrr.annotations@\spxentry{deepdrr.annotations}}\index{deepdrr.annotations@\spxentry{deepdrr.annotations}!module@\spxentry{module}}\index{Fiducial (class in deepdrr.annotations)@\spxentry{Fiducial}\spxextra{class in deepdrr.annotations}}

\begin{fulllineitems}
\phantomsection\label{\detokenize{deepdrr.annotations:deepdrr.annotations.Fiducial}}
\pysigstartsignatures
\pysiglinewithargsret{\sphinxbfcode{\sphinxupquote{class\DUrole{w,w}{  }}}\sphinxcode{\sphinxupquote{deepdrr.annotations.}}\sphinxbfcode{\sphinxupquote{Fiducial}}}{\sphinxparam{\DUrole{n,n}{data}\DUrole{p,p}{:}\DUrole{w,w}{  }\DUrole{n,n}{ndarray}}}{}
\pysigstopsignatures
\sphinxAtStartPar
Bases: {\hyperref[\detokenize{deepdrr.geo:deepdrr.geo.core.Point3D}]{\sphinxcrossref{\sphinxcode{\sphinxupquote{Point3D}}}}}
\index{data (deepdrr.annotations.Fiducial attribute)@\spxentry{data}\spxextra{deepdrr.annotations.Fiducial attribute}}

\begin{fulllineitems}
\phantomsection\label{\detokenize{deepdrr.annotations:deepdrr.annotations.Fiducial.data}}
\pysigstartsignatures
\pysigline{\sphinxbfcode{\sphinxupquote{data}}\sphinxbfcode{\sphinxupquote{\DUrole{p,p}{:}\DUrole{w,w}{  }ndarray}}}
\pysigstopsignatures
\end{fulllineitems}

\index{from\_fcsv() (deepdrr.annotations.Fiducial class method)@\spxentry{from\_fcsv()}\spxextra{deepdrr.annotations.Fiducial class method}}

\begin{fulllineitems}
\phantomsection\label{\detokenize{deepdrr.annotations:deepdrr.annotations.Fiducial.from_fcsv}}
\pysigstartsignatures
\pysiglinewithargsret{\sphinxbfcode{\sphinxupquote{classmethod\DUrole{w,w}{  }}}\sphinxbfcode{\sphinxupquote{from\_fcsv}}}{\sphinxparam{\DUrole{n,n}{path}\DUrole{p,p}{:}\DUrole{w,w}{  }\DUrole{n,n}{Path}}\sphinxparamcomma \sphinxparam{\DUrole{n,n}{world\_from\_anatomical}\DUrole{p,p}{:}\DUrole{w,w}{  }\DUrole{n,n}{{\hyperref[\detokenize{deepdrr.geo:deepdrr.geo.core.FrameTransform}]{\sphinxcrossref{FrameTransform}}}\DUrole{w,w}{  }\DUrole{p,p}{|}\DUrole{w,w}{  }None}\DUrole{w,w}{  }\DUrole{o,o}{=}\DUrole{w,w}{  }\DUrole{default_value}{None}}}{}
\pysigstopsignatures
\end{fulllineitems}

\index{from\_json() (deepdrr.annotations.Fiducial class method)@\spxentry{from\_json()}\spxextra{deepdrr.annotations.Fiducial class method}}

\begin{fulllineitems}
\phantomsection\label{\detokenize{deepdrr.annotations:deepdrr.annotations.Fiducial.from_json}}
\pysigstartsignatures
\pysiglinewithargsret{\sphinxbfcode{\sphinxupquote{classmethod\DUrole{w,w}{  }}}\sphinxbfcode{\sphinxupquote{from\_json}}}{\sphinxparam{\DUrole{n,n}{path}\DUrole{p,p}{:}\DUrole{w,w}{  }\DUrole{n,n}{Path}}\sphinxparamcomma \sphinxparam{\DUrole{n,n}{world\_from\_anatomical}\DUrole{p,p}{:}\DUrole{w,w}{  }\DUrole{n,n}{{\hyperref[\detokenize{deepdrr.geo:deepdrr.geo.core.FrameTransform}]{\sphinxcrossref{FrameTransform}}}\DUrole{w,w}{  }\DUrole{p,p}{|}\DUrole{w,w}{  }None}\DUrole{w,w}{  }\DUrole{o,o}{=}\DUrole{w,w}{  }\DUrole{default_value}{None}}}{}
\pysigstopsignatures
\end{fulllineitems}

\index{save() (deepdrr.annotations.Fiducial method)@\spxentry{save()}\spxextra{deepdrr.annotations.Fiducial method}}

\begin{fulllineitems}
\phantomsection\label{\detokenize{deepdrr.annotations:deepdrr.annotations.Fiducial.save}}
\pysigstartsignatures
\pysiglinewithargsret{\sphinxbfcode{\sphinxupquote{save}}}{\sphinxparam{\DUrole{n,n}{path}\DUrole{p,p}{:}\DUrole{w,w}{  }\DUrole{n,n}{Path}}}{}
\pysigstopsignatures
\end{fulllineitems}


\end{fulllineitems}

\index{FiducialList (class in deepdrr.annotations)@\spxentry{FiducialList}\spxextra{class in deepdrr.annotations}}

\begin{fulllineitems}
\phantomsection\label{\detokenize{deepdrr.annotations:deepdrr.annotations.FiducialList}}
\pysigstartsignatures
\pysiglinewithargsret{\sphinxbfcode{\sphinxupquote{class\DUrole{w,w}{  }}}\sphinxcode{\sphinxupquote{deepdrr.annotations.}}\sphinxbfcode{\sphinxupquote{FiducialList}}}{\sphinxparam{\DUrole{n,n}{points}\DUrole{p,p}{:}\DUrole{w,w}{  }\DUrole{n,n}{List\DUrole{p,p}{{[}}{\hyperref[\detokenize{deepdrr.geo:deepdrr.geo.core.Point3D}]{\sphinxcrossref{Point3D}}}\DUrole{p,p}{{]}}}}\sphinxparamcomma \sphinxparam{\DUrole{n,n}{world\_from\_anatomical}\DUrole{p,p}{:}\DUrole{w,w}{  }\DUrole{n,n}{{\hyperref[\detokenize{deepdrr.geo:deepdrr.geo.core.FrameTransform}]{\sphinxcrossref{FrameTransform}}}\DUrole{w,w}{  }\DUrole{p,p}{|}\DUrole{w,w}{  }None}\DUrole{w,w}{  }\DUrole{o,o}{=}\DUrole{w,w}{  }\DUrole{default_value}{None}}\sphinxparamcomma \sphinxparam{\DUrole{n,n}{anatomical\_coordinate\_system}\DUrole{p,p}{:}\DUrole{w,w}{  }\DUrole{n,n}{Literal\DUrole{p,p}{{[}}\DUrole{s,s}{\textquotesingle{}RAS\textquotesingle{}}\DUrole{p,p}{,}\DUrole{w,w}{  }\DUrole{s,s}{\textquotesingle{}LPS\textquotesingle{}}\DUrole{p,p}{{]}}}\DUrole{w,w}{  }\DUrole{o,o}{=}\DUrole{w,w}{  }\DUrole{default_value}{\textquotesingle{}RAS\textquotesingle{}}}}{}
\pysigstopsignatures
\sphinxAtStartPar
Bases: \sphinxcode{\sphinxupquote{object}}
\index{from\_fcsv() (deepdrr.annotations.FiducialList class method)@\spxentry{from\_fcsv()}\spxextra{deepdrr.annotations.FiducialList class method}}

\begin{fulllineitems}
\phantomsection\label{\detokenize{deepdrr.annotations:deepdrr.annotations.FiducialList.from_fcsv}}
\pysigstartsignatures
\pysiglinewithargsret{\sphinxbfcode{\sphinxupquote{classmethod\DUrole{w,w}{  }}}\sphinxbfcode{\sphinxupquote{from\_fcsv}}}{\sphinxparam{\DUrole{n,n}{path}\DUrole{p,p}{:}\DUrole{w,w}{  }\DUrole{n,n}{Path}}\sphinxparamcomma \sphinxparam{\DUrole{n,n}{world\_from\_anatomical}\DUrole{p,p}{:}\DUrole{w,w}{  }\DUrole{n,n}{{\hyperref[\detokenize{deepdrr.geo:deepdrr.geo.core.FrameTransform}]{\sphinxcrossref{FrameTransform}}}\DUrole{w,w}{  }\DUrole{p,p}{|}\DUrole{w,w}{  }None}\DUrole{w,w}{  }\DUrole{o,o}{=}\DUrole{w,w}{  }\DUrole{default_value}{None}}}{{ $\rightarrow$ {\hyperref[\detokenize{deepdrr.annotations:deepdrr.annotations.fiducials.FiducialList}]{\sphinxcrossref{FiducialList}}}}}
\pysigstopsignatures
\sphinxAtStartPar
Load a FCSV file from Slicer3D
\begin{quote}\begin{description}
\sphinxlineitem{Parameters}
\sphinxAtStartPar
\sphinxstyleliteralstrong{\sphinxupquote{path}} (\sphinxstyleliteralemphasis{\sphinxupquote{Path}}) \textendash{} Path to the FCSV file

\sphinxlineitem{Returns}
\sphinxAtStartPar
Array of 3D points

\sphinxlineitem{Return type}
\sphinxAtStartPar
np.ndarray

\end{description}\end{quote}

\end{fulllineitems}

\index{from\_json() (deepdrr.annotations.FiducialList class method)@\spxentry{from\_json()}\spxextra{deepdrr.annotations.FiducialList class method}}

\begin{fulllineitems}
\phantomsection\label{\detokenize{deepdrr.annotations:deepdrr.annotations.FiducialList.from_json}}
\pysigstartsignatures
\pysiglinewithargsret{\sphinxbfcode{\sphinxupquote{classmethod\DUrole{w,w}{  }}}\sphinxbfcode{\sphinxupquote{from\_json}}}{\sphinxparam{\DUrole{n,n}{path}\DUrole{p,p}{:}\DUrole{w,w}{  }\DUrole{n,n}{Path}}\sphinxparamcomma \sphinxparam{\DUrole{n,n}{world\_from\_anatomical}\DUrole{p,p}{:}\DUrole{w,w}{  }\DUrole{n,n}{{\hyperref[\detokenize{deepdrr.geo:deepdrr.geo.core.FrameTransform}]{\sphinxcrossref{FrameTransform}}}\DUrole{w,w}{  }\DUrole{p,p}{|}\DUrole{w,w}{  }None}\DUrole{w,w}{  }\DUrole{o,o}{=}\DUrole{w,w}{  }\DUrole{default_value}{None}}}{}
\pysigstopsignatures
\end{fulllineitems}

\index{save() (deepdrr.annotations.FiducialList method)@\spxentry{save()}\spxextra{deepdrr.annotations.FiducialList method}}

\begin{fulllineitems}
\phantomsection\label{\detokenize{deepdrr.annotations:deepdrr.annotations.FiducialList.save}}
\pysigstartsignatures
\pysiglinewithargsret{\sphinxbfcode{\sphinxupquote{save}}}{\sphinxparam{\DUrole{n,n}{path}\DUrole{p,p}{:}\DUrole{w,w}{  }\DUrole{n,n}{Path}}}{}
\pysigstopsignatures
\end{fulllineitems}

\index{to\_LPS() (deepdrr.annotations.FiducialList method)@\spxentry{to\_LPS()}\spxextra{deepdrr.annotations.FiducialList method}}

\begin{fulllineitems}
\phantomsection\label{\detokenize{deepdrr.annotations:deepdrr.annotations.FiducialList.to_LPS}}
\pysigstartsignatures
\pysiglinewithargsret{\sphinxbfcode{\sphinxupquote{to\_LPS}}}{}{{ $\rightarrow$ {\hyperref[\detokenize{deepdrr.annotations:deepdrr.annotations.fiducials.FiducialList}]{\sphinxcrossref{FiducialList}}}}}
\pysigstopsignatures
\end{fulllineitems}

\index{to\_RAS() (deepdrr.annotations.FiducialList method)@\spxentry{to\_RAS()}\spxextra{deepdrr.annotations.FiducialList method}}

\begin{fulllineitems}
\phantomsection\label{\detokenize{deepdrr.annotations:deepdrr.annotations.FiducialList.to_RAS}}
\pysigstartsignatures
\pysiglinewithargsret{\sphinxbfcode{\sphinxupquote{to\_RAS}}}{}{{ $\rightarrow$ {\hyperref[\detokenize{deepdrr.annotations:deepdrr.annotations.fiducials.FiducialList}]{\sphinxcrossref{FiducialList}}}}}
\pysigstopsignatures
\end{fulllineitems}


\end{fulllineitems}

\index{LineAnnotation (class in deepdrr.annotations)@\spxentry{LineAnnotation}\spxextra{class in deepdrr.annotations}}

\begin{fulllineitems}
\phantomsection\label{\detokenize{deepdrr.annotations:deepdrr.annotations.LineAnnotation}}
\pysigstartsignatures
\pysiglinewithargsret{\sphinxbfcode{\sphinxupquote{class\DUrole{w,w}{  }}}\sphinxcode{\sphinxupquote{deepdrr.annotations.}}\sphinxbfcode{\sphinxupquote{LineAnnotation}}}{\sphinxparam{\DUrole{n,n}{startpoint}\DUrole{p,p}{:}\DUrole{w,w}{  }\DUrole{n,n}{{\hyperref[\detokenize{deepdrr.geo:deepdrr.geo.core.Point3D}]{\sphinxcrossref{Point3D}}}}}\sphinxparamcomma \sphinxparam{\DUrole{n,n}{endpoint}\DUrole{p,p}{:}\DUrole{w,w}{  }\DUrole{n,n}{{\hyperref[\detokenize{deepdrr.geo:deepdrr.geo.core.Point3D}]{\sphinxcrossref{Point3D}}}}}\sphinxparamcomma \sphinxparam{\DUrole{n,n}{volume}\DUrole{p,p}{:}\DUrole{w,w}{  }\DUrole{n,n}{{\hyperref[\detokenize{deepdrr.vol:deepdrr.vol.volume.Volume}]{\sphinxcrossref{Volume}}}\DUrole{w,w}{  }\DUrole{p,p}{|}\DUrole{w,w}{  }None}\DUrole{w,w}{  }\DUrole{o,o}{=}\DUrole{w,w}{  }\DUrole{default_value}{None}}\sphinxparamcomma \sphinxparam{\DUrole{n,n}{world\_from\_anatomical}\DUrole{p,p}{:}\DUrole{w,w}{  }\DUrole{n,n}{{\hyperref[\detokenize{deepdrr.geo:deepdrr.geo.core.FrameTransform}]{\sphinxcrossref{FrameTransform}}}\DUrole{w,w}{  }\DUrole{p,p}{|}\DUrole{w,w}{  }None}\DUrole{w,w}{  }\DUrole{o,o}{=}\DUrole{w,w}{  }\DUrole{default_value}{None}}\sphinxparamcomma \sphinxparam{\DUrole{n,n}{anatomical\_coordinate\_system}\DUrole{p,p}{:}\DUrole{w,w}{  }\DUrole{n,n}{str\DUrole{w,w}{  }\DUrole{p,p}{|}\DUrole{w,w}{  }None}\DUrole{w,w}{  }\DUrole{o,o}{=}\DUrole{w,w}{  }\DUrole{default_value}{None}}}{}
\pysigstopsignatures
\sphinxAtStartPar
Bases: \sphinxcode{\sphinxupquote{object}}

\sphinxAtStartPar
Really a “segment annotation”, but Slicer calls it a line.
\index{startpoint (deepdrr.annotations.LineAnnotation attribute)@\spxentry{startpoint}\spxextra{deepdrr.annotations.LineAnnotation attribute}}

\begin{fulllineitems}
\phantomsection\label{\detokenize{deepdrr.annotations:deepdrr.annotations.LineAnnotation.startpoint}}
\pysigstartsignatures
\pysigline{\sphinxbfcode{\sphinxupquote{startpoint}}}
\pysigstopsignatures
\sphinxAtStartPar
The startpoint in anatomical coordinates.
\begin{quote}\begin{description}
\sphinxlineitem{Type}
\sphinxAtStartPar
{\hyperref[\detokenize{deepdrr.geo:deepdrr.geo.Point}]{\sphinxcrossref{geo.Point}}}

\end{description}\end{quote}

\end{fulllineitems}

\index{endpoint (deepdrr.annotations.LineAnnotation attribute)@\spxentry{endpoint}\spxextra{deepdrr.annotations.LineAnnotation attribute}}

\begin{fulllineitems}
\phantomsection\label{\detokenize{deepdrr.annotations:deepdrr.annotations.LineAnnotation.endpoint}}
\pysigstartsignatures
\pysigline{\sphinxbfcode{\sphinxupquote{endpoint}}}
\pysigstopsignatures
\sphinxAtStartPar
The endpoint in anatomical coordinates.
\begin{quote}\begin{description}
\sphinxlineitem{Type}
\sphinxAtStartPar
{\hyperref[\detokenize{deepdrr.geo:deepdrr.geo.Point}]{\sphinxcrossref{geo.Point}}}

\end{description}\end{quote}

\end{fulllineitems}

\index{anatomical\_coordinate\_system (deepdrr.annotations.LineAnnotation property)@\spxentry{anatomical\_coordinate\_system}\spxextra{deepdrr.annotations.LineAnnotation property}}

\begin{fulllineitems}
\phantomsection\label{\detokenize{deepdrr.annotations:deepdrr.annotations.LineAnnotation.anatomical_coordinate_system}}
\pysigstartsignatures
\pysigline{\sphinxbfcode{\sphinxupquote{property\DUrole{w,w}{  }}}\sphinxbfcode{\sphinxupquote{anatomical\_coordinate\_system}}\sphinxbfcode{\sphinxupquote{\DUrole{p,p}{:}\DUrole{w,w}{  }str}}}
\pysigstopsignatures
\end{fulllineitems}

\index{anatomical\_from\_world (deepdrr.annotations.LineAnnotation property)@\spxentry{anatomical\_from\_world}\spxextra{deepdrr.annotations.LineAnnotation property}}

\begin{fulllineitems}
\phantomsection\label{\detokenize{deepdrr.annotations:deepdrr.annotations.LineAnnotation.anatomical_from_world}}
\pysigstartsignatures
\pysigline{\sphinxbfcode{\sphinxupquote{property\DUrole{w,w}{  }}}\sphinxbfcode{\sphinxupquote{anatomical\_from\_world}}\sphinxbfcode{\sphinxupquote{\DUrole{p,p}{:}\DUrole{w,w}{  }{\hyperref[\detokenize{deepdrr.geo:deepdrr.geo.core.FrameTransform}]{\sphinxcrossref{FrameTransform}}}}}}
\pysigstopsignatures
\end{fulllineitems}

\index{direction\_in\_world (deepdrr.annotations.LineAnnotation property)@\spxentry{direction\_in\_world}\spxextra{deepdrr.annotations.LineAnnotation property}}

\begin{fulllineitems}
\phantomsection\label{\detokenize{deepdrr.annotations:deepdrr.annotations.LineAnnotation.direction_in_world}}
\pysigstartsignatures
\pysigline{\sphinxbfcode{\sphinxupquote{property\DUrole{w,w}{  }}}\sphinxbfcode{\sphinxupquote{direction\_in\_world}}\sphinxbfcode{\sphinxupquote{\DUrole{p,p}{:}\DUrole{w,w}{  }{\hyperref[\detokenize{deepdrr.geo:deepdrr.geo.core.Vector3D}]{\sphinxcrossref{Vector3D}}}}}}
\pysigstopsignatures
\end{fulllineitems}

\index{endpoint\_in\_world (deepdrr.annotations.LineAnnotation property)@\spxentry{endpoint\_in\_world}\spxextra{deepdrr.annotations.LineAnnotation property}}

\begin{fulllineitems}
\phantomsection\label{\detokenize{deepdrr.annotations:deepdrr.annotations.LineAnnotation.endpoint_in_world}}
\pysigstartsignatures
\pysigline{\sphinxbfcode{\sphinxupquote{property\DUrole{w,w}{  }}}\sphinxbfcode{\sphinxupquote{endpoint\_in\_world}}\sphinxbfcode{\sphinxupquote{\DUrole{p,p}{:}\DUrole{w,w}{  }{\hyperref[\detokenize{deepdrr.geo:deepdrr.geo.core.Point3D}]{\sphinxcrossref{Point3D}}}}}}
\pysigstopsignatures
\end{fulllineitems}

\index{from\_fcsv() (deepdrr.annotations.LineAnnotation class method)@\spxentry{from\_fcsv()}\spxextra{deepdrr.annotations.LineAnnotation class method}}

\begin{fulllineitems}
\phantomsection\label{\detokenize{deepdrr.annotations:deepdrr.annotations.LineAnnotation.from_fcsv}}
\pysigstartsignatures
\pysiglinewithargsret{\sphinxbfcode{\sphinxupquote{classmethod\DUrole{w,w}{  }}}\sphinxbfcode{\sphinxupquote{from\_fcsv}}}{\sphinxparam{\DUrole{n,n}{path}\DUrole{p,p}{:}\DUrole{w,w}{  }\DUrole{n,n}{Path}}\sphinxparamcomma \sphinxparam{\DUrole{o,o}{**}\DUrole{n,n}{kwargs}}}{{ $\rightarrow$ {\hyperref[\detokenize{deepdrr.annotations:deepdrr.annotations.line_annotation.LineAnnotation}]{\sphinxcrossref{LineAnnotation}}}}}
\pysigstopsignatures
\sphinxAtStartPar
Load a LineAnnotation from a .fcsv file.
\begin{quote}\begin{description}
\sphinxlineitem{Parameters}\begin{itemize}
\item {} 
\sphinxAtStartPar
\sphinxstyleliteralstrong{\sphinxupquote{path}} (\sphinxstyleliteralemphasis{\sphinxupquote{Path}}) \textendash{} Path to the .fcsv file.

\item {} 
\sphinxAtStartPar
\sphinxstyleliteralstrong{\sphinxupquote{world\_from\_anatomical}} (\sphinxstyleliteralemphasis{\sphinxupquote{Optional}}\sphinxstyleliteralemphasis{\sphinxupquote{{[}}}{\hyperref[\detokenize{deepdrr.geo:deepdrr.geo.FrameTransform}]{\sphinxcrossref{\sphinxstyleliteralemphasis{\sphinxupquote{geo.FrameTransform}}}}}\sphinxstyleliteralemphasis{\sphinxupquote{{]}}}\sphinxstyleliteralemphasis{\sphinxupquote{, }}\sphinxstyleliteralemphasis{\sphinxupquote{optional}}) \textendash{} The pose of the volume in world coordinates. Defaults to None.

\end{itemize}

\sphinxlineitem{Returns}
\sphinxAtStartPar
The loaded annotation.

\sphinxlineitem{Return type}
\sphinxAtStartPar
{\hyperref[\detokenize{deepdrr.annotations:deepdrr.annotations.LineAnnotation}]{\sphinxcrossref{LineAnnotation}}}

\end{description}\end{quote}

\end{fulllineitems}

\index{from\_json() (deepdrr.annotations.LineAnnotation class method)@\spxentry{from\_json()}\spxextra{deepdrr.annotations.LineAnnotation class method}}

\begin{fulllineitems}
\phantomsection\label{\detokenize{deepdrr.annotations:deepdrr.annotations.LineAnnotation.from_json}}
\pysigstartsignatures
\pysiglinewithargsret{\sphinxbfcode{\sphinxupquote{classmethod\DUrole{w,w}{  }}}\sphinxbfcode{\sphinxupquote{from\_json}}}{\sphinxparam{\DUrole{n,n}{path}\DUrole{p,p}{:}\DUrole{w,w}{  }\DUrole{n,n}{str}}\sphinxparamcomma \sphinxparam{\DUrole{n,n}{volume}\DUrole{p,p}{:}\DUrole{w,w}{  }\DUrole{n,n}{{\hyperref[\detokenize{deepdrr.vol:deepdrr.vol.volume.Volume}]{\sphinxcrossref{Volume}}}\DUrole{w,w}{  }\DUrole{p,p}{|}\DUrole{w,w}{  }None}\DUrole{w,w}{  }\DUrole{o,o}{=}\DUrole{w,w}{  }\DUrole{default_value}{None}}\sphinxparamcomma \sphinxparam{\DUrole{n,n}{world\_from\_anatomical}\DUrole{p,p}{:}\DUrole{w,w}{  }\DUrole{n,n}{{\hyperref[\detokenize{deepdrr.geo:deepdrr.geo.core.FrameTransform}]{\sphinxcrossref{FrameTransform}}}\DUrole{w,w}{  }\DUrole{p,p}{|}\DUrole{w,w}{  }None}\DUrole{w,w}{  }\DUrole{o,o}{=}\DUrole{w,w}{  }\DUrole{default_value}{None}}\sphinxparamcomma \sphinxparam{\DUrole{n,n}{anatomical\_coordinate\_system}\DUrole{p,p}{:}\DUrole{w,w}{  }\DUrole{n,n}{str\DUrole{w,w}{  }\DUrole{p,p}{|}\DUrole{w,w}{  }None}\DUrole{w,w}{  }\DUrole{o,o}{=}\DUrole{w,w}{  }\DUrole{default_value}{None}}}{{ $\rightarrow$ {\hyperref[\detokenize{deepdrr.annotations:deepdrr.annotations.line_annotation.LineAnnotation}]{\sphinxcrossref{LineAnnotation}}}}}
\pysigstopsignatures
\end{fulllineitems}

\index{from\_markup() (deepdrr.annotations.LineAnnotation class method)@\spxentry{from\_markup()}\spxextra{deepdrr.annotations.LineAnnotation class method}}

\begin{fulllineitems}
\phantomsection\label{\detokenize{deepdrr.annotations:deepdrr.annotations.LineAnnotation.from_markup}}
\pysigstartsignatures
\pysiglinewithargsret{\sphinxbfcode{\sphinxupquote{classmethod\DUrole{w,w}{  }}}\sphinxbfcode{\sphinxupquote{from\_markup}}}{\sphinxparam{\DUrole{o,o}{*}\DUrole{n,n}{args}}\sphinxparamcomma \sphinxparam{\DUrole{o,o}{**}\DUrole{n,n}{kwargs}}}{}
\pysigstopsignatures
\end{fulllineitems}

\index{get\_direction() (deepdrr.annotations.LineAnnotation method)@\spxentry{get\_direction()}\spxextra{deepdrr.annotations.LineAnnotation method}}

\begin{fulllineitems}
\phantomsection\label{\detokenize{deepdrr.annotations:deepdrr.annotations.LineAnnotation.get_direction}}
\pysigstartsignatures
\pysiglinewithargsret{\sphinxbfcode{\sphinxupquote{get\_direction}}}{}{{ $\rightarrow$ {\hyperref[\detokenize{deepdrr.geo:deepdrr.geo.core.Vector3D}]{\sphinxcrossref{Vector3D}}}}}
\pysigstopsignatures
\end{fulllineitems}

\index{get\_mesh() (deepdrr.annotations.LineAnnotation method)@\spxentry{get\_mesh()}\spxextra{deepdrr.annotations.LineAnnotation method}}

\begin{fulllineitems}
\phantomsection\label{\detokenize{deepdrr.annotations:deepdrr.annotations.LineAnnotation.get_mesh}}
\pysigstartsignatures
\pysiglinewithargsret{\sphinxbfcode{\sphinxupquote{get\_mesh}}}{}{}
\pysigstopsignatures
\sphinxAtStartPar
Get the mesh in anatomical coordinates.

\end{fulllineitems}

\index{get\_mesh\_in\_world() (deepdrr.annotations.LineAnnotation method)@\spxentry{get\_mesh\_in\_world()}\spxextra{deepdrr.annotations.LineAnnotation method}}

\begin{fulllineitems}
\phantomsection\label{\detokenize{deepdrr.annotations:deepdrr.annotations.LineAnnotation.get_mesh_in_world}}
\pysigstartsignatures
\pysiglinewithargsret{\sphinxbfcode{\sphinxupquote{get\_mesh\_in\_world}}}{\sphinxparam{\DUrole{n,n}{full}\DUrole{p,p}{:}\DUrole{w,w}{  }\DUrole{n,n}{bool}\DUrole{w,w}{  }\DUrole{o,o}{=}\DUrole{w,w}{  }\DUrole{default_value}{True}}\sphinxparamcomma \sphinxparam{\DUrole{n,n}{use\_cached}\DUrole{p,p}{:}\DUrole{w,w}{  }\DUrole{n,n}{bool}\DUrole{w,w}{  }\DUrole{o,o}{=}\DUrole{w,w}{  }\DUrole{default_value}{False}}}{{ $\rightarrow$ PolyData}}
\pysigstopsignatures
\end{fulllineitems}

\index{midpoint\_in\_world (deepdrr.annotations.LineAnnotation property)@\spxentry{midpoint\_in\_world}\spxextra{deepdrr.annotations.LineAnnotation property}}

\begin{fulllineitems}
\phantomsection\label{\detokenize{deepdrr.annotations:deepdrr.annotations.LineAnnotation.midpoint_in_world}}
\pysigstartsignatures
\pysigline{\sphinxbfcode{\sphinxupquote{property\DUrole{w,w}{  }}}\sphinxbfcode{\sphinxupquote{midpoint\_in\_world}}\sphinxbfcode{\sphinxupquote{\DUrole{p,p}{:}\DUrole{w,w}{  }{\hyperref[\detokenize{deepdrr.geo:deepdrr.geo.core.Point3D}]{\sphinxcrossref{Point3D}}}}}}
\pysigstopsignatures
\end{fulllineitems}

\index{save() (deepdrr.annotations.LineAnnotation method)@\spxentry{save()}\spxextra{deepdrr.annotations.LineAnnotation method}}

\begin{fulllineitems}
\phantomsection\label{\detokenize{deepdrr.annotations:deepdrr.annotations.LineAnnotation.save}}
\pysigstartsignatures
\pysiglinewithargsret{\sphinxbfcode{\sphinxupquote{save}}}{\sphinxparam{\DUrole{n,n}{path}\DUrole{p,p}{:}\DUrole{w,w}{  }\DUrole{n,n}{str}}\sphinxparamcomma \sphinxparam{\DUrole{n,n}{color}\DUrole{p,p}{:}\DUrole{w,w}{  }\DUrole{n,n}{List\DUrole{p,p}{{[}}float\DUrole{p,p}{{]}}}\DUrole{w,w}{  }\DUrole{o,o}{=}\DUrole{w,w}{  }\DUrole{default_value}{{[}1.0, 0.5000076295109483, 0.5000076295109483{]}}}}{}
\pysigstopsignatures
\sphinxAtStartPar
Save the Line annotation to a mrk.json file, which can be opened by 3D Slicer.
\begin{quote}\begin{description}
\sphinxlineitem{Parameters}\begin{itemize}
\item {} 
\sphinxAtStartPar
\sphinxstyleliteralstrong{\sphinxupquote{path}} (\sphinxstyleliteralemphasis{\sphinxupquote{str}}) \textendash{} Output path to the file.

\item {} 
\sphinxAtStartPar
\sphinxstyleliteralstrong{\sphinxupquote{color}} (\sphinxstyleliteralemphasis{\sphinxupquote{List}}\sphinxstyleliteralemphasis{\sphinxupquote{{[}}}\sphinxstyleliteralemphasis{\sphinxupquote{int}}\sphinxstyleliteralemphasis{\sphinxupquote{{]}}}\sphinxstyleliteralemphasis{\sphinxupquote{, }}\sphinxstyleliteralemphasis{\sphinxupquote{optional}}) \textendash{} The color of the saved annotation.

\end{itemize}

\end{description}\end{quote}

\end{fulllineitems}

\index{startpoint\_in\_world (deepdrr.annotations.LineAnnotation property)@\spxentry{startpoint\_in\_world}\spxextra{deepdrr.annotations.LineAnnotation property}}

\begin{fulllineitems}
\phantomsection\label{\detokenize{deepdrr.annotations:deepdrr.annotations.LineAnnotation.startpoint_in_world}}
\pysigstartsignatures
\pysigline{\sphinxbfcode{\sphinxupquote{property\DUrole{w,w}{  }}}\sphinxbfcode{\sphinxupquote{startpoint\_in\_world}}\sphinxbfcode{\sphinxupquote{\DUrole{p,p}{:}\DUrole{w,w}{  }{\hyperref[\detokenize{deepdrr.geo:deepdrr.geo.core.Point3D}]{\sphinxcrossref{Point3D}}}}}}
\pysigstopsignatures
\end{fulllineitems}

\index{trajectory\_in\_world (deepdrr.annotations.LineAnnotation property)@\spxentry{trajectory\_in\_world}\spxextra{deepdrr.annotations.LineAnnotation property}}

\begin{fulllineitems}
\phantomsection\label{\detokenize{deepdrr.annotations:deepdrr.annotations.LineAnnotation.trajectory_in_world}}
\pysigstartsignatures
\pysigline{\sphinxbfcode{\sphinxupquote{property\DUrole{w,w}{  }}}\sphinxbfcode{\sphinxupquote{trajectory\_in\_world}}\sphinxbfcode{\sphinxupquote{\DUrole{p,p}{:}\DUrole{w,w}{  }{\hyperref[\detokenize{deepdrr.geo:deepdrr.geo.core.Vector3D}]{\sphinxcrossref{Vector3D}}}}}}
\pysigstopsignatures
\end{fulllineitems}

\index{world\_from\_anatomical (deepdrr.annotations.LineAnnotation property)@\spxentry{world\_from\_anatomical}\spxextra{deepdrr.annotations.LineAnnotation property}}

\begin{fulllineitems}
\phantomsection\label{\detokenize{deepdrr.annotations:deepdrr.annotations.LineAnnotation.world_from_anatomical}}
\pysigstartsignatures
\pysigline{\sphinxbfcode{\sphinxupquote{property\DUrole{w,w}{  }}}\sphinxbfcode{\sphinxupquote{world\_from\_anatomical}}\sphinxbfcode{\sphinxupquote{\DUrole{p,p}{:}\DUrole{w,w}{  }{\hyperref[\detokenize{deepdrr.geo:deepdrr.geo.core.FrameTransform}]{\sphinxcrossref{FrameTransform}}}}}}
\pysigstopsignatures
\end{fulllineitems}


\end{fulllineitems}


\sphinxstepscope


\section{deepdrr.device package}
\label{\detokenize{deepdrr.device:deepdrr-device-package}}\label{\detokenize{deepdrr.device::doc}}

\subsection{deepdrr.device.carm}
\label{\detokenize{deepdrr.device:module-deepdrr.device.carm}}\label{\detokenize{deepdrr.device:deepdrr-device-carm}}\index{module@\spxentry{module}!deepdrr.device.carm@\spxentry{deepdrr.device.carm}}\index{deepdrr.device.carm@\spxentry{deepdrr.device.carm}!module@\spxentry{module}}\index{CArm (class in deepdrr.device.carm)@\spxentry{CArm}\spxextra{class in deepdrr.device.carm}}

\begin{fulllineitems}
\phantomsection\label{\detokenize{deepdrr.device:deepdrr.device.carm.CArm}}
\pysigstartsignatures
\pysiglinewithargsret{\sphinxbfcode{\sphinxupquote{class\DUrole{w,w}{  }}}\sphinxcode{\sphinxupquote{deepdrr.device.carm.}}\sphinxbfcode{\sphinxupquote{CArm}}}{\sphinxparam{\DUrole{n,n}{isocenter\_distance}\DUrole{p,p}{:}\DUrole{w,w}{  }\DUrole{n,n}{float}}\sphinxparamcomma \sphinxparam{\DUrole{n,n}{isocenter}\DUrole{p,p}{:}\DUrole{w,w}{  }\DUrole{n,n}{{\hyperref[\detokenize{deepdrr.geo:deepdrr.geo.core.Point3D}]{\sphinxcrossref{Point3D}}}\DUrole{w,w}{  }\DUrole{p,p}{|}\DUrole{w,w}{  }None}\DUrole{w,w}{  }\DUrole{o,o}{=}\DUrole{w,w}{  }\DUrole{default_value}{None}}\sphinxparamcomma \sphinxparam{\DUrole{n,n}{phi}\DUrole{p,p}{:}\DUrole{w,w}{  }\DUrole{n,n}{float}\DUrole{w,w}{  }\DUrole{o,o}{=}\DUrole{w,w}{  }\DUrole{default_value}{0}}\sphinxparamcomma \sphinxparam{\DUrole{n,n}{theta}\DUrole{p,p}{:}\DUrole{w,w}{  }\DUrole{n,n}{float}\DUrole{w,w}{  }\DUrole{o,o}{=}\DUrole{w,w}{  }\DUrole{default_value}{0}}\sphinxparamcomma \sphinxparam{\DUrole{n,n}{rho}\DUrole{p,p}{:}\DUrole{w,w}{  }\DUrole{n,n}{float}\DUrole{w,w}{  }\DUrole{o,o}{=}\DUrole{w,w}{  }\DUrole{default_value}{0}}\sphinxparamcomma \sphinxparam{\DUrole{n,n}{degrees}\DUrole{p,p}{:}\DUrole{w,w}{  }\DUrole{n,n}{bool}\DUrole{w,w}{  }\DUrole{o,o}{=}\DUrole{w,w}{  }\DUrole{default_value}{False}}}{}
\pysigstopsignatures
\sphinxAtStartPar
Bases: \sphinxcode{\sphinxupquote{object}}

\sphinxAtStartPar
C\sphinxhyphen{}arm device for positioning a camera in space.

\sphinxAtStartPar
It is suggested to use MobileCArm instead.
\index{camera3d\_from\_world (deepdrr.device.carm.CArm property)@\spxentry{camera3d\_from\_world}\spxextra{deepdrr.device.carm.CArm property}}

\begin{fulllineitems}
\phantomsection\label{\detokenize{deepdrr.device:deepdrr.device.carm.CArm.camera3d_from_world}}
\pysigstartsignatures
\pysigline{\sphinxbfcode{\sphinxupquote{property\DUrole{w,w}{  }}}\sphinxbfcode{\sphinxupquote{camera3d\_from\_world}}\sphinxbfcode{\sphinxupquote{\DUrole{p,p}{:}\DUrole{w,w}{  }{\hyperref[\detokenize{deepdrr.geo:deepdrr.geo.core.FrameTransform}]{\sphinxcrossref{FrameTransform}}}}}}
\pysigstopsignatures
\end{fulllineitems}

\index{get\_camera3d\_from\_world() (deepdrr.device.carm.CArm method)@\spxentry{get\_camera3d\_from\_world()}\spxextra{deepdrr.device.carm.CArm method}}

\begin{fulllineitems}
\phantomsection\label{\detokenize{deepdrr.device:deepdrr.device.carm.CArm.get_camera3d_from_world}}
\pysigstartsignatures
\pysiglinewithargsret{\sphinxbfcode{\sphinxupquote{get\_camera3d\_from\_world}}}{\sphinxparam{\DUrole{n,n}{isocenter}\DUrole{p,p}{:}\DUrole{w,w}{  }\DUrole{n,n}{{\hyperref[\detokenize{deepdrr.geo:deepdrr.geo.core.Point3D}]{\sphinxcrossref{Point3D}}}}}\sphinxparamcomma \sphinxparam{\DUrole{n,n}{phi}\DUrole{p,p}{:}\DUrole{w,w}{  }\DUrole{n,n}{float}}\sphinxparamcomma \sphinxparam{\DUrole{n,n}{theta}\DUrole{p,p}{:}\DUrole{w,w}{  }\DUrole{n,n}{float}}\sphinxparamcomma \sphinxparam{\DUrole{n,n}{rho}\DUrole{p,p}{:}\DUrole{w,w}{  }\DUrole{n,n}{float\DUrole{w,w}{  }\DUrole{p,p}{|}\DUrole{w,w}{  }None}\DUrole{w,w}{  }\DUrole{o,o}{=}\DUrole{w,w}{  }\DUrole{default_value}{0}}\sphinxparamcomma \sphinxparam{\DUrole{n,n}{degrees}\DUrole{p,p}{:}\DUrole{w,w}{  }\DUrole{n,n}{bool}\DUrole{w,w}{  }\DUrole{o,o}{=}\DUrole{w,w}{  }\DUrole{default_value}{False}}}{{ $\rightarrow$ {\hyperref[\detokenize{deepdrr.geo:deepdrr.geo.core.FrameTransform}]{\sphinxcrossref{FrameTransform}}}}}
\pysigstopsignatures
\sphinxAtStartPar
Get the FrameTransform for the C\sphinxhyphen{}Arm device at the given pose.

\sphinxAtStartPar
This ignores the internal state except for the isocenter\_distance.
\begin{quote}\begin{description}
\sphinxlineitem{Parameters}\begin{itemize}
\item {} 
\sphinxAtStartPar
\sphinxstyleliteralstrong{\sphinxupquote{isocenter}} ({\hyperref[\detokenize{deepdrr.geo:deepdrr.geo.Point3D}]{\sphinxcrossref{\sphinxstyleliteralemphasis{\sphinxupquote{geo.Point3D}}}}}) \textendash{} isocenter of the device.

\item {} 
\sphinxAtStartPar
\sphinxstyleliteralstrong{\sphinxupquote{phi}} (\sphinxstyleliteralemphasis{\sphinxupquote{float}}) \textendash{} CRAN/CAUD angle of the C\sphinxhyphen{}Arm (along the actual arc of the arm)

\item {} 
\sphinxAtStartPar
\sphinxstyleliteralstrong{\sphinxupquote{theta}} (\sphinxstyleliteralemphasis{\sphinxupquote{float}}) \textendash{} Lect/Right angulation of C\sphinxhyphen{}arm (rotation at the base)

\item {} 
\sphinxAtStartPar
\sphinxstyleliteralstrong{\sphinxupquote{rho}} (\sphinxstyleliteralemphasis{\sphinxupquote{Optional}}\sphinxstyleliteralemphasis{\sphinxupquote{{[}}}\sphinxstyleliteralemphasis{\sphinxupquote{float}}\sphinxstyleliteralemphasis{\sphinxupquote{{]}}}\sphinxstyleliteralemphasis{\sphinxupquote{, }}\sphinxstyleliteralemphasis{\sphinxupquote{optional}}) \textendash{} rotation about principle axis, after main rotation. Defaults to 0.

\item {} 
\sphinxAtStartPar
\sphinxstyleliteralstrong{\sphinxupquote{degrees}} (\sphinxstyleliteralemphasis{\sphinxupquote{bool}}\sphinxstyleliteralemphasis{\sphinxupquote{, }}\sphinxstyleliteralemphasis{\sphinxupquote{optional}}) \textendash{} Whether given angles are in degrees. Defaults to False.

\item {} 
\sphinxAtStartPar
\sphinxstyleliteralstrong{\sphinxupquote{offset}} (\sphinxstyleliteralemphasis{\sphinxupquote{Optional}}\sphinxstyleliteralemphasis{\sphinxupquote{{[}}}{\hyperref[\detokenize{deepdrr.geo:deepdrr.geo.core.Vector3D}]{\sphinxcrossref{\sphinxstyleliteralemphasis{\sphinxupquote{Vector3D}}}}}\sphinxstyleliteralemphasis{\sphinxupquote{{]}}}\sphinxstyleliteralemphasis{\sphinxupquote{, }}\sphinxstyleliteralemphasis{\sphinxupquote{optional}}) \textendash{} world\sphinxhyphen{}space offset to add to the initial C\sphinxhyphen{}arm isocenter. Defaults to None.

\end{itemize}

\sphinxlineitem{Returns}
\sphinxAtStartPar
the extrinsic matrix or “camera3d\_from\_world” frame transformation for the oriented C\sphinxhyphen{}Arm camera.

\sphinxlineitem{Return type}
\sphinxAtStartPar
{\hyperref[\detokenize{deepdrr.geo:deepdrr.geo.core.FrameTransform}]{\sphinxcrossref{FrameTransform}}}

\end{description}\end{quote}

\end{fulllineitems}

\index{move\_by() (deepdrr.device.carm.CArm method)@\spxentry{move\_by()}\spxextra{deepdrr.device.carm.CArm method}}

\begin{fulllineitems}
\phantomsection\label{\detokenize{deepdrr.device:deepdrr.device.carm.CArm.move_by}}
\pysigstartsignatures
\pysiglinewithargsret{\sphinxbfcode{\sphinxupquote{move\_by}}}{\sphinxparam{\DUrole{n,n}{delta\_isocenter}\DUrole{p,p}{:}\DUrole{w,w}{  }\DUrole{n,n}{{\hyperref[\detokenize{deepdrr.geo:deepdrr.geo.core.Vector3D}]{\sphinxcrossref{Vector3D}}}\DUrole{w,w}{  }\DUrole{p,p}{|}\DUrole{w,w}{  }None}\DUrole{w,w}{  }\DUrole{o,o}{=}\DUrole{w,w}{  }\DUrole{default_value}{None}}\sphinxparamcomma \sphinxparam{\DUrole{n,n}{delta\_phi}\DUrole{p,p}{:}\DUrole{w,w}{  }\DUrole{n,n}{float\DUrole{w,w}{  }\DUrole{p,p}{|}\DUrole{w,w}{  }None}\DUrole{w,w}{  }\DUrole{o,o}{=}\DUrole{w,w}{  }\DUrole{default_value}{None}}\sphinxparamcomma \sphinxparam{\DUrole{n,n}{delta\_theta}\DUrole{p,p}{:}\DUrole{w,w}{  }\DUrole{n,n}{float\DUrole{w,w}{  }\DUrole{p,p}{|}\DUrole{w,w}{  }None}\DUrole{w,w}{  }\DUrole{o,o}{=}\DUrole{w,w}{  }\DUrole{default_value}{None}}\sphinxparamcomma \sphinxparam{\DUrole{n,n}{delta\_rho}\DUrole{p,p}{:}\DUrole{w,w}{  }\DUrole{n,n}{float\DUrole{w,w}{  }\DUrole{p,p}{|}\DUrole{w,w}{  }None}\DUrole{w,w}{  }\DUrole{o,o}{=}\DUrole{w,w}{  }\DUrole{default_value}{None}}\sphinxparamcomma \sphinxparam{\DUrole{n,n}{degrees}\DUrole{p,p}{:}\DUrole{w,w}{  }\DUrole{n,n}{bool}\DUrole{w,w}{  }\DUrole{o,o}{=}\DUrole{w,w}{  }\DUrole{default_value}{False}}\sphinxparamcomma \sphinxparam{\DUrole{n,n}{min\_isocenter}\DUrole{p,p}{:}\DUrole{w,w}{  }\DUrole{n,n}{{\hyperref[\detokenize{deepdrr.geo:deepdrr.geo.core.Point3D}]{\sphinxcrossref{Point3D}}}\DUrole{w,w}{  }\DUrole{p,p}{|}\DUrole{w,w}{  }None}\DUrole{w,w}{  }\DUrole{o,o}{=}\DUrole{w,w}{  }\DUrole{default_value}{None}}\sphinxparamcomma \sphinxparam{\DUrole{n,n}{max\_isocenter}\DUrole{p,p}{:}\DUrole{w,w}{  }\DUrole{n,n}{{\hyperref[\detokenize{deepdrr.geo:deepdrr.geo.core.Point3D}]{\sphinxcrossref{Point3D}}}\DUrole{w,w}{  }\DUrole{p,p}{|}\DUrole{w,w}{  }None}\DUrole{w,w}{  }\DUrole{o,o}{=}\DUrole{w,w}{  }\DUrole{default_value}{None}}\sphinxparamcomma \sphinxparam{\DUrole{n,n}{min\_phi}\DUrole{p,p}{:}\DUrole{w,w}{  }\DUrole{n,n}{float\DUrole{w,w}{  }\DUrole{p,p}{|}\DUrole{w,w}{  }None}\DUrole{w,w}{  }\DUrole{o,o}{=}\DUrole{w,w}{  }\DUrole{default_value}{None}}\sphinxparamcomma \sphinxparam{\DUrole{n,n}{max\_phi}\DUrole{p,p}{:}\DUrole{w,w}{  }\DUrole{n,n}{float\DUrole{w,w}{  }\DUrole{p,p}{|}\DUrole{w,w}{  }None}\DUrole{w,w}{  }\DUrole{o,o}{=}\DUrole{w,w}{  }\DUrole{default_value}{None}}\sphinxparamcomma \sphinxparam{\DUrole{n,n}{min\_theta}\DUrole{p,p}{:}\DUrole{w,w}{  }\DUrole{n,n}{float\DUrole{w,w}{  }\DUrole{p,p}{|}\DUrole{w,w}{  }None}\DUrole{w,w}{  }\DUrole{o,o}{=}\DUrole{w,w}{  }\DUrole{default_value}{None}}\sphinxparamcomma \sphinxparam{\DUrole{n,n}{max\_theta}\DUrole{p,p}{:}\DUrole{w,w}{  }\DUrole{n,n}{float\DUrole{w,w}{  }\DUrole{p,p}{|}\DUrole{w,w}{  }None}\DUrole{w,w}{  }\DUrole{o,o}{=}\DUrole{w,w}{  }\DUrole{default_value}{None}}}{{ $\rightarrow$ None}}
\pysigstopsignatures
\sphinxAtStartPar
Move the C\sphinxhyphen{}arm by the specified deltas.

\sphinxAtStartPar
Clips the internal state by the provided values if not None.
\begin{quote}\begin{description}
\sphinxlineitem{Parameters}\begin{itemize}
\item {} 
\sphinxAtStartPar
\sphinxstyleliteralstrong{\sphinxupquote{delta\_isocenter}} ({\hyperref[\detokenize{deepdrr.geo:deepdrr.geo.core.Vector3D}]{\sphinxcrossref{\sphinxstyleliteralemphasis{\sphinxupquote{Vector3D}}}}}) \textendash{} offset for the isocenter of the C\sphinxhyphen{}arm in world\sphinxhyphen{}space. This is the center about which rotations are performed.

\item {} 
\sphinxAtStartPar
\sphinxstyleliteralstrong{\sphinxupquote{phi}} (\sphinxstyleliteralemphasis{\sphinxupquote{float}}) \textendash{} CRAN/CAUD angle of the C\sphinxhyphen{}Arm (along the actual arc of the arm)

\item {} 
\sphinxAtStartPar
\sphinxstyleliteralstrong{\sphinxupquote{theta}} (\sphinxstyleliteralemphasis{\sphinxupquote{float}}) \textendash{} Lect/Right angulation of C\sphinxhyphen{}arm (rotation at the base)

\item {} 
\sphinxAtStartPar
\sphinxstyleliteralstrong{\sphinxupquote{rho}} (\sphinxstyleliteralemphasis{\sphinxupquote{float}}\sphinxstyleliteralemphasis{\sphinxupquote{, }}\sphinxstyleliteralemphasis{\sphinxupquote{optional}}) \textendash{} rotation about principle axis, after main rotation. Defaults to 0.

\item {} 
\sphinxAtStartPar
\sphinxstyleliteralstrong{\sphinxupquote{degrees}} (\sphinxstyleliteralemphasis{\sphinxupquote{bool}}\sphinxstyleliteralemphasis{\sphinxupquote{, }}\sphinxstyleliteralemphasis{\sphinxupquote{optional}}) \textendash{} Whether given angles are in degrees. Defaults to False.

\end{itemize}

\end{description}\end{quote}

\end{fulllineitems}

\index{move\_to() (deepdrr.device.carm.CArm method)@\spxentry{move\_to()}\spxextra{deepdrr.device.carm.CArm method}}

\begin{fulllineitems}
\phantomsection\label{\detokenize{deepdrr.device:deepdrr.device.carm.CArm.move_to}}
\pysigstartsignatures
\pysiglinewithargsret{\sphinxbfcode{\sphinxupquote{move\_to}}}{\sphinxparam{\DUrole{n,n}{isocenter}\DUrole{p,p}{:}\DUrole{w,w}{  }\DUrole{n,n}{{\hyperref[\detokenize{deepdrr.geo:deepdrr.geo.core.Point3D}]{\sphinxcrossref{Point3D}}}\DUrole{w,w}{  }\DUrole{p,p}{|}\DUrole{w,w}{  }None}\DUrole{w,w}{  }\DUrole{o,o}{=}\DUrole{w,w}{  }\DUrole{default_value}{None}}\sphinxparamcomma \sphinxparam{\DUrole{n,n}{phi}\DUrole{p,p}{:}\DUrole{w,w}{  }\DUrole{n,n}{float\DUrole{w,w}{  }\DUrole{p,p}{|}\DUrole{w,w}{  }None}\DUrole{w,w}{  }\DUrole{o,o}{=}\DUrole{w,w}{  }\DUrole{default_value}{None}}\sphinxparamcomma \sphinxparam{\DUrole{n,n}{theta}\DUrole{p,p}{:}\DUrole{w,w}{  }\DUrole{n,n}{float\DUrole{w,w}{  }\DUrole{p,p}{|}\DUrole{w,w}{  }None}\DUrole{w,w}{  }\DUrole{o,o}{=}\DUrole{w,w}{  }\DUrole{default_value}{None}}\sphinxparamcomma \sphinxparam{\DUrole{n,n}{rho}\DUrole{p,p}{:}\DUrole{w,w}{  }\DUrole{n,n}{float\DUrole{w,w}{  }\DUrole{p,p}{|}\DUrole{w,w}{  }None}\DUrole{w,w}{  }\DUrole{o,o}{=}\DUrole{w,w}{  }\DUrole{default_value}{None}}\sphinxparamcomma \sphinxparam{\DUrole{n,n}{degrees}\DUrole{p,p}{:}\DUrole{w,w}{  }\DUrole{n,n}{bool}\DUrole{w,w}{  }\DUrole{o,o}{=}\DUrole{w,w}{  }\DUrole{default_value}{False}}}{{ $\rightarrow$ None}}
\pysigstopsignatures
\sphinxAtStartPar
Move the C\sphinxhyphen{}arm to the specified pose.
\begin{quote}\begin{description}
\sphinxlineitem{Parameters}\begin{itemize}
\item {} 
\sphinxAtStartPar
\sphinxstyleliteralstrong{\sphinxupquote{isocenter}} ({\hyperref[\detokenize{deepdrr.geo:deepdrr.geo.core.Point3D}]{\sphinxcrossref{\sphinxstyleliteralemphasis{\sphinxupquote{Point3D}}}}}) \textendash{} New isocenter of the C\sphinxhyphen{}arm in device space. This is the center about which rotations are performed.

\item {} 
\sphinxAtStartPar
\sphinxstyleliteralstrong{\sphinxupquote{phi}} (\sphinxstyleliteralemphasis{\sphinxupquote{float}}) \textendash{} CRAN/CAUD angle of the C\sphinxhyphen{}Arm (along the actual arc of the arm)

\item {} 
\sphinxAtStartPar
\sphinxstyleliteralstrong{\sphinxupquote{theta}} (\sphinxstyleliteralemphasis{\sphinxupquote{float}}) \textendash{} Lect/Right angulation of C\sphinxhyphen{}arm (rotation at the base)

\item {} 
\sphinxAtStartPar
\sphinxstyleliteralstrong{\sphinxupquote{rho}} (\sphinxstyleliteralemphasis{\sphinxupquote{float}}\sphinxstyleliteralemphasis{\sphinxupquote{, }}\sphinxstyleliteralemphasis{\sphinxupquote{optional}}) \textendash{} rotation about principle axis, after main rotation. Defaults to 0.

\item {} 
\sphinxAtStartPar
\sphinxstyleliteralstrong{\sphinxupquote{degrees}} (\sphinxstyleliteralemphasis{\sphinxupquote{bool}}\sphinxstyleliteralemphasis{\sphinxupquote{, }}\sphinxstyleliteralemphasis{\sphinxupquote{optional}}) \textendash{} Whether given angles are in degrees. Defaults to False.

\end{itemize}

\end{description}\end{quote}

\end{fulllineitems}


\end{fulllineitems}

\index{make\_detector\_rotation() (in module deepdrr.device.carm)@\spxentry{make\_detector\_rotation()}\spxextra{in module deepdrr.device.carm}}

\begin{fulllineitems}
\phantomsection\label{\detokenize{deepdrr.device:deepdrr.device.carm.make_detector_rotation}}
\pysigstartsignatures
\pysiglinewithargsret{\sphinxcode{\sphinxupquote{deepdrr.device.carm.}}\sphinxbfcode{\sphinxupquote{make\_detector\_rotation}}}{\sphinxparam{\DUrole{n,n}{phi}\DUrole{p,p}{:}\DUrole{w,w}{  }\DUrole{n,n}{float}}\sphinxparamcomma \sphinxparam{\DUrole{n,n}{theta}\DUrole{p,p}{:}\DUrole{w,w}{  }\DUrole{n,n}{float}}\sphinxparamcomma \sphinxparam{\DUrole{n,n}{rho}\DUrole{p,p}{:}\DUrole{w,w}{  }\DUrole{n,n}{float}}}{}
\pysigstopsignatures
\sphinxAtStartPar
Make the rotation matrix for a CArm detector at the given angles.
\begin{quote}\begin{description}
\sphinxlineitem{Parameters}\begin{itemize}
\item {} 
\sphinxAtStartPar
\sphinxstyleliteralstrong{\sphinxupquote{phi}} (\sphinxstyleliteralemphasis{\sphinxupquote{float}}) \textendash{} phi.

\item {} 
\sphinxAtStartPar
\sphinxstyleliteralstrong{\sphinxupquote{theta}} (\sphinxstyleliteralemphasis{\sphinxupquote{float}}) \textendash{} theta.

\item {} 
\sphinxAtStartPar
\sphinxstyleliteralstrong{\sphinxupquote{rho}} (\sphinxstyleliteralemphasis{\sphinxupquote{float}}) \textendash{} rho

\end{itemize}

\sphinxlineitem{Returns}
\sphinxAtStartPar
Rotation matrix.

\sphinxlineitem{Return type}
\sphinxAtStartPar
np.ndarray

\end{description}\end{quote}

\end{fulllineitems}



\subsection{deepdrr.device.device}
\label{\detokenize{deepdrr.device:module-deepdrr.device.device}}\label{\detokenize{deepdrr.device:deepdrr-device-device}}\index{module@\spxentry{module}!deepdrr.device.device@\spxentry{deepdrr.device.device}}\index{deepdrr.device.device@\spxentry{deepdrr.device.device}!module@\spxentry{module}}\index{Device (class in deepdrr.device.device)@\spxentry{Device}\spxextra{class in deepdrr.device.device}}

\begin{fulllineitems}
\phantomsection\label{\detokenize{deepdrr.device:deepdrr.device.device.Device}}
\pysigstartsignatures
\pysigline{\sphinxbfcode{\sphinxupquote{class\DUrole{w,w}{  }}}\sphinxcode{\sphinxupquote{deepdrr.device.device.}}\sphinxbfcode{\sphinxupquote{Device}}}
\pysigstopsignatures
\sphinxAtStartPar
Bases: \sphinxcode{\sphinxupquote{ABC}}

\sphinxAtStartPar
A parent class representing X\sphinxhyphen{}ray device interfaces in DeepDRR.
\begin{description}
\sphinxlineitem{To implement a sub class, the following methods/attributes must be implemented:}\begin{itemize}
\item {} 
\sphinxAtStartPar
device\_from\_camera3d

\end{itemize}

\end{description}
\index{sensor\_height (deepdrr.device.device.Device attribute)@\spxentry{sensor\_height}\spxextra{deepdrr.device.device.Device attribute}}

\begin{fulllineitems}
\phantomsection\label{\detokenize{deepdrr.device:deepdrr.device.device.Device.sensor_height}}
\pysigstartsignatures
\pysigline{\sphinxbfcode{\sphinxupquote{sensor\_height}}}
\pysigstopsignatures
\sphinxAtStartPar
the height of the sensor in pixels.
\begin{quote}\begin{description}
\sphinxlineitem{Type}
\sphinxAtStartPar
int

\end{description}\end{quote}

\end{fulllineitems}

\index{sensor\_width (deepdrr.device.device.Device attribute)@\spxentry{sensor\_width}\spxextra{deepdrr.device.device.Device attribute}}

\begin{fulllineitems}
\phantomsection\label{\detokenize{deepdrr.device:deepdrr.device.device.Device.sensor_width}}
\pysigstartsignatures
\pysigline{\sphinxbfcode{\sphinxupquote{sensor\_width}}}
\pysigstopsignatures
\sphinxAtStartPar
the width of the sensor in pixels.
\begin{quote}\begin{description}
\sphinxlineitem{Type}
\sphinxAtStartPar
int

\end{description}\end{quote}

\end{fulllineitems}

\index{pixel\_size (deepdrr.device.device.Device attribute)@\spxentry{pixel\_size}\spxextra{deepdrr.device.device.Device attribute}}

\begin{fulllineitems}
\phantomsection\label{\detokenize{deepdrr.device:deepdrr.device.device.Device.pixel_size}}
\pysigstartsignatures
\pysigline{\sphinxbfcode{\sphinxupquote{pixel\_size}}}
\pysigstopsignatures
\sphinxAtStartPar
the size of a pixel in mm.
\begin{quote}\begin{description}
\sphinxlineitem{Type}
\sphinxAtStartPar
float

\end{description}\end{quote}

\end{fulllineitems}

\index{camera3d\_from\_device (deepdrr.device.device.Device property)@\spxentry{camera3d\_from\_device}\spxextra{deepdrr.device.device.Device property}}

\begin{fulllineitems}
\phantomsection\label{\detokenize{deepdrr.device:deepdrr.device.device.Device.camera3d_from_device}}
\pysigstartsignatures
\pysigline{\sphinxbfcode{\sphinxupquote{property\DUrole{w,w}{  }}}\sphinxbfcode{\sphinxupquote{camera3d\_from\_device}}\sphinxbfcode{\sphinxupquote{\DUrole{p,p}{:}\DUrole{w,w}{  }{\hyperref[\detokenize{deepdrr.geo:deepdrr.geo.core.FrameTransform}]{\sphinxcrossref{FrameTransform}}}}}}
\pysigstopsignatures
\sphinxAtStartPar
Get the FrameTransform for the device’s camera3d\_from\_device frame (in the current pose).
\begin{quote}\begin{description}
\sphinxlineitem{Returns}
\sphinxAtStartPar
the “camera3d\_from\_device” frame transformation for the device.

\sphinxlineitem{Return type}
\sphinxAtStartPar
{\hyperref[\detokenize{deepdrr.geo:deepdrr.geo.core.FrameTransform}]{\sphinxcrossref{FrameTransform}}}

\end{description}\end{quote}

\end{fulllineitems}

\index{camera3d\_from\_index (deepdrr.device.device.Device property)@\spxentry{camera3d\_from\_index}\spxextra{deepdrr.device.device.Device property}}

\begin{fulllineitems}
\phantomsection\label{\detokenize{deepdrr.device:deepdrr.device.device.Device.camera3d_from_index}}
\pysigstartsignatures
\pysigline{\sphinxbfcode{\sphinxupquote{property\DUrole{w,w}{  }}}\sphinxbfcode{\sphinxupquote{camera3d\_from\_index}}\sphinxbfcode{\sphinxupquote{\DUrole{p,p}{:}\DUrole{w,w}{  }{\hyperref[\detokenize{deepdrr.geo:deepdrr.geo.core.Transform}]{\sphinxcrossref{Transform}}}}}}
\pysigstopsignatures
\end{fulllineitems}

\index{camera3d\_from\_world (deepdrr.device.device.Device property)@\spxentry{camera3d\_from\_world}\spxextra{deepdrr.device.device.Device property}}

\begin{fulllineitems}
\phantomsection\label{\detokenize{deepdrr.device:deepdrr.device.device.Device.camera3d_from_world}}
\pysigstartsignatures
\pysigline{\sphinxbfcode{\sphinxupquote{property\DUrole{w,w}{  }}}\sphinxbfcode{\sphinxupquote{camera3d\_from\_world}}\sphinxbfcode{\sphinxupquote{\DUrole{p,p}{:}\DUrole{w,w}{  }{\hyperref[\detokenize{deepdrr.geo:deepdrr.geo.core.FrameTransform}]{\sphinxcrossref{FrameTransform}}}}}}
\pysigstopsignatures
\sphinxAtStartPar
Get the FrameTransform for the device’s camera3d\_from\_world frame (in the current pose).
\begin{quote}\begin{description}
\sphinxlineitem{Returns}
\sphinxAtStartPar
the “camera3d\_from\_world” frame transformation for the device.

\sphinxlineitem{Return type}
\sphinxAtStartPar
{\hyperref[\detokenize{deepdrr.geo:deepdrr.geo.core.FrameTransform}]{\sphinxcrossref{FrameTransform}}}

\end{description}\end{quote}

\end{fulllineitems}

\index{camera\_intrinsics (deepdrr.device.device.Device attribute)@\spxentry{camera\_intrinsics}\spxextra{deepdrr.device.device.Device attribute}}

\begin{fulllineitems}
\phantomsection\label{\detokenize{deepdrr.device:deepdrr.device.device.Device.camera_intrinsics}}
\pysigstartsignatures
\pysigline{\sphinxbfcode{\sphinxupquote{camera\_intrinsics}}\sphinxbfcode{\sphinxupquote{\DUrole{p,p}{:}\DUrole{w,w}{  }{\hyperref[\detokenize{deepdrr.geo:deepdrr.geo.core.CameraIntrinsicTransform}]{\sphinxcrossref{CameraIntrinsicTransform}}}}}}
\pysigstopsignatures
\end{fulllineitems}

\index{detector\_height (deepdrr.device.device.Device property)@\spxentry{detector\_height}\spxextra{deepdrr.device.device.Device property}}

\begin{fulllineitems}
\phantomsection\label{\detokenize{deepdrr.device:deepdrr.device.device.Device.detector_height}}
\pysigstartsignatures
\pysigline{\sphinxbfcode{\sphinxupquote{property\DUrole{w,w}{  }}}\sphinxbfcode{\sphinxupquote{detector\_height}}\sphinxbfcode{\sphinxupquote{\DUrole{p,p}{:}\DUrole{w,w}{  }float}}}
\pysigstopsignatures
\sphinxAtStartPar
Height of the detector in mm.

\end{fulllineitems}

\index{detector\_width (deepdrr.device.device.Device property)@\spxentry{detector\_width}\spxextra{deepdrr.device.device.Device property}}

\begin{fulllineitems}
\phantomsection\label{\detokenize{deepdrr.device:deepdrr.device.device.Device.detector_width}}
\pysigstartsignatures
\pysigline{\sphinxbfcode{\sphinxupquote{property\DUrole{w,w}{  }}}\sphinxbfcode{\sphinxupquote{detector\_width}}\sphinxbfcode{\sphinxupquote{\DUrole{p,p}{:}\DUrole{w,w}{  }float}}}
\pysigstopsignatures
\sphinxAtStartPar
Width of the detector in mm.

\end{fulllineitems}

\index{device\_from\_camera3d (deepdrr.device.device.Device property)@\spxentry{device\_from\_camera3d}\spxextra{deepdrr.device.device.Device property}}

\begin{fulllineitems}
\phantomsection\label{\detokenize{deepdrr.device:deepdrr.device.device.Device.device_from_camera3d}}
\pysigstartsignatures
\pysigline{\sphinxbfcode{\sphinxupquote{abstract\DUrole{w,w}{  }property\DUrole{w,w}{  }}}\sphinxbfcode{\sphinxupquote{device\_from\_camera3d}}\sphinxbfcode{\sphinxupquote{\DUrole{p,p}{:}\DUrole{w,w}{  }{\hyperref[\detokenize{deepdrr.geo:deepdrr.geo.core.FrameTransform}]{\sphinxcrossref{FrameTransform}}}}}}
\pysigstopsignatures
\sphinxAtStartPar
Get the FrameTransform for the device’s camera3d\_from\_device frame (in the current pose).
\begin{quote}\begin{description}
\sphinxlineitem{Parameters}
\sphinxAtStartPar
\sphinxstyleliteralstrong{\sphinxupquote{camera3d\_transform}} ({\hyperref[\detokenize{deepdrr.geo:deepdrr.geo.core.FrameTransform}]{\sphinxcrossref{\sphinxstyleliteralemphasis{\sphinxupquote{FrameTransform}}}}}) \textendash{} the “camera3d\_from\_device” frame transformation for the device.

\sphinxlineitem{Returns}
\sphinxAtStartPar
the “device\_from\_camera3d” frame transformation for the device.

\sphinxlineitem{Return type}
\sphinxAtStartPar
{\hyperref[\detokenize{deepdrr.geo:deepdrr.geo.core.FrameTransform}]{\sphinxcrossref{FrameTransform}}}

\end{description}\end{quote}

\end{fulllineitems}

\index{device\_from\_world (deepdrr.device.device.Device property)@\spxentry{device\_from\_world}\spxextra{deepdrr.device.device.Device property}}

\begin{fulllineitems}
\phantomsection\label{\detokenize{deepdrr.device:deepdrr.device.device.Device.device_from_world}}
\pysigstartsignatures
\pysigline{\sphinxbfcode{\sphinxupquote{property\DUrole{w,w}{  }}}\sphinxbfcode{\sphinxupquote{device\_from\_world}}\sphinxbfcode{\sphinxupquote{\DUrole{p,p}{:}\DUrole{w,w}{  }{\hyperref[\detokenize{deepdrr.geo:deepdrr.geo.core.FrameTransform}]{\sphinxcrossref{FrameTransform}}}}}}
\pysigstopsignatures
\sphinxAtStartPar
Get the FrameTransform for the device’s local frame.
\begin{quote}\begin{description}
\sphinxlineitem{Parameters}
\sphinxAtStartPar
\sphinxstyleliteralstrong{\sphinxupquote{world\_transform}} ({\hyperref[\detokenize{deepdrr.geo:deepdrr.geo.core.FrameTransform}]{\sphinxcrossref{\sphinxstyleliteralemphasis{\sphinxupquote{FrameTransform}}}}}) \textendash{} the “world\_from\_device” frame transformation for the device.

\sphinxlineitem{Returns}
\sphinxAtStartPar
the “device\_from\_world” frame transformation for the device.

\sphinxlineitem{Return type}
\sphinxAtStartPar
{\hyperref[\detokenize{deepdrr.geo:deepdrr.geo.core.FrameTransform}]{\sphinxcrossref{FrameTransform}}}

\end{description}\end{quote}

\end{fulllineitems}

\index{get\_camera\_projection() (deepdrr.device.device.Device method)@\spxentry{get\_camera\_projection()}\spxextra{deepdrr.device.device.Device method}}

\begin{fulllineitems}
\phantomsection\label{\detokenize{deepdrr.device:deepdrr.device.device.Device.get_camera_projection}}
\pysigstartsignatures
\pysiglinewithargsret{\sphinxbfcode{\sphinxupquote{get\_camera\_projection}}}{}{{ $\rightarrow$ {\hyperref[\detokenize{deepdrr.geo:deepdrr.geo.core.CameraProjection}]{\sphinxcrossref{CameraProjection}}}}}
\pysigstopsignatures
\sphinxAtStartPar
Get the camera projection for the device in the current pose.
\begin{quote}\begin{description}
\sphinxlineitem{Returns}
\sphinxAtStartPar
the “index\_from\_world” camera projection for the device.

\sphinxlineitem{Return type}
\sphinxAtStartPar
{\hyperref[\detokenize{deepdrr.geo:deepdrr.geo.core.CameraProjection}]{\sphinxcrossref{CameraProjection}}}

\end{description}\end{quote}

\end{fulllineitems}

\index{get\_mesh\_in\_world() (deepdrr.device.device.Device method)@\spxentry{get\_mesh\_in\_world()}\spxextra{deepdrr.device.device.Device method}}

\begin{fulllineitems}
\phantomsection\label{\detokenize{deepdrr.device:deepdrr.device.device.Device.get_mesh_in_world}}
\pysigstartsignatures
\pysiglinewithargsret{\sphinxbfcode{\sphinxupquote{get\_mesh\_in\_world}}}{\sphinxparam{\DUrole{n,n}{full}\DUrole{o,o}{=}\DUrole{default_value}{False}}\sphinxparamcomma \sphinxparam{\DUrole{n,n}{use\_cached}\DUrole{o,o}{=}\DUrole{default_value}{True}}}{}
\pysigstopsignatures
\sphinxAtStartPar
Get a really simple camera mesh for the device in the current pose.

\sphinxAtStartPar
Subclasses may want to override this with more detailed meshes (full=True).

\end{fulllineitems}

\index{index\_from\_camera3d (deepdrr.device.device.Device property)@\spxentry{index\_from\_camera3d}\spxextra{deepdrr.device.device.Device property}}

\begin{fulllineitems}
\phantomsection\label{\detokenize{deepdrr.device:deepdrr.device.device.Device.index_from_camera3d}}
\pysigstartsignatures
\pysigline{\sphinxbfcode{\sphinxupquote{property\DUrole{w,w}{  }}}\sphinxbfcode{\sphinxupquote{index\_from\_camera3d}}\sphinxbfcode{\sphinxupquote{\DUrole{p,p}{:}\DUrole{w,w}{  }{\hyperref[\detokenize{deepdrr.geo:deepdrr.geo.core.CameraProjection}]{\sphinxcrossref{CameraProjection}}}}}}
\pysigstopsignatures
\sphinxAtStartPar
Get the CameraIntrinsicTransform for the device’s camera3d\_from\_index frame (in the current pose).
\begin{quote}\begin{description}
\sphinxlineitem{Returns}
\sphinxAtStartPar
the “index\_from\_camera3d” frame transformation for the device.

\sphinxlineitem{Return type}
\sphinxAtStartPar
{\hyperref[\detokenize{deepdrr.geo:deepdrr.geo.core.CameraIntrinsicTransform}]{\sphinxcrossref{CameraIntrinsicTransform}}}

\end{description}\end{quote}

\end{fulllineitems}

\index{index\_from\_world (deepdrr.device.device.Device property)@\spxentry{index\_from\_world}\spxextra{deepdrr.device.device.Device property}}

\begin{fulllineitems}
\phantomsection\label{\detokenize{deepdrr.device:deepdrr.device.device.Device.index_from_world}}
\pysigstartsignatures
\pysigline{\sphinxbfcode{\sphinxupquote{property\DUrole{w,w}{  }}}\sphinxbfcode{\sphinxupquote{index\_from\_world}}\sphinxbfcode{\sphinxupquote{\DUrole{p,p}{:}\DUrole{w,w}{  }{\hyperref[\detokenize{deepdrr.geo:deepdrr.geo.core.CameraProjection}]{\sphinxcrossref{CameraProjection}}}}}}
\pysigstopsignatures
\sphinxAtStartPar
Get the camera projection for the device in the current pose.
\begin{quote}\begin{description}
\sphinxlineitem{Returns}
\sphinxAtStartPar
the “index\_from\_world” camera projection for the device.

\sphinxlineitem{Return type}
\sphinxAtStartPar
{\hyperref[\detokenize{deepdrr.geo:deepdrr.geo.core.CameraProjection}]{\sphinxcrossref{CameraProjection}}}

\end{description}\end{quote}

\end{fulllineitems}

\index{pixel\_size (deepdrr.device.device.Device attribute)@\spxentry{pixel\_size}\spxextra{deepdrr.device.device.Device attribute}}

\begin{fulllineitems}
\phantomsection\label{\detokenize{deepdrr.device:id0}}
\pysigstartsignatures
\pysigline{\sphinxbfcode{\sphinxupquote{pixel\_size}}\sphinxbfcode{\sphinxupquote{\DUrole{p,p}{:}\DUrole{w,w}{  }float}}}
\pysigstopsignatures
\end{fulllineitems}

\index{principle\_ray (deepdrr.device.device.Device property)@\spxentry{principle\_ray}\spxextra{deepdrr.device.device.Device property}}

\begin{fulllineitems}
\phantomsection\label{\detokenize{deepdrr.device:deepdrr.device.device.Device.principle_ray}}
\pysigstartsignatures
\pysigline{\sphinxbfcode{\sphinxupquote{property\DUrole{w,w}{  }}}\sphinxbfcode{\sphinxupquote{principle\_ray}}\sphinxbfcode{\sphinxupquote{\DUrole{p,p}{:}\DUrole{w,w}{  }{\hyperref[\detokenize{deepdrr.geo:deepdrr.geo.core.Vector3D}]{\sphinxcrossref{Vector3D}}}}}}
\pysigstopsignatures
\sphinxAtStartPar
Get the principle ray for the device in the current pose in the device frame.

\sphinxAtStartPar
The principle ray is the direction of the ray that passes through the center of the
image. It points from the source toward the detector.

\sphinxAtStartPar
By default, this is just the z axis, but this can be overridden by sub classes.
\begin{quote}\begin{description}
\sphinxlineitem{Returns}
\sphinxAtStartPar
the principle ray for the device as a unit vector.

\sphinxlineitem{Return type}
\sphinxAtStartPar
{\hyperref[\detokenize{deepdrr.geo:deepdrr.geo.core.Vector3D}]{\sphinxcrossref{Vector3D}}}

\end{description}\end{quote}

\end{fulllineitems}

\index{principle\_ray\_in\_world (deepdrr.device.device.Device property)@\spxentry{principle\_ray\_in\_world}\spxextra{deepdrr.device.device.Device property}}

\begin{fulllineitems}
\phantomsection\label{\detokenize{deepdrr.device:deepdrr.device.device.Device.principle_ray_in_world}}
\pysigstartsignatures
\pysigline{\sphinxbfcode{\sphinxupquote{property\DUrole{w,w}{  }}}\sphinxbfcode{\sphinxupquote{principle\_ray\_in\_world}}\sphinxbfcode{\sphinxupquote{\DUrole{p,p}{:}\DUrole{w,w}{  }{\hyperref[\detokenize{deepdrr.geo:deepdrr.geo.core.Vector3D}]{\sphinxcrossref{Vector3D}}}}}}
\pysigstopsignatures
\sphinxAtStartPar
Get the principle ray for the device in the current pose in the world frame.

\sphinxAtStartPar
The principle ray is the direction of the ray that passes through the center of the
image. It points from the source toward the detector.
\begin{quote}\begin{description}
\sphinxlineitem{Returns}
\sphinxAtStartPar
the principle ray for the device as a unit vector.

\sphinxlineitem{Return type}
\sphinxAtStartPar
{\hyperref[\detokenize{deepdrr.geo:deepdrr.geo.core.Vector3D}]{\sphinxcrossref{Vector3D}}}

\end{description}\end{quote}

\end{fulllineitems}

\index{sensor\_height (deepdrr.device.device.Device attribute)@\spxentry{sensor\_height}\spxextra{deepdrr.device.device.Device attribute}}

\begin{fulllineitems}
\phantomsection\label{\detokenize{deepdrr.device:id1}}
\pysigstartsignatures
\pysigline{\sphinxbfcode{\sphinxupquote{sensor\_height}}\sphinxbfcode{\sphinxupquote{\DUrole{p,p}{:}\DUrole{w,w}{  }int}}}
\pysigstopsignatures
\end{fulllineitems}

\index{sensor\_width (deepdrr.device.device.Device attribute)@\spxentry{sensor\_width}\spxextra{deepdrr.device.device.Device attribute}}

\begin{fulllineitems}
\phantomsection\label{\detokenize{deepdrr.device:id2}}
\pysigstartsignatures
\pysigline{\sphinxbfcode{\sphinxupquote{sensor\_width}}\sphinxbfcode{\sphinxupquote{\DUrole{p,p}{:}\DUrole{w,w}{  }int}}}
\pysigstopsignatures
\end{fulllineitems}

\index{source\_in\_world (deepdrr.device.device.Device property)@\spxentry{source\_in\_world}\spxextra{deepdrr.device.device.Device property}}

\begin{fulllineitems}
\phantomsection\label{\detokenize{deepdrr.device:deepdrr.device.device.Device.source_in_world}}
\pysigstartsignatures
\pysigline{\sphinxbfcode{\sphinxupquote{property\DUrole{w,w}{  }}}\sphinxbfcode{\sphinxupquote{source\_in\_world}}\sphinxbfcode{\sphinxupquote{\DUrole{p,p}{:}\DUrole{w,w}{  }{\hyperref[\detokenize{deepdrr.geo:deepdrr.geo.core.Point3D}]{\sphinxcrossref{Point3D}}}}}}
\pysigstopsignatures
\end{fulllineitems}

\index{source\_to\_detector\_distance (deepdrr.device.device.Device attribute)@\spxentry{source\_to\_detector\_distance}\spxextra{deepdrr.device.device.Device attribute}}

\begin{fulllineitems}
\phantomsection\label{\detokenize{deepdrr.device:deepdrr.device.device.Device.source_to_detector_distance}}
\pysigstartsignatures
\pysigline{\sphinxbfcode{\sphinxupquote{source\_to\_detector\_distance}}\sphinxbfcode{\sphinxupquote{\DUrole{p,p}{:}\DUrole{w,w}{  }float}}}
\pysigstopsignatures
\end{fulllineitems}

\index{world\_from\_camera3d (deepdrr.device.device.Device property)@\spxentry{world\_from\_camera3d}\spxextra{deepdrr.device.device.Device property}}

\begin{fulllineitems}
\phantomsection\label{\detokenize{deepdrr.device:deepdrr.device.device.Device.world_from_camera3d}}
\pysigstartsignatures
\pysigline{\sphinxbfcode{\sphinxupquote{property\DUrole{w,w}{  }}}\sphinxbfcode{\sphinxupquote{world\_from\_camera3d}}\sphinxbfcode{\sphinxupquote{\DUrole{p,p}{:}\DUrole{w,w}{  }{\hyperref[\detokenize{deepdrr.geo:deepdrr.geo.core.FrameTransform}]{\sphinxcrossref{FrameTransform}}}}}}
\pysigstopsignatures
\sphinxAtStartPar
Get the FrameTransform for the device’s camera3d\_from\_world frame (in the current pose).
\begin{quote}\begin{description}
\sphinxlineitem{Returns}
\sphinxAtStartPar
the “world\_from\_camera3d” frame transformation for the device.

\sphinxlineitem{Return type}
\sphinxAtStartPar
{\hyperref[\detokenize{deepdrr.geo:deepdrr.geo.core.FrameTransform}]{\sphinxcrossref{FrameTransform}}}

\end{description}\end{quote}

\end{fulllineitems}

\index{world\_from\_device (deepdrr.device.device.Device attribute)@\spxentry{world\_from\_device}\spxextra{deepdrr.device.device.Device attribute}}

\begin{fulllineitems}
\phantomsection\label{\detokenize{deepdrr.device:deepdrr.device.device.Device.world_from_device}}
\pysigstartsignatures
\pysigline{\sphinxbfcode{\sphinxupquote{world\_from\_device}}\sphinxbfcode{\sphinxupquote{\DUrole{p,p}{:}\DUrole{w,w}{  }{\hyperref[\detokenize{deepdrr.geo:deepdrr.geo.core.FrameTransform}]{\sphinxcrossref{FrameTransform}}}}}}
\pysigstopsignatures
\end{fulllineitems}

\index{world\_from\_index (deepdrr.device.device.Device property)@\spxentry{world\_from\_index}\spxextra{deepdrr.device.device.Device property}}

\begin{fulllineitems}
\phantomsection\label{\detokenize{deepdrr.device:deepdrr.device.device.Device.world_from_index}}
\pysigstartsignatures
\pysigline{\sphinxbfcode{\sphinxupquote{property\DUrole{w,w}{  }}}\sphinxbfcode{\sphinxupquote{world\_from\_index}}\sphinxbfcode{\sphinxupquote{\DUrole{p,p}{:}\DUrole{w,w}{  }{\hyperref[\detokenize{deepdrr.geo:deepdrr.geo.core.Transform}]{\sphinxcrossref{Transform}}}}}}
\pysigstopsignatures
\sphinxAtStartPar
Get the world\_from\_index transform for the device in the current pose.
\begin{quote}\begin{description}
\sphinxlineitem{Returns}
\sphinxAtStartPar
the “world\_from\_index” transform for the device.

\sphinxlineitem{Return type}
\sphinxAtStartPar
{\hyperref[\detokenize{deepdrr.geo:deepdrr.geo.core.Transform}]{\sphinxcrossref{Transform}}}

\end{description}\end{quote}

\end{fulllineitems}


\end{fulllineitems}



\subsection{deepdrr.device.mobile\_carm}
\label{\detokenize{deepdrr.device:module-deepdrr.device.mobile_carm}}\label{\detokenize{deepdrr.device:deepdrr-device-mobile-carm}}\index{module@\spxentry{module}!deepdrr.device.mobile\_carm@\spxentry{deepdrr.device.mobile\_carm}}\index{deepdrr.device.mobile\_carm@\spxentry{deepdrr.device.mobile\_carm}!module@\spxentry{module}}\index{MobileCArm (class in deepdrr.device.mobile\_carm)@\spxentry{MobileCArm}\spxextra{class in deepdrr.device.mobile\_carm}}

\begin{fulllineitems}
\phantomsection\label{\detokenize{deepdrr.device:deepdrr.device.mobile_carm.MobileCArm}}
\pysigstartsignatures
\pysiglinewithargsret{\sphinxbfcode{\sphinxupquote{class\DUrole{w,w}{  }}}\sphinxcode{\sphinxupquote{deepdrr.device.mobile\_carm.}}\sphinxbfcode{\sphinxupquote{MobileCArm}}}{\sphinxparam{\DUrole{n,n}{world\_from\_device}\DUrole{p,p}{:}\DUrole{w,w}{  }\DUrole{n,n}{{\hyperref[\detokenize{deepdrr.geo:deepdrr.geo.core.FrameTransform}]{\sphinxcrossref{FrameTransform}}}\DUrole{w,w}{  }\DUrole{p,p}{|}\DUrole{w,w}{  }None}\DUrole{w,w}{  }\DUrole{o,o}{=}\DUrole{w,w}{  }\DUrole{default_value}{None}}\sphinxparamcomma \sphinxparam{\DUrole{n,n}{isocenter}\DUrole{p,p}{:}\DUrole{w,w}{  }\DUrole{n,n}{{\hyperref[\detokenize{deepdrr.geo:deepdrr.geo.core.Point3D}]{\sphinxcrossref{Point3D}}}}\DUrole{w,w}{  }\DUrole{o,o}{=}\DUrole{w,w}{  }\DUrole{default_value}{{[}0, 0, 0{]}}}\sphinxparamcomma \sphinxparam{\DUrole{n,n}{alpha}\DUrole{p,p}{:}\DUrole{w,w}{  }\DUrole{n,n}{float}\DUrole{w,w}{  }\DUrole{o,o}{=}\DUrole{w,w}{  }\DUrole{default_value}{0}}\sphinxparamcomma \sphinxparam{\DUrole{n,n}{beta}\DUrole{p,p}{:}\DUrole{w,w}{  }\DUrole{n,n}{float}\DUrole{w,w}{  }\DUrole{o,o}{=}\DUrole{w,w}{  }\DUrole{default_value}{0}}\sphinxparamcomma \sphinxparam{\DUrole{n,n}{gamma}\DUrole{p,p}{:}\DUrole{w,w}{  }\DUrole{n,n}{float}\DUrole{w,w}{  }\DUrole{o,o}{=}\DUrole{w,w}{  }\DUrole{default_value}{0}}\sphinxparamcomma \sphinxparam{\DUrole{n,n}{degrees}\DUrole{p,p}{:}\DUrole{w,w}{  }\DUrole{n,n}{bool}\DUrole{w,w}{  }\DUrole{o,o}{=}\DUrole{w,w}{  }\DUrole{default_value}{True}}\sphinxparamcomma \sphinxparam{\DUrole{n,n}{horizontal\_movement}\DUrole{p,p}{:}\DUrole{w,w}{  }\DUrole{n,n}{float}\DUrole{w,w}{  }\DUrole{o,o}{=}\DUrole{w,w}{  }\DUrole{default_value}{200}}\sphinxparamcomma \sphinxparam{\DUrole{n,n}{vertical\_travel}\DUrole{p,p}{:}\DUrole{w,w}{  }\DUrole{n,n}{float}\DUrole{w,w}{  }\DUrole{o,o}{=}\DUrole{w,w}{  }\DUrole{default_value}{430}}\sphinxparamcomma \sphinxparam{\DUrole{n,n}{min\_alpha}\DUrole{p,p}{:}\DUrole{w,w}{  }\DUrole{n,n}{float}\DUrole{w,w}{  }\DUrole{o,o}{=}\DUrole{w,w}{  }\DUrole{default_value}{\sphinxhyphen{}40}}\sphinxparamcomma \sphinxparam{\DUrole{n,n}{max\_alpha}\DUrole{p,p}{:}\DUrole{w,w}{  }\DUrole{n,n}{float}\DUrole{w,w}{  }\DUrole{o,o}{=}\DUrole{w,w}{  }\DUrole{default_value}{110}}\sphinxparamcomma \sphinxparam{\DUrole{n,n}{min\_beta}\DUrole{p,p}{:}\DUrole{w,w}{  }\DUrole{n,n}{float}\DUrole{w,w}{  }\DUrole{o,o}{=}\DUrole{w,w}{  }\DUrole{default_value}{\sphinxhyphen{}225}}\sphinxparamcomma \sphinxparam{\DUrole{n,n}{max\_beta}\DUrole{p,p}{:}\DUrole{w,w}{  }\DUrole{n,n}{float}\DUrole{w,w}{  }\DUrole{o,o}{=}\DUrole{w,w}{  }\DUrole{default_value}{225}}\sphinxparamcomma \sphinxparam{\DUrole{n,n}{source\_to\_detector\_distance}\DUrole{p,p}{:}\DUrole{w,w}{  }\DUrole{n,n}{float}\DUrole{w,w}{  }\DUrole{o,o}{=}\DUrole{w,w}{  }\DUrole{default_value}{1020}}\sphinxparamcomma \sphinxparam{\DUrole{n,n}{source\_to\_isocenter\_vertical\_distance}\DUrole{p,p}{:}\DUrole{w,w}{  }\DUrole{n,n}{float}\DUrole{w,w}{  }\DUrole{o,o}{=}\DUrole{w,w}{  }\DUrole{default_value}{530}}\sphinxparamcomma \sphinxparam{\DUrole{n,n}{source\_to\_isocenter\_horizontal\_offset}\DUrole{p,p}{:}\DUrole{w,w}{  }\DUrole{n,n}{float}\DUrole{w,w}{  }\DUrole{o,o}{=}\DUrole{w,w}{  }\DUrole{default_value}{0}}\sphinxparamcomma \sphinxparam{\DUrole{n,n}{immersion\_depth}\DUrole{p,p}{:}\DUrole{w,w}{  }\DUrole{n,n}{float}\DUrole{w,w}{  }\DUrole{o,o}{=}\DUrole{w,w}{  }\DUrole{default_value}{730}}\sphinxparamcomma \sphinxparam{\DUrole{n,n}{free\_space}\DUrole{p,p}{:}\DUrole{w,w}{  }\DUrole{n,n}{float}\DUrole{w,w}{  }\DUrole{o,o}{=}\DUrole{w,w}{  }\DUrole{default_value}{820}}\sphinxparamcomma \sphinxparam{\DUrole{n,n}{sensor\_height}\DUrole{p,p}{:}\DUrole{w,w}{  }\DUrole{n,n}{int}\DUrole{w,w}{  }\DUrole{o,o}{=}\DUrole{w,w}{  }\DUrole{default_value}{1536}}\sphinxparamcomma \sphinxparam{\DUrole{n,n}{sensor\_width}\DUrole{p,p}{:}\DUrole{w,w}{  }\DUrole{n,n}{int}\DUrole{w,w}{  }\DUrole{o,o}{=}\DUrole{w,w}{  }\DUrole{default_value}{1536}}\sphinxparamcomma \sphinxparam{\DUrole{n,n}{pixel\_size}\DUrole{p,p}{:}\DUrole{w,w}{  }\DUrole{n,n}{float}\DUrole{w,w}{  }\DUrole{o,o}{=}\DUrole{w,w}{  }\DUrole{default_value}{0.194}}\sphinxparamcomma \sphinxparam{\DUrole{n,n}{rotate\_camera\_left}\DUrole{p,p}{:}\DUrole{w,w}{  }\DUrole{n,n}{bool}\DUrole{w,w}{  }\DUrole{o,o}{=}\DUrole{w,w}{  }\DUrole{default_value}{True}}\sphinxparamcomma \sphinxparam{\DUrole{n,n}{enforce\_isocenter\_bounds}\DUrole{p,p}{:}\DUrole{w,w}{  }\DUrole{n,n}{bool}\DUrole{w,w}{  }\DUrole{o,o}{=}\DUrole{w,w}{  }\DUrole{default_value}{False}}}{}
\pysigstopsignatures
\sphinxAtStartPar
Bases: {\hyperref[\detokenize{deepdrr.device:deepdrr.device.device.Device}]{\sphinxcrossref{\sphinxcode{\sphinxupquote{Device}}}}}

\sphinxAtStartPar
A C\sphinxhyphen{}arm imaging device with orbital movement (alpha, beta) and isocenter movement (x, y, z).

\sphinxAtStartPar
Default parameters are based on the Siemens CIOS Spin.
\index{alpha (deepdrr.device.mobile\_carm.MobileCArm attribute)@\spxentry{alpha}\spxextra{deepdrr.device.mobile\_carm.MobileCArm attribute}}

\begin{fulllineitems}
\phantomsection\label{\detokenize{deepdrr.device:deepdrr.device.mobile_carm.MobileCArm.alpha}}
\pysigstartsignatures
\pysigline{\sphinxbfcode{\sphinxupquote{alpha}}\sphinxbfcode{\sphinxupquote{\DUrole{p,p}{:}\DUrole{w,w}{  }float}}}
\pysigstopsignatures
\end{fulllineitems}

\index{arm\_from\_device (deepdrr.device.mobile\_carm.MobileCArm property)@\spxentry{arm\_from\_device}\spxextra{deepdrr.device.mobile\_carm.MobileCArm property}}

\begin{fulllineitems}
\phantomsection\label{\detokenize{deepdrr.device:deepdrr.device.mobile_carm.MobileCArm.arm_from_device}}
\pysigstartsignatures
\pysigline{\sphinxbfcode{\sphinxupquote{property\DUrole{w,w}{  }}}\sphinxbfcode{\sphinxupquote{arm\_from\_device}}\sphinxbfcode{\sphinxupquote{\DUrole{p,p}{:}\DUrole{w,w}{  }{\hyperref[\detokenize{deepdrr.geo:deepdrr.geo.core.FrameTransform}]{\sphinxcrossref{FrameTransform}}}}}}
\pysigstopsignatures
\sphinxAtStartPar
Transformation from the device frame (which doesn’t move) to the arm frame (which rotates and translates with the arm, origin at the isocenter).

\end{fulllineitems}

\index{arm\_width (deepdrr.device.mobile\_carm.MobileCArm attribute)@\spxentry{arm\_width}\spxextra{deepdrr.device.mobile\_carm.MobileCArm attribute}}

\begin{fulllineitems}
\phantomsection\label{\detokenize{deepdrr.device:deepdrr.device.mobile_carm.MobileCArm.arm_width}}
\pysigstartsignatures
\pysigline{\sphinxbfcode{\sphinxupquote{arm\_width}}\sphinxbfcode{\sphinxupquote{\DUrole{w,w}{  }\DUrole{p,p}{=}\DUrole{w,w}{  }100}}}
\pysigstopsignatures
\end{fulllineitems}

\index{beta (deepdrr.device.mobile\_carm.MobileCArm attribute)@\spxentry{beta}\spxextra{deepdrr.device.mobile\_carm.MobileCArm attribute}}

\begin{fulllineitems}
\phantomsection\label{\detokenize{deepdrr.device:deepdrr.device.mobile_carm.MobileCArm.beta}}
\pysigstartsignatures
\pysigline{\sphinxbfcode{\sphinxupquote{beta}}\sphinxbfcode{\sphinxupquote{\DUrole{p,p}{:}\DUrole{w,w}{  }float}}}
\pysigstopsignatures
\end{fulllineitems}

\index{camera3d\_from\_device (deepdrr.device.mobile\_carm.MobileCArm property)@\spxentry{camera3d\_from\_device}\spxextra{deepdrr.device.mobile\_carm.MobileCArm property}}

\begin{fulllineitems}
\phantomsection\label{\detokenize{deepdrr.device:deepdrr.device.mobile_carm.MobileCArm.camera3d_from_device}}
\pysigstartsignatures
\pysigline{\sphinxbfcode{\sphinxupquote{property\DUrole{w,w}{  }}}\sphinxbfcode{\sphinxupquote{camera3d\_from\_device}}\sphinxbfcode{\sphinxupquote{\DUrole{p,p}{:}\DUrole{w,w}{  }{\hyperref[\detokenize{deepdrr.geo:deepdrr.geo.core.FrameTransform}]{\sphinxcrossref{FrameTransform}}}}}}
\pysigstopsignatures
\sphinxAtStartPar
Get the camera3d frame from device coordinates

\sphinxAtStartPar
The Z axis points from the source to the detector.

\end{fulllineitems}

\index{camera3d\_from\_world (deepdrr.device.mobile\_carm.MobileCArm property)@\spxentry{camera3d\_from\_world}\spxextra{deepdrr.device.mobile\_carm.MobileCArm property}}

\begin{fulllineitems}
\phantomsection\label{\detokenize{deepdrr.device:deepdrr.device.mobile_carm.MobileCArm.camera3d_from_world}}
\pysigstartsignatures
\pysigline{\sphinxbfcode{\sphinxupquote{property\DUrole{w,w}{  }}}\sphinxbfcode{\sphinxupquote{camera3d\_from\_world}}\sphinxbfcode{\sphinxupquote{\DUrole{p,p}{:}\DUrole{w,w}{  }{\hyperref[\detokenize{deepdrr.geo:deepdrr.geo.core.FrameTransform}]{\sphinxcrossref{FrameTransform}}}}}}
\pysigstopsignatures
\sphinxAtStartPar
Rigid transformation of the C\sphinxhyphen{}arm camera pose.

\end{fulllineitems}

\index{camera\_intrinsics (deepdrr.device.mobile\_carm.MobileCArm attribute)@\spxentry{camera\_intrinsics}\spxextra{deepdrr.device.mobile\_carm.MobileCArm attribute}}

\begin{fulllineitems}
\phantomsection\label{\detokenize{deepdrr.device:deepdrr.device.mobile_carm.MobileCArm.camera_intrinsics}}
\pysigstartsignatures
\pysigline{\sphinxbfcode{\sphinxupquote{camera\_intrinsics}}\sphinxbfcode{\sphinxupquote{\DUrole{p,p}{:}\DUrole{w,w}{  }{\hyperref[\detokenize{deepdrr.geo:deepdrr.geo.core.CameraIntrinsicTransform}]{\sphinxcrossref{CameraIntrinsicTransform}}}}}}
\pysigstopsignatures
\end{fulllineitems}

\index{detector\_height (deepdrr.device.mobile\_carm.MobileCArm attribute)@\spxentry{detector\_height}\spxextra{deepdrr.device.mobile\_carm.MobileCArm attribute}}

\begin{fulllineitems}
\phantomsection\label{\detokenize{deepdrr.device:deepdrr.device.mobile_carm.MobileCArm.detector_height}}
\pysigstartsignatures
\pysigline{\sphinxbfcode{\sphinxupquote{detector\_height}}\sphinxbfcode{\sphinxupquote{\DUrole{w,w}{  }\DUrole{p,p}{=}\DUrole{w,w}{  }100}}}
\pysigstopsignatures
\end{fulllineitems}

\index{device\_from\_arm (deepdrr.device.mobile\_carm.MobileCArm property)@\spxentry{device\_from\_arm}\spxextra{deepdrr.device.mobile\_carm.MobileCArm property}}

\begin{fulllineitems}
\phantomsection\label{\detokenize{deepdrr.device:deepdrr.device.mobile_carm.MobileCArm.device_from_arm}}
\pysigstartsignatures
\pysigline{\sphinxbfcode{\sphinxupquote{property\DUrole{w,w}{  }}}\sphinxbfcode{\sphinxupquote{device\_from\_arm}}\sphinxbfcode{\sphinxupquote{\DUrole{p,p}{:}\DUrole{w,w}{  }{\hyperref[\detokenize{deepdrr.geo:deepdrr.geo.core.FrameTransform}]{\sphinxcrossref{FrameTransform}}}}}}
\pysigstopsignatures
\end{fulllineitems}

\index{device\_from\_camera3d (deepdrr.device.mobile\_carm.MobileCArm property)@\spxentry{device\_from\_camera3d}\spxextra{deepdrr.device.mobile\_carm.MobileCArm property}}

\begin{fulllineitems}
\phantomsection\label{\detokenize{deepdrr.device:deepdrr.device.mobile_carm.MobileCArm.device_from_camera3d}}
\pysigstartsignatures
\pysigline{\sphinxbfcode{\sphinxupquote{property\DUrole{w,w}{  }}}\sphinxbfcode{\sphinxupquote{device\_from\_camera3d}}\sphinxbfcode{\sphinxupquote{\DUrole{p,p}{:}\DUrole{w,w}{  }{\hyperref[\detokenize{deepdrr.geo:deepdrr.geo.core.FrameTransform}]{\sphinxcrossref{FrameTransform}}}}}}
\pysigstopsignatures
\sphinxAtStartPar
Get the FrameTransform for the device’s camera3d\_from\_device frame (in the current pose).
\begin{quote}\begin{description}
\sphinxlineitem{Parameters}
\sphinxAtStartPar
\sphinxstyleliteralstrong{\sphinxupquote{camera3d\_transform}} ({\hyperref[\detokenize{deepdrr.geo:deepdrr.geo.core.FrameTransform}]{\sphinxcrossref{\sphinxstyleliteralemphasis{\sphinxupquote{FrameTransform}}}}}) \textendash{} the “camera3d\_from\_device” frame transformation for the device.

\sphinxlineitem{Returns}
\sphinxAtStartPar
the “device\_from\_camera3d” frame transformation for the device.

\sphinxlineitem{Return type}
\sphinxAtStartPar
{\hyperref[\detokenize{deepdrr.geo:deepdrr.geo.core.FrameTransform}]{\sphinxcrossref{FrameTransform}}}

\end{description}\end{quote}

\end{fulllineitems}

\index{get\_camera3d\_from\_world() (deepdrr.device.mobile\_carm.MobileCArm method)@\spxentry{get\_camera3d\_from\_world()}\spxextra{deepdrr.device.mobile\_carm.MobileCArm method}}

\begin{fulllineitems}
\phantomsection\label{\detokenize{deepdrr.device:deepdrr.device.mobile_carm.MobileCArm.get_camera3d_from_world}}
\pysigstartsignatures
\pysiglinewithargsret{\sphinxbfcode{\sphinxupquote{get\_camera3d\_from\_world}}}{}{{ $\rightarrow$ {\hyperref[\detokenize{deepdrr.geo:deepdrr.geo.core.FrameTransform}]{\sphinxcrossref{FrameTransform}}}}}
\pysigstopsignatures
\end{fulllineitems}

\index{get\_camera\_projection() (deepdrr.device.mobile\_carm.MobileCArm method)@\spxentry{get\_camera\_projection()}\spxextra{deepdrr.device.mobile\_carm.MobileCArm method}}

\begin{fulllineitems}
\phantomsection\label{\detokenize{deepdrr.device:deepdrr.device.mobile_carm.MobileCArm.get_camera_projection}}
\pysigstartsignatures
\pysiglinewithargsret{\sphinxbfcode{\sphinxupquote{get\_camera\_projection}}}{}{{ $\rightarrow$ {\hyperref[\detokenize{deepdrr.geo:deepdrr.geo.core.CameraProjection}]{\sphinxcrossref{CameraProjection}}}}}
\pysigstopsignatures
\sphinxAtStartPar
Get the camera projection for the device in the current pose.
\begin{quote}\begin{description}
\sphinxlineitem{Returns}
\sphinxAtStartPar
the “index\_from\_world” camera projection for the device.

\sphinxlineitem{Return type}
\sphinxAtStartPar
{\hyperref[\detokenize{deepdrr.geo:deepdrr.geo.core.CameraProjection}]{\sphinxcrossref{CameraProjection}}}

\end{description}\end{quote}

\end{fulllineitems}

\index{get\_mesh\_in\_world() (deepdrr.device.mobile\_carm.MobileCArm method)@\spxentry{get\_mesh\_in\_world()}\spxextra{deepdrr.device.mobile\_carm.MobileCArm method}}

\begin{fulllineitems}
\phantomsection\label{\detokenize{deepdrr.device:deepdrr.device.mobile_carm.MobileCArm.get_mesh_in_world}}
\pysigstartsignatures
\pysiglinewithargsret{\sphinxbfcode{\sphinxupquote{get\_mesh\_in\_world}}}{\sphinxparam{\DUrole{n,n}{full}\DUrole{o,o}{=}\DUrole{default_value}{False}}\sphinxparamcomma \sphinxparam{\DUrole{n,n}{use\_cached}\DUrole{o,o}{=}\DUrole{default_value}{True}}}{}
\pysigstopsignatures
\sphinxAtStartPar
Get the pyvista mesh for the C\sphinxhyphen{}arm, in its world\sphinxhyphen{}space orientation.
\begin{quote}\begin{description}
\sphinxlineitem{Raises}
\sphinxAtStartPar
\sphinxstyleliteralstrong{\sphinxupquote{RuntimeError}} \textendash{} if pyvista is not available.

\end{description}\end{quote}

\end{fulllineitems}

\index{isocenter (deepdrr.device.mobile\_carm.MobileCArm attribute)@\spxentry{isocenter}\spxextra{deepdrr.device.mobile\_carm.MobileCArm attribute}}

\begin{fulllineitems}
\phantomsection\label{\detokenize{deepdrr.device:deepdrr.device.mobile_carm.MobileCArm.isocenter}}
\pysigstartsignatures
\pysigline{\sphinxbfcode{\sphinxupquote{isocenter}}\sphinxbfcode{\sphinxupquote{\DUrole{p,p}{:}\DUrole{w,w}{  }{\hyperref[\detokenize{deepdrr.geo:deepdrr.geo.core.Point3D}]{\sphinxcrossref{Point3D}}}}}}
\pysigstopsignatures
\end{fulllineitems}

\index{isocenter\_in\_world (deepdrr.device.mobile\_carm.MobileCArm property)@\spxentry{isocenter\_in\_world}\spxextra{deepdrr.device.mobile\_carm.MobileCArm property}}

\begin{fulllineitems}
\phantomsection\label{\detokenize{deepdrr.device:deepdrr.device.mobile_carm.MobileCArm.isocenter_in_world}}
\pysigstartsignatures
\pysigline{\sphinxbfcode{\sphinxupquote{property\DUrole{w,w}{  }}}\sphinxbfcode{\sphinxupquote{isocenter\_in\_world}}\sphinxbfcode{\sphinxupquote{\DUrole{p,p}{:}\DUrole{w,w}{  }{\hyperref[\detokenize{deepdrr.geo:deepdrr.geo.core.Point3D}]{\sphinxcrossref{Point3D}}}}}}
\pysigstopsignatures
\end{fulllineitems}

\index{jitter() (deepdrr.device.mobile\_carm.MobileCArm method)@\spxentry{jitter()}\spxextra{deepdrr.device.mobile\_carm.MobileCArm method}}

\begin{fulllineitems}
\phantomsection\label{\detokenize{deepdrr.device:deepdrr.device.mobile_carm.MobileCArm.jitter}}
\pysigstartsignatures
\pysiglinewithargsret{\sphinxbfcode{\sphinxupquote{jitter}}}{}{}
\pysigstopsignatures
\end{fulllineitems}

\index{max\_isocenter (deepdrr.device.mobile\_carm.MobileCArm property)@\spxentry{max\_isocenter}\spxextra{deepdrr.device.mobile\_carm.MobileCArm property}}

\begin{fulllineitems}
\phantomsection\label{\detokenize{deepdrr.device:deepdrr.device.mobile_carm.MobileCArm.max_isocenter}}
\pysigstartsignatures
\pysigline{\sphinxbfcode{\sphinxupquote{property\DUrole{w,w}{  }}}\sphinxbfcode{\sphinxupquote{max\_isocenter}}\sphinxbfcode{\sphinxupquote{\DUrole{p,p}{:}\DUrole{w,w}{  }ndarray}}}
\pysigstopsignatures
\end{fulllineitems}

\index{min\_isocenter (deepdrr.device.mobile\_carm.MobileCArm property)@\spxentry{min\_isocenter}\spxextra{deepdrr.device.mobile\_carm.MobileCArm property}}

\begin{fulllineitems}
\phantomsection\label{\detokenize{deepdrr.device:deepdrr.device.mobile_carm.MobileCArm.min_isocenter}}
\pysigstartsignatures
\pysigline{\sphinxbfcode{\sphinxupquote{property\DUrole{w,w}{  }}}\sphinxbfcode{\sphinxupquote{min\_isocenter}}\sphinxbfcode{\sphinxupquote{\DUrole{p,p}{:}\DUrole{w,w}{  }ndarray}}}
\pysigstopsignatures
\end{fulllineitems}

\index{move\_by() (deepdrr.device.mobile\_carm.MobileCArm method)@\spxentry{move\_by()}\spxextra{deepdrr.device.mobile\_carm.MobileCArm method}}

\begin{fulllineitems}
\phantomsection\label{\detokenize{deepdrr.device:deepdrr.device.mobile_carm.MobileCArm.move_by}}
\pysigstartsignatures
\pysiglinewithargsret{\sphinxbfcode{\sphinxupquote{move\_by}}}{\sphinxparam{\DUrole{n,n}{delta\_isocenter}\DUrole{p,p}{:}\DUrole{w,w}{  }\DUrole{n,n}{{\hyperref[\detokenize{deepdrr.geo:deepdrr.geo.core.Vector3D}]{\sphinxcrossref{Vector3D}}}\DUrole{w,w}{  }\DUrole{p,p}{|}\DUrole{w,w}{  }None}\DUrole{w,w}{  }\DUrole{o,o}{=}\DUrole{w,w}{  }\DUrole{default_value}{None}}\sphinxparamcomma \sphinxparam{\DUrole{n,n}{delta\_alpha}\DUrole{p,p}{:}\DUrole{w,w}{  }\DUrole{n,n}{float\DUrole{w,w}{  }\DUrole{p,p}{|}\DUrole{w,w}{  }None}\DUrole{w,w}{  }\DUrole{o,o}{=}\DUrole{w,w}{  }\DUrole{default_value}{None}}\sphinxparamcomma \sphinxparam{\DUrole{n,n}{delta\_beta}\DUrole{p,p}{:}\DUrole{w,w}{  }\DUrole{n,n}{float\DUrole{w,w}{  }\DUrole{p,p}{|}\DUrole{w,w}{  }None}\DUrole{w,w}{  }\DUrole{o,o}{=}\DUrole{w,w}{  }\DUrole{default_value}{None}}\sphinxparamcomma \sphinxparam{\DUrole{n,n}{delta\_gamma}\DUrole{p,p}{:}\DUrole{w,w}{  }\DUrole{n,n}{float\DUrole{w,w}{  }\DUrole{p,p}{|}\DUrole{w,w}{  }None}\DUrole{w,w}{  }\DUrole{o,o}{=}\DUrole{w,w}{  }\DUrole{default_value}{None}}\sphinxparamcomma \sphinxparam{\DUrole{n,n}{degrees}\DUrole{p,p}{:}\DUrole{w,w}{  }\DUrole{n,n}{bool}\DUrole{w,w}{  }\DUrole{o,o}{=}\DUrole{w,w}{  }\DUrole{default_value}{True}}}{{ $\rightarrow$ None}}
\pysigstopsignatures
\sphinxAtStartPar
Move the C\sphinxhyphen{}arm to the specified pose.
\begin{quote}\begin{description}
\sphinxlineitem{Parameters}\begin{itemize}
\item {} 
\sphinxAtStartPar
\sphinxstyleliteralstrong{\sphinxupquote{delta\_isocenter}} (\sphinxstyleliteralemphasis{\sphinxupquote{Optional}}\sphinxstyleliteralemphasis{\sphinxupquote{{[}}}{\hyperref[\detokenize{deepdrr.geo:deepdrr.geo.Vector3D}]{\sphinxcrossref{\sphinxstyleliteralemphasis{\sphinxupquote{geo.Vector3D}}}}}\sphinxstyleliteralemphasis{\sphinxupquote{{]}}}\sphinxstyleliteralemphasis{\sphinxupquote{, }}\sphinxstyleliteralemphasis{\sphinxupquote{optional}}) \textendash{} change to the isocenter in DEVICE space
(as a vector, this only matters if the scaling/rotation is different).
This is the center about which rotations are performed. Defaults to None.

\item {} 
\sphinxAtStartPar
\sphinxstyleliteralstrong{\sphinxupquote{delta\_alpha}} (\sphinxstyleliteralemphasis{\sphinxupquote{Optional}}\sphinxstyleliteralemphasis{\sphinxupquote{{[}}}\sphinxstyleliteralemphasis{\sphinxupquote{float}}\sphinxstyleliteralemphasis{\sphinxupquote{{]}}}\sphinxstyleliteralemphasis{\sphinxupquote{, }}\sphinxstyleliteralemphasis{\sphinxupquote{optional}}) \textendash{} change in alpha. Defaults to None.

\item {} 
\sphinxAtStartPar
\sphinxstyleliteralstrong{\sphinxupquote{delta\_beta}} (\sphinxstyleliteralemphasis{\sphinxupquote{Optional}}\sphinxstyleliteralemphasis{\sphinxupquote{{[}}}\sphinxstyleliteralemphasis{\sphinxupquote{float}}\sphinxstyleliteralemphasis{\sphinxupquote{{]}}}\sphinxstyleliteralemphasis{\sphinxupquote{, }}\sphinxstyleliteralemphasis{\sphinxupquote{optional}}) \textendash{} change in beta. Defaults to None.

\item {} 
\sphinxAtStartPar
\sphinxstyleliteralstrong{\sphinxupquote{degrees}} (\sphinxstyleliteralemphasis{\sphinxupquote{bool}}\sphinxstyleliteralemphasis{\sphinxupquote{, }}\sphinxstyleliteralemphasis{\sphinxupquote{optional}}) \textendash{} whether the given angles are in degrees. Defaults to False.

\end{itemize}

\end{description}\end{quote}

\end{fulllineitems}

\index{move\_to() (deepdrr.device.mobile\_carm.MobileCArm method)@\spxentry{move\_to()}\spxextra{deepdrr.device.mobile\_carm.MobileCArm method}}

\begin{fulllineitems}
\phantomsection\label{\detokenize{deepdrr.device:deepdrr.device.mobile_carm.MobileCArm.move_to}}
\pysigstartsignatures
\pysiglinewithargsret{\sphinxbfcode{\sphinxupquote{move\_to}}}{\sphinxparam{\DUrole{n,n}{isocenter}\DUrole{p,p}{:}\DUrole{w,w}{  }\DUrole{n,n}{{\hyperref[\detokenize{deepdrr.geo:deepdrr.geo.core.Point3D}]{\sphinxcrossref{Point3D}}}\DUrole{w,w}{  }\DUrole{p,p}{|}\DUrole{w,w}{  }None}\DUrole{w,w}{  }\DUrole{o,o}{=}\DUrole{w,w}{  }\DUrole{default_value}{None}}\sphinxparamcomma \sphinxparam{\DUrole{n,n}{isocenter\_in\_world}\DUrole{p,p}{:}\DUrole{w,w}{  }\DUrole{n,n}{{\hyperref[\detokenize{deepdrr.geo:deepdrr.geo.core.Point3D}]{\sphinxcrossref{Point3D}}}\DUrole{w,w}{  }\DUrole{p,p}{|}\DUrole{w,w}{  }None}\DUrole{w,w}{  }\DUrole{o,o}{=}\DUrole{w,w}{  }\DUrole{default_value}{None}}\sphinxparamcomma \sphinxparam{\DUrole{n,n}{alpha}\DUrole{p,p}{:}\DUrole{w,w}{  }\DUrole{n,n}{float\DUrole{w,w}{  }\DUrole{p,p}{|}\DUrole{w,w}{  }None}\DUrole{w,w}{  }\DUrole{o,o}{=}\DUrole{w,w}{  }\DUrole{default_value}{None}}\sphinxparamcomma \sphinxparam{\DUrole{n,n}{beta}\DUrole{p,p}{:}\DUrole{w,w}{  }\DUrole{n,n}{float\DUrole{w,w}{  }\DUrole{p,p}{|}\DUrole{w,w}{  }None}\DUrole{w,w}{  }\DUrole{o,o}{=}\DUrole{w,w}{  }\DUrole{default_value}{None}}\sphinxparamcomma \sphinxparam{\DUrole{n,n}{gamma}\DUrole{p,p}{:}\DUrole{w,w}{  }\DUrole{n,n}{float\DUrole{w,w}{  }\DUrole{p,p}{|}\DUrole{w,w}{  }None}\DUrole{w,w}{  }\DUrole{o,o}{=}\DUrole{w,w}{  }\DUrole{default_value}{None}}\sphinxparamcomma \sphinxparam{\DUrole{n,n}{degrees}\DUrole{p,p}{:}\DUrole{w,w}{  }\DUrole{n,n}{bool}\DUrole{w,w}{  }\DUrole{o,o}{=}\DUrole{w,w}{  }\DUrole{default_value}{True}}\sphinxparamcomma \sphinxparam{\DUrole{n,n}{interest\_point\_in\_world}\DUrole{p,p}{:}\DUrole{w,w}{  }\DUrole{n,n}{{\hyperref[\detokenize{deepdrr.geo:deepdrr.geo.core.Point3D}]{\sphinxcrossref{Point3D}}}\DUrole{w,w}{  }\DUrole{p,p}{|}\DUrole{w,w}{  }None}\DUrole{w,w}{  }\DUrole{o,o}{=}\DUrole{w,w}{  }\DUrole{default_value}{None}}\sphinxparamcomma \sphinxparam{\DUrole{n,n}{principle\_ray\_in\_world}\DUrole{p,p}{:}\DUrole{w,w}{  }\DUrole{n,n}{{\hyperref[\detokenize{deepdrr.geo:deepdrr.geo.core.Vector3D}]{\sphinxcrossref{Vector3D}}}\DUrole{w,w}{  }\DUrole{p,p}{|}\DUrole{w,w}{  }None}\DUrole{w,w}{  }\DUrole{o,o}{=}\DUrole{w,w}{  }\DUrole{default_value}{None}}}{{ $\rightarrow$ None}}
\pysigstopsignatures
\sphinxAtStartPar
Move to the specified point.
\begin{quote}\begin{description}
\sphinxlineitem{Parameters}\begin{itemize}
\item {} 
\sphinxAtStartPar
\sphinxstyleliteralstrong{\sphinxupquote{isocenter\_in\_world}} (\sphinxstyleliteralemphasis{\sphinxupquote{Optional}}\sphinxstyleliteralemphasis{\sphinxupquote{{[}}}{\hyperref[\detokenize{deepdrr.geo:deepdrr.geo.Point3D}]{\sphinxcrossref{\sphinxstyleliteralemphasis{\sphinxupquote{geo.Point3D}}}}}\sphinxstyleliteralemphasis{\sphinxupquote{{]}}}\sphinxstyleliteralemphasis{\sphinxupquote{, }}\sphinxstyleliteralemphasis{\sphinxupquote{optional}}) \textendash{} the desired isocenter in world coordinates.
Overrides \sphinxtitleref{isocenter} if provided. Defaults to None.

\item {} 
\sphinxAtStartPar
\sphinxstyleliteralstrong{\sphinxupquote{isocenter}} \textendash{} Desired isocenter in device coordinates.

\item {} 
\sphinxAtStartPar
\sphinxstyleliteralstrong{\sphinxupquote{alpha}} (\sphinxstyleliteralemphasis{\sphinxupquote{Optional}}\sphinxstyleliteralemphasis{\sphinxupquote{{[}}}\sphinxstyleliteralemphasis{\sphinxupquote{float}}\sphinxstyleliteralemphasis{\sphinxupquote{{]}}}\sphinxstyleliteralemphasis{\sphinxupquote{, }}\sphinxstyleliteralemphasis{\sphinxupquote{optional}}) \textendash{} the desired alpha angulation. Defaults to None.

\item {} 
\sphinxAtStartPar
\sphinxstyleliteralstrong{\sphinxupquote{beta}} (\sphinxstyleliteralemphasis{\sphinxupquote{Optional}}\sphinxstyleliteralemphasis{\sphinxupquote{{[}}}\sphinxstyleliteralemphasis{\sphinxupquote{float}}\sphinxstyleliteralemphasis{\sphinxupquote{{]}}}\sphinxstyleliteralemphasis{\sphinxupquote{, }}\sphinxstyleliteralemphasis{\sphinxupquote{optional}}) \textendash{} the desired secondary angulation. Defaults to None.

\item {} 
\sphinxAtStartPar
\sphinxstyleliteralstrong{\sphinxupquote{degrees}} (\sphinxstyleliteralemphasis{\sphinxupquote{bool}}\sphinxstyleliteralemphasis{\sphinxupquote{, }}\sphinxstyleliteralemphasis{\sphinxupquote{optional}}) \textendash{} whether angles are in degrees or radians. Defaults to False.

\item {} 
\sphinxAtStartPar
\sphinxstyleliteralstrong{\sphinxupquote{interest\_point}} ({\hyperref[\detokenize{deepdrr.geo:deepdrr.geo.core.Point3D}]{\sphinxcrossref{\sphinxstyleliteralemphasis{\sphinxupquote{Point3D}}}}}\sphinxstyleliteralemphasis{\sphinxupquote{, }}\sphinxstyleliteralemphasis{\sphinxupquote{optional}}) \textendash{} If this world\sphinxhyphen{}space point is provided, add a translation such that the rotation
maintains the camera\sphinxhyphen{}space position of this point. Overrides \sphinxtitleref{isocenter}. Defaults to None.

\item {} 
\sphinxAtStartPar
\sphinxstyleliteralstrong{\sphinxupquote{principle\_ray\_in\_world}} (\sphinxstyleliteralemphasis{\sphinxupquote{Optional}}\sphinxstyleliteralemphasis{\sphinxupquote{{[}}}{\hyperref[\detokenize{deepdrr.geo:deepdrr.geo.core.Vector3D}]{\sphinxcrossref{\sphinxstyleliteralemphasis{\sphinxupquote{Vector3D}}}}}\sphinxstyleliteralemphasis{\sphinxupquote{{]}}}\sphinxstyleliteralemphasis{\sphinxupquote{, }}\sphinxstyleliteralemphasis{\sphinxupquote{optional}}) \textendash{} If this world\sphinxhyphen{}space vector is provided, override alpha, beta so the C\sphinxhyphen{}arm points along this vector.

\end{itemize}

\end{description}\end{quote}

\end{fulllineitems}

\index{pixel\_size (deepdrr.device.mobile\_carm.MobileCArm attribute)@\spxentry{pixel\_size}\spxextra{deepdrr.device.mobile\_carm.MobileCArm attribute}}

\begin{fulllineitems}
\phantomsection\label{\detokenize{deepdrr.device:deepdrr.device.mobile_carm.MobileCArm.pixel_size}}
\pysigstartsignatures
\pysigline{\sphinxbfcode{\sphinxupquote{pixel\_size}}\sphinxbfcode{\sphinxupquote{\DUrole{p,p}{:}\DUrole{w,w}{  }float}}}
\pysigstopsignatures
\end{fulllineitems}

\index{principle\_ray (deepdrr.device.mobile\_carm.MobileCArm property)@\spxentry{principle\_ray}\spxextra{deepdrr.device.mobile\_carm.MobileCArm property}}

\begin{fulllineitems}
\phantomsection\label{\detokenize{deepdrr.device:deepdrr.device.mobile_carm.MobileCArm.principle_ray}}
\pysigstartsignatures
\pysigline{\sphinxbfcode{\sphinxupquote{property\DUrole{w,w}{  }}}\sphinxbfcode{\sphinxupquote{principle\_ray}}\sphinxbfcode{\sphinxupquote{\DUrole{p,p}{:}\DUrole{w,w}{  }{\hyperref[\detokenize{deepdrr.geo:deepdrr.geo.core.Vector3D}]{\sphinxcrossref{Vector3D}}}}}}
\pysigstopsignatures
\sphinxAtStartPar
Unit vector along principle ray.

\end{fulllineitems}

\index{reposition() (deepdrr.device.mobile\_carm.MobileCArm method)@\spxentry{reposition()}\spxextra{deepdrr.device.mobile\_carm.MobileCArm method}}

\begin{fulllineitems}
\phantomsection\label{\detokenize{deepdrr.device:deepdrr.device.mobile_carm.MobileCArm.reposition}}
\pysigstartsignatures
\pysiglinewithargsret{\sphinxbfcode{\sphinxupquote{reposition}}}{\sphinxparam{\DUrole{n,n}{viewpoint\_in\_world}\DUrole{p,p}{:}\DUrole{w,w}{  }\DUrole{n,n}{{\hyperref[\detokenize{deepdrr.geo:deepdrr.geo.core.Point3D}]{\sphinxcrossref{Point3D}}}\DUrole{w,w}{  }\DUrole{p,p}{|}\DUrole{w,w}{  }None}\DUrole{w,w}{  }\DUrole{o,o}{=}\DUrole{w,w}{  }\DUrole{default_value}{None}}\sphinxparamcomma \sphinxparam{\DUrole{n,n}{device\_in\_world}\DUrole{p,p}{:}\DUrole{w,w}{  }\DUrole{n,n}{{\hyperref[\detokenize{deepdrr.geo:deepdrr.geo.core.Point3D}]{\sphinxcrossref{Point3D}}}\DUrole{w,w}{  }\DUrole{p,p}{|}\DUrole{w,w}{  }None}\DUrole{w,w}{  }\DUrole{o,o}{=}\DUrole{w,w}{  }\DUrole{default_value}{None}}}{{ $\rightarrow$ None}}
\pysigstopsignatures
\sphinxAtStartPar
Reposition the C\sphinxhyphen{}arm by resetting its internal pose to the parameters and adjusting the world\_from\_device transform.

\sphinxAtStartPar
TODO: currently, this eliminates any scaling/rotation of the device in world.

\sphinxAtStartPar
May provide either the isocenter location (device\_in\_world) or viewpoint (viewpoint\_in\_world)
\begin{quote}\begin{description}
\sphinxlineitem{Parameters}\begin{itemize}
\item {} 
\sphinxAtStartPar
\sphinxstyleliteralstrong{\sphinxupquote{viewpoint\_in\_world}} ({\hyperref[\detokenize{deepdrr.geo:deepdrr.geo.Point3D}]{\sphinxcrossref{\sphinxstyleliteralemphasis{\sphinxupquote{geo.Point3D}}}}}) \textendash{} the initial viewpoint the device should have.

\item {} 
\sphinxAtStartPar
\sphinxstyleliteralstrong{\sphinxupquote{(}}\sphinxstyleliteralstrong{\sphinxupquote{)}} (\sphinxstyleliteralemphasis{\sphinxupquote{device\_in\_world}}) \textendash{} initial isocenter.

\end{itemize}

\end{description}\end{quote}

\end{fulllineitems}

\index{sensor\_height (deepdrr.device.mobile\_carm.MobileCArm attribute)@\spxentry{sensor\_height}\spxextra{deepdrr.device.mobile\_carm.MobileCArm attribute}}

\begin{fulllineitems}
\phantomsection\label{\detokenize{deepdrr.device:deepdrr.device.mobile_carm.MobileCArm.sensor_height}}
\pysigstartsignatures
\pysigline{\sphinxbfcode{\sphinxupquote{sensor\_height}}\sphinxbfcode{\sphinxupquote{\DUrole{p,p}{:}\DUrole{w,w}{  }int}}}
\pysigstopsignatures
\end{fulllineitems}

\index{sensor\_width (deepdrr.device.mobile\_carm.MobileCArm attribute)@\spxentry{sensor\_width}\spxextra{deepdrr.device.mobile\_carm.MobileCArm attribute}}

\begin{fulllineitems}
\phantomsection\label{\detokenize{deepdrr.device:deepdrr.device.mobile_carm.MobileCArm.sensor_width}}
\pysigstartsignatures
\pysigline{\sphinxbfcode{\sphinxupquote{sensor\_width}}\sphinxbfcode{\sphinxupquote{\DUrole{p,p}{:}\DUrole{w,w}{  }int}}}
\pysigstopsignatures
\end{fulllineitems}

\index{source\_height (deepdrr.device.mobile\_carm.MobileCArm attribute)@\spxentry{source\_height}\spxextra{deepdrr.device.mobile\_carm.MobileCArm attribute}}

\begin{fulllineitems}
\phantomsection\label{\detokenize{deepdrr.device:deepdrr.device.mobile_carm.MobileCArm.source_height}}
\pysigstartsignatures
\pysigline{\sphinxbfcode{\sphinxupquote{source\_height}}\sphinxbfcode{\sphinxupquote{\DUrole{w,w}{  }\DUrole{p,p}{=}\DUrole{w,w}{  }200}}}
\pysigstopsignatures
\end{fulllineitems}

\index{source\_in\_arm (deepdrr.device.mobile\_carm.MobileCArm property)@\spxentry{source\_in\_arm}\spxextra{deepdrr.device.mobile\_carm.MobileCArm property}}

\begin{fulllineitems}
\phantomsection\label{\detokenize{deepdrr.device:deepdrr.device.mobile_carm.MobileCArm.source_in_arm}}
\pysigstartsignatures
\pysigline{\sphinxbfcode{\sphinxupquote{property\DUrole{w,w}{  }}}\sphinxbfcode{\sphinxupquote{source\_in\_arm}}\sphinxbfcode{\sphinxupquote{\DUrole{p,p}{:}\DUrole{w,w}{  }{\hyperref[\detokenize{deepdrr.geo:deepdrr.geo.core.Point3D}]{\sphinxcrossref{Point3D}}}}}}
\pysigstopsignatures
\end{fulllineitems}

\index{source\_in\_device (deepdrr.device.mobile\_carm.MobileCArm property)@\spxentry{source\_in\_device}\spxextra{deepdrr.device.mobile\_carm.MobileCArm property}}

\begin{fulllineitems}
\phantomsection\label{\detokenize{deepdrr.device:deepdrr.device.mobile_carm.MobileCArm.source_in_device}}
\pysigstartsignatures
\pysigline{\sphinxbfcode{\sphinxupquote{property\DUrole{w,w}{  }}}\sphinxbfcode{\sphinxupquote{source\_in\_device}}\sphinxbfcode{\sphinxupquote{\DUrole{p,p}{:}\DUrole{w,w}{  }{\hyperref[\detokenize{deepdrr.geo:deepdrr.geo.core.Point3D}]{\sphinxcrossref{Point3D}}}}}}
\pysigstopsignatures
\end{fulllineitems}

\index{source\_radius (deepdrr.device.mobile\_carm.MobileCArm attribute)@\spxentry{source\_radius}\spxextra{deepdrr.device.mobile\_carm.MobileCArm attribute}}

\begin{fulllineitems}
\phantomsection\label{\detokenize{deepdrr.device:deepdrr.device.mobile_carm.MobileCArm.source_radius}}
\pysigstartsignatures
\pysigline{\sphinxbfcode{\sphinxupquote{source\_radius}}\sphinxbfcode{\sphinxupquote{\DUrole{w,w}{  }\DUrole{p,p}{=}\DUrole{w,w}{  }100}}}
\pysigstopsignatures
\end{fulllineitems}

\index{source\_to\_detector\_distance (deepdrr.device.mobile\_carm.MobileCArm attribute)@\spxentry{source\_to\_detector\_distance}\spxextra{deepdrr.device.mobile\_carm.MobileCArm attribute}}

\begin{fulllineitems}
\phantomsection\label{\detokenize{deepdrr.device:deepdrr.device.mobile_carm.MobileCArm.source_to_detector_distance}}
\pysigstartsignatures
\pysigline{\sphinxbfcode{\sphinxupquote{source\_to\_detector\_distance}}\sphinxbfcode{\sphinxupquote{\DUrole{p,p}{:}\DUrole{w,w}{  }float}}}
\pysigstopsignatures
\end{fulllineitems}

\index{to\_config() (deepdrr.device.mobile\_carm.MobileCArm method)@\spxentry{to\_config()}\spxextra{deepdrr.device.mobile\_carm.MobileCArm method}}

\begin{fulllineitems}
\phantomsection\label{\detokenize{deepdrr.device:deepdrr.device.mobile_carm.MobileCArm.to_config}}
\pysigstartsignatures
\pysiglinewithargsret{\sphinxbfcode{\sphinxupquote{to\_config}}}{}{{ $\rightarrow$ Dict\DUrole{p,p}{{[}}str\DUrole{p,p}{,}\DUrole{w,w}{  }Any\DUrole{p,p}{{]}}}}
\pysigstopsignatures
\sphinxAtStartPar
Get a json\sphinxhyphen{}safe dictionary that can be used to initialize the C\sphinxhyphen{}arm in its current pose.

\end{fulllineitems}

\index{viewpoint (deepdrr.device.mobile\_carm.MobileCArm property)@\spxentry{viewpoint}\spxextra{deepdrr.device.mobile\_carm.MobileCArm property}}

\begin{fulllineitems}
\phantomsection\label{\detokenize{deepdrr.device:deepdrr.device.mobile_carm.MobileCArm.viewpoint}}
\pysigstartsignatures
\pysigline{\sphinxbfcode{\sphinxupquote{property\DUrole{w,w}{  }}}\sphinxbfcode{\sphinxupquote{viewpoint}}\sphinxbfcode{\sphinxupquote{\DUrole{p,p}{:}\DUrole{w,w}{  }{\hyperref[\detokenize{deepdrr.geo:deepdrr.geo.core.Point3D}]{\sphinxcrossref{Point3D}}}}}}
\pysigstopsignatures
\sphinxAtStartPar
Get the point along the principle ray, where objects of interest should ideally be placed.
\begin{quote}\begin{description}
\sphinxlineitem{Returns}
\sphinxAtStartPar
the viewpoint in the device frame.

\sphinxlineitem{Return type}
\sphinxAtStartPar
{\hyperref[\detokenize{deepdrr.geo:deepdrr.geo.Point3D}]{\sphinxcrossref{geo.Point3D}}}

\end{description}\end{quote}

\end{fulllineitems}

\index{viewpoint\_in\_world (deepdrr.device.mobile\_carm.MobileCArm property)@\spxentry{viewpoint\_in\_world}\spxextra{deepdrr.device.mobile\_carm.MobileCArm property}}

\begin{fulllineitems}
\phantomsection\label{\detokenize{deepdrr.device:deepdrr.device.mobile_carm.MobileCArm.viewpoint_in_world}}
\pysigstartsignatures
\pysigline{\sphinxbfcode{\sphinxupquote{property\DUrole{w,w}{  }}}\sphinxbfcode{\sphinxupquote{viewpoint\_in\_world}}\sphinxbfcode{\sphinxupquote{\DUrole{p,p}{:}\DUrole{w,w}{  }{\hyperref[\detokenize{deepdrr.geo:deepdrr.geo.core.Point3D}]{\sphinxcrossref{Point3D}}}}}}
\pysigstopsignatures
\end{fulllineitems}

\index{world\_from\_device (deepdrr.device.mobile\_carm.MobileCArm attribute)@\spxentry{world\_from\_device}\spxextra{deepdrr.device.mobile\_carm.MobileCArm attribute}}

\begin{fulllineitems}
\phantomsection\label{\detokenize{deepdrr.device:deepdrr.device.mobile_carm.MobileCArm.world_from_device}}
\pysigstartsignatures
\pysigline{\sphinxbfcode{\sphinxupquote{world\_from\_device}}\sphinxbfcode{\sphinxupquote{\DUrole{p,p}{:}\DUrole{w,w}{  }{\hyperref[\detokenize{deepdrr.geo:deepdrr.geo.core.FrameTransform}]{\sphinxcrossref{FrameTransform}}}}}}
\pysigstopsignatures
\end{fulllineitems}


\end{fulllineitems}

\index{pose\_vector\_angles() (in module deepdrr.device.mobile\_carm)@\spxentry{pose\_vector\_angles()}\spxextra{in module deepdrr.device.mobile\_carm}}

\begin{fulllineitems}
\phantomsection\label{\detokenize{deepdrr.device:deepdrr.device.mobile_carm.pose_vector_angles}}
\pysigstartsignatures
\pysiglinewithargsret{\sphinxcode{\sphinxupquote{deepdrr.device.mobile\_carm.}}\sphinxbfcode{\sphinxupquote{pose\_vector\_angles}}}{\sphinxparam{\DUrole{n,n}{pose}\DUrole{p,p}{:}\DUrole{w,w}{  }\DUrole{n,n}{{\hyperref[\detokenize{deepdrr.geo:deepdrr.geo.core.Vector3D}]{\sphinxcrossref{Vector3D}}}}}}{{ $\rightarrow$ Tuple\DUrole{p,p}{{[}}float\DUrole{p,p}{,}\DUrole{w,w}{  }float\DUrole{p,p}{{]}}}}
\pysigstopsignatures
\sphinxAtStartPar
Get the C\sphinxhyphen{}arm angles alpha, beta corrsponding the the pose vector.

\sphinxAtStartPar
TODO(killeen): make a part of the MobileCArm object, to convert from a world\sphinxhyphen{}space vector. Get rid of the Gimble lock problem.
\begin{quote}\begin{description}
\sphinxlineitem{Parameters}
\sphinxAtStartPar
\sphinxstyleliteralstrong{\sphinxupquote{pose}} ({\hyperref[\detokenize{deepdrr.geo:deepdrr.geo.Vector3D}]{\sphinxcrossref{\sphinxstyleliteralemphasis{\sphinxupquote{geo.Vector3D}}}}}) \textendash{} the vector pointing from the isocenter (or camera) to the detector. (Along the principle ray.)

\sphinxlineitem{Returns}
\sphinxAtStartPar
carm angulation (alpha, beta) in radians.

\sphinxlineitem{Return type}
\sphinxAtStartPar
Tuple{[}float, float{]}

\end{description}\end{quote}

\end{fulllineitems}



\subsection{deepdrr.device.simple\_device}
\label{\detokenize{deepdrr.device:module-deepdrr.device.simple_device}}\label{\detokenize{deepdrr.device:deepdrr-device-simple-device}}\index{module@\spxentry{module}!deepdrr.device.simple\_device@\spxentry{deepdrr.device.simple\_device}}\index{deepdrr.device.simple\_device@\spxentry{deepdrr.device.simple\_device}!module@\spxentry{module}}\index{SimpleDevice (class in deepdrr.device.simple\_device)@\spxentry{SimpleDevice}\spxextra{class in deepdrr.device.simple\_device}}

\begin{fulllineitems}
\phantomsection\label{\detokenize{deepdrr.device:deepdrr.device.simple_device.SimpleDevice}}
\pysigstartsignatures
\pysiglinewithargsret{\sphinxbfcode{\sphinxupquote{class\DUrole{w,w}{  }}}\sphinxcode{\sphinxupquote{deepdrr.device.simple\_device.}}\sphinxbfcode{\sphinxupquote{SimpleDevice}}}{\sphinxparam{\DUrole{n,n}{sensor\_height}\DUrole{p,p}{:}\DUrole{w,w}{  }\DUrole{n,n}{int}\DUrole{w,w}{  }\DUrole{o,o}{=}\DUrole{w,w}{  }\DUrole{default_value}{300}}\sphinxparamcomma \sphinxparam{\DUrole{n,n}{sensor\_width}\DUrole{p,p}{:}\DUrole{w,w}{  }\DUrole{n,n}{int}\DUrole{w,w}{  }\DUrole{o,o}{=}\DUrole{w,w}{  }\DUrole{default_value}{300}}\sphinxparamcomma \sphinxparam{\DUrole{n,n}{pixel\_size}\DUrole{p,p}{:}\DUrole{w,w}{  }\DUrole{n,n}{float}\DUrole{w,w}{  }\DUrole{o,o}{=}\DUrole{w,w}{  }\DUrole{default_value}{0.1}}\sphinxparamcomma \sphinxparam{\DUrole{n,n}{source\_to\_detector\_distance}\DUrole{p,p}{:}\DUrole{w,w}{  }\DUrole{n,n}{float}\DUrole{w,w}{  }\DUrole{o,o}{=}\DUrole{w,w}{  }\DUrole{default_value}{1000.0}}\sphinxparamcomma \sphinxparam{\DUrole{n,n}{world\_from\_device}\DUrole{p,p}{:}\DUrole{w,w}{  }\DUrole{n,n}{{\hyperref[\detokenize{deepdrr.geo:deepdrr.geo.core.FrameTransform}]{\sphinxcrossref{FrameTransform}}}\DUrole{w,w}{  }\DUrole{p,p}{|}\DUrole{w,w}{  }None}\DUrole{w,w}{  }\DUrole{o,o}{=}\DUrole{w,w}{  }\DUrole{default_value}{None}}}{}
\pysigstopsignatures
\sphinxAtStartPar
Bases: {\hyperref[\detokenize{deepdrr.device:deepdrr.device.device.Device}]{\sphinxcrossref{\sphinxcode{\sphinxupquote{Device}}}}}

\sphinxAtStartPar
A simple C\sphinxhyphen{}arm with a point, direction interface to set views.

\sphinxAtStartPar
The “point” being positioned is always at the midpoint of the source and detector. The direction
is the direction from the source to the detector. The up\sphinxhyphen{}vector of images can also be provided,
not necessarily in the image plane (projected onto it).

\sphinxAtStartPar
Any of the device’s attributes can be set directly. The default values are not based on any
particular device.

\sphinxAtStartPar
This class may be useful for understanding basic concepts.
\index{sensor\_height (deepdrr.device.simple\_device.SimpleDevice attribute)@\spxentry{sensor\_height}\spxextra{deepdrr.device.simple\_device.SimpleDevice attribute}}

\begin{fulllineitems}
\phantomsection\label{\detokenize{deepdrr.device:deepdrr.device.simple_device.SimpleDevice.sensor_height}}
\pysigstartsignatures
\pysigline{\sphinxbfcode{\sphinxupquote{sensor\_height}}}
\pysigstopsignatures
\sphinxAtStartPar
the height of the sensor in pixels.
\begin{quote}\begin{description}
\sphinxlineitem{Type}
\sphinxAtStartPar
int

\end{description}\end{quote}

\end{fulllineitems}

\index{sensor\_width (deepdrr.device.simple\_device.SimpleDevice attribute)@\spxentry{sensor\_width}\spxextra{deepdrr.device.simple\_device.SimpleDevice attribute}}

\begin{fulllineitems}
\phantomsection\label{\detokenize{deepdrr.device:deepdrr.device.simple_device.SimpleDevice.sensor_width}}
\pysigstartsignatures
\pysigline{\sphinxbfcode{\sphinxupquote{sensor\_width}}}
\pysigstopsignatures
\sphinxAtStartPar
the width of the sensor in pixels.
\begin{quote}\begin{description}
\sphinxlineitem{Type}
\sphinxAtStartPar
int

\end{description}\end{quote}

\end{fulllineitems}

\index{pixel\_size (deepdrr.device.simple\_device.SimpleDevice attribute)@\spxentry{pixel\_size}\spxextra{deepdrr.device.simple\_device.SimpleDevice attribute}}

\begin{fulllineitems}
\phantomsection\label{\detokenize{deepdrr.device:deepdrr.device.simple_device.SimpleDevice.pixel_size}}
\pysigstartsignatures
\pysigline{\sphinxbfcode{\sphinxupquote{pixel\_size}}}
\pysigstopsignatures
\sphinxAtStartPar
the size of a pixel in mm.
\begin{quote}\begin{description}
\sphinxlineitem{Type}
\sphinxAtStartPar
float

\end{description}\end{quote}

\end{fulllineitems}

\index{source\_to\_detector\_distance (deepdrr.device.simple\_device.SimpleDevice attribute)@\spxentry{source\_to\_detector\_distance}\spxextra{deepdrr.device.simple\_device.SimpleDevice attribute}}

\begin{fulllineitems}
\phantomsection\label{\detokenize{deepdrr.device:deepdrr.device.simple_device.SimpleDevice.source_to_detector_distance}}
\pysigstartsignatures
\pysigline{\sphinxbfcode{\sphinxupquote{source\_to\_detector\_distance}}}
\pysigstopsignatures
\sphinxAtStartPar
the distance from the source to the detector in mm.
\begin{quote}\begin{description}
\sphinxlineitem{Type}
\sphinxAtStartPar
float

\end{description}\end{quote}

\end{fulllineitems}

\index{world\_from\_device (deepdrr.device.simple\_device.SimpleDevice attribute)@\spxentry{world\_from\_device}\spxextra{deepdrr.device.simple\_device.SimpleDevice attribute}}

\begin{fulllineitems}
\phantomsection\label{\detokenize{deepdrr.device:deepdrr.device.simple_device.SimpleDevice.world_from_device}}
\pysigstartsignatures
\pysigline{\sphinxbfcode{\sphinxupquote{world\_from\_device}}}
\pysigstopsignatures
\sphinxAtStartPar
the “world\_from\_device” frame transformation for the device.
\begin{quote}\begin{description}
\sphinxlineitem{Type}
\sphinxAtStartPar
{\hyperref[\detokenize{deepdrr.geo:deepdrr.geo.core.FrameTransform}]{\sphinxcrossref{FrameTransform}}}

\end{description}\end{quote}

\end{fulllineitems}

\index{camera\_intrinsics (deepdrr.device.simple\_device.SimpleDevice property)@\spxentry{camera\_intrinsics}\spxextra{deepdrr.device.simple\_device.SimpleDevice property}}

\begin{fulllineitems}
\phantomsection\label{\detokenize{deepdrr.device:deepdrr.device.simple_device.SimpleDevice.camera_intrinsics}}
\pysigstartsignatures
\pysigline{\sphinxbfcode{\sphinxupquote{property\DUrole{w,w}{  }}}\sphinxbfcode{\sphinxupquote{camera\_intrinsics}}\sphinxbfcode{\sphinxupquote{\DUrole{p,p}{:}\DUrole{w,w}{  }{\hyperref[\detokenize{deepdrr.geo:deepdrr.geo.core.CameraIntrinsicTransform}]{\sphinxcrossref{CameraIntrinsicTransform}}}}}}
\pysigstopsignatures
\sphinxAtStartPar
Get the camera intrinsics for the device.
\begin{quote}\begin{description}
\sphinxlineitem{Returns}
\sphinxAtStartPar
the camera intrinsics for the device.

\sphinxlineitem{Return type}
\sphinxAtStartPar
{\hyperref[\detokenize{deepdrr.geo:deepdrr.geo.core.CameraIntrinsicTransform}]{\sphinxcrossref{CameraIntrinsicTransform}}}

\end{description}\end{quote}

\end{fulllineitems}

\index{device\_from\_camera3d (deepdrr.device.simple\_device.SimpleDevice property)@\spxentry{device\_from\_camera3d}\spxextra{deepdrr.device.simple\_device.SimpleDevice property}}

\begin{fulllineitems}
\phantomsection\label{\detokenize{deepdrr.device:deepdrr.device.simple_device.SimpleDevice.device_from_camera3d}}
\pysigstartsignatures
\pysigline{\sphinxbfcode{\sphinxupquote{property\DUrole{w,w}{  }}}\sphinxbfcode{\sphinxupquote{device\_from\_camera3d}}\sphinxbfcode{\sphinxupquote{\DUrole{p,p}{:}\DUrole{w,w}{  }{\hyperref[\detokenize{deepdrr.geo:deepdrr.geo.core.FrameTransform}]{\sphinxcrossref{FrameTransform}}}}}}
\pysigstopsignatures
\sphinxAtStartPar
Get the FrameTransform for the device’s camera3d\_from\_device frame (in the current pose).
\begin{quote}\begin{description}
\sphinxlineitem{Parameters}
\sphinxAtStartPar
\sphinxstyleliteralstrong{\sphinxupquote{camera3d\_transform}} ({\hyperref[\detokenize{deepdrr.geo:deepdrr.geo.core.FrameTransform}]{\sphinxcrossref{\sphinxstyleliteralemphasis{\sphinxupquote{FrameTransform}}}}}) \textendash{} the “camera3d\_from\_device” frame transformation for the device.

\sphinxlineitem{Returns}
\sphinxAtStartPar
the “device\_from\_camera3d” frame transformation for the device.

\sphinxlineitem{Return type}
\sphinxAtStartPar
{\hyperref[\detokenize{deepdrr.geo:deepdrr.geo.core.FrameTransform}]{\sphinxcrossref{FrameTransform}}}

\end{description}\end{quote}

\end{fulllineitems}

\index{pixel\_size (deepdrr.device.simple\_device.SimpleDevice attribute)@\spxentry{pixel\_size}\spxextra{deepdrr.device.simple\_device.SimpleDevice attribute}}

\begin{fulllineitems}
\phantomsection\label{\detokenize{deepdrr.device:id3}}
\pysigstartsignatures
\pysigline{\sphinxbfcode{\sphinxupquote{pixel\_size}}\sphinxbfcode{\sphinxupquote{\DUrole{p,p}{:}\DUrole{w,w}{  }float}}}
\pysigstopsignatures
\end{fulllineitems}

\index{sensor\_height (deepdrr.device.simple\_device.SimpleDevice attribute)@\spxentry{sensor\_height}\spxextra{deepdrr.device.simple\_device.SimpleDevice attribute}}

\begin{fulllineitems}
\phantomsection\label{\detokenize{deepdrr.device:id4}}
\pysigstartsignatures
\pysigline{\sphinxbfcode{\sphinxupquote{sensor\_height}}\sphinxbfcode{\sphinxupquote{\DUrole{p,p}{:}\DUrole{w,w}{  }int}}}
\pysigstopsignatures
\end{fulllineitems}

\index{sensor\_width (deepdrr.device.simple\_device.SimpleDevice attribute)@\spxentry{sensor\_width}\spxextra{deepdrr.device.simple\_device.SimpleDevice attribute}}

\begin{fulllineitems}
\phantomsection\label{\detokenize{deepdrr.device:id5}}
\pysigstartsignatures
\pysigline{\sphinxbfcode{\sphinxupquote{sensor\_width}}\sphinxbfcode{\sphinxupquote{\DUrole{p,p}{:}\DUrole{w,w}{  }int}}}
\pysigstopsignatures
\end{fulllineitems}

\index{set\_view() (deepdrr.device.simple\_device.SimpleDevice method)@\spxentry{set\_view()}\spxextra{deepdrr.device.simple\_device.SimpleDevice method}}

\begin{fulllineitems}
\phantomsection\label{\detokenize{deepdrr.device:deepdrr.device.simple_device.SimpleDevice.set_view}}
\pysigstartsignatures
\pysiglinewithargsret{\sphinxbfcode{\sphinxupquote{set\_view}}}{\sphinxparam{\DUrole{n,n}{point}\DUrole{p,p}{:}\DUrole{w,w}{  }\DUrole{n,n}{{\hyperref[\detokenize{deepdrr.geo:deepdrr.geo.core.Point3D}]{\sphinxcrossref{Point3D}}}\DUrole{w,w}{  }\DUrole{p,p}{|}\DUrole{w,w}{  }None}\DUrole{w,w}{  }\DUrole{o,o}{=}\DUrole{w,w}{  }\DUrole{default_value}{None}}\sphinxparamcomma \sphinxparam{\DUrole{n,n}{direction}\DUrole{p,p}{:}\DUrole{w,w}{  }\DUrole{n,n}{{\hyperref[\detokenize{deepdrr.geo:deepdrr.geo.core.Vector3D}]{\sphinxcrossref{Vector3D}}}\DUrole{w,w}{  }\DUrole{p,p}{|}\DUrole{w,w}{  }None}\DUrole{w,w}{  }\DUrole{o,o}{=}\DUrole{w,w}{  }\DUrole{default_value}{None}}\sphinxparamcomma \sphinxparam{\DUrole{n,n}{up}\DUrole{p,p}{:}\DUrole{w,w}{  }\DUrole{n,n}{{\hyperref[\detokenize{deepdrr.geo:deepdrr.geo.core.Vector3D}]{\sphinxcrossref{Vector3D}}}\DUrole{w,w}{  }\DUrole{p,p}{|}\DUrole{w,w}{  }None}\DUrole{w,w}{  }\DUrole{o,o}{=}\DUrole{w,w}{  }\DUrole{default_value}{None}}\sphinxparamcomma \sphinxparam{\DUrole{n,n}{source\_to\_point\_distance}\DUrole{p,p}{:}\DUrole{w,w}{  }\DUrole{n,n}{float\DUrole{w,w}{  }\DUrole{p,p}{|}\DUrole{w,w}{  }None}\DUrole{w,w}{  }\DUrole{o,o}{=}\DUrole{w,w}{  }\DUrole{default_value}{None}}\sphinxparamcomma \sphinxparam{\DUrole{n,n}{source\_to\_point\_fraction}\DUrole{p,p}{:}\DUrole{w,w}{  }\DUrole{n,n}{float}\DUrole{w,w}{  }\DUrole{o,o}{=}\DUrole{w,w}{  }\DUrole{default_value}{0.5}}}{}
\pysigstopsignatures
\sphinxAtStartPar
Set the view of the device.

\sphinxAtStartPar
Can be called with a Ray3D as the first argument, by doing \sphinxtitleref{device.set\_view(*ray)}.
\begin{quote}\begin{description}
\sphinxlineitem{Parameters}\begin{itemize}
\item {} 
\sphinxAtStartPar
\sphinxstyleliteralstrong{\sphinxupquote{center}} ({\hyperref[\detokenize{deepdrr.geo:deepdrr.geo.core.Point3D}]{\sphinxcrossref{\sphinxstyleliteralemphasis{\sphinxupquote{Point3D}}}}}) \textendash{} the point at the center of the source and detector, in world coordinates. If None,
the point is left unchanged (rotation only). Default: None.

\item {} 
\sphinxAtStartPar
\sphinxstyleliteralstrong{\sphinxupquote{direction}} ({\hyperref[\detokenize{deepdrr.geo:deepdrr.geo.core.Vector3D}]{\sphinxcrossref{\sphinxstyleliteralemphasis{\sphinxupquote{Vector3D}}}}}) \textendash{} the direction from the source to the detector, in world coordinates. If None,
the direction is set to the +Z axis. Default: None.

\item {} 
\sphinxAtStartPar
\sphinxstyleliteralstrong{\sphinxupquote{up}} ({\hyperref[\detokenize{deepdrr.geo:deepdrr.geo.core.Vector3D}]{\sphinxcrossref{\sphinxstyleliteralemphasis{\sphinxupquote{Vector3D}}}}}) \textendash{} the up vector of the image, in world\_coordinates. It’s projection
corresponds to the \sphinxhyphen{}Y axis in the camera3d frame. If None, the up vector is set to the \sphinxhyphen{}Y
axis of the device frame.

\item {} 
\sphinxAtStartPar
\sphinxstyleliteralstrong{\sphinxupquote{source\_to\_point\_distance}} (\sphinxstyleliteralemphasis{\sphinxupquote{float}}) \textendash{} the distance from the source to the point. If None, the distance
is \sphinxtitleref{source\_to\_point\_fraction} of the source\sphinxhyphen{}to\sphinxhyphen{}detector distance. Default: None.

\item {} 
\sphinxAtStartPar
\sphinxstyleliteralstrong{\sphinxupquote{source\_to\_point\_fraction}} (\sphinxstyleliteralemphasis{\sphinxupquote{float}}) \textendash{} the fraction of the source\sphinxhyphen{}to\sphinxhyphen{}detector distance to use as the
source\sphinxhyphen{}to\sphinxhyphen{}point distance. Default: 0.5.

\end{itemize}

\end{description}\end{quote}

\end{fulllineitems}

\index{source\_to\_detector\_distance (deepdrr.device.simple\_device.SimpleDevice attribute)@\spxentry{source\_to\_detector\_distance}\spxextra{deepdrr.device.simple\_device.SimpleDevice attribute}}

\begin{fulllineitems}
\phantomsection\label{\detokenize{deepdrr.device:id6}}
\pysigstartsignatures
\pysigline{\sphinxbfcode{\sphinxupquote{source\_to\_detector\_distance}}\sphinxbfcode{\sphinxupquote{\DUrole{p,p}{:}\DUrole{w,w}{  }float}}}
\pysigstopsignatures
\end{fulllineitems}

\index{world\_from\_device (deepdrr.device.simple\_device.SimpleDevice attribute)@\spxentry{world\_from\_device}\spxextra{deepdrr.device.simple\_device.SimpleDevice attribute}}

\begin{fulllineitems}
\phantomsection\label{\detokenize{deepdrr.device:id7}}
\pysigstartsignatures
\pysigline{\sphinxbfcode{\sphinxupquote{world\_from\_device}}\sphinxbfcode{\sphinxupquote{\DUrole{p,p}{:}\DUrole{w,w}{  }{\hyperref[\detokenize{deepdrr.geo:deepdrr.geo.core.FrameTransform}]{\sphinxcrossref{FrameTransform}}}}}}
\pysigstopsignatures
\end{fulllineitems}


\end{fulllineitems}



\subsection{Module contents}
\label{\detokenize{deepdrr.device:module-deepdrr.device}}\label{\detokenize{deepdrr.device:module-contents}}\index{module@\spxentry{module}!deepdrr.device@\spxentry{deepdrr.device}}\index{deepdrr.device@\spxentry{deepdrr.device}!module@\spxentry{module}}\index{CArm (class in deepdrr.device)@\spxentry{CArm}\spxextra{class in deepdrr.device}}

\begin{fulllineitems}
\phantomsection\label{\detokenize{deepdrr.device:deepdrr.device.CArm}}
\pysigstartsignatures
\pysiglinewithargsret{\sphinxbfcode{\sphinxupquote{class\DUrole{w,w}{  }}}\sphinxcode{\sphinxupquote{deepdrr.device.}}\sphinxbfcode{\sphinxupquote{CArm}}}{\sphinxparam{\DUrole{n,n}{isocenter\_distance}\DUrole{p,p}{:}\DUrole{w,w}{  }\DUrole{n,n}{float}}\sphinxparamcomma \sphinxparam{\DUrole{n,n}{isocenter}\DUrole{p,p}{:}\DUrole{w,w}{  }\DUrole{n,n}{{\hyperref[\detokenize{deepdrr.geo:deepdrr.geo.core.Point3D}]{\sphinxcrossref{Point3D}}}\DUrole{w,w}{  }\DUrole{p,p}{|}\DUrole{w,w}{  }None}\DUrole{w,w}{  }\DUrole{o,o}{=}\DUrole{w,w}{  }\DUrole{default_value}{None}}\sphinxparamcomma \sphinxparam{\DUrole{n,n}{phi}\DUrole{p,p}{:}\DUrole{w,w}{  }\DUrole{n,n}{float}\DUrole{w,w}{  }\DUrole{o,o}{=}\DUrole{w,w}{  }\DUrole{default_value}{0}}\sphinxparamcomma \sphinxparam{\DUrole{n,n}{theta}\DUrole{p,p}{:}\DUrole{w,w}{  }\DUrole{n,n}{float}\DUrole{w,w}{  }\DUrole{o,o}{=}\DUrole{w,w}{  }\DUrole{default_value}{0}}\sphinxparamcomma \sphinxparam{\DUrole{n,n}{rho}\DUrole{p,p}{:}\DUrole{w,w}{  }\DUrole{n,n}{float}\DUrole{w,w}{  }\DUrole{o,o}{=}\DUrole{w,w}{  }\DUrole{default_value}{0}}\sphinxparamcomma \sphinxparam{\DUrole{n,n}{degrees}\DUrole{p,p}{:}\DUrole{w,w}{  }\DUrole{n,n}{bool}\DUrole{w,w}{  }\DUrole{o,o}{=}\DUrole{w,w}{  }\DUrole{default_value}{False}}}{}
\pysigstopsignatures
\sphinxAtStartPar
Bases: \sphinxcode{\sphinxupquote{object}}

\sphinxAtStartPar
C\sphinxhyphen{}arm device for positioning a camera in space.

\sphinxAtStartPar
It is suggested to use MobileCArm instead.
\index{camera3d\_from\_world (deepdrr.device.CArm property)@\spxentry{camera3d\_from\_world}\spxextra{deepdrr.device.CArm property}}

\begin{fulllineitems}
\phantomsection\label{\detokenize{deepdrr.device:deepdrr.device.CArm.camera3d_from_world}}
\pysigstartsignatures
\pysigline{\sphinxbfcode{\sphinxupquote{property\DUrole{w,w}{  }}}\sphinxbfcode{\sphinxupquote{camera3d\_from\_world}}\sphinxbfcode{\sphinxupquote{\DUrole{p,p}{:}\DUrole{w,w}{  }{\hyperref[\detokenize{deepdrr.geo:deepdrr.geo.core.FrameTransform}]{\sphinxcrossref{FrameTransform}}}}}}
\pysigstopsignatures
\end{fulllineitems}

\index{get\_camera3d\_from\_world() (deepdrr.device.CArm method)@\spxentry{get\_camera3d\_from\_world()}\spxextra{deepdrr.device.CArm method}}

\begin{fulllineitems}
\phantomsection\label{\detokenize{deepdrr.device:deepdrr.device.CArm.get_camera3d_from_world}}
\pysigstartsignatures
\pysiglinewithargsret{\sphinxbfcode{\sphinxupquote{get\_camera3d\_from\_world}}}{\sphinxparam{\DUrole{n,n}{isocenter}\DUrole{p,p}{:}\DUrole{w,w}{  }\DUrole{n,n}{{\hyperref[\detokenize{deepdrr.geo:deepdrr.geo.core.Point3D}]{\sphinxcrossref{Point3D}}}}}\sphinxparamcomma \sphinxparam{\DUrole{n,n}{phi}\DUrole{p,p}{:}\DUrole{w,w}{  }\DUrole{n,n}{float}}\sphinxparamcomma \sphinxparam{\DUrole{n,n}{theta}\DUrole{p,p}{:}\DUrole{w,w}{  }\DUrole{n,n}{float}}\sphinxparamcomma \sphinxparam{\DUrole{n,n}{rho}\DUrole{p,p}{:}\DUrole{w,w}{  }\DUrole{n,n}{float\DUrole{w,w}{  }\DUrole{p,p}{|}\DUrole{w,w}{  }None}\DUrole{w,w}{  }\DUrole{o,o}{=}\DUrole{w,w}{  }\DUrole{default_value}{0}}\sphinxparamcomma \sphinxparam{\DUrole{n,n}{degrees}\DUrole{p,p}{:}\DUrole{w,w}{  }\DUrole{n,n}{bool}\DUrole{w,w}{  }\DUrole{o,o}{=}\DUrole{w,w}{  }\DUrole{default_value}{False}}}{{ $\rightarrow$ {\hyperref[\detokenize{deepdrr.geo:deepdrr.geo.core.FrameTransform}]{\sphinxcrossref{FrameTransform}}}}}
\pysigstopsignatures
\sphinxAtStartPar
Get the FrameTransform for the C\sphinxhyphen{}Arm device at the given pose.

\sphinxAtStartPar
This ignores the internal state except for the isocenter\_distance.
\begin{quote}\begin{description}
\sphinxlineitem{Parameters}\begin{itemize}
\item {} 
\sphinxAtStartPar
\sphinxstyleliteralstrong{\sphinxupquote{isocenter}} ({\hyperref[\detokenize{deepdrr.geo:deepdrr.geo.Point3D}]{\sphinxcrossref{\sphinxstyleliteralemphasis{\sphinxupquote{geo.Point3D}}}}}) \textendash{} isocenter of the device.

\item {} 
\sphinxAtStartPar
\sphinxstyleliteralstrong{\sphinxupquote{phi}} (\sphinxstyleliteralemphasis{\sphinxupquote{float}}) \textendash{} CRAN/CAUD angle of the C\sphinxhyphen{}Arm (along the actual arc of the arm)

\item {} 
\sphinxAtStartPar
\sphinxstyleliteralstrong{\sphinxupquote{theta}} (\sphinxstyleliteralemphasis{\sphinxupquote{float}}) \textendash{} Lect/Right angulation of C\sphinxhyphen{}arm (rotation at the base)

\item {} 
\sphinxAtStartPar
\sphinxstyleliteralstrong{\sphinxupquote{rho}} (\sphinxstyleliteralemphasis{\sphinxupquote{Optional}}\sphinxstyleliteralemphasis{\sphinxupquote{{[}}}\sphinxstyleliteralemphasis{\sphinxupquote{float}}\sphinxstyleliteralemphasis{\sphinxupquote{{]}}}\sphinxstyleliteralemphasis{\sphinxupquote{, }}\sphinxstyleliteralemphasis{\sphinxupquote{optional}}) \textendash{} rotation about principle axis, after main rotation. Defaults to 0.

\item {} 
\sphinxAtStartPar
\sphinxstyleliteralstrong{\sphinxupquote{degrees}} (\sphinxstyleliteralemphasis{\sphinxupquote{bool}}\sphinxstyleliteralemphasis{\sphinxupquote{, }}\sphinxstyleliteralemphasis{\sphinxupquote{optional}}) \textendash{} Whether given angles are in degrees. Defaults to False.

\item {} 
\sphinxAtStartPar
\sphinxstyleliteralstrong{\sphinxupquote{offset}} (\sphinxstyleliteralemphasis{\sphinxupquote{Optional}}\sphinxstyleliteralemphasis{\sphinxupquote{{[}}}{\hyperref[\detokenize{deepdrr.geo:deepdrr.geo.core.Vector3D}]{\sphinxcrossref{\sphinxstyleliteralemphasis{\sphinxupquote{Vector3D}}}}}\sphinxstyleliteralemphasis{\sphinxupquote{{]}}}\sphinxstyleliteralemphasis{\sphinxupquote{, }}\sphinxstyleliteralemphasis{\sphinxupquote{optional}}) \textendash{} world\sphinxhyphen{}space offset to add to the initial C\sphinxhyphen{}arm isocenter. Defaults to None.

\end{itemize}

\sphinxlineitem{Returns}
\sphinxAtStartPar
the extrinsic matrix or “camera3d\_from\_world” frame transformation for the oriented C\sphinxhyphen{}Arm camera.

\sphinxlineitem{Return type}
\sphinxAtStartPar
{\hyperref[\detokenize{deepdrr.geo:deepdrr.geo.core.FrameTransform}]{\sphinxcrossref{FrameTransform}}}

\end{description}\end{quote}

\end{fulllineitems}

\index{move\_by() (deepdrr.device.CArm method)@\spxentry{move\_by()}\spxextra{deepdrr.device.CArm method}}

\begin{fulllineitems}
\phantomsection\label{\detokenize{deepdrr.device:deepdrr.device.CArm.move_by}}
\pysigstartsignatures
\pysiglinewithargsret{\sphinxbfcode{\sphinxupquote{move\_by}}}{\sphinxparam{\DUrole{n,n}{delta\_isocenter}\DUrole{p,p}{:}\DUrole{w,w}{  }\DUrole{n,n}{{\hyperref[\detokenize{deepdrr.geo:deepdrr.geo.core.Vector3D}]{\sphinxcrossref{Vector3D}}}\DUrole{w,w}{  }\DUrole{p,p}{|}\DUrole{w,w}{  }None}\DUrole{w,w}{  }\DUrole{o,o}{=}\DUrole{w,w}{  }\DUrole{default_value}{None}}\sphinxparamcomma \sphinxparam{\DUrole{n,n}{delta\_phi}\DUrole{p,p}{:}\DUrole{w,w}{  }\DUrole{n,n}{float\DUrole{w,w}{  }\DUrole{p,p}{|}\DUrole{w,w}{  }None}\DUrole{w,w}{  }\DUrole{o,o}{=}\DUrole{w,w}{  }\DUrole{default_value}{None}}\sphinxparamcomma \sphinxparam{\DUrole{n,n}{delta\_theta}\DUrole{p,p}{:}\DUrole{w,w}{  }\DUrole{n,n}{float\DUrole{w,w}{  }\DUrole{p,p}{|}\DUrole{w,w}{  }None}\DUrole{w,w}{  }\DUrole{o,o}{=}\DUrole{w,w}{  }\DUrole{default_value}{None}}\sphinxparamcomma \sphinxparam{\DUrole{n,n}{delta\_rho}\DUrole{p,p}{:}\DUrole{w,w}{  }\DUrole{n,n}{float\DUrole{w,w}{  }\DUrole{p,p}{|}\DUrole{w,w}{  }None}\DUrole{w,w}{  }\DUrole{o,o}{=}\DUrole{w,w}{  }\DUrole{default_value}{None}}\sphinxparamcomma \sphinxparam{\DUrole{n,n}{degrees}\DUrole{p,p}{:}\DUrole{w,w}{  }\DUrole{n,n}{bool}\DUrole{w,w}{  }\DUrole{o,o}{=}\DUrole{w,w}{  }\DUrole{default_value}{False}}\sphinxparamcomma \sphinxparam{\DUrole{n,n}{min\_isocenter}\DUrole{p,p}{:}\DUrole{w,w}{  }\DUrole{n,n}{{\hyperref[\detokenize{deepdrr.geo:deepdrr.geo.core.Point3D}]{\sphinxcrossref{Point3D}}}\DUrole{w,w}{  }\DUrole{p,p}{|}\DUrole{w,w}{  }None}\DUrole{w,w}{  }\DUrole{o,o}{=}\DUrole{w,w}{  }\DUrole{default_value}{None}}\sphinxparamcomma \sphinxparam{\DUrole{n,n}{max\_isocenter}\DUrole{p,p}{:}\DUrole{w,w}{  }\DUrole{n,n}{{\hyperref[\detokenize{deepdrr.geo:deepdrr.geo.core.Point3D}]{\sphinxcrossref{Point3D}}}\DUrole{w,w}{  }\DUrole{p,p}{|}\DUrole{w,w}{  }None}\DUrole{w,w}{  }\DUrole{o,o}{=}\DUrole{w,w}{  }\DUrole{default_value}{None}}\sphinxparamcomma \sphinxparam{\DUrole{n,n}{min\_phi}\DUrole{p,p}{:}\DUrole{w,w}{  }\DUrole{n,n}{float\DUrole{w,w}{  }\DUrole{p,p}{|}\DUrole{w,w}{  }None}\DUrole{w,w}{  }\DUrole{o,o}{=}\DUrole{w,w}{  }\DUrole{default_value}{None}}\sphinxparamcomma \sphinxparam{\DUrole{n,n}{max\_phi}\DUrole{p,p}{:}\DUrole{w,w}{  }\DUrole{n,n}{float\DUrole{w,w}{  }\DUrole{p,p}{|}\DUrole{w,w}{  }None}\DUrole{w,w}{  }\DUrole{o,o}{=}\DUrole{w,w}{  }\DUrole{default_value}{None}}\sphinxparamcomma \sphinxparam{\DUrole{n,n}{min\_theta}\DUrole{p,p}{:}\DUrole{w,w}{  }\DUrole{n,n}{float\DUrole{w,w}{  }\DUrole{p,p}{|}\DUrole{w,w}{  }None}\DUrole{w,w}{  }\DUrole{o,o}{=}\DUrole{w,w}{  }\DUrole{default_value}{None}}\sphinxparamcomma \sphinxparam{\DUrole{n,n}{max\_theta}\DUrole{p,p}{:}\DUrole{w,w}{  }\DUrole{n,n}{float\DUrole{w,w}{  }\DUrole{p,p}{|}\DUrole{w,w}{  }None}\DUrole{w,w}{  }\DUrole{o,o}{=}\DUrole{w,w}{  }\DUrole{default_value}{None}}}{{ $\rightarrow$ None}}
\pysigstopsignatures
\sphinxAtStartPar
Move the C\sphinxhyphen{}arm by the specified deltas.

\sphinxAtStartPar
Clips the internal state by the provided values if not None.
\begin{quote}\begin{description}
\sphinxlineitem{Parameters}\begin{itemize}
\item {} 
\sphinxAtStartPar
\sphinxstyleliteralstrong{\sphinxupquote{delta\_isocenter}} ({\hyperref[\detokenize{deepdrr.geo:deepdrr.geo.core.Vector3D}]{\sphinxcrossref{\sphinxstyleliteralemphasis{\sphinxupquote{Vector3D}}}}}) \textendash{} offset for the isocenter of the C\sphinxhyphen{}arm in world\sphinxhyphen{}space. This is the center about which rotations are performed.

\item {} 
\sphinxAtStartPar
\sphinxstyleliteralstrong{\sphinxupquote{phi}} (\sphinxstyleliteralemphasis{\sphinxupquote{float}}) \textendash{} CRAN/CAUD angle of the C\sphinxhyphen{}Arm (along the actual arc of the arm)

\item {} 
\sphinxAtStartPar
\sphinxstyleliteralstrong{\sphinxupquote{theta}} (\sphinxstyleliteralemphasis{\sphinxupquote{float}}) \textendash{} Lect/Right angulation of C\sphinxhyphen{}arm (rotation at the base)

\item {} 
\sphinxAtStartPar
\sphinxstyleliteralstrong{\sphinxupquote{rho}} (\sphinxstyleliteralemphasis{\sphinxupquote{float}}\sphinxstyleliteralemphasis{\sphinxupquote{, }}\sphinxstyleliteralemphasis{\sphinxupquote{optional}}) \textendash{} rotation about principle axis, after main rotation. Defaults to 0.

\item {} 
\sphinxAtStartPar
\sphinxstyleliteralstrong{\sphinxupquote{degrees}} (\sphinxstyleliteralemphasis{\sphinxupquote{bool}}\sphinxstyleliteralemphasis{\sphinxupquote{, }}\sphinxstyleliteralemphasis{\sphinxupquote{optional}}) \textendash{} Whether given angles are in degrees. Defaults to False.

\end{itemize}

\end{description}\end{quote}

\end{fulllineitems}

\index{move\_to() (deepdrr.device.CArm method)@\spxentry{move\_to()}\spxextra{deepdrr.device.CArm method}}

\begin{fulllineitems}
\phantomsection\label{\detokenize{deepdrr.device:deepdrr.device.CArm.move_to}}
\pysigstartsignatures
\pysiglinewithargsret{\sphinxbfcode{\sphinxupquote{move\_to}}}{\sphinxparam{\DUrole{n,n}{isocenter}\DUrole{p,p}{:}\DUrole{w,w}{  }\DUrole{n,n}{{\hyperref[\detokenize{deepdrr.geo:deepdrr.geo.core.Point3D}]{\sphinxcrossref{Point3D}}}\DUrole{w,w}{  }\DUrole{p,p}{|}\DUrole{w,w}{  }None}\DUrole{w,w}{  }\DUrole{o,o}{=}\DUrole{w,w}{  }\DUrole{default_value}{None}}\sphinxparamcomma \sphinxparam{\DUrole{n,n}{phi}\DUrole{p,p}{:}\DUrole{w,w}{  }\DUrole{n,n}{float\DUrole{w,w}{  }\DUrole{p,p}{|}\DUrole{w,w}{  }None}\DUrole{w,w}{  }\DUrole{o,o}{=}\DUrole{w,w}{  }\DUrole{default_value}{None}}\sphinxparamcomma \sphinxparam{\DUrole{n,n}{theta}\DUrole{p,p}{:}\DUrole{w,w}{  }\DUrole{n,n}{float\DUrole{w,w}{  }\DUrole{p,p}{|}\DUrole{w,w}{  }None}\DUrole{w,w}{  }\DUrole{o,o}{=}\DUrole{w,w}{  }\DUrole{default_value}{None}}\sphinxparamcomma \sphinxparam{\DUrole{n,n}{rho}\DUrole{p,p}{:}\DUrole{w,w}{  }\DUrole{n,n}{float\DUrole{w,w}{  }\DUrole{p,p}{|}\DUrole{w,w}{  }None}\DUrole{w,w}{  }\DUrole{o,o}{=}\DUrole{w,w}{  }\DUrole{default_value}{None}}\sphinxparamcomma \sphinxparam{\DUrole{n,n}{degrees}\DUrole{p,p}{:}\DUrole{w,w}{  }\DUrole{n,n}{bool}\DUrole{w,w}{  }\DUrole{o,o}{=}\DUrole{w,w}{  }\DUrole{default_value}{False}}}{{ $\rightarrow$ None}}
\pysigstopsignatures
\sphinxAtStartPar
Move the C\sphinxhyphen{}arm to the specified pose.
\begin{quote}\begin{description}
\sphinxlineitem{Parameters}\begin{itemize}
\item {} 
\sphinxAtStartPar
\sphinxstyleliteralstrong{\sphinxupquote{isocenter}} ({\hyperref[\detokenize{deepdrr.geo:deepdrr.geo.core.Point3D}]{\sphinxcrossref{\sphinxstyleliteralemphasis{\sphinxupquote{Point3D}}}}}) \textendash{} New isocenter of the C\sphinxhyphen{}arm in device space. This is the center about which rotations are performed.

\item {} 
\sphinxAtStartPar
\sphinxstyleliteralstrong{\sphinxupquote{phi}} (\sphinxstyleliteralemphasis{\sphinxupquote{float}}) \textendash{} CRAN/CAUD angle of the C\sphinxhyphen{}Arm (along the actual arc of the arm)

\item {} 
\sphinxAtStartPar
\sphinxstyleliteralstrong{\sphinxupquote{theta}} (\sphinxstyleliteralemphasis{\sphinxupquote{float}}) \textendash{} Lect/Right angulation of C\sphinxhyphen{}arm (rotation at the base)

\item {} 
\sphinxAtStartPar
\sphinxstyleliteralstrong{\sphinxupquote{rho}} (\sphinxstyleliteralemphasis{\sphinxupquote{float}}\sphinxstyleliteralemphasis{\sphinxupquote{, }}\sphinxstyleliteralemphasis{\sphinxupquote{optional}}) \textendash{} rotation about principle axis, after main rotation. Defaults to 0.

\item {} 
\sphinxAtStartPar
\sphinxstyleliteralstrong{\sphinxupquote{degrees}} (\sphinxstyleliteralemphasis{\sphinxupquote{bool}}\sphinxstyleliteralemphasis{\sphinxupquote{, }}\sphinxstyleliteralemphasis{\sphinxupquote{optional}}) \textendash{} Whether given angles are in degrees. Defaults to False.

\end{itemize}

\end{description}\end{quote}

\end{fulllineitems}


\end{fulllineitems}

\index{Device (class in deepdrr.device)@\spxentry{Device}\spxextra{class in deepdrr.device}}

\begin{fulllineitems}
\phantomsection\label{\detokenize{deepdrr.device:deepdrr.device.Device}}
\pysigstartsignatures
\pysigline{\sphinxbfcode{\sphinxupquote{class\DUrole{w,w}{  }}}\sphinxcode{\sphinxupquote{deepdrr.device.}}\sphinxbfcode{\sphinxupquote{Device}}}
\pysigstopsignatures
\sphinxAtStartPar
Bases: \sphinxcode{\sphinxupquote{ABC}}

\sphinxAtStartPar
A parent class representing X\sphinxhyphen{}ray device interfaces in DeepDRR.
\begin{description}
\sphinxlineitem{To implement a sub class, the following methods/attributes must be implemented:}\begin{itemize}
\item {} 
\sphinxAtStartPar
device\_from\_camera3d

\end{itemize}

\end{description}
\index{sensor\_height (deepdrr.device.Device attribute)@\spxentry{sensor\_height}\spxextra{deepdrr.device.Device attribute}}

\begin{fulllineitems}
\phantomsection\label{\detokenize{deepdrr.device:deepdrr.device.Device.sensor_height}}
\pysigstartsignatures
\pysigline{\sphinxbfcode{\sphinxupquote{sensor\_height}}}
\pysigstopsignatures
\sphinxAtStartPar
the height of the sensor in pixels.
\begin{quote}\begin{description}
\sphinxlineitem{Type}
\sphinxAtStartPar
int

\end{description}\end{quote}

\end{fulllineitems}

\index{sensor\_width (deepdrr.device.Device attribute)@\spxentry{sensor\_width}\spxextra{deepdrr.device.Device attribute}}

\begin{fulllineitems}
\phantomsection\label{\detokenize{deepdrr.device:deepdrr.device.Device.sensor_width}}
\pysigstartsignatures
\pysigline{\sphinxbfcode{\sphinxupquote{sensor\_width}}}
\pysigstopsignatures
\sphinxAtStartPar
the width of the sensor in pixels.
\begin{quote}\begin{description}
\sphinxlineitem{Type}
\sphinxAtStartPar
int

\end{description}\end{quote}

\end{fulllineitems}

\index{pixel\_size (deepdrr.device.Device attribute)@\spxentry{pixel\_size}\spxextra{deepdrr.device.Device attribute}}

\begin{fulllineitems}
\phantomsection\label{\detokenize{deepdrr.device:deepdrr.device.Device.pixel_size}}
\pysigstartsignatures
\pysigline{\sphinxbfcode{\sphinxupquote{pixel\_size}}}
\pysigstopsignatures
\sphinxAtStartPar
the size of a pixel in mm.
\begin{quote}\begin{description}
\sphinxlineitem{Type}
\sphinxAtStartPar
float

\end{description}\end{quote}

\end{fulllineitems}

\index{camera3d\_from\_device (deepdrr.device.Device property)@\spxentry{camera3d\_from\_device}\spxextra{deepdrr.device.Device property}}

\begin{fulllineitems}
\phantomsection\label{\detokenize{deepdrr.device:deepdrr.device.Device.camera3d_from_device}}
\pysigstartsignatures
\pysigline{\sphinxbfcode{\sphinxupquote{property\DUrole{w,w}{  }}}\sphinxbfcode{\sphinxupquote{camera3d\_from\_device}}\sphinxbfcode{\sphinxupquote{\DUrole{p,p}{:}\DUrole{w,w}{  }{\hyperref[\detokenize{deepdrr.geo:deepdrr.geo.core.FrameTransform}]{\sphinxcrossref{FrameTransform}}}}}}
\pysigstopsignatures
\sphinxAtStartPar
Get the FrameTransform for the device’s camera3d\_from\_device frame (in the current pose).
\begin{quote}\begin{description}
\sphinxlineitem{Returns}
\sphinxAtStartPar
the “camera3d\_from\_device” frame transformation for the device.

\sphinxlineitem{Return type}
\sphinxAtStartPar
{\hyperref[\detokenize{deepdrr.geo:deepdrr.geo.core.FrameTransform}]{\sphinxcrossref{FrameTransform}}}

\end{description}\end{quote}

\end{fulllineitems}

\index{camera3d\_from\_index (deepdrr.device.Device property)@\spxentry{camera3d\_from\_index}\spxextra{deepdrr.device.Device property}}

\begin{fulllineitems}
\phantomsection\label{\detokenize{deepdrr.device:deepdrr.device.Device.camera3d_from_index}}
\pysigstartsignatures
\pysigline{\sphinxbfcode{\sphinxupquote{property\DUrole{w,w}{  }}}\sphinxbfcode{\sphinxupquote{camera3d\_from\_index}}\sphinxbfcode{\sphinxupquote{\DUrole{p,p}{:}\DUrole{w,w}{  }{\hyperref[\detokenize{deepdrr.geo:deepdrr.geo.core.Transform}]{\sphinxcrossref{Transform}}}}}}
\pysigstopsignatures
\end{fulllineitems}

\index{camera3d\_from\_world (deepdrr.device.Device property)@\spxentry{camera3d\_from\_world}\spxextra{deepdrr.device.Device property}}

\begin{fulllineitems}
\phantomsection\label{\detokenize{deepdrr.device:deepdrr.device.Device.camera3d_from_world}}
\pysigstartsignatures
\pysigline{\sphinxbfcode{\sphinxupquote{property\DUrole{w,w}{  }}}\sphinxbfcode{\sphinxupquote{camera3d\_from\_world}}\sphinxbfcode{\sphinxupquote{\DUrole{p,p}{:}\DUrole{w,w}{  }{\hyperref[\detokenize{deepdrr.geo:deepdrr.geo.core.FrameTransform}]{\sphinxcrossref{FrameTransform}}}}}}
\pysigstopsignatures
\sphinxAtStartPar
Get the FrameTransform for the device’s camera3d\_from\_world frame (in the current pose).
\begin{quote}\begin{description}
\sphinxlineitem{Returns}
\sphinxAtStartPar
the “camera3d\_from\_world” frame transformation for the device.

\sphinxlineitem{Return type}
\sphinxAtStartPar
{\hyperref[\detokenize{deepdrr.geo:deepdrr.geo.core.FrameTransform}]{\sphinxcrossref{FrameTransform}}}

\end{description}\end{quote}

\end{fulllineitems}

\index{camera\_intrinsics (deepdrr.device.Device attribute)@\spxentry{camera\_intrinsics}\spxextra{deepdrr.device.Device attribute}}

\begin{fulllineitems}
\phantomsection\label{\detokenize{deepdrr.device:deepdrr.device.Device.camera_intrinsics}}
\pysigstartsignatures
\pysigline{\sphinxbfcode{\sphinxupquote{camera\_intrinsics}}\sphinxbfcode{\sphinxupquote{\DUrole{p,p}{:}\DUrole{w,w}{  }{\hyperref[\detokenize{deepdrr.geo:deepdrr.geo.core.CameraIntrinsicTransform}]{\sphinxcrossref{CameraIntrinsicTransform}}}}}}
\pysigstopsignatures
\end{fulllineitems}

\index{detector\_height (deepdrr.device.Device property)@\spxentry{detector\_height}\spxextra{deepdrr.device.Device property}}

\begin{fulllineitems}
\phantomsection\label{\detokenize{deepdrr.device:deepdrr.device.Device.detector_height}}
\pysigstartsignatures
\pysigline{\sphinxbfcode{\sphinxupquote{property\DUrole{w,w}{  }}}\sphinxbfcode{\sphinxupquote{detector\_height}}\sphinxbfcode{\sphinxupquote{\DUrole{p,p}{:}\DUrole{w,w}{  }float}}}
\pysigstopsignatures
\sphinxAtStartPar
Height of the detector in mm.

\end{fulllineitems}

\index{detector\_width (deepdrr.device.Device property)@\spxentry{detector\_width}\spxextra{deepdrr.device.Device property}}

\begin{fulllineitems}
\phantomsection\label{\detokenize{deepdrr.device:deepdrr.device.Device.detector_width}}
\pysigstartsignatures
\pysigline{\sphinxbfcode{\sphinxupquote{property\DUrole{w,w}{  }}}\sphinxbfcode{\sphinxupquote{detector\_width}}\sphinxbfcode{\sphinxupquote{\DUrole{p,p}{:}\DUrole{w,w}{  }float}}}
\pysigstopsignatures
\sphinxAtStartPar
Width of the detector in mm.

\end{fulllineitems}

\index{device\_from\_camera3d (deepdrr.device.Device property)@\spxentry{device\_from\_camera3d}\spxextra{deepdrr.device.Device property}}

\begin{fulllineitems}
\phantomsection\label{\detokenize{deepdrr.device:deepdrr.device.Device.device_from_camera3d}}
\pysigstartsignatures
\pysigline{\sphinxbfcode{\sphinxupquote{abstract\DUrole{w,w}{  }property\DUrole{w,w}{  }}}\sphinxbfcode{\sphinxupquote{device\_from\_camera3d}}\sphinxbfcode{\sphinxupquote{\DUrole{p,p}{:}\DUrole{w,w}{  }{\hyperref[\detokenize{deepdrr.geo:deepdrr.geo.core.FrameTransform}]{\sphinxcrossref{FrameTransform}}}}}}
\pysigstopsignatures
\sphinxAtStartPar
Get the FrameTransform for the device’s camera3d\_from\_device frame (in the current pose).
\begin{quote}\begin{description}
\sphinxlineitem{Parameters}
\sphinxAtStartPar
\sphinxstyleliteralstrong{\sphinxupquote{camera3d\_transform}} ({\hyperref[\detokenize{deepdrr.geo:deepdrr.geo.core.FrameTransform}]{\sphinxcrossref{\sphinxstyleliteralemphasis{\sphinxupquote{FrameTransform}}}}}) \textendash{} the “camera3d\_from\_device” frame transformation for the device.

\sphinxlineitem{Returns}
\sphinxAtStartPar
the “device\_from\_camera3d” frame transformation for the device.

\sphinxlineitem{Return type}
\sphinxAtStartPar
{\hyperref[\detokenize{deepdrr.geo:deepdrr.geo.core.FrameTransform}]{\sphinxcrossref{FrameTransform}}}

\end{description}\end{quote}

\end{fulllineitems}

\index{device\_from\_world (deepdrr.device.Device property)@\spxentry{device\_from\_world}\spxextra{deepdrr.device.Device property}}

\begin{fulllineitems}
\phantomsection\label{\detokenize{deepdrr.device:deepdrr.device.Device.device_from_world}}
\pysigstartsignatures
\pysigline{\sphinxbfcode{\sphinxupquote{property\DUrole{w,w}{  }}}\sphinxbfcode{\sphinxupquote{device\_from\_world}}\sphinxbfcode{\sphinxupquote{\DUrole{p,p}{:}\DUrole{w,w}{  }{\hyperref[\detokenize{deepdrr.geo:deepdrr.geo.core.FrameTransform}]{\sphinxcrossref{FrameTransform}}}}}}
\pysigstopsignatures
\sphinxAtStartPar
Get the FrameTransform for the device’s local frame.
\begin{quote}\begin{description}
\sphinxlineitem{Parameters}
\sphinxAtStartPar
\sphinxstyleliteralstrong{\sphinxupquote{world\_transform}} ({\hyperref[\detokenize{deepdrr.geo:deepdrr.geo.core.FrameTransform}]{\sphinxcrossref{\sphinxstyleliteralemphasis{\sphinxupquote{FrameTransform}}}}}) \textendash{} the “world\_from\_device” frame transformation for the device.

\sphinxlineitem{Returns}
\sphinxAtStartPar
the “device\_from\_world” frame transformation for the device.

\sphinxlineitem{Return type}
\sphinxAtStartPar
{\hyperref[\detokenize{deepdrr.geo:deepdrr.geo.core.FrameTransform}]{\sphinxcrossref{FrameTransform}}}

\end{description}\end{quote}

\end{fulllineitems}

\index{get\_camera\_projection() (deepdrr.device.Device method)@\spxentry{get\_camera\_projection()}\spxextra{deepdrr.device.Device method}}

\begin{fulllineitems}
\phantomsection\label{\detokenize{deepdrr.device:deepdrr.device.Device.get_camera_projection}}
\pysigstartsignatures
\pysiglinewithargsret{\sphinxbfcode{\sphinxupquote{get\_camera\_projection}}}{}{{ $\rightarrow$ {\hyperref[\detokenize{deepdrr.geo:deepdrr.geo.core.CameraProjection}]{\sphinxcrossref{CameraProjection}}}}}
\pysigstopsignatures
\sphinxAtStartPar
Get the camera projection for the device in the current pose.
\begin{quote}\begin{description}
\sphinxlineitem{Returns}
\sphinxAtStartPar
the “index\_from\_world” camera projection for the device.

\sphinxlineitem{Return type}
\sphinxAtStartPar
{\hyperref[\detokenize{deepdrr.geo:deepdrr.geo.core.CameraProjection}]{\sphinxcrossref{CameraProjection}}}

\end{description}\end{quote}

\end{fulllineitems}

\index{get\_mesh\_in\_world() (deepdrr.device.Device method)@\spxentry{get\_mesh\_in\_world()}\spxextra{deepdrr.device.Device method}}

\begin{fulllineitems}
\phantomsection\label{\detokenize{deepdrr.device:deepdrr.device.Device.get_mesh_in_world}}
\pysigstartsignatures
\pysiglinewithargsret{\sphinxbfcode{\sphinxupquote{get\_mesh\_in\_world}}}{\sphinxparam{\DUrole{n,n}{full}\DUrole{o,o}{=}\DUrole{default_value}{False}}\sphinxparamcomma \sphinxparam{\DUrole{n,n}{use\_cached}\DUrole{o,o}{=}\DUrole{default_value}{True}}}{}
\pysigstopsignatures
\sphinxAtStartPar
Get a really simple camera mesh for the device in the current pose.

\sphinxAtStartPar
Subclasses may want to override this with more detailed meshes (full=True).

\end{fulllineitems}

\index{index\_from\_camera3d (deepdrr.device.Device property)@\spxentry{index\_from\_camera3d}\spxextra{deepdrr.device.Device property}}

\begin{fulllineitems}
\phantomsection\label{\detokenize{deepdrr.device:deepdrr.device.Device.index_from_camera3d}}
\pysigstartsignatures
\pysigline{\sphinxbfcode{\sphinxupquote{property\DUrole{w,w}{  }}}\sphinxbfcode{\sphinxupquote{index\_from\_camera3d}}\sphinxbfcode{\sphinxupquote{\DUrole{p,p}{:}\DUrole{w,w}{  }{\hyperref[\detokenize{deepdrr.geo:deepdrr.geo.core.CameraProjection}]{\sphinxcrossref{CameraProjection}}}}}}
\pysigstopsignatures
\sphinxAtStartPar
Get the CameraIntrinsicTransform for the device’s camera3d\_from\_index frame (in the current pose).
\begin{quote}\begin{description}
\sphinxlineitem{Returns}
\sphinxAtStartPar
the “index\_from\_camera3d” frame transformation for the device.

\sphinxlineitem{Return type}
\sphinxAtStartPar
{\hyperref[\detokenize{deepdrr.geo:deepdrr.geo.core.CameraIntrinsicTransform}]{\sphinxcrossref{CameraIntrinsicTransform}}}

\end{description}\end{quote}

\end{fulllineitems}

\index{index\_from\_world (deepdrr.device.Device property)@\spxentry{index\_from\_world}\spxextra{deepdrr.device.Device property}}

\begin{fulllineitems}
\phantomsection\label{\detokenize{deepdrr.device:deepdrr.device.Device.index_from_world}}
\pysigstartsignatures
\pysigline{\sphinxbfcode{\sphinxupquote{property\DUrole{w,w}{  }}}\sphinxbfcode{\sphinxupquote{index\_from\_world}}\sphinxbfcode{\sphinxupquote{\DUrole{p,p}{:}\DUrole{w,w}{  }{\hyperref[\detokenize{deepdrr.geo:deepdrr.geo.core.CameraProjection}]{\sphinxcrossref{CameraProjection}}}}}}
\pysigstopsignatures
\sphinxAtStartPar
Get the camera projection for the device in the current pose.
\begin{quote}\begin{description}
\sphinxlineitem{Returns}
\sphinxAtStartPar
the “index\_from\_world” camera projection for the device.

\sphinxlineitem{Return type}
\sphinxAtStartPar
{\hyperref[\detokenize{deepdrr.geo:deepdrr.geo.core.CameraProjection}]{\sphinxcrossref{CameraProjection}}}

\end{description}\end{quote}

\end{fulllineitems}

\index{pixel\_size (deepdrr.device.Device attribute)@\spxentry{pixel\_size}\spxextra{deepdrr.device.Device attribute}}

\begin{fulllineitems}
\phantomsection\label{\detokenize{deepdrr.device:id8}}
\pysigstartsignatures
\pysigline{\sphinxbfcode{\sphinxupquote{pixel\_size}}\sphinxbfcode{\sphinxupquote{\DUrole{p,p}{:}\DUrole{w,w}{  }float}}}
\pysigstopsignatures
\end{fulllineitems}

\index{principle\_ray (deepdrr.device.Device property)@\spxentry{principle\_ray}\spxextra{deepdrr.device.Device property}}

\begin{fulllineitems}
\phantomsection\label{\detokenize{deepdrr.device:deepdrr.device.Device.principle_ray}}
\pysigstartsignatures
\pysigline{\sphinxbfcode{\sphinxupquote{property\DUrole{w,w}{  }}}\sphinxbfcode{\sphinxupquote{principle\_ray}}\sphinxbfcode{\sphinxupquote{\DUrole{p,p}{:}\DUrole{w,w}{  }{\hyperref[\detokenize{deepdrr.geo:deepdrr.geo.core.Vector3D}]{\sphinxcrossref{Vector3D}}}}}}
\pysigstopsignatures
\sphinxAtStartPar
Get the principle ray for the device in the current pose in the device frame.

\sphinxAtStartPar
The principle ray is the direction of the ray that passes through the center of the
image. It points from the source toward the detector.

\sphinxAtStartPar
By default, this is just the z axis, but this can be overridden by sub classes.
\begin{quote}\begin{description}
\sphinxlineitem{Returns}
\sphinxAtStartPar
the principle ray for the device as a unit vector.

\sphinxlineitem{Return type}
\sphinxAtStartPar
{\hyperref[\detokenize{deepdrr.geo:deepdrr.geo.core.Vector3D}]{\sphinxcrossref{Vector3D}}}

\end{description}\end{quote}

\end{fulllineitems}

\index{principle\_ray\_in\_world (deepdrr.device.Device property)@\spxentry{principle\_ray\_in\_world}\spxextra{deepdrr.device.Device property}}

\begin{fulllineitems}
\phantomsection\label{\detokenize{deepdrr.device:deepdrr.device.Device.principle_ray_in_world}}
\pysigstartsignatures
\pysigline{\sphinxbfcode{\sphinxupquote{property\DUrole{w,w}{  }}}\sphinxbfcode{\sphinxupquote{principle\_ray\_in\_world}}\sphinxbfcode{\sphinxupquote{\DUrole{p,p}{:}\DUrole{w,w}{  }{\hyperref[\detokenize{deepdrr.geo:deepdrr.geo.core.Vector3D}]{\sphinxcrossref{Vector3D}}}}}}
\pysigstopsignatures
\sphinxAtStartPar
Get the principle ray for the device in the current pose in the world frame.

\sphinxAtStartPar
The principle ray is the direction of the ray that passes through the center of the
image. It points from the source toward the detector.
\begin{quote}\begin{description}
\sphinxlineitem{Returns}
\sphinxAtStartPar
the principle ray for the device as a unit vector.

\sphinxlineitem{Return type}
\sphinxAtStartPar
{\hyperref[\detokenize{deepdrr.geo:deepdrr.geo.core.Vector3D}]{\sphinxcrossref{Vector3D}}}

\end{description}\end{quote}

\end{fulllineitems}

\index{sensor\_height (deepdrr.device.Device attribute)@\spxentry{sensor\_height}\spxextra{deepdrr.device.Device attribute}}

\begin{fulllineitems}
\phantomsection\label{\detokenize{deepdrr.device:id9}}
\pysigstartsignatures
\pysigline{\sphinxbfcode{\sphinxupquote{sensor\_height}}\sphinxbfcode{\sphinxupquote{\DUrole{p,p}{:}\DUrole{w,w}{  }int}}}
\pysigstopsignatures
\end{fulllineitems}

\index{sensor\_width (deepdrr.device.Device attribute)@\spxentry{sensor\_width}\spxextra{deepdrr.device.Device attribute}}

\begin{fulllineitems}
\phantomsection\label{\detokenize{deepdrr.device:id10}}
\pysigstartsignatures
\pysigline{\sphinxbfcode{\sphinxupquote{sensor\_width}}\sphinxbfcode{\sphinxupquote{\DUrole{p,p}{:}\DUrole{w,w}{  }int}}}
\pysigstopsignatures
\end{fulllineitems}

\index{source\_in\_world (deepdrr.device.Device property)@\spxentry{source\_in\_world}\spxextra{deepdrr.device.Device property}}

\begin{fulllineitems}
\phantomsection\label{\detokenize{deepdrr.device:deepdrr.device.Device.source_in_world}}
\pysigstartsignatures
\pysigline{\sphinxbfcode{\sphinxupquote{property\DUrole{w,w}{  }}}\sphinxbfcode{\sphinxupquote{source\_in\_world}}\sphinxbfcode{\sphinxupquote{\DUrole{p,p}{:}\DUrole{w,w}{  }{\hyperref[\detokenize{deepdrr.geo:deepdrr.geo.core.Point3D}]{\sphinxcrossref{Point3D}}}}}}
\pysigstopsignatures
\end{fulllineitems}

\index{source\_to\_detector\_distance (deepdrr.device.Device attribute)@\spxentry{source\_to\_detector\_distance}\spxextra{deepdrr.device.Device attribute}}

\begin{fulllineitems}
\phantomsection\label{\detokenize{deepdrr.device:deepdrr.device.Device.source_to_detector_distance}}
\pysigstartsignatures
\pysigline{\sphinxbfcode{\sphinxupquote{source\_to\_detector\_distance}}\sphinxbfcode{\sphinxupquote{\DUrole{p,p}{:}\DUrole{w,w}{  }float}}}
\pysigstopsignatures
\end{fulllineitems}

\index{world\_from\_camera3d (deepdrr.device.Device property)@\spxentry{world\_from\_camera3d}\spxextra{deepdrr.device.Device property}}

\begin{fulllineitems}
\phantomsection\label{\detokenize{deepdrr.device:deepdrr.device.Device.world_from_camera3d}}
\pysigstartsignatures
\pysigline{\sphinxbfcode{\sphinxupquote{property\DUrole{w,w}{  }}}\sphinxbfcode{\sphinxupquote{world\_from\_camera3d}}\sphinxbfcode{\sphinxupquote{\DUrole{p,p}{:}\DUrole{w,w}{  }{\hyperref[\detokenize{deepdrr.geo:deepdrr.geo.core.FrameTransform}]{\sphinxcrossref{FrameTransform}}}}}}
\pysigstopsignatures
\sphinxAtStartPar
Get the FrameTransform for the device’s camera3d\_from\_world frame (in the current pose).
\begin{quote}\begin{description}
\sphinxlineitem{Returns}
\sphinxAtStartPar
the “world\_from\_camera3d” frame transformation for the device.

\sphinxlineitem{Return type}
\sphinxAtStartPar
{\hyperref[\detokenize{deepdrr.geo:deepdrr.geo.core.FrameTransform}]{\sphinxcrossref{FrameTransform}}}

\end{description}\end{quote}

\end{fulllineitems}

\index{world\_from\_device (deepdrr.device.Device attribute)@\spxentry{world\_from\_device}\spxextra{deepdrr.device.Device attribute}}

\begin{fulllineitems}
\phantomsection\label{\detokenize{deepdrr.device:deepdrr.device.Device.world_from_device}}
\pysigstartsignatures
\pysigline{\sphinxbfcode{\sphinxupquote{world\_from\_device}}\sphinxbfcode{\sphinxupquote{\DUrole{p,p}{:}\DUrole{w,w}{  }{\hyperref[\detokenize{deepdrr.geo:deepdrr.geo.core.FrameTransform}]{\sphinxcrossref{FrameTransform}}}}}}
\pysigstopsignatures
\end{fulllineitems}

\index{world\_from\_index (deepdrr.device.Device property)@\spxentry{world\_from\_index}\spxextra{deepdrr.device.Device property}}

\begin{fulllineitems}
\phantomsection\label{\detokenize{deepdrr.device:deepdrr.device.Device.world_from_index}}
\pysigstartsignatures
\pysigline{\sphinxbfcode{\sphinxupquote{property\DUrole{w,w}{  }}}\sphinxbfcode{\sphinxupquote{world\_from\_index}}\sphinxbfcode{\sphinxupquote{\DUrole{p,p}{:}\DUrole{w,w}{  }{\hyperref[\detokenize{deepdrr.geo:deepdrr.geo.core.Transform}]{\sphinxcrossref{Transform}}}}}}
\pysigstopsignatures
\sphinxAtStartPar
Get the world\_from\_index transform for the device in the current pose.
\begin{quote}\begin{description}
\sphinxlineitem{Returns}
\sphinxAtStartPar
the “world\_from\_index” transform for the device.

\sphinxlineitem{Return type}
\sphinxAtStartPar
{\hyperref[\detokenize{deepdrr.geo:deepdrr.geo.core.Transform}]{\sphinxcrossref{Transform}}}

\end{description}\end{quote}

\end{fulllineitems}


\end{fulllineitems}

\index{MobileCArm (class in deepdrr.device)@\spxentry{MobileCArm}\spxextra{class in deepdrr.device}}

\begin{fulllineitems}
\phantomsection\label{\detokenize{deepdrr.device:deepdrr.device.MobileCArm}}
\pysigstartsignatures
\pysiglinewithargsret{\sphinxbfcode{\sphinxupquote{class\DUrole{w,w}{  }}}\sphinxcode{\sphinxupquote{deepdrr.device.}}\sphinxbfcode{\sphinxupquote{MobileCArm}}}{\sphinxparam{\DUrole{n,n}{world\_from\_device}\DUrole{p,p}{:}\DUrole{w,w}{  }\DUrole{n,n}{{\hyperref[\detokenize{deepdrr.geo:deepdrr.geo.core.FrameTransform}]{\sphinxcrossref{FrameTransform}}}\DUrole{w,w}{  }\DUrole{p,p}{|}\DUrole{w,w}{  }None}\DUrole{w,w}{  }\DUrole{o,o}{=}\DUrole{w,w}{  }\DUrole{default_value}{None}}\sphinxparamcomma \sphinxparam{\DUrole{n,n}{isocenter}\DUrole{p,p}{:}\DUrole{w,w}{  }\DUrole{n,n}{{\hyperref[\detokenize{deepdrr.geo:deepdrr.geo.core.Point3D}]{\sphinxcrossref{Point3D}}}}\DUrole{w,w}{  }\DUrole{o,o}{=}\DUrole{w,w}{  }\DUrole{default_value}{{[}0, 0, 0{]}}}\sphinxparamcomma \sphinxparam{\DUrole{n,n}{alpha}\DUrole{p,p}{:}\DUrole{w,w}{  }\DUrole{n,n}{float}\DUrole{w,w}{  }\DUrole{o,o}{=}\DUrole{w,w}{  }\DUrole{default_value}{0}}\sphinxparamcomma \sphinxparam{\DUrole{n,n}{beta}\DUrole{p,p}{:}\DUrole{w,w}{  }\DUrole{n,n}{float}\DUrole{w,w}{  }\DUrole{o,o}{=}\DUrole{w,w}{  }\DUrole{default_value}{0}}\sphinxparamcomma \sphinxparam{\DUrole{n,n}{gamma}\DUrole{p,p}{:}\DUrole{w,w}{  }\DUrole{n,n}{float}\DUrole{w,w}{  }\DUrole{o,o}{=}\DUrole{w,w}{  }\DUrole{default_value}{0}}\sphinxparamcomma \sphinxparam{\DUrole{n,n}{degrees}\DUrole{p,p}{:}\DUrole{w,w}{  }\DUrole{n,n}{bool}\DUrole{w,w}{  }\DUrole{o,o}{=}\DUrole{w,w}{  }\DUrole{default_value}{True}}\sphinxparamcomma \sphinxparam{\DUrole{n,n}{horizontal\_movement}\DUrole{p,p}{:}\DUrole{w,w}{  }\DUrole{n,n}{float}\DUrole{w,w}{  }\DUrole{o,o}{=}\DUrole{w,w}{  }\DUrole{default_value}{200}}\sphinxparamcomma \sphinxparam{\DUrole{n,n}{vertical\_travel}\DUrole{p,p}{:}\DUrole{w,w}{  }\DUrole{n,n}{float}\DUrole{w,w}{  }\DUrole{o,o}{=}\DUrole{w,w}{  }\DUrole{default_value}{430}}\sphinxparamcomma \sphinxparam{\DUrole{n,n}{min\_alpha}\DUrole{p,p}{:}\DUrole{w,w}{  }\DUrole{n,n}{float}\DUrole{w,w}{  }\DUrole{o,o}{=}\DUrole{w,w}{  }\DUrole{default_value}{\sphinxhyphen{}40}}\sphinxparamcomma \sphinxparam{\DUrole{n,n}{max\_alpha}\DUrole{p,p}{:}\DUrole{w,w}{  }\DUrole{n,n}{float}\DUrole{w,w}{  }\DUrole{o,o}{=}\DUrole{w,w}{  }\DUrole{default_value}{110}}\sphinxparamcomma \sphinxparam{\DUrole{n,n}{min\_beta}\DUrole{p,p}{:}\DUrole{w,w}{  }\DUrole{n,n}{float}\DUrole{w,w}{  }\DUrole{o,o}{=}\DUrole{w,w}{  }\DUrole{default_value}{\sphinxhyphen{}225}}\sphinxparamcomma \sphinxparam{\DUrole{n,n}{max\_beta}\DUrole{p,p}{:}\DUrole{w,w}{  }\DUrole{n,n}{float}\DUrole{w,w}{  }\DUrole{o,o}{=}\DUrole{w,w}{  }\DUrole{default_value}{225}}\sphinxparamcomma \sphinxparam{\DUrole{n,n}{source\_to\_detector\_distance}\DUrole{p,p}{:}\DUrole{w,w}{  }\DUrole{n,n}{float}\DUrole{w,w}{  }\DUrole{o,o}{=}\DUrole{w,w}{  }\DUrole{default_value}{1020}}\sphinxparamcomma \sphinxparam{\DUrole{n,n}{source\_to\_isocenter\_vertical\_distance}\DUrole{p,p}{:}\DUrole{w,w}{  }\DUrole{n,n}{float}\DUrole{w,w}{  }\DUrole{o,o}{=}\DUrole{w,w}{  }\DUrole{default_value}{530}}\sphinxparamcomma \sphinxparam{\DUrole{n,n}{source\_to\_isocenter\_horizontal\_offset}\DUrole{p,p}{:}\DUrole{w,w}{  }\DUrole{n,n}{float}\DUrole{w,w}{  }\DUrole{o,o}{=}\DUrole{w,w}{  }\DUrole{default_value}{0}}\sphinxparamcomma \sphinxparam{\DUrole{n,n}{immersion\_depth}\DUrole{p,p}{:}\DUrole{w,w}{  }\DUrole{n,n}{float}\DUrole{w,w}{  }\DUrole{o,o}{=}\DUrole{w,w}{  }\DUrole{default_value}{730}}\sphinxparamcomma \sphinxparam{\DUrole{n,n}{free\_space}\DUrole{p,p}{:}\DUrole{w,w}{  }\DUrole{n,n}{float}\DUrole{w,w}{  }\DUrole{o,o}{=}\DUrole{w,w}{  }\DUrole{default_value}{820}}\sphinxparamcomma \sphinxparam{\DUrole{n,n}{sensor\_height}\DUrole{p,p}{:}\DUrole{w,w}{  }\DUrole{n,n}{int}\DUrole{w,w}{  }\DUrole{o,o}{=}\DUrole{w,w}{  }\DUrole{default_value}{1536}}\sphinxparamcomma \sphinxparam{\DUrole{n,n}{sensor\_width}\DUrole{p,p}{:}\DUrole{w,w}{  }\DUrole{n,n}{int}\DUrole{w,w}{  }\DUrole{o,o}{=}\DUrole{w,w}{  }\DUrole{default_value}{1536}}\sphinxparamcomma \sphinxparam{\DUrole{n,n}{pixel\_size}\DUrole{p,p}{:}\DUrole{w,w}{  }\DUrole{n,n}{float}\DUrole{w,w}{  }\DUrole{o,o}{=}\DUrole{w,w}{  }\DUrole{default_value}{0.194}}\sphinxparamcomma \sphinxparam{\DUrole{n,n}{rotate\_camera\_left}\DUrole{p,p}{:}\DUrole{w,w}{  }\DUrole{n,n}{bool}\DUrole{w,w}{  }\DUrole{o,o}{=}\DUrole{w,w}{  }\DUrole{default_value}{True}}\sphinxparamcomma \sphinxparam{\DUrole{n,n}{enforce\_isocenter\_bounds}\DUrole{p,p}{:}\DUrole{w,w}{  }\DUrole{n,n}{bool}\DUrole{w,w}{  }\DUrole{o,o}{=}\DUrole{w,w}{  }\DUrole{default_value}{False}}}{}
\pysigstopsignatures
\sphinxAtStartPar
Bases: {\hyperref[\detokenize{deepdrr.device:deepdrr.device.device.Device}]{\sphinxcrossref{\sphinxcode{\sphinxupquote{Device}}}}}

\sphinxAtStartPar
A C\sphinxhyphen{}arm imaging device with orbital movement (alpha, beta) and isocenter movement (x, y, z).

\sphinxAtStartPar
Default parameters are based on the Siemens CIOS Spin.
\index{alpha (deepdrr.device.MobileCArm attribute)@\spxentry{alpha}\spxextra{deepdrr.device.MobileCArm attribute}}

\begin{fulllineitems}
\phantomsection\label{\detokenize{deepdrr.device:deepdrr.device.MobileCArm.alpha}}
\pysigstartsignatures
\pysigline{\sphinxbfcode{\sphinxupquote{alpha}}\sphinxbfcode{\sphinxupquote{\DUrole{p,p}{:}\DUrole{w,w}{  }float}}}
\pysigstopsignatures
\end{fulllineitems}

\index{arm\_from\_device (deepdrr.device.MobileCArm property)@\spxentry{arm\_from\_device}\spxextra{deepdrr.device.MobileCArm property}}

\begin{fulllineitems}
\phantomsection\label{\detokenize{deepdrr.device:deepdrr.device.MobileCArm.arm_from_device}}
\pysigstartsignatures
\pysigline{\sphinxbfcode{\sphinxupquote{property\DUrole{w,w}{  }}}\sphinxbfcode{\sphinxupquote{arm\_from\_device}}\sphinxbfcode{\sphinxupquote{\DUrole{p,p}{:}\DUrole{w,w}{  }{\hyperref[\detokenize{deepdrr.geo:deepdrr.geo.core.FrameTransform}]{\sphinxcrossref{FrameTransform}}}}}}
\pysigstopsignatures
\sphinxAtStartPar
Transformation from the device frame (which doesn’t move) to the arm frame (which rotates and translates with the arm, origin at the isocenter).

\end{fulllineitems}

\index{arm\_width (deepdrr.device.MobileCArm attribute)@\spxentry{arm\_width}\spxextra{deepdrr.device.MobileCArm attribute}}

\begin{fulllineitems}
\phantomsection\label{\detokenize{deepdrr.device:deepdrr.device.MobileCArm.arm_width}}
\pysigstartsignatures
\pysigline{\sphinxbfcode{\sphinxupquote{arm\_width}}\sphinxbfcode{\sphinxupquote{\DUrole{w,w}{  }\DUrole{p,p}{=}\DUrole{w,w}{  }100}}}
\pysigstopsignatures
\end{fulllineitems}

\index{beta (deepdrr.device.MobileCArm attribute)@\spxentry{beta}\spxextra{deepdrr.device.MobileCArm attribute}}

\begin{fulllineitems}
\phantomsection\label{\detokenize{deepdrr.device:deepdrr.device.MobileCArm.beta}}
\pysigstartsignatures
\pysigline{\sphinxbfcode{\sphinxupquote{beta}}\sphinxbfcode{\sphinxupquote{\DUrole{p,p}{:}\DUrole{w,w}{  }float}}}
\pysigstopsignatures
\end{fulllineitems}

\index{camera3d\_from\_device (deepdrr.device.MobileCArm property)@\spxentry{camera3d\_from\_device}\spxextra{deepdrr.device.MobileCArm property}}

\begin{fulllineitems}
\phantomsection\label{\detokenize{deepdrr.device:deepdrr.device.MobileCArm.camera3d_from_device}}
\pysigstartsignatures
\pysigline{\sphinxbfcode{\sphinxupquote{property\DUrole{w,w}{  }}}\sphinxbfcode{\sphinxupquote{camera3d\_from\_device}}\sphinxbfcode{\sphinxupquote{\DUrole{p,p}{:}\DUrole{w,w}{  }{\hyperref[\detokenize{deepdrr.geo:deepdrr.geo.core.FrameTransform}]{\sphinxcrossref{FrameTransform}}}}}}
\pysigstopsignatures
\sphinxAtStartPar
Get the camera3d frame from device coordinates

\sphinxAtStartPar
The Z axis points from the source to the detector.

\end{fulllineitems}

\index{camera3d\_from\_world (deepdrr.device.MobileCArm property)@\spxentry{camera3d\_from\_world}\spxextra{deepdrr.device.MobileCArm property}}

\begin{fulllineitems}
\phantomsection\label{\detokenize{deepdrr.device:deepdrr.device.MobileCArm.camera3d_from_world}}
\pysigstartsignatures
\pysigline{\sphinxbfcode{\sphinxupquote{property\DUrole{w,w}{  }}}\sphinxbfcode{\sphinxupquote{camera3d\_from\_world}}\sphinxbfcode{\sphinxupquote{\DUrole{p,p}{:}\DUrole{w,w}{  }{\hyperref[\detokenize{deepdrr.geo:deepdrr.geo.core.FrameTransform}]{\sphinxcrossref{FrameTransform}}}}}}
\pysigstopsignatures
\sphinxAtStartPar
Rigid transformation of the C\sphinxhyphen{}arm camera pose.

\end{fulllineitems}

\index{camera\_intrinsics (deepdrr.device.MobileCArm attribute)@\spxentry{camera\_intrinsics}\spxextra{deepdrr.device.MobileCArm attribute}}

\begin{fulllineitems}
\phantomsection\label{\detokenize{deepdrr.device:deepdrr.device.MobileCArm.camera_intrinsics}}
\pysigstartsignatures
\pysigline{\sphinxbfcode{\sphinxupquote{camera\_intrinsics}}\sphinxbfcode{\sphinxupquote{\DUrole{p,p}{:}\DUrole{w,w}{  }{\hyperref[\detokenize{deepdrr.geo:deepdrr.geo.core.CameraIntrinsicTransform}]{\sphinxcrossref{CameraIntrinsicTransform}}}}}}
\pysigstopsignatures
\end{fulllineitems}

\index{detector\_height (deepdrr.device.MobileCArm attribute)@\spxentry{detector\_height}\spxextra{deepdrr.device.MobileCArm attribute}}

\begin{fulllineitems}
\phantomsection\label{\detokenize{deepdrr.device:deepdrr.device.MobileCArm.detector_height}}
\pysigstartsignatures
\pysigline{\sphinxbfcode{\sphinxupquote{detector\_height}}\sphinxbfcode{\sphinxupquote{\DUrole{w,w}{  }\DUrole{p,p}{=}\DUrole{w,w}{  }100}}}
\pysigstopsignatures
\end{fulllineitems}

\index{device\_from\_arm (deepdrr.device.MobileCArm property)@\spxentry{device\_from\_arm}\spxextra{deepdrr.device.MobileCArm property}}

\begin{fulllineitems}
\phantomsection\label{\detokenize{deepdrr.device:deepdrr.device.MobileCArm.device_from_arm}}
\pysigstartsignatures
\pysigline{\sphinxbfcode{\sphinxupquote{property\DUrole{w,w}{  }}}\sphinxbfcode{\sphinxupquote{device\_from\_arm}}\sphinxbfcode{\sphinxupquote{\DUrole{p,p}{:}\DUrole{w,w}{  }{\hyperref[\detokenize{deepdrr.geo:deepdrr.geo.core.FrameTransform}]{\sphinxcrossref{FrameTransform}}}}}}
\pysigstopsignatures
\end{fulllineitems}

\index{device\_from\_camera3d (deepdrr.device.MobileCArm property)@\spxentry{device\_from\_camera3d}\spxextra{deepdrr.device.MobileCArm property}}

\begin{fulllineitems}
\phantomsection\label{\detokenize{deepdrr.device:deepdrr.device.MobileCArm.device_from_camera3d}}
\pysigstartsignatures
\pysigline{\sphinxbfcode{\sphinxupquote{property\DUrole{w,w}{  }}}\sphinxbfcode{\sphinxupquote{device\_from\_camera3d}}\sphinxbfcode{\sphinxupquote{\DUrole{p,p}{:}\DUrole{w,w}{  }{\hyperref[\detokenize{deepdrr.geo:deepdrr.geo.core.FrameTransform}]{\sphinxcrossref{FrameTransform}}}}}}
\pysigstopsignatures
\sphinxAtStartPar
Get the FrameTransform for the device’s camera3d\_from\_device frame (in the current pose).
\begin{quote}\begin{description}
\sphinxlineitem{Parameters}
\sphinxAtStartPar
\sphinxstyleliteralstrong{\sphinxupquote{camera3d\_transform}} ({\hyperref[\detokenize{deepdrr.geo:deepdrr.geo.core.FrameTransform}]{\sphinxcrossref{\sphinxstyleliteralemphasis{\sphinxupquote{FrameTransform}}}}}) \textendash{} the “camera3d\_from\_device” frame transformation for the device.

\sphinxlineitem{Returns}
\sphinxAtStartPar
the “device\_from\_camera3d” frame transformation for the device.

\sphinxlineitem{Return type}
\sphinxAtStartPar
{\hyperref[\detokenize{deepdrr.geo:deepdrr.geo.core.FrameTransform}]{\sphinxcrossref{FrameTransform}}}

\end{description}\end{quote}

\end{fulllineitems}

\index{get\_camera3d\_from\_world() (deepdrr.device.MobileCArm method)@\spxentry{get\_camera3d\_from\_world()}\spxextra{deepdrr.device.MobileCArm method}}

\begin{fulllineitems}
\phantomsection\label{\detokenize{deepdrr.device:deepdrr.device.MobileCArm.get_camera3d_from_world}}
\pysigstartsignatures
\pysiglinewithargsret{\sphinxbfcode{\sphinxupquote{get\_camera3d\_from\_world}}}{}{{ $\rightarrow$ {\hyperref[\detokenize{deepdrr.geo:deepdrr.geo.core.FrameTransform}]{\sphinxcrossref{FrameTransform}}}}}
\pysigstopsignatures
\end{fulllineitems}

\index{get\_camera\_projection() (deepdrr.device.MobileCArm method)@\spxentry{get\_camera\_projection()}\spxextra{deepdrr.device.MobileCArm method}}

\begin{fulllineitems}
\phantomsection\label{\detokenize{deepdrr.device:deepdrr.device.MobileCArm.get_camera_projection}}
\pysigstartsignatures
\pysiglinewithargsret{\sphinxbfcode{\sphinxupquote{get\_camera\_projection}}}{}{{ $\rightarrow$ {\hyperref[\detokenize{deepdrr.geo:deepdrr.geo.core.CameraProjection}]{\sphinxcrossref{CameraProjection}}}}}
\pysigstopsignatures
\sphinxAtStartPar
Get the camera projection for the device in the current pose.
\begin{quote}\begin{description}
\sphinxlineitem{Returns}
\sphinxAtStartPar
the “index\_from\_world” camera projection for the device.

\sphinxlineitem{Return type}
\sphinxAtStartPar
{\hyperref[\detokenize{deepdrr.geo:deepdrr.geo.core.CameraProjection}]{\sphinxcrossref{CameraProjection}}}

\end{description}\end{quote}

\end{fulllineitems}

\index{get\_mesh\_in\_world() (deepdrr.device.MobileCArm method)@\spxentry{get\_mesh\_in\_world()}\spxextra{deepdrr.device.MobileCArm method}}

\begin{fulllineitems}
\phantomsection\label{\detokenize{deepdrr.device:deepdrr.device.MobileCArm.get_mesh_in_world}}
\pysigstartsignatures
\pysiglinewithargsret{\sphinxbfcode{\sphinxupquote{get\_mesh\_in\_world}}}{\sphinxparam{\DUrole{n,n}{full}\DUrole{o,o}{=}\DUrole{default_value}{False}}\sphinxparamcomma \sphinxparam{\DUrole{n,n}{use\_cached}\DUrole{o,o}{=}\DUrole{default_value}{True}}}{}
\pysigstopsignatures
\sphinxAtStartPar
Get the pyvista mesh for the C\sphinxhyphen{}arm, in its world\sphinxhyphen{}space orientation.
\begin{quote}\begin{description}
\sphinxlineitem{Raises}
\sphinxAtStartPar
\sphinxstyleliteralstrong{\sphinxupquote{RuntimeError}} \textendash{} if pyvista is not available.

\end{description}\end{quote}

\end{fulllineitems}

\index{isocenter (deepdrr.device.MobileCArm attribute)@\spxentry{isocenter}\spxextra{deepdrr.device.MobileCArm attribute}}

\begin{fulllineitems}
\phantomsection\label{\detokenize{deepdrr.device:deepdrr.device.MobileCArm.isocenter}}
\pysigstartsignatures
\pysigline{\sphinxbfcode{\sphinxupquote{isocenter}}\sphinxbfcode{\sphinxupquote{\DUrole{p,p}{:}\DUrole{w,w}{  }{\hyperref[\detokenize{deepdrr.geo:deepdrr.geo.core.Point3D}]{\sphinxcrossref{Point3D}}}}}}
\pysigstopsignatures
\end{fulllineitems}

\index{isocenter\_in\_world (deepdrr.device.MobileCArm property)@\spxentry{isocenter\_in\_world}\spxextra{deepdrr.device.MobileCArm property}}

\begin{fulllineitems}
\phantomsection\label{\detokenize{deepdrr.device:deepdrr.device.MobileCArm.isocenter_in_world}}
\pysigstartsignatures
\pysigline{\sphinxbfcode{\sphinxupquote{property\DUrole{w,w}{  }}}\sphinxbfcode{\sphinxupquote{isocenter\_in\_world}}\sphinxbfcode{\sphinxupquote{\DUrole{p,p}{:}\DUrole{w,w}{  }{\hyperref[\detokenize{deepdrr.geo:deepdrr.geo.core.Point3D}]{\sphinxcrossref{Point3D}}}}}}
\pysigstopsignatures
\end{fulllineitems}

\index{jitter() (deepdrr.device.MobileCArm method)@\spxentry{jitter()}\spxextra{deepdrr.device.MobileCArm method}}

\begin{fulllineitems}
\phantomsection\label{\detokenize{deepdrr.device:deepdrr.device.MobileCArm.jitter}}
\pysigstartsignatures
\pysiglinewithargsret{\sphinxbfcode{\sphinxupquote{jitter}}}{}{}
\pysigstopsignatures
\end{fulllineitems}

\index{max\_isocenter (deepdrr.device.MobileCArm property)@\spxentry{max\_isocenter}\spxextra{deepdrr.device.MobileCArm property}}

\begin{fulllineitems}
\phantomsection\label{\detokenize{deepdrr.device:deepdrr.device.MobileCArm.max_isocenter}}
\pysigstartsignatures
\pysigline{\sphinxbfcode{\sphinxupquote{property\DUrole{w,w}{  }}}\sphinxbfcode{\sphinxupquote{max\_isocenter}}\sphinxbfcode{\sphinxupquote{\DUrole{p,p}{:}\DUrole{w,w}{  }ndarray}}}
\pysigstopsignatures
\end{fulllineitems}

\index{min\_isocenter (deepdrr.device.MobileCArm property)@\spxentry{min\_isocenter}\spxextra{deepdrr.device.MobileCArm property}}

\begin{fulllineitems}
\phantomsection\label{\detokenize{deepdrr.device:deepdrr.device.MobileCArm.min_isocenter}}
\pysigstartsignatures
\pysigline{\sphinxbfcode{\sphinxupquote{property\DUrole{w,w}{  }}}\sphinxbfcode{\sphinxupquote{min\_isocenter}}\sphinxbfcode{\sphinxupquote{\DUrole{p,p}{:}\DUrole{w,w}{  }ndarray}}}
\pysigstopsignatures
\end{fulllineitems}

\index{move\_by() (deepdrr.device.MobileCArm method)@\spxentry{move\_by()}\spxextra{deepdrr.device.MobileCArm method}}

\begin{fulllineitems}
\phantomsection\label{\detokenize{deepdrr.device:deepdrr.device.MobileCArm.move_by}}
\pysigstartsignatures
\pysiglinewithargsret{\sphinxbfcode{\sphinxupquote{move\_by}}}{\sphinxparam{\DUrole{n,n}{delta\_isocenter}\DUrole{p,p}{:}\DUrole{w,w}{  }\DUrole{n,n}{{\hyperref[\detokenize{deepdrr.geo:deepdrr.geo.core.Vector3D}]{\sphinxcrossref{Vector3D}}}\DUrole{w,w}{  }\DUrole{p,p}{|}\DUrole{w,w}{  }None}\DUrole{w,w}{  }\DUrole{o,o}{=}\DUrole{w,w}{  }\DUrole{default_value}{None}}\sphinxparamcomma \sphinxparam{\DUrole{n,n}{delta\_alpha}\DUrole{p,p}{:}\DUrole{w,w}{  }\DUrole{n,n}{float\DUrole{w,w}{  }\DUrole{p,p}{|}\DUrole{w,w}{  }None}\DUrole{w,w}{  }\DUrole{o,o}{=}\DUrole{w,w}{  }\DUrole{default_value}{None}}\sphinxparamcomma \sphinxparam{\DUrole{n,n}{delta\_beta}\DUrole{p,p}{:}\DUrole{w,w}{  }\DUrole{n,n}{float\DUrole{w,w}{  }\DUrole{p,p}{|}\DUrole{w,w}{  }None}\DUrole{w,w}{  }\DUrole{o,o}{=}\DUrole{w,w}{  }\DUrole{default_value}{None}}\sphinxparamcomma \sphinxparam{\DUrole{n,n}{delta\_gamma}\DUrole{p,p}{:}\DUrole{w,w}{  }\DUrole{n,n}{float\DUrole{w,w}{  }\DUrole{p,p}{|}\DUrole{w,w}{  }None}\DUrole{w,w}{  }\DUrole{o,o}{=}\DUrole{w,w}{  }\DUrole{default_value}{None}}\sphinxparamcomma \sphinxparam{\DUrole{n,n}{degrees}\DUrole{p,p}{:}\DUrole{w,w}{  }\DUrole{n,n}{bool}\DUrole{w,w}{  }\DUrole{o,o}{=}\DUrole{w,w}{  }\DUrole{default_value}{True}}}{{ $\rightarrow$ None}}
\pysigstopsignatures
\sphinxAtStartPar
Move the C\sphinxhyphen{}arm to the specified pose.
\begin{quote}\begin{description}
\sphinxlineitem{Parameters}\begin{itemize}
\item {} 
\sphinxAtStartPar
\sphinxstyleliteralstrong{\sphinxupquote{delta\_isocenter}} (\sphinxstyleliteralemphasis{\sphinxupquote{Optional}}\sphinxstyleliteralemphasis{\sphinxupquote{{[}}}{\hyperref[\detokenize{deepdrr.geo:deepdrr.geo.Vector3D}]{\sphinxcrossref{\sphinxstyleliteralemphasis{\sphinxupquote{geo.Vector3D}}}}}\sphinxstyleliteralemphasis{\sphinxupquote{{]}}}\sphinxstyleliteralemphasis{\sphinxupquote{, }}\sphinxstyleliteralemphasis{\sphinxupquote{optional}}) \textendash{} change to the isocenter in DEVICE space
(as a vector, this only matters if the scaling/rotation is different).
This is the center about which rotations are performed. Defaults to None.

\item {} 
\sphinxAtStartPar
\sphinxstyleliteralstrong{\sphinxupquote{delta\_alpha}} (\sphinxstyleliteralemphasis{\sphinxupquote{Optional}}\sphinxstyleliteralemphasis{\sphinxupquote{{[}}}\sphinxstyleliteralemphasis{\sphinxupquote{float}}\sphinxstyleliteralemphasis{\sphinxupquote{{]}}}\sphinxstyleliteralemphasis{\sphinxupquote{, }}\sphinxstyleliteralemphasis{\sphinxupquote{optional}}) \textendash{} change in alpha. Defaults to None.

\item {} 
\sphinxAtStartPar
\sphinxstyleliteralstrong{\sphinxupquote{delta\_beta}} (\sphinxstyleliteralemphasis{\sphinxupquote{Optional}}\sphinxstyleliteralemphasis{\sphinxupquote{{[}}}\sphinxstyleliteralemphasis{\sphinxupquote{float}}\sphinxstyleliteralemphasis{\sphinxupquote{{]}}}\sphinxstyleliteralemphasis{\sphinxupquote{, }}\sphinxstyleliteralemphasis{\sphinxupquote{optional}}) \textendash{} change in beta. Defaults to None.

\item {} 
\sphinxAtStartPar
\sphinxstyleliteralstrong{\sphinxupquote{degrees}} (\sphinxstyleliteralemphasis{\sphinxupquote{bool}}\sphinxstyleliteralemphasis{\sphinxupquote{, }}\sphinxstyleliteralemphasis{\sphinxupquote{optional}}) \textendash{} whether the given angles are in degrees. Defaults to False.

\end{itemize}

\end{description}\end{quote}

\end{fulllineitems}

\index{move\_to() (deepdrr.device.MobileCArm method)@\spxentry{move\_to()}\spxextra{deepdrr.device.MobileCArm method}}

\begin{fulllineitems}
\phantomsection\label{\detokenize{deepdrr.device:deepdrr.device.MobileCArm.move_to}}
\pysigstartsignatures
\pysiglinewithargsret{\sphinxbfcode{\sphinxupquote{move\_to}}}{\sphinxparam{\DUrole{n,n}{isocenter}\DUrole{p,p}{:}\DUrole{w,w}{  }\DUrole{n,n}{{\hyperref[\detokenize{deepdrr.geo:deepdrr.geo.core.Point3D}]{\sphinxcrossref{Point3D}}}\DUrole{w,w}{  }\DUrole{p,p}{|}\DUrole{w,w}{  }None}\DUrole{w,w}{  }\DUrole{o,o}{=}\DUrole{w,w}{  }\DUrole{default_value}{None}}\sphinxparamcomma \sphinxparam{\DUrole{n,n}{isocenter\_in\_world}\DUrole{p,p}{:}\DUrole{w,w}{  }\DUrole{n,n}{{\hyperref[\detokenize{deepdrr.geo:deepdrr.geo.core.Point3D}]{\sphinxcrossref{Point3D}}}\DUrole{w,w}{  }\DUrole{p,p}{|}\DUrole{w,w}{  }None}\DUrole{w,w}{  }\DUrole{o,o}{=}\DUrole{w,w}{  }\DUrole{default_value}{None}}\sphinxparamcomma \sphinxparam{\DUrole{n,n}{alpha}\DUrole{p,p}{:}\DUrole{w,w}{  }\DUrole{n,n}{float\DUrole{w,w}{  }\DUrole{p,p}{|}\DUrole{w,w}{  }None}\DUrole{w,w}{  }\DUrole{o,o}{=}\DUrole{w,w}{  }\DUrole{default_value}{None}}\sphinxparamcomma \sphinxparam{\DUrole{n,n}{beta}\DUrole{p,p}{:}\DUrole{w,w}{  }\DUrole{n,n}{float\DUrole{w,w}{  }\DUrole{p,p}{|}\DUrole{w,w}{  }None}\DUrole{w,w}{  }\DUrole{o,o}{=}\DUrole{w,w}{  }\DUrole{default_value}{None}}\sphinxparamcomma \sphinxparam{\DUrole{n,n}{gamma}\DUrole{p,p}{:}\DUrole{w,w}{  }\DUrole{n,n}{float\DUrole{w,w}{  }\DUrole{p,p}{|}\DUrole{w,w}{  }None}\DUrole{w,w}{  }\DUrole{o,o}{=}\DUrole{w,w}{  }\DUrole{default_value}{None}}\sphinxparamcomma \sphinxparam{\DUrole{n,n}{degrees}\DUrole{p,p}{:}\DUrole{w,w}{  }\DUrole{n,n}{bool}\DUrole{w,w}{  }\DUrole{o,o}{=}\DUrole{w,w}{  }\DUrole{default_value}{True}}\sphinxparamcomma \sphinxparam{\DUrole{n,n}{interest\_point\_in\_world}\DUrole{p,p}{:}\DUrole{w,w}{  }\DUrole{n,n}{{\hyperref[\detokenize{deepdrr.geo:deepdrr.geo.core.Point3D}]{\sphinxcrossref{Point3D}}}\DUrole{w,w}{  }\DUrole{p,p}{|}\DUrole{w,w}{  }None}\DUrole{w,w}{  }\DUrole{o,o}{=}\DUrole{w,w}{  }\DUrole{default_value}{None}}\sphinxparamcomma \sphinxparam{\DUrole{n,n}{principle\_ray\_in\_world}\DUrole{p,p}{:}\DUrole{w,w}{  }\DUrole{n,n}{{\hyperref[\detokenize{deepdrr.geo:deepdrr.geo.core.Vector3D}]{\sphinxcrossref{Vector3D}}}\DUrole{w,w}{  }\DUrole{p,p}{|}\DUrole{w,w}{  }None}\DUrole{w,w}{  }\DUrole{o,o}{=}\DUrole{w,w}{  }\DUrole{default_value}{None}}}{{ $\rightarrow$ None}}
\pysigstopsignatures
\sphinxAtStartPar
Move to the specified point.
\begin{quote}\begin{description}
\sphinxlineitem{Parameters}\begin{itemize}
\item {} 
\sphinxAtStartPar
\sphinxstyleliteralstrong{\sphinxupquote{isocenter\_in\_world}} (\sphinxstyleliteralemphasis{\sphinxupquote{Optional}}\sphinxstyleliteralemphasis{\sphinxupquote{{[}}}{\hyperref[\detokenize{deepdrr.geo:deepdrr.geo.Point3D}]{\sphinxcrossref{\sphinxstyleliteralemphasis{\sphinxupquote{geo.Point3D}}}}}\sphinxstyleliteralemphasis{\sphinxupquote{{]}}}\sphinxstyleliteralemphasis{\sphinxupquote{, }}\sphinxstyleliteralemphasis{\sphinxupquote{optional}}) \textendash{} the desired isocenter in world coordinates.
Overrides \sphinxtitleref{isocenter} if provided. Defaults to None.

\item {} 
\sphinxAtStartPar
\sphinxstyleliteralstrong{\sphinxupquote{isocenter}} \textendash{} Desired isocenter in device coordinates.

\item {} 
\sphinxAtStartPar
\sphinxstyleliteralstrong{\sphinxupquote{alpha}} (\sphinxstyleliteralemphasis{\sphinxupquote{Optional}}\sphinxstyleliteralemphasis{\sphinxupquote{{[}}}\sphinxstyleliteralemphasis{\sphinxupquote{float}}\sphinxstyleliteralemphasis{\sphinxupquote{{]}}}\sphinxstyleliteralemphasis{\sphinxupquote{, }}\sphinxstyleliteralemphasis{\sphinxupquote{optional}}) \textendash{} the desired alpha angulation. Defaults to None.

\item {} 
\sphinxAtStartPar
\sphinxstyleliteralstrong{\sphinxupquote{beta}} (\sphinxstyleliteralemphasis{\sphinxupquote{Optional}}\sphinxstyleliteralemphasis{\sphinxupquote{{[}}}\sphinxstyleliteralemphasis{\sphinxupquote{float}}\sphinxstyleliteralemphasis{\sphinxupquote{{]}}}\sphinxstyleliteralemphasis{\sphinxupquote{, }}\sphinxstyleliteralemphasis{\sphinxupquote{optional}}) \textendash{} the desired secondary angulation. Defaults to None.

\item {} 
\sphinxAtStartPar
\sphinxstyleliteralstrong{\sphinxupquote{degrees}} (\sphinxstyleliteralemphasis{\sphinxupquote{bool}}\sphinxstyleliteralemphasis{\sphinxupquote{, }}\sphinxstyleliteralemphasis{\sphinxupquote{optional}}) \textendash{} whether angles are in degrees or radians. Defaults to False.

\item {} 
\sphinxAtStartPar
\sphinxstyleliteralstrong{\sphinxupquote{interest\_point}} ({\hyperref[\detokenize{deepdrr.geo:deepdrr.geo.core.Point3D}]{\sphinxcrossref{\sphinxstyleliteralemphasis{\sphinxupquote{Point3D}}}}}\sphinxstyleliteralemphasis{\sphinxupquote{, }}\sphinxstyleliteralemphasis{\sphinxupquote{optional}}) \textendash{} If this world\sphinxhyphen{}space point is provided, add a translation such that the rotation
maintains the camera\sphinxhyphen{}space position of this point. Overrides \sphinxtitleref{isocenter}. Defaults to None.

\item {} 
\sphinxAtStartPar
\sphinxstyleliteralstrong{\sphinxupquote{principle\_ray\_in\_world}} (\sphinxstyleliteralemphasis{\sphinxupquote{Optional}}\sphinxstyleliteralemphasis{\sphinxupquote{{[}}}{\hyperref[\detokenize{deepdrr.geo:deepdrr.geo.core.Vector3D}]{\sphinxcrossref{\sphinxstyleliteralemphasis{\sphinxupquote{Vector3D}}}}}\sphinxstyleliteralemphasis{\sphinxupquote{{]}}}\sphinxstyleliteralemphasis{\sphinxupquote{, }}\sphinxstyleliteralemphasis{\sphinxupquote{optional}}) \textendash{} If this world\sphinxhyphen{}space vector is provided, override alpha, beta so the C\sphinxhyphen{}arm points along this vector.

\end{itemize}

\end{description}\end{quote}

\end{fulllineitems}

\index{pixel\_size (deepdrr.device.MobileCArm attribute)@\spxentry{pixel\_size}\spxextra{deepdrr.device.MobileCArm attribute}}

\begin{fulllineitems}
\phantomsection\label{\detokenize{deepdrr.device:deepdrr.device.MobileCArm.pixel_size}}
\pysigstartsignatures
\pysigline{\sphinxbfcode{\sphinxupquote{pixel\_size}}\sphinxbfcode{\sphinxupquote{\DUrole{p,p}{:}\DUrole{w,w}{  }float}}}
\pysigstopsignatures
\end{fulllineitems}

\index{principle\_ray (deepdrr.device.MobileCArm property)@\spxentry{principle\_ray}\spxextra{deepdrr.device.MobileCArm property}}

\begin{fulllineitems}
\phantomsection\label{\detokenize{deepdrr.device:deepdrr.device.MobileCArm.principle_ray}}
\pysigstartsignatures
\pysigline{\sphinxbfcode{\sphinxupquote{property\DUrole{w,w}{  }}}\sphinxbfcode{\sphinxupquote{principle\_ray}}\sphinxbfcode{\sphinxupquote{\DUrole{p,p}{:}\DUrole{w,w}{  }{\hyperref[\detokenize{deepdrr.geo:deepdrr.geo.core.Vector3D}]{\sphinxcrossref{Vector3D}}}}}}
\pysigstopsignatures
\sphinxAtStartPar
Unit vector along principle ray.

\end{fulllineitems}

\index{reposition() (deepdrr.device.MobileCArm method)@\spxentry{reposition()}\spxextra{deepdrr.device.MobileCArm method}}

\begin{fulllineitems}
\phantomsection\label{\detokenize{deepdrr.device:deepdrr.device.MobileCArm.reposition}}
\pysigstartsignatures
\pysiglinewithargsret{\sphinxbfcode{\sphinxupquote{reposition}}}{\sphinxparam{\DUrole{n,n}{viewpoint\_in\_world}\DUrole{p,p}{:}\DUrole{w,w}{  }\DUrole{n,n}{{\hyperref[\detokenize{deepdrr.geo:deepdrr.geo.core.Point3D}]{\sphinxcrossref{Point3D}}}\DUrole{w,w}{  }\DUrole{p,p}{|}\DUrole{w,w}{  }None}\DUrole{w,w}{  }\DUrole{o,o}{=}\DUrole{w,w}{  }\DUrole{default_value}{None}}\sphinxparamcomma \sphinxparam{\DUrole{n,n}{device\_in\_world}\DUrole{p,p}{:}\DUrole{w,w}{  }\DUrole{n,n}{{\hyperref[\detokenize{deepdrr.geo:deepdrr.geo.core.Point3D}]{\sphinxcrossref{Point3D}}}\DUrole{w,w}{  }\DUrole{p,p}{|}\DUrole{w,w}{  }None}\DUrole{w,w}{  }\DUrole{o,o}{=}\DUrole{w,w}{  }\DUrole{default_value}{None}}}{{ $\rightarrow$ None}}
\pysigstopsignatures
\sphinxAtStartPar
Reposition the C\sphinxhyphen{}arm by resetting its internal pose to the parameters and adjusting the world\_from\_device transform.

\sphinxAtStartPar
TODO: currently, this eliminates any scaling/rotation of the device in world.

\sphinxAtStartPar
May provide either the isocenter location (device\_in\_world) or viewpoint (viewpoint\_in\_world)
\begin{quote}\begin{description}
\sphinxlineitem{Parameters}\begin{itemize}
\item {} 
\sphinxAtStartPar
\sphinxstyleliteralstrong{\sphinxupquote{viewpoint\_in\_world}} ({\hyperref[\detokenize{deepdrr.geo:deepdrr.geo.Point3D}]{\sphinxcrossref{\sphinxstyleliteralemphasis{\sphinxupquote{geo.Point3D}}}}}) \textendash{} the initial viewpoint the device should have.

\item {} 
\sphinxAtStartPar
\sphinxstyleliteralstrong{\sphinxupquote{(}}\sphinxstyleliteralstrong{\sphinxupquote{)}} (\sphinxstyleliteralemphasis{\sphinxupquote{device\_in\_world}}) \textendash{} initial isocenter.

\end{itemize}

\end{description}\end{quote}

\end{fulllineitems}

\index{sensor\_height (deepdrr.device.MobileCArm attribute)@\spxentry{sensor\_height}\spxextra{deepdrr.device.MobileCArm attribute}}

\begin{fulllineitems}
\phantomsection\label{\detokenize{deepdrr.device:deepdrr.device.MobileCArm.sensor_height}}
\pysigstartsignatures
\pysigline{\sphinxbfcode{\sphinxupquote{sensor\_height}}\sphinxbfcode{\sphinxupquote{\DUrole{p,p}{:}\DUrole{w,w}{  }int}}}
\pysigstopsignatures
\end{fulllineitems}

\index{sensor\_width (deepdrr.device.MobileCArm attribute)@\spxentry{sensor\_width}\spxextra{deepdrr.device.MobileCArm attribute}}

\begin{fulllineitems}
\phantomsection\label{\detokenize{deepdrr.device:deepdrr.device.MobileCArm.sensor_width}}
\pysigstartsignatures
\pysigline{\sphinxbfcode{\sphinxupquote{sensor\_width}}\sphinxbfcode{\sphinxupquote{\DUrole{p,p}{:}\DUrole{w,w}{  }int}}}
\pysigstopsignatures
\end{fulllineitems}

\index{source\_height (deepdrr.device.MobileCArm attribute)@\spxentry{source\_height}\spxextra{deepdrr.device.MobileCArm attribute}}

\begin{fulllineitems}
\phantomsection\label{\detokenize{deepdrr.device:deepdrr.device.MobileCArm.source_height}}
\pysigstartsignatures
\pysigline{\sphinxbfcode{\sphinxupquote{source\_height}}\sphinxbfcode{\sphinxupquote{\DUrole{w,w}{  }\DUrole{p,p}{=}\DUrole{w,w}{  }200}}}
\pysigstopsignatures
\end{fulllineitems}

\index{source\_in\_arm (deepdrr.device.MobileCArm property)@\spxentry{source\_in\_arm}\spxextra{deepdrr.device.MobileCArm property}}

\begin{fulllineitems}
\phantomsection\label{\detokenize{deepdrr.device:deepdrr.device.MobileCArm.source_in_arm}}
\pysigstartsignatures
\pysigline{\sphinxbfcode{\sphinxupquote{property\DUrole{w,w}{  }}}\sphinxbfcode{\sphinxupquote{source\_in\_arm}}\sphinxbfcode{\sphinxupquote{\DUrole{p,p}{:}\DUrole{w,w}{  }{\hyperref[\detokenize{deepdrr.geo:deepdrr.geo.core.Point3D}]{\sphinxcrossref{Point3D}}}}}}
\pysigstopsignatures
\end{fulllineitems}

\index{source\_in\_device (deepdrr.device.MobileCArm property)@\spxentry{source\_in\_device}\spxextra{deepdrr.device.MobileCArm property}}

\begin{fulllineitems}
\phantomsection\label{\detokenize{deepdrr.device:deepdrr.device.MobileCArm.source_in_device}}
\pysigstartsignatures
\pysigline{\sphinxbfcode{\sphinxupquote{property\DUrole{w,w}{  }}}\sphinxbfcode{\sphinxupquote{source\_in\_device}}\sphinxbfcode{\sphinxupquote{\DUrole{p,p}{:}\DUrole{w,w}{  }{\hyperref[\detokenize{deepdrr.geo:deepdrr.geo.core.Point3D}]{\sphinxcrossref{Point3D}}}}}}
\pysigstopsignatures
\end{fulllineitems}

\index{source\_radius (deepdrr.device.MobileCArm attribute)@\spxentry{source\_radius}\spxextra{deepdrr.device.MobileCArm attribute}}

\begin{fulllineitems}
\phantomsection\label{\detokenize{deepdrr.device:deepdrr.device.MobileCArm.source_radius}}
\pysigstartsignatures
\pysigline{\sphinxbfcode{\sphinxupquote{source\_radius}}\sphinxbfcode{\sphinxupquote{\DUrole{w,w}{  }\DUrole{p,p}{=}\DUrole{w,w}{  }100}}}
\pysigstopsignatures
\end{fulllineitems}

\index{source\_to\_detector\_distance (deepdrr.device.MobileCArm attribute)@\spxentry{source\_to\_detector\_distance}\spxextra{deepdrr.device.MobileCArm attribute}}

\begin{fulllineitems}
\phantomsection\label{\detokenize{deepdrr.device:deepdrr.device.MobileCArm.source_to_detector_distance}}
\pysigstartsignatures
\pysigline{\sphinxbfcode{\sphinxupquote{source\_to\_detector\_distance}}\sphinxbfcode{\sphinxupquote{\DUrole{p,p}{:}\DUrole{w,w}{  }float}}}
\pysigstopsignatures
\end{fulllineitems}

\index{to\_config() (deepdrr.device.MobileCArm method)@\spxentry{to\_config()}\spxextra{deepdrr.device.MobileCArm method}}

\begin{fulllineitems}
\phantomsection\label{\detokenize{deepdrr.device:deepdrr.device.MobileCArm.to_config}}
\pysigstartsignatures
\pysiglinewithargsret{\sphinxbfcode{\sphinxupquote{to\_config}}}{}{{ $\rightarrow$ Dict\DUrole{p,p}{{[}}str\DUrole{p,p}{,}\DUrole{w,w}{  }Any\DUrole{p,p}{{]}}}}
\pysigstopsignatures
\sphinxAtStartPar
Get a json\sphinxhyphen{}safe dictionary that can be used to initialize the C\sphinxhyphen{}arm in its current pose.

\end{fulllineitems}

\index{viewpoint (deepdrr.device.MobileCArm property)@\spxentry{viewpoint}\spxextra{deepdrr.device.MobileCArm property}}

\begin{fulllineitems}
\phantomsection\label{\detokenize{deepdrr.device:deepdrr.device.MobileCArm.viewpoint}}
\pysigstartsignatures
\pysigline{\sphinxbfcode{\sphinxupquote{property\DUrole{w,w}{  }}}\sphinxbfcode{\sphinxupquote{viewpoint}}\sphinxbfcode{\sphinxupquote{\DUrole{p,p}{:}\DUrole{w,w}{  }{\hyperref[\detokenize{deepdrr.geo:deepdrr.geo.core.Point3D}]{\sphinxcrossref{Point3D}}}}}}
\pysigstopsignatures
\sphinxAtStartPar
Get the point along the principle ray, where objects of interest should ideally be placed.
\begin{quote}\begin{description}
\sphinxlineitem{Returns}
\sphinxAtStartPar
the viewpoint in the device frame.

\sphinxlineitem{Return type}
\sphinxAtStartPar
{\hyperref[\detokenize{deepdrr.geo:deepdrr.geo.Point3D}]{\sphinxcrossref{geo.Point3D}}}

\end{description}\end{quote}

\end{fulllineitems}

\index{viewpoint\_in\_world (deepdrr.device.MobileCArm property)@\spxentry{viewpoint\_in\_world}\spxextra{deepdrr.device.MobileCArm property}}

\begin{fulllineitems}
\phantomsection\label{\detokenize{deepdrr.device:deepdrr.device.MobileCArm.viewpoint_in_world}}
\pysigstartsignatures
\pysigline{\sphinxbfcode{\sphinxupquote{property\DUrole{w,w}{  }}}\sphinxbfcode{\sphinxupquote{viewpoint\_in\_world}}\sphinxbfcode{\sphinxupquote{\DUrole{p,p}{:}\DUrole{w,w}{  }{\hyperref[\detokenize{deepdrr.geo:deepdrr.geo.core.Point3D}]{\sphinxcrossref{Point3D}}}}}}
\pysigstopsignatures
\end{fulllineitems}

\index{world\_from\_device (deepdrr.device.MobileCArm attribute)@\spxentry{world\_from\_device}\spxextra{deepdrr.device.MobileCArm attribute}}

\begin{fulllineitems}
\phantomsection\label{\detokenize{deepdrr.device:deepdrr.device.MobileCArm.world_from_device}}
\pysigstartsignatures
\pysigline{\sphinxbfcode{\sphinxupquote{world\_from\_device}}\sphinxbfcode{\sphinxupquote{\DUrole{p,p}{:}\DUrole{w,w}{  }{\hyperref[\detokenize{deepdrr.geo:deepdrr.geo.core.FrameTransform}]{\sphinxcrossref{FrameTransform}}}}}}
\pysigstopsignatures
\end{fulllineitems}


\end{fulllineitems}

\index{SimpleDevice (class in deepdrr.device)@\spxentry{SimpleDevice}\spxextra{class in deepdrr.device}}

\begin{fulllineitems}
\phantomsection\label{\detokenize{deepdrr.device:deepdrr.device.SimpleDevice}}
\pysigstartsignatures
\pysiglinewithargsret{\sphinxbfcode{\sphinxupquote{class\DUrole{w,w}{  }}}\sphinxcode{\sphinxupquote{deepdrr.device.}}\sphinxbfcode{\sphinxupquote{SimpleDevice}}}{\sphinxparam{\DUrole{n,n}{sensor\_height}\DUrole{p,p}{:}\DUrole{w,w}{  }\DUrole{n,n}{int}\DUrole{w,w}{  }\DUrole{o,o}{=}\DUrole{w,w}{  }\DUrole{default_value}{300}}\sphinxparamcomma \sphinxparam{\DUrole{n,n}{sensor\_width}\DUrole{p,p}{:}\DUrole{w,w}{  }\DUrole{n,n}{int}\DUrole{w,w}{  }\DUrole{o,o}{=}\DUrole{w,w}{  }\DUrole{default_value}{300}}\sphinxparamcomma \sphinxparam{\DUrole{n,n}{pixel\_size}\DUrole{p,p}{:}\DUrole{w,w}{  }\DUrole{n,n}{float}\DUrole{w,w}{  }\DUrole{o,o}{=}\DUrole{w,w}{  }\DUrole{default_value}{0.1}}\sphinxparamcomma \sphinxparam{\DUrole{n,n}{source\_to\_detector\_distance}\DUrole{p,p}{:}\DUrole{w,w}{  }\DUrole{n,n}{float}\DUrole{w,w}{  }\DUrole{o,o}{=}\DUrole{w,w}{  }\DUrole{default_value}{1000.0}}\sphinxparamcomma \sphinxparam{\DUrole{n,n}{world\_from\_device}\DUrole{p,p}{:}\DUrole{w,w}{  }\DUrole{n,n}{{\hyperref[\detokenize{deepdrr.geo:deepdrr.geo.core.FrameTransform}]{\sphinxcrossref{FrameTransform}}}\DUrole{w,w}{  }\DUrole{p,p}{|}\DUrole{w,w}{  }None}\DUrole{w,w}{  }\DUrole{o,o}{=}\DUrole{w,w}{  }\DUrole{default_value}{None}}}{}
\pysigstopsignatures
\sphinxAtStartPar
Bases: {\hyperref[\detokenize{deepdrr.device:deepdrr.device.device.Device}]{\sphinxcrossref{\sphinxcode{\sphinxupquote{Device}}}}}

\sphinxAtStartPar
A simple C\sphinxhyphen{}arm with a point, direction interface to set views.

\sphinxAtStartPar
The “point” being positioned is always at the midpoint of the source and detector. The direction
is the direction from the source to the detector. The up\sphinxhyphen{}vector of images can also be provided,
not necessarily in the image plane (projected onto it).

\sphinxAtStartPar
Any of the device’s attributes can be set directly. The default values are not based on any
particular device.

\sphinxAtStartPar
This class may be useful for understanding basic concepts.
\index{sensor\_height (deepdrr.device.SimpleDevice attribute)@\spxentry{sensor\_height}\spxextra{deepdrr.device.SimpleDevice attribute}}

\begin{fulllineitems}
\phantomsection\label{\detokenize{deepdrr.device:deepdrr.device.SimpleDevice.sensor_height}}
\pysigstartsignatures
\pysigline{\sphinxbfcode{\sphinxupquote{sensor\_height}}}
\pysigstopsignatures
\sphinxAtStartPar
the height of the sensor in pixels.
\begin{quote}\begin{description}
\sphinxlineitem{Type}
\sphinxAtStartPar
int

\end{description}\end{quote}

\end{fulllineitems}

\index{sensor\_width (deepdrr.device.SimpleDevice attribute)@\spxentry{sensor\_width}\spxextra{deepdrr.device.SimpleDevice attribute}}

\begin{fulllineitems}
\phantomsection\label{\detokenize{deepdrr.device:deepdrr.device.SimpleDevice.sensor_width}}
\pysigstartsignatures
\pysigline{\sphinxbfcode{\sphinxupquote{sensor\_width}}}
\pysigstopsignatures
\sphinxAtStartPar
the width of the sensor in pixels.
\begin{quote}\begin{description}
\sphinxlineitem{Type}
\sphinxAtStartPar
int

\end{description}\end{quote}

\end{fulllineitems}

\index{pixel\_size (deepdrr.device.SimpleDevice attribute)@\spxentry{pixel\_size}\spxextra{deepdrr.device.SimpleDevice attribute}}

\begin{fulllineitems}
\phantomsection\label{\detokenize{deepdrr.device:deepdrr.device.SimpleDevice.pixel_size}}
\pysigstartsignatures
\pysigline{\sphinxbfcode{\sphinxupquote{pixel\_size}}}
\pysigstopsignatures
\sphinxAtStartPar
the size of a pixel in mm.
\begin{quote}\begin{description}
\sphinxlineitem{Type}
\sphinxAtStartPar
float

\end{description}\end{quote}

\end{fulllineitems}

\index{source\_to\_detector\_distance (deepdrr.device.SimpleDevice attribute)@\spxentry{source\_to\_detector\_distance}\spxextra{deepdrr.device.SimpleDevice attribute}}

\begin{fulllineitems}
\phantomsection\label{\detokenize{deepdrr.device:deepdrr.device.SimpleDevice.source_to_detector_distance}}
\pysigstartsignatures
\pysigline{\sphinxbfcode{\sphinxupquote{source\_to\_detector\_distance}}}
\pysigstopsignatures
\sphinxAtStartPar
the distance from the source to the detector in mm.
\begin{quote}\begin{description}
\sphinxlineitem{Type}
\sphinxAtStartPar
float

\end{description}\end{quote}

\end{fulllineitems}

\index{world\_from\_device (deepdrr.device.SimpleDevice attribute)@\spxentry{world\_from\_device}\spxextra{deepdrr.device.SimpleDevice attribute}}

\begin{fulllineitems}
\phantomsection\label{\detokenize{deepdrr.device:deepdrr.device.SimpleDevice.world_from_device}}
\pysigstartsignatures
\pysigline{\sphinxbfcode{\sphinxupquote{world\_from\_device}}}
\pysigstopsignatures
\sphinxAtStartPar
the “world\_from\_device” frame transformation for the device.
\begin{quote}\begin{description}
\sphinxlineitem{Type}
\sphinxAtStartPar
{\hyperref[\detokenize{deepdrr.geo:deepdrr.geo.core.FrameTransform}]{\sphinxcrossref{FrameTransform}}}

\end{description}\end{quote}

\end{fulllineitems}

\index{camera\_intrinsics (deepdrr.device.SimpleDevice property)@\spxentry{camera\_intrinsics}\spxextra{deepdrr.device.SimpleDevice property}}

\begin{fulllineitems}
\phantomsection\label{\detokenize{deepdrr.device:deepdrr.device.SimpleDevice.camera_intrinsics}}
\pysigstartsignatures
\pysigline{\sphinxbfcode{\sphinxupquote{property\DUrole{w,w}{  }}}\sphinxbfcode{\sphinxupquote{camera\_intrinsics}}\sphinxbfcode{\sphinxupquote{\DUrole{p,p}{:}\DUrole{w,w}{  }{\hyperref[\detokenize{deepdrr.geo:deepdrr.geo.core.CameraIntrinsicTransform}]{\sphinxcrossref{CameraIntrinsicTransform}}}}}}
\pysigstopsignatures
\sphinxAtStartPar
Get the camera intrinsics for the device.
\begin{quote}\begin{description}
\sphinxlineitem{Returns}
\sphinxAtStartPar
the camera intrinsics for the device.

\sphinxlineitem{Return type}
\sphinxAtStartPar
{\hyperref[\detokenize{deepdrr.geo:deepdrr.geo.core.CameraIntrinsicTransform}]{\sphinxcrossref{CameraIntrinsicTransform}}}

\end{description}\end{quote}

\end{fulllineitems}

\index{device\_from\_camera3d (deepdrr.device.SimpleDevice property)@\spxentry{device\_from\_camera3d}\spxextra{deepdrr.device.SimpleDevice property}}

\begin{fulllineitems}
\phantomsection\label{\detokenize{deepdrr.device:deepdrr.device.SimpleDevice.device_from_camera3d}}
\pysigstartsignatures
\pysigline{\sphinxbfcode{\sphinxupquote{property\DUrole{w,w}{  }}}\sphinxbfcode{\sphinxupquote{device\_from\_camera3d}}\sphinxbfcode{\sphinxupquote{\DUrole{p,p}{:}\DUrole{w,w}{  }{\hyperref[\detokenize{deepdrr.geo:deepdrr.geo.core.FrameTransform}]{\sphinxcrossref{FrameTransform}}}}}}
\pysigstopsignatures
\sphinxAtStartPar
Get the FrameTransform for the device’s camera3d\_from\_device frame (in the current pose).
\begin{quote}\begin{description}
\sphinxlineitem{Parameters}
\sphinxAtStartPar
\sphinxstyleliteralstrong{\sphinxupquote{camera3d\_transform}} ({\hyperref[\detokenize{deepdrr.geo:deepdrr.geo.core.FrameTransform}]{\sphinxcrossref{\sphinxstyleliteralemphasis{\sphinxupquote{FrameTransform}}}}}) \textendash{} the “camera3d\_from\_device” frame transformation for the device.

\sphinxlineitem{Returns}
\sphinxAtStartPar
the “device\_from\_camera3d” frame transformation for the device.

\sphinxlineitem{Return type}
\sphinxAtStartPar
{\hyperref[\detokenize{deepdrr.geo:deepdrr.geo.core.FrameTransform}]{\sphinxcrossref{FrameTransform}}}

\end{description}\end{quote}

\end{fulllineitems}

\index{pixel\_size (deepdrr.device.SimpleDevice attribute)@\spxentry{pixel\_size}\spxextra{deepdrr.device.SimpleDevice attribute}}

\begin{fulllineitems}
\phantomsection\label{\detokenize{deepdrr.device:id11}}
\pysigstartsignatures
\pysigline{\sphinxbfcode{\sphinxupquote{pixel\_size}}\sphinxbfcode{\sphinxupquote{\DUrole{p,p}{:}\DUrole{w,w}{  }float}}}
\pysigstopsignatures
\end{fulllineitems}

\index{sensor\_height (deepdrr.device.SimpleDevice attribute)@\spxentry{sensor\_height}\spxextra{deepdrr.device.SimpleDevice attribute}}

\begin{fulllineitems}
\phantomsection\label{\detokenize{deepdrr.device:id12}}
\pysigstartsignatures
\pysigline{\sphinxbfcode{\sphinxupquote{sensor\_height}}\sphinxbfcode{\sphinxupquote{\DUrole{p,p}{:}\DUrole{w,w}{  }int}}}
\pysigstopsignatures
\end{fulllineitems}

\index{sensor\_width (deepdrr.device.SimpleDevice attribute)@\spxentry{sensor\_width}\spxextra{deepdrr.device.SimpleDevice attribute}}

\begin{fulllineitems}
\phantomsection\label{\detokenize{deepdrr.device:id13}}
\pysigstartsignatures
\pysigline{\sphinxbfcode{\sphinxupquote{sensor\_width}}\sphinxbfcode{\sphinxupquote{\DUrole{p,p}{:}\DUrole{w,w}{  }int}}}
\pysigstopsignatures
\end{fulllineitems}

\index{set\_view() (deepdrr.device.SimpleDevice method)@\spxentry{set\_view()}\spxextra{deepdrr.device.SimpleDevice method}}

\begin{fulllineitems}
\phantomsection\label{\detokenize{deepdrr.device:deepdrr.device.SimpleDevice.set_view}}
\pysigstartsignatures
\pysiglinewithargsret{\sphinxbfcode{\sphinxupquote{set\_view}}}{\sphinxparam{\DUrole{n,n}{point}\DUrole{p,p}{:}\DUrole{w,w}{  }\DUrole{n,n}{{\hyperref[\detokenize{deepdrr.geo:deepdrr.geo.core.Point3D}]{\sphinxcrossref{Point3D}}}\DUrole{w,w}{  }\DUrole{p,p}{|}\DUrole{w,w}{  }None}\DUrole{w,w}{  }\DUrole{o,o}{=}\DUrole{w,w}{  }\DUrole{default_value}{None}}\sphinxparamcomma \sphinxparam{\DUrole{n,n}{direction}\DUrole{p,p}{:}\DUrole{w,w}{  }\DUrole{n,n}{{\hyperref[\detokenize{deepdrr.geo:deepdrr.geo.core.Vector3D}]{\sphinxcrossref{Vector3D}}}\DUrole{w,w}{  }\DUrole{p,p}{|}\DUrole{w,w}{  }None}\DUrole{w,w}{  }\DUrole{o,o}{=}\DUrole{w,w}{  }\DUrole{default_value}{None}}\sphinxparamcomma \sphinxparam{\DUrole{n,n}{up}\DUrole{p,p}{:}\DUrole{w,w}{  }\DUrole{n,n}{{\hyperref[\detokenize{deepdrr.geo:deepdrr.geo.core.Vector3D}]{\sphinxcrossref{Vector3D}}}\DUrole{w,w}{  }\DUrole{p,p}{|}\DUrole{w,w}{  }None}\DUrole{w,w}{  }\DUrole{o,o}{=}\DUrole{w,w}{  }\DUrole{default_value}{None}}\sphinxparamcomma \sphinxparam{\DUrole{n,n}{source\_to\_point\_distance}\DUrole{p,p}{:}\DUrole{w,w}{  }\DUrole{n,n}{float\DUrole{w,w}{  }\DUrole{p,p}{|}\DUrole{w,w}{  }None}\DUrole{w,w}{  }\DUrole{o,o}{=}\DUrole{w,w}{  }\DUrole{default_value}{None}}\sphinxparamcomma \sphinxparam{\DUrole{n,n}{source\_to\_point\_fraction}\DUrole{p,p}{:}\DUrole{w,w}{  }\DUrole{n,n}{float}\DUrole{w,w}{  }\DUrole{o,o}{=}\DUrole{w,w}{  }\DUrole{default_value}{0.5}}}{}
\pysigstopsignatures
\sphinxAtStartPar
Set the view of the device.

\sphinxAtStartPar
Can be called with a Ray3D as the first argument, by doing \sphinxtitleref{device.set\_view(*ray)}.
\begin{quote}\begin{description}
\sphinxlineitem{Parameters}\begin{itemize}
\item {} 
\sphinxAtStartPar
\sphinxstyleliteralstrong{\sphinxupquote{center}} ({\hyperref[\detokenize{deepdrr.geo:deepdrr.geo.core.Point3D}]{\sphinxcrossref{\sphinxstyleliteralemphasis{\sphinxupquote{Point3D}}}}}) \textendash{} the point at the center of the source and detector, in world coordinates. If None,
the point is left unchanged (rotation only). Default: None.

\item {} 
\sphinxAtStartPar
\sphinxstyleliteralstrong{\sphinxupquote{direction}} ({\hyperref[\detokenize{deepdrr.geo:deepdrr.geo.core.Vector3D}]{\sphinxcrossref{\sphinxstyleliteralemphasis{\sphinxupquote{Vector3D}}}}}) \textendash{} the direction from the source to the detector, in world coordinates. If None,
the direction is set to the +Z axis. Default: None.

\item {} 
\sphinxAtStartPar
\sphinxstyleliteralstrong{\sphinxupquote{up}} ({\hyperref[\detokenize{deepdrr.geo:deepdrr.geo.core.Vector3D}]{\sphinxcrossref{\sphinxstyleliteralemphasis{\sphinxupquote{Vector3D}}}}}) \textendash{} the up vector of the image, in world\_coordinates. It’s projection
corresponds to the \sphinxhyphen{}Y axis in the camera3d frame. If None, the up vector is set to the \sphinxhyphen{}Y
axis of the device frame.

\item {} 
\sphinxAtStartPar
\sphinxstyleliteralstrong{\sphinxupquote{source\_to\_point\_distance}} (\sphinxstyleliteralemphasis{\sphinxupquote{float}}) \textendash{} the distance from the source to the point. If None, the distance
is \sphinxtitleref{source\_to\_point\_fraction} of the source\sphinxhyphen{}to\sphinxhyphen{}detector distance. Default: None.

\item {} 
\sphinxAtStartPar
\sphinxstyleliteralstrong{\sphinxupquote{source\_to\_point\_fraction}} (\sphinxstyleliteralemphasis{\sphinxupquote{float}}) \textendash{} the fraction of the source\sphinxhyphen{}to\sphinxhyphen{}detector distance to use as the
source\sphinxhyphen{}to\sphinxhyphen{}point distance. Default: 0.5.

\end{itemize}

\end{description}\end{quote}

\end{fulllineitems}

\index{source\_to\_detector\_distance (deepdrr.device.SimpleDevice attribute)@\spxentry{source\_to\_detector\_distance}\spxextra{deepdrr.device.SimpleDevice attribute}}

\begin{fulllineitems}
\phantomsection\label{\detokenize{deepdrr.device:id14}}
\pysigstartsignatures
\pysigline{\sphinxbfcode{\sphinxupquote{source\_to\_detector\_distance}}\sphinxbfcode{\sphinxupquote{\DUrole{p,p}{:}\DUrole{w,w}{  }float}}}
\pysigstopsignatures
\end{fulllineitems}

\index{world\_from\_device (deepdrr.device.SimpleDevice attribute)@\spxentry{world\_from\_device}\spxextra{deepdrr.device.SimpleDevice attribute}}

\begin{fulllineitems}
\phantomsection\label{\detokenize{deepdrr.device:id15}}
\pysigstartsignatures
\pysigline{\sphinxbfcode{\sphinxupquote{world\_from\_device}}\sphinxbfcode{\sphinxupquote{\DUrole{p,p}{:}\DUrole{w,w}{  }{\hyperref[\detokenize{deepdrr.geo:deepdrr.geo.core.FrameTransform}]{\sphinxcrossref{FrameTransform}}}}}}
\pysigstopsignatures
\end{fulllineitems}


\end{fulllineitems}


\sphinxstepscope


\section{deepdrr.geo package}
\label{\detokenize{deepdrr.geo:deepdrr-geo-package}}\label{\detokenize{deepdrr.geo::doc}}

\subsection{deepdrr.geo.core}
\label{\detokenize{deepdrr.geo:module-deepdrr.geo.core}}\label{\detokenize{deepdrr.geo:deepdrr-geo-core}}\index{module@\spxentry{module}!deepdrr.geo.core@\spxentry{deepdrr.geo.core}}\index{deepdrr.geo.core@\spxentry{deepdrr.geo.core}!module@\spxentry{module}}
\sphinxAtStartPar
Homogeneous geometry library.

\sphinxAtStartPar
Copyright (c) 2022, Benjamin D. Killeen. MIT License.

\sphinxAtStartPar
KNOWN ISSUES:
\begin{itemize}
\item {} 
\sphinxAtStartPar
When multiplying vectors by scalars it is safer to put the vector on the left. This is because
your float or int may actually by a numpy scalar, in which case numpy will greedily convert the
vector (which has an \_\_array\_\_ method) to a numpy array, so the multiplication will return an
np.ndarray and not a geo.Vector. It will still be the \sphinxstyleemphasis{correct} result, just the wrong type (and
no longer homogeneous).

\end{itemize}
\index{CameraIntrinsicTransform (class in deepdrr.geo.core)@\spxentry{CameraIntrinsicTransform}\spxextra{class in deepdrr.geo.core}}

\begin{fulllineitems}
\phantomsection\label{\detokenize{deepdrr.geo:deepdrr.geo.core.CameraIntrinsicTransform}}
\pysigstartsignatures
\pysiglinewithargsret{\sphinxbfcode{\sphinxupquote{class\DUrole{w,w}{  }}}\sphinxcode{\sphinxupquote{deepdrr.geo.core.}}\sphinxbfcode{\sphinxupquote{CameraIntrinsicTransform}}}{\sphinxparam{\DUrole{n,n}{data}\DUrole{p,p}{:}\DUrole{w,w}{  }\DUrole{n,n}{ndarray}}\sphinxparamcomma \sphinxparam{\DUrole{n,n}{sensor\_height}\DUrole{p,p}{:}\DUrole{w,w}{  }\DUrole{n,n}{int\DUrole{w,w}{  }\DUrole{p,p}{|}\DUrole{w,w}{  }None}\DUrole{w,w}{  }\DUrole{o,o}{=}\DUrole{w,w}{  }\DUrole{default_value}{None}}\sphinxparamcomma \sphinxparam{\DUrole{n,n}{sensor\_width}\DUrole{p,p}{:}\DUrole{w,w}{  }\DUrole{n,n}{int\DUrole{w,w}{  }\DUrole{p,p}{|}\DUrole{w,w}{  }None}\DUrole{w,w}{  }\DUrole{o,o}{=}\DUrole{w,w}{  }\DUrole{default_value}{None}}}{}
\pysigstopsignatures
\sphinxAtStartPar
Bases: {\hyperref[\detokenize{deepdrr.geo:deepdrr.geo.core.FrameTransform}]{\sphinxcrossref{\sphinxcode{\sphinxupquote{FrameTransform}}}}}
\index{aspect\_ratio (deepdrr.geo.core.CameraIntrinsicTransform property)@\spxentry{aspect\_ratio}\spxextra{deepdrr.geo.core.CameraIntrinsicTransform property}}

\begin{fulllineitems}
\phantomsection\label{\detokenize{deepdrr.geo:deepdrr.geo.core.CameraIntrinsicTransform.aspect_ratio}}
\pysigstartsignatures
\pysigline{\sphinxbfcode{\sphinxupquote{property\DUrole{w,w}{  }}}\sphinxbfcode{\sphinxupquote{aspect\_ratio}}\sphinxbfcode{\sphinxupquote{\DUrole{p,p}{:}\DUrole{w,w}{  }float}}}
\pysigstopsignatures
\sphinxAtStartPar
Image aspect ratio.

\end{fulllineitems}

\index{cx (deepdrr.geo.core.CameraIntrinsicTransform property)@\spxentry{cx}\spxextra{deepdrr.geo.core.CameraIntrinsicTransform property}}

\begin{fulllineitems}
\phantomsection\label{\detokenize{deepdrr.geo:deepdrr.geo.core.CameraIntrinsicTransform.cx}}
\pysigstartsignatures
\pysigline{\sphinxbfcode{\sphinxupquote{property\DUrole{w,w}{  }}}\sphinxbfcode{\sphinxupquote{cx}}\sphinxbfcode{\sphinxupquote{\DUrole{p,p}{:}\DUrole{w,w}{  }float}}}
\pysigstopsignatures
\end{fulllineitems}

\index{cy (deepdrr.geo.core.CameraIntrinsicTransform property)@\spxentry{cy}\spxextra{deepdrr.geo.core.CameraIntrinsicTransform property}}

\begin{fulllineitems}
\phantomsection\label{\detokenize{deepdrr.geo:deepdrr.geo.core.CameraIntrinsicTransform.cy}}
\pysigstartsignatures
\pysigline{\sphinxbfcode{\sphinxupquote{property\DUrole{w,w}{  }}}\sphinxbfcode{\sphinxupquote{cy}}\sphinxbfcode{\sphinxupquote{\DUrole{p,p}{:}\DUrole{w,w}{  }float}}}
\pysigstopsignatures
\end{fulllineitems}

\index{dim (deepdrr.geo.core.CameraIntrinsicTransform attribute)@\spxentry{dim}\spxextra{deepdrr.geo.core.CameraIntrinsicTransform attribute}}

\begin{fulllineitems}
\phantomsection\label{\detokenize{deepdrr.geo:deepdrr.geo.core.CameraIntrinsicTransform.dim}}
\pysigstartsignatures
\pysigline{\sphinxbfcode{\sphinxupquote{dim}}\sphinxbfcode{\sphinxupquote{\DUrole{p,p}{:}\DUrole{w,w}{  }int}}\sphinxbfcode{\sphinxupquote{\DUrole{w,w}{  }\DUrole{p,p}{=}\DUrole{w,w}{  }2}}}
\pysigstopsignatures
\end{fulllineitems}

\index{focal\_length (deepdrr.geo.core.CameraIntrinsicTransform property)@\spxentry{focal\_length}\spxextra{deepdrr.geo.core.CameraIntrinsicTransform property}}

\begin{fulllineitems}
\phantomsection\label{\detokenize{deepdrr.geo:deepdrr.geo.core.CameraIntrinsicTransform.focal_length}}
\pysigstartsignatures
\pysigline{\sphinxbfcode{\sphinxupquote{property\DUrole{w,w}{  }}}\sphinxbfcode{\sphinxupquote{focal\_length}}\sphinxbfcode{\sphinxupquote{\DUrole{p,p}{:}\DUrole{w,w}{  }float}}}
\pysigstopsignatures
\sphinxAtStartPar
Focal length in the matrix units.

\end{fulllineitems}

\index{from\_parameters() (deepdrr.geo.core.CameraIntrinsicTransform class method)@\spxentry{from\_parameters()}\spxextra{deepdrr.geo.core.CameraIntrinsicTransform class method}}

\begin{fulllineitems}
\phantomsection\label{\detokenize{deepdrr.geo:deepdrr.geo.core.CameraIntrinsicTransform.from_parameters}}
\pysigstartsignatures
\pysiglinewithargsret{\sphinxbfcode{\sphinxupquote{classmethod\DUrole{w,w}{  }}}\sphinxbfcode{\sphinxupquote{from\_parameters}}}{\sphinxparam{\DUrole{n,n}{optical\_center}\DUrole{p,p}{:}\DUrole{w,w}{  }\DUrole{n,n}{{\hyperref[\detokenize{deepdrr.geo:deepdrr.geo.core.Point2D}]{\sphinxcrossref{Point2D}}}}}\sphinxparamcomma \sphinxparam{\DUrole{n,n}{focal\_length}\DUrole{p,p}{:}\DUrole{w,w}{  }\DUrole{n,n}{float\DUrole{w,w}{  }\DUrole{p,p}{|}\DUrole{w,w}{  }Tuple\DUrole{p,p}{{[}}float\DUrole{p,p}{,}\DUrole{w,w}{  }float\DUrole{p,p}{{]}}}\DUrole{w,w}{  }\DUrole{o,o}{=}\DUrole{w,w}{  }\DUrole{default_value}{1}}\sphinxparamcomma \sphinxparam{\DUrole{n,n}{shear}\DUrole{p,p}{:}\DUrole{w,w}{  }\DUrole{n,n}{float}\DUrole{w,w}{  }\DUrole{o,o}{=}\DUrole{w,w}{  }\DUrole{default_value}{0}}\sphinxparamcomma \sphinxparam{\DUrole{n,n}{aspect\_ratio}\DUrole{p,p}{:}\DUrole{w,w}{  }\DUrole{n,n}{float\DUrole{w,w}{  }\DUrole{p,p}{|}\DUrole{w,w}{  }None}\DUrole{w,w}{  }\DUrole{o,o}{=}\DUrole{w,w}{  }\DUrole{default_value}{None}}}{{ $\rightarrow$ {\hyperref[\detokenize{deepdrr.geo:deepdrr.geo.core.CameraIntrinsicTransform}]{\sphinxcrossref{CameraIntrinsicTransform}}}}}
\pysigstopsignatures
\sphinxAtStartPar
The camera intrinsic matrix.

\sphinxAtStartPar
The intrinsic matrix is fundamentally a FrameTransform in 2D, namely \sphinxtitleref{index\_from\_camera2d}.
It transforms to the index\sphinxhyphen{}space of the image (as mapped on the sensor)
from the index\sphinxhyphen{}space centered on the principle ray.

\begin{sphinxadmonition}{note}{Note:}
\sphinxAtStartPar
Focal lengths are usually measured in world units (e.g. millimeters.). This function handles the conversion.
\end{sphinxadmonition}

\sphinxAtStartPar
Useful references include Szeliski’s “Computer Vision”
\sphinxhyphen{} \sphinxurl{https://ksimek.github.io/2013/08/13/intrinsic/}
\begin{quote}\begin{description}
\sphinxlineitem{Parameters}\begin{itemize}
\item {} 
\sphinxAtStartPar
\sphinxstyleliteralstrong{\sphinxupquote{optical\_center}} ({\hyperref[\detokenize{deepdrr.geo:deepdrr.geo.core.Point2D}]{\sphinxcrossref{\sphinxstyleliteralemphasis{\sphinxupquote{Point2D}}}}}) \textendash{} the index\sphinxhyphen{}space point where the isocenter (or pinhole) is centered.

\item {} 
\sphinxAtStartPar
\sphinxstyleliteralstrong{\sphinxupquote{focal\_length}} (\sphinxstyleliteralemphasis{\sphinxupquote{Union}}\sphinxstyleliteralemphasis{\sphinxupquote{{[}}}\sphinxstyleliteralemphasis{\sphinxupquote{float}}\sphinxstyleliteralemphasis{\sphinxupquote{, }}\sphinxstyleliteralemphasis{\sphinxupquote{Tuple}}\sphinxstyleliteralemphasis{\sphinxupquote{{[}}}\sphinxstyleliteralemphasis{\sphinxupquote{float}}\sphinxstyleliteralemphasis{\sphinxupquote{, }}\sphinxstyleliteralemphasis{\sphinxupquote{float}}\sphinxstyleliteralemphasis{\sphinxupquote{{]}}}\sphinxstyleliteralemphasis{\sphinxupquote{{]}}}) \textendash{} the focal length in index units. Can be a tubple (f\_x, f\_y),
or a scalar used for both, or a scalar modified by aspect\_ratio, in index units.

\item {} 
\sphinxAtStartPar
\sphinxstyleliteralstrong{\sphinxupquote{shear}} (\sphinxstyleliteralemphasis{\sphinxupquote{float}}) \textendash{} the shear \sphinxtitleref{s} of the camera.

\item {} 
\sphinxAtStartPar
\sphinxstyleliteralstrong{\sphinxupquote{aspect\_ratio}} (\sphinxstyleliteralemphasis{\sphinxupquote{Optional}}\sphinxstyleliteralemphasis{\sphinxupquote{{[}}}\sphinxstyleliteralemphasis{\sphinxupquote{float}}\sphinxstyleliteralemphasis{\sphinxupquote{{]}}}\sphinxstyleliteralemphasis{\sphinxupquote{, }}\sphinxstyleliteralemphasis{\sphinxupquote{optional}}) \textendash{} the aspect ratio \sphinxtitleref{a} (for use with one focal length). If not provided, aspect
ratio is 1. Defaults to None.

\end{itemize}

\sphinxlineitem{Returns}
\sphinxAtStartPar
The camera intrinsic matrix.

\sphinxlineitem{Return type}
\sphinxAtStartPar
{\hyperref[\detokenize{deepdrr.geo:deepdrr.geo.core.CameraIntrinsicTransform}]{\sphinxcrossref{CameraIntrinsicTransform}}}

\end{description}\end{quote}

\end{fulllineitems}

\index{from\_sizes() (deepdrr.geo.core.CameraIntrinsicTransform class method)@\spxentry{from\_sizes()}\spxextra{deepdrr.geo.core.CameraIntrinsicTransform class method}}

\begin{fulllineitems}
\phantomsection\label{\detokenize{deepdrr.geo:deepdrr.geo.core.CameraIntrinsicTransform.from_sizes}}
\pysigstartsignatures
\pysiglinewithargsret{\sphinxbfcode{\sphinxupquote{classmethod\DUrole{w,w}{  }}}\sphinxbfcode{\sphinxupquote{from\_sizes}}}{\sphinxparam{\DUrole{n,n}{sensor\_size}\DUrole{p,p}{:}\DUrole{w,w}{  }\DUrole{n,n}{int\DUrole{w,w}{  }\DUrole{p,p}{|}\DUrole{w,w}{  }Tuple\DUrole{p,p}{{[}}int\DUrole{p,p}{,}\DUrole{w,w}{  }int\DUrole{p,p}{{]}}}}\sphinxparamcomma \sphinxparam{\DUrole{n,n}{pixel\_size}\DUrole{p,p}{:}\DUrole{w,w}{  }\DUrole{n,n}{float\DUrole{w,w}{  }\DUrole{p,p}{|}\DUrole{w,w}{  }Tuple\DUrole{p,p}{{[}}float\DUrole{p,p}{,}\DUrole{w,w}{  }float\DUrole{p,p}{{]}}}}\sphinxparamcomma \sphinxparam{\DUrole{n,n}{source\_to\_detector\_distance}\DUrole{p,p}{:}\DUrole{w,w}{  }\DUrole{n,n}{float}}}{{ $\rightarrow$ {\hyperref[\detokenize{deepdrr.geo:deepdrr.geo.core.CameraIntrinsicTransform}]{\sphinxcrossref{CameraIntrinsicTransform}}}}}
\pysigstopsignatures
\sphinxAtStartPar
Generate the camera from human\sphinxhyphen{}readable parameters.

\sphinxAtStartPar
This is the recommended way to create the camera. Note that although pixel\_size and source\_to\_detector distance are measured in world units,
the camera intrinsic matrix contains no information about the world, as these are merely used to compute the focal length in pixels.
\begin{quote}\begin{description}
\sphinxlineitem{Parameters}\begin{itemize}
\item {} 
\sphinxAtStartPar
\sphinxstyleliteralstrong{\sphinxupquote{sensor\_size}} (\sphinxstyleliteralemphasis{\sphinxupquote{Union}}\sphinxstyleliteralemphasis{\sphinxupquote{{[}}}\sphinxstyleliteralemphasis{\sphinxupquote{float}}\sphinxstyleliteralemphasis{\sphinxupquote{, }}\sphinxstyleliteralemphasis{\sphinxupquote{Tuple}}\sphinxstyleliteralemphasis{\sphinxupquote{{[}}}\sphinxstyleliteralemphasis{\sphinxupquote{float}}\sphinxstyleliteralemphasis{\sphinxupquote{, }}\sphinxstyleliteralemphasis{\sphinxupquote{float}}\sphinxstyleliteralemphasis{\sphinxupquote{{]}}}\sphinxstyleliteralemphasis{\sphinxupquote{{]}}}) \textendash{} (width, height) of the sensor, or a single value for both, in pixels.

\item {} 
\sphinxAtStartPar
\sphinxstyleliteralstrong{\sphinxupquote{pixel\_size}} (\sphinxstyleliteralemphasis{\sphinxupquote{Union}}\sphinxstyleliteralemphasis{\sphinxupquote{{[}}}\sphinxstyleliteralemphasis{\sphinxupquote{float}}\sphinxstyleliteralemphasis{\sphinxupquote{, }}\sphinxstyleliteralemphasis{\sphinxupquote{Tuple}}\sphinxstyleliteralemphasis{\sphinxupquote{{[}}}\sphinxstyleliteralemphasis{\sphinxupquote{float}}\sphinxstyleliteralemphasis{\sphinxupquote{, }}\sphinxstyleliteralemphasis{\sphinxupquote{float}}\sphinxstyleliteralemphasis{\sphinxupquote{{]}}}\sphinxstyleliteralemphasis{\sphinxupquote{{]}}}) \textendash{} (width, height) of a pixel, or a single value for both, in world units (e.g. mm).

\item {} 
\sphinxAtStartPar
\sphinxstyleliteralstrong{\sphinxupquote{source\_to\_detector\_distance}} (\sphinxstyleliteralemphasis{\sphinxupquote{float}}) \textendash{} distance from source to detector in world units.

\end{itemize}

\end{description}\end{quote}

\sphinxAtStartPar
Returns:

\end{fulllineitems}

\index{fx (deepdrr.geo.core.CameraIntrinsicTransform property)@\spxentry{fx}\spxextra{deepdrr.geo.core.CameraIntrinsicTransform property}}

\begin{fulllineitems}
\phantomsection\label{\detokenize{deepdrr.geo:deepdrr.geo.core.CameraIntrinsicTransform.fx}}
\pysigstartsignatures
\pysigline{\sphinxbfcode{\sphinxupquote{property\DUrole{w,w}{  }}}\sphinxbfcode{\sphinxupquote{fx}}\sphinxbfcode{\sphinxupquote{\DUrole{p,p}{:}\DUrole{w,w}{  }float}}}
\pysigstopsignatures
\end{fulllineitems}

\index{fy (deepdrr.geo.core.CameraIntrinsicTransform property)@\spxentry{fy}\spxextra{deepdrr.geo.core.CameraIntrinsicTransform property}}

\begin{fulllineitems}
\phantomsection\label{\detokenize{deepdrr.geo:deepdrr.geo.core.CameraIntrinsicTransform.fy}}
\pysigstartsignatures
\pysigline{\sphinxbfcode{\sphinxupquote{property\DUrole{w,w}{  }}}\sphinxbfcode{\sphinxupquote{fy}}\sphinxbfcode{\sphinxupquote{\DUrole{p,p}{:}\DUrole{w,w}{  }float}}}
\pysigstopsignatures
\end{fulllineitems}

\index{get\_config() (deepdrr.geo.core.CameraIntrinsicTransform method)@\spxentry{get\_config()}\spxextra{deepdrr.geo.core.CameraIntrinsicTransform method}}

\begin{fulllineitems}
\phantomsection\label{\detokenize{deepdrr.geo:deepdrr.geo.core.CameraIntrinsicTransform.get_config}}
\pysigstartsignatures
\pysiglinewithargsret{\sphinxbfcode{\sphinxupquote{get\_config}}}{}{}
\pysigstopsignatures
\sphinxAtStartPar
Get a config dict with the data in this object.

\end{fulllineitems}

\index{input\_dim (deepdrr.geo.core.CameraIntrinsicTransform attribute)@\spxentry{input\_dim}\spxextra{deepdrr.geo.core.CameraIntrinsicTransform attribute}}

\begin{fulllineitems}
\phantomsection\label{\detokenize{deepdrr.geo:deepdrr.geo.core.CameraIntrinsicTransform.input_dim}}
\pysigstartsignatures
\pysigline{\sphinxbfcode{\sphinxupquote{input\_dim}}\sphinxbfcode{\sphinxupquote{\DUrole{p,p}{:}\DUrole{w,w}{  }int}}\sphinxbfcode{\sphinxupquote{\DUrole{w,w}{  }\DUrole{p,p}{=}\DUrole{w,w}{  }2}}}
\pysigstopsignatures
\sphinxAtStartPar
The intrinsic camera transform.

\sphinxAtStartPar
It should be scaled such that the units of the matrix (including the focal length) are in pixels

\end{fulllineitems}

\index{optical\_center (deepdrr.geo.core.CameraIntrinsicTransform property)@\spxentry{optical\_center}\spxextra{deepdrr.geo.core.CameraIntrinsicTransform property}}

\begin{fulllineitems}
\phantomsection\label{\detokenize{deepdrr.geo:deepdrr.geo.core.CameraIntrinsicTransform.optical_center}}
\pysigstartsignatures
\pysigline{\sphinxbfcode{\sphinxupquote{property\DUrole{w,w}{  }}}\sphinxbfcode{\sphinxupquote{optical\_center}}\sphinxbfcode{\sphinxupquote{\DUrole{p,p}{:}\DUrole{w,w}{  }{\hyperref[\detokenize{deepdrr.geo:deepdrr.geo.core.Point2D}]{\sphinxcrossref{Point2D}}}}}}
\pysigstopsignatures
\end{fulllineitems}

\index{sensor\_height (deepdrr.geo.core.CameraIntrinsicTransform property)@\spxentry{sensor\_height}\spxextra{deepdrr.geo.core.CameraIntrinsicTransform property}}

\begin{fulllineitems}
\phantomsection\label{\detokenize{deepdrr.geo:deepdrr.geo.core.CameraIntrinsicTransform.sensor_height}}
\pysigstartsignatures
\pysigline{\sphinxbfcode{\sphinxupquote{property\DUrole{w,w}{  }}}\sphinxbfcode{\sphinxupquote{sensor\_height}}\sphinxbfcode{\sphinxupquote{\DUrole{p,p}{:}\DUrole{w,w}{  }int}}}
\pysigstopsignatures
\sphinxAtStartPar
Get the sensor height in pixels.

\sphinxAtStartPar
Assumes optical center is at the center of the sensor.

\sphinxAtStartPar
Based on the convention of origin in top left, with x pointing to the right and y pointing down.

\end{fulllineitems}

\index{sensor\_size (deepdrr.geo.core.CameraIntrinsicTransform property)@\spxentry{sensor\_size}\spxextra{deepdrr.geo.core.CameraIntrinsicTransform property}}

\begin{fulllineitems}
\phantomsection\label{\detokenize{deepdrr.geo:deepdrr.geo.core.CameraIntrinsicTransform.sensor_size}}
\pysigstartsignatures
\pysigline{\sphinxbfcode{\sphinxupquote{property\DUrole{w,w}{  }}}\sphinxbfcode{\sphinxupquote{sensor\_size}}\sphinxbfcode{\sphinxupquote{\DUrole{p,p}{:}\DUrole{w,w}{  }Tuple\DUrole{p,p}{{[}}int\DUrole{p,p}{,}\DUrole{w,w}{  }int\DUrole{p,p}{{]}}}}}
\pysigstopsignatures
\sphinxAtStartPar
Tuple with the (width, height) of the sense/image, in matrix units.

\end{fulllineitems}

\index{sensor\_width (deepdrr.geo.core.CameraIntrinsicTransform property)@\spxentry{sensor\_width}\spxextra{deepdrr.geo.core.CameraIntrinsicTransform property}}

\begin{fulllineitems}
\phantomsection\label{\detokenize{deepdrr.geo:deepdrr.geo.core.CameraIntrinsicTransform.sensor_width}}
\pysigstartsignatures
\pysigline{\sphinxbfcode{\sphinxupquote{property\DUrole{w,w}{  }}}\sphinxbfcode{\sphinxupquote{sensor\_width}}\sphinxbfcode{\sphinxupquote{\DUrole{p,p}{:}\DUrole{w,w}{  }int}}}
\pysigstopsignatures
\sphinxAtStartPar
Get the sensor width in the matrix units.

\sphinxAtStartPar
Assumes optical center is at the center of the sensor.

\sphinxAtStartPar
Based on the convention of origin in top left, with x pointing to the right and y pointing down.

\end{fulllineitems}


\end{fulllineitems}

\index{CameraProjection (class in deepdrr.geo.core)@\spxentry{CameraProjection}\spxextra{class in deepdrr.geo.core}}

\begin{fulllineitems}
\phantomsection\label{\detokenize{deepdrr.geo:deepdrr.geo.core.CameraProjection}}
\pysigstartsignatures
\pysiglinewithargsret{\sphinxbfcode{\sphinxupquote{class\DUrole{w,w}{  }}}\sphinxcode{\sphinxupquote{deepdrr.geo.core.}}\sphinxbfcode{\sphinxupquote{CameraProjection}}}{\sphinxparam{\DUrole{n,n}{intrinsic}\DUrole{p,p}{:}\DUrole{w,w}{  }\DUrole{n,n}{{\hyperref[\detokenize{deepdrr.geo:deepdrr.geo.core.CameraIntrinsicTransform}]{\sphinxcrossref{CameraIntrinsicTransform}}}\DUrole{w,w}{  }\DUrole{p,p}{|}\DUrole{w,w}{  }ndarray}}\sphinxparamcomma \sphinxparam{\DUrole{n,n}{extrinsic}\DUrole{p,p}{:}\DUrole{w,w}{  }\DUrole{n,n}{{\hyperref[\detokenize{deepdrr.geo:deepdrr.geo.core.FrameTransform}]{\sphinxcrossref{FrameTransform}}}\DUrole{w,w}{  }\DUrole{p,p}{|}\DUrole{w,w}{  }ndarray}}}{}
\pysigstopsignatures
\sphinxAtStartPar
Bases: {\hyperref[\detokenize{deepdrr.geo:deepdrr.geo.core.Transform}]{\sphinxcrossref{\sphinxcode{\sphinxupquote{Transform}}}}}
\index{K (deepdrr.geo.core.CameraProjection property)@\spxentry{K}\spxextra{deepdrr.geo.core.CameraProjection property}}

\begin{fulllineitems}
\phantomsection\label{\detokenize{deepdrr.geo:deepdrr.geo.core.CameraProjection.K}}
\pysigstartsignatures
\pysigline{\sphinxbfcode{\sphinxupquote{property\DUrole{w,w}{  }}}\sphinxbfcode{\sphinxupquote{K}}}
\pysigstopsignatures
\end{fulllineitems}

\index{R (deepdrr.geo.core.CameraProjection property)@\spxentry{R}\spxextra{deepdrr.geo.core.CameraProjection property}}

\begin{fulllineitems}
\phantomsection\label{\detokenize{deepdrr.geo:deepdrr.geo.core.CameraProjection.R}}
\pysigstartsignatures
\pysigline{\sphinxbfcode{\sphinxupquote{property\DUrole{w,w}{  }}}\sphinxbfcode{\sphinxupquote{R}}}
\pysigstopsignatures
\end{fulllineitems}

\index{camera3d\_from\_index (deepdrr.geo.core.CameraProjection property)@\spxentry{camera3d\_from\_index}\spxextra{deepdrr.geo.core.CameraProjection property}}

\begin{fulllineitems}
\phantomsection\label{\detokenize{deepdrr.geo:deepdrr.geo.core.CameraProjection.camera3d_from_index}}
\pysigstartsignatures
\pysigline{\sphinxbfcode{\sphinxupquote{property\DUrole{w,w}{  }}}\sphinxbfcode{\sphinxupquote{camera3d\_from\_index}}\sphinxbfcode{\sphinxupquote{\DUrole{p,p}{:}\DUrole{w,w}{  }{\hyperref[\detokenize{deepdrr.geo:deepdrr.geo.core.Transform}]{\sphinxcrossref{Transform}}}}}}
\pysigstopsignatures
\end{fulllineitems}

\index{camera3d\_from\_world (deepdrr.geo.core.CameraProjection attribute)@\spxentry{camera3d\_from\_world}\spxextra{deepdrr.geo.core.CameraProjection attribute}}

\begin{fulllineitems}
\phantomsection\label{\detokenize{deepdrr.geo:deepdrr.geo.core.CameraProjection.camera3d_from_world}}
\pysigstartsignatures
\pysigline{\sphinxbfcode{\sphinxupquote{camera3d\_from\_world}}\sphinxbfcode{\sphinxupquote{\DUrole{p,p}{:}\DUrole{w,w}{  }{\hyperref[\detokenize{deepdrr.geo:deepdrr.geo.core.FrameTransform}]{\sphinxcrossref{FrameTransform}}}}}}
\pysigstopsignatures
\end{fulllineitems}

\index{center\_in\_world (deepdrr.geo.core.CameraProjection property)@\spxentry{center\_in\_world}\spxextra{deepdrr.geo.core.CameraProjection property}}

\begin{fulllineitems}
\phantomsection\label{\detokenize{deepdrr.geo:deepdrr.geo.core.CameraProjection.center_in_world}}
\pysigstartsignatures
\pysigline{\sphinxbfcode{\sphinxupquote{property\DUrole{w,w}{  }}}\sphinxbfcode{\sphinxupquote{center\_in\_world}}\sphinxbfcode{\sphinxupquote{\DUrole{p,p}{:}\DUrole{w,w}{  }{\hyperref[\detokenize{deepdrr.geo:deepdrr.geo.core.Point3D}]{\sphinxcrossref{Point3D}}}}}}
\pysigstopsignatures
\end{fulllineitems}

\index{dim (deepdrr.geo.core.CameraProjection attribute)@\spxentry{dim}\spxextra{deepdrr.geo.core.CameraProjection attribute}}

\begin{fulllineitems}
\phantomsection\label{\detokenize{deepdrr.geo:deepdrr.geo.core.CameraProjection.dim}}
\pysigstartsignatures
\pysigline{\sphinxbfcode{\sphinxupquote{dim}}\sphinxbfcode{\sphinxupquote{\DUrole{w,w}{  }\DUrole{p,p}{=}\DUrole{w,w}{  }3}}}
\pysigstopsignatures
\end{fulllineitems}

\index{extrinsic (deepdrr.geo.core.CameraProjection property)@\spxentry{extrinsic}\spxextra{deepdrr.geo.core.CameraProjection property}}

\begin{fulllineitems}
\phantomsection\label{\detokenize{deepdrr.geo:deepdrr.geo.core.CameraProjection.extrinsic}}
\pysigstartsignatures
\pysigline{\sphinxbfcode{\sphinxupquote{property\DUrole{w,w}{  }}}\sphinxbfcode{\sphinxupquote{extrinsic}}\sphinxbfcode{\sphinxupquote{\DUrole{p,p}{:}\DUrole{w,w}{  }{\hyperref[\detokenize{deepdrr.geo:deepdrr.geo.core.FrameTransform}]{\sphinxcrossref{FrameTransform}}}}}}
\pysigstopsignatures
\end{fulllineitems}

\index{from\_krt() (deepdrr.geo.core.CameraProjection class method)@\spxentry{from\_krt()}\spxextra{deepdrr.geo.core.CameraProjection class method}}

\begin{fulllineitems}
\phantomsection\label{\detokenize{deepdrr.geo:deepdrr.geo.core.CameraProjection.from_krt}}
\pysigstartsignatures
\pysiglinewithargsret{\sphinxbfcode{\sphinxupquote{classmethod\DUrole{w,w}{  }}}\sphinxbfcode{\sphinxupquote{from\_krt}}}{\sphinxparam{\DUrole{n,n}{K}\DUrole{p,p}{:}\DUrole{w,w}{  }\DUrole{n,n}{ndarray}}\sphinxparamcomma \sphinxparam{\DUrole{n,n}{R}\DUrole{p,p}{:}\DUrole{w,w}{  }\DUrole{n,n}{ndarray}}\sphinxparamcomma \sphinxparam{\DUrole{n,n}{t}\DUrole{p,p}{:}\DUrole{w,w}{  }\DUrole{n,n}{ndarray}}}{{ $\rightarrow$ {\hyperref[\detokenize{deepdrr.geo:deepdrr.geo.core.CameraProjection}]{\sphinxcrossref{CameraProjection}}}}}
\pysigstopsignatures
\sphinxAtStartPar
Create a CameraProjection from a camera intrinsic matrix and extrinsic matrix.
\begin{quote}\begin{description}
\sphinxlineitem{Parameters}\begin{itemize}
\item {} 
\sphinxAtStartPar
\sphinxstyleliteralstrong{\sphinxupquote{K}} (\sphinxstyleliteralemphasis{\sphinxupquote{np.ndarray}}) \textendash{} the camera intrinsic matrix.

\item {} 
\sphinxAtStartPar
\sphinxstyleliteralstrong{\sphinxupquote{R}} (\sphinxstyleliteralemphasis{\sphinxupquote{np.ndarray}}) \textendash{} the camera extrinsic matrix.

\item {} 
\sphinxAtStartPar
\sphinxstyleliteralstrong{\sphinxupquote{t}} (\sphinxstyleliteralemphasis{\sphinxupquote{np.ndarray}}) \textendash{} the camera extrinsic translation vector.

\end{itemize}

\sphinxlineitem{Returns}
\sphinxAtStartPar
the camera projection.

\sphinxlineitem{Return type}
\sphinxAtStartPar
{\hyperref[\detokenize{deepdrr.geo:deepdrr.geo.core.CameraProjection}]{\sphinxcrossref{CameraProjection}}}

\end{description}\end{quote}

\end{fulllineitems}

\index{from\_rtk() (deepdrr.geo.core.CameraProjection class method)@\spxentry{from\_rtk()}\spxextra{deepdrr.geo.core.CameraProjection class method}}

\begin{fulllineitems}
\phantomsection\label{\detokenize{deepdrr.geo:deepdrr.geo.core.CameraProjection.from_rtk}}
\pysigstartsignatures
\pysiglinewithargsret{\sphinxbfcode{\sphinxupquote{classmethod\DUrole{w,w}{  }}}\sphinxbfcode{\sphinxupquote{from\_rtk}}}{\sphinxparam{\DUrole{n,n}{R}\DUrole{p,p}{:}\DUrole{w,w}{  }\DUrole{n,n}{ndarray}}\sphinxparamcomma \sphinxparam{\DUrole{n,n}{t}\DUrole{p,p}{:}\DUrole{w,w}{  }\DUrole{n,n}{{\hyperref[\detokenize{deepdrr.geo:deepdrr.geo.core.Point3D}]{\sphinxcrossref{Point3D}}}}}\sphinxparamcomma \sphinxparam{\DUrole{n,n}{K}\DUrole{p,p}{:}\DUrole{w,w}{  }\DUrole{n,n}{{\hyperref[\detokenize{deepdrr.geo:deepdrr.geo.core.CameraIntrinsicTransform}]{\sphinxcrossref{CameraIntrinsicTransform}}}\DUrole{w,w}{  }\DUrole{p,p}{|}\DUrole{w,w}{  }ndarray}}}{}
\pysigstopsignatures
\end{fulllineitems}

\index{get\_center\_in\_volume() (deepdrr.geo.core.CameraProjection method)@\spxentry{get\_center\_in\_volume()}\spxextra{deepdrr.geo.core.CameraProjection method}}

\begin{fulllineitems}
\phantomsection\label{\detokenize{deepdrr.geo:deepdrr.geo.core.CameraProjection.get_center_in_volume}}
\pysigstartsignatures
\pysiglinewithargsret{\sphinxbfcode{\sphinxupquote{get\_center\_in\_volume}}}{\sphinxparam{\DUrole{n,n}{volume}\DUrole{p,p}{:}\DUrole{w,w}{  }\DUrole{n,n}{{\hyperref[\detokenize{deepdrr:deepdrr.Volume}]{\sphinxcrossref{Volume}}}}}}{{ $\rightarrow$ {\hyperref[\detokenize{deepdrr.geo:deepdrr.geo.core.Point3D}]{\sphinxcrossref{Point3D}}}}}
\pysigstopsignatures
\sphinxAtStartPar
Get the camera center in IJK\sphinxhyphen{}space.

\sphinxAtStartPar
In original deepdrr, this is the \sphinxtitleref{source\_point} of \sphinxtitleref{get\_canonical\_proj\_matrix()}
\begin{quote}\begin{description}
\sphinxlineitem{Parameters}
\sphinxAtStartPar
\sphinxstyleliteralstrong{\sphinxupquote{volume}} (\sphinxstyleliteralemphasis{\sphinxupquote{AnyVolume}}) \textendash{} the volume to get the camera center in.

\sphinxlineitem{Returns}
\sphinxAtStartPar
the camera center in the volume’s IJK\sphinxhyphen{}space.

\sphinxlineitem{Return type}
\sphinxAtStartPar
{\hyperref[\detokenize{deepdrr.geo:deepdrr.geo.core.Point3D}]{\sphinxcrossref{Point3D}}}

\end{description}\end{quote}

\end{fulllineitems}

\index{get\_center\_in\_world() (deepdrr.geo.core.CameraProjection method)@\spxentry{get\_center\_in\_world()}\spxextra{deepdrr.geo.core.CameraProjection method}}

\begin{fulllineitems}
\phantomsection\label{\detokenize{deepdrr.geo:deepdrr.geo.core.CameraProjection.get_center_in_world}}
\pysigstartsignatures
\pysiglinewithargsret{\sphinxbfcode{\sphinxupquote{get\_center\_in\_world}}}{}{{ $\rightarrow$ {\hyperref[\detokenize{deepdrr.geo:deepdrr.geo.core.Point3D}]{\sphinxcrossref{Point3D}}}}}
\pysigstopsignatures
\sphinxAtStartPar
Get the center of the camera (origin of camera3d frame) in world coordinates.

\sphinxAtStartPar
That is, get the translation vector of the world\_from\_camera3d FrameTransform

\sphinxAtStartPar
This is comparable to the function get\_camera\_center() in DeepDRR.
\begin{quote}\begin{description}
\sphinxlineitem{Returns}
\sphinxAtStartPar
the center of the camera in center.

\sphinxlineitem{Return type}
\sphinxAtStartPar
{\hyperref[\detokenize{deepdrr.geo:deepdrr.geo.core.Point3D}]{\sphinxcrossref{Point3D}}}

\end{description}\end{quote}

\end{fulllineitems}

\index{get\_config() (deepdrr.geo.core.CameraProjection method)@\spxentry{get\_config()}\spxextra{deepdrr.geo.core.CameraProjection method}}

\begin{fulllineitems}
\phantomsection\label{\detokenize{deepdrr.geo:deepdrr.geo.core.CameraProjection.get_config}}
\pysigstartsignatures
\pysiglinewithargsret{\sphinxbfcode{\sphinxupquote{get\_config}}}{}{{ $\rightarrow$ dict\DUrole{p,p}{{[}}str\DUrole{p,p}{,}\DUrole{w,w}{  }Any\DUrole{p,p}{{]}}}}
\pysigstopsignatures
\sphinxAtStartPar
Get the configuration of the camera projection.
\begin{quote}\begin{description}
\sphinxlineitem{Returns}
\sphinxAtStartPar
the configuration of the camera projection.

\sphinxlineitem{Return type}
\sphinxAtStartPar
dict{[}str, Any{]}

\end{description}\end{quote}

\end{fulllineitems}

\index{get\_ray\_transform() (deepdrr.geo.core.CameraProjection method)@\spxentry{get\_ray\_transform()}\spxextra{deepdrr.geo.core.CameraProjection method}}

\begin{fulllineitems}
\phantomsection\label{\detokenize{deepdrr.geo:deepdrr.geo.core.CameraProjection.get_ray_transform}}
\pysigstartsignatures
\pysiglinewithargsret{\sphinxbfcode{\sphinxupquote{get\_ray\_transform}}}{\sphinxparam{\DUrole{n,n}{volume}\DUrole{p,p}{:}\DUrole{w,w}{  }\DUrole{n,n}{{\hyperref[\detokenize{deepdrr:deepdrr.Volume}]{\sphinxcrossref{Volume}}}}}}{{ $\rightarrow$ {\hyperref[\detokenize{deepdrr.geo:deepdrr.geo.core.Transform}]{\sphinxcrossref{Transform}}}}}
\pysigstopsignatures
\sphinxAtStartPar
Get the ray transform for the camera, in IJK\sphinxhyphen{}space.

\sphinxAtStartPar
ijk\_from\_index transformation that goes from Point2D to Vector3D, with the vector in the
Point2D frame.

\sphinxAtStartPar
The ray transform takes a Point2D and converts it to a Vector3D. This is the vector in
the direction pointing between the camera center (or source) and a given index\sphinxhyphen{}space
point on the detector.
\begin{quote}\begin{description}
\sphinxlineitem{Parameters}
\sphinxAtStartPar
\sphinxstyleliteralstrong{\sphinxupquote{volume}} (\sphinxstyleliteralemphasis{\sphinxupquote{AnyVolume}}) \textendash{} the volume to get get the ray transfrom through.

\sphinxlineitem{Returns}
\sphinxAtStartPar
the \sphinxtitleref{ijk\_from\_index} transform.

\sphinxlineitem{Return type}
\sphinxAtStartPar
{\hyperref[\detokenize{deepdrr.geo:deepdrr.geo.core.Transform}]{\sphinxcrossref{Transform}}}

\end{description}\end{quote}

\end{fulllineitems}

\index{index\_from\_camera2d (deepdrr.geo.core.CameraProjection attribute)@\spxentry{index\_from\_camera2d}\spxextra{deepdrr.geo.core.CameraProjection attribute}}

\begin{fulllineitems}
\phantomsection\label{\detokenize{deepdrr.geo:deepdrr.geo.core.CameraProjection.index_from_camera2d}}
\pysigstartsignatures
\pysigline{\sphinxbfcode{\sphinxupquote{index\_from\_camera2d}}\sphinxbfcode{\sphinxupquote{\DUrole{p,p}{:}\DUrole{w,w}{  }{\hyperref[\detokenize{deepdrr.geo:deepdrr.geo.core.CameraIntrinsicTransform}]{\sphinxcrossref{CameraIntrinsicTransform}}}}}}
\pysigstopsignatures
\end{fulllineitems}

\index{index\_from\_camera3d (deepdrr.geo.core.CameraProjection property)@\spxentry{index\_from\_camera3d}\spxextra{deepdrr.geo.core.CameraProjection property}}

\begin{fulllineitems}
\phantomsection\label{\detokenize{deepdrr.geo:deepdrr.geo.core.CameraProjection.index_from_camera3d}}
\pysigstartsignatures
\pysigline{\sphinxbfcode{\sphinxupquote{property\DUrole{w,w}{  }}}\sphinxbfcode{\sphinxupquote{index\_from\_camera3d}}\sphinxbfcode{\sphinxupquote{\DUrole{p,p}{:}\DUrole{w,w}{  }{\hyperref[\detokenize{deepdrr.geo:deepdrr.geo.core.Transform}]{\sphinxcrossref{Transform}}}}}}
\pysigstopsignatures
\end{fulllineitems}

\index{index\_from\_world (deepdrr.geo.core.CameraProjection property)@\spxentry{index\_from\_world}\spxextra{deepdrr.geo.core.CameraProjection property}}

\begin{fulllineitems}
\phantomsection\label{\detokenize{deepdrr.geo:deepdrr.geo.core.CameraProjection.index_from_world}}
\pysigstartsignatures
\pysigline{\sphinxbfcode{\sphinxupquote{property\DUrole{w,w}{  }}}\sphinxbfcode{\sphinxupquote{index\_from\_world}}\sphinxbfcode{\sphinxupquote{\DUrole{p,p}{:}\DUrole{w,w}{  }{\hyperref[\detokenize{deepdrr.geo:deepdrr.geo.core.Transform}]{\sphinxcrossref{Transform}}}}}}
\pysigstopsignatures
\end{fulllineitems}

\index{intrinsic (deepdrr.geo.core.CameraProjection property)@\spxentry{intrinsic}\spxextra{deepdrr.geo.core.CameraProjection property}}

\begin{fulllineitems}
\phantomsection\label{\detokenize{deepdrr.geo:deepdrr.geo.core.CameraProjection.intrinsic}}
\pysigstartsignatures
\pysigline{\sphinxbfcode{\sphinxupquote{property\DUrole{w,w}{  }}}\sphinxbfcode{\sphinxupquote{intrinsic}}\sphinxbfcode{\sphinxupquote{\DUrole{p,p}{:}\DUrole{w,w}{  }{\hyperref[\detokenize{deepdrr.geo:deepdrr.geo.core.CameraIntrinsicTransform}]{\sphinxcrossref{CameraIntrinsicTransform}}}}}}
\pysigstopsignatures
\end{fulllineitems}

\index{principle\_ray\_in\_world (deepdrr.geo.core.CameraProjection property)@\spxentry{principle\_ray\_in\_world}\spxextra{deepdrr.geo.core.CameraProjection property}}

\begin{fulllineitems}
\phantomsection\label{\detokenize{deepdrr.geo:deepdrr.geo.core.CameraProjection.principle_ray_in_world}}
\pysigstartsignatures
\pysigline{\sphinxbfcode{\sphinxupquote{property\DUrole{w,w}{  }}}\sphinxbfcode{\sphinxupquote{principle\_ray\_in\_world}}\sphinxbfcode{\sphinxupquote{\DUrole{p,p}{:}\DUrole{w,w}{  }{\hyperref[\detokenize{deepdrr.geo:deepdrr.geo.core.Vector3D}]{\sphinxcrossref{Vector3D}}}}}}
\pysigstopsignatures
\sphinxAtStartPar
Get the principle ray in world coordinates.

\end{fulllineitems}

\index{sensor\_height (deepdrr.geo.core.CameraProjection property)@\spxentry{sensor\_height}\spxextra{deepdrr.geo.core.CameraProjection property}}

\begin{fulllineitems}
\phantomsection\label{\detokenize{deepdrr.geo:deepdrr.geo.core.CameraProjection.sensor_height}}
\pysigstartsignatures
\pysigline{\sphinxbfcode{\sphinxupquote{property\DUrole{w,w}{  }}}\sphinxbfcode{\sphinxupquote{sensor\_height}}\sphinxbfcode{\sphinxupquote{\DUrole{p,p}{:}\DUrole{w,w}{  }int}}}
\pysigstopsignatures
\end{fulllineitems}

\index{sensor\_width (deepdrr.geo.core.CameraProjection property)@\spxentry{sensor\_width}\spxextra{deepdrr.geo.core.CameraProjection property}}

\begin{fulllineitems}
\phantomsection\label{\detokenize{deepdrr.geo:deepdrr.geo.core.CameraProjection.sensor_width}}
\pysigstartsignatures
\pysigline{\sphinxbfcode{\sphinxupquote{property\DUrole{w,w}{  }}}\sphinxbfcode{\sphinxupquote{sensor\_width}}\sphinxbfcode{\sphinxupquote{\DUrole{p,p}{:}\DUrole{w,w}{  }int}}}
\pysigstopsignatures
\end{fulllineitems}

\index{t (deepdrr.geo.core.CameraProjection property)@\spxentry{t}\spxextra{deepdrr.geo.core.CameraProjection property}}

\begin{fulllineitems}
\phantomsection\label{\detokenize{deepdrr.geo:deepdrr.geo.core.CameraProjection.t}}
\pysigstartsignatures
\pysigline{\sphinxbfcode{\sphinxupquote{property\DUrole{w,w}{  }}}\sphinxbfcode{\sphinxupquote{t}}}
\pysigstopsignatures
\end{fulllineitems}

\index{world\_from\_camera3d (deepdrr.geo.core.CameraProjection property)@\spxentry{world\_from\_camera3d}\spxextra{deepdrr.geo.core.CameraProjection property}}

\begin{fulllineitems}
\phantomsection\label{\detokenize{deepdrr.geo:deepdrr.geo.core.CameraProjection.world_from_camera3d}}
\pysigstartsignatures
\pysigline{\sphinxbfcode{\sphinxupquote{property\DUrole{w,w}{  }}}\sphinxbfcode{\sphinxupquote{world\_from\_camera3d}}\sphinxbfcode{\sphinxupquote{\DUrole{p,p}{:}\DUrole{w,w}{  }{\hyperref[\detokenize{deepdrr.geo:deepdrr.geo.core.FrameTransform}]{\sphinxcrossref{FrameTransform}}}}}}
\pysigstopsignatures
\end{fulllineitems}

\index{world\_from\_index (deepdrr.geo.core.CameraProjection property)@\spxentry{world\_from\_index}\spxextra{deepdrr.geo.core.CameraProjection property}}

\begin{fulllineitems}
\phantomsection\label{\detokenize{deepdrr.geo:deepdrr.geo.core.CameraProjection.world_from_index}}
\pysigstartsignatures
\pysigline{\sphinxbfcode{\sphinxupquote{property\DUrole{w,w}{  }}}\sphinxbfcode{\sphinxupquote{world\_from\_index}}\sphinxbfcode{\sphinxupquote{\DUrole{p,p}{:}\DUrole{w,w}{  }{\hyperref[\detokenize{deepdrr.geo:deepdrr.geo.core.Transform}]{\sphinxcrossref{Transform}}}}}}
\pysigstopsignatures
\sphinxAtStartPar
Gets the world\sphinxhyphen{}space vector between the source in world and the given point in index space.

\end{fulllineitems}

\index{world\_from\_index\_on\_image\_plane (deepdrr.geo.core.CameraProjection property)@\spxentry{world\_from\_index\_on\_image\_plane}\spxextra{deepdrr.geo.core.CameraProjection property}}

\begin{fulllineitems}
\phantomsection\label{\detokenize{deepdrr.geo:deepdrr.geo.core.CameraProjection.world_from_index_on_image_plane}}
\pysigstartsignatures
\pysigline{\sphinxbfcode{\sphinxupquote{property\DUrole{w,w}{  }}}\sphinxbfcode{\sphinxupquote{world\_from\_index\_on\_image\_plane}}\sphinxbfcode{\sphinxupquote{\DUrole{p,p}{:}\DUrole{w,w}{  }{\hyperref[\detokenize{deepdrr.geo:deepdrr.geo.core.FrameTransform}]{\sphinxcrossref{FrameTransform}}}}}}
\pysigstopsignatures
\sphinxAtStartPar
Get the transform to points in world on the image (detector) plane from image indices.

\sphinxAtStartPar
The point input point should still be 3D, with a 0 in the z coordinate.

\end{fulllineitems}


\end{fulllineitems}

\index{F (class in deepdrr.geo.core)@\spxentry{F}\spxextra{class in deepdrr.geo.core}}

\begin{fulllineitems}
\phantomsection\label{\detokenize{deepdrr.geo:deepdrr.geo.core.F}}
\pysigstartsignatures
\pysiglinewithargsret{\sphinxbfcode{\sphinxupquote{class\DUrole{w,w}{  }}}\sphinxcode{\sphinxupquote{deepdrr.geo.core.}}\sphinxbfcode{\sphinxupquote{F}}}{\sphinxparam{\DUrole{n,n}{data}\DUrole{p,p}{:}\DUrole{w,w}{  }\DUrole{n,n}{ndarray}}}{}
\pysigstopsignatures
\sphinxAtStartPar
Bases: {\hyperref[\detokenize{deepdrr.geo:deepdrr.geo.core.FrameTransform}]{\sphinxcrossref{\sphinxcode{\sphinxupquote{FrameTransform}}}}}

\sphinxAtStartPar
Alias for FrameTransform.
\index{data (deepdrr.geo.core.F attribute)@\spxentry{data}\spxextra{deepdrr.geo.core.F attribute}}

\begin{fulllineitems}
\phantomsection\label{\detokenize{deepdrr.geo:deepdrr.geo.core.F.data}}
\pysigstartsignatures
\pysigline{\sphinxbfcode{\sphinxupquote{data}}\sphinxbfcode{\sphinxupquote{\DUrole{p,p}{:}\DUrole{w,w}{  }ndarray}}}
\pysigstopsignatures
\end{fulllineitems}


\end{fulllineitems}

\index{FrameTransform (class in deepdrr.geo.core)@\spxentry{FrameTransform}\spxextra{class in deepdrr.geo.core}}

\begin{fulllineitems}
\phantomsection\label{\detokenize{deepdrr.geo:deepdrr.geo.core.FrameTransform}}
\pysigstartsignatures
\pysiglinewithargsret{\sphinxbfcode{\sphinxupquote{class\DUrole{w,w}{  }}}\sphinxcode{\sphinxupquote{deepdrr.geo.core.}}\sphinxbfcode{\sphinxupquote{FrameTransform}}}{\sphinxparam{\DUrole{n,n}{data}\DUrole{p,p}{:}\DUrole{w,w}{  }\DUrole{n,n}{ndarray}}}{}
\pysigstopsignatures
\sphinxAtStartPar
Bases: {\hyperref[\detokenize{deepdrr.geo:deepdrr.geo.core.Transform}]{\sphinxcrossref{\sphinxcode{\sphinxupquote{Transform}}}}}, {\hyperref[\detokenize{deepdrr.geo:deepdrr.geo.core.HasLocation}]{\sphinxcrossref{\sphinxcode{\sphinxupquote{HasLocation}}}}}
\index{R (deepdrr.geo.core.FrameTransform property)@\spxentry{R}\spxextra{deepdrr.geo.core.FrameTransform property}}

\begin{fulllineitems}
\phantomsection\label{\detokenize{deepdrr.geo:deepdrr.geo.core.FrameTransform.R}}
\pysigstartsignatures
\pysigline{\sphinxbfcode{\sphinxupquote{property\DUrole{w,w}{  }}}\sphinxbfcode{\sphinxupquote{R}}}
\pysigstopsignatures
\end{fulllineitems}

\index{as\_quatpos() (deepdrr.geo.core.FrameTransform method)@\spxentry{as\_quatpos()}\spxextra{deepdrr.geo.core.FrameTransform method}}

\begin{fulllineitems}
\phantomsection\label{\detokenize{deepdrr.geo:deepdrr.geo.core.FrameTransform.as_quatpos}}
\pysigstartsignatures
\pysiglinewithargsret{\sphinxbfcode{\sphinxupquote{as\_quatpos}}}{}{{ $\rightarrow$ ndarray}}
\pysigstopsignatures
\sphinxAtStartPar
Return the transform as a quaternion and position.
\begin{quote}\begin{description}
\sphinxlineitem{Returns}
\sphinxAtStartPar
A 7\sphinxhyphen{}element array, with the first 4 elements being the quaternion, and the last 3 being the position.

\sphinxlineitem{Return type}
\sphinxAtStartPar
np.ndarray

\end{description}\end{quote}

\end{fulllineitems}

\index{data (deepdrr.geo.core.FrameTransform attribute)@\spxentry{data}\spxextra{deepdrr.geo.core.FrameTransform attribute}}

\begin{fulllineitems}
\phantomsection\label{\detokenize{deepdrr.geo:deepdrr.geo.core.FrameTransform.data}}
\pysigstartsignatures
\pysigline{\sphinxbfcode{\sphinxupquote{data}}\sphinxbfcode{\sphinxupquote{\DUrole{p,p}{:}\DUrole{w,w}{  }ndarray}}}
\pysigstopsignatures
\end{fulllineitems}

\index{dim (deepdrr.geo.core.FrameTransform property)@\spxentry{dim}\spxextra{deepdrr.geo.core.FrameTransform property}}

\begin{fulllineitems}
\phantomsection\label{\detokenize{deepdrr.geo:deepdrr.geo.core.FrameTransform.dim}}
\pysigstartsignatures
\pysigline{\sphinxbfcode{\sphinxupquote{property\DUrole{w,w}{  }}}\sphinxbfcode{\sphinxupquote{dim}}}
\pysigstopsignatures
\sphinxAtStartPar
The output dimension of the transformation.

\end{fulllineitems}

\index{from\_line\_segments() (deepdrr.geo.core.FrameTransform class method)@\spxentry{from\_line\_segments()}\spxextra{deepdrr.geo.core.FrameTransform class method}}

\begin{fulllineitems}
\phantomsection\label{\detokenize{deepdrr.geo:deepdrr.geo.core.FrameTransform.from_line_segments}}
\pysigstartsignatures
\pysiglinewithargsret{\sphinxbfcode{\sphinxupquote{classmethod\DUrole{w,w}{  }}}\sphinxbfcode{\sphinxupquote{from\_line\_segments}}}{\sphinxparam{\DUrole{n,n}{x\_B}\DUrole{p,p}{:}\DUrole{w,w}{  }\DUrole{n,n}{{\hyperref[\detokenize{deepdrr.geo:deepdrr.geo.core.Point3D}]{\sphinxcrossref{Point3D}}}}}\sphinxparamcomma \sphinxparam{\DUrole{n,n}{y\_B}\DUrole{p,p}{:}\DUrole{w,w}{  }\DUrole{n,n}{{\hyperref[\detokenize{deepdrr.geo:deepdrr.geo.core.Point3D}]{\sphinxcrossref{Point3D}}}}}\sphinxparamcomma \sphinxparam{\DUrole{n,n}{x\_A}\DUrole{p,p}{:}\DUrole{w,w}{  }\DUrole{n,n}{{\hyperref[\detokenize{deepdrr.geo:deepdrr.geo.core.Point3D}]{\sphinxcrossref{Point3D}}}}}\sphinxparamcomma \sphinxparam{\DUrole{n,n}{y\_A}\DUrole{p,p}{:}\DUrole{w,w}{  }\DUrole{n,n}{{\hyperref[\detokenize{deepdrr.geo:deepdrr.geo.core.Point3D}]{\sphinxcrossref{Point3D}}}}}}{{ $\rightarrow$ {\hyperref[\detokenize{deepdrr.geo:deepdrr.geo.core.FrameTransform}]{\sphinxcrossref{FrameTransform}}}}}
\pysigstopsignatures
\sphinxAtStartPar
Get the \sphinxtitleref{B\_from\_A} frame transform that aligns the line segments, given by endpoints.

\sphinxAtStartPar
Perfectly aligns the two line segments, so there is possibly some scaling.
\begin{quote}\begin{description}
\sphinxlineitem{Parameters}\begin{itemize}
\item {} 
\sphinxAtStartPar
\sphinxstyleliteralstrong{\sphinxupquote{x\_B}} ({\hyperref[\detokenize{deepdrr.geo:deepdrr.geo.core.Point3D}]{\sphinxcrossref{\sphinxstyleliteralemphasis{\sphinxupquote{Point3D}}}}}) \textendash{} The first endpoint, in frame B.

\item {} 
\sphinxAtStartPar
\sphinxstyleliteralstrong{\sphinxupquote{y\_B}} ({\hyperref[\detokenize{deepdrr.geo:deepdrr.geo.core.Point3D}]{\sphinxcrossref{\sphinxstyleliteralemphasis{\sphinxupquote{Point3D}}}}}) \textendash{} The second endpoint, in frame B.

\item {} 
\sphinxAtStartPar
\sphinxstyleliteralstrong{\sphinxupquote{x\_A}} ({\hyperref[\detokenize{deepdrr.geo:deepdrr.geo.core.Point3D}]{\sphinxcrossref{\sphinxstyleliteralemphasis{\sphinxupquote{Point3D}}}}}) \textendash{} The first endpoint, in frame A.

\item {} 
\sphinxAtStartPar
\sphinxstyleliteralstrong{\sphinxupquote{y\_A}} ({\hyperref[\detokenize{deepdrr.geo:deepdrr.geo.core.Point3D}]{\sphinxcrossref{\sphinxstyleliteralemphasis{\sphinxupquote{Point3D}}}}}) \textendash{} The second endpoint, in frame A.

\end{itemize}

\sphinxlineitem{Returns}
\sphinxAtStartPar
\begin{description}
\sphinxlineitem{A \sphinxtitleref{B\_from\_A} transform that aligns the points.}
\sphinxAtStartPar
Note that this is not unique, due to rotation about the axis between the points.

\end{description}


\sphinxlineitem{Return type}
\sphinxAtStartPar
{\hyperref[\detokenize{deepdrr.geo:deepdrr.geo.core.FrameTransform}]{\sphinxcrossref{FrameTransform}}}

\end{description}\end{quote}

\end{fulllineitems}

\index{from\_origin() (deepdrr.geo.core.FrameTransform class method)@\spxentry{from\_origin()}\spxextra{deepdrr.geo.core.FrameTransform class method}}

\begin{fulllineitems}
\phantomsection\label{\detokenize{deepdrr.geo:deepdrr.geo.core.FrameTransform.from_origin}}
\pysigstartsignatures
\pysiglinewithargsret{\sphinxbfcode{\sphinxupquote{classmethod\DUrole{w,w}{  }}}\sphinxbfcode{\sphinxupquote{from\_origin}}}{\sphinxparam{\DUrole{n,n}{origin}\DUrole{p,p}{:}\DUrole{w,w}{  }\DUrole{n,n}{{\hyperref[\detokenize{deepdrr.geo:deepdrr.geo.core.Point}]{\sphinxcrossref{Point}}}}}}{{ $\rightarrow$ {\hyperref[\detokenize{deepdrr.geo:deepdrr.geo.core.FrameTransform}]{\sphinxcrossref{FrameTransform}}}}}
\pysigstopsignatures
\sphinxAtStartPar
Make a transfrom to a frame knowing the origin.

\sphinxAtStartPar
Suppose \sphinxtitleref{origin} is point where frame \sphinxtitleref{B} has its origin, as a point
in frame \sphinxtitleref{A}. Make the \sphinxtitleref{B\_from\_A} transform.
This just negates \sphinxtitleref{origin}, but this is often counterintuitive.
\begin{quote}\begin{description}
\sphinxlineitem{Parameters}
\sphinxAtStartPar
\sphinxstyleliteralstrong{\sphinxupquote{origin}} ({\hyperref[\detokenize{deepdrr.geo:deepdrr.geo.core.Point}]{\sphinxcrossref{\sphinxstyleliteralemphasis{\sphinxupquote{Point}}}}}) \textendash{} origin of the target frame in the world frame

\sphinxlineitem{Returns}
\sphinxAtStartPar
the B\_from\_A transform.

\sphinxlineitem{Return type}
\sphinxAtStartPar
{\hyperref[\detokenize{deepdrr.geo:deepdrr.geo.core.FrameTransform}]{\sphinxcrossref{FrameTransform}}}

\end{description}\end{quote}

\end{fulllineitems}

\index{from\_pd() (deepdrr.geo.core.FrameTransform class method)@\spxentry{from\_pd()}\spxextra{deepdrr.geo.core.FrameTransform class method}}

\begin{fulllineitems}
\phantomsection\label{\detokenize{deepdrr.geo:deepdrr.geo.core.FrameTransform.from_pd}}
\pysigstartsignatures
\pysiglinewithargsret{\sphinxbfcode{\sphinxupquote{classmethod\DUrole{w,w}{  }}}\sphinxbfcode{\sphinxupquote{from\_pd}}}{\sphinxparam{\DUrole{n,n}{origin}\DUrole{p,p}{:}\DUrole{w,w}{  }\DUrole{n,n}{{\hyperref[\detokenize{deepdrr.geo:deepdrr.geo.core.Point3D}]{\sphinxcrossref{Point3D}}}}}\sphinxparamcomma \sphinxparam{\DUrole{n,n}{direction}\DUrole{p,p}{:}\DUrole{w,w}{  }\DUrole{n,n}{{\hyperref[\detokenize{deepdrr.geo:deepdrr.geo.core.Vector3D}]{\sphinxcrossref{Vector3D}}}}}\sphinxparamcomma \sphinxparam{\DUrole{n,n}{axis}\DUrole{p,p}{:}\DUrole{w,w}{  }\DUrole{n,n}{str\DUrole{w,w}{  }\DUrole{p,p}{|}\DUrole{w,w}{  }{\hyperref[\detokenize{deepdrr.geo:deepdrr.geo.core.Vector3D}]{\sphinxcrossref{Vector3D}}}}\DUrole{w,w}{  }\DUrole{o,o}{=}\DUrole{w,w}{  }\DUrole{default_value}{\textquotesingle{}z\textquotesingle{}}}}{{ $\rightarrow$ {\hyperref[\detokenize{deepdrr.geo:deepdrr.geo.core.FrameTransform}]{\sphinxcrossref{FrameTransform}}}}}
\pysigstopsignatures
\sphinxAtStartPar
Get the pose of the coordinate frame given its origin and direction of the given axis.
\begin{quote}\begin{description}
\sphinxlineitem{Parameters}\begin{itemize}
\item {} 
\sphinxAtStartPar
\sphinxstyleliteralstrong{\sphinxupquote{origin}} ({\hyperref[\detokenize{deepdrr.geo:deepdrr.geo.core.Point3D}]{\sphinxcrossref{\sphinxstyleliteralemphasis{\sphinxupquote{Point3D}}}}}) \textendash{} The origin of the frame.

\item {} 
\sphinxAtStartPar
\sphinxstyleliteralstrong{\sphinxupquote{direction}} ({\hyperref[\detokenize{deepdrr.geo:deepdrr.geo.core.Vector3D}]{\sphinxcrossref{\sphinxstyleliteralemphasis{\sphinxupquote{Vector3D}}}}}) \textendash{} The direction of the axis.

\item {} 
\sphinxAtStartPar
\sphinxstyleliteralstrong{\sphinxupquote{axis}} (\sphinxstyleliteralemphasis{\sphinxupquote{Union}}\sphinxstyleliteralemphasis{\sphinxupquote{{[}}}\sphinxstyleliteralemphasis{\sphinxupquote{str}}\sphinxstyleliteralemphasis{\sphinxupquote{, }}{\hyperref[\detokenize{deepdrr.geo:deepdrr.geo.core.Vector3D}]{\sphinxcrossref{\sphinxstyleliteralemphasis{\sphinxupquote{Vector3D}}}}}\sphinxstyleliteralemphasis{\sphinxupquote{{]}}}\sphinxstyleliteralemphasis{\sphinxupquote{, }}\sphinxstyleliteralemphasis{\sphinxupquote{optional}}) \textendash{} The axis to align with the direction. Defaults to “z”.

\end{itemize}

\end{description}\end{quote}

\end{fulllineitems}

\index{from\_point\_correspondence() (deepdrr.geo.core.FrameTransform class method)@\spxentry{from\_point\_correspondence()}\spxextra{deepdrr.geo.core.FrameTransform class method}}

\begin{fulllineitems}
\phantomsection\label{\detokenize{deepdrr.geo:deepdrr.geo.core.FrameTransform.from_point_correspondence}}
\pysigstartsignatures
\pysiglinewithargsret{\sphinxbfcode{\sphinxupquote{classmethod\DUrole{w,w}{  }}}\sphinxbfcode{\sphinxupquote{from\_point\_correspondence}}}{\sphinxparam{\DUrole{n,n}{points\_B}\DUrole{p,p}{:}\DUrole{w,w}{  }\DUrole{n,n}{List\DUrole{p,p}{{[}}{\hyperref[\detokenize{deepdrr.geo:deepdrr.geo.core.Point}]{\sphinxcrossref{Point}}}\DUrole{p,p}{{]}}\DUrole{w,w}{  }\DUrole{p,p}{|}\DUrole{w,w}{  }ndarray}}\sphinxparamcomma \sphinxparam{\DUrole{n,n}{points\_A}\DUrole{p,p}{:}\DUrole{w,w}{  }\DUrole{n,n}{List\DUrole{p,p}{{[}}{\hyperref[\detokenize{deepdrr.geo:deepdrr.geo.core.Point}]{\sphinxcrossref{Point}}}\DUrole{p,p}{{]}}\DUrole{w,w}{  }\DUrole{p,p}{|}\DUrole{w,w}{  }ndarray}}}{}
\pysigstopsignatures
\sphinxAtStartPar
Create a (rigid) frame transform from a known point correspondence.
\begin{quote}\begin{description}
\sphinxlineitem{Parameters}\begin{itemize}
\item {} 
\sphinxAtStartPar
\sphinxstyleliteralstrong{\sphinxupquote{points\_B}} \textendash{} a list of N corresponding points in the B frame.

\item {} 
\sphinxAtStartPar
\sphinxstyleliteralstrong{\sphinxupquote{points\_A}} \textendash{} a list of N points in the A frame (or an array with shape {[}N, 3{]}).

\end{itemize}

\sphinxlineitem{Returns}
\sphinxAtStartPar
\begin{description}
\sphinxlineitem{the \sphinxtitleref{B\_from\_A} transform that minimizes the mean squared distance}
\sphinxAtStartPar
between matching points.

\end{description}


\sphinxlineitem{Return type}
\sphinxAtStartPar
{\hyperref[\detokenize{deepdrr.geo:deepdrr.geo.core.FrameTransform}]{\sphinxcrossref{FrameTransform}}}

\end{description}\end{quote}

\end{fulllineitems}

\index{from\_pointdir() (deepdrr.geo.core.FrameTransform class method)@\spxentry{from\_pointdir()}\spxextra{deepdrr.geo.core.FrameTransform class method}}

\begin{fulllineitems}
\phantomsection\label{\detokenize{deepdrr.geo:deepdrr.geo.core.FrameTransform.from_pointdir}}
\pysigstartsignatures
\pysiglinewithargsret{\sphinxbfcode{\sphinxupquote{classmethod\DUrole{w,w}{  }}}\sphinxbfcode{\sphinxupquote{from\_pointdir}}}{\sphinxparam{\DUrole{o,o}{*}\DUrole{n,n}{args}}\sphinxparamcomma \sphinxparam{\DUrole{o,o}{**}\DUrole{n,n}{kwargs}}}{{ $\rightarrow$ {\hyperref[\detokenize{deepdrr.geo:deepdrr.geo.core.FrameTransform}]{\sphinxcrossref{FrameTransform}}}}}
\pysigstopsignatures
\sphinxAtStartPar
Alias for \sphinxtitleref{from\_pd}.

\end{fulllineitems}

\index{from\_points() (deepdrr.geo.core.FrameTransform class method)@\spxentry{from\_points()}\spxextra{deepdrr.geo.core.FrameTransform class method}}

\begin{fulllineitems}
\phantomsection\label{\detokenize{deepdrr.geo:deepdrr.geo.core.FrameTransform.from_points}}
\pysigstartsignatures
\pysiglinewithargsret{\sphinxbfcode{\sphinxupquote{classmethod\DUrole{w,w}{  }}}\sphinxbfcode{\sphinxupquote{from\_points}}}{\sphinxparam{\DUrole{n,n}{points\_B}\DUrole{p,p}{:}\DUrole{w,w}{  }\DUrole{n,n}{List\DUrole{p,p}{{[}}{\hyperref[\detokenize{deepdrr.geo:deepdrr.geo.core.Point3D}]{\sphinxcrossref{Point3D}}}\DUrole{p,p}{{]}}\DUrole{w,w}{  }\DUrole{p,p}{|}\DUrole{w,w}{  }ndarray}}\sphinxparamcomma \sphinxparam{\DUrole{n,n}{points\_A}\DUrole{p,p}{:}\DUrole{w,w}{  }\DUrole{n,n}{List\DUrole{p,p}{{[}}{\hyperref[\detokenize{deepdrr.geo:deepdrr.geo.core.Point3D}]{\sphinxcrossref{Point3D}}}\DUrole{p,p}{{]}}\DUrole{w,w}{  }\DUrole{p,p}{|}\DUrole{w,w}{  }ndarray}}\sphinxparamcomma \sphinxparam{\DUrole{n,n}{max\_iterations}\DUrole{p,p}{:}\DUrole{w,w}{  }\DUrole{n,n}{int}\DUrole{w,w}{  }\DUrole{o,o}{=}\DUrole{w,w}{  }\DUrole{default_value}{1000}}}{}
\pysigstopsignatures
\sphinxAtStartPar
Create a (rigid) frame transform from ICP between the two point clouds.
\begin{quote}\begin{description}
\sphinxlineitem{Parameters}\begin{itemize}
\item {} 
\sphinxAtStartPar
\sphinxstyleliteralstrong{\sphinxupquote{points\_B}} \textendash{} a list of M points in the B frame.

\item {} 
\sphinxAtStartPar
\sphinxstyleliteralstrong{\sphinxupquote{points\_A}} \textendash{} a list of N points in the A frame, corresponding to the same shape as \sphinxtitleref{points\_B}.

\end{itemize}

\sphinxlineitem{Returns}
\sphinxAtStartPar
\begin{description}
\sphinxlineitem{the \sphinxtitleref{B\_from\_A} transform that minimizes the mean squared distance}
\sphinxAtStartPar
between matching points.

\end{description}


\sphinxlineitem{Return type}
\sphinxAtStartPar
{\hyperref[\detokenize{deepdrr.geo:deepdrr.geo.core.FrameTransform}]{\sphinxcrossref{FrameTransform}}}

\end{description}\end{quote}

\end{fulllineitems}

\index{from\_quatpos() (deepdrr.geo.core.FrameTransform class method)@\spxentry{from\_quatpos()}\spxextra{deepdrr.geo.core.FrameTransform class method}}

\begin{fulllineitems}
\phantomsection\label{\detokenize{deepdrr.geo:deepdrr.geo.core.FrameTransform.from_quatpos}}
\pysigstartsignatures
\pysiglinewithargsret{\sphinxbfcode{\sphinxupquote{classmethod\DUrole{w,w}{  }}}\sphinxbfcode{\sphinxupquote{from\_quatpos}}}{\sphinxparam{\DUrole{n,n}{quatpos}\DUrole{p,p}{:}\DUrole{w,w}{  }\DUrole{n,n}{ndarray}}}{{ $\rightarrow$ {\hyperref[\detokenize{deepdrr.geo:deepdrr.geo.core.FrameTransform}]{\sphinxcrossref{FrameTransform}}}}}
\pysigstopsignatures
\sphinxAtStartPar
Create a transform from a quaternion and position.
\begin{quote}\begin{description}
\sphinxlineitem{Parameters}
\sphinxAtStartPar
\sphinxstyleliteralstrong{\sphinxupquote{quatpos}} (\sphinxstyleliteralemphasis{\sphinxupquote{np.ndarray}}) \textendash{} A 7\sphinxhyphen{}element array, with the first 4 elements being the quaternion, and the last 3 being the position.

\sphinxlineitem{Returns}
\sphinxAtStartPar
The transform.

\sphinxlineitem{Return type}
\sphinxAtStartPar
{\hyperref[\detokenize{deepdrr.geo:deepdrr.geo.core.FrameTransform}]{\sphinxcrossref{FrameTransform}}}

\end{description}\end{quote}

\end{fulllineitems}

\index{from\_rotation() (deepdrr.geo.core.FrameTransform class method)@\spxentry{from\_rotation()}\spxextra{deepdrr.geo.core.FrameTransform class method}}

\begin{fulllineitems}
\phantomsection\label{\detokenize{deepdrr.geo:deepdrr.geo.core.FrameTransform.from_rotation}}
\pysigstartsignatures
\pysiglinewithargsret{\sphinxbfcode{\sphinxupquote{classmethod\DUrole{w,w}{  }}}\sphinxbfcode{\sphinxupquote{from\_rotation}}}{\sphinxparam{\DUrole{n,n}{rotation}\DUrole{p,p}{:}\DUrole{w,w}{  }\DUrole{n,n}{{\hyperref[\detokenize{deepdrr.geo:deepdrr.geo.Rotation}]{\sphinxcrossref{Rotation}}}\DUrole{w,w}{  }\DUrole{p,p}{|}\DUrole{w,w}{  }ndarray}}}{{ $\rightarrow$ {\hyperref[\detokenize{deepdrr.geo:deepdrr.geo.core.FrameTransform}]{\sphinxcrossref{FrameTransform}}}}}
\pysigstopsignatures
\sphinxAtStartPar
Wrapper around from\_rt.

\end{fulllineitems}

\index{from\_rt() (deepdrr.geo.core.FrameTransform class method)@\spxentry{from\_rt()}\spxextra{deepdrr.geo.core.FrameTransform class method}}

\begin{fulllineitems}
\phantomsection\label{\detokenize{deepdrr.geo:deepdrr.geo.core.FrameTransform.from_rt}}
\pysigstartsignatures
\pysiglinewithargsret{\sphinxbfcode{\sphinxupquote{classmethod\DUrole{w,w}{  }}}\sphinxbfcode{\sphinxupquote{from\_rt}}}{\sphinxparam{\DUrole{n,n}{rotation}\DUrole{p,p}{:}\DUrole{w,w}{  }\DUrole{n,n}{{\hyperref[\detokenize{deepdrr.geo:deepdrr.geo.Rotation}]{\sphinxcrossref{Rotation}}}\DUrole{w,w}{  }\DUrole{p,p}{|}\DUrole{w,w}{  }ndarray\DUrole{w,w}{  }\DUrole{p,p}{|}\DUrole{w,w}{  }None}\DUrole{w,w}{  }\DUrole{o,o}{=}\DUrole{w,w}{  }\DUrole{default_value}{None}}\sphinxparamcomma \sphinxparam{\DUrole{n,n}{translation}\DUrole{p,p}{:}\DUrole{w,w}{  }\DUrole{n,n}{{\hyperref[\detokenize{deepdrr.geo:deepdrr.geo.core.Point3D}]{\sphinxcrossref{Point3D}}}\DUrole{w,w}{  }\DUrole{p,p}{|}\DUrole{w,w}{  }{\hyperref[\detokenize{deepdrr.geo:deepdrr.geo.core.Vector3D}]{\sphinxcrossref{Vector3D}}}\DUrole{w,w}{  }\DUrole{p,p}{|}\DUrole{w,w}{  }ndarray\DUrole{w,w}{  }\DUrole{p,p}{|}\DUrole{w,w}{  }None}\DUrole{w,w}{  }\DUrole{o,o}{=}\DUrole{w,w}{  }\DUrole{default_value}{None}}\sphinxparamcomma \sphinxparam{\DUrole{n,n}{dim}\DUrole{p,p}{:}\DUrole{w,w}{  }\DUrole{n,n}{int\DUrole{w,w}{  }\DUrole{p,p}{|}\DUrole{w,w}{  }None}\DUrole{w,w}{  }\DUrole{o,o}{=}\DUrole{w,w}{  }\DUrole{default_value}{None}}}{{ $\rightarrow$ {\hyperref[\detokenize{deepdrr.geo:deepdrr.geo.core.FrameTransform}]{\sphinxcrossref{FrameTransform}}}}}
\pysigstopsignatures
\sphinxAtStartPar
Make a frame translation from a rotation and translation, as {[}R,t{]}, where x’ = Rx + t.
\begin{quote}\begin{description}
\sphinxlineitem{Parameters}\begin{itemize}
\item {} 
\sphinxAtStartPar
\sphinxstyleliteralstrong{\sphinxupquote{rotation}} (\sphinxstyleliteralemphasis{\sphinxupquote{Optional}}\sphinxstyleliteralemphasis{\sphinxupquote{{[}}}\sphinxstyleliteralemphasis{\sphinxupquote{np.ndarray}}\sphinxstyleliteralemphasis{\sphinxupquote{{]}}}\sphinxstyleliteralemphasis{\sphinxupquote{, }}\sphinxstyleliteralemphasis{\sphinxupquote{optional}}) \textendash{} Rotation matrix. If None, uses the identity. Defaults to None.

\item {} 
\sphinxAtStartPar
\sphinxstyleliteralstrong{\sphinxupquote{translation}} \textendash{} Optional{[}Union{[}Point3D, np.ndarray{]}{]}: Translation of the transformation. If None, no translation. Defaults to None.

\item {} 
\sphinxAtStartPar
\sphinxstyleliteralstrong{\sphinxupquote{dim}} (\sphinxstyleliteralemphasis{\sphinxupquote{Optional}}\sphinxstyleliteralemphasis{\sphinxupquote{{[}}}\sphinxstyleliteralemphasis{\sphinxupquote{int}}\sphinxstyleliteralemphasis{\sphinxupquote{{]}}}\sphinxstyleliteralemphasis{\sphinxupquote{, }}\sphinxstyleliteralemphasis{\sphinxupquote{optional}}) \textendash{} Must be provided if both  Defaults to None.

\end{itemize}

\end{description}\end{quote}

\sphinxAtStartPar
If both args are None,
\begin{quote}\begin{description}
\sphinxlineitem{Returns}
\sphinxAtStartPar
The transformation \sphinxtitleref{F} such that \sphinxtitleref{F(x) = rotation @ x + translation}

\sphinxlineitem{Return type}
\sphinxAtStartPar
{\hyperref[\detokenize{deepdrr.geo:deepdrr.geo.core.FrameTransform}]{\sphinxcrossref{FrameTransform}}}

\end{description}\end{quote}

\end{fulllineitems}

\index{from\_scaling() (deepdrr.geo.core.FrameTransform class method)@\spxentry{from\_scaling()}\spxextra{deepdrr.geo.core.FrameTransform class method}}

\begin{fulllineitems}
\phantomsection\label{\detokenize{deepdrr.geo:deepdrr.geo.core.FrameTransform.from_scaling}}
\pysigstartsignatures
\pysiglinewithargsret{\sphinxbfcode{\sphinxupquote{classmethod\DUrole{w,w}{  }}}\sphinxbfcode{\sphinxupquote{from\_scaling}}}{\sphinxparam{\DUrole{n,n}{scaling}\DUrole{p,p}{:}\DUrole{w,w}{  }\DUrole{n,n}{int\DUrole{w,w}{  }\DUrole{p,p}{|}\DUrole{w,w}{  }float}}\sphinxparamcomma \sphinxparam{\DUrole{n,n}{dim}\DUrole{p,p}{:}\DUrole{w,w}{  }\DUrole{n,n}{int}\DUrole{w,w}{  }\DUrole{o,o}{=}\DUrole{w,w}{  }\DUrole{default_value}{3}}}{{ $\rightarrow$ {\hyperref[\detokenize{deepdrr.geo:deepdrr.geo.core.FrameTransform}]{\sphinxcrossref{FrameTransform}}}}}
\pysigstopsignatures
\sphinxAtStartPar
Create a frame based on scaling dimensions uniformly.
\begin{quote}\begin{description}
\sphinxlineitem{Parameters}\begin{itemize}
\item {} 
\sphinxAtStartPar
\sphinxstyleliteralstrong{\sphinxupquote{cls}} (\sphinxstyleliteralemphasis{\sphinxupquote{Type}}\sphinxstyleliteralemphasis{\sphinxupquote{{[}}}{\hyperref[\detokenize{deepdrr.geo:deepdrr.geo.core.FrameTransform}]{\sphinxcrossref{\sphinxstyleliteralemphasis{\sphinxupquote{FrameTransform}}}}}\sphinxstyleliteralemphasis{\sphinxupquote{{]}}}) \textendash{} the class.

\item {} 
\sphinxAtStartPar
\sphinxstyleliteralstrong{\sphinxupquote{dim}} (\sphinxstyleliteralemphasis{\sphinxupquote{int}}\sphinxstyleliteralemphasis{\sphinxupquote{, }}\sphinxstyleliteralemphasis{\sphinxupquote{optional}}) \textendash{} the dimension of the frame. Defaults to 3.

\end{itemize}

\sphinxlineitem{Return type}
\sphinxAtStartPar
{\hyperref[\detokenize{deepdrr.geo:deepdrr.geo.core.FrameTransform}]{\sphinxcrossref{FrameTransform}}}

\end{description}\end{quote}

\end{fulllineitems}

\index{from\_translation() (deepdrr.geo.core.FrameTransform class method)@\spxentry{from\_translation()}\spxextra{deepdrr.geo.core.FrameTransform class method}}

\begin{fulllineitems}
\phantomsection\label{\detokenize{deepdrr.geo:deepdrr.geo.core.FrameTransform.from_translation}}
\pysigstartsignatures
\pysiglinewithargsret{\sphinxbfcode{\sphinxupquote{classmethod\DUrole{w,w}{  }}}\sphinxbfcode{\sphinxupquote{from\_translation}}}{\sphinxparam{\DUrole{n,n}{translation}\DUrole{p,p}{:}\DUrole{w,w}{  }\DUrole{n,n}{ndarray}}}{{ $\rightarrow$ {\hyperref[\detokenize{deepdrr.geo:deepdrr.geo.core.FrameTransform}]{\sphinxcrossref{FrameTransform}}}}}
\pysigstopsignatures
\sphinxAtStartPar
Wrapper around from\_rt.

\end{fulllineitems}

\index{get\_point() (deepdrr.geo.core.FrameTransform method)@\spxentry{get\_point()}\spxextra{deepdrr.geo.core.FrameTransform method}}

\begin{fulllineitems}
\phantomsection\label{\detokenize{deepdrr.geo:deepdrr.geo.core.FrameTransform.get_point}}
\pysigstartsignatures
\pysiglinewithargsret{\sphinxbfcode{\sphinxupquote{get\_point}}}{\sphinxparam{\DUrole{n,n}{point}\DUrole{p,p}{:}\DUrole{w,w}{  }\DUrole{n,n}{{\hyperref[\detokenize{deepdrr.geo:deepdrr.geo.core.Point}]{\sphinxcrossref{Point}}}}}}{{ $\rightarrow$ {\hyperref[\detokenize{deepdrr.geo:deepdrr.geo.core.Point}]{\sphinxcrossref{Point}}}}}
\pysigstopsignatures
\sphinxAtStartPar
Transform a point.
\begin{quote}\begin{description}
\sphinxlineitem{Parameters}
\sphinxAtStartPar
\sphinxstyleliteralstrong{\sphinxupquote{point}} ({\hyperref[\detokenize{deepdrr.geo:deepdrr.geo.core.Point}]{\sphinxcrossref{\sphinxstyleliteralemphasis{\sphinxupquote{Point}}}}}) \textendash{} The point to transform.

\sphinxlineitem{Returns}
\sphinxAtStartPar
The transformed point.

\sphinxlineitem{Return type}
\sphinxAtStartPar
{\hyperref[\detokenize{deepdrr.geo:deepdrr.geo.core.Point}]{\sphinxcrossref{Point}}}

\end{description}\end{quote}

\end{fulllineitems}

\index{i (deepdrr.geo.core.FrameTransform property)@\spxentry{i}\spxextra{deepdrr.geo.core.FrameTransform property}}

\begin{fulllineitems}
\phantomsection\label{\detokenize{deepdrr.geo:deepdrr.geo.core.FrameTransform.i}}
\pysigstartsignatures
\pysigline{\sphinxbfcode{\sphinxupquote{property\DUrole{w,w}{  }}}\sphinxbfcode{\sphinxupquote{i}}\sphinxbfcode{\sphinxupquote{\DUrole{p,p}{:}\DUrole{w,w}{  }{\hyperref[\detokenize{deepdrr.geo:deepdrr.geo.core.Vector}]{\sphinxcrossref{Vector}}}}}}
\pysigstopsignatures
\end{fulllineitems}

\index{identity() (deepdrr.geo.core.FrameTransform class method)@\spxentry{identity()}\spxextra{deepdrr.geo.core.FrameTransform class method}}

\begin{fulllineitems}
\phantomsection\label{\detokenize{deepdrr.geo:deepdrr.geo.core.FrameTransform.identity}}
\pysigstartsignatures
\pysiglinewithargsret{\sphinxbfcode{\sphinxupquote{classmethod\DUrole{w,w}{  }}}\sphinxbfcode{\sphinxupquote{identity}}}{\sphinxparam{\DUrole{n,n}{dim}\DUrole{p,p}{:}\DUrole{w,w}{  }\DUrole{n,n}{int}\DUrole{w,w}{  }\DUrole{o,o}{=}\DUrole{w,w}{  }\DUrole{default_value}{3}}}{{ $\rightarrow$ {\hyperref[\detokenize{deepdrr.geo:deepdrr.geo.core.FrameTransform}]{\sphinxcrossref{FrameTransform}}}}}
\pysigstopsignatures
\sphinxAtStartPar
Get the identity FrameTransform.

\end{fulllineitems}

\index{inv (deepdrr.geo.core.FrameTransform property)@\spxentry{inv}\spxextra{deepdrr.geo.core.FrameTransform property}}

\begin{fulllineitems}
\phantomsection\label{\detokenize{deepdrr.geo:deepdrr.geo.core.FrameTransform.inv}}
\pysigstartsignatures
\pysigline{\sphinxbfcode{\sphinxupquote{property\DUrole{w,w}{  }}}\sphinxbfcode{\sphinxupquote{inv}}}
\pysigstopsignatures
\sphinxAtStartPar
Get the inverse of the Transform.
\begin{quote}\begin{description}
\sphinxlineitem{Returns}
\sphinxAtStartPar
a Transform (or subclass) that is well\sphinxhyphen{}defined as the inverse of this transform.

\sphinxlineitem{Return type}
\sphinxAtStartPar
({\hyperref[\detokenize{deepdrr.geo:deepdrr.geo.core.Transform}]{\sphinxcrossref{Transform}}})

\sphinxlineitem{Raises}
\sphinxAtStartPar
\sphinxstyleliteralstrong{\sphinxupquote{NotImplementedError}} \textendash{} if \_inv is None and method is not overriden.

\end{description}\end{quote}

\end{fulllineitems}

\index{j (deepdrr.geo.core.FrameTransform property)@\spxentry{j}\spxextra{deepdrr.geo.core.FrameTransform property}}

\begin{fulllineitems}
\phantomsection\label{\detokenize{deepdrr.geo:deepdrr.geo.core.FrameTransform.j}}
\pysigstartsignatures
\pysigline{\sphinxbfcode{\sphinxupquote{property\DUrole{w,w}{  }}}\sphinxbfcode{\sphinxupquote{j}}\sphinxbfcode{\sphinxupquote{\DUrole{p,p}{:}\DUrole{w,w}{  }{\hyperref[\detokenize{deepdrr.geo:deepdrr.geo.core.Vector}]{\sphinxcrossref{Vector}}}}}}
\pysigstopsignatures
\end{fulllineitems}

\index{k (deepdrr.geo.core.FrameTransform property)@\spxentry{k}\spxextra{deepdrr.geo.core.FrameTransform property}}

\begin{fulllineitems}
\phantomsection\label{\detokenize{deepdrr.geo:deepdrr.geo.core.FrameTransform.k}}
\pysigstartsignatures
\pysigline{\sphinxbfcode{\sphinxupquote{property\DUrole{w,w}{  }}}\sphinxbfcode{\sphinxupquote{k}}\sphinxbfcode{\sphinxupquote{\DUrole{p,p}{:}\DUrole{w,w}{  }{\hyperref[\detokenize{deepdrr.geo:deepdrr.geo.core.Vector}]{\sphinxcrossref{Vector}}}}}}
\pysigstopsignatures
\end{fulllineitems}

\index{load() (deepdrr.geo.core.FrameTransform class method)@\spxentry{load()}\spxextra{deepdrr.geo.core.FrameTransform class method}}

\begin{fulllineitems}
\phantomsection\label{\detokenize{deepdrr.geo:deepdrr.geo.core.FrameTransform.load}}
\pysigstartsignatures
\pysiglinewithargsret{\sphinxbfcode{\sphinxupquote{classmethod\DUrole{w,w}{  }}}\sphinxbfcode{\sphinxupquote{load}}}{\sphinxparam{\DUrole{n,n}{path}\DUrole{p,p}{:}\DUrole{w,w}{  }\DUrole{n,n}{str\DUrole{w,w}{  }\DUrole{p,p}{|}\DUrole{w,w}{  }Path}}}{{ $\rightarrow$ None}}
\pysigstopsignatures
\sphinxAtStartPar
Load the transform from a file.
\begin{quote}\begin{description}
\sphinxlineitem{Parameters}
\sphinxAtStartPar
\sphinxstyleliteralstrong{\sphinxupquote{path}} (\sphinxstyleliteralemphasis{\sphinxupquote{Union}}\sphinxstyleliteralemphasis{\sphinxupquote{{[}}}\sphinxstyleliteralemphasis{\sphinxupquote{str}}\sphinxstyleliteralemphasis{\sphinxupquote{, }}\sphinxstyleliteralemphasis{\sphinxupquote{Path}}\sphinxstyleliteralemphasis{\sphinxupquote{{]}}}) \textendash{} path to load the transform from.

\end{description}\end{quote}

\end{fulllineitems}

\index{load\_txt() (deepdrr.geo.core.FrameTransform class method)@\spxentry{load\_txt()}\spxextra{deepdrr.geo.core.FrameTransform class method}}

\begin{fulllineitems}
\phantomsection\label{\detokenize{deepdrr.geo:deepdrr.geo.core.FrameTransform.load_txt}}
\pysigstartsignatures
\pysiglinewithargsret{\sphinxbfcode{\sphinxupquote{classmethod\DUrole{w,w}{  }}}\sphinxbfcode{\sphinxupquote{load\_txt}}}{\sphinxparam{\DUrole{n,n}{path}\DUrole{p,p}{:}\DUrole{w,w}{  }\DUrole{n,n}{str\DUrole{w,w}{  }\DUrole{p,p}{|}\DUrole{w,w}{  }Path}}}{{ $\rightarrow$ {\hyperref[\detokenize{deepdrr.geo:deepdrr.geo.core.FrameTransform}]{\sphinxcrossref{FrameTransform}}}}}
\pysigstopsignatures
\sphinxAtStartPar
Load a transform from a text file.
\begin{quote}\begin{description}
\sphinxlineitem{Parameters}
\sphinxAtStartPar
\sphinxstyleliteralstrong{\sphinxupquote{path}} (\sphinxstyleliteralemphasis{\sphinxupquote{Union}}\sphinxstyleliteralemphasis{\sphinxupquote{{[}}}\sphinxstyleliteralemphasis{\sphinxupquote{str}}\sphinxstyleliteralemphasis{\sphinxupquote{, }}\sphinxstyleliteralemphasis{\sphinxupquote{Path}}\sphinxstyleliteralemphasis{\sphinxupquote{{]}}}) \textendash{} path to load the transform from.

\sphinxlineitem{Returns}
\sphinxAtStartPar
the loaded transform.

\sphinxlineitem{Return type}
\sphinxAtStartPar
{\hyperref[\detokenize{deepdrr.geo:deepdrr.geo.core.FrameTransform}]{\sphinxcrossref{FrameTransform}}}

\end{description}\end{quote}

\end{fulllineitems}

\index{o (deepdrr.geo.core.FrameTransform property)@\spxentry{o}\spxextra{deepdrr.geo.core.FrameTransform property}}

\begin{fulllineitems}
\phantomsection\label{\detokenize{deepdrr.geo:deepdrr.geo.core.FrameTransform.o}}
\pysigstartsignatures
\pysigline{\sphinxbfcode{\sphinxupquote{property\DUrole{w,w}{  }}}\sphinxbfcode{\sphinxupquote{o}}}
\pysigstopsignatures
\sphinxAtStartPar
If this is the A\_from\_B transform, return the origin of frame B in frame A.

\end{fulllineitems}

\index{save() (deepdrr.geo.core.FrameTransform method)@\spxentry{save()}\spxextra{deepdrr.geo.core.FrameTransform method}}

\begin{fulllineitems}
\phantomsection\label{\detokenize{deepdrr.geo:deepdrr.geo.core.FrameTransform.save}}
\pysigstartsignatures
\pysiglinewithargsret{\sphinxbfcode{\sphinxupquote{save}}}{\sphinxparam{\DUrole{n,n}{path}\DUrole{p,p}{:}\DUrole{w,w}{  }\DUrole{n,n}{str\DUrole{w,w}{  }\DUrole{p,p}{|}\DUrole{w,w}{  }Path}}}{{ $\rightarrow$ None}}
\pysigstopsignatures
\sphinxAtStartPar
Save the transform to a file.
\begin{quote}\begin{description}
\sphinxlineitem{Parameters}
\sphinxAtStartPar
\sphinxstyleliteralstrong{\sphinxupquote{path}} (\sphinxstyleliteralemphasis{\sphinxupquote{Union}}\sphinxstyleliteralemphasis{\sphinxupquote{{[}}}\sphinxstyleliteralemphasis{\sphinxupquote{str}}\sphinxstyleliteralemphasis{\sphinxupquote{, }}\sphinxstyleliteralemphasis{\sphinxupquote{Path}}\sphinxstyleliteralemphasis{\sphinxupquote{{]}}}) \textendash{} path to save the transform to.

\end{description}\end{quote}

\end{fulllineitems}

\index{save\_txt() (deepdrr.geo.core.FrameTransform method)@\spxentry{save\_txt()}\spxextra{deepdrr.geo.core.FrameTransform method}}

\begin{fulllineitems}
\phantomsection\label{\detokenize{deepdrr.geo:deepdrr.geo.core.FrameTransform.save_txt}}
\pysigstartsignatures
\pysiglinewithargsret{\sphinxbfcode{\sphinxupquote{save\_txt}}}{\sphinxparam{\DUrole{n,n}{path}\DUrole{p,p}{:}\DUrole{w,w}{  }\DUrole{n,n}{str\DUrole{w,w}{  }\DUrole{p,p}{|}\DUrole{w,w}{  }Path}}}{{ $\rightarrow$ None}}
\pysigstopsignatures
\sphinxAtStartPar
Save the transform to a text file.
\begin{quote}\begin{description}
\sphinxlineitem{Parameters}
\sphinxAtStartPar
\sphinxstyleliteralstrong{\sphinxupquote{path}} (\sphinxstyleliteralemphasis{\sphinxupquote{Union}}\sphinxstyleliteralemphasis{\sphinxupquote{{[}}}\sphinxstyleliteralemphasis{\sphinxupquote{str}}\sphinxstyleliteralemphasis{\sphinxupquote{, }}\sphinxstyleliteralemphasis{\sphinxupquote{Path}}\sphinxstyleliteralemphasis{\sphinxupquote{{]}}}) \textendash{} path to save the transform to.

\end{description}\end{quote}

\end{fulllineitems}

\index{t (deepdrr.geo.core.FrameTransform property)@\spxentry{t}\spxextra{deepdrr.geo.core.FrameTransform property}}

\begin{fulllineitems}
\phantomsection\label{\detokenize{deepdrr.geo:deepdrr.geo.core.FrameTransform.t}}
\pysigstartsignatures
\pysigline{\sphinxbfcode{\sphinxupquote{property\DUrole{w,w}{  }}}\sphinxbfcode{\sphinxupquote{t}}}
\pysigstopsignatures
\end{fulllineitems}

\index{toarray() (deepdrr.geo.core.FrameTransform method)@\spxentry{toarray()}\spxextra{deepdrr.geo.core.FrameTransform method}}

\begin{fulllineitems}
\phantomsection\label{\detokenize{deepdrr.geo:deepdrr.geo.core.FrameTransform.toarray}}
\pysigstartsignatures
\pysiglinewithargsret{\sphinxbfcode{\sphinxupquote{toarray}}}{}{}
\pysigstopsignatures
\sphinxAtStartPar
Return the transform as a 3x4 numpy array.

\sphinxAtStartPar
This is different from calling np.array() on the transform, which returns a 4x4 array.

\end{fulllineitems}

\index{tostring() (deepdrr.geo.core.FrameTransform method)@\spxentry{tostring()}\spxextra{deepdrr.geo.core.FrameTransform method}}

\begin{fulllineitems}
\phantomsection\label{\detokenize{deepdrr.geo:deepdrr.geo.core.FrameTransform.tostring}}
\pysigstartsignatures
\pysiglinewithargsret{\sphinxbfcode{\sphinxupquote{tostring}}}{}{}
\pysigstopsignatures
\end{fulllineitems}

\index{transform\_points() (deepdrr.geo.core.FrameTransform method)@\spxentry{transform\_points()}\spxextra{deepdrr.geo.core.FrameTransform method}}

\begin{fulllineitems}
\phantomsection\label{\detokenize{deepdrr.geo:deepdrr.geo.core.FrameTransform.transform_points}}
\pysigstartsignatures
\pysiglinewithargsret{\sphinxbfcode{\sphinxupquote{transform\_points}}}{\sphinxparam{\DUrole{n,n}{points}\DUrole{p,p}{:}\DUrole{w,w}{  }\DUrole{n,n}{ndarray}}}{{ $\rightarrow$ ndarray}}
\pysigstopsignatures
\sphinxAtStartPar
Transform a set of points.
\begin{quote}\begin{description}
\sphinxlineitem{Parameters}
\sphinxAtStartPar
\sphinxstyleliteralstrong{\sphinxupquote{points}} (\sphinxstyleliteralemphasis{\sphinxupquote{np.ndarray}}) \textendash{} {[}N, D{]} array of nonhomogeneous points.

\sphinxlineitem{Returns}
\sphinxAtStartPar
The transformed points.

\sphinxlineitem{Return type}
\sphinxAtStartPar
np.ndarray

\end{description}\end{quote}

\end{fulllineitems}


\end{fulllineitems}

\index{HasDirection (class in deepdrr.geo.core)@\spxentry{HasDirection}\spxextra{class in deepdrr.geo.core}}

\begin{fulllineitems}
\phantomsection\label{\detokenize{deepdrr.geo:deepdrr.geo.core.HasDirection}}
\pysigstartsignatures
\pysiglinewithargsret{\sphinxbfcode{\sphinxupquote{class\DUrole{w,w}{  }}}\sphinxcode{\sphinxupquote{deepdrr.geo.core.}}\sphinxbfcode{\sphinxupquote{HasDirection}}}{\sphinxparam{\DUrole{n,n}{data}\DUrole{p,p}{:}\DUrole{w,w}{  }\DUrole{n,n}{ndarray}}}{}
\pysigstopsignatures
\sphinxAtStartPar
Bases: {\hyperref[\detokenize{deepdrr.geo:deepdrr.geo.core.Primitive}]{\sphinxcrossref{\sphinxcode{\sphinxupquote{Primitive}}}}}
\index{angle() (deepdrr.geo.core.HasDirection method)@\spxentry{angle()}\spxextra{deepdrr.geo.core.HasDirection method}}

\begin{fulllineitems}
\phantomsection\label{\detokenize{deepdrr.geo:deepdrr.geo.core.HasDirection.angle}}
\pysigstartsignatures
\pysiglinewithargsret{\sphinxbfcode{\sphinxupquote{angle}}}{\sphinxparam{\DUrole{n,n}{other}\DUrole{p,p}{:}\DUrole{w,w}{  }\DUrole{n,n}{{\hyperref[\detokenize{deepdrr.geo:deepdrr.geo.core.HasDirection}]{\sphinxcrossref{HasDirection}}}}}}{{ $\rightarrow$ float}}
\pysigstopsignatures
\sphinxAtStartPar
Get the angle between self and other in radians.
\begin{quote}\begin{description}
\sphinxlineitem{Parameters}
\sphinxAtStartPar
\sphinxstyleliteralstrong{\sphinxupquote{other}} ({\hyperref[\detokenize{deepdrr.geo:deepdrr.geo.core.Vector}]{\sphinxcrossref{\sphinxstyleliteralemphasis{\sphinxupquote{Vector}}}}}) \textendash{} the other vector.

\sphinxlineitem{Returns}
\sphinxAtStartPar
the angle between self and other in radians.

\sphinxlineitem{Return type}
\sphinxAtStartPar
float

\end{description}\end{quote}

\end{fulllineitems}

\index{cosine\_distance() (deepdrr.geo.core.HasDirection method)@\spxentry{cosine\_distance()}\spxextra{deepdrr.geo.core.HasDirection method}}

\begin{fulllineitems}
\phantomsection\label{\detokenize{deepdrr.geo:deepdrr.geo.core.HasDirection.cosine_distance}}
\pysigstartsignatures
\pysiglinewithargsret{\sphinxbfcode{\sphinxupquote{cosine\_distance}}}{\sphinxparam{\DUrole{n,n}{other}\DUrole{p,p}{:}\DUrole{w,w}{  }\DUrole{n,n}{{\hyperref[\detokenize{deepdrr.geo:deepdrr.geo.core.Vector}]{\sphinxcrossref{Vector}}}}}}{{ $\rightarrow$ float}}
\pysigstopsignatures
\sphinxAtStartPar
Get the cosine distance between the angles.
\begin{quote}\begin{description}
\sphinxlineitem{Parameters}
\sphinxAtStartPar
\sphinxstyleliteralstrong{\sphinxupquote{other}} ({\hyperref[\detokenize{deepdrr.geo:deepdrr.geo.core.Vector}]{\sphinxcrossref{\sphinxstyleliteralemphasis{\sphinxupquote{Vector}}}}}) \textendash{} the other vector.

\sphinxlineitem{Returns}
\sphinxAtStartPar
\sphinxtitleref{1 \sphinxhyphen{} cos(angle)}, where \sphinxtitleref{angle} is between self and other.

\sphinxlineitem{Return type}
\sphinxAtStartPar
float

\end{description}\end{quote}

\end{fulllineitems}

\index{data (deepdrr.geo.core.HasDirection attribute)@\spxentry{data}\spxextra{deepdrr.geo.core.HasDirection attribute}}

\begin{fulllineitems}
\phantomsection\label{\detokenize{deepdrr.geo:deepdrr.geo.core.HasDirection.data}}
\pysigstartsignatures
\pysigline{\sphinxbfcode{\sphinxupquote{data}}\sphinxbfcode{\sphinxupquote{\DUrole{p,p}{:}\DUrole{w,w}{  }ndarray}}}
\pysigstopsignatures
\end{fulllineitems}

\index{get\_direction() (deepdrr.geo.core.HasDirection method)@\spxentry{get\_direction()}\spxextra{deepdrr.geo.core.HasDirection method}}

\begin{fulllineitems}
\phantomsection\label{\detokenize{deepdrr.geo:deepdrr.geo.core.HasDirection.get_direction}}
\pysigstartsignatures
\pysiglinewithargsret{\sphinxbfcode{\sphinxupquote{abstract\DUrole{w,w}{  }}}\sphinxbfcode{\sphinxupquote{get\_direction}}}{}{{ $\rightarrow$ {\hyperref[\detokenize{deepdrr.geo:deepdrr.geo.core.Vector}]{\sphinxcrossref{Vector}}}}}
\pysigstopsignatures
\sphinxAtStartPar
Get the direction associated with the object.
\begin{quote}\begin{description}
\sphinxlineitem{Returns}
\sphinxAtStartPar
the direction of the object.

\sphinxlineitem{Return type}
\sphinxAtStartPar
{\hyperref[\detokenize{deepdrr.geo:deepdrr.geo.core.Vector}]{\sphinxcrossref{Vector}}}

\end{description}\end{quote}

\end{fulllineitems}

\index{rotfrom() (deepdrr.geo.core.HasDirection method)@\spxentry{rotfrom()}\spxextra{deepdrr.geo.core.HasDirection method}}

\begin{fulllineitems}
\phantomsection\label{\detokenize{deepdrr.geo:deepdrr.geo.core.HasDirection.rotfrom}}
\pysigstartsignatures
\pysiglinewithargsret{\sphinxbfcode{\sphinxupquote{rotfrom}}}{\sphinxparam{\DUrole{n,n}{other}\DUrole{p,p}{:}\DUrole{w,w}{  }\DUrole{n,n}{{\hyperref[\detokenize{deepdrr.geo:deepdrr.geo.core.HasDirection}]{\sphinxcrossref{HasDirection}}}}}}{{ $\rightarrow$ {\hyperref[\detokenize{deepdrr.geo:deepdrr.geo.core.FrameTransform}]{\sphinxcrossref{FrameTransform}}}}}
\pysigstopsignatures
\sphinxAtStartPar
Get the rotation such that \sphinxtitleref{self = self.rotfrom(other) @ other}.

\sphinxAtStartPar
NOTE: not tested with 2D vectors.
\begin{quote}\begin{description}
\sphinxlineitem{Parameters}
\sphinxAtStartPar
\sphinxstyleliteralstrong{\sphinxupquote{other}} ({\hyperref[\detokenize{deepdrr.geo:deepdrr.geo.core.Vector}]{\sphinxcrossref{\sphinxstyleliteralemphasis{\sphinxupquote{Vector}}}}}) \textendash{} the vector to rotate to.

\sphinxlineitem{Returns}
\sphinxAtStartPar
the rotation that rotates other to self.

\sphinxlineitem{Return type}
\sphinxAtStartPar
{\hyperref[\detokenize{deepdrr.geo:deepdrr.geo.core.FrameTransform}]{\sphinxcrossref{FrameTransform}}}

\end{description}\end{quote}

\end{fulllineitems}


\end{fulllineitems}

\index{HasLocation (class in deepdrr.geo.core)@\spxentry{HasLocation}\spxextra{class in deepdrr.geo.core}}

\begin{fulllineitems}
\phantomsection\label{\detokenize{deepdrr.geo:deepdrr.geo.core.HasLocation}}
\pysigstartsignatures
\pysiglinewithargsret{\sphinxbfcode{\sphinxupquote{class\DUrole{w,w}{  }}}\sphinxcode{\sphinxupquote{deepdrr.geo.core.}}\sphinxbfcode{\sphinxupquote{HasLocation}}}{\sphinxparam{\DUrole{n,n}{data}\DUrole{p,p}{:}\DUrole{w,w}{  }\DUrole{n,n}{ndarray}}}{}
\pysigstopsignatures
\sphinxAtStartPar
Bases: {\hyperref[\detokenize{deepdrr.geo:deepdrr.geo.core.Primitive}]{\sphinxcrossref{\sphinxcode{\sphinxupquote{Primitive}}}}}
\index{data (deepdrr.geo.core.HasLocation attribute)@\spxentry{data}\spxextra{deepdrr.geo.core.HasLocation attribute}}

\begin{fulllineitems}
\phantomsection\label{\detokenize{deepdrr.geo:deepdrr.geo.core.HasLocation.data}}
\pysigstartsignatures
\pysigline{\sphinxbfcode{\sphinxupquote{data}}\sphinxbfcode{\sphinxupquote{\DUrole{p,p}{:}\DUrole{w,w}{  }ndarray}}}
\pysigstopsignatures
\end{fulllineitems}

\index{get\_point() (deepdrr.geo.core.HasLocation method)@\spxentry{get\_point()}\spxextra{deepdrr.geo.core.HasLocation method}}

\begin{fulllineitems}
\phantomsection\label{\detokenize{deepdrr.geo:deepdrr.geo.core.HasLocation.get_point}}
\pysigstartsignatures
\pysiglinewithargsret{\sphinxbfcode{\sphinxupquote{abstract\DUrole{w,w}{  }}}\sphinxbfcode{\sphinxupquote{get\_point}}}{}{{ $\rightarrow$ {\hyperref[\detokenize{deepdrr.geo:deepdrr.geo.core.Point}]{\sphinxcrossref{Point}}}}}
\pysigstopsignatures
\sphinxAtStartPar
Get the location of the object.
\begin{quote}\begin{description}
\sphinxlineitem{Returns}
\sphinxAtStartPar
the location of the object.

\sphinxlineitem{Return type}
\sphinxAtStartPar
{\hyperref[\detokenize{deepdrr.geo:deepdrr.geo.core.Point}]{\sphinxcrossref{Point}}}

\end{description}\end{quote}

\end{fulllineitems}


\end{fulllineitems}

\index{HasLocationAndDirection (class in deepdrr.geo.core)@\spxentry{HasLocationAndDirection}\spxextra{class in deepdrr.geo.core}}

\begin{fulllineitems}
\phantomsection\label{\detokenize{deepdrr.geo:deepdrr.geo.core.HasLocationAndDirection}}
\pysigstartsignatures
\pysiglinewithargsret{\sphinxbfcode{\sphinxupquote{class\DUrole{w,w}{  }}}\sphinxcode{\sphinxupquote{deepdrr.geo.core.}}\sphinxbfcode{\sphinxupquote{HasLocationAndDirection}}}{\sphinxparam{\DUrole{n,n}{data}\DUrole{p,p}{:}\DUrole{w,w}{  }\DUrole{n,n}{ndarray}}}{}
\pysigstopsignatures
\sphinxAtStartPar
Bases: {\hyperref[\detokenize{deepdrr.geo:deepdrr.geo.core.HasLocation}]{\sphinxcrossref{\sphinxcode{\sphinxupquote{HasLocation}}}}}, {\hyperref[\detokenize{deepdrr.geo:deepdrr.geo.core.HasDirection}]{\sphinxcrossref{\sphinxcode{\sphinxupquote{HasDirection}}}}}
\index{data (deepdrr.geo.core.HasLocationAndDirection attribute)@\spxentry{data}\spxextra{deepdrr.geo.core.HasLocationAndDirection attribute}}

\begin{fulllineitems}
\phantomsection\label{\detokenize{deepdrr.geo:deepdrr.geo.core.HasLocationAndDirection.data}}
\pysigstartsignatures
\pysigline{\sphinxbfcode{\sphinxupquote{data}}\sphinxbfcode{\sphinxupquote{\DUrole{p,p}{:}\DUrole{w,w}{  }ndarray}}}
\pysigstopsignatures
\end{fulllineitems}

\index{from\_point\_direction() (deepdrr.geo.core.HasLocationAndDirection class method)@\spxentry{from\_point\_direction()}\spxextra{deepdrr.geo.core.HasLocationAndDirection class method}}

\begin{fulllineitems}
\phantomsection\label{\detokenize{deepdrr.geo:deepdrr.geo.core.HasLocationAndDirection.from_point_direction}}
\pysigstartsignatures
\pysiglinewithargsret{\sphinxbfcode{\sphinxupquote{abstract\DUrole{w,w}{  }classmethod\DUrole{w,w}{  }}}\sphinxbfcode{\sphinxupquote{from\_point\_direction}}}{\sphinxparam{\DUrole{n,n}{point}\DUrole{p,p}{:}\DUrole{w,w}{  }\DUrole{n,n}{{\hyperref[\detokenize{deepdrr.geo:deepdrr.geo.core.Point}]{\sphinxcrossref{Point}}}}}\sphinxparamcomma \sphinxparam{\DUrole{n,n}{direction}\DUrole{p,p}{:}\DUrole{w,w}{  }\DUrole{n,n}{{\hyperref[\detokenize{deepdrr.geo:deepdrr.geo.core.Vector}]{\sphinxcrossref{Vector}}}}}}{{ $\rightarrow$ Self}}
\pysigstopsignatures
\sphinxAtStartPar
Create an object from a point and a direction.
\begin{quote}\begin{description}
\sphinxlineitem{Parameters}\begin{itemize}
\item {} 
\sphinxAtStartPar
\sphinxstyleliteralstrong{\sphinxupquote{point}} ({\hyperref[\detokenize{deepdrr.geo:deepdrr.geo.core.Point}]{\sphinxcrossref{\sphinxstyleliteralemphasis{\sphinxupquote{Point}}}}}) \textendash{} the point.

\item {} 
\sphinxAtStartPar
\sphinxstyleliteralstrong{\sphinxupquote{direction}} ({\hyperref[\detokenize{deepdrr.geo:deepdrr.geo.core.Vector}]{\sphinxcrossref{\sphinxstyleliteralemphasis{\sphinxupquote{Vector}}}}}) \textendash{} the direction.

\end{itemize}

\sphinxlineitem{Returns}
\sphinxAtStartPar
the object.

\sphinxlineitem{Return type}
\sphinxAtStartPar
Self

\end{description}\end{quote}

\end{fulllineitems}


\end{fulllineitems}

\index{HasProjection (class in deepdrr.geo.core)@\spxentry{HasProjection}\spxextra{class in deepdrr.geo.core}}

\begin{fulllineitems}
\phantomsection\label{\detokenize{deepdrr.geo:deepdrr.geo.core.HasProjection}}
\pysigstartsignatures
\pysigline{\sphinxbfcode{\sphinxupquote{class\DUrole{w,w}{  }}}\sphinxcode{\sphinxupquote{deepdrr.geo.core.}}\sphinxbfcode{\sphinxupquote{HasProjection}}}
\pysigstopsignatures
\sphinxAtStartPar
Bases: \sphinxcode{\sphinxupquote{ABC}}
\index{projection\_type() (deepdrr.geo.core.HasProjection class method)@\spxentry{projection\_type()}\spxextra{deepdrr.geo.core.HasProjection class method}}

\begin{fulllineitems}
\phantomsection\label{\detokenize{deepdrr.geo:deepdrr.geo.core.HasProjection.projection_type}}
\pysigstartsignatures
\pysiglinewithargsret{\sphinxbfcode{\sphinxupquote{abstract\DUrole{w,w}{  }classmethod\DUrole{w,w}{  }}}\sphinxbfcode{\sphinxupquote{projection\_type}}}{}{{ $\rightarrow$ Type\DUrole{p,p}{{[}}{\hyperref[\detokenize{deepdrr.geo:deepdrr.geo.core.Primitive}]{\sphinxcrossref{Primitive}}}\DUrole{p,p}{{]}}}}
\pysigstopsignatures
\sphinxAtStartPar
Get the type of the projection of the object.
\begin{quote}\begin{description}
\sphinxlineitem{Returns}
\sphinxAtStartPar
the type of the projection of the object.

\sphinxlineitem{Return type}
\sphinxAtStartPar
Type{[}{\hyperref[\detokenize{deepdrr.geo:deepdrr.geo.core.Primitive}]{\sphinxcrossref{Primitive}}}{]}

\end{description}\end{quote}

\end{fulllineitems}


\end{fulllineitems}

\index{HomogeneousObject (class in deepdrr.geo.core)@\spxentry{HomogeneousObject}\spxextra{class in deepdrr.geo.core}}

\begin{fulllineitems}
\phantomsection\label{\detokenize{deepdrr.geo:deepdrr.geo.core.HomogeneousObject}}
\pysigstartsignatures
\pysiglinewithargsret{\sphinxbfcode{\sphinxupquote{class\DUrole{w,w}{  }}}\sphinxcode{\sphinxupquote{deepdrr.geo.core.}}\sphinxbfcode{\sphinxupquote{HomogeneousObject}}}{\sphinxparam{\DUrole{n,n}{data}\DUrole{p,p}{:}\DUrole{w,w}{  }\DUrole{n,n}{ndarray}}}{}
\pysigstopsignatures
\sphinxAtStartPar
Bases: \sphinxcode{\sphinxupquote{ABC}}

\sphinxAtStartPar
Any of the objects that rely on homogeneous transforms, all of which wrap a single array called \sphinxtitleref{data}.
\index{copy() (deepdrr.geo.core.HomogeneousObject method)@\spxentry{copy()}\spxextra{deepdrr.geo.core.HomogeneousObject method}}

\begin{fulllineitems}
\phantomsection\label{\detokenize{deepdrr.geo:deepdrr.geo.core.HomogeneousObject.copy}}
\pysigstartsignatures
\pysiglinewithargsret{\sphinxbfcode{\sphinxupquote{copy}}}{}{{ $\rightarrow$ Self}}
\pysigstopsignatures
\end{fulllineitems}

\index{data (deepdrr.geo.core.HomogeneousObject attribute)@\spxentry{data}\spxextra{deepdrr.geo.core.HomogeneousObject attribute}}

\begin{fulllineitems}
\phantomsection\label{\detokenize{deepdrr.geo:deepdrr.geo.core.HomogeneousObject.data}}
\pysigstartsignatures
\pysigline{\sphinxbfcode{\sphinxupquote{data}}\sphinxbfcode{\sphinxupquote{\DUrole{p,p}{:}\DUrole{w,w}{  }ndarray}}}
\pysigstopsignatures
\end{fulllineitems}

\index{dim (deepdrr.geo.core.HomogeneousObject property)@\spxentry{dim}\spxextra{deepdrr.geo.core.HomogeneousObject property}}

\begin{fulllineitems}
\phantomsection\label{\detokenize{deepdrr.geo:deepdrr.geo.core.HomogeneousObject.dim}}
\pysigstartsignatures
\pysigline{\sphinxbfcode{\sphinxupquote{abstract\DUrole{w,w}{  }property\DUrole{w,w}{  }}}\sphinxbfcode{\sphinxupquote{dim}}\sphinxbfcode{\sphinxupquote{\DUrole{p,p}{:}\DUrole{w,w}{  }int}}}
\pysigstopsignatures
\sphinxAtStartPar
Get the dimension of the space the object lives in. For transforms, this is the OUTPUT dim.

\end{fulllineitems}

\index{dtype (deepdrr.geo.core.HomogeneousObject attribute)@\spxentry{dtype}\spxextra{deepdrr.geo.core.HomogeneousObject attribute}}

\begin{fulllineitems}
\phantomsection\label{\detokenize{deepdrr.geo:deepdrr.geo.core.HomogeneousObject.dtype}}
\pysigstartsignatures
\pysigline{\sphinxbfcode{\sphinxupquote{dtype}}}
\pysigstopsignatures
\sphinxAtStartPar
alias of \sphinxcode{\sphinxupquote{float32}}

\end{fulllineitems}

\index{from\_array() (deepdrr.geo.core.HomogeneousObject class method)@\spxentry{from\_array()}\spxextra{deepdrr.geo.core.HomogeneousObject class method}}

\begin{fulllineitems}
\phantomsection\label{\detokenize{deepdrr.geo:deepdrr.geo.core.HomogeneousObject.from_array}}
\pysigstartsignatures
\pysiglinewithargsret{\sphinxbfcode{\sphinxupquote{classmethod\DUrole{w,w}{  }}}\sphinxbfcode{\sphinxupquote{from\_array}}}{\sphinxparam{\DUrole{n,n}{x}\DUrole{p,p}{:}\DUrole{w,w}{  }\DUrole{n,n}{ndarray}}}{{ $\rightarrow$ T}}
\pysigstopsignatures
\sphinxAtStartPar
Create a homogeneous object from its non\sphinxhyphen{}homogeous representation as an array.

\end{fulllineitems}

\index{get\_config() (deepdrr.geo.core.HomogeneousObject method)@\spxentry{get\_config()}\spxextra{deepdrr.geo.core.HomogeneousObject method}}

\begin{fulllineitems}
\phantomsection\label{\detokenize{deepdrr.geo:deepdrr.geo.core.HomogeneousObject.get_config}}
\pysigstartsignatures
\pysiglinewithargsret{\sphinxbfcode{\sphinxupquote{get\_config}}}{}{{ $\rightarrow$ dict}}
\pysigstopsignatures
\sphinxAtStartPar
Get a config dict with the data in this object.

\end{fulllineitems}

\index{get\_data() (deepdrr.geo.core.HomogeneousObject method)@\spxentry{get\_data()}\spxextra{deepdrr.geo.core.HomogeneousObject method}}

\begin{fulllineitems}
\phantomsection\label{\detokenize{deepdrr.geo:deepdrr.geo.core.HomogeneousObject.get_data}}
\pysigstartsignatures
\pysiglinewithargsret{\sphinxbfcode{\sphinxupquote{get\_data}}}{}{{ $\rightarrow$ ndarray}}
\pysigstopsignatures
\end{fulllineitems}

\index{shape (deepdrr.geo.core.HomogeneousObject property)@\spxentry{shape}\spxextra{deepdrr.geo.core.HomogeneousObject property}}

\begin{fulllineitems}
\phantomsection\label{\detokenize{deepdrr.geo:deepdrr.geo.core.HomogeneousObject.shape}}
\pysigstartsignatures
\pysigline{\sphinxbfcode{\sphinxupquote{property\DUrole{w,w}{  }}}\sphinxbfcode{\sphinxupquote{shape}}\sphinxbfcode{\sphinxupquote{\DUrole{p,p}{:}\DUrole{w,w}{  }Tuple\DUrole{p,p}{{[}}int\DUrole{p,p}{,}\DUrole{w,w}{  }\DUrole{p,p}{...}\DUrole{p,p}{{]}}}}}
\pysigstopsignatures
\end{fulllineitems}

\index{tolist() (deepdrr.geo.core.HomogeneousObject method)@\spxentry{tolist()}\spxextra{deepdrr.geo.core.HomogeneousObject method}}

\begin{fulllineitems}
\phantomsection\label{\detokenize{deepdrr.geo:deepdrr.geo.core.HomogeneousObject.tolist}}
\pysigstartsignatures
\pysiglinewithargsret{\sphinxbfcode{\sphinxupquote{tolist}}}{}{{ $\rightarrow$ List}}
\pysigstopsignatures
\sphinxAtStartPar
Get a json\sphinxhyphen{}save list with the data in this object.

\end{fulllineitems}


\end{fulllineitems}

\index{Joinable (class in deepdrr.geo.core)@\spxentry{Joinable}\spxextra{class in deepdrr.geo.core}}

\begin{fulllineitems}
\phantomsection\label{\detokenize{deepdrr.geo:deepdrr.geo.core.Joinable}}
\pysigstartsignatures
\pysigline{\sphinxbfcode{\sphinxupquote{class\DUrole{w,w}{  }}}\sphinxcode{\sphinxupquote{deepdrr.geo.core.}}\sphinxbfcode{\sphinxupquote{Joinable}}}
\pysigstopsignatures
\sphinxAtStartPar
Bases: \sphinxcode{\sphinxupquote{ABC}}

\sphinxAtStartPar
Abstract class for objects that can be joined together.
\index{join() (deepdrr.geo.core.Joinable method)@\spxentry{join()}\spxextra{deepdrr.geo.core.Joinable method}}

\begin{fulllineitems}
\phantomsection\label{\detokenize{deepdrr.geo:deepdrr.geo.core.Joinable.join}}
\pysigstartsignatures
\pysiglinewithargsret{\sphinxbfcode{\sphinxupquote{abstract\DUrole{w,w}{  }}}\sphinxbfcode{\sphinxupquote{join}}}{\sphinxparam{\DUrole{n,n}{other}\DUrole{p,p}{:}\DUrole{w,w}{  }\DUrole{n,n}{{\hyperref[\detokenize{deepdrr.geo:deepdrr.geo.core.Joinable}]{\sphinxcrossref{Joinable}}}}}}{{ $\rightarrow$ {\hyperref[\detokenize{deepdrr.geo:deepdrr.geo.core.Primitive}]{\sphinxcrossref{Primitive}}}}}
\pysigstopsignatures
\sphinxAtStartPar
Join two objects.

\sphinxAtStartPar
For example, given two points, get the line that connects them.
\begin{quote}\begin{description}
\sphinxlineitem{Parameters}
\sphinxAtStartPar
\sphinxstyleliteralstrong{\sphinxupquote{other}} ({\hyperref[\detokenize{deepdrr.geo:deepdrr.geo.core.Primitive}]{\sphinxcrossref{\sphinxstyleliteralemphasis{\sphinxupquote{Primitive}}}}}) \textendash{} the other primitive.

\sphinxlineitem{Returns}
\sphinxAtStartPar
the joined primitive.

\sphinxlineitem{Return type}
\sphinxAtStartPar
{\hyperref[\detokenize{deepdrr.geo:deepdrr.geo.core.Primitive}]{\sphinxcrossref{Primitive}}}

\end{description}\end{quote}

\end{fulllineitems}


\end{fulllineitems}

\index{Meetable (class in deepdrr.geo.core)@\spxentry{Meetable}\spxextra{class in deepdrr.geo.core}}

\begin{fulllineitems}
\phantomsection\label{\detokenize{deepdrr.geo:deepdrr.geo.core.Meetable}}
\pysigstartsignatures
\pysigline{\sphinxbfcode{\sphinxupquote{class\DUrole{w,w}{  }}}\sphinxcode{\sphinxupquote{deepdrr.geo.core.}}\sphinxbfcode{\sphinxupquote{Meetable}}}
\pysigstopsignatures
\sphinxAtStartPar
Bases: \sphinxcode{\sphinxupquote{ABC}}

\sphinxAtStartPar
Abstract class for objects that can be intersected.
\index{intersects() (deepdrr.geo.core.Meetable method)@\spxentry{intersects()}\spxextra{deepdrr.geo.core.Meetable method}}

\begin{fulllineitems}
\phantomsection\label{\detokenize{deepdrr.geo:deepdrr.geo.core.Meetable.intersects}}
\pysigstartsignatures
\pysiglinewithargsret{\sphinxbfcode{\sphinxupquote{intersects}}}{\sphinxparam{\DUrole{n,n}{other}\DUrole{p,p}{:}\DUrole{w,w}{  }\DUrole{n,n}{{\hyperref[\detokenize{deepdrr.geo:deepdrr.geo.core.Meetable}]{\sphinxcrossref{Meetable}}}}}}{{ $\rightarrow$ bool}}
\pysigstopsignatures
\sphinxAtStartPar
Check if two objects intersect.
\begin{quote}\begin{description}
\sphinxlineitem{Parameters}
\sphinxAtStartPar
\sphinxstyleliteralstrong{\sphinxupquote{other}} ({\hyperref[\detokenize{deepdrr.geo:deepdrr.geo.core.Primitive}]{\sphinxcrossref{\sphinxstyleliteralemphasis{\sphinxupquote{Primitive}}}}}) \textendash{} the other primitive.

\sphinxlineitem{Returns}
\sphinxAtStartPar
True if the objects intersect, False otherwise.

\sphinxlineitem{Return type}
\sphinxAtStartPar
bool

\end{description}\end{quote}

\end{fulllineitems}

\index{meet() (deepdrr.geo.core.Meetable method)@\spxentry{meet()}\spxextra{deepdrr.geo.core.Meetable method}}

\begin{fulllineitems}
\phantomsection\label{\detokenize{deepdrr.geo:deepdrr.geo.core.Meetable.meet}}
\pysigstartsignatures
\pysiglinewithargsret{\sphinxbfcode{\sphinxupquote{abstract\DUrole{w,w}{  }}}\sphinxbfcode{\sphinxupquote{meet}}}{\sphinxparam{\DUrole{n,n}{other}\DUrole{p,p}{:}\DUrole{w,w}{  }\DUrole{n,n}{{\hyperref[\detokenize{deepdrr.geo:deepdrr.geo.core.Meetable}]{\sphinxcrossref{Meetable}}}}}}{{ $\rightarrow$ {\hyperref[\detokenize{deepdrr.geo:deepdrr.geo.core.Primitive}]{\sphinxcrossref{Primitive}}}}}
\pysigstopsignatures
\sphinxAtStartPar
Get the intersection of two objects.

\sphinxAtStartPar
For example, given two lines, get the line that is the intersection of them.
\begin{quote}\begin{description}
\sphinxlineitem{Parameters}
\sphinxAtStartPar
\sphinxstyleliteralstrong{\sphinxupquote{other}} ({\hyperref[\detokenize{deepdrr.geo:deepdrr.geo.core.Primitive}]{\sphinxcrossref{\sphinxstyleliteralemphasis{\sphinxupquote{Primitive}}}}}) \textendash{} the other primitive.

\sphinxlineitem{Returns}
\sphinxAtStartPar
the intersection of \sphinxtitleref{self} and \sphinxtitleref{other}.

\sphinxlineitem{Return type}
\sphinxAtStartPar
{\hyperref[\detokenize{deepdrr.geo:deepdrr.geo.core.Primitive}]{\sphinxcrossref{Primitive}}}

\sphinxlineitem{Raises}
\sphinxAtStartPar
{\hyperref[\detokenize{deepdrr.geo:deepdrr.geo.exceptions.MeetError}]{\sphinxcrossref{\sphinxstyleliteralstrong{\sphinxupquote{MeetError}}}}} \textendash{} if the objects cannot be intersected.

\end{description}\end{quote}

\end{fulllineitems}


\end{fulllineitems}

\index{Point (class in deepdrr.geo.core)@\spxentry{Point}\spxextra{class in deepdrr.geo.core}}

\begin{fulllineitems}
\phantomsection\label{\detokenize{deepdrr.geo:deepdrr.geo.core.Point}}
\pysigstartsignatures
\pysiglinewithargsret{\sphinxbfcode{\sphinxupquote{class\DUrole{w,w}{  }}}\sphinxcode{\sphinxupquote{deepdrr.geo.core.}}\sphinxbfcode{\sphinxupquote{Point}}}{\sphinxparam{\DUrole{n,n}{data}\DUrole{p,p}{:}\DUrole{w,w}{  }\DUrole{n,n}{ndarray}}}{}
\pysigstopsignatures
\sphinxAtStartPar
Bases: {\hyperref[\detokenize{deepdrr.geo:deepdrr.geo.core.PointOrVector}]{\sphinxcrossref{\sphinxcode{\sphinxupquote{PointOrVector}}}}}, {\hyperref[\detokenize{deepdrr.geo:deepdrr.geo.core.Joinable}]{\sphinxcrossref{\sphinxcode{\sphinxupquote{Joinable}}}}}, {\hyperref[\detokenize{deepdrr.geo:deepdrr.geo.core.HasLocation}]{\sphinxcrossref{\sphinxcode{\sphinxupquote{HasLocation}}}}}
\index{as\_vector() (deepdrr.geo.core.Point method)@\spxentry{as\_vector()}\spxextra{deepdrr.geo.core.Point method}}

\begin{fulllineitems}
\phantomsection\label{\detokenize{deepdrr.geo:deepdrr.geo.core.Point.as_vector}}
\pysigstartsignatures
\pysiglinewithargsret{\sphinxbfcode{\sphinxupquote{as\_vector}}}{}{{ $\rightarrow$ {\hyperref[\detokenize{deepdrr.geo:deepdrr.geo.core.Vector}]{\sphinxcrossref{Vector}}}}}
\pysigstopsignatures
\sphinxAtStartPar
Get the vector with the same numerical representation as this point.

\end{fulllineitems}

\index{data (deepdrr.geo.core.Point attribute)@\spxentry{data}\spxextra{deepdrr.geo.core.Point attribute}}

\begin{fulllineitems}
\phantomsection\label{\detokenize{deepdrr.geo:deepdrr.geo.core.Point.data}}
\pysigstartsignatures
\pysigline{\sphinxbfcode{\sphinxupquote{data}}\sphinxbfcode{\sphinxupquote{\DUrole{p,p}{:}\DUrole{w,w}{  }ndarray}}}
\pysigstopsignatures
\end{fulllineitems}

\index{from\_any() (deepdrr.geo.core.Point class method)@\spxentry{from\_any()}\spxextra{deepdrr.geo.core.Point class method}}

\begin{fulllineitems}
\phantomsection\label{\detokenize{deepdrr.geo:deepdrr.geo.core.Point.from_any}}
\pysigstartsignatures
\pysiglinewithargsret{\sphinxbfcode{\sphinxupquote{classmethod\DUrole{w,w}{  }}}\sphinxbfcode{\sphinxupquote{from\_any}}}{\sphinxparam{\DUrole{n,n}{other}\DUrole{p,p}{:}\DUrole{w,w}{  }\DUrole{n,n}{ndarray\DUrole{w,w}{  }\DUrole{p,p}{|}\DUrole{w,w}{  }{\hyperref[\detokenize{deepdrr.geo:deepdrr.geo.core.Point}]{\sphinxcrossref{Point}}}}}}{}
\pysigstopsignatures
\sphinxAtStartPar
If other is not a point, make it one.

\end{fulllineitems}

\index{from\_array() (deepdrr.geo.core.Point class method)@\spxentry{from\_array()}\spxextra{deepdrr.geo.core.Point class method}}

\begin{fulllineitems}
\phantomsection\label{\detokenize{deepdrr.geo:deepdrr.geo.core.Point.from_array}}
\pysigstartsignatures
\pysiglinewithargsret{\sphinxbfcode{\sphinxupquote{classmethod\DUrole{w,w}{  }}}\sphinxbfcode{\sphinxupquote{from\_array}}}{\sphinxparam{\DUrole{n,n}{x}\DUrole{p,p}{:}\DUrole{w,w}{  }\DUrole{n,n}{ndarray}}}{{ $\rightarrow$ T}}
\pysigstopsignatures
\sphinxAtStartPar
Create a homogeneous object from its non\sphinxhyphen{}homogeous representation as an array.

\end{fulllineitems}

\index{get\_point() (deepdrr.geo.core.Point method)@\spxentry{get\_point()}\spxextra{deepdrr.geo.core.Point method}}

\begin{fulllineitems}
\phantomsection\label{\detokenize{deepdrr.geo:deepdrr.geo.core.Point.get_point}}
\pysigstartsignatures
\pysiglinewithargsret{\sphinxbfcode{\sphinxupquote{get\_point}}}{}{{ $\rightarrow$ Self}}
\pysigstopsignatures
\sphinxAtStartPar
Get the point with the same numerical representation as this point.

\end{fulllineitems}

\index{lerp() (deepdrr.geo.core.Point method)@\spxentry{lerp()}\spxextra{deepdrr.geo.core.Point method}}

\begin{fulllineitems}
\phantomsection\label{\detokenize{deepdrr.geo:deepdrr.geo.core.Point.lerp}}
\pysigstartsignatures
\pysiglinewithargsret{\sphinxbfcode{\sphinxupquote{lerp}}}{\sphinxparam{\DUrole{n,n}{other}\DUrole{p,p}{:}\DUrole{w,w}{  }\DUrole{n,n}{{\hyperref[\detokenize{deepdrr.geo:deepdrr.geo.core.Point}]{\sphinxcrossref{Point}}}}}\sphinxparamcomma \sphinxparam{\DUrole{n,n}{alpha}\DUrole{p,p}{:}\DUrole{w,w}{  }\DUrole{n,n}{float}\DUrole{w,w}{  }\DUrole{o,o}{=}\DUrole{w,w}{  }\DUrole{default_value}{0.5}}}{{ $\rightarrow$ Self}}
\pysigstopsignatures
\sphinxAtStartPar
Linearly interpolate between one point and another.
\begin{quote}\begin{description}
\sphinxlineitem{Parameters}\begin{itemize}
\item {} 
\sphinxAtStartPar
\sphinxstyleliteralstrong{\sphinxupquote{other}} ({\hyperref[\detokenize{deepdrr.geo:deepdrr.geo.core.Point}]{\sphinxcrossref{\sphinxstyleliteralemphasis{\sphinxupquote{Point}}}}}) \textendash{} other point.

\item {} 
\sphinxAtStartPar
\sphinxstyleliteralstrong{\sphinxupquote{alpha}} (\sphinxstyleliteralemphasis{\sphinxupquote{float}}) \textendash{} fraction of the distance from self to other to travel. Defaults to 0.5 (the midpoint).

\end{itemize}

\sphinxlineitem{Returns}
\sphinxAtStartPar
the point that is \sphinxtitleref{alpha} of the way between self and other.

\sphinxlineitem{Return type}
\sphinxAtStartPar
{\hyperref[\detokenize{deepdrr.geo:deepdrr.geo.core.Point}]{\sphinxcrossref{Point}}}

\end{description}\end{quote}

\end{fulllineitems}


\end{fulllineitems}

\index{Point2D (class in deepdrr.geo.core)@\spxentry{Point2D}\spxextra{class in deepdrr.geo.core}}

\begin{fulllineitems}
\phantomsection\label{\detokenize{deepdrr.geo:deepdrr.geo.core.Point2D}}
\pysigstartsignatures
\pysiglinewithargsret{\sphinxbfcode{\sphinxupquote{class\DUrole{w,w}{  }}}\sphinxcode{\sphinxupquote{deepdrr.geo.core.}}\sphinxbfcode{\sphinxupquote{Point2D}}}{\sphinxparam{\DUrole{n,n}{data}\DUrole{p,p}{:}\DUrole{w,w}{  }\DUrole{n,n}{ndarray}}}{}
\pysigstopsignatures
\sphinxAtStartPar
Bases: {\hyperref[\detokenize{deepdrr.geo:deepdrr.geo.core.Point}]{\sphinxcrossref{\sphinxcode{\sphinxupquote{Point}}}}}

\sphinxAtStartPar
Homogeneous point in 2D, represented as an array with {[}x, y, 1{]}
\index{backproject() (deepdrr.geo.core.Point2D method)@\spxentry{backproject()}\spxextra{deepdrr.geo.core.Point2D method}}

\begin{fulllineitems}
\phantomsection\label{\detokenize{deepdrr.geo:deepdrr.geo.core.Point2D.backproject}}
\pysigstartsignatures
\pysiglinewithargsret{\sphinxbfcode{\sphinxupquote{backproject}}}{\sphinxparam{\DUrole{n,n}{index\_from\_world}\DUrole{p,p}{:}\DUrole{w,w}{  }\DUrole{n,n}{{\hyperref[\detokenize{deepdrr.geo:deepdrr.geo.core.CameraProjection}]{\sphinxcrossref{CameraProjection}}}}}}{{ $\rightarrow$ {\hyperref[\detokenize{deepdrr.geo:deepdrr.geo.hyperplane.Line3D}]{\sphinxcrossref{Line3D}}}}}
\pysigstopsignatures
\sphinxAtStartPar
Backproject this point into a line.
\begin{quote}\begin{description}
\sphinxlineitem{Parameters}
\sphinxAtStartPar
\sphinxstyleliteralstrong{\sphinxupquote{index\_from\_world}} ({\hyperref[\detokenize{deepdrr.geo:deepdrr.geo.core.Transform}]{\sphinxcrossref{\sphinxstyleliteralemphasis{\sphinxupquote{Transform}}}}}) \textendash{} The transform from the world to the index.

\sphinxlineitem{Returns}
\sphinxAtStartPar
The line in 3D space through the source of \sphinxtitleref{index\_from\_world} and self.

\sphinxlineitem{Return type}
\sphinxAtStartPar
{\hyperref[\detokenize{deepdrr.geo:deepdrr.geo.hyperplane.Line3D}]{\sphinxcrossref{Line3D}}}

\end{description}\end{quote}

\end{fulllineitems}

\index{data (deepdrr.geo.core.Point2D attribute)@\spxentry{data}\spxextra{deepdrr.geo.core.Point2D attribute}}

\begin{fulllineitems}
\phantomsection\label{\detokenize{deepdrr.geo:deepdrr.geo.core.Point2D.data}}
\pysigstartsignatures
\pysigline{\sphinxbfcode{\sphinxupquote{data}}\sphinxbfcode{\sphinxupquote{\DUrole{p,p}{:}\DUrole{w,w}{  }ndarray}}}
\pysigstopsignatures
\end{fulllineitems}

\index{dim (deepdrr.geo.core.Point2D attribute)@\spxentry{dim}\spxextra{deepdrr.geo.core.Point2D attribute}}

\begin{fulllineitems}
\phantomsection\label{\detokenize{deepdrr.geo:deepdrr.geo.core.Point2D.dim}}
\pysigstartsignatures
\pysigline{\sphinxbfcode{\sphinxupquote{dim}}\sphinxbfcode{\sphinxupquote{\DUrole{w,w}{  }\DUrole{p,p}{=}\DUrole{w,w}{  }2}}}
\pysigstopsignatures
\end{fulllineitems}

\index{join() (deepdrr.geo.core.Point2D method)@\spxentry{join()}\spxextra{deepdrr.geo.core.Point2D method}}

\begin{fulllineitems}
\phantomsection\label{\detokenize{deepdrr.geo:deepdrr.geo.core.Point2D.join}}
\pysigstartsignatures
\pysiglinewithargsret{\sphinxbfcode{\sphinxupquote{join}}}{\sphinxparam{\DUrole{n,n}{other}\DUrole{p,p}{:}\DUrole{w,w}{  }\DUrole{n,n}{{\hyperref[\detokenize{deepdrr.geo:deepdrr.geo.core.Point2D}]{\sphinxcrossref{Point2D}}}}}}{{ $\rightarrow$ {\hyperref[\detokenize{deepdrr.geo:deepdrr.geo.hyperplane.Line2D}]{\sphinxcrossref{Line2D}}}}}
\pysiglinewithargsret{\sphinxbfcode{\sphinxupquote{join}}}{\sphinxparam{\DUrole{n,n}{other}\DUrole{p,p}{:}\DUrole{w,w}{  }\DUrole{n,n}{{\hyperref[\detokenize{deepdrr.geo:deepdrr.geo.hyperplane.Line2D}]{\sphinxcrossref{Line2D}}}}}}{{ $\rightarrow$ {\hyperref[\detokenize{deepdrr.geo:deepdrr.geo.core.Vector2D}]{\sphinxcrossref{Vector2D}}}}}
\pysigstopsignatures
\sphinxAtStartPar
Join two objects.

\sphinxAtStartPar
For example, given two points, get the line that connects them.
\begin{quote}\begin{description}
\sphinxlineitem{Parameters}
\sphinxAtStartPar
\sphinxstyleliteralstrong{\sphinxupquote{other}} ({\hyperref[\detokenize{deepdrr.geo:deepdrr.geo.core.Primitive}]{\sphinxcrossref{\sphinxstyleliteralemphasis{\sphinxupquote{Primitive}}}}}) \textendash{} the other primitive.

\sphinxlineitem{Returns}
\sphinxAtStartPar
the joined primitive.

\sphinxlineitem{Return type}
\sphinxAtStartPar
{\hyperref[\detokenize{deepdrr.geo:deepdrr.geo.core.Primitive}]{\sphinxcrossref{Primitive}}}

\end{description}\end{quote}

\end{fulllineitems}


\end{fulllineitems}

\index{Point3D (class in deepdrr.geo.core)@\spxentry{Point3D}\spxextra{class in deepdrr.geo.core}}

\begin{fulllineitems}
\phantomsection\label{\detokenize{deepdrr.geo:deepdrr.geo.core.Point3D}}
\pysigstartsignatures
\pysiglinewithargsret{\sphinxbfcode{\sphinxupquote{class\DUrole{w,w}{  }}}\sphinxcode{\sphinxupquote{deepdrr.geo.core.}}\sphinxbfcode{\sphinxupquote{Point3D}}}{\sphinxparam{\DUrole{n,n}{data}\DUrole{p,p}{:}\DUrole{w,w}{  }\DUrole{n,n}{ndarray}}}{}
\pysigstopsignatures
\sphinxAtStartPar
Bases: {\hyperref[\detokenize{deepdrr.geo:deepdrr.geo.core.Point}]{\sphinxcrossref{\sphinxcode{\sphinxupquote{Point}}}}}

\sphinxAtStartPar
Homogeneous point in 3D, represented as an array with {[}x, y, z, 1{]}
\index{data (deepdrr.geo.core.Point3D attribute)@\spxentry{data}\spxextra{deepdrr.geo.core.Point3D attribute}}

\begin{fulllineitems}
\phantomsection\label{\detokenize{deepdrr.geo:deepdrr.geo.core.Point3D.data}}
\pysigstartsignatures
\pysigline{\sphinxbfcode{\sphinxupquote{data}}\sphinxbfcode{\sphinxupquote{\DUrole{p,p}{:}\DUrole{w,w}{  }ndarray}}}
\pysigstopsignatures
\end{fulllineitems}

\index{dim (deepdrr.geo.core.Point3D attribute)@\spxentry{dim}\spxextra{deepdrr.geo.core.Point3D attribute}}

\begin{fulllineitems}
\phantomsection\label{\detokenize{deepdrr.geo:deepdrr.geo.core.Point3D.dim}}
\pysigstartsignatures
\pysigline{\sphinxbfcode{\sphinxupquote{dim}}\sphinxbfcode{\sphinxupquote{\DUrole{w,w}{  }\DUrole{p,p}{=}\DUrole{w,w}{  }3}}}
\pysigstopsignatures
\end{fulllineitems}

\index{join() (deepdrr.geo.core.Point3D method)@\spxentry{join()}\spxextra{deepdrr.geo.core.Point3D method}}

\begin{fulllineitems}
\phantomsection\label{\detokenize{deepdrr.geo:deepdrr.geo.core.Point3D.join}}
\pysigstartsignatures
\pysiglinewithargsret{\sphinxbfcode{\sphinxupquote{join}}}{\sphinxparam{\DUrole{n,n}{other}\DUrole{p,p}{:}\DUrole{w,w}{  }\DUrole{n,n}{{\hyperref[\detokenize{deepdrr.geo:deepdrr.geo.core.Point3D}]{\sphinxcrossref{Point3D}}}}}}{{ $\rightarrow$ {\hyperref[\detokenize{deepdrr.geo:deepdrr.geo.hyperplane.Line3D}]{\sphinxcrossref{Line3D}}}}}
\pysiglinewithargsret{\sphinxbfcode{\sphinxupquote{join}}}{\sphinxparam{\DUrole{n,n}{other}\DUrole{p,p}{:}\DUrole{w,w}{  }\DUrole{n,n}{{\hyperref[\detokenize{deepdrr.geo:deepdrr.geo.hyperplane.Line3D}]{\sphinxcrossref{Line3D}}}}}}{{ $\rightarrow$ {\hyperref[\detokenize{deepdrr.geo:deepdrr.geo.hyperplane.Plane}]{\sphinxcrossref{Plane}}}}}
\pysigstopsignatures
\sphinxAtStartPar
Join two objects.

\sphinxAtStartPar
For example, given two points, get the line that connects them.
\begin{quote}\begin{description}
\sphinxlineitem{Parameters}
\sphinxAtStartPar
\sphinxstyleliteralstrong{\sphinxupquote{other}} ({\hyperref[\detokenize{deepdrr.geo:deepdrr.geo.core.Primitive}]{\sphinxcrossref{\sphinxstyleliteralemphasis{\sphinxupquote{Primitive}}}}}) \textendash{} the other primitive.

\sphinxlineitem{Returns}
\sphinxAtStartPar
the joined primitive.

\sphinxlineitem{Return type}
\sphinxAtStartPar
{\hyperref[\detokenize{deepdrr.geo:deepdrr.geo.core.Primitive}]{\sphinxcrossref{Primitive}}}

\end{description}\end{quote}

\end{fulllineitems}

\index{projection\_type() (deepdrr.geo.core.Point3D class method)@\spxentry{projection\_type()}\spxextra{deepdrr.geo.core.Point3D class method}}

\begin{fulllineitems}
\phantomsection\label{\detokenize{deepdrr.geo:deepdrr.geo.core.Point3D.projection_type}}
\pysigstartsignatures
\pysiglinewithargsret{\sphinxbfcode{\sphinxupquote{classmethod\DUrole{w,w}{  }}}\sphinxbfcode{\sphinxupquote{projection\_type}}}{}{{ $\rightarrow$ Type\DUrole{p,p}{{[}}{\hyperref[\detokenize{deepdrr.geo:deepdrr.geo.core.Point2D}]{\sphinxcrossref{Point2D}}}\DUrole{p,p}{{]}}}}
\pysigstopsignatures
\end{fulllineitems}


\end{fulllineitems}

\index{PointOrVector (class in deepdrr.geo.core)@\spxentry{PointOrVector}\spxextra{class in deepdrr.geo.core}}

\begin{fulllineitems}
\phantomsection\label{\detokenize{deepdrr.geo:deepdrr.geo.core.PointOrVector}}
\pysigstartsignatures
\pysiglinewithargsret{\sphinxbfcode{\sphinxupquote{class\DUrole{w,w}{  }}}\sphinxcode{\sphinxupquote{deepdrr.geo.core.}}\sphinxbfcode{\sphinxupquote{PointOrVector}}}{\sphinxparam{\DUrole{n,n}{data}\DUrole{p,p}{:}\DUrole{w,w}{  }\DUrole{n,n}{ndarray}}}{}
\pysigstopsignatures
\sphinxAtStartPar
Bases: {\hyperref[\detokenize{deepdrr.geo:deepdrr.geo.core.Primitive}]{\sphinxcrossref{\sphinxcode{\sphinxupquote{Primitive}}}}}

\sphinxAtStartPar
A Homogeneous point or vector in any dimension.
\index{data (deepdrr.geo.core.PointOrVector attribute)@\spxentry{data}\spxextra{deepdrr.geo.core.PointOrVector attribute}}

\begin{fulllineitems}
\phantomsection\label{\detokenize{deepdrr.geo:deepdrr.geo.core.PointOrVector.data}}
\pysigstartsignatures
\pysigline{\sphinxbfcode{\sphinxupquote{data}}\sphinxbfcode{\sphinxupquote{\DUrole{p,p}{:}\DUrole{w,w}{  }ndarray}}}
\pysigstopsignatures
\end{fulllineitems}

\index{norm() (deepdrr.geo.core.PointOrVector method)@\spxentry{norm()}\spxextra{deepdrr.geo.core.PointOrVector method}}

\begin{fulllineitems}
\phantomsection\label{\detokenize{deepdrr.geo:deepdrr.geo.core.PointOrVector.norm}}
\pysigstartsignatures
\pysiglinewithargsret{\sphinxbfcode{\sphinxupquote{norm}}}{\sphinxparam{\DUrole{o,o}{*}\DUrole{n,n}{args}}\sphinxparamcomma \sphinxparam{\DUrole{o,o}{**}\DUrole{n,n}{kwargs}}}{{ $\rightarrow$ float}}
\pysigstopsignatures
\sphinxAtStartPar
Get the norm of the vector. Pass any arguments to \sphinxtitleref{np.linalg.norm}.

\end{fulllineitems}

\index{normsqr() (deepdrr.geo.core.PointOrVector method)@\spxentry{normsqr()}\spxextra{deepdrr.geo.core.PointOrVector method}}

\begin{fulllineitems}
\phantomsection\label{\detokenize{deepdrr.geo:deepdrr.geo.core.PointOrVector.normsqr}}
\pysigstartsignatures
\pysiglinewithargsret{\sphinxbfcode{\sphinxupquote{normsqr}}}{\sphinxparam{\DUrole{n,n}{order}\DUrole{p,p}{:}\DUrole{w,w}{  }\DUrole{n,n}{int}\DUrole{w,w}{  }\DUrole{o,o}{=}\DUrole{w,w}{  }\DUrole{default_value}{2}}}{{ $\rightarrow$ float}}
\pysigstopsignatures
\sphinxAtStartPar
Get the squared L\sphinxhyphen{}order norm of the vector.

\end{fulllineitems}

\index{w (deepdrr.geo.core.PointOrVector property)@\spxentry{w}\spxextra{deepdrr.geo.core.PointOrVector property}}

\begin{fulllineitems}
\phantomsection\label{\detokenize{deepdrr.geo:deepdrr.geo.core.PointOrVector.w}}
\pysigstartsignatures
\pysigline{\sphinxbfcode{\sphinxupquote{property\DUrole{w,w}{  }}}\sphinxbfcode{\sphinxupquote{w}}\sphinxbfcode{\sphinxupquote{\DUrole{p,p}{:}\DUrole{w,w}{  }float}}}
\pysigstopsignatures
\end{fulllineitems}

\index{x (deepdrr.geo.core.PointOrVector property)@\spxentry{x}\spxextra{deepdrr.geo.core.PointOrVector property}}

\begin{fulllineitems}
\phantomsection\label{\detokenize{deepdrr.geo:deepdrr.geo.core.PointOrVector.x}}
\pysigstartsignatures
\pysigline{\sphinxbfcode{\sphinxupquote{property\DUrole{w,w}{  }}}\sphinxbfcode{\sphinxupquote{x}}\sphinxbfcode{\sphinxupquote{\DUrole{p,p}{:}\DUrole{w,w}{  }float}}}
\pysigstopsignatures
\end{fulllineitems}

\index{y (deepdrr.geo.core.PointOrVector property)@\spxentry{y}\spxextra{deepdrr.geo.core.PointOrVector property}}

\begin{fulllineitems}
\phantomsection\label{\detokenize{deepdrr.geo:deepdrr.geo.core.PointOrVector.y}}
\pysigstartsignatures
\pysigline{\sphinxbfcode{\sphinxupquote{property\DUrole{w,w}{  }}}\sphinxbfcode{\sphinxupquote{y}}\sphinxbfcode{\sphinxupquote{\DUrole{p,p}{:}\DUrole{w,w}{  }float}}}
\pysigstopsignatures
\end{fulllineitems}

\index{z (deepdrr.geo.core.PointOrVector property)@\spxentry{z}\spxextra{deepdrr.geo.core.PointOrVector property}}

\begin{fulllineitems}
\phantomsection\label{\detokenize{deepdrr.geo:deepdrr.geo.core.PointOrVector.z}}
\pysigstartsignatures
\pysigline{\sphinxbfcode{\sphinxupquote{property\DUrole{w,w}{  }}}\sphinxbfcode{\sphinxupquote{z}}\sphinxbfcode{\sphinxupquote{\DUrole{p,p}{:}\DUrole{w,w}{  }float}}}
\pysigstopsignatures
\end{fulllineitems}


\end{fulllineitems}

\index{Primitive (class in deepdrr.geo.core)@\spxentry{Primitive}\spxextra{class in deepdrr.geo.core}}

\begin{fulllineitems}
\phantomsection\label{\detokenize{deepdrr.geo:deepdrr.geo.core.Primitive}}
\pysigstartsignatures
\pysiglinewithargsret{\sphinxbfcode{\sphinxupquote{class\DUrole{w,w}{  }}}\sphinxcode{\sphinxupquote{deepdrr.geo.core.}}\sphinxbfcode{\sphinxupquote{Primitive}}}{\sphinxparam{\DUrole{n,n}{data}\DUrole{p,p}{:}\DUrole{w,w}{  }\DUrole{n,n}{ndarray}}}{}
\pysigstopsignatures
\sphinxAtStartPar
Bases: {\hyperref[\detokenize{deepdrr.geo:deepdrr.geo.core.HomogeneousObject}]{\sphinxcrossref{\sphinxcode{\sphinxupquote{HomogeneousObject}}}}}

\sphinxAtStartPar
Abstract class for geometric primitives.

\sphinxAtStartPar
Primitives are the objects contained in a homogeneous frame, like points, vectors, lines, shapes, etc.
\index{data (deepdrr.geo.core.Primitive attribute)@\spxentry{data}\spxextra{deepdrr.geo.core.Primitive attribute}}

\begin{fulllineitems}
\phantomsection\label{\detokenize{deepdrr.geo:deepdrr.geo.core.Primitive.data}}
\pysigstartsignatures
\pysigline{\sphinxbfcode{\sphinxupquote{data}}\sphinxbfcode{\sphinxupquote{\DUrole{p,p}{:}\DUrole{w,w}{  }ndarray}}}
\pysigstopsignatures
\end{fulllineitems}


\end{fulllineitems}

\index{Transform (class in deepdrr.geo.core)@\spxentry{Transform}\spxextra{class in deepdrr.geo.core}}

\begin{fulllineitems}
\phantomsection\label{\detokenize{deepdrr.geo:deepdrr.geo.core.Transform}}
\pysigstartsignatures
\pysiglinewithargsret{\sphinxbfcode{\sphinxupquote{class\DUrole{w,w}{  }}}\sphinxcode{\sphinxupquote{deepdrr.geo.core.}}\sphinxbfcode{\sphinxupquote{Transform}}}{\sphinxparam{\DUrole{n,n}{data}\DUrole{p,p}{:}\DUrole{w,w}{  }\DUrole{n,n}{ndarray}}\sphinxparamcomma \sphinxparam{\DUrole{n,n}{\_inv}\DUrole{p,p}{:}\DUrole{w,w}{  }\DUrole{n,n}{ndarray\DUrole{w,w}{  }\DUrole{p,p}{|}\DUrole{w,w}{  }None}\DUrole{w,w}{  }\DUrole{o,o}{=}\DUrole{w,w}{  }\DUrole{default_value}{None}}}{}
\pysigstopsignatures
\sphinxAtStartPar
Bases: {\hyperref[\detokenize{deepdrr.geo:deepdrr.geo.core.HomogeneousObject}]{\sphinxcrossref{\sphinxcode{\sphinxupquote{HomogeneousObject}}}}}
\index{data (deepdrr.geo.core.Transform attribute)@\spxentry{data}\spxextra{deepdrr.geo.core.Transform attribute}}

\begin{fulllineitems}
\phantomsection\label{\detokenize{deepdrr.geo:deepdrr.geo.core.Transform.data}}
\pysigstartsignatures
\pysigline{\sphinxbfcode{\sphinxupquote{data}}\sphinxbfcode{\sphinxupquote{\DUrole{p,p}{:}\DUrole{w,w}{  }ndarray}}}
\pysigstopsignatures
\end{fulllineitems}

\index{dim (deepdrr.geo.core.Transform property)@\spxentry{dim}\spxextra{deepdrr.geo.core.Transform property}}

\begin{fulllineitems}
\phantomsection\label{\detokenize{deepdrr.geo:deepdrr.geo.core.Transform.dim}}
\pysigstartsignatures
\pysigline{\sphinxbfcode{\sphinxupquote{property\DUrole{w,w}{  }}}\sphinxbfcode{\sphinxupquote{dim}}}
\pysigstopsignatures
\sphinxAtStartPar
The output dimension of the transformation.

\end{fulllineitems}

\index{from\_array() (deepdrr.geo.core.Transform class method)@\spxentry{from\_array()}\spxextra{deepdrr.geo.core.Transform class method}}

\begin{fulllineitems}
\phantomsection\label{\detokenize{deepdrr.geo:deepdrr.geo.core.Transform.from_array}}
\pysigstartsignatures
\pysiglinewithargsret{\sphinxbfcode{\sphinxupquote{classmethod\DUrole{w,w}{  }}}\sphinxbfcode{\sphinxupquote{from\_array}}}{\sphinxparam{\DUrole{n,n}{array}\DUrole{p,p}{:}\DUrole{w,w}{  }\DUrole{n,n}{ndarray}}}{{ $\rightarrow$ {\hyperref[\detokenize{deepdrr.geo:deepdrr.geo.core.Transform}]{\sphinxcrossref{Transform}}}}}
\pysigstopsignatures
\sphinxAtStartPar
Convert non\sphinxhyphen{}homogeneous matrix to homogeneous transform.

\sphinxAtStartPar
Usually, one would instantiate Transforms directly from the homogeneous matrix \sphinxtitleref{data} or using one of the other classmethods.
\begin{quote}\begin{description}
\sphinxlineitem{Parameters}
\sphinxAtStartPar
\sphinxstyleliteralstrong{\sphinxupquote{array}} (\sphinxstyleliteralemphasis{\sphinxupquote{np.ndarray}}) \textendash{} transformation matrix.

\sphinxlineitem{Returns}
\sphinxAtStartPar
the transform.

\sphinxlineitem{Return type}
\sphinxAtStartPar
{\hyperref[\detokenize{deepdrr.geo:deepdrr.geo.core.Transform}]{\sphinxcrossref{Transform}}}

\end{description}\end{quote}

\end{fulllineitems}

\index{get\_center() (deepdrr.geo.core.Transform method)@\spxentry{get\_center()}\spxextra{deepdrr.geo.core.Transform method}}

\begin{fulllineitems}
\phantomsection\label{\detokenize{deepdrr.geo:deepdrr.geo.core.Transform.get_center}}
\pysigstartsignatures
\pysiglinewithargsret{\sphinxbfcode{\sphinxupquote{get\_center}}}{}{{ $\rightarrow$ {\hyperref[\detokenize{deepdrr.geo:deepdrr.geo.core.Point3D}]{\sphinxcrossref{Point3D}}}}}
\pysigstopsignatures
\sphinxAtStartPar
If the transform is a projection, get the center of the projection.
\begin{quote}\begin{description}
\sphinxlineitem{Returns}
\sphinxAtStartPar
the center of the projection.

\sphinxlineitem{Return type}
\sphinxAtStartPar
({\hyperref[\detokenize{deepdrr.geo:deepdrr.geo.core.Point3D}]{\sphinxcrossref{Point3D}}})

\sphinxlineitem{Raises}
\sphinxAtStartPar
\sphinxstyleliteralstrong{\sphinxupquote{ValueError}} \textendash{} if the transform is not a projection.

\end{description}\end{quote}

\end{fulllineitems}

\index{input\_dim (deepdrr.geo.core.Transform property)@\spxentry{input\_dim}\spxextra{deepdrr.geo.core.Transform property}}

\begin{fulllineitems}
\phantomsection\label{\detokenize{deepdrr.geo:deepdrr.geo.core.Transform.input_dim}}
\pysigstartsignatures
\pysigline{\sphinxbfcode{\sphinxupquote{property\DUrole{w,w}{  }}}\sphinxbfcode{\sphinxupquote{input\_dim}}}
\pysigstopsignatures
\sphinxAtStartPar
The input dimension of the transformation.

\end{fulllineitems}

\index{inv (deepdrr.geo.core.Transform property)@\spxentry{inv}\spxextra{deepdrr.geo.core.Transform property}}

\begin{fulllineitems}
\phantomsection\label{\detokenize{deepdrr.geo:deepdrr.geo.core.Transform.inv}}
\pysigstartsignatures
\pysigline{\sphinxbfcode{\sphinxupquote{property\DUrole{w,w}{  }}}\sphinxbfcode{\sphinxupquote{inv}}\sphinxbfcode{\sphinxupquote{\DUrole{p,p}{:}\DUrole{w,w}{  }Self}}}
\pysigstopsignatures
\sphinxAtStartPar
Get the inverse of the Transform.
\begin{quote}\begin{description}
\sphinxlineitem{Returns}
\sphinxAtStartPar
a Transform (or subclass) that is well\sphinxhyphen{}defined as the inverse of this transform.

\sphinxlineitem{Return type}
\sphinxAtStartPar
({\hyperref[\detokenize{deepdrr.geo:deepdrr.geo.core.Transform}]{\sphinxcrossref{Transform}}})

\sphinxlineitem{Raises}
\sphinxAtStartPar
\sphinxstyleliteralstrong{\sphinxupquote{NotImplementedError}} \textendash{} if \_inv is None and method is not overriden.

\end{description}\end{quote}

\end{fulllineitems}

\index{inverse() (deepdrr.geo.core.Transform method)@\spxentry{inverse()}\spxextra{deepdrr.geo.core.Transform method}}

\begin{fulllineitems}
\phantomsection\label{\detokenize{deepdrr.geo:deepdrr.geo.core.Transform.inverse}}
\pysigstartsignatures
\pysiglinewithargsret{\sphinxbfcode{\sphinxupquote{inverse}}}{}{{ $\rightarrow$ {\hyperref[\detokenize{deepdrr.geo:deepdrr.geo.core.FrameTransform}]{\sphinxcrossref{FrameTransform}}}}}
\pysiglinewithargsret{\sphinxbfcode{\sphinxupquote{inverse}}}{}{{ $\rightarrow$ {\hyperref[\detokenize{deepdrr.geo:deepdrr.geo.core.Transform}]{\sphinxcrossref{Transform}}}}}
\pysigstopsignatures
\sphinxAtStartPar
Get the inverse of the Transform.
\begin{quote}\begin{description}
\sphinxlineitem{Returns}
\sphinxAtStartPar
a Transform (or subclass) that is well\sphinxhyphen{}defined as the inverse of this transform.

\sphinxlineitem{Return type}
\sphinxAtStartPar
({\hyperref[\detokenize{deepdrr.geo:deepdrr.geo.core.Transform}]{\sphinxcrossref{Transform}}})

\sphinxlineitem{Raises}
\sphinxAtStartPar
\sphinxstyleliteralstrong{\sphinxupquote{NotImplementedError}} \textendash{} if \_inv is None and method is not overriden.

\end{description}\end{quote}

\end{fulllineitems}


\end{fulllineitems}

\index{Vector (class in deepdrr.geo.core)@\spxentry{Vector}\spxextra{class in deepdrr.geo.core}}

\begin{fulllineitems}
\phantomsection\label{\detokenize{deepdrr.geo:deepdrr.geo.core.Vector}}
\pysigstartsignatures
\pysiglinewithargsret{\sphinxbfcode{\sphinxupquote{class\DUrole{w,w}{  }}}\sphinxcode{\sphinxupquote{deepdrr.geo.core.}}\sphinxbfcode{\sphinxupquote{Vector}}}{\sphinxparam{\DUrole{n,n}{data}\DUrole{p,p}{:}\DUrole{w,w}{  }\DUrole{n,n}{ndarray}}}{}
\pysigstopsignatures
\sphinxAtStartPar
Bases: {\hyperref[\detokenize{deepdrr.geo:deepdrr.geo.core.PointOrVector}]{\sphinxcrossref{\sphinxcode{\sphinxupquote{PointOrVector}}}}}, {\hyperref[\detokenize{deepdrr.geo:deepdrr.geo.core.HasDirection}]{\sphinxcrossref{\sphinxcode{\sphinxupquote{HasDirection}}}}}
\index{as\_point() (deepdrr.geo.core.Vector method)@\spxentry{as\_point()}\spxextra{deepdrr.geo.core.Vector method}}

\begin{fulllineitems}
\phantomsection\label{\detokenize{deepdrr.geo:deepdrr.geo.core.Vector.as_point}}
\pysigstartsignatures
\pysiglinewithargsret{\sphinxbfcode{\sphinxupquote{as\_point}}}{}{{ $\rightarrow$ {\hyperref[\detokenize{deepdrr.geo:deepdrr.geo.core.Point}]{\sphinxcrossref{Point}}}}}
\pysigstopsignatures
\sphinxAtStartPar
Gets the point with the same numerical representation as this vector.

\end{fulllineitems}

\index{cross() (deepdrr.geo.core.Vector method)@\spxentry{cross()}\spxextra{deepdrr.geo.core.Vector method}}

\begin{fulllineitems}
\phantomsection\label{\detokenize{deepdrr.geo:deepdrr.geo.core.Vector.cross}}
\pysigstartsignatures
\pysiglinewithargsret{\sphinxbfcode{\sphinxupquote{cross}}}{\sphinxparam{\DUrole{n,n}{other}\DUrole{p,p}{:}\DUrole{w,w}{  }\DUrole{n,n}{{\hyperref[\detokenize{deepdrr.geo:deepdrr.geo.core.Vector}]{\sphinxcrossref{Vector}}}}}}{{ $\rightarrow$ {\hyperref[\detokenize{deepdrr.geo:deepdrr.geo.core.Vector3D}]{\sphinxcrossref{Vector3D}}}}}
\pysigstopsignatures
\end{fulllineitems}

\index{data (deepdrr.geo.core.Vector attribute)@\spxentry{data}\spxextra{deepdrr.geo.core.Vector attribute}}

\begin{fulllineitems}
\phantomsection\label{\detokenize{deepdrr.geo:deepdrr.geo.core.Vector.data}}
\pysigstartsignatures
\pysigline{\sphinxbfcode{\sphinxupquote{data}}\sphinxbfcode{\sphinxupquote{\DUrole{p,p}{:}\DUrole{w,w}{  }ndarray}}}
\pysigstopsignatures
\end{fulllineitems}

\index{dot() (deepdrr.geo.core.Vector method)@\spxentry{dot()}\spxextra{deepdrr.geo.core.Vector method}}

\begin{fulllineitems}
\phantomsection\label{\detokenize{deepdrr.geo:deepdrr.geo.core.Vector.dot}}
\pysigstartsignatures
\pysiglinewithargsret{\sphinxbfcode{\sphinxupquote{dot}}}{\sphinxparam{\DUrole{n,n}{other}}}{{ $\rightarrow$ float}}
\pysigstopsignatures
\end{fulllineitems}

\index{from\_any() (deepdrr.geo.core.Vector class method)@\spxentry{from\_any()}\spxextra{deepdrr.geo.core.Vector class method}}

\begin{fulllineitems}
\phantomsection\label{\detokenize{deepdrr.geo:deepdrr.geo.core.Vector.from_any}}
\pysigstartsignatures
\pysiglinewithargsret{\sphinxbfcode{\sphinxupquote{classmethod\DUrole{w,w}{  }}}\sphinxbfcode{\sphinxupquote{from\_any}}}{\sphinxparam{\DUrole{n,n}{other}\DUrole{p,p}{:}\DUrole{w,w}{  }\DUrole{n,n}{ndarray\DUrole{w,w}{  }\DUrole{p,p}{|}\DUrole{w,w}{  }{\hyperref[\detokenize{deepdrr.geo:deepdrr.geo.core.Vector}]{\sphinxcrossref{Vector}}}}}}{}
\pysigstopsignatures
\sphinxAtStartPar
If other is not a Vector, make it one.

\end{fulllineitems}

\index{from\_array() (deepdrr.geo.core.Vector class method)@\spxentry{from\_array()}\spxextra{deepdrr.geo.core.Vector class method}}

\begin{fulllineitems}
\phantomsection\label{\detokenize{deepdrr.geo:deepdrr.geo.core.Vector.from_array}}
\pysigstartsignatures
\pysiglinewithargsret{\sphinxbfcode{\sphinxupquote{classmethod\DUrole{w,w}{  }}}\sphinxbfcode{\sphinxupquote{from\_array}}}{\sphinxparam{\DUrole{n,n}{v}\DUrole{p,p}{:}\DUrole{w,w}{  }\DUrole{n,n}{ndarray}}}{{ $\rightarrow$ T}}
\pysigstopsignatures
\sphinxAtStartPar
Create a homogeneous object from its non\sphinxhyphen{}homogeous representation as an array.

\end{fulllineitems}

\index{get\_direction() (deepdrr.geo.core.Vector method)@\spxentry{get\_direction()}\spxextra{deepdrr.geo.core.Vector method}}

\begin{fulllineitems}
\phantomsection\label{\detokenize{deepdrr.geo:deepdrr.geo.core.Vector.get_direction}}
\pysigstartsignatures
\pysiglinewithargsret{\sphinxbfcode{\sphinxupquote{get\_direction}}}{}{{ $\rightarrow$ Self}}
\pysigstopsignatures
\sphinxAtStartPar
Gets the vector with the same numerical representation as this vector.

\end{fulllineitems}

\index{hat() (deepdrr.geo.core.Vector method)@\spxentry{hat()}\spxextra{deepdrr.geo.core.Vector method}}

\begin{fulllineitems}
\phantomsection\label{\detokenize{deepdrr.geo:deepdrr.geo.core.Vector.hat}}
\pysigstartsignatures
\pysiglinewithargsret{\sphinxbfcode{\sphinxupquote{hat}}}{}{{ $\rightarrow$ Self}}
\pysigstopsignatures
\end{fulllineitems}

\index{normalized() (deepdrr.geo.core.Vector method)@\spxentry{normalized()}\spxextra{deepdrr.geo.core.Vector method}}

\begin{fulllineitems}
\phantomsection\label{\detokenize{deepdrr.geo:deepdrr.geo.core.Vector.normalized}}
\pysigstartsignatures
\pysiglinewithargsret{\sphinxbfcode{\sphinxupquote{normalized}}}{}{{ $\rightarrow$ Self}}
\pysigstopsignatures
\end{fulllineitems}


\end{fulllineitems}

\index{Vector2D (class in deepdrr.geo.core)@\spxentry{Vector2D}\spxextra{class in deepdrr.geo.core}}

\begin{fulllineitems}
\phantomsection\label{\detokenize{deepdrr.geo:deepdrr.geo.core.Vector2D}}
\pysigstartsignatures
\pysiglinewithargsret{\sphinxbfcode{\sphinxupquote{class\DUrole{w,w}{  }}}\sphinxcode{\sphinxupquote{deepdrr.geo.core.}}\sphinxbfcode{\sphinxupquote{Vector2D}}}{\sphinxparam{\DUrole{n,n}{data}\DUrole{p,p}{:}\DUrole{w,w}{  }\DUrole{n,n}{ndarray}}}{}
\pysigstopsignatures
\sphinxAtStartPar
Bases: {\hyperref[\detokenize{deepdrr.geo:deepdrr.geo.core.Vector}]{\sphinxcrossref{\sphinxcode{\sphinxupquote{Vector}}}}}

\sphinxAtStartPar
Homogeneous vector in 2D, represented as an array with {[}x, y, 0{]}
\index{data (deepdrr.geo.core.Vector2D attribute)@\spxentry{data}\spxextra{deepdrr.geo.core.Vector2D attribute}}

\begin{fulllineitems}
\phantomsection\label{\detokenize{deepdrr.geo:deepdrr.geo.core.Vector2D.data}}
\pysigstartsignatures
\pysigline{\sphinxbfcode{\sphinxupquote{data}}\sphinxbfcode{\sphinxupquote{\DUrole{p,p}{:}\DUrole{w,w}{  }ndarray}}}
\pysigstopsignatures
\end{fulllineitems}

\index{dim (deepdrr.geo.core.Vector2D attribute)@\spxentry{dim}\spxextra{deepdrr.geo.core.Vector2D attribute}}

\begin{fulllineitems}
\phantomsection\label{\detokenize{deepdrr.geo:deepdrr.geo.core.Vector2D.dim}}
\pysigstartsignatures
\pysigline{\sphinxbfcode{\sphinxupquote{dim}}\sphinxbfcode{\sphinxupquote{\DUrole{w,w}{  }\DUrole{p,p}{=}\DUrole{w,w}{  }2}}}
\pysigstopsignatures
\end{fulllineitems}

\index{perpendicular() (deepdrr.geo.core.Vector2D method)@\spxentry{perpendicular()}\spxextra{deepdrr.geo.core.Vector2D method}}

\begin{fulllineitems}
\phantomsection\label{\detokenize{deepdrr.geo:deepdrr.geo.core.Vector2D.perpendicular}}
\pysigstartsignatures
\pysiglinewithargsret{\sphinxbfcode{\sphinxupquote{perpendicular}}}{\sphinxparam{\DUrole{n,n}{random}\DUrole{p,p}{:}\DUrole{w,w}{  }\DUrole{n,n}{bool}\DUrole{w,w}{  }\DUrole{o,o}{=}\DUrole{w,w}{  }\DUrole{default_value}{False}}}{{ $\rightarrow$ {\hyperref[\detokenize{deepdrr.geo:deepdrr.geo.core.Vector2D}]{\sphinxcrossref{Vector2D}}}}}
\pysigstopsignatures
\sphinxAtStartPar
Find an arbitrary perpendicular vector to self.
\begin{quote}\begin{description}
\sphinxlineitem{Parameters}
\sphinxAtStartPar
\sphinxstyleliteralstrong{\sphinxupquote{random}} \textendash{} Whether to randomize the vector’s direction in
the perpendicular plane, drawing from {[}0, 2pi).
Defaults to False.

\sphinxlineitem{Returns}
\sphinxAtStartPar
\begin{description}
\sphinxlineitem{A vector in 3D space, perpendicular}
\sphinxAtStartPar
to the original.

\end{description}


\sphinxlineitem{Return type}
\sphinxAtStartPar
{\hyperref[\detokenize{deepdrr.geo:deepdrr.geo.core.Vector3D}]{\sphinxcrossref{Vector3D}}}

\end{description}\end{quote}

\end{fulllineitems}


\end{fulllineitems}

\index{Vector3D (class in deepdrr.geo.core)@\spxentry{Vector3D}\spxextra{class in deepdrr.geo.core}}

\begin{fulllineitems}
\phantomsection\label{\detokenize{deepdrr.geo:deepdrr.geo.core.Vector3D}}
\pysigstartsignatures
\pysiglinewithargsret{\sphinxbfcode{\sphinxupquote{class\DUrole{w,w}{  }}}\sphinxcode{\sphinxupquote{deepdrr.geo.core.}}\sphinxbfcode{\sphinxupquote{Vector3D}}}{\sphinxparam{\DUrole{n,n}{data}\DUrole{p,p}{:}\DUrole{w,w}{  }\DUrole{n,n}{ndarray}}}{}
\pysigstopsignatures
\sphinxAtStartPar
Bases: {\hyperref[\detokenize{deepdrr.geo:deepdrr.geo.core.Vector}]{\sphinxcrossref{\sphinxcode{\sphinxupquote{Vector}}}}}

\sphinxAtStartPar
Homogeneous vector in 3D, represented as an array with {[}x, y, z, 0{]}.

\sphinxAtStartPar
A 3d vector still cannot be projected by a camera, because it doesn’t have a location.
\index{as\_plane() (deepdrr.geo.core.Vector3D method)@\spxentry{as\_plane()}\spxextra{deepdrr.geo.core.Vector3D method}}

\begin{fulllineitems}
\phantomsection\label{\detokenize{deepdrr.geo:deepdrr.geo.core.Vector3D.as_plane}}
\pysigstartsignatures
\pysiglinewithargsret{\sphinxbfcode{\sphinxupquote{as\_plane}}}{}{{ $\rightarrow$ {\hyperref[\detokenize{deepdrr.geo:deepdrr.geo.hyperplane.Plane}]{\sphinxcrossref{Plane}}}}}
\pysigstopsignatures
\sphinxAtStartPar
Get the plane through the origin with this vector as its normal.

\end{fulllineitems}

\index{data (deepdrr.geo.core.Vector3D attribute)@\spxentry{data}\spxextra{deepdrr.geo.core.Vector3D attribute}}

\begin{fulllineitems}
\phantomsection\label{\detokenize{deepdrr.geo:deepdrr.geo.core.Vector3D.data}}
\pysigstartsignatures
\pysigline{\sphinxbfcode{\sphinxupquote{data}}\sphinxbfcode{\sphinxupquote{\DUrole{p,p}{:}\DUrole{w,w}{  }ndarray}}}
\pysigstopsignatures
\end{fulllineitems}

\index{dim (deepdrr.geo.core.Vector3D attribute)@\spxentry{dim}\spxextra{deepdrr.geo.core.Vector3D attribute}}

\begin{fulllineitems}
\phantomsection\label{\detokenize{deepdrr.geo:deepdrr.geo.core.Vector3D.dim}}
\pysigstartsignatures
\pysigline{\sphinxbfcode{\sphinxupquote{dim}}\sphinxbfcode{\sphinxupquote{\DUrole{w,w}{  }\DUrole{p,p}{=}\DUrole{w,w}{  }3}}}
\pysigstopsignatures
\end{fulllineitems}

\index{perpendicular() (deepdrr.geo.core.Vector3D method)@\spxentry{perpendicular()}\spxextra{deepdrr.geo.core.Vector3D method}}

\begin{fulllineitems}
\phantomsection\label{\detokenize{deepdrr.geo:deepdrr.geo.core.Vector3D.perpendicular}}
\pysigstartsignatures
\pysiglinewithargsret{\sphinxbfcode{\sphinxupquote{perpendicular}}}{\sphinxparam{\DUrole{n,n}{random}\DUrole{p,p}{:}\DUrole{w,w}{  }\DUrole{n,n}{bool}\DUrole{w,w}{  }\DUrole{o,o}{=}\DUrole{w,w}{  }\DUrole{default_value}{False}}}{{ $\rightarrow$ {\hyperref[\detokenize{deepdrr.geo:deepdrr.geo.core.Vector3D}]{\sphinxcrossref{Vector3D}}}}}
\pysigstopsignatures
\sphinxAtStartPar
Find an arbitrary perpendicular vector to self.
\begin{quote}\begin{description}
\sphinxlineitem{Parameters}
\sphinxAtStartPar
\sphinxstyleliteralstrong{\sphinxupquote{random}} \textendash{} Whether to randomize the vector’s direction in
the perpendicular plane, drawing from {[}0, 2pi).
Defaults to False.

\sphinxlineitem{Returns}
\sphinxAtStartPar
\begin{description}
\sphinxlineitem{A vector in 3D space, perpendicular}
\sphinxAtStartPar
to the original.

\end{description}


\sphinxlineitem{Return type}
\sphinxAtStartPar
{\hyperref[\detokenize{deepdrr.geo:deepdrr.geo.core.Vector3D}]{\sphinxcrossref{Vector3D}}}

\end{description}\end{quote}

\end{fulllineitems}

\index{rotate() (deepdrr.geo.core.Vector3D method)@\spxentry{rotate()}\spxextra{deepdrr.geo.core.Vector3D method}}

\begin{fulllineitems}
\phantomsection\label{\detokenize{deepdrr.geo:deepdrr.geo.core.Vector3D.rotate}}
\pysigstartsignatures
\pysiglinewithargsret{\sphinxbfcode{\sphinxupquote{rotate}}}{\sphinxparam{\DUrole{n,n}{n}\DUrole{p,p}{:}\DUrole{w,w}{  }\DUrole{n,n}{{\hyperref[\detokenize{deepdrr.geo:deepdrr.geo.core.Vector3D}]{\sphinxcrossref{Vector3D}}}}}\sphinxparamcomma \sphinxparam{\DUrole{n,n}{theta}\DUrole{p,p}{:}\DUrole{w,w}{  }\DUrole{n,n}{float\DUrole{w,w}{  }\DUrole{p,p}{|}\DUrole{w,w}{  }None}\DUrole{w,w}{  }\DUrole{o,o}{=}\DUrole{w,w}{  }\DUrole{default_value}{None}}}{{ $\rightarrow$ {\hyperref[\detokenize{deepdrr.geo:deepdrr.geo.core.Vector3D}]{\sphinxcrossref{Vector3D}}}}}
\pysigstopsignatures
\sphinxAtStartPar
Rotate self by the given vector.
\begin{quote}\begin{description}
\sphinxlineitem{Parameters}\begin{itemize}
\item {} 
\sphinxAtStartPar
\sphinxstyleliteralstrong{\sphinxupquote{n}} ({\hyperref[\detokenize{deepdrr.geo:deepdrr.geo.core.Vector}]{\sphinxcrossref{\sphinxstyleliteralemphasis{\sphinxupquote{Vector}}}}}) \textendash{} the axis of rotation. If theta is None, the magnitude of this vector is
used. Otherwise, it is ignored.

\item {} 
\sphinxAtStartPar
\sphinxstyleliteralstrong{\sphinxupquote{theta}} (\sphinxstyleliteralemphasis{\sphinxupquote{float}}\sphinxstyleliteralemphasis{\sphinxupquote{, }}\sphinxstyleliteralemphasis{\sphinxupquote{optional}}) \textendash{} the angle of rotation. Defaults to None.

\end{itemize}

\sphinxlineitem{Returns}
\sphinxAtStartPar
the rotated vector.

\sphinxlineitem{Return type}
\sphinxAtStartPar
{\hyperref[\detokenize{deepdrr.geo:deepdrr.geo.core.Vector}]{\sphinxcrossref{Vector}}}

\end{description}\end{quote}

\end{fulllineitems}

\index{rotvec\_to() (deepdrr.geo.core.Vector3D method)@\spxentry{rotvec\_to()}\spxextra{deepdrr.geo.core.Vector3D method}}

\begin{fulllineitems}
\phantomsection\label{\detokenize{deepdrr.geo:deepdrr.geo.core.Vector3D.rotvec_to}}
\pysigstartsignatures
\pysiglinewithargsret{\sphinxbfcode{\sphinxupquote{rotvec\_to}}}{\sphinxparam{\DUrole{n,n}{other}\DUrole{p,p}{:}\DUrole{w,w}{  }\DUrole{n,n}{{\hyperref[\detokenize{deepdrr.geo:deepdrr.geo.core.Vector3D}]{\sphinxcrossref{Vector3D}}}}}}{{ $\rightarrow$ {\hyperref[\detokenize{deepdrr.geo:deepdrr.geo.core.Vector3D}]{\sphinxcrossref{Vector3D}}}}}
\pysigstopsignatures
\sphinxAtStartPar
Get the rotvec that rotates self to other.

\end{fulllineitems}


\end{fulllineitems}

\index{f() (in module deepdrr.geo.core)@\spxentry{f()}\spxextra{in module deepdrr.geo.core}}

\begin{fulllineitems}
\phantomsection\label{\detokenize{deepdrr.geo:deepdrr.geo.core.f}}
\pysigstartsignatures
\pysiglinewithargsret{\sphinxcode{\sphinxupquote{deepdrr.geo.core.}}\sphinxbfcode{\sphinxupquote{f}}}{\sphinxparam{\DUrole{o,o}{*}\DUrole{n,n}{args}}}{{ $\rightarrow$ {\hyperref[\detokenize{deepdrr.geo:deepdrr.geo.core.FrameTransform}]{\sphinxcrossref{FrameTransform}}}}}
\pysigstopsignatures
\sphinxAtStartPar
Convenience function for creating a 3D frame transform.

\sphinxAtStartPar
The output depends on how the function is called:
frame\_transform() \sphinxhyphen{}\textgreater{} 3D identity transform
frame\_transform(None) \sphinxhyphen{}\textgreater{} 3D identity transform
frame\_transform(scalar) \sphinxhyphen{}\textgreater{} FrameTransform.from\_scaling(scalar)
frame\_transform(ft: FrameTransform) \sphinxhyphen{}\textgreater{} ft
frame\_transform(data: np.ndarray{[}4,4{]}) \sphinxhyphen{}\textgreater{} FrameTransform(data)
frame\_transform(R: Rotation | np.ndarray{[}3,3{]}) \sphinxhyphen{}\textgreater{} FrameTransform.from\_rt(R)
frame\_transform(t: Point | np.ndarray{[}3{]}) \sphinxhyphen{}\textgreater{} FrameTransform.from\_translation(t)
frame\_transform((R, t)) \sphinxhyphen{}\textgreater{} FrameTransform.from\_rt(R, t)
frame\_transform(R, t) \sphinxhyphen{}\textgreater{} FrameTransform.from\_rt(R, t)
frame\_transform({[}a00, a10, a20, a01, a11, a21, a02, a12, a22, a03, a13, a23{]}) \sphinxhyphen{}\textgreater{} FrameTransform({[}
\begin{quote}

\sphinxAtStartPar
{[}a00, a01, a02, a03{]},
{[}a10, a11, a12, a13{]},
{[}a20, a21, a22, a23{]},
{[}0, 0, 0, 1{]}{]}
\end{quote}

\sphinxAtStartPar
)

\sphinxAtStartPar
R maybe be given as a (3,3) matrix or as a 9\sphinxhyphen{}vector. If provided as a 9\sphinxhyphen{}vector, column major order is assumed,
such that (a11, a21, a31, a12, a22, a32, a13, a23, a33) is the rotation matrix.
{[}{[}a11, a12, a13{]},
\begin{quote}

\sphinxAtStartPar
{[}a21, a22, a23{]},
{[}a31, a32, a33{]}{]}
\end{quote}

\sphinxAtStartPar
{[}R | t{]} may be given as a (12,) array\sphinxhyphen{}like, where the first 9 elements are the rotation in column major order,  and the last 3 are the translation.

\sphinxAtStartPar
If a string provided, it is converted to an array with whitespace separator.
\begin{quote}\begin{description}
\sphinxlineitem{Returns}
\sphinxAtStartPar
{[}description{]}

\sphinxlineitem{Return type}
\sphinxAtStartPar
{\hyperref[\detokenize{deepdrr.geo:deepdrr.geo.core.FrameTransform}]{\sphinxcrossref{FrameTransform}}}

\end{description}\end{quote}

\end{fulllineitems}

\index{frame\_transform() (in module deepdrr.geo.core)@\spxentry{frame\_transform()}\spxextra{in module deepdrr.geo.core}}

\begin{fulllineitems}
\phantomsection\label{\detokenize{deepdrr.geo:deepdrr.geo.core.frame_transform}}
\pysigstartsignatures
\pysiglinewithargsret{\sphinxcode{\sphinxupquote{deepdrr.geo.core.}}\sphinxbfcode{\sphinxupquote{frame\_transform}}}{\sphinxparam{\DUrole{o,o}{*}\DUrole{n,n}{args}}}{{ $\rightarrow$ {\hyperref[\detokenize{deepdrr.geo:deepdrr.geo.core.FrameTransform}]{\sphinxcrossref{FrameTransform}}}}}
\pysigstopsignatures
\sphinxAtStartPar
Convenience function for creating a 3D frame transform.

\sphinxAtStartPar
The output depends on how the function is called:
frame\_transform() \sphinxhyphen{}\textgreater{} 3D identity transform
frame\_transform(None) \sphinxhyphen{}\textgreater{} 3D identity transform
frame\_transform(scalar) \sphinxhyphen{}\textgreater{} FrameTransform.from\_scaling(scalar)
frame\_transform(ft: FrameTransform) \sphinxhyphen{}\textgreater{} ft
frame\_transform(data: np.ndarray{[}4,4{]}) \sphinxhyphen{}\textgreater{} FrameTransform(data)
frame\_transform(R: Rotation | np.ndarray{[}3,3{]}) \sphinxhyphen{}\textgreater{} FrameTransform.from\_rt(R)
frame\_transform(t: Point | np.ndarray{[}3{]}) \sphinxhyphen{}\textgreater{} FrameTransform.from\_translation(t)
frame\_transform((R, t)) \sphinxhyphen{}\textgreater{} FrameTransform.from\_rt(R, t)
frame\_transform(R, t) \sphinxhyphen{}\textgreater{} FrameTransform.from\_rt(R, t)
frame\_transform({[}a00, a10, a20, a01, a11, a21, a02, a12, a22, a03, a13, a23{]}) \sphinxhyphen{}\textgreater{} FrameTransform({[}
\begin{quote}

\sphinxAtStartPar
{[}a00, a01, a02, a03{]},
{[}a10, a11, a12, a13{]},
{[}a20, a21, a22, a23{]},
{[}0, 0, 0, 1{]}{]}
\end{quote}

\sphinxAtStartPar
)

\sphinxAtStartPar
R maybe be given as a (3,3) matrix or as a 9\sphinxhyphen{}vector. If provided as a 9\sphinxhyphen{}vector, column major order is assumed,
such that (a11, a21, a31, a12, a22, a32, a13, a23, a33) is the rotation matrix.
{[}{[}a11, a12, a13{]},
\begin{quote}

\sphinxAtStartPar
{[}a21, a22, a23{]},
{[}a31, a32, a33{]}{]}
\end{quote}

\sphinxAtStartPar
{[}R | t{]} may be given as a (12,) array\sphinxhyphen{}like, where the first 9 elements are the rotation in column major order,  and the last 3 are the translation.

\sphinxAtStartPar
If a string provided, it is converted to an array with whitespace separator.
\begin{quote}\begin{description}
\sphinxlineitem{Returns}
\sphinxAtStartPar
{[}description{]}

\sphinxlineitem{Return type}
\sphinxAtStartPar
{\hyperref[\detokenize{deepdrr.geo:deepdrr.geo.core.FrameTransform}]{\sphinxcrossref{FrameTransform}}}

\end{description}\end{quote}

\end{fulllineitems}

\index{get\_data() (in module deepdrr.geo.core)@\spxentry{get\_data()}\spxextra{in module deepdrr.geo.core}}

\begin{fulllineitems}
\phantomsection\label{\detokenize{deepdrr.geo:deepdrr.geo.core.get_data}}
\pysigstartsignatures
\pysiglinewithargsret{\sphinxcode{\sphinxupquote{deepdrr.geo.core.}}\sphinxbfcode{\sphinxupquote{get\_data}}}{\sphinxparam{\DUrole{n,n}{x}\DUrole{p,p}{:}\DUrole{w,w}{  }\DUrole{n,n}{{\hyperref[\detokenize{deepdrr.geo:deepdrr.geo.core.HomogeneousObject}]{\sphinxcrossref{HomogeneousObject}}}\DUrole{w,w}{  }\DUrole{p,p}{|}\DUrole{w,w}{  }List\DUrole{p,p}{{[}}{\hyperref[\detokenize{deepdrr.geo:deepdrr.geo.core.HomogeneousObject}]{\sphinxcrossref{HomogeneousObject}}}\DUrole{p,p}{{]}}}}}{{ $\rightarrow$ ndarray}}
\pysigstopsignatures
\end{fulllineitems}

\index{p() (in module deepdrr.geo.core)@\spxentry{p()}\spxextra{in module deepdrr.geo.core}}

\begin{fulllineitems}
\phantomsection\label{\detokenize{deepdrr.geo:deepdrr.geo.core.p}}
\pysigstartsignatures
\pysiglinewithargsret{\sphinxcode{\sphinxupquote{deepdrr.geo.core.}}\sphinxbfcode{\sphinxupquote{p}}}{\sphinxparam{\DUrole{o,o}{*}\DUrole{n,n}{args}}}{}
\pysigstopsignatures
\end{fulllineitems}

\index{point() (in module deepdrr.geo.core)@\spxentry{point()}\spxextra{in module deepdrr.geo.core}}

\begin{fulllineitems}
\phantomsection\label{\detokenize{deepdrr.geo:deepdrr.geo.core.point}}
\pysigstartsignatures
\pysiglinewithargsret{\sphinxcode{\sphinxupquote{deepdrr.geo.core.}}\sphinxbfcode{\sphinxupquote{point}}}{\sphinxparam{\DUrole{n,n}{p}\DUrole{p,p}{:}\DUrole{w,w}{  }\DUrole{n,n}{P}}}{{ $\rightarrow$ P}}
\pysiglinewithargsret{\sphinxcode{\sphinxupquote{deepdrr.geo.core.}}\sphinxbfcode{\sphinxupquote{point}}}{\sphinxparam{\DUrole{n,n}{v}\DUrole{p,p}{:}\DUrole{w,w}{  }\DUrole{n,n}{{\hyperref[\detokenize{deepdrr.geo:deepdrr.geo.core.Vector2D}]{\sphinxcrossref{Vector2D}}}}}}{{ $\rightarrow$ {\hyperref[\detokenize{deepdrr.geo:deepdrr.geo.core.Point2D}]{\sphinxcrossref{Point2D}}}}}
\pysiglinewithargsret{\sphinxcode{\sphinxupquote{deepdrr.geo.core.}}\sphinxbfcode{\sphinxupquote{point}}}{\sphinxparam{\DUrole{n,n}{v}\DUrole{p,p}{:}\DUrole{w,w}{  }\DUrole{n,n}{{\hyperref[\detokenize{deepdrr.geo:deepdrr.geo.core.Vector3D}]{\sphinxcrossref{Vector3D}}}}}}{{ $\rightarrow$ {\hyperref[\detokenize{deepdrr.geo:deepdrr.geo.core.Point3D}]{\sphinxcrossref{Point3D}}}}}
\pysiglinewithargsret{\sphinxcode{\sphinxupquote{deepdrr.geo.core.}}\sphinxbfcode{\sphinxupquote{point}}}{\sphinxparam{\DUrole{n,n}{x}\DUrole{p,p}{:}\DUrole{w,w}{  }\DUrole{n,n}{float}}\sphinxparamcomma \sphinxparam{\DUrole{n,n}{y}\DUrole{p,p}{:}\DUrole{w,w}{  }\DUrole{n,n}{float}}}{{ $\rightarrow$ {\hyperref[\detokenize{deepdrr.geo:deepdrr.geo.core.Point2D}]{\sphinxcrossref{Point2D}}}}}
\pysiglinewithargsret{\sphinxcode{\sphinxupquote{deepdrr.geo.core.}}\sphinxbfcode{\sphinxupquote{point}}}{\sphinxparam{\DUrole{n,n}{x}\DUrole{p,p}{:}\DUrole{w,w}{  }\DUrole{n,n}{float}}\sphinxparamcomma \sphinxparam{\DUrole{n,n}{y}\DUrole{p,p}{:}\DUrole{w,w}{  }\DUrole{n,n}{float}}\sphinxparamcomma \sphinxparam{\DUrole{n,n}{z}\DUrole{p,p}{:}\DUrole{w,w}{  }\DUrole{n,n}{float}}}{{ $\rightarrow$ {\hyperref[\detokenize{deepdrr.geo:deepdrr.geo.core.Point3D}]{\sphinxcrossref{Point3D}}}}}
\pysiglinewithargsret{\sphinxcode{\sphinxupquote{deepdrr.geo.core.}}\sphinxbfcode{\sphinxupquote{point}}}{\sphinxparam{\DUrole{n,n}{x}\DUrole{p,p}{:}\DUrole{w,w}{  }\DUrole{n,n}{ndarray}}}{{ $\rightarrow$ {\hyperref[\detokenize{deepdrr.geo:deepdrr.geo.core.Point}]{\sphinxcrossref{Point}}}}}
\pysiglinewithargsret{\sphinxcode{\sphinxupquote{deepdrr.geo.core.}}\sphinxbfcode{\sphinxupquote{point}}}{\sphinxparam{\DUrole{n,n}{x}\DUrole{p,p}{:}\DUrole{w,w}{  }\DUrole{n,n}{{\hyperref[\detokenize{deepdrr.geo:deepdrr.geo.core.HasLocation}]{\sphinxcrossref{HasLocation}}}}}}{{ $\rightarrow$ {\hyperref[\detokenize{deepdrr.geo:deepdrr.geo.core.Point}]{\sphinxcrossref{Point}}}}}
\pysiglinewithargsret{\sphinxcode{\sphinxupquote{deepdrr.geo.core.}}\sphinxbfcode{\sphinxupquote{point}}}{\sphinxparam{\DUrole{o,o}{*}\DUrole{n,n}{args}\DUrole{p,p}{:}\DUrole{w,w}{  }\DUrole{n,n}{Any}}}{{ $\rightarrow$ {\hyperref[\detokenize{deepdrr.geo:deepdrr.geo.core.Point}]{\sphinxcrossref{Point}}}}}
\pysigstopsignatures
\sphinxAtStartPar
The preferred method for creating a point.

\sphinxAtStartPar
There are three ways to create a point using \sphinxtitleref{point()}.
\sphinxhyphen{} Pass the coordinates as separate arguments. For instance, \sphinxtitleref{point(0, 0)} returns the 2D homogeneous point for the origin \sphinxtitleref{Point2D({[}0, 0, 1{]})}.
\sphinxhyphen{} Pass a numpy array containing the non\sphinxhyphen{}homogeneous representation of the point. For example \sphinxtitleref{point(np.ndarray({[}0, 1, 2{]}))} is the 3D homogeneous point \sphinxtitleref{Point3D({[}0, 1, 2, 1{]})}.
\sphinxhyphen{} Pass a Point2D or Point3D instance, in which case \sphinxtitleref{point()} just returns the first argument.

\sphinxAtStartPar
\sphinxtitleref{point()} shoud NOT be given a numpy array containing the homogeneous data. In this case, use the \sphinxtitleref{Point2D} and \sphinxtitleref{Point3D} constructors directly.
\begin{quote}\begin{description}
\sphinxlineitem{Raises}
\sphinxAtStartPar
\sphinxstyleliteralstrong{\sphinxupquote{ValueError}} \textendash{} if arguments cannot be parsed as data for a point.

\sphinxlineitem{Returns}
\sphinxAtStartPar
Point2D or Point3D.

\sphinxlineitem{Return type}
\sphinxAtStartPar
Union{[}{\hyperref[\detokenize{deepdrr.geo:deepdrr.geo.core.Point2D}]{\sphinxcrossref{Point2D}}}, {\hyperref[\detokenize{deepdrr.geo:deepdrr.geo.core.Point3D}]{\sphinxcrossref{Point3D}}}{]}

\end{description}\end{quote}

\end{fulllineitems}

\index{v() (in module deepdrr.geo.core)@\spxentry{v()}\spxextra{in module deepdrr.geo.core}}

\begin{fulllineitems}
\phantomsection\label{\detokenize{deepdrr.geo:deepdrr.geo.core.v}}
\pysigstartsignatures
\pysiglinewithargsret{\sphinxcode{\sphinxupquote{deepdrr.geo.core.}}\sphinxbfcode{\sphinxupquote{v}}}{\sphinxparam{\DUrole{o,o}{*}\DUrole{n,n}{args}}}{}
\pysigstopsignatures
\end{fulllineitems}

\index{vector() (in module deepdrr.geo.core)@\spxentry{vector()}\spxextra{in module deepdrr.geo.core}}

\begin{fulllineitems}
\phantomsection\label{\detokenize{deepdrr.geo:deepdrr.geo.core.vector}}
\pysigstartsignatures
\pysiglinewithargsret{\sphinxcode{\sphinxupquote{deepdrr.geo.core.}}\sphinxbfcode{\sphinxupquote{vector}}}{\sphinxparam{\DUrole{n,n}{v}\DUrole{p,p}{:}\DUrole{w,w}{  }\DUrole{n,n}{V}}}{{ $\rightarrow$ V}}
\pysiglinewithargsret{\sphinxcode{\sphinxupquote{deepdrr.geo.core.}}\sphinxbfcode{\sphinxupquote{vector}}}{\sphinxparam{\DUrole{n,n}{p}\DUrole{p,p}{:}\DUrole{w,w}{  }\DUrole{n,n}{{\hyperref[\detokenize{deepdrr.geo:deepdrr.geo.core.Point2D}]{\sphinxcrossref{Point2D}}}}}}{{ $\rightarrow$ {\hyperref[\detokenize{deepdrr.geo:deepdrr.geo.core.Vector2D}]{\sphinxcrossref{Vector2D}}}}}
\pysiglinewithargsret{\sphinxcode{\sphinxupquote{deepdrr.geo.core.}}\sphinxbfcode{\sphinxupquote{vector}}}{\sphinxparam{\DUrole{n,n}{p}\DUrole{p,p}{:}\DUrole{w,w}{  }\DUrole{n,n}{{\hyperref[\detokenize{deepdrr.geo:deepdrr.geo.core.Point3D}]{\sphinxcrossref{Point3D}}}}}}{{ $\rightarrow$ {\hyperref[\detokenize{deepdrr.geo:deepdrr.geo.core.Vector3D}]{\sphinxcrossref{Vector3D}}}}}
\pysiglinewithargsret{\sphinxcode{\sphinxupquote{deepdrr.geo.core.}}\sphinxbfcode{\sphinxupquote{vector}}}{\sphinxparam{\DUrole{n,n}{x}\DUrole{p,p}{:}\DUrole{w,w}{  }\DUrole{n,n}{float}}\sphinxparamcomma \sphinxparam{\DUrole{n,n}{y}\DUrole{p,p}{:}\DUrole{w,w}{  }\DUrole{n,n}{float}}}{{ $\rightarrow$ {\hyperref[\detokenize{deepdrr.geo:deepdrr.geo.core.Vector2D}]{\sphinxcrossref{Vector2D}}}}}
\pysiglinewithargsret{\sphinxcode{\sphinxupquote{deepdrr.geo.core.}}\sphinxbfcode{\sphinxupquote{vector}}}{\sphinxparam{\DUrole{n,n}{x}\DUrole{p,p}{:}\DUrole{w,w}{  }\DUrole{n,n}{float}}\sphinxparamcomma \sphinxparam{\DUrole{n,n}{y}\DUrole{p,p}{:}\DUrole{w,w}{  }\DUrole{n,n}{float}}\sphinxparamcomma \sphinxparam{\DUrole{n,n}{z}\DUrole{p,p}{:}\DUrole{w,w}{  }\DUrole{n,n}{float}}}{{ $\rightarrow$ {\hyperref[\detokenize{deepdrr.geo:deepdrr.geo.core.Vector3D}]{\sphinxcrossref{Vector3D}}}}}
\pysiglinewithargsret{\sphinxcode{\sphinxupquote{deepdrr.geo.core.}}\sphinxbfcode{\sphinxupquote{vector}}}{\sphinxparam{\DUrole{n,n}{x}\DUrole{p,p}{:}\DUrole{w,w}{  }\DUrole{n,n}{ndarray}}}{{ $\rightarrow$ {\hyperref[\detokenize{deepdrr.geo:deepdrr.geo.core.Vector}]{\sphinxcrossref{Vector}}}}}
\pysiglinewithargsret{\sphinxcode{\sphinxupquote{deepdrr.geo.core.}}\sphinxbfcode{\sphinxupquote{vector}}}{\sphinxparam{\DUrole{n,n}{v}\DUrole{p,p}{:}\DUrole{w,w}{  }\DUrole{n,n}{{\hyperref[\detokenize{deepdrr.geo:deepdrr.geo.core.HasDirection}]{\sphinxcrossref{HasDirection}}}}}}{{ $\rightarrow$ {\hyperref[\detokenize{deepdrr.geo:deepdrr.geo.core.Vector}]{\sphinxcrossref{Vector}}}}}
\pysiglinewithargsret{\sphinxcode{\sphinxupquote{deepdrr.geo.core.}}\sphinxbfcode{\sphinxupquote{vector}}}{\sphinxparam{\DUrole{o,o}{*}\DUrole{n,n}{args}\DUrole{p,p}{:}\DUrole{w,w}{  }\DUrole{n,n}{Any}}}{{ $\rightarrow$ {\hyperref[\detokenize{deepdrr.geo:deepdrr.geo.core.Vector}]{\sphinxcrossref{Vector}}}}}
\pysigstopsignatures
\sphinxAtStartPar
The preferred method for creating a vector.

\sphinxAtStartPar
There are three ways to create a point using \sphinxtitleref{vector()}.
\begin{itemize}
\item {} 
\sphinxAtStartPar
Pass the coordinates as separate arguments. For instance, \sphinxtitleref{vector(0, 0)} returns the 2D homogeneous vector \sphinxtitleref{Vector2D({[}0, 0, 0{]})}.

\item {} 
\sphinxAtStartPar
Pass a numpy array containing the non\sphinxhyphen{}homogeneous representation of the vector.
For example \sphinxtitleref{vector(np.ndarray({[}0, 1, 2{]}))} is the 3D homogeneous veector \sphinxtitleref{Vector3D({[}0, 1, 2, 0{]})}.

\item {} 
\sphinxAtStartPar
Pass a Vector2D or Vector3D instance, in which case \sphinxtitleref{vector()} just returns the first argument.

\end{itemize}

\sphinxAtStartPar
\sphinxtitleref{point()} should NOT be given a numpy array containing the homogeneous data. In this case, use the \sphinxtitleref{Vector2D} and \sphinxtitleref{Vector3D} constructors directly.
\begin{quote}\begin{description}
\sphinxlineitem{Raises}
\sphinxAtStartPar
\sphinxstyleliteralstrong{\sphinxupquote{ValueError}} \textendash{} if arguments cannot be parsed as data for a point.

\sphinxlineitem{Returns}
\sphinxAtStartPar
Point2D or Point3D.

\sphinxlineitem{Return type}
\sphinxAtStartPar
Union{[}{\hyperref[\detokenize{deepdrr.geo:deepdrr.geo.core.Point2D}]{\sphinxcrossref{Point2D}}}, {\hyperref[\detokenize{deepdrr.geo:deepdrr.geo.core.Point3D}]{\sphinxcrossref{Point3D}}}{]}

\end{description}\end{quote}

\end{fulllineitems}



\subsection{deepdrr.geo.exceptions}
\label{\detokenize{deepdrr.geo:module-deepdrr.geo.exceptions}}\label{\detokenize{deepdrr.geo:deepdrr-geo-exceptions}}\index{module@\spxentry{module}!deepdrr.geo.exceptions@\spxentry{deepdrr.geo.exceptions}}\index{deepdrr.geo.exceptions@\spxentry{deepdrr.geo.exceptions}!module@\spxentry{module}}\index{JoinError@\spxentry{JoinError}}

\begin{fulllineitems}
\phantomsection\label{\detokenize{deepdrr.geo:deepdrr.geo.exceptions.JoinError}}
\pysigstartsignatures
\pysigline{\sphinxbfcode{\sphinxupquote{exception\DUrole{w,w}{  }}}\sphinxcode{\sphinxupquote{deepdrr.geo.exceptions.}}\sphinxbfcode{\sphinxupquote{JoinError}}}
\pysigstopsignatures
\sphinxAtStartPar
Bases: \sphinxcode{\sphinxupquote{Exception}}

\sphinxAtStartPar
Represents an error when joining two primitives.

\end{fulllineitems}

\index{MeetError@\spxentry{MeetError}}

\begin{fulllineitems}
\phantomsection\label{\detokenize{deepdrr.geo:deepdrr.geo.exceptions.MeetError}}
\pysigstartsignatures
\pysigline{\sphinxbfcode{\sphinxupquote{exception\DUrole{w,w}{  }}}\sphinxcode{\sphinxupquote{deepdrr.geo.exceptions.}}\sphinxbfcode{\sphinxupquote{MeetError}}}
\pysigstopsignatures
\sphinxAtStartPar
Bases: \sphinxcode{\sphinxupquote{Exception}}

\sphinxAtStartPar
Represents an error when finding the intersection of two primitives.

\end{fulllineitems}



\subsection{deepdrr.geo.functional}
\label{\detokenize{deepdrr.geo:deepdrr-geo-functional}}

\subsection{deepdrr.geo.hyperplane}
\label{\detokenize{deepdrr.geo:module-deepdrr.geo.hyperplane}}\label{\detokenize{deepdrr.geo:deepdrr-geo-hyperplane}}\index{module@\spxentry{module}!deepdrr.geo.hyperplane@\spxentry{deepdrr.geo.hyperplane}}\index{deepdrr.geo.hyperplane@\spxentry{deepdrr.geo.hyperplane}!module@\spxentry{module}}\index{HyperPlane (class in deepdrr.geo.hyperplane)@\spxentry{HyperPlane}\spxextra{class in deepdrr.geo.hyperplane}}

\begin{fulllineitems}
\phantomsection\label{\detokenize{deepdrr.geo:deepdrr.geo.hyperplane.HyperPlane}}
\pysigstartsignatures
\pysiglinewithargsret{\sphinxbfcode{\sphinxupquote{class\DUrole{w,w}{  }}}\sphinxcode{\sphinxupquote{deepdrr.geo.hyperplane.}}\sphinxbfcode{\sphinxupquote{HyperPlane}}}{\sphinxparam{\DUrole{n,n}{data}\DUrole{p,p}{:}\DUrole{w,w}{  }\DUrole{n,n}{ndarray}}}{}
\pysigstopsignatures
\sphinxAtStartPar
Bases: {\hyperref[\detokenize{deepdrr.geo:deepdrr.geo.core.HasLocationAndDirection}]{\sphinxcrossref{\sphinxcode{\sphinxupquote{HasLocationAndDirection}}}}}, {\hyperref[\detokenize{deepdrr.geo:deepdrr.geo.core.Meetable}]{\sphinxcrossref{\sphinxcode{\sphinxupquote{Meetable}}}}}

\sphinxAtStartPar
Represents a hyperplane in 2D (a line) or 3D (a plane).

\sphinxAtStartPar
Hyperplanes can be intersected with other hyperplanes or lower dimensional objects, but they are
not joinable.
\index{a (deepdrr.geo.hyperplane.HyperPlane property)@\spxentry{a}\spxextra{deepdrr.geo.hyperplane.HyperPlane property}}

\begin{fulllineitems}
\phantomsection\label{\detokenize{deepdrr.geo:deepdrr.geo.hyperplane.HyperPlane.a}}
\pysigstartsignatures
\pysigline{\sphinxbfcode{\sphinxupquote{property\DUrole{w,w}{  }}}\sphinxbfcode{\sphinxupquote{a}}\sphinxbfcode{\sphinxupquote{\DUrole{p,p}{:}\DUrole{w,w}{  }float}}}
\pysigstopsignatures
\sphinxAtStartPar
Get the coefficient of the first variable.
\begin{quote}\begin{description}
\sphinxlineitem{Returns}
\sphinxAtStartPar
The coefficient of the first variable.

\sphinxlineitem{Return type}
\sphinxAtStartPar
float

\end{description}\end{quote}

\end{fulllineitems}

\index{b (deepdrr.geo.hyperplane.HyperPlane property)@\spxentry{b}\spxextra{deepdrr.geo.hyperplane.HyperPlane property}}

\begin{fulllineitems}
\phantomsection\label{\detokenize{deepdrr.geo:deepdrr.geo.hyperplane.HyperPlane.b}}
\pysigstartsignatures
\pysigline{\sphinxbfcode{\sphinxupquote{property\DUrole{w,w}{  }}}\sphinxbfcode{\sphinxupquote{b}}\sphinxbfcode{\sphinxupquote{\DUrole{p,p}{:}\DUrole{w,w}{  }float}}}
\pysigstopsignatures
\sphinxAtStartPar
Get the coefficient of the second variable.
\begin{quote}\begin{description}
\sphinxlineitem{Returns}
\sphinxAtStartPar
The coefficient of the second variable.

\sphinxlineitem{Return type}
\sphinxAtStartPar
float

\end{description}\end{quote}

\end{fulllineitems}

\index{c (deepdrr.geo.hyperplane.HyperPlane property)@\spxentry{c}\spxextra{deepdrr.geo.hyperplane.HyperPlane property}}

\begin{fulllineitems}
\phantomsection\label{\detokenize{deepdrr.geo:deepdrr.geo.hyperplane.HyperPlane.c}}
\pysigstartsignatures
\pysigline{\sphinxbfcode{\sphinxupquote{property\DUrole{w,w}{  }}}\sphinxbfcode{\sphinxupquote{c}}\sphinxbfcode{\sphinxupquote{\DUrole{p,p}{:}\DUrole{w,w}{  }float}}}
\pysigstopsignatures
\sphinxAtStartPar
Get the coefficient of the third variable.
\begin{quote}\begin{description}
\sphinxlineitem{Returns}
\sphinxAtStartPar
The coefficient of the third variable.

\sphinxlineitem{Return type}
\sphinxAtStartPar
float

\end{description}\end{quote}

\end{fulllineitems}

\index{d (deepdrr.geo.hyperplane.HyperPlane property)@\spxentry{d}\spxextra{deepdrr.geo.hyperplane.HyperPlane property}}

\begin{fulllineitems}
\phantomsection\label{\detokenize{deepdrr.geo:deepdrr.geo.hyperplane.HyperPlane.d}}
\pysigstartsignatures
\pysigline{\sphinxbfcode{\sphinxupquote{property\DUrole{w,w}{  }}}\sphinxbfcode{\sphinxupquote{d}}\sphinxbfcode{\sphinxupquote{\DUrole{p,p}{:}\DUrole{w,w}{  }float}}}
\pysigstopsignatures
\sphinxAtStartPar
Get the constant term.
\begin{quote}\begin{description}
\sphinxlineitem{Returns}
\sphinxAtStartPar
The constant term.

\sphinxlineitem{Return type}
\sphinxAtStartPar
float

\end{description}\end{quote}

\end{fulllineitems}

\index{data (deepdrr.geo.hyperplane.HyperPlane attribute)@\spxentry{data}\spxextra{deepdrr.geo.hyperplane.HyperPlane attribute}}

\begin{fulllineitems}
\phantomsection\label{\detokenize{deepdrr.geo:deepdrr.geo.hyperplane.HyperPlane.data}}
\pysigstartsignatures
\pysigline{\sphinxbfcode{\sphinxupquote{data}}\sphinxbfcode{\sphinxupquote{\DUrole{p,p}{:}\DUrole{w,w}{  }ndarray}}}
\pysigstopsignatures
\end{fulllineitems}

\index{distance() (deepdrr.geo.hyperplane.HyperPlane method)@\spxentry{distance()}\spxextra{deepdrr.geo.hyperplane.HyperPlane method}}

\begin{fulllineitems}
\phantomsection\label{\detokenize{deepdrr.geo:deepdrr.geo.hyperplane.HyperPlane.distance}}
\pysigstartsignatures
\pysiglinewithargsret{\sphinxbfcode{\sphinxupquote{distance}}}{\sphinxparam{\DUrole{n,n}{p}\DUrole{p,p}{:}\DUrole{w,w}{  }\DUrole{n,n}{{\hyperref[\detokenize{deepdrr.geo:deepdrr.geo.core.Point}]{\sphinxcrossref{Point}}}}}}{{ $\rightarrow$ float}}
\pysigstopsignatures
\sphinxAtStartPar
Get the distance of the point to the hyperplane.
\begin{quote}\begin{description}
\sphinxlineitem{Parameters}
\sphinxAtStartPar
\sphinxstyleliteralstrong{\sphinxupquote{p}} ({\hyperref[\detokenize{deepdrr.geo:deepdrr.geo.core.Point}]{\sphinxcrossref{\sphinxstyleliteralemphasis{\sphinxupquote{Point}}}}}) \textendash{} the point to evaluate at.

\sphinxlineitem{Returns}
\sphinxAtStartPar
the distance of the point to the hyperplane.

\sphinxlineitem{Return type}
\sphinxAtStartPar
float

\end{description}\end{quote}

\end{fulllineitems}

\index{evaluate() (deepdrr.geo.hyperplane.HyperPlane method)@\spxentry{evaluate()}\spxextra{deepdrr.geo.hyperplane.HyperPlane method}}

\begin{fulllineitems}
\phantomsection\label{\detokenize{deepdrr.geo:deepdrr.geo.hyperplane.HyperPlane.evaluate}}
\pysigstartsignatures
\pysiglinewithargsret{\sphinxbfcode{\sphinxupquote{evaluate}}}{\sphinxparam{\DUrole{n,n}{p}\DUrole{p,p}{:}\DUrole{w,w}{  }\DUrole{n,n}{{\hyperref[\detokenize{deepdrr.geo:deepdrr.geo.core.Point}]{\sphinxcrossref{Point}}}}}}{{ $\rightarrow$ float}}
\pysigstopsignatures
\sphinxAtStartPar
Evaluate the hyperplane at the given point.

\sphinxAtStartPar
The sign of this value tells you which side of the hyperplane the point is on.
\begin{quote}\begin{description}
\sphinxlineitem{Parameters}
\sphinxAtStartPar
\sphinxstyleliteralstrong{\sphinxupquote{p}} ({\hyperref[\detokenize{deepdrr.geo:deepdrr.geo.core.Point}]{\sphinxcrossref{\sphinxstyleliteralemphasis{\sphinxupquote{Point}}}}}) \textendash{} the point to evaluate at.

\sphinxlineitem{Returns}
\sphinxAtStartPar
the value of the hyperplane at the given point.

\sphinxlineitem{Return type}
\sphinxAtStartPar
float

\end{description}\end{quote}

\end{fulllineitems}

\index{get\_normal() (deepdrr.geo.hyperplane.HyperPlane method)@\spxentry{get\_normal()}\spxextra{deepdrr.geo.hyperplane.HyperPlane method}}

\begin{fulllineitems}
\phantomsection\label{\detokenize{deepdrr.geo:deepdrr.geo.hyperplane.HyperPlane.get_normal}}
\pysigstartsignatures
\pysiglinewithargsret{\sphinxbfcode{\sphinxupquote{get\_normal}}}{}{{ $\rightarrow$ {\hyperref[\detokenize{deepdrr.geo:deepdrr.geo.core.Vector3D}]{\sphinxcrossref{Vector3D}}}}}
\pysigstopsignatures
\sphinxAtStartPar
Get the normal vector of the plane.
\begin{quote}\begin{description}
\sphinxlineitem{Returns}
\sphinxAtStartPar
The normal vector of the plane.

\sphinxlineitem{Return type}
\sphinxAtStartPar
{\hyperref[\detokenize{deepdrr.geo:deepdrr.geo.core.Vector3D}]{\sphinxcrossref{Vector3D}}}

\end{description}\end{quote}

\end{fulllineitems}

\index{n (deepdrr.geo.hyperplane.HyperPlane property)@\spxentry{n}\spxextra{deepdrr.geo.hyperplane.HyperPlane property}}

\begin{fulllineitems}
\phantomsection\label{\detokenize{deepdrr.geo:deepdrr.geo.hyperplane.HyperPlane.n}}
\pysigstartsignatures
\pysigline{\sphinxbfcode{\sphinxupquote{property\DUrole{w,w}{  }}}\sphinxbfcode{\sphinxupquote{n}}\sphinxbfcode{\sphinxupquote{\DUrole{p,p}{:}\DUrole{w,w}{  }{\hyperref[\detokenize{deepdrr.geo:deepdrr.geo.core.Vector3D}]{\sphinxcrossref{Vector3D}}}}}}
\pysigstopsignatures
\end{fulllineitems}

\index{normal() (deepdrr.geo.hyperplane.HyperPlane method)@\spxentry{normal()}\spxextra{deepdrr.geo.hyperplane.HyperPlane method}}

\begin{fulllineitems}
\phantomsection\label{\detokenize{deepdrr.geo:deepdrr.geo.hyperplane.HyperPlane.normal}}
\pysigstartsignatures
\pysiglinewithargsret{\sphinxbfcode{\sphinxupquote{normal}}}{}{{ $\rightarrow$ {\hyperref[\detokenize{deepdrr.geo:deepdrr.geo.core.Vector3D}]{\sphinxcrossref{Vector3D}}}}}
\pysigstopsignatures
\end{fulllineitems}

\index{project() (deepdrr.geo.hyperplane.HyperPlane method)@\spxentry{project()}\spxextra{deepdrr.geo.hyperplane.HyperPlane method}}

\begin{fulllineitems}
\phantomsection\label{\detokenize{deepdrr.geo:deepdrr.geo.hyperplane.HyperPlane.project}}
\pysigstartsignatures
\pysiglinewithargsret{\sphinxbfcode{\sphinxupquote{project}}}{\sphinxparam{\DUrole{n,n}{p}\DUrole{p,p}{:}\DUrole{w,w}{  }\DUrole{n,n}{P}}}{{ $\rightarrow$ P}}
\pysigstopsignatures
\sphinxAtStartPar
Get the closest point on the hyperplane to p.
\begin{quote}\begin{description}
\sphinxlineitem{Parameters}
\sphinxAtStartPar
\sphinxstyleliteralstrong{\sphinxupquote{p}} ({\hyperref[\detokenize{deepdrr.geo:deepdrr.geo.core.Point}]{\sphinxcrossref{\sphinxstyleliteralemphasis{\sphinxupquote{Point}}}}}) \textendash{} The point to project.

\sphinxlineitem{Returns}
\sphinxAtStartPar
The closest point on the hyperplane to p.

\sphinxlineitem{Return type}
\sphinxAtStartPar
{\hyperref[\detokenize{deepdrr.geo:deepdrr.geo.core.Point}]{\sphinxcrossref{Point}}}

\end{description}\end{quote}

\end{fulllineitems}

\index{signed\_distance() (deepdrr.geo.hyperplane.HyperPlane method)@\spxentry{signed\_distance()}\spxextra{deepdrr.geo.hyperplane.HyperPlane method}}

\begin{fulllineitems}
\phantomsection\label{\detokenize{deepdrr.geo:deepdrr.geo.hyperplane.HyperPlane.signed_distance}}
\pysigstartsignatures
\pysiglinewithargsret{\sphinxbfcode{\sphinxupquote{signed\_distance}}}{\sphinxparam{\DUrole{n,n}{p}\DUrole{p,p}{:}\DUrole{w,w}{  }\DUrole{n,n}{{\hyperref[\detokenize{deepdrr.geo:deepdrr.geo.core.Point}]{\sphinxcrossref{Point}}}}}}{{ $\rightarrow$ float}}
\pysigstopsignatures
\sphinxAtStartPar
Get the signed distance from the given point to the hyperplane.
\begin{quote}\begin{description}
\sphinxlineitem{Parameters}
\sphinxAtStartPar
\sphinxstyleliteralstrong{\sphinxupquote{p}} ({\hyperref[\detokenize{deepdrr.geo:deepdrr.geo.core.Point}]{\sphinxcrossref{\sphinxstyleliteralemphasis{\sphinxupquote{Point}}}}}) \textendash{} the point to measure the distance from.

\sphinxlineitem{Returns}
\sphinxAtStartPar
the signed distance from the point to the hyperplane.

\sphinxlineitem{Return type}
\sphinxAtStartPar
float

\end{description}\end{quote}

\end{fulllineitems}


\end{fulllineitems}

\index{Line (class in deepdrr.geo.hyperplane)@\spxentry{Line}\spxextra{class in deepdrr.geo.hyperplane}}

\begin{fulllineitems}
\phantomsection\label{\detokenize{deepdrr.geo:deepdrr.geo.hyperplane.Line}}
\pysigstartsignatures
\pysiglinewithargsret{\sphinxbfcode{\sphinxupquote{class\DUrole{w,w}{  }}}\sphinxcode{\sphinxupquote{deepdrr.geo.hyperplane.}}\sphinxbfcode{\sphinxupquote{Line}}}{\sphinxparam{\DUrole{n,n}{data}\DUrole{p,p}{:}\DUrole{w,w}{  }\DUrole{n,n}{ndarray}}}{}
\pysigstopsignatures
\sphinxAtStartPar
Bases: {\hyperref[\detokenize{deepdrr.geo:deepdrr.geo.core.HasLocationAndDirection}]{\sphinxcrossref{\sphinxcode{\sphinxupquote{HasLocationAndDirection}}}}}, {\hyperref[\detokenize{deepdrr.geo:deepdrr.geo.core.Meetable}]{\sphinxcrossref{\sphinxcode{\sphinxupquote{Meetable}}}}}

\sphinxAtStartPar
Abstract parent class for lines and line\sphinxhyphen{}like objects.
\index{angle() (deepdrr.geo.hyperplane.Line method)@\spxentry{angle()}\spxextra{deepdrr.geo.hyperplane.Line method}}

\begin{fulllineitems}
\phantomsection\label{\detokenize{deepdrr.geo:deepdrr.geo.hyperplane.Line.angle}}
\pysigstartsignatures
\pysiglinewithargsret{\sphinxbfcode{\sphinxupquote{angle}}}{\sphinxparam{\DUrole{n,n}{other}\DUrole{p,p}{:}\DUrole{w,w}{  }\DUrole{n,n}{{\hyperref[\detokenize{deepdrr.geo:deepdrr.geo.hyperplane.Line}]{\sphinxcrossref{Line}}}\DUrole{w,w}{  }\DUrole{p,p}{|}\DUrole{w,w}{  }{\hyperref[\detokenize{deepdrr.geo:deepdrr.geo.core.Vector}]{\sphinxcrossref{Vector}}}}}}{{ $\rightarrow$ float}}
\pysigstopsignatures
\sphinxAtStartPar
Get the acute angle between the two lines.

\end{fulllineitems}

\index{as\_points() (deepdrr.geo.hyperplane.Line method)@\spxentry{as\_points()}\spxextra{deepdrr.geo.hyperplane.Line method}}

\begin{fulllineitems}
\phantomsection\label{\detokenize{deepdrr.geo:deepdrr.geo.hyperplane.Line.as_points}}
\pysigstartsignatures
\pysiglinewithargsret{\sphinxbfcode{\sphinxupquote{as\_points}}}{}{{ $\rightarrow$ Tuple\DUrole{p,p}{{[}}{\hyperref[\detokenize{deepdrr.geo:deepdrr.geo.core.Point2D}]{\sphinxcrossref{Point2D}}}\DUrole{p,p}{,}\DUrole{w,w}{  }{\hyperref[\detokenize{deepdrr.geo:deepdrr.geo.core.Point2D}]{\sphinxcrossref{Point2D}}}\DUrole{p,p}{{]}}}}
\pysiglinewithargsret{\sphinxbfcode{\sphinxupquote{as\_points}}}{}{{ $\rightarrow$ Tuple\DUrole{p,p}{{[}}{\hyperref[\detokenize{deepdrr.geo:deepdrr.geo.core.Point3D}]{\sphinxcrossref{Point3D}}}\DUrole{p,p}{,}\DUrole{w,w}{  }{\hyperref[\detokenize{deepdrr.geo:deepdrr.geo.core.Point3D}]{\sphinxcrossref{Point3D}}}\DUrole{p,p}{{]}}}}
\pysigstopsignatures
\sphinxAtStartPar
Get two points on the line.
\begin{quote}\begin{description}
\sphinxlineitem{Returns}
\sphinxAtStartPar
Two points on the line.

\sphinxlineitem{Return type}
\sphinxAtStartPar
Tuple{[}{\hyperref[\detokenize{deepdrr.geo:deepdrr.geo.core.Point}]{\sphinxcrossref{Point}}}, {\hyperref[\detokenize{deepdrr.geo:deepdrr.geo.core.Point}]{\sphinxcrossref{Point}}}{]}

\end{description}\end{quote}

\end{fulllineitems}

\index{data (deepdrr.geo.hyperplane.Line attribute)@\spxentry{data}\spxextra{deepdrr.geo.hyperplane.Line attribute}}

\begin{fulllineitems}
\phantomsection\label{\detokenize{deepdrr.geo:deepdrr.geo.hyperplane.Line.data}}
\pysigstartsignatures
\pysigline{\sphinxbfcode{\sphinxupquote{data}}\sphinxbfcode{\sphinxupquote{\DUrole{p,p}{:}\DUrole{w,w}{  }ndarray}}}
\pysigstopsignatures
\end{fulllineitems}

\index{distance() (deepdrr.geo.hyperplane.Line method)@\spxentry{distance()}\spxextra{deepdrr.geo.hyperplane.Line method}}

\begin{fulllineitems}
\phantomsection\label{\detokenize{deepdrr.geo:deepdrr.geo.hyperplane.Line.distance}}
\pysigstartsignatures
\pysiglinewithargsret{\sphinxbfcode{\sphinxupquote{distance}}}{\sphinxparam{\DUrole{n,n}{other}\DUrole{p,p}{:}\DUrole{w,w}{  }\DUrole{n,n}{{\hyperref[\detokenize{deepdrr.geo:deepdrr.geo.core.Point}]{\sphinxcrossref{Point}}}}}}{{ $\rightarrow$ float}}
\pysigstopsignatures
\sphinxAtStartPar
Get the distance from the line to another point.
\begin{quote}\begin{description}
\sphinxlineitem{Parameters}
\sphinxAtStartPar
\sphinxstyleliteralstrong{\sphinxupquote{other}} ({\hyperref[\detokenize{deepdrr.geo:deepdrr.geo.core.Point}]{\sphinxcrossref{\sphinxstyleliteralemphasis{\sphinxupquote{Point}}}}}) \textendash{} The point to which the distance is sought.

\sphinxlineitem{Returns}
\sphinxAtStartPar
The distance from the line to the other point.

\sphinxlineitem{Return type}
\sphinxAtStartPar
float

\end{description}\end{quote}

\end{fulllineitems}

\index{from\_point\_direction() (deepdrr.geo.hyperplane.Line class method)@\spxentry{from\_point\_direction()}\spxextra{deepdrr.geo.hyperplane.Line class method}}

\begin{fulllineitems}
\phantomsection\label{\detokenize{deepdrr.geo:deepdrr.geo.hyperplane.Line.from_point_direction}}
\pysigstartsignatures
\pysiglinewithargsret{\sphinxbfcode{\sphinxupquote{classmethod\DUrole{w,w}{  }}}\sphinxbfcode{\sphinxupquote{from\_point\_direction}}}{\sphinxparam{\DUrole{n,n}{p}\DUrole{p,p}{:}\DUrole{w,w}{  }\DUrole{n,n}{{\hyperref[\detokenize{deepdrr.geo:deepdrr.geo.core.Point}]{\sphinxcrossref{Point}}}}}\sphinxparamcomma \sphinxparam{\DUrole{n,n}{v}\DUrole{p,p}{:}\DUrole{w,w}{  }\DUrole{n,n}{{\hyperref[\detokenize{deepdrr.geo:deepdrr.geo.core.Vector}]{\sphinxcrossref{Vector}}}}}}{{ $\rightarrow$ {\hyperref[\detokenize{deepdrr.geo:deepdrr.geo.hyperplane.Line}]{\sphinxcrossref{Line}}}}}
\pysigstopsignatures
\sphinxAtStartPar
Construct a line from a point and a direction vector.
\begin{quote}\begin{description}
\sphinxlineitem{Parameters}\begin{itemize}
\item {} 
\sphinxAtStartPar
\sphinxstyleliteralstrong{\sphinxupquote{p}} ({\hyperref[\detokenize{deepdrr.geo:deepdrr.geo.core.Point}]{\sphinxcrossref{\sphinxstyleliteralemphasis{\sphinxupquote{Point}}}}}) \textendash{} The point on the line.

\item {} 
\sphinxAtStartPar
\sphinxstyleliteralstrong{\sphinxupquote{v}} ({\hyperref[\detokenize{deepdrr.geo:deepdrr.geo.core.Vector}]{\sphinxcrossref{\sphinxstyleliteralemphasis{\sphinxupquote{Vector}}}}}) \textendash{} The direction vector.

\end{itemize}

\sphinxlineitem{Returns}
\sphinxAtStartPar
The line through the point in the direction of the vector.

\sphinxlineitem{Return type}
\sphinxAtStartPar
{\hyperref[\detokenize{deepdrr.geo:deepdrr.geo.hyperplane.Line}]{\sphinxcrossref{Line}}}

\end{description}\end{quote}

\end{fulllineitems}

\index{project() (deepdrr.geo.hyperplane.Line method)@\spxentry{project()}\spxextra{deepdrr.geo.hyperplane.Line method}}

\begin{fulllineitems}
\phantomsection\label{\detokenize{deepdrr.geo:deepdrr.geo.hyperplane.Line.project}}
\pysigstartsignatures
\pysiglinewithargsret{\sphinxbfcode{\sphinxupquote{project}}}{\sphinxparam{\DUrole{n,n}{other}\DUrole{p,p}{:}\DUrole{w,w}{  }\DUrole{n,n}{{\hyperref[\detokenize{deepdrr.geo:deepdrr.geo.core.Point2D}]{\sphinxcrossref{Point2D}}}}}}{{ $\rightarrow$ {\hyperref[\detokenize{deepdrr.geo:deepdrr.geo.core.Point2D}]{\sphinxcrossref{Point2D}}}}}
\pysiglinewithargsret{\sphinxbfcode{\sphinxupquote{project}}}{\sphinxparam{\DUrole{n,n}{other}\DUrole{p,p}{:}\DUrole{w,w}{  }\DUrole{n,n}{{\hyperref[\detokenize{deepdrr.geo:deepdrr.geo.core.Point3D}]{\sphinxcrossref{Point3D}}}}}}{{ $\rightarrow$ {\hyperref[\detokenize{deepdrr.geo:deepdrr.geo.core.Point3D}]{\sphinxcrossref{Point3D}}}}}
\pysigstopsignatures
\sphinxAtStartPar
Get the closest point on the line to another point.
\begin{quote}\begin{description}
\sphinxlineitem{Parameters}
\sphinxAtStartPar
\sphinxstyleliteralstrong{\sphinxupquote{other}} ({\hyperref[\detokenize{deepdrr.geo:deepdrr.geo.core.Point}]{\sphinxcrossref{\sphinxstyleliteralemphasis{\sphinxupquote{Point}}}}}) \textendash{} The point to which the closest point is sought.

\sphinxlineitem{Returns}
\sphinxAtStartPar
The closest point on the line to the other point.

\sphinxlineitem{Return type}
\sphinxAtStartPar
{\hyperref[\detokenize{deepdrr.geo:deepdrr.geo.core.Point}]{\sphinxcrossref{Point}}}

\end{description}\end{quote}

\end{fulllineitems}


\end{fulllineitems}

\index{Line2D (class in deepdrr.geo.hyperplane)@\spxentry{Line2D}\spxextra{class in deepdrr.geo.hyperplane}}

\begin{fulllineitems}
\phantomsection\label{\detokenize{deepdrr.geo:deepdrr.geo.hyperplane.Line2D}}
\pysigstartsignatures
\pysiglinewithargsret{\sphinxbfcode{\sphinxupquote{class\DUrole{w,w}{  }}}\sphinxcode{\sphinxupquote{deepdrr.geo.hyperplane.}}\sphinxbfcode{\sphinxupquote{Line2D}}}{\sphinxparam{\DUrole{n,n}{data}\DUrole{p,p}{:}\DUrole{w,w}{  }\DUrole{n,n}{ndarray}}}{}
\pysigstopsignatures
\sphinxAtStartPar
Bases: {\hyperref[\detokenize{deepdrr.geo:deepdrr.geo.hyperplane.Line}]{\sphinxcrossref{\sphinxcode{\sphinxupquote{Line}}}}}, {\hyperref[\detokenize{deepdrr.geo:deepdrr.geo.hyperplane.HyperPlane}]{\sphinxcrossref{\sphinxcode{\sphinxupquote{HyperPlane}}}}}

\sphinxAtStartPar
Represents a line in 2D.

\sphinxAtStartPar
Consists of a 3\sphinxhyphen{}vector \(\mathbf{p} = [a, b, c]\) such that the line is all the points (x,y)
such that \(ax + by + c = 0\) or, alternatively, all the homogeneous points
\(\mathbf{x} = [x,y,w]\) such that \(p^T x = 0\).
\index{backproject() (deepdrr.geo.hyperplane.Line2D method)@\spxentry{backproject()}\spxextra{deepdrr.geo.hyperplane.Line2D method}}

\begin{fulllineitems}
\phantomsection\label{\detokenize{deepdrr.geo:deepdrr.geo.hyperplane.Line2D.backproject}}
\pysigstartsignatures
\pysiglinewithargsret{\sphinxbfcode{\sphinxupquote{backproject}}}{\sphinxparam{\DUrole{n,n}{index\_from\_world}\DUrole{p,p}{:}\DUrole{w,w}{  }\DUrole{n,n}{{\hyperref[\detokenize{deepdrr.geo:deepdrr.geo.core.CameraProjection}]{\sphinxcrossref{CameraProjection}}}}}}{{ $\rightarrow$ {\hyperref[\detokenize{deepdrr.geo:deepdrr.geo.hyperplane.Plane}]{\sphinxcrossref{Plane}}}}}
\pysigstopsignatures
\sphinxAtStartPar
Get the plane containing all the points that \sphinxtitleref{P} projects onto this line.
\begin{quote}\begin{description}
\sphinxlineitem{Parameters}
\sphinxAtStartPar
\sphinxstyleliteralstrong{\sphinxupquote{P}} ({\hyperref[\detokenize{deepdrr.geo:deepdrr.geo.core.Transform}]{\sphinxcrossref{\sphinxstyleliteralemphasis{\sphinxupquote{Transform}}}}}) \textendash{} A so\sphinxhyphen{}called \sphinxtitleref{index\_from\_world} projection transform.

\sphinxlineitem{Return type}
\sphinxAtStartPar
{\hyperref[\detokenize{deepdrr.geo:deepdrr.geo.hyperplane.Plane}]{\sphinxcrossref{Plane}}}

\end{description}\end{quote}

\end{fulllineitems}

\index{data (deepdrr.geo.hyperplane.Line2D attribute)@\spxentry{data}\spxextra{deepdrr.geo.hyperplane.Line2D attribute}}

\begin{fulllineitems}
\phantomsection\label{\detokenize{deepdrr.geo:deepdrr.geo.hyperplane.Line2D.data}}
\pysigstartsignatures
\pysigline{\sphinxbfcode{\sphinxupquote{data}}\sphinxbfcode{\sphinxupquote{\DUrole{p,p}{:}\DUrole{w,w}{  }ndarray}}}
\pysigstopsignatures
\end{fulllineitems}

\index{dim (deepdrr.geo.hyperplane.Line2D attribute)@\spxentry{dim}\spxextra{deepdrr.geo.hyperplane.Line2D attribute}}

\begin{fulllineitems}
\phantomsection\label{\detokenize{deepdrr.geo:deepdrr.geo.hyperplane.Line2D.dim}}
\pysigstartsignatures
\pysigline{\sphinxbfcode{\sphinxupquote{dim}}\sphinxbfcode{\sphinxupquote{\DUrole{w,w}{  }\DUrole{p,p}{=}\DUrole{w,w}{  }2}}}
\pysigstopsignatures
\end{fulllineitems}

\index{get\_direction() (deepdrr.geo.hyperplane.Line2D method)@\spxentry{get\_direction()}\spxextra{deepdrr.geo.hyperplane.Line2D method}}

\begin{fulllineitems}
\phantomsection\label{\detokenize{deepdrr.geo:deepdrr.geo.hyperplane.Line2D.get_direction}}
\pysigstartsignatures
\pysiglinewithargsret{\sphinxbfcode{\sphinxupquote{get\_direction}}}{}{{ $\rightarrow$ {\hyperref[\detokenize{deepdrr.geo:deepdrr.geo.core.Vector2D}]{\sphinxcrossref{Vector2D}}}}}
\pysigstopsignatures
\sphinxAtStartPar
Get the direction of the line.
\begin{quote}\begin{description}
\sphinxlineitem{Returns}
\sphinxAtStartPar
The unit\sphinxhyphen{}length direction of the line.

\sphinxlineitem{Return type}
\sphinxAtStartPar
{\hyperref[\detokenize{deepdrr.geo:deepdrr.geo.core.Vector2D}]{\sphinxcrossref{Vector2D}}}

\end{description}\end{quote}

\end{fulllineitems}

\index{get\_point() (deepdrr.geo.hyperplane.Line2D method)@\spxentry{get\_point()}\spxextra{deepdrr.geo.hyperplane.Line2D method}}

\begin{fulllineitems}
\phantomsection\label{\detokenize{deepdrr.geo:deepdrr.geo.hyperplane.Line2D.get_point}}
\pysigstartsignatures
\pysiglinewithargsret{\sphinxbfcode{\sphinxupquote{get\_point}}}{}{{ $\rightarrow$ {\hyperref[\detokenize{deepdrr.geo:deepdrr.geo.core.Point}]{\sphinxcrossref{Point}}}}}
\pysigstopsignatures
\sphinxAtStartPar
Get an arbitrary point on the line.
\begin{quote}\begin{description}
\sphinxlineitem{Returns}
\sphinxAtStartPar
A point on the line.

\sphinxlineitem{Return type}
\sphinxAtStartPar
{\hyperref[\detokenize{deepdrr.geo:deepdrr.geo.core.Point}]{\sphinxcrossref{Point}}}

\end{description}\end{quote}

\end{fulllineitems}

\index{meet() (deepdrr.geo.hyperplane.Line2D method)@\spxentry{meet()}\spxextra{deepdrr.geo.hyperplane.Line2D method}}

\begin{fulllineitems}
\phantomsection\label{\detokenize{deepdrr.geo:deepdrr.geo.hyperplane.Line2D.meet}}
\pysigstartsignatures
\pysiglinewithargsret{\sphinxbfcode{\sphinxupquote{meet}}}{\sphinxparam{\DUrole{n,n}{other}\DUrole{p,p}{:}\DUrole{w,w}{  }\DUrole{n,n}{{\hyperref[\detokenize{deepdrr.geo:deepdrr.geo.hyperplane.Line2D}]{\sphinxcrossref{Line2D}}}}}}{{ $\rightarrow$ {\hyperref[\detokenize{deepdrr.geo:deepdrr.geo.core.Point2D}]{\sphinxcrossref{Point2D}}}}}
\pysiglinewithargsret{\sphinxbfcode{\sphinxupquote{meet}}}{\sphinxparam{\DUrole{n,n}{other}\DUrole{p,p}{:}\DUrole{w,w}{  }\DUrole{n,n}{{\hyperref[\detokenize{deepdrr.geo:deepdrr.geo.segment.Segment2D}]{\sphinxcrossref{Segment2D}}}}}}{{ $\rightarrow$ {\hyperref[\detokenize{deepdrr.geo:deepdrr.geo.core.Point2D}]{\sphinxcrossref{Point2D}}}}}
\pysigstopsignatures
\sphinxAtStartPar
Get the intersection of two objects.

\sphinxAtStartPar
For example, given two lines, get the line that is the intersection of them.
\begin{quote}\begin{description}
\sphinxlineitem{Parameters}
\sphinxAtStartPar
\sphinxstyleliteralstrong{\sphinxupquote{other}} ({\hyperref[\detokenize{deepdrr.geo:deepdrr.geo.core.Primitive}]{\sphinxcrossref{\sphinxstyleliteralemphasis{\sphinxupquote{Primitive}}}}}) \textendash{} the other primitive.

\sphinxlineitem{Returns}
\sphinxAtStartPar
the intersection of \sphinxtitleref{self} and \sphinxtitleref{other}.

\sphinxlineitem{Return type}
\sphinxAtStartPar
{\hyperref[\detokenize{deepdrr.geo:deepdrr.geo.core.Primitive}]{\sphinxcrossref{Primitive}}}

\sphinxlineitem{Raises}
\sphinxAtStartPar
{\hyperref[\detokenize{deepdrr.geo:deepdrr.geo.exceptions.MeetError}]{\sphinxcrossref{\sphinxstyleliteralstrong{\sphinxupquote{MeetError}}}}} \textendash{} if the objects cannot be intersected.

\end{description}\end{quote}

\end{fulllineitems}


\end{fulllineitems}

\index{Line3D (class in deepdrr.geo.hyperplane)@\spxentry{Line3D}\spxextra{class in deepdrr.geo.hyperplane}}

\begin{fulllineitems}
\phantomsection\label{\detokenize{deepdrr.geo:deepdrr.geo.hyperplane.Line3D}}
\pysigstartsignatures
\pysiglinewithargsret{\sphinxbfcode{\sphinxupquote{class\DUrole{w,w}{  }}}\sphinxcode{\sphinxupquote{deepdrr.geo.hyperplane.}}\sphinxbfcode{\sphinxupquote{Line3D}}}{\sphinxparam{\DUrole{n,n}{data}\DUrole{p,p}{:}\DUrole{w,w}{  }\DUrole{n,n}{ndarray}}}{}
\pysigstopsignatures
\sphinxAtStartPar
Bases: {\hyperref[\detokenize{deepdrr.geo:deepdrr.geo.hyperplane.Line}]{\sphinxcrossref{\sphinxcode{\sphinxupquote{Line}}}}}, {\hyperref[\detokenize{deepdrr.geo:deepdrr.geo.core.Primitive}]{\sphinxcrossref{\sphinxcode{\sphinxupquote{Primitive}}}}}, {\hyperref[\detokenize{deepdrr.geo:deepdrr.geo.core.Joinable}]{\sphinxcrossref{\sphinxcode{\sphinxupquote{Joinable}}}}}, {\hyperref[\detokenize{deepdrr.geo:deepdrr.geo.core.Meetable}]{\sphinxcrossref{\sphinxcode{\sphinxupquote{Meetable}}}}}, {\hyperref[\detokenize{deepdrr.geo:deepdrr.geo.core.HasProjection}]{\sphinxcrossref{\sphinxcode{\sphinxupquote{HasProjection}}}}}

\sphinxAtStartPar
Represents a line in 3D as a 6\sphinxhyphen{}vector (p,q,r,s,t,u).

\sphinxAtStartPar
Based on \sphinxurl{https://dl.acm.org/doi/pdf/10.1145/965141.563900}.
\index{K (deepdrr.geo.hyperplane.Line3D property)@\spxentry{K}\spxextra{deepdrr.geo.hyperplane.Line3D property}}

\begin{fulllineitems}
\phantomsection\label{\detokenize{deepdrr.geo:deepdrr.geo.hyperplane.Line3D.K}}
\pysigstartsignatures
\pysigline{\sphinxbfcode{\sphinxupquote{property\DUrole{w,w}{  }}}\sphinxbfcode{\sphinxupquote{K}}\sphinxbfcode{\sphinxupquote{\DUrole{p,p}{:}\DUrole{w,w}{  }ndarray}}}
\pysigstopsignatures
\sphinxAtStartPar
Get the dual form of the line.

\end{fulllineitems}

\index{L (deepdrr.geo.hyperplane.Line3D property)@\spxentry{L}\spxextra{deepdrr.geo.hyperplane.Line3D property}}

\begin{fulllineitems}
\phantomsection\label{\detokenize{deepdrr.geo:deepdrr.geo.hyperplane.Line3D.L}}
\pysigstartsignatures
\pysigline{\sphinxbfcode{\sphinxupquote{property\DUrole{w,w}{  }}}\sphinxbfcode{\sphinxupquote{L}}\sphinxbfcode{\sphinxupquote{\DUrole{p,p}{:}\DUrole{w,w}{  }ndarray}}}
\pysigstopsignatures
\sphinxAtStartPar
Get the primal matrix of the line.

\end{fulllineitems}

\index{data (deepdrr.geo.hyperplane.Line3D attribute)@\spxentry{data}\spxextra{deepdrr.geo.hyperplane.Line3D attribute}}

\begin{fulllineitems}
\phantomsection\label{\detokenize{deepdrr.geo:deepdrr.geo.hyperplane.Line3D.data}}
\pysigstartsignatures
\pysigline{\sphinxbfcode{\sphinxupquote{data}}\sphinxbfcode{\sphinxupquote{\DUrole{p,p}{:}\DUrole{w,w}{  }ndarray}}}
\pysigstopsignatures
\end{fulllineitems}

\index{dim (deepdrr.geo.hyperplane.Line3D attribute)@\spxentry{dim}\spxextra{deepdrr.geo.hyperplane.Line3D attribute}}

\begin{fulllineitems}
\phantomsection\label{\detokenize{deepdrr.geo:deepdrr.geo.hyperplane.Line3D.dim}}
\pysigstartsignatures
\pysigline{\sphinxbfcode{\sphinxupquote{dim}}\sphinxbfcode{\sphinxupquote{\DUrole{w,w}{  }\DUrole{p,p}{=}\DUrole{w,w}{  }3}}}
\pysigstopsignatures
\end{fulllineitems}

\index{dual() (deepdrr.geo.hyperplane.Line3D method)@\spxentry{dual()}\spxextra{deepdrr.geo.hyperplane.Line3D method}}

\begin{fulllineitems}
\phantomsection\label{\detokenize{deepdrr.geo:deepdrr.geo.hyperplane.Line3D.dual}}
\pysigstartsignatures
\pysiglinewithargsret{\sphinxbfcode{\sphinxupquote{dual}}}{}{{ $\rightarrow$ ndarray}}
\pysigstopsignatures
\sphinxAtStartPar
Get the dual form of the line.

\end{fulllineitems}

\index{from\_dual() (deepdrr.geo.hyperplane.Line3D class method)@\spxentry{from\_dual()}\spxextra{deepdrr.geo.hyperplane.Line3D class method}}

\begin{fulllineitems}
\phantomsection\label{\detokenize{deepdrr.geo:deepdrr.geo.hyperplane.Line3D.from_dual}}
\pysigstartsignatures
\pysiglinewithargsret{\sphinxbfcode{\sphinxupquote{classmethod\DUrole{w,w}{  }}}\sphinxbfcode{\sphinxupquote{from\_dual}}}{\sphinxparam{\DUrole{n,n}{lk}\DUrole{p,p}{:}\DUrole{w,w}{  }\DUrole{n,n}{ndarray}}}{{ $\rightarrow$ {\hyperref[\detokenize{deepdrr.geo:deepdrr.geo.hyperplane.Line3D}]{\sphinxcrossref{Line3D}}}}}
\pysigstopsignatures
\end{fulllineitems}

\index{from\_primal() (deepdrr.geo.hyperplane.Line3D class method)@\spxentry{from\_primal()}\spxextra{deepdrr.geo.hyperplane.Line3D class method}}

\begin{fulllineitems}
\phantomsection\label{\detokenize{deepdrr.geo:deepdrr.geo.hyperplane.Line3D.from_primal}}
\pysigstartsignatures
\pysiglinewithargsret{\sphinxbfcode{\sphinxupquote{classmethod\DUrole{w,w}{  }}}\sphinxbfcode{\sphinxupquote{from\_primal}}}{\sphinxparam{\DUrole{n,n}{lp}\DUrole{p,p}{:}\DUrole{w,w}{  }\DUrole{n,n}{ndarray}}}{{ $\rightarrow$ {\hyperref[\detokenize{deepdrr.geo:deepdrr.geo.hyperplane.Line3D}]{\sphinxcrossref{Line3D}}}}}
\pysigstopsignatures
\end{fulllineitems}

\index{get\_direction() (deepdrr.geo.hyperplane.Line3D method)@\spxentry{get\_direction()}\spxextra{deepdrr.geo.hyperplane.Line3D method}}

\begin{fulllineitems}
\phantomsection\label{\detokenize{deepdrr.geo:deepdrr.geo.hyperplane.Line3D.get_direction}}
\pysigstartsignatures
\pysiglinewithargsret{\sphinxbfcode{\sphinxupquote{get\_direction}}}{}{{ $\rightarrow$ {\hyperref[\detokenize{deepdrr.geo:deepdrr.geo.core.Vector3D}]{\sphinxcrossref{Vector3D}}}}}
\pysigstopsignatures
\sphinxAtStartPar
Get the direction of the line.

\end{fulllineitems}

\index{get\_point() (deepdrr.geo.hyperplane.Line3D method)@\spxentry{get\_point()}\spxextra{deepdrr.geo.hyperplane.Line3D method}}

\begin{fulllineitems}
\phantomsection\label{\detokenize{deepdrr.geo:deepdrr.geo.hyperplane.Line3D.get_point}}
\pysigstartsignatures
\pysiglinewithargsret{\sphinxbfcode{\sphinxupquote{get\_point}}}{}{{ $\rightarrow$ {\hyperref[\detokenize{deepdrr.geo:deepdrr.geo.core.Point3D}]{\sphinxcrossref{Point3D}}}}}
\pysigstopsignatures
\sphinxAtStartPar
Get a point on the line.

\end{fulllineitems}

\index{join() (deepdrr.geo.hyperplane.Line3D method)@\spxentry{join()}\spxextra{deepdrr.geo.hyperplane.Line3D method}}

\begin{fulllineitems}
\phantomsection\label{\detokenize{deepdrr.geo:deepdrr.geo.hyperplane.Line3D.join}}
\pysigstartsignatures
\pysiglinewithargsret{\sphinxbfcode{\sphinxupquote{join}}}{\sphinxparam{\DUrole{n,n}{other}\DUrole{p,p}{:}\DUrole{w,w}{  }\DUrole{n,n}{{\hyperref[\detokenize{deepdrr.geo:deepdrr.geo.core.Point3D}]{\sphinxcrossref{Point3D}}}}}}{{ $\rightarrow$ {\hyperref[\detokenize{deepdrr.geo:deepdrr.geo.hyperplane.Plane}]{\sphinxcrossref{Plane}}}}}
\pysigstopsignatures
\sphinxAtStartPar
Join two objects.

\sphinxAtStartPar
For example, given two points, get the line that connects them.
\begin{quote}\begin{description}
\sphinxlineitem{Parameters}
\sphinxAtStartPar
\sphinxstyleliteralstrong{\sphinxupquote{other}} ({\hyperref[\detokenize{deepdrr.geo:deepdrr.geo.core.Primitive}]{\sphinxcrossref{\sphinxstyleliteralemphasis{\sphinxupquote{Primitive}}}}}) \textendash{} the other primitive.

\sphinxlineitem{Returns}
\sphinxAtStartPar
the joined primitive.

\sphinxlineitem{Return type}
\sphinxAtStartPar
{\hyperref[\detokenize{deepdrr.geo:deepdrr.geo.core.Primitive}]{\sphinxcrossref{Primitive}}}

\end{description}\end{quote}

\end{fulllineitems}

\index{meet() (deepdrr.geo.hyperplane.Line3D method)@\spxentry{meet()}\spxextra{deepdrr.geo.hyperplane.Line3D method}}

\begin{fulllineitems}
\phantomsection\label{\detokenize{deepdrr.geo:deepdrr.geo.hyperplane.Line3D.meet}}
\pysigstartsignatures
\pysiglinewithargsret{\sphinxbfcode{\sphinxupquote{meet}}}{\sphinxparam{\DUrole{n,n}{other}\DUrole{p,p}{:}\DUrole{w,w}{  }\DUrole{n,n}{{\hyperref[\detokenize{deepdrr.geo:deepdrr.geo.hyperplane.Plane}]{\sphinxcrossref{Plane}}}}}}{{ $\rightarrow$ {\hyperref[\detokenize{deepdrr.geo:deepdrr.geo.core.Point3D}]{\sphinxcrossref{Point3D}}}}}
\pysigstopsignatures
\sphinxAtStartPar
Get the intersection of two objects.

\sphinxAtStartPar
For example, given two lines, get the line that is the intersection of them.
\begin{quote}\begin{description}
\sphinxlineitem{Parameters}
\sphinxAtStartPar
\sphinxstyleliteralstrong{\sphinxupquote{other}} ({\hyperref[\detokenize{deepdrr.geo:deepdrr.geo.core.Primitive}]{\sphinxcrossref{\sphinxstyleliteralemphasis{\sphinxupquote{Primitive}}}}}) \textendash{} the other primitive.

\sphinxlineitem{Returns}
\sphinxAtStartPar
the intersection of \sphinxtitleref{self} and \sphinxtitleref{other}.

\sphinxlineitem{Return type}
\sphinxAtStartPar
{\hyperref[\detokenize{deepdrr.geo:deepdrr.geo.core.Primitive}]{\sphinxcrossref{Primitive}}}

\sphinxlineitem{Raises}
\sphinxAtStartPar
{\hyperref[\detokenize{deepdrr.geo:deepdrr.geo.exceptions.MeetError}]{\sphinxcrossref{\sphinxstyleliteralstrong{\sphinxupquote{MeetError}}}}} \textendash{} if the objects cannot be intersected.

\end{description}\end{quote}

\end{fulllineitems}

\index{p (deepdrr.geo.hyperplane.Line3D property)@\spxentry{p}\spxextra{deepdrr.geo.hyperplane.Line3D property}}

\begin{fulllineitems}
\phantomsection\label{\detokenize{deepdrr.geo:deepdrr.geo.hyperplane.Line3D.p}}
\pysigstartsignatures
\pysigline{\sphinxbfcode{\sphinxupquote{property\DUrole{w,w}{  }}}\sphinxbfcode{\sphinxupquote{p}}\sphinxbfcode{\sphinxupquote{\DUrole{p,p}{:}\DUrole{w,w}{  }float}}}
\pysigstopsignatures
\sphinxAtStartPar
Get the first parameter of the line.

\end{fulllineitems}

\index{primal() (deepdrr.geo.hyperplane.Line3D method)@\spxentry{primal()}\spxextra{deepdrr.geo.hyperplane.Line3D method}}

\begin{fulllineitems}
\phantomsection\label{\detokenize{deepdrr.geo:deepdrr.geo.hyperplane.Line3D.primal}}
\pysigstartsignatures
\pysiglinewithargsret{\sphinxbfcode{\sphinxupquote{primal}}}{}{{ $\rightarrow$ ndarray}}
\pysigstopsignatures
\sphinxAtStartPar
Get the primal matrix of the line.

\end{fulllineitems}

\index{projection\_type() (deepdrr.geo.hyperplane.Line3D class method)@\spxentry{projection\_type()}\spxextra{deepdrr.geo.hyperplane.Line3D class method}}

\begin{fulllineitems}
\phantomsection\label{\detokenize{deepdrr.geo:deepdrr.geo.hyperplane.Line3D.projection_type}}
\pysigstartsignatures
\pysiglinewithargsret{\sphinxbfcode{\sphinxupquote{classmethod\DUrole{w,w}{  }}}\sphinxbfcode{\sphinxupquote{projection\_type}}}{}{{ $\rightarrow$ Type\DUrole{p,p}{{[}}{\hyperref[\detokenize{deepdrr.geo:deepdrr.geo.hyperplane.Line2D}]{\sphinxcrossref{Line2D}}}\DUrole{p,p}{{]}}}}
\pysigstopsignatures
\sphinxAtStartPar
Get the type of the projection of the object.
\begin{quote}\begin{description}
\sphinxlineitem{Returns}
\sphinxAtStartPar
the type of the projection of the object.

\sphinxlineitem{Return type}
\sphinxAtStartPar
Type{[}{\hyperref[\detokenize{deepdrr.geo:deepdrr.geo.core.Primitive}]{\sphinxcrossref{Primitive}}}{]}

\end{description}\end{quote}

\end{fulllineitems}

\index{q (deepdrr.geo.hyperplane.Line3D property)@\spxentry{q}\spxextra{deepdrr.geo.hyperplane.Line3D property}}

\begin{fulllineitems}
\phantomsection\label{\detokenize{deepdrr.geo:deepdrr.geo.hyperplane.Line3D.q}}
\pysigstartsignatures
\pysigline{\sphinxbfcode{\sphinxupquote{property\DUrole{w,w}{  }}}\sphinxbfcode{\sphinxupquote{q}}\sphinxbfcode{\sphinxupquote{\DUrole{p,p}{:}\DUrole{w,w}{  }float}}}
\pysigstopsignatures
\sphinxAtStartPar
Get the second parameter of the line.

\end{fulllineitems}

\index{r (deepdrr.geo.hyperplane.Line3D property)@\spxentry{r}\spxextra{deepdrr.geo.hyperplane.Line3D property}}

\begin{fulllineitems}
\phantomsection\label{\detokenize{deepdrr.geo:deepdrr.geo.hyperplane.Line3D.r}}
\pysigstartsignatures
\pysigline{\sphinxbfcode{\sphinxupquote{property\DUrole{w,w}{  }}}\sphinxbfcode{\sphinxupquote{r}}\sphinxbfcode{\sphinxupquote{\DUrole{p,p}{:}\DUrole{w,w}{  }float}}}
\pysigstopsignatures
\sphinxAtStartPar
Get the third parameter of the line.

\end{fulllineitems}

\index{s (deepdrr.geo.hyperplane.Line3D property)@\spxentry{s}\spxextra{deepdrr.geo.hyperplane.Line3D property}}

\begin{fulllineitems}
\phantomsection\label{\detokenize{deepdrr.geo:deepdrr.geo.hyperplane.Line3D.s}}
\pysigstartsignatures
\pysigline{\sphinxbfcode{\sphinxupquote{property\DUrole{w,w}{  }}}\sphinxbfcode{\sphinxupquote{s}}\sphinxbfcode{\sphinxupquote{\DUrole{p,p}{:}\DUrole{w,w}{  }float}}}
\pysigstopsignatures
\sphinxAtStartPar
Get the fourth parameter of the line.

\end{fulllineitems}

\index{t (deepdrr.geo.hyperplane.Line3D property)@\spxentry{t}\spxextra{deepdrr.geo.hyperplane.Line3D property}}

\begin{fulllineitems}
\phantomsection\label{\detokenize{deepdrr.geo:deepdrr.geo.hyperplane.Line3D.t}}
\pysigstartsignatures
\pysigline{\sphinxbfcode{\sphinxupquote{property\DUrole{w,w}{  }}}\sphinxbfcode{\sphinxupquote{t}}\sphinxbfcode{\sphinxupquote{\DUrole{p,p}{:}\DUrole{w,w}{  }float}}}
\pysigstopsignatures
\sphinxAtStartPar
Get the fifth parameter of the line.

\end{fulllineitems}

\index{u (deepdrr.geo.hyperplane.Line3D property)@\spxentry{u}\spxextra{deepdrr.geo.hyperplane.Line3D property}}

\begin{fulllineitems}
\phantomsection\label{\detokenize{deepdrr.geo:deepdrr.geo.hyperplane.Line3D.u}}
\pysigstartsignatures
\pysigline{\sphinxbfcode{\sphinxupquote{property\DUrole{w,w}{  }}}\sphinxbfcode{\sphinxupquote{u}}\sphinxbfcode{\sphinxupquote{\DUrole{p,p}{:}\DUrole{w,w}{  }float}}}
\pysigstopsignatures
\sphinxAtStartPar
Get the sixth parameter of the line.

\end{fulllineitems}


\end{fulllineitems}

\index{Plane (class in deepdrr.geo.hyperplane)@\spxentry{Plane}\spxextra{class in deepdrr.geo.hyperplane}}

\begin{fulllineitems}
\phantomsection\label{\detokenize{deepdrr.geo:deepdrr.geo.hyperplane.Plane}}
\pysigstartsignatures
\pysiglinewithargsret{\sphinxbfcode{\sphinxupquote{class\DUrole{w,w}{  }}}\sphinxcode{\sphinxupquote{deepdrr.geo.hyperplane.}}\sphinxbfcode{\sphinxupquote{Plane}}}{\sphinxparam{\DUrole{n,n}{data}\DUrole{p,p}{:}\DUrole{w,w}{  }\DUrole{n,n}{ndarray}}}{}
\pysigstopsignatures
\sphinxAtStartPar
Bases: {\hyperref[\detokenize{deepdrr.geo:deepdrr.geo.hyperplane.HyperPlane}]{\sphinxcrossref{\sphinxcode{\sphinxupquote{HyperPlane}}}}}

\sphinxAtStartPar
Represents a plane in 3D
\index{data (deepdrr.geo.hyperplane.Plane attribute)@\spxentry{data}\spxextra{deepdrr.geo.hyperplane.Plane attribute}}

\begin{fulllineitems}
\phantomsection\label{\detokenize{deepdrr.geo:deepdrr.geo.hyperplane.Plane.data}}
\pysigstartsignatures
\pysigline{\sphinxbfcode{\sphinxupquote{data}}\sphinxbfcode{\sphinxupquote{\DUrole{p,p}{:}\DUrole{w,w}{  }ndarray}}}
\pysigstopsignatures
\end{fulllineitems}

\index{dim (deepdrr.geo.hyperplane.Plane attribute)@\spxentry{dim}\spxextra{deepdrr.geo.hyperplane.Plane attribute}}

\begin{fulllineitems}
\phantomsection\label{\detokenize{deepdrr.geo:deepdrr.geo.hyperplane.Plane.dim}}
\pysigstartsignatures
\pysigline{\sphinxbfcode{\sphinxupquote{dim}}\sphinxbfcode{\sphinxupquote{\DUrole{w,w}{  }\DUrole{p,p}{=}\DUrole{w,w}{  }3}}}
\pysigstopsignatures
\end{fulllineitems}

\index{from\_point\_direction() (deepdrr.geo.hyperplane.Plane class method)@\spxentry{from\_point\_direction()}\spxextra{deepdrr.geo.hyperplane.Plane class method}}

\begin{fulllineitems}
\phantomsection\label{\detokenize{deepdrr.geo:deepdrr.geo.hyperplane.Plane.from_point_direction}}
\pysigstartsignatures
\pysiglinewithargsret{\sphinxbfcode{\sphinxupquote{classmethod\DUrole{w,w}{  }}}\sphinxbfcode{\sphinxupquote{from\_point\_direction}}}{\sphinxparam{\DUrole{n,n}{r}\DUrole{p,p}{:}\DUrole{w,w}{  }\DUrole{n,n}{{\hyperref[\detokenize{deepdrr.geo:deepdrr.geo.core.Point3D}]{\sphinxcrossref{Point3D}}}}}\sphinxparamcomma \sphinxparam{\DUrole{n,n}{d}\DUrole{p,p}{:}\DUrole{w,w}{  }\DUrole{n,n}{{\hyperref[\detokenize{deepdrr.geo:deepdrr.geo.core.Vector3D}]{\sphinxcrossref{Vector3D}}}}}}{}
\pysigstopsignatures
\sphinxAtStartPar
Make a plane from a point and a direction vector.
\begin{quote}\begin{description}
\sphinxlineitem{Parameters}\begin{itemize}
\item {} 
\sphinxAtStartPar
\sphinxstyleliteralstrong{\sphinxupquote{r}} ({\hyperref[\detokenize{deepdrr.geo:deepdrr.geo.core.Point3D}]{\sphinxcrossref{\sphinxstyleliteralemphasis{\sphinxupquote{Point3D}}}}}) \textendash{} The point on the plane.

\item {} 
\sphinxAtStartPar
\sphinxstyleliteralstrong{\sphinxupquote{d}} ({\hyperref[\detokenize{deepdrr.geo:deepdrr.geo.core.Vector3D}]{\sphinxcrossref{\sphinxstyleliteralemphasis{\sphinxupquote{Vector3D}}}}}) \textendash{} The direction vector of the plane.

\end{itemize}

\sphinxlineitem{Returns}
\sphinxAtStartPar
The plane.

\sphinxlineitem{Return type}
\sphinxAtStartPar
{\hyperref[\detokenize{deepdrr.geo:deepdrr.geo.hyperplane.Plane}]{\sphinxcrossref{Plane}}}

\end{description}\end{quote}

\end{fulllineitems}

\index{from\_point\_normal() (deepdrr.geo.hyperplane.Plane class method)@\spxentry{from\_point\_normal()}\spxextra{deepdrr.geo.hyperplane.Plane class method}}

\begin{fulllineitems}
\phantomsection\label{\detokenize{deepdrr.geo:deepdrr.geo.hyperplane.Plane.from_point_normal}}
\pysigstartsignatures
\pysiglinewithargsret{\sphinxbfcode{\sphinxupquote{classmethod\DUrole{w,w}{  }}}\sphinxbfcode{\sphinxupquote{from\_point\_normal}}}{\sphinxparam{\DUrole{n,n}{r}\DUrole{p,p}{:}\DUrole{w,w}{  }\DUrole{n,n}{{\hyperref[\detokenize{deepdrr.geo:deepdrr.geo.core.Point3D}]{\sphinxcrossref{Point3D}}}}}\sphinxparamcomma \sphinxparam{\DUrole{n,n}{n}\DUrole{p,p}{:}\DUrole{w,w}{  }\DUrole{n,n}{{\hyperref[\detokenize{deepdrr.geo:deepdrr.geo.core.Vector3D}]{\sphinxcrossref{Vector3D}}}}}}{}
\pysigstopsignatures
\sphinxAtStartPar
Make a plane from a point and a normal vector.
\begin{quote}\begin{description}
\sphinxlineitem{Parameters}\begin{itemize}
\item {} 
\sphinxAtStartPar
\sphinxstyleliteralstrong{\sphinxupquote{r}} ({\hyperref[\detokenize{deepdrr.geo:deepdrr.geo.core.Point3D}]{\sphinxcrossref{\sphinxstyleliteralemphasis{\sphinxupquote{Point3D}}}}}) \textendash{} The point on the plane.

\item {} 
\sphinxAtStartPar
\sphinxstyleliteralstrong{\sphinxupquote{n}} ({\hyperref[\detokenize{deepdrr.geo:deepdrr.geo.core.Vector3D}]{\sphinxcrossref{\sphinxstyleliteralemphasis{\sphinxupquote{Vector3D}}}}}) \textendash{} The normal vector of the plane.

\end{itemize}

\sphinxlineitem{Returns}
\sphinxAtStartPar
The plane.

\sphinxlineitem{Return type}
\sphinxAtStartPar
{\hyperref[\detokenize{deepdrr.geo:deepdrr.geo.hyperplane.Plane}]{\sphinxcrossref{Plane}}}

\end{description}\end{quote}

\end{fulllineitems}

\index{from\_points() (deepdrr.geo.hyperplane.Plane class method)@\spxentry{from\_points()}\spxextra{deepdrr.geo.hyperplane.Plane class method}}

\begin{fulllineitems}
\phantomsection\label{\detokenize{deepdrr.geo:deepdrr.geo.hyperplane.Plane.from_points}}
\pysigstartsignatures
\pysiglinewithargsret{\sphinxbfcode{\sphinxupquote{classmethod\DUrole{w,w}{  }}}\sphinxbfcode{\sphinxupquote{from\_points}}}{\sphinxparam{\DUrole{n,n}{a}\DUrole{p,p}{:}\DUrole{w,w}{  }\DUrole{n,n}{{\hyperref[\detokenize{deepdrr.geo:deepdrr.geo.core.Point3D}]{\sphinxcrossref{Point3D}}}}}\sphinxparamcomma \sphinxparam{\DUrole{n,n}{b}\DUrole{p,p}{:}\DUrole{w,w}{  }\DUrole{n,n}{{\hyperref[\detokenize{deepdrr.geo:deepdrr.geo.core.Point3D}]{\sphinxcrossref{Point3D}}}}}\sphinxparamcomma \sphinxparam{\DUrole{n,n}{c}\DUrole{p,p}{:}\DUrole{w,w}{  }\DUrole{n,n}{{\hyperref[\detokenize{deepdrr.geo:deepdrr.geo.core.Point3D}]{\sphinxcrossref{Point3D}}}}}}{{ $\rightarrow$ None}}
\pysigstopsignatures
\sphinxAtStartPar
Initialize the plane containing three points.
\begin{quote}\begin{description}
\sphinxlineitem{Parameters}\begin{itemize}
\item {} 
\sphinxAtStartPar
\sphinxstyleliteralstrong{\sphinxupquote{a}} ({\hyperref[\detokenize{deepdrr.geo:deepdrr.geo.core.Point3D}]{\sphinxcrossref{\sphinxstyleliteralemphasis{\sphinxupquote{Point3D}}}}}) \textendash{} a point on the plane.

\item {} 
\sphinxAtStartPar
\sphinxstyleliteralstrong{\sphinxupquote{b}} ({\hyperref[\detokenize{deepdrr.geo:deepdrr.geo.core.Point3D}]{\sphinxcrossref{\sphinxstyleliteralemphasis{\sphinxupquote{Point3D}}}}}) \textendash{} a point on the plane.

\item {} 
\sphinxAtStartPar
\sphinxstyleliteralstrong{\sphinxupquote{c}} ({\hyperref[\detokenize{deepdrr.geo:deepdrr.geo.core.Point3D}]{\sphinxcrossref{\sphinxstyleliteralemphasis{\sphinxupquote{Point3D}}}}}) \textendash{} a point on the plane.

\end{itemize}

\sphinxlineitem{Returns}
\sphinxAtStartPar
The plane.

\sphinxlineitem{Return type}
\sphinxAtStartPar
{\hyperref[\detokenize{deepdrr.geo:deepdrr.geo.hyperplane.Plane}]{\sphinxcrossref{Plane}}}

\end{description}\end{quote}

\end{fulllineitems}

\index{get\_direction() (deepdrr.geo.hyperplane.Plane method)@\spxentry{get\_direction()}\spxextra{deepdrr.geo.hyperplane.Plane method}}

\begin{fulllineitems}
\phantomsection\label{\detokenize{deepdrr.geo:deepdrr.geo.hyperplane.Plane.get_direction}}
\pysigstartsignatures
\pysiglinewithargsret{\sphinxbfcode{\sphinxupquote{get\_direction}}}{}{{ $\rightarrow$ {\hyperref[\detokenize{deepdrr.geo:deepdrr.geo.core.Vector3D}]{\sphinxcrossref{Vector3D}}}}}
\pysigstopsignatures
\sphinxAtStartPar
Get the direction of the plane.
\begin{quote}\begin{description}
\sphinxlineitem{Returns}
\sphinxAtStartPar
The unit\sphinxhyphen{}length direction of the plane.

\sphinxlineitem{Return type}
\sphinxAtStartPar
{\hyperref[\detokenize{deepdrr.geo:deepdrr.geo.core.Vector3D}]{\sphinxcrossref{Vector3D}}}

\end{description}\end{quote}

\end{fulllineitems}

\index{get\_point() (deepdrr.geo.hyperplane.Plane method)@\spxentry{get\_point()}\spxextra{deepdrr.geo.hyperplane.Plane method}}

\begin{fulllineitems}
\phantomsection\label{\detokenize{deepdrr.geo:deepdrr.geo.hyperplane.Plane.get_point}}
\pysigstartsignatures
\pysiglinewithargsret{\sphinxbfcode{\sphinxupquote{get\_point}}}{}{{ $\rightarrow$ {\hyperref[\detokenize{deepdrr.geo:deepdrr.geo.core.Point3D}]{\sphinxcrossref{Point3D}}}}}
\pysigstopsignatures
\sphinxAtStartPar
Get an arbitrary point on the plane.
\begin{quote}\begin{description}
\sphinxlineitem{Returns}
\sphinxAtStartPar
A point on the plane.

\sphinxlineitem{Return type}
\sphinxAtStartPar
{\hyperref[\detokenize{deepdrr.geo:deepdrr.geo.core.Point3D}]{\sphinxcrossref{Point3D}}}

\end{description}\end{quote}

\end{fulllineitems}

\index{meet() (deepdrr.geo.hyperplane.Plane method)@\spxentry{meet()}\spxextra{deepdrr.geo.hyperplane.Plane method}}

\begin{fulllineitems}
\phantomsection\label{\detokenize{deepdrr.geo:deepdrr.geo.hyperplane.Plane.meet}}
\pysigstartsignatures
\pysiglinewithargsret{\sphinxbfcode{\sphinxupquote{meet}}}{\sphinxparam{\DUrole{n,n}{other}\DUrole{p,p}{:}\DUrole{w,w}{  }\DUrole{n,n}{{\hyperref[\detokenize{deepdrr.geo:deepdrr.geo.hyperplane.Plane}]{\sphinxcrossref{Plane}}}}}}{{ $\rightarrow$ {\hyperref[\detokenize{deepdrr.geo:deepdrr.geo.hyperplane.Line3D}]{\sphinxcrossref{Line3D}}}}}
\pysiglinewithargsret{\sphinxbfcode{\sphinxupquote{meet}}}{\sphinxparam{\DUrole{n,n}{other}\DUrole{p,p}{:}\DUrole{w,w}{  }\DUrole{n,n}{{\hyperref[\detokenize{deepdrr.geo:deepdrr.geo.hyperplane.Line3D}]{\sphinxcrossref{Line3D}}}}}}{{ $\rightarrow$ {\hyperref[\detokenize{deepdrr.geo:deepdrr.geo.core.Point3D}]{\sphinxcrossref{Point3D}}}}}
\pysigstopsignatures
\sphinxAtStartPar
Get the intersection of two objects.

\sphinxAtStartPar
For example, given two lines, get the line that is the intersection of them.
\begin{quote}\begin{description}
\sphinxlineitem{Parameters}
\sphinxAtStartPar
\sphinxstyleliteralstrong{\sphinxupquote{other}} ({\hyperref[\detokenize{deepdrr.geo:deepdrr.geo.core.Primitive}]{\sphinxcrossref{\sphinxstyleliteralemphasis{\sphinxupquote{Primitive}}}}}) \textendash{} the other primitive.

\sphinxlineitem{Returns}
\sphinxAtStartPar
the intersection of \sphinxtitleref{self} and \sphinxtitleref{other}.

\sphinxlineitem{Return type}
\sphinxAtStartPar
{\hyperref[\detokenize{deepdrr.geo:deepdrr.geo.core.Primitive}]{\sphinxcrossref{Primitive}}}

\sphinxlineitem{Raises}
\sphinxAtStartPar
{\hyperref[\detokenize{deepdrr.geo:deepdrr.geo.exceptions.MeetError}]{\sphinxcrossref{\sphinxstyleliteralstrong{\sphinxupquote{MeetError}}}}} \textendash{} if the objects cannot be intersected.

\end{description}\end{quote}

\end{fulllineitems}


\end{fulllineitems}

\index{l() (in module deepdrr.geo.hyperplane)@\spxentry{l()}\spxextra{in module deepdrr.geo.hyperplane}}

\begin{fulllineitems}
\phantomsection\label{\detokenize{deepdrr.geo:deepdrr.geo.hyperplane.l}}
\pysigstartsignatures
\pysiglinewithargsret{\sphinxcode{\sphinxupquote{deepdrr.geo.hyperplane.}}\sphinxbfcode{\sphinxupquote{l}}}{\sphinxparam{\DUrole{o,o}{*}\DUrole{n,n}{args}}}{}
\pysigstopsignatures
\end{fulllineitems}

\index{line() (in module deepdrr.geo.hyperplane)@\spxentry{line()}\spxextra{in module deepdrr.geo.hyperplane}}

\begin{fulllineitems}
\phantomsection\label{\detokenize{deepdrr.geo:deepdrr.geo.hyperplane.line}}
\pysigstartsignatures
\pysiglinewithargsret{\sphinxcode{\sphinxupquote{deepdrr.geo.hyperplane.}}\sphinxbfcode{\sphinxupquote{line}}}{\sphinxparam{\DUrole{n,n}{l}\DUrole{p,p}{:}\DUrole{w,w}{  }\DUrole{n,n}{{\hyperref[\detokenize{deepdrr.geo:deepdrr.geo.hyperplane.Line2D}]{\sphinxcrossref{Line2D}}}}}}{{ $\rightarrow$ {\hyperref[\detokenize{deepdrr.geo:deepdrr.geo.hyperplane.Line2D}]{\sphinxcrossref{Line2D}}}}}
\pysiglinewithargsret{\sphinxcode{\sphinxupquote{deepdrr.geo.hyperplane.}}\sphinxbfcode{\sphinxupquote{line}}}{\sphinxparam{\DUrole{n,n}{l}\DUrole{p,p}{:}\DUrole{w,w}{  }\DUrole{n,n}{{\hyperref[\detokenize{deepdrr.geo:deepdrr.geo.hyperplane.Line3D}]{\sphinxcrossref{Line3D}}}}}}{{ $\rightarrow$ {\hyperref[\detokenize{deepdrr.geo:deepdrr.geo.hyperplane.Line3D}]{\sphinxcrossref{Line3D}}}}}
\pysiglinewithargsret{\sphinxcode{\sphinxupquote{deepdrr.geo.hyperplane.}}\sphinxbfcode{\sphinxupquote{line}}}{\sphinxparam{\DUrole{n,n}{r}\DUrole{p,p}{:}\DUrole{w,w}{  }\DUrole{n,n}{{\hyperref[\detokenize{deepdrr.geo:deepdrr.geo.ray.Ray2D}]{\sphinxcrossref{Ray2D}}}}}}{{ $\rightarrow$ {\hyperref[\detokenize{deepdrr.geo:deepdrr.geo.hyperplane.Line2D}]{\sphinxcrossref{Line2D}}}}}
\pysiglinewithargsret{\sphinxcode{\sphinxupquote{deepdrr.geo.hyperplane.}}\sphinxbfcode{\sphinxupquote{line}}}{\sphinxparam{\DUrole{n,n}{r}\DUrole{p,p}{:}\DUrole{w,w}{  }\DUrole{n,n}{{\hyperref[\detokenize{deepdrr.geo:deepdrr.geo.ray.Ray3D}]{\sphinxcrossref{Ray3D}}}}}}{{ $\rightarrow$ {\hyperref[\detokenize{deepdrr.geo:deepdrr.geo.hyperplane.Line3D}]{\sphinxcrossref{Line3D}}}}}
\pysiglinewithargsret{\sphinxcode{\sphinxupquote{deepdrr.geo.hyperplane.}}\sphinxbfcode{\sphinxupquote{line}}}{\sphinxparam{\DUrole{n,n}{s}\DUrole{p,p}{:}\DUrole{w,w}{  }\DUrole{n,n}{{\hyperref[\detokenize{deepdrr.geo:deepdrr.geo.segment.Segment2D}]{\sphinxcrossref{Segment2D}}}}}}{{ $\rightarrow$ {\hyperref[\detokenize{deepdrr.geo:deepdrr.geo.hyperplane.Line2D}]{\sphinxcrossref{Line2D}}}}}
\pysiglinewithargsret{\sphinxcode{\sphinxupquote{deepdrr.geo.hyperplane.}}\sphinxbfcode{\sphinxupquote{line}}}{\sphinxparam{\DUrole{n,n}{s}\DUrole{p,p}{:}\DUrole{w,w}{  }\DUrole{n,n}{{\hyperref[\detokenize{deepdrr.geo:deepdrr.geo.segment.Segment3D}]{\sphinxcrossref{Segment3D}}}}}}{{ $\rightarrow$ {\hyperref[\detokenize{deepdrr.geo:deepdrr.geo.hyperplane.Line3D}]{\sphinxcrossref{Line3D}}}}}
\pysiglinewithargsret{\sphinxcode{\sphinxupquote{deepdrr.geo.hyperplane.}}\sphinxbfcode{\sphinxupquote{line}}}{\sphinxparam{\DUrole{n,n}{a}\DUrole{p,p}{:}\DUrole{w,w}{  }\DUrole{n,n}{float}}\sphinxparamcomma \sphinxparam{\DUrole{n,n}{b}\DUrole{p,p}{:}\DUrole{w,w}{  }\DUrole{n,n}{float}}\sphinxparamcomma \sphinxparam{\DUrole{n,n}{c}\DUrole{p,p}{:}\DUrole{w,w}{  }\DUrole{n,n}{float}}}{{ $\rightarrow$ {\hyperref[\detokenize{deepdrr.geo:deepdrr.geo.hyperplane.Line2D}]{\sphinxcrossref{Line2D}}}}}
\pysiglinewithargsret{\sphinxcode{\sphinxupquote{deepdrr.geo.hyperplane.}}\sphinxbfcode{\sphinxupquote{line}}}{\sphinxparam{\DUrole{n,n}{l}\DUrole{p,p}{:}\DUrole{w,w}{  }\DUrole{n,n}{ndarray}}}{{ $\rightarrow$ {\hyperref[\detokenize{deepdrr.geo:deepdrr.geo.hyperplane.Line}]{\sphinxcrossref{Line}}}}}
\pysiglinewithargsret{\sphinxcode{\sphinxupquote{deepdrr.geo.hyperplane.}}\sphinxbfcode{\sphinxupquote{line}}}{\sphinxparam{\DUrole{n,n}{p}\DUrole{p,p}{:}\DUrole{w,w}{  }\DUrole{n,n}{float}}\sphinxparamcomma \sphinxparam{\DUrole{n,n}{q}\DUrole{p,p}{:}\DUrole{w,w}{  }\DUrole{n,n}{float}}\sphinxparamcomma \sphinxparam{\DUrole{n,n}{r}\DUrole{p,p}{:}\DUrole{w,w}{  }\DUrole{n,n}{float}}\sphinxparamcomma \sphinxparam{\DUrole{n,n}{s}\DUrole{p,p}{:}\DUrole{w,w}{  }\DUrole{n,n}{float}}\sphinxparamcomma \sphinxparam{\DUrole{n,n}{t}\DUrole{p,p}{:}\DUrole{w,w}{  }\DUrole{n,n}{float}}\sphinxparamcomma \sphinxparam{\DUrole{n,n}{u}\DUrole{p,p}{:}\DUrole{w,w}{  }\DUrole{n,n}{float}}}{{ $\rightarrow$ {\hyperref[\detokenize{deepdrr.geo:deepdrr.geo.hyperplane.Line3D}]{\sphinxcrossref{Line3D}}}}}
\pysiglinewithargsret{\sphinxcode{\sphinxupquote{deepdrr.geo.hyperplane.}}\sphinxbfcode{\sphinxupquote{line}}}{\sphinxparam{\DUrole{n,n}{x}\DUrole{p,p}{:}\DUrole{w,w}{  }\DUrole{n,n}{{\hyperref[\detokenize{deepdrr.geo:deepdrr.geo.core.Point2D}]{\sphinxcrossref{Point2D}}}}}\sphinxparamcomma \sphinxparam{\DUrole{n,n}{y}\DUrole{p,p}{:}\DUrole{w,w}{  }\DUrole{n,n}{{\hyperref[\detokenize{deepdrr.geo:deepdrr.geo.core.Point2D}]{\sphinxcrossref{Point2D}}}}}}{{ $\rightarrow$ {\hyperref[\detokenize{deepdrr.geo:deepdrr.geo.hyperplane.Line2D}]{\sphinxcrossref{Line2D}}}}}
\pysiglinewithargsret{\sphinxcode{\sphinxupquote{deepdrr.geo.hyperplane.}}\sphinxbfcode{\sphinxupquote{line}}}{\sphinxparam{\DUrole{n,n}{x}\DUrole{p,p}{:}\DUrole{w,w}{  }\DUrole{n,n}{{\hyperref[\detokenize{deepdrr.geo:deepdrr.geo.core.Point3D}]{\sphinxcrossref{Point3D}}}}}\sphinxparamcomma \sphinxparam{\DUrole{n,n}{y}\DUrole{p,p}{:}\DUrole{w,w}{  }\DUrole{n,n}{{\hyperref[\detokenize{deepdrr.geo:deepdrr.geo.core.Point3D}]{\sphinxcrossref{Point3D}}}}}}{{ $\rightarrow$ {\hyperref[\detokenize{deepdrr.geo:deepdrr.geo.hyperplane.Line3D}]{\sphinxcrossref{Line3D}}}}}
\pysiglinewithargsret{\sphinxcode{\sphinxupquote{deepdrr.geo.hyperplane.}}\sphinxbfcode{\sphinxupquote{line}}}{\sphinxparam{\DUrole{n,n}{a}\DUrole{p,p}{:}\DUrole{w,w}{  }\DUrole{n,n}{{\hyperref[\detokenize{deepdrr.geo:deepdrr.geo.hyperplane.Plane}]{\sphinxcrossref{Plane}}}}}\sphinxparamcomma \sphinxparam{\DUrole{n,n}{b}\DUrole{p,p}{:}\DUrole{w,w}{  }\DUrole{n,n}{{\hyperref[\detokenize{deepdrr.geo:deepdrr.geo.hyperplane.Plane}]{\sphinxcrossref{Plane}}}}}}{{ $\rightarrow$ {\hyperref[\detokenize{deepdrr.geo:deepdrr.geo.hyperplane.Line3D}]{\sphinxcrossref{Line3D}}}}}
\pysiglinewithargsret{\sphinxcode{\sphinxupquote{deepdrr.geo.hyperplane.}}\sphinxbfcode{\sphinxupquote{line}}}{\sphinxparam{\DUrole{n,n}{x}\DUrole{p,p}{:}\DUrole{w,w}{  }\DUrole{n,n}{{\hyperref[\detokenize{deepdrr.geo:deepdrr.geo.core.Point2D}]{\sphinxcrossref{Point2D}}}}}\sphinxparamcomma \sphinxparam{\DUrole{n,n}{v}\DUrole{p,p}{:}\DUrole{w,w}{  }\DUrole{n,n}{{\hyperref[\detokenize{deepdrr.geo:deepdrr.geo.core.Vector2D}]{\sphinxcrossref{Vector2D}}}}}}{{ $\rightarrow$ {\hyperref[\detokenize{deepdrr.geo:deepdrr.geo.hyperplane.Line2D}]{\sphinxcrossref{Line2D}}}}}
\pysiglinewithargsret{\sphinxcode{\sphinxupquote{deepdrr.geo.hyperplane.}}\sphinxbfcode{\sphinxupquote{line}}}{\sphinxparam{\DUrole{o,o}{*}\DUrole{n,n}{args}\DUrole{p,p}{:}\DUrole{w,w}{  }\DUrole{n,n}{Any}}}{{ $\rightarrow$ {\hyperref[\detokenize{deepdrr.geo:deepdrr.geo.hyperplane.Line}]{\sphinxcrossref{Line}}}}}
\pysigstopsignatures
\sphinxAtStartPar
The preferred method for creating a line.

\sphinxAtStartPar
Can create a line using one of the following methods:
\sphinxhyphen{} Pass the coordinates as separate arguments. For instance, \sphinxtitleref{line(1, 2, 3)} returns the 2D homogeneous line \sphinxtitleref{1x + 2y + 3 = 0}.
\sphinxhyphen{} Pass a numpy array with the homogeneous coordinates (NOTE THE DIFFERENCE WITH \sphinxtitleref{point} and \sphinxtitleref{vector}).
\sphinxhyphen{} Pass a Line2D or Line3D instance, in which case \sphinxtitleref{line()} is a no\sphinxhyphen{}op.
\sphinxhyphen{} Pass two points of the same dimension, in which case \sphinxtitleref{line()} returns the line through the points.
\sphinxhyphen{} Pass two planes, in which case \sphinxtitleref{line()} returns the line of intersection of the planes.

\end{fulllineitems}

\index{pl() (in module deepdrr.geo.hyperplane)@\spxentry{pl()}\spxextra{in module deepdrr.geo.hyperplane}}

\begin{fulllineitems}
\phantomsection\label{\detokenize{deepdrr.geo:deepdrr.geo.hyperplane.pl}}
\pysigstartsignatures
\pysiglinewithargsret{\sphinxcode{\sphinxupquote{deepdrr.geo.hyperplane.}}\sphinxbfcode{\sphinxupquote{pl}}}{\sphinxparam{\DUrole{o,o}{*}\DUrole{n,n}{args}}}{}
\pysigstopsignatures
\end{fulllineitems}

\index{plane() (in module deepdrr.geo.hyperplane)@\spxentry{plane()}\spxextra{in module deepdrr.geo.hyperplane}}

\begin{fulllineitems}
\phantomsection\label{\detokenize{deepdrr.geo:deepdrr.geo.hyperplane.plane}}
\pysigstartsignatures
\pysiglinewithargsret{\sphinxcode{\sphinxupquote{deepdrr.geo.hyperplane.}}\sphinxbfcode{\sphinxupquote{plane}}}{\sphinxparam{\DUrole{n,n}{p}\DUrole{p,p}{:}\DUrole{w,w}{  }\DUrole{n,n}{{\hyperref[\detokenize{deepdrr.geo:deepdrr.geo.hyperplane.Plane}]{\sphinxcrossref{Plane}}}}}}{{ $\rightarrow$ {\hyperref[\detokenize{deepdrr.geo:deepdrr.geo.hyperplane.Plane}]{\sphinxcrossref{Plane}}}}}
\pysiglinewithargsret{\sphinxcode{\sphinxupquote{deepdrr.geo.hyperplane.}}\sphinxbfcode{\sphinxupquote{plane}}}{\sphinxparam{\DUrole{n,n}{r}\DUrole{p,p}{:}\DUrole{w,w}{  }\DUrole{n,n}{{\hyperref[\detokenize{deepdrr.geo:deepdrr.geo.ray.Ray3D}]{\sphinxcrossref{Ray3D}}}}}}{{ $\rightarrow$ {\hyperref[\detokenize{deepdrr.geo:deepdrr.geo.ray.Ray3D}]{\sphinxcrossref{Ray3D}}}}}
\pysiglinewithargsret{\sphinxcode{\sphinxupquote{deepdrr.geo.hyperplane.}}\sphinxbfcode{\sphinxupquote{plane}}}{\sphinxparam{\DUrole{n,n}{a}\DUrole{p,p}{:}\DUrole{w,w}{  }\DUrole{n,n}{float}}\sphinxparamcomma \sphinxparam{\DUrole{n,n}{b}\DUrole{p,p}{:}\DUrole{w,w}{  }\DUrole{n,n}{float}}\sphinxparamcomma \sphinxparam{\DUrole{n,n}{c}\DUrole{p,p}{:}\DUrole{w,w}{  }\DUrole{n,n}{float}}\sphinxparamcomma \sphinxparam{\DUrole{n,n}{d}\DUrole{p,p}{:}\DUrole{w,w}{  }\DUrole{n,n}{float}}}{{ $\rightarrow$ {\hyperref[\detokenize{deepdrr.geo:deepdrr.geo.hyperplane.Plane}]{\sphinxcrossref{Plane}}}}}
\pysiglinewithargsret{\sphinxcode{\sphinxupquote{deepdrr.geo.hyperplane.}}\sphinxbfcode{\sphinxupquote{plane}}}{\sphinxparam{\DUrole{n,n}{x}\DUrole{p,p}{:}\DUrole{w,w}{  }\DUrole{n,n}{ndarray}}}{{ $\rightarrow$ {\hyperref[\detokenize{deepdrr.geo:deepdrr.geo.hyperplane.Plane}]{\sphinxcrossref{Plane}}}}}
\pysiglinewithargsret{\sphinxcode{\sphinxupquote{deepdrr.geo.hyperplane.}}\sphinxbfcode{\sphinxupquote{plane}}}{\sphinxparam{\DUrole{n,n}{p}\DUrole{p,p}{:}\DUrole{w,w}{  }\DUrole{n,n}{{\hyperref[\detokenize{deepdrr.geo:deepdrr.geo.core.Point3D}]{\sphinxcrossref{Point3D}}}}}\sphinxparamcomma \sphinxparam{\DUrole{n,n}{n}\DUrole{p,p}{:}\DUrole{w,w}{  }\DUrole{n,n}{{\hyperref[\detokenize{deepdrr.geo:deepdrr.geo.core.Vector3D}]{\sphinxcrossref{Vector3D}}}}}}{{ $\rightarrow$ {\hyperref[\detokenize{deepdrr.geo:deepdrr.geo.hyperplane.Plane}]{\sphinxcrossref{Plane}}}}}
\pysigstopsignatures
\sphinxAtStartPar
The preferred method for creating a plane.

\sphinxAtStartPar
Can create a plane using one of the following methods:
\sphinxhyphen{} Pass the coordinates as separate arguments. For instance, \sphinxtitleref{plane(1, 2, 3, 4)} returns the 2D homogeneous plane \sphinxtitleref{1x + 2y + 3z + 4 = 0}.
\sphinxhyphen{} Pass a numpy array with the homogeneous coordinates.
\sphinxhyphen{} Pass a Plane instance, in which case \sphinxtitleref{plane()} is a no\sphinxhyphen{}op.
\sphinxhyphen{} Pass a Point3D and Vector3D instance, in which case \sphinxtitleref{plane(p, n)} returns the plane corresponding to
\sphinxhyphen{} Pass a ray, which defines r, n as above.

\end{fulllineitems}



\subsection{deepdrr.geo.random}
\label{\detokenize{deepdrr.geo:module-deepdrr.geo.random}}\label{\detokenize{deepdrr.geo:deepdrr-geo-random}}\index{module@\spxentry{module}!deepdrr.geo.random@\spxentry{deepdrr.geo.random}}\index{deepdrr.geo.random@\spxentry{deepdrr.geo.random}!module@\spxentry{module}}\index{normal() (in module deepdrr.geo.random)@\spxentry{normal()}\spxextra{in module deepdrr.geo.random}}

\begin{fulllineitems}
\phantomsection\label{\detokenize{deepdrr.geo:deepdrr.geo.random.normal}}
\pysigstartsignatures
\pysiglinewithargsret{\sphinxcode{\sphinxupquote{deepdrr.geo.random.}}\sphinxbfcode{\sphinxupquote{normal}}}{\sphinxparam{\DUrole{n,n}{center}\DUrole{p,p}{:}\DUrole{w,w}{  }\DUrole{n,n}{{\hyperref[\detokenize{deepdrr.geo:deepdrr.geo.core.Point3D}]{\sphinxcrossref{Point3D}}}}}\sphinxparamcomma \sphinxparam{\DUrole{n,n}{scale}\DUrole{p,p}{:}\DUrole{w,w}{  }\DUrole{n,n}{float}}\sphinxparamcomma \sphinxparam{\DUrole{n,n}{radius}\DUrole{p,p}{:}\DUrole{w,w}{  }\DUrole{n,n}{float\DUrole{w,w}{  }\DUrole{p,p}{|}\DUrole{w,w}{  }None}}\sphinxparamcomma \sphinxparam{\DUrole{n,n}{n}\DUrole{p,p}{:}\DUrole{w,w}{  }\DUrole{n,n}{int}}}{{ $\rightarrow$ List\DUrole{p,p}{{[}}{\hyperref[\detokenize{deepdrr.geo:deepdrr.geo.core.Point3D}]{\sphinxcrossref{Point3D}}}\DUrole{p,p}{{]}}}}
\pysiglinewithargsret{\sphinxcode{\sphinxupquote{deepdrr.geo.random.}}\sphinxbfcode{\sphinxupquote{normal}}}{\sphinxparam{\DUrole{n,n}{center}\DUrole{p,p}{:}\DUrole{w,w}{  }\DUrole{n,n}{{\hyperref[\detokenize{deepdrr.geo:deepdrr.geo.core.Point3D}]{\sphinxcrossref{Point3D}}}}}\sphinxparamcomma \sphinxparam{\DUrole{n,n}{scale}\DUrole{p,p}{:}\DUrole{w,w}{  }\DUrole{n,n}{float}}\sphinxparamcomma \sphinxparam{\DUrole{n,n}{radius}\DUrole{p,p}{:}\DUrole{w,w}{  }\DUrole{n,n}{float\DUrole{w,w}{  }\DUrole{p,p}{|}\DUrole{w,w}{  }None}}\sphinxparamcomma \sphinxparam{\DUrole{n,n}{n}\DUrole{p,p}{:}\DUrole{w,w}{  }\DUrole{n,n}{None}}}{{ $\rightarrow$ {\hyperref[\detokenize{deepdrr.geo:deepdrr.geo.core.Point3D}]{\sphinxcrossref{Point3D}}}}}
\pysigstopsignatures
\sphinxAtStartPar
Sample points from a clipped normal distribution.
\begin{quote}\begin{description}
\sphinxlineitem{Parameters}\begin{itemize}
\item {} 
\sphinxAtStartPar
\sphinxstyleliteralstrong{\sphinxupquote{center}} ({\hyperref[\detokenize{deepdrr.geo:deepdrr.geo.core.Point3D}]{\sphinxcrossref{\sphinxstyleliteralemphasis{\sphinxupquote{Point3D}}}}}) \textendash{} The center of the distribution.

\item {} 
\sphinxAtStartPar
\sphinxstyleliteralstrong{\sphinxupquote{scale}} (\sphinxstyleliteralemphasis{\sphinxupquote{float}}) \textendash{} The standard deviation of the distribution.

\item {} 
\sphinxAtStartPar
\sphinxstyleliteralstrong{\sphinxupquote{radius}} (\sphinxstyleliteralemphasis{\sphinxupquote{float}}) \textendash{} The radius of the distribution.

\item {} 
\sphinxAtStartPar
\sphinxstyleliteralstrong{\sphinxupquote{n}} (\sphinxstyleliteralemphasis{\sphinxupquote{int}}) \textendash{} The number of points to sample.

\end{itemize}

\sphinxlineitem{Returns}
\sphinxAtStartPar
The sampled point or points, if n is not None.

\sphinxlineitem{Return type}
\sphinxAtStartPar
{\hyperref[\detokenize{deepdrr.geo:deepdrr.geo.core.Point3D}]{\sphinxcrossref{Point3D}}}

\end{description}\end{quote}

\end{fulllineitems}

\index{spherical\_uniform() (in module deepdrr.geo.random)@\spxentry{spherical\_uniform()}\spxextra{in module deepdrr.geo.random}}

\begin{fulllineitems}
\phantomsection\label{\detokenize{deepdrr.geo:deepdrr.geo.random.spherical_uniform}}
\pysigstartsignatures
\pysiglinewithargsret{\sphinxcode{\sphinxupquote{deepdrr.geo.random.}}\sphinxbfcode{\sphinxupquote{spherical\_uniform}}}{\sphinxparam{\DUrole{n,n}{center}\DUrole{p,p}{:}\DUrole{w,w}{  }\DUrole{n,n}{{\hyperref[\detokenize{deepdrr.geo:deepdrr.geo.core.Vector3D}]{\sphinxcrossref{Vector3D}}}}}\sphinxparamcomma \sphinxparam{\DUrole{n,n}{d\_phi}\DUrole{p,p}{:}\DUrole{w,w}{  }\DUrole{n,n}{float}}\sphinxparamcomma \sphinxparam{\DUrole{n,n}{n}\DUrole{p,p}{:}\DUrole{w,w}{  }\DUrole{n,n}{int}}}{{ $\rightarrow$ List\DUrole{p,p}{{[}}{\hyperref[\detokenize{deepdrr.geo:deepdrr.geo.core.Vector3D}]{\sphinxcrossref{Vector3D}}}\DUrole{p,p}{{]}}}}
\pysiglinewithargsret{\sphinxcode{\sphinxupquote{deepdrr.geo.random.}}\sphinxbfcode{\sphinxupquote{spherical\_uniform}}}{\sphinxparam{\DUrole{n,n}{center}\DUrole{p,p}{:}\DUrole{w,w}{  }\DUrole{n,n}{{\hyperref[\detokenize{deepdrr.geo:deepdrr.geo.core.Vector3D}]{\sphinxcrossref{Vector3D}}}}}\sphinxparamcomma \sphinxparam{\DUrole{n,n}{d\_phi}\DUrole{p,p}{:}\DUrole{w,w}{  }\DUrole{n,n}{float}}\sphinxparamcomma \sphinxparam{\DUrole{n,n}{n}\DUrole{p,p}{:}\DUrole{w,w}{  }\DUrole{n,n}{None}}}{{ $\rightarrow$ {\hyperref[\detokenize{deepdrr.geo:deepdrr.geo.core.Vector3D}]{\sphinxcrossref{Vector3D}}}}}
\pysigstopsignatures
\sphinxAtStartPar
Sample unit vectors on the surface of the sphere within \sphinxtitleref{d\_phi} radians of \sphinxtitleref{v}.

\end{fulllineitems}

\index{uniform() (in module deepdrr.geo.random)@\spxentry{uniform()}\spxextra{in module deepdrr.geo.random}}

\begin{fulllineitems}
\phantomsection\label{\detokenize{deepdrr.geo:deepdrr.geo.random.uniform}}
\pysigstartsignatures
\pysiglinewithargsret{\sphinxcode{\sphinxupquote{deepdrr.geo.random.}}\sphinxbfcode{\sphinxupquote{uniform}}}{\sphinxparam{\DUrole{n,n}{center}\DUrole{p,p}{:}\DUrole{w,w}{  }\DUrole{n,n}{{\hyperref[\detokenize{deepdrr.geo:deepdrr.geo.core.Point3D}]{\sphinxcrossref{Point3D}}}}}\sphinxparamcomma \sphinxparam{\DUrole{n,n}{radius}\DUrole{p,p}{:}\DUrole{w,w}{  }\DUrole{n,n}{float}}\sphinxparamcomma \sphinxparam{\DUrole{n,n}{n}\DUrole{p,p}{:}\DUrole{w,w}{  }\DUrole{n,n}{int}}}{{ $\rightarrow$ List\DUrole{p,p}{{[}}{\hyperref[\detokenize{deepdrr.geo:deepdrr.geo.core.Point3D}]{\sphinxcrossref{Point3D}}}\DUrole{p,p}{{]}}}}
\pysiglinewithargsret{\sphinxcode{\sphinxupquote{deepdrr.geo.random.}}\sphinxbfcode{\sphinxupquote{uniform}}}{\sphinxparam{\DUrole{n,n}{center}\DUrole{p,p}{:}\DUrole{w,w}{  }\DUrole{n,n}{{\hyperref[\detokenize{deepdrr.geo:deepdrr.geo.core.Point3D}]{\sphinxcrossref{Point3D}}}}}\sphinxparamcomma \sphinxparam{\DUrole{n,n}{radius}\DUrole{p,p}{:}\DUrole{w,w}{  }\DUrole{n,n}{float}}\sphinxparamcomma \sphinxparam{\DUrole{n,n}{n}\DUrole{p,p}{:}\DUrole{w,w}{  }\DUrole{n,n}{None}}}{{ $\rightarrow$ {\hyperref[\detokenize{deepdrr.geo:deepdrr.geo.core.Point3D}]{\sphinxcrossref{Point3D}}}}}
\pysigstopsignatures
\sphinxAtStartPar
Sample points from a uniform distribution, bounded by a sphere.
\begin{quote}\begin{description}
\sphinxlineitem{Parameters}\begin{itemize}
\item {} 
\sphinxAtStartPar
\sphinxstyleliteralstrong{\sphinxupquote{center}} ({\hyperref[\detokenize{deepdrr.geo:deepdrr.geo.core.Point3D}]{\sphinxcrossref{\sphinxstyleliteralemphasis{\sphinxupquote{Point3D}}}}}) \textendash{} The center of the distribution.

\item {} 
\sphinxAtStartPar
\sphinxstyleliteralstrong{\sphinxupquote{radius}} (\sphinxstyleliteralemphasis{\sphinxupquote{float}}) \textendash{} The radius of the distribution. Defaults to 1.

\item {} 
\sphinxAtStartPar
\sphinxstyleliteralstrong{\sphinxupquote{n}} (\sphinxstyleliteralemphasis{\sphinxupquote{int}}) \textendash{} The number of points to sample. Defaults to None.

\end{itemize}

\sphinxlineitem{Returns}
\sphinxAtStartPar
The sampled point or points, if n is not None.

\sphinxlineitem{Return type}
\sphinxAtStartPar
{\hyperref[\detokenize{deepdrr.geo:deepdrr.geo.core.Point3D}]{\sphinxcrossref{Point3D}}}

\end{description}\end{quote}

\end{fulllineitems}



\subsection{deepdrr.geo.ray}
\label{\detokenize{deepdrr.geo:module-deepdrr.geo.ray}}\label{\detokenize{deepdrr.geo:deepdrr-geo-ray}}\index{module@\spxentry{module}!deepdrr.geo.ray@\spxentry{deepdrr.geo.ray}}\index{deepdrr.geo.ray@\spxentry{deepdrr.geo.ray}!module@\spxentry{module}}\index{Ray (class in deepdrr.geo.ray)@\spxentry{Ray}\spxextra{class in deepdrr.geo.ray}}

\begin{fulllineitems}
\phantomsection\label{\detokenize{deepdrr.geo:deepdrr.geo.ray.Ray}}
\pysigstartsignatures
\pysiglinewithargsret{\sphinxbfcode{\sphinxupquote{class\DUrole{w,w}{  }}}\sphinxcode{\sphinxupquote{deepdrr.geo.ray.}}\sphinxbfcode{\sphinxupquote{Ray}}}{\sphinxparam{\DUrole{n,n}{data}\DUrole{p,p}{:}\DUrole{w,w}{  }\DUrole{n,n}{ndarray}}}{}
\pysigstopsignatures
\sphinxAtStartPar
Bases: {\hyperref[\detokenize{deepdrr.geo:deepdrr.geo.core.HasLocationAndDirection}]{\sphinxcrossref{\sphinxcode{\sphinxupquote{HasLocationAndDirection}}}}}, {\hyperref[\detokenize{deepdrr.geo:deepdrr.geo.core.Meetable}]{\sphinxcrossref{\sphinxcode{\sphinxupquote{Meetable}}}}}
\index{data (deepdrr.geo.ray.Ray attribute)@\spxentry{data}\spxextra{deepdrr.geo.ray.Ray attribute}}

\begin{fulllineitems}
\phantomsection\label{\detokenize{deepdrr.geo:deepdrr.geo.ray.Ray.data}}
\pysigstartsignatures
\pysigline{\sphinxbfcode{\sphinxupquote{data}}\sphinxbfcode{\sphinxupquote{\DUrole{p,p}{:}\DUrole{w,w}{  }ndarray}}}
\pysigstopsignatures
\end{fulllineitems}

\index{from\_pn() (deepdrr.geo.ray.Ray class method)@\spxentry{from\_pn()}\spxextra{deepdrr.geo.ray.Ray class method}}

\begin{fulllineitems}
\phantomsection\label{\detokenize{deepdrr.geo:deepdrr.geo.ray.Ray.from_pn}}
\pysigstartsignatures
\pysiglinewithargsret{\sphinxbfcode{\sphinxupquote{classmethod\DUrole{w,w}{  }}}\sphinxbfcode{\sphinxupquote{from\_pn}}}{\sphinxparam{\DUrole{n,n}{p}\DUrole{p,p}{:}\DUrole{w,w}{  }\DUrole{n,n}{{\hyperref[\detokenize{deepdrr.geo:deepdrr.geo.core.Point}]{\sphinxcrossref{Point}}}}}\sphinxparamcomma \sphinxparam{\DUrole{n,n}{d}\DUrole{p,p}{:}\DUrole{w,w}{  }\DUrole{n,n}{{\hyperref[\detokenize{deepdrr.geo:deepdrr.geo.core.Vector}]{\sphinxcrossref{Vector}}}}}}{{ $\rightarrow$ R}}
\pysigstopsignatures
\sphinxAtStartPar
Create a ray from a point and a direction.

\end{fulllineitems}

\index{from\_point\_direction() (deepdrr.geo.ray.Ray class method)@\spxentry{from\_point\_direction()}\spxextra{deepdrr.geo.ray.Ray class method}}

\begin{fulllineitems}
\phantomsection\label{\detokenize{deepdrr.geo:deepdrr.geo.ray.Ray.from_point_direction}}
\pysigstartsignatures
\pysiglinewithargsret{\sphinxbfcode{\sphinxupquote{classmethod\DUrole{w,w}{  }}}\sphinxbfcode{\sphinxupquote{from\_point\_direction}}}{\sphinxparam{\DUrole{n,n}{p}\DUrole{p,p}{:}\DUrole{w,w}{  }\DUrole{n,n}{{\hyperref[\detokenize{deepdrr.geo:deepdrr.geo.core.Point}]{\sphinxcrossref{Point}}}}}\sphinxparamcomma \sphinxparam{\DUrole{n,n}{d}\DUrole{p,p}{:}\DUrole{w,w}{  }\DUrole{n,n}{{\hyperref[\detokenize{deepdrr.geo:deepdrr.geo.core.Vector}]{\sphinxcrossref{Vector}}}}}}{{ $\rightarrow$ {\hyperref[\detokenize{deepdrr.geo:deepdrr.geo.ray.Ray}]{\sphinxcrossref{Ray}}}}}
\pysigstopsignatures
\sphinxAtStartPar
Create a ray from a point and a direction.

\end{fulllineitems}

\index{from\_pq() (deepdrr.geo.ray.Ray class method)@\spxentry{from\_pq()}\spxextra{deepdrr.geo.ray.Ray class method}}

\begin{fulllineitems}
\phantomsection\label{\detokenize{deepdrr.geo:deepdrr.geo.ray.Ray.from_pq}}
\pysigstartsignatures
\pysiglinewithargsret{\sphinxbfcode{\sphinxupquote{classmethod\DUrole{w,w}{  }}}\sphinxbfcode{\sphinxupquote{from\_pq}}}{\sphinxparam{\DUrole{n,n}{p}\DUrole{p,p}{:}\DUrole{w,w}{  }\DUrole{n,n}{{\hyperref[\detokenize{deepdrr.geo:deepdrr.geo.core.Point}]{\sphinxcrossref{Point}}}}}\sphinxparamcomma \sphinxparam{\DUrole{n,n}{q}\DUrole{p,p}{:}\DUrole{w,w}{  }\DUrole{n,n}{{\hyperref[\detokenize{deepdrr.geo:deepdrr.geo.core.Point}]{\sphinxcrossref{Point}}}}}}{{ $\rightarrow$ R}}
\pysigstopsignatures
\sphinxAtStartPar
Create a ray from two points.

\sphinxAtStartPar
The point q is not preserved in the ray.
\begin{quote}\begin{description}
\sphinxlineitem{Parameters}\begin{itemize}
\item {} 
\sphinxAtStartPar
\sphinxstyleliteralstrong{\sphinxupquote{p}} ({\hyperref[\detokenize{deepdrr.geo:deepdrr.geo.core.Point3D}]{\sphinxcrossref{\sphinxstyleliteralemphasis{\sphinxupquote{Point3D}}}}}) \textendash{} The origin of the ray.

\item {} 
\sphinxAtStartPar
\sphinxstyleliteralstrong{\sphinxupquote{q}} ({\hyperref[\detokenize{deepdrr.geo:deepdrr.geo.core.Point3D}]{\sphinxcrossref{\sphinxstyleliteralemphasis{\sphinxupquote{Point3D}}}}}) \textendash{} A point on the ray.

\end{itemize}

\end{description}\end{quote}

\end{fulllineitems}

\index{get\_direction() (deepdrr.geo.ray.Ray method)@\spxentry{get\_direction()}\spxextra{deepdrr.geo.ray.Ray method}}

\begin{fulllineitems}
\phantomsection\label{\detokenize{deepdrr.geo:deepdrr.geo.ray.Ray.get_direction}}
\pysigstartsignatures
\pysiglinewithargsret{\sphinxbfcode{\sphinxupquote{get\_direction}}}{}{{ $\rightarrow$ {\hyperref[\detokenize{deepdrr.geo:deepdrr.geo.core.Vector3D}]{\sphinxcrossref{Vector3D}}}}}
\pysigstopsignatures
\sphinxAtStartPar
Get the direction associated with the object.
\begin{quote}\begin{description}
\sphinxlineitem{Returns}
\sphinxAtStartPar
the direction of the object.

\sphinxlineitem{Return type}
\sphinxAtStartPar
{\hyperref[\detokenize{deepdrr.geo:deepdrr.geo.core.Vector}]{\sphinxcrossref{Vector}}}

\end{description}\end{quote}

\end{fulllineitems}

\index{get\_point() (deepdrr.geo.ray.Ray method)@\spxentry{get\_point()}\spxextra{deepdrr.geo.ray.Ray method}}

\begin{fulllineitems}
\phantomsection\label{\detokenize{deepdrr.geo:deepdrr.geo.ray.Ray.get_point}}
\pysigstartsignatures
\pysiglinewithargsret{\sphinxbfcode{\sphinxupquote{get\_point}}}{}{{ $\rightarrow$ {\hyperref[\detokenize{deepdrr.geo:deepdrr.geo.core.Point3D}]{\sphinxcrossref{Point3D}}}}}
\pysigstopsignatures
\sphinxAtStartPar
Get the location of the object.
\begin{quote}\begin{description}
\sphinxlineitem{Returns}
\sphinxAtStartPar
the location of the object.

\sphinxlineitem{Return type}
\sphinxAtStartPar
{\hyperref[\detokenize{deepdrr.geo:deepdrr.geo.core.Point}]{\sphinxcrossref{Point}}}

\end{description}\end{quote}

\end{fulllineitems}

\index{n (deepdrr.geo.ray.Ray property)@\spxentry{n}\spxextra{deepdrr.geo.ray.Ray property}}

\begin{fulllineitems}
\phantomsection\label{\detokenize{deepdrr.geo:deepdrr.geo.ray.Ray.n}}
\pysigstartsignatures
\pysigline{\sphinxbfcode{\sphinxupquote{property\DUrole{w,w}{  }}}\sphinxbfcode{\sphinxupquote{n}}\sphinxbfcode{\sphinxupquote{\DUrole{p,p}{:}\DUrole{w,w}{  }{\hyperref[\detokenize{deepdrr.geo:deepdrr.geo.core.Vector}]{\sphinxcrossref{Vector}}}}}}
\pysigstopsignatures
\end{fulllineitems}

\index{p (deepdrr.geo.ray.Ray property)@\spxentry{p}\spxextra{deepdrr.geo.ray.Ray property}}

\begin{fulllineitems}
\phantomsection\label{\detokenize{deepdrr.geo:deepdrr.geo.ray.Ray.p}}
\pysigstartsignatures
\pysigline{\sphinxbfcode{\sphinxupquote{property\DUrole{w,w}{  }}}\sphinxbfcode{\sphinxupquote{p}}\sphinxbfcode{\sphinxupquote{\DUrole{p,p}{:}\DUrole{w,w}{  }{\hyperref[\detokenize{deepdrr.geo:deepdrr.geo.core.Point3D}]{\sphinxcrossref{Point3D}}}}}}
\pysigstopsignatures
\end{fulllineitems}


\end{fulllineitems}

\index{Ray2D (class in deepdrr.geo.ray)@\spxentry{Ray2D}\spxextra{class in deepdrr.geo.ray}}

\begin{fulllineitems}
\phantomsection\label{\detokenize{deepdrr.geo:deepdrr.geo.ray.Ray2D}}
\pysigstartsignatures
\pysiglinewithargsret{\sphinxbfcode{\sphinxupquote{class\DUrole{w,w}{  }}}\sphinxcode{\sphinxupquote{deepdrr.geo.ray.}}\sphinxbfcode{\sphinxupquote{Ray2D}}}{\sphinxparam{\DUrole{n,n}{data}\DUrole{p,p}{:}\DUrole{w,w}{  }\DUrole{n,n}{ndarray}}}{}
\pysigstopsignatures
\sphinxAtStartPar
Bases: {\hyperref[\detokenize{deepdrr.geo:deepdrr.geo.ray.Ray}]{\sphinxcrossref{\sphinxcode{\sphinxupquote{Ray}}}}}
\index{data (deepdrr.geo.ray.Ray2D attribute)@\spxentry{data}\spxextra{deepdrr.geo.ray.Ray2D attribute}}

\begin{fulllineitems}
\phantomsection\label{\detokenize{deepdrr.geo:deepdrr.geo.ray.Ray2D.data}}
\pysigstartsignatures
\pysigline{\sphinxbfcode{\sphinxupquote{data}}\sphinxbfcode{\sphinxupquote{\DUrole{p,p}{:}\DUrole{w,w}{  }ndarray}}}
\pysigstopsignatures
\end{fulllineitems}

\index{dim (deepdrr.geo.ray.Ray2D attribute)@\spxentry{dim}\spxextra{deepdrr.geo.ray.Ray2D attribute}}

\begin{fulllineitems}
\phantomsection\label{\detokenize{deepdrr.geo:deepdrr.geo.ray.Ray2D.dim}}
\pysigstartsignatures
\pysigline{\sphinxbfcode{\sphinxupquote{dim}}\sphinxbfcode{\sphinxupquote{\DUrole{w,w}{  }\DUrole{p,p}{=}\DUrole{w,w}{  }2}}}
\pysigstopsignatures
\end{fulllineitems}

\index{meet() (deepdrr.geo.ray.Ray2D method)@\spxentry{meet()}\spxextra{deepdrr.geo.ray.Ray2D method}}

\begin{fulllineitems}
\phantomsection\label{\detokenize{deepdrr.geo:deepdrr.geo.ray.Ray2D.meet}}
\pysigstartsignatures
\pysiglinewithargsret{\sphinxbfcode{\sphinxupquote{meet}}}{\sphinxparam{\DUrole{n,n}{other}\DUrole{p,p}{:}\DUrole{w,w}{  }\DUrole{n,n}{{\hyperref[\detokenize{deepdrr.geo:deepdrr.geo.hyperplane.Line2D}]{\sphinxcrossref{Line2D}}}\DUrole{w,w}{  }\DUrole{p,p}{|}\DUrole{w,w}{  }{\hyperref[\detokenize{deepdrr.geo:deepdrr.geo.segment.Segment2D}]{\sphinxcrossref{Segment2D}}}}}}{{ $\rightarrow$ {\hyperref[\detokenize{deepdrr.geo:deepdrr.geo.core.Point2D}]{\sphinxcrossref{Point2D}}}}}
\pysigstopsignatures
\sphinxAtStartPar
Get the point of intersection between this ray and another line.
\begin{quote}\begin{description}
\sphinxlineitem{Parameters}
\sphinxAtStartPar
\sphinxstyleliteralstrong{\sphinxupquote{other}} ({\hyperref[\detokenize{deepdrr.geo:deepdrr.geo.hyperplane.Line2D}]{\sphinxcrossref{\sphinxstyleliteralemphasis{\sphinxupquote{Line2D}}}}}) \textendash{} The other line.

\sphinxlineitem{Returns}
\sphinxAtStartPar
The point of intersection.

\sphinxlineitem{Return type}
\sphinxAtStartPar
{\hyperref[\detokenize{deepdrr.geo:deepdrr.geo.core.Point2D}]{\sphinxcrossref{Point2D}}}

\end{description}\end{quote}

\end{fulllineitems}


\end{fulllineitems}

\index{Ray3D (class in deepdrr.geo.ray)@\spxentry{Ray3D}\spxextra{class in deepdrr.geo.ray}}

\begin{fulllineitems}
\phantomsection\label{\detokenize{deepdrr.geo:deepdrr.geo.ray.Ray3D}}
\pysigstartsignatures
\pysiglinewithargsret{\sphinxbfcode{\sphinxupquote{class\DUrole{w,w}{  }}}\sphinxcode{\sphinxupquote{deepdrr.geo.ray.}}\sphinxbfcode{\sphinxupquote{Ray3D}}}{\sphinxparam{\DUrole{n,n}{data}\DUrole{p,p}{:}\DUrole{w,w}{  }\DUrole{n,n}{ndarray}}}{}
\pysigstopsignatures
\sphinxAtStartPar
Bases: {\hyperref[\detokenize{deepdrr.geo:deepdrr.geo.ray.Ray}]{\sphinxcrossref{\sphinxcode{\sphinxupquote{Ray}}}}}, {\hyperref[\detokenize{deepdrr.geo:deepdrr.geo.core.Joinable}]{\sphinxcrossref{\sphinxcode{\sphinxupquote{Joinable}}}}}, {\hyperref[\detokenize{deepdrr.geo:deepdrr.geo.core.HasProjection}]{\sphinxcrossref{\sphinxcode{\sphinxupquote{HasProjection}}}}}

\sphinxAtStartPar
A homogeneous representation of a ray.

\sphinxAtStartPar
This is just a (4,2) array with the homogeneous coordinates of the
origin and the direction, respectively.
\index{data (deepdrr.geo.ray.Ray3D attribute)@\spxentry{data}\spxextra{deepdrr.geo.ray.Ray3D attribute}}

\begin{fulllineitems}
\phantomsection\label{\detokenize{deepdrr.geo:deepdrr.geo.ray.Ray3D.data}}
\pysigstartsignatures
\pysigline{\sphinxbfcode{\sphinxupquote{data}}\sphinxbfcode{\sphinxupquote{\DUrole{p,p}{:}\DUrole{w,w}{  }ndarray}}}
\pysigstopsignatures
\end{fulllineitems}

\index{dim (deepdrr.geo.ray.Ray3D attribute)@\spxentry{dim}\spxextra{deepdrr.geo.ray.Ray3D attribute}}

\begin{fulllineitems}
\phantomsection\label{\detokenize{deepdrr.geo:deepdrr.geo.ray.Ray3D.dim}}
\pysigstartsignatures
\pysigline{\sphinxbfcode{\sphinxupquote{dim}}\sphinxbfcode{\sphinxupquote{\DUrole{w,w}{  }\DUrole{p,p}{=}\DUrole{w,w}{  }3}}}
\pysigstopsignatures
\end{fulllineitems}

\index{join() (deepdrr.geo.ray.Ray3D method)@\spxentry{join()}\spxextra{deepdrr.geo.ray.Ray3D method}}

\begin{fulllineitems}
\phantomsection\label{\detokenize{deepdrr.geo:deepdrr.geo.ray.Ray3D.join}}
\pysigstartsignatures
\pysiglinewithargsret{\sphinxbfcode{\sphinxupquote{join}}}{\sphinxparam{\DUrole{n,n}{other}\DUrole{p,p}{:}\DUrole{w,w}{  }\DUrole{n,n}{{\hyperref[\detokenize{deepdrr.geo:deepdrr.geo.core.Point3D}]{\sphinxcrossref{Point3D}}}}}}{{ $\rightarrow$ {\hyperref[\detokenize{deepdrr.geo:deepdrr.geo.hyperplane.Plane}]{\sphinxcrossref{Plane}}}}}
\pysigstopsignatures
\sphinxAtStartPar
Join two objects.

\sphinxAtStartPar
For example, given two points, get the line that connects them.
\begin{quote}\begin{description}
\sphinxlineitem{Parameters}
\sphinxAtStartPar
\sphinxstyleliteralstrong{\sphinxupquote{other}} ({\hyperref[\detokenize{deepdrr.geo:deepdrr.geo.core.Primitive}]{\sphinxcrossref{\sphinxstyleliteralemphasis{\sphinxupquote{Primitive}}}}}) \textendash{} the other primitive.

\sphinxlineitem{Returns}
\sphinxAtStartPar
the joined primitive.

\sphinxlineitem{Return type}
\sphinxAtStartPar
{\hyperref[\detokenize{deepdrr.geo:deepdrr.geo.core.Primitive}]{\sphinxcrossref{Primitive}}}

\end{description}\end{quote}

\end{fulllineitems}

\index{meet() (deepdrr.geo.ray.Ray3D method)@\spxentry{meet()}\spxextra{deepdrr.geo.ray.Ray3D method}}

\begin{fulllineitems}
\phantomsection\label{\detokenize{deepdrr.geo:deepdrr.geo.ray.Ray3D.meet}}
\pysigstartsignatures
\pysiglinewithargsret{\sphinxbfcode{\sphinxupquote{meet}}}{\sphinxparam{\DUrole{n,n}{other}\DUrole{p,p}{:}\DUrole{w,w}{  }\DUrole{n,n}{{\hyperref[\detokenize{deepdrr.geo:deepdrr.geo.hyperplane.Plane}]{\sphinxcrossref{Plane}}}}}}{{ $\rightarrow$ {\hyperref[\detokenize{deepdrr.geo:deepdrr.geo.core.Point3D}]{\sphinxcrossref{Point3D}}}}}
\pysigstopsignatures
\sphinxAtStartPar
Get the intersection of two objects.

\sphinxAtStartPar
For example, given two lines, get the line that is the intersection of them.
\begin{quote}\begin{description}
\sphinxlineitem{Parameters}
\sphinxAtStartPar
\sphinxstyleliteralstrong{\sphinxupquote{other}} ({\hyperref[\detokenize{deepdrr.geo:deepdrr.geo.core.Primitive}]{\sphinxcrossref{\sphinxstyleliteralemphasis{\sphinxupquote{Primitive}}}}}) \textendash{} the other primitive.

\sphinxlineitem{Returns}
\sphinxAtStartPar
the intersection of \sphinxtitleref{self} and \sphinxtitleref{other}.

\sphinxlineitem{Return type}
\sphinxAtStartPar
{\hyperref[\detokenize{deepdrr.geo:deepdrr.geo.core.Primitive}]{\sphinxcrossref{Primitive}}}

\sphinxlineitem{Raises}
\sphinxAtStartPar
{\hyperref[\detokenize{deepdrr.geo:deepdrr.geo.exceptions.MeetError}]{\sphinxcrossref{\sphinxstyleliteralstrong{\sphinxupquote{MeetError}}}}} \textendash{} if the objects cannot be intersected.

\end{description}\end{quote}

\end{fulllineitems}

\index{projection\_type() (deepdrr.geo.ray.Ray3D class method)@\spxentry{projection\_type()}\spxextra{deepdrr.geo.ray.Ray3D class method}}

\begin{fulllineitems}
\phantomsection\label{\detokenize{deepdrr.geo:deepdrr.geo.ray.Ray3D.projection_type}}
\pysigstartsignatures
\pysiglinewithargsret{\sphinxbfcode{\sphinxupquote{classmethod\DUrole{w,w}{  }}}\sphinxbfcode{\sphinxupquote{projection\_type}}}{}{{ $\rightarrow$ Type\DUrole{p,p}{{[}}{\hyperref[\detokenize{deepdrr.geo:deepdrr.geo.ray.Ray2D}]{\sphinxcrossref{Ray2D}}}\DUrole{p,p}{{]}}}}
\pysigstopsignatures
\sphinxAtStartPar
Get the type of the projection of the object.
\begin{quote}\begin{description}
\sphinxlineitem{Returns}
\sphinxAtStartPar
the type of the projection of the object.

\sphinxlineitem{Return type}
\sphinxAtStartPar
Type{[}{\hyperref[\detokenize{deepdrr.geo:deepdrr.geo.core.Primitive}]{\sphinxcrossref{Primitive}}}{]}

\end{description}\end{quote}

\end{fulllineitems}


\end{fulllineitems}

\index{ray() (in module deepdrr.geo.ray)@\spxentry{ray()}\spxextra{in module deepdrr.geo.ray}}

\begin{fulllineitems}
\phantomsection\label{\detokenize{deepdrr.geo:deepdrr.geo.ray.ray}}
\pysigstartsignatures
\pysiglinewithargsret{\sphinxcode{\sphinxupquote{deepdrr.geo.ray.}}\sphinxbfcode{\sphinxupquote{ray}}}{\sphinxparam{\DUrole{n,n}{r}\DUrole{p,p}{:}\DUrole{w,w}{  }\DUrole{n,n}{R}}}{{ $\rightarrow$ R}}
\pysiglinewithargsret{\sphinxcode{\sphinxupquote{deepdrr.geo.ray.}}\sphinxbfcode{\sphinxupquote{ray}}}{\sphinxparam{\DUrole{n,n}{l}\DUrole{p,p}{:}\DUrole{w,w}{  }\DUrole{n,n}{{\hyperref[\detokenize{deepdrr.geo:deepdrr.geo.hyperplane.Line2D}]{\sphinxcrossref{Line2D}}}}}}{{ $\rightarrow$ {\hyperref[\detokenize{deepdrr.geo:deepdrr.geo.ray.Ray2D}]{\sphinxcrossref{Ray2D}}}}}
\pysiglinewithargsret{\sphinxcode{\sphinxupquote{deepdrr.geo.ray.}}\sphinxbfcode{\sphinxupquote{ray}}}{\sphinxparam{\DUrole{n,n}{l}\DUrole{p,p}{:}\DUrole{w,w}{  }\DUrole{n,n}{{\hyperref[\detokenize{deepdrr.geo:deepdrr.geo.hyperplane.Line3D}]{\sphinxcrossref{Line3D}}}}}}{{ $\rightarrow$ {\hyperref[\detokenize{deepdrr.geo:deepdrr.geo.ray.Ray3D}]{\sphinxcrossref{Ray3D}}}}}
\pysiglinewithargsret{\sphinxcode{\sphinxupquote{deepdrr.geo.ray.}}\sphinxbfcode{\sphinxupquote{ray}}}{\sphinxparam{\DUrole{n,n}{p}\DUrole{p,p}{:}\DUrole{w,w}{  }\DUrole{n,n}{{\hyperref[\detokenize{deepdrr.geo:deepdrr.geo.core.Point2D}]{\sphinxcrossref{Point2D}}}}}\sphinxparamcomma \sphinxparam{\DUrole{n,n}{n}\DUrole{p,p}{:}\DUrole{w,w}{  }\DUrole{n,n}{{\hyperref[\detokenize{deepdrr.geo:deepdrr.geo.core.Vector2D}]{\sphinxcrossref{Vector2D}}}}}}{{ $\rightarrow$ {\hyperref[\detokenize{deepdrr.geo:deepdrr.geo.ray.Ray2D}]{\sphinxcrossref{Ray2D}}}}}
\pysiglinewithargsret{\sphinxcode{\sphinxupquote{deepdrr.geo.ray.}}\sphinxbfcode{\sphinxupquote{ray}}}{\sphinxparam{\DUrole{n,n}{p}\DUrole{p,p}{:}\DUrole{w,w}{  }\DUrole{n,n}{{\hyperref[\detokenize{deepdrr.geo:deepdrr.geo.core.Point3D}]{\sphinxcrossref{Point3D}}}}}\sphinxparamcomma \sphinxparam{\DUrole{n,n}{n}\DUrole{p,p}{:}\DUrole{w,w}{  }\DUrole{n,n}{{\hyperref[\detokenize{deepdrr.geo:deepdrr.geo.core.Vector3D}]{\sphinxcrossref{Vector3D}}}}}}{{ $\rightarrow$ {\hyperref[\detokenize{deepdrr.geo:deepdrr.geo.ray.Ray3D}]{\sphinxcrossref{Ray3D}}}}}
\pysiglinewithargsret{\sphinxcode{\sphinxupquote{deepdrr.geo.ray.}}\sphinxbfcode{\sphinxupquote{ray}}}{\sphinxparam{\DUrole{n,n}{p}\DUrole{p,p}{:}\DUrole{w,w}{  }\DUrole{n,n}{{\hyperref[\detokenize{deepdrr.geo:deepdrr.geo.core.Point3D}]{\sphinxcrossref{Point3D}}}}}\sphinxparamcomma \sphinxparam{\DUrole{n,n}{q}\DUrole{p,p}{:}\DUrole{w,w}{  }\DUrole{n,n}{{\hyperref[\detokenize{deepdrr.geo:deepdrr.geo.core.Point3D}]{\sphinxcrossref{Point3D}}}}}}{{ $\rightarrow$ {\hyperref[\detokenize{deepdrr.geo:deepdrr.geo.ray.Ray3D}]{\sphinxcrossref{Ray3D}}}}}
\pysiglinewithargsret{\sphinxcode{\sphinxupquote{deepdrr.geo.ray.}}\sphinxbfcode{\sphinxupquote{ray}}}{\sphinxparam{\DUrole{n,n}{a}\DUrole{p,p}{:}\DUrole{w,w}{  }\DUrole{n,n}{float}}\sphinxparamcomma \sphinxparam{\DUrole{n,n}{b}\DUrole{p,p}{:}\DUrole{w,w}{  }\DUrole{n,n}{float}}\sphinxparamcomma \sphinxparam{\DUrole{n,n}{c}\DUrole{p,p}{:}\DUrole{w,w}{  }\DUrole{n,n}{float}}\sphinxparamcomma \sphinxparam{\DUrole{n,n}{d}\DUrole{p,p}{:}\DUrole{w,w}{  }\DUrole{n,n}{float}}}{{ $\rightarrow$ {\hyperref[\detokenize{deepdrr.geo:deepdrr.geo.ray.Ray2D}]{\sphinxcrossref{Ray2D}}}}}
\pysiglinewithargsret{\sphinxcode{\sphinxupquote{deepdrr.geo.ray.}}\sphinxbfcode{\sphinxupquote{ray}}}{\sphinxparam{\DUrole{n,n}{a}\DUrole{p,p}{:}\DUrole{w,w}{  }\DUrole{n,n}{float}}\sphinxparamcomma \sphinxparam{\DUrole{n,n}{b}\DUrole{p,p}{:}\DUrole{w,w}{  }\DUrole{n,n}{float}}\sphinxparamcomma \sphinxparam{\DUrole{n,n}{c}\DUrole{p,p}{:}\DUrole{w,w}{  }\DUrole{n,n}{float}}\sphinxparamcomma \sphinxparam{\DUrole{n,n}{d}\DUrole{p,p}{:}\DUrole{w,w}{  }\DUrole{n,n}{float}}\sphinxparamcomma \sphinxparam{\DUrole{n,n}{e}\DUrole{p,p}{:}\DUrole{w,w}{  }\DUrole{n,n}{float}}\sphinxparamcomma \sphinxparam{\DUrole{n,n}{f}\DUrole{p,p}{:}\DUrole{w,w}{  }\DUrole{n,n}{float}}}{{ $\rightarrow$ {\hyperref[\detokenize{deepdrr.geo:deepdrr.geo.ray.Ray3D}]{\sphinxcrossref{Ray3D}}}}}
\pysiglinewithargsret{\sphinxcode{\sphinxupquote{deepdrr.geo.ray.}}\sphinxbfcode{\sphinxupquote{ray}}}{\sphinxparam{\DUrole{n,n}{x}\DUrole{p,p}{:}\DUrole{w,w}{  }\DUrole{n,n}{ndarray}}}{{ $\rightarrow$ {\hyperref[\detokenize{deepdrr.geo:deepdrr.geo.ray.Ray}]{\sphinxcrossref{Ray}}}}}
\pysigstopsignatures
\sphinxAtStartPar
More flexible method for creating a ray.

\end{fulllineitems}



\subsection{deepdrr.geo.segment}
\label{\detokenize{deepdrr.geo:module-deepdrr.geo.segment}}\label{\detokenize{deepdrr.geo:deepdrr-geo-segment}}\index{module@\spxentry{module}!deepdrr.geo.segment@\spxentry{deepdrr.geo.segment}}\index{deepdrr.geo.segment@\spxentry{deepdrr.geo.segment}!module@\spxentry{module}}\index{Segment (class in deepdrr.geo.segment)@\spxentry{Segment}\spxextra{class in deepdrr.geo.segment}}

\begin{fulllineitems}
\phantomsection\label{\detokenize{deepdrr.geo:deepdrr.geo.segment.Segment}}
\pysigstartsignatures
\pysiglinewithargsret{\sphinxbfcode{\sphinxupquote{class\DUrole{w,w}{  }}}\sphinxcode{\sphinxupquote{deepdrr.geo.segment.}}\sphinxbfcode{\sphinxupquote{Segment}}}{\sphinxparam{\DUrole{n,n}{data}\DUrole{p,p}{:}\DUrole{w,w}{  }\DUrole{n,n}{ndarray}}}{}
\pysigstopsignatures
\sphinxAtStartPar
Bases: {\hyperref[\detokenize{deepdrr.geo:deepdrr.geo.core.HasLocationAndDirection}]{\sphinxcrossref{\sphinxcode{\sphinxupquote{HasLocationAndDirection}}}}}, {\hyperref[\detokenize{deepdrr.geo:deepdrr.geo.core.Meetable}]{\sphinxcrossref{\sphinxcode{\sphinxupquote{Meetable}}}}}
\index{data (deepdrr.geo.segment.Segment attribute)@\spxentry{data}\spxextra{deepdrr.geo.segment.Segment attribute}}

\begin{fulllineitems}
\phantomsection\label{\detokenize{deepdrr.geo:deepdrr.geo.segment.Segment.data}}
\pysigstartsignatures
\pysigline{\sphinxbfcode{\sphinxupquote{data}}\sphinxbfcode{\sphinxupquote{\DUrole{p,p}{:}\DUrole{w,w}{  }ndarray}}}
\pysigstopsignatures
\end{fulllineitems}

\index{from\_pn() (deepdrr.geo.segment.Segment class method)@\spxentry{from\_pn()}\spxextra{deepdrr.geo.segment.Segment class method}}

\begin{fulllineitems}
\phantomsection\label{\detokenize{deepdrr.geo:deepdrr.geo.segment.Segment.from_pn}}
\pysigstartsignatures
\pysiglinewithargsret{\sphinxbfcode{\sphinxupquote{classmethod\DUrole{w,w}{  }}}\sphinxbfcode{\sphinxupquote{from\_pn}}}{\sphinxparam{\DUrole{n,n}{p}\DUrole{p,p}{:}\DUrole{w,w}{  }\DUrole{n,n}{{\hyperref[\detokenize{deepdrr.geo:deepdrr.geo.core.Point}]{\sphinxcrossref{Point}}}}}\sphinxparamcomma \sphinxparam{\DUrole{n,n}{n}\DUrole{p,p}{:}\DUrole{w,w}{  }\DUrole{n,n}{{\hyperref[\detokenize{deepdrr.geo:deepdrr.geo.core.Vector}]{\sphinxcrossref{Vector}}}}}}{{ $\rightarrow$ S}}
\pysigstopsignatures
\sphinxAtStartPar
Initialize the segment with a poind and a direction.
\begin{quote}\begin{description}
\sphinxlineitem{Parameters}\begin{itemize}
\item {} 
\sphinxAtStartPar
\sphinxstyleliteralstrong{\sphinxupquote{p}} ({\hyperref[\detokenize{deepdrr.geo:deepdrr.geo.core.Point}]{\sphinxcrossref{\sphinxstyleliteralemphasis{\sphinxupquote{Point}}}}}) \textendash{} The first point.

\item {} 
\sphinxAtStartPar
\sphinxstyleliteralstrong{\sphinxupquote{n}} ({\hyperref[\detokenize{deepdrr.geo:deepdrr.geo.core.Vector}]{\sphinxcrossref{\sphinxstyleliteralemphasis{\sphinxupquote{Vector}}}}}) \textendash{} The direction vector.

\end{itemize}

\sphinxlineitem{Returns}
\sphinxAtStartPar
The segment.

\sphinxlineitem{Return type}
\sphinxAtStartPar
{\hyperref[\detokenize{deepdrr.geo:deepdrr.geo.segment.Segment}]{\sphinxcrossref{Segment}}}

\end{description}\end{quote}

\end{fulllineitems}

\index{from\_point\_direction() (deepdrr.geo.segment.Segment class method)@\spxentry{from\_point\_direction()}\spxextra{deepdrr.geo.segment.Segment class method}}

\begin{fulllineitems}
\phantomsection\label{\detokenize{deepdrr.geo:deepdrr.geo.segment.Segment.from_point_direction}}
\pysigstartsignatures
\pysiglinewithargsret{\sphinxbfcode{\sphinxupquote{classmethod\DUrole{w,w}{  }}}\sphinxbfcode{\sphinxupquote{from\_point\_direction}}}{\sphinxparam{\DUrole{n,n}{p}\DUrole{p,p}{:}\DUrole{w,w}{  }\DUrole{n,n}{{\hyperref[\detokenize{deepdrr.geo:deepdrr.geo.core.Point}]{\sphinxcrossref{Point}}}}}\sphinxparamcomma \sphinxparam{\DUrole{n,n}{n}\DUrole{p,p}{:}\DUrole{w,w}{  }\DUrole{n,n}{{\hyperref[\detokenize{deepdrr.geo:deepdrr.geo.core.Vector}]{\sphinxcrossref{Vector}}}}}}{{ $\rightarrow$ S}}
\pysigstopsignatures
\sphinxAtStartPar
Initialize the segment with a poind and a direction.
\begin{quote}\begin{description}
\sphinxlineitem{Parameters}\begin{itemize}
\item {} 
\sphinxAtStartPar
\sphinxstyleliteralstrong{\sphinxupquote{p}} ({\hyperref[\detokenize{deepdrr.geo:deepdrr.geo.core.Point}]{\sphinxcrossref{\sphinxstyleliteralemphasis{\sphinxupquote{Point}}}}}) \textendash{} The first point.

\item {} 
\sphinxAtStartPar
\sphinxstyleliteralstrong{\sphinxupquote{n}} ({\hyperref[\detokenize{deepdrr.geo:deepdrr.geo.core.Vector}]{\sphinxcrossref{\sphinxstyleliteralemphasis{\sphinxupquote{Vector}}}}}) \textendash{} The direction vector.

\end{itemize}

\sphinxlineitem{Returns}
\sphinxAtStartPar
The segment.

\sphinxlineitem{Return type}
\sphinxAtStartPar
{\hyperref[\detokenize{deepdrr.geo:deepdrr.geo.segment.Segment}]{\sphinxcrossref{Segment}}}

\end{description}\end{quote}

\end{fulllineitems}

\index{from\_pq() (deepdrr.geo.segment.Segment class method)@\spxentry{from\_pq()}\spxextra{deepdrr.geo.segment.Segment class method}}

\begin{fulllineitems}
\phantomsection\label{\detokenize{deepdrr.geo:deepdrr.geo.segment.Segment.from_pq}}
\pysigstartsignatures
\pysiglinewithargsret{\sphinxbfcode{\sphinxupquote{classmethod\DUrole{w,w}{  }}}\sphinxbfcode{\sphinxupquote{from\_pq}}}{\sphinxparam{\DUrole{n,n}{p}\DUrole{p,p}{:}\DUrole{w,w}{  }\DUrole{n,n}{{\hyperref[\detokenize{deepdrr.geo:deepdrr.geo.core.Point}]{\sphinxcrossref{Point}}}}}\sphinxparamcomma \sphinxparam{\DUrole{n,n}{q}\DUrole{p,p}{:}\DUrole{w,w}{  }\DUrole{n,n}{{\hyperref[\detokenize{deepdrr.geo:deepdrr.geo.core.Point}]{\sphinxcrossref{Point}}}}}}{{ $\rightarrow$ S}}
\pysigstopsignatures
\sphinxAtStartPar
Initialize the segment containing two points.
\begin{quote}\begin{description}
\sphinxlineitem{Parameters}\begin{itemize}
\item {} 
\sphinxAtStartPar
\sphinxstyleliteralstrong{\sphinxupquote{p}} ({\hyperref[\detokenize{deepdrr.geo:deepdrr.geo.core.Point}]{\sphinxcrossref{\sphinxstyleliteralemphasis{\sphinxupquote{Point}}}}}) \textendash{} The first point.

\item {} 
\sphinxAtStartPar
\sphinxstyleliteralstrong{\sphinxupquote{q}} ({\hyperref[\detokenize{deepdrr.geo:deepdrr.geo.core.Point}]{\sphinxcrossref{\sphinxstyleliteralemphasis{\sphinxupquote{Point}}}}}) \textendash{} The second point.

\end{itemize}

\sphinxlineitem{Returns}
\sphinxAtStartPar
The segment.

\sphinxlineitem{Return type}
\sphinxAtStartPar
{\hyperref[\detokenize{deepdrr.geo:deepdrr.geo.segment.Segment}]{\sphinxcrossref{Segment}}}

\end{description}\end{quote}

\end{fulllineitems}

\index{get\_direction() (deepdrr.geo.segment.Segment method)@\spxentry{get\_direction()}\spxextra{deepdrr.geo.segment.Segment method}}

\begin{fulllineitems}
\phantomsection\label{\detokenize{deepdrr.geo:deepdrr.geo.segment.Segment.get_direction}}
\pysigstartsignatures
\pysiglinewithargsret{\sphinxbfcode{\sphinxupquote{get\_direction}}}{}{{ $\rightarrow$ {\hyperref[\detokenize{deepdrr.geo:deepdrr.geo.core.Vector}]{\sphinxcrossref{Vector}}}}}
\pysigstopsignatures
\sphinxAtStartPar
Get the direction associated with the object.
\begin{quote}\begin{description}
\sphinxlineitem{Returns}
\sphinxAtStartPar
the direction of the object.

\sphinxlineitem{Return type}
\sphinxAtStartPar
{\hyperref[\detokenize{deepdrr.geo:deepdrr.geo.core.Vector}]{\sphinxcrossref{Vector}}}

\end{description}\end{quote}

\end{fulllineitems}

\index{get\_point() (deepdrr.geo.segment.Segment method)@\spxentry{get\_point()}\spxextra{deepdrr.geo.segment.Segment method}}

\begin{fulllineitems}
\phantomsection\label{\detokenize{deepdrr.geo:deepdrr.geo.segment.Segment.get_point}}
\pysigstartsignatures
\pysiglinewithargsret{\sphinxbfcode{\sphinxupquote{get\_point}}}{}{{ $\rightarrow$ {\hyperref[\detokenize{deepdrr.geo:deepdrr.geo.core.Point}]{\sphinxcrossref{Point}}}}}
\pysigstopsignatures
\sphinxAtStartPar
Get the location of the object.
\begin{quote}\begin{description}
\sphinxlineitem{Returns}
\sphinxAtStartPar
the location of the object.

\sphinxlineitem{Return type}
\sphinxAtStartPar
{\hyperref[\detokenize{deepdrr.geo:deepdrr.geo.core.Point}]{\sphinxcrossref{Point}}}

\end{description}\end{quote}

\end{fulllineitems}

\index{length() (deepdrr.geo.segment.Segment method)@\spxentry{length()}\spxextra{deepdrr.geo.segment.Segment method}}

\begin{fulllineitems}
\phantomsection\label{\detokenize{deepdrr.geo:deepdrr.geo.segment.Segment.length}}
\pysigstartsignatures
\pysiglinewithargsret{\sphinxbfcode{\sphinxupquote{length}}}{}{{ $\rightarrow$ float}}
\pysigstopsignatures
\sphinxAtStartPar
Get the length of the segment.
\begin{quote}\begin{description}
\sphinxlineitem{Returns}
\sphinxAtStartPar
The length of the segment.

\sphinxlineitem{Return type}
\sphinxAtStartPar
float

\end{description}\end{quote}

\end{fulllineitems}

\index{line() (deepdrr.geo.segment.Segment method)@\spxentry{line()}\spxextra{deepdrr.geo.segment.Segment method}}

\begin{fulllineitems}
\phantomsection\label{\detokenize{deepdrr.geo:deepdrr.geo.segment.Segment.line}}
\pysigstartsignatures
\pysiglinewithargsret{\sphinxbfcode{\sphinxupquote{line}}}{}{{ $\rightarrow$ {\hyperref[\detokenize{deepdrr.geo:deepdrr.geo.hyperplane.Line2D}]{\sphinxcrossref{Line2D}}}}}
\pysiglinewithargsret{\sphinxbfcode{\sphinxupquote{line}}}{}{{ $\rightarrow$ {\hyperref[\detokenize{deepdrr.geo:deepdrr.geo.hyperplane.Line3D}]{\sphinxcrossref{Line3D}}}}}
\pysigstopsignatures
\end{fulllineitems}

\index{midpoint() (deepdrr.geo.segment.Segment method)@\spxentry{midpoint()}\spxextra{deepdrr.geo.segment.Segment method}}

\begin{fulllineitems}
\phantomsection\label{\detokenize{deepdrr.geo:deepdrr.geo.segment.Segment.midpoint}}
\pysigstartsignatures
\pysiglinewithargsret{\sphinxbfcode{\sphinxupquote{midpoint}}}{}{{ $\rightarrow$ {\hyperref[\detokenize{deepdrr.geo:deepdrr.geo.core.Point}]{\sphinxcrossref{Point}}}}}
\pysigstopsignatures
\end{fulllineitems}

\index{p (deepdrr.geo.segment.Segment property)@\spxentry{p}\spxextra{deepdrr.geo.segment.Segment property}}

\begin{fulllineitems}
\phantomsection\label{\detokenize{deepdrr.geo:deepdrr.geo.segment.Segment.p}}
\pysigstartsignatures
\pysigline{\sphinxbfcode{\sphinxupquote{property\DUrole{w,w}{  }}}\sphinxbfcode{\sphinxupquote{p}}\sphinxbfcode{\sphinxupquote{\DUrole{p,p}{:}\DUrole{w,w}{  }{\hyperref[\detokenize{deepdrr.geo:deepdrr.geo.core.Point}]{\sphinxcrossref{Point}}}}}}
\pysigstopsignatures
\sphinxAtStartPar
Get the first point of the segment.
\begin{quote}\begin{description}
\sphinxlineitem{Returns}
\sphinxAtStartPar
The first point of the segment.

\sphinxlineitem{Return type}
\sphinxAtStartPar
{\hyperref[\detokenize{deepdrr.geo:deepdrr.geo.core.Point2D}]{\sphinxcrossref{Point2D}}}

\end{description}\end{quote}

\end{fulllineitems}

\index{q (deepdrr.geo.segment.Segment property)@\spxentry{q}\spxextra{deepdrr.geo.segment.Segment property}}

\begin{fulllineitems}
\phantomsection\label{\detokenize{deepdrr.geo:deepdrr.geo.segment.Segment.q}}
\pysigstartsignatures
\pysigline{\sphinxbfcode{\sphinxupquote{property\DUrole{w,w}{  }}}\sphinxbfcode{\sphinxupquote{q}}\sphinxbfcode{\sphinxupquote{\DUrole{p,p}{:}\DUrole{w,w}{  }{\hyperref[\detokenize{deepdrr.geo:deepdrr.geo.core.Point}]{\sphinxcrossref{Point}}}}}}
\pysigstopsignatures
\sphinxAtStartPar
Get the second point of the segment.
\begin{quote}\begin{description}
\sphinxlineitem{Returns}
\sphinxAtStartPar
The second point of the segment.

\sphinxlineitem{Return type}
\sphinxAtStartPar
{\hyperref[\detokenize{deepdrr.geo:deepdrr.geo.core.Point2D}]{\sphinxcrossref{Point2D}}}

\end{description}\end{quote}

\end{fulllineitems}


\end{fulllineitems}

\index{Segment2D (class in deepdrr.geo.segment)@\spxentry{Segment2D}\spxextra{class in deepdrr.geo.segment}}

\begin{fulllineitems}
\phantomsection\label{\detokenize{deepdrr.geo:deepdrr.geo.segment.Segment2D}}
\pysigstartsignatures
\pysiglinewithargsret{\sphinxbfcode{\sphinxupquote{class\DUrole{w,w}{  }}}\sphinxcode{\sphinxupquote{deepdrr.geo.segment.}}\sphinxbfcode{\sphinxupquote{Segment2D}}}{\sphinxparam{\DUrole{n,n}{data}\DUrole{p,p}{:}\DUrole{w,w}{  }\DUrole{n,n}{ndarray}}}{}
\pysigstopsignatures
\sphinxAtStartPar
Bases: {\hyperref[\detokenize{deepdrr.geo:deepdrr.geo.segment.Segment}]{\sphinxcrossref{\sphinxcode{\sphinxupquote{Segment}}}}}

\sphinxAtStartPar
Represents a line segment in 2D.
\index{data (deepdrr.geo.segment.Segment2D attribute)@\spxentry{data}\spxextra{deepdrr.geo.segment.Segment2D attribute}}

\begin{fulllineitems}
\phantomsection\label{\detokenize{deepdrr.geo:deepdrr.geo.segment.Segment2D.data}}
\pysigstartsignatures
\pysigline{\sphinxbfcode{\sphinxupquote{data}}\sphinxbfcode{\sphinxupquote{\DUrole{p,p}{:}\DUrole{w,w}{  }ndarray}}}
\pysigstopsignatures
\end{fulllineitems}

\index{dim (deepdrr.geo.segment.Segment2D attribute)@\spxentry{dim}\spxextra{deepdrr.geo.segment.Segment2D attribute}}

\begin{fulllineitems}
\phantomsection\label{\detokenize{deepdrr.geo:deepdrr.geo.segment.Segment2D.dim}}
\pysigstartsignatures
\pysigline{\sphinxbfcode{\sphinxupquote{dim}}\sphinxbfcode{\sphinxupquote{\DUrole{w,w}{  }\DUrole{p,p}{=}\DUrole{w,w}{  }2}}}
\pysigstopsignatures
\end{fulllineitems}

\index{meet() (deepdrr.geo.segment.Segment2D method)@\spxentry{meet()}\spxextra{deepdrr.geo.segment.Segment2D method}}

\begin{fulllineitems}
\phantomsection\label{\detokenize{deepdrr.geo:deepdrr.geo.segment.Segment2D.meet}}
\pysigstartsignatures
\pysiglinewithargsret{\sphinxbfcode{\sphinxupquote{meet}}}{\sphinxparam{\DUrole{n,n}{other}\DUrole{p,p}{:}\DUrole{w,w}{  }\DUrole{n,n}{{\hyperref[\detokenize{deepdrr.geo:deepdrr.geo.hyperplane.Line2D}]{\sphinxcrossref{Line2D}}}\DUrole{w,w}{  }\DUrole{p,p}{|}\DUrole{w,w}{  }{\hyperref[\detokenize{deepdrr.geo:deepdrr.geo.segment.Segment2D}]{\sphinxcrossref{Segment2D}}}}}}{{ $\rightarrow$ {\hyperref[\detokenize{deepdrr.geo:deepdrr.geo.core.Point2D}]{\sphinxcrossref{Point2D}}}}}
\pysigstopsignatures
\sphinxAtStartPar
Get the point of intersection between this segment and another line.
\begin{quote}\begin{description}
\sphinxlineitem{Parameters}
\sphinxAtStartPar
\sphinxstyleliteralstrong{\sphinxupquote{other}} ({\hyperref[\detokenize{deepdrr.geo:deepdrr.geo.hyperplane.Line2D}]{\sphinxcrossref{\sphinxstyleliteralemphasis{\sphinxupquote{Line2D}}}}}) \textendash{} The other line.

\sphinxlineitem{Returns}
\sphinxAtStartPar
The point of intersection.

\sphinxlineitem{Return type}
\sphinxAtStartPar
{\hyperref[\detokenize{deepdrr.geo:deepdrr.geo.core.Point2D}]{\sphinxcrossref{Point2D}}}

\end{description}\end{quote}

\end{fulllineitems}


\end{fulllineitems}

\index{Segment3D (class in deepdrr.geo.segment)@\spxentry{Segment3D}\spxextra{class in deepdrr.geo.segment}}

\begin{fulllineitems}
\phantomsection\label{\detokenize{deepdrr.geo:deepdrr.geo.segment.Segment3D}}
\pysigstartsignatures
\pysiglinewithargsret{\sphinxbfcode{\sphinxupquote{class\DUrole{w,w}{  }}}\sphinxcode{\sphinxupquote{deepdrr.geo.segment.}}\sphinxbfcode{\sphinxupquote{Segment3D}}}{\sphinxparam{\DUrole{n,n}{data}\DUrole{p,p}{:}\DUrole{w,w}{  }\DUrole{n,n}{ndarray}}}{}
\pysigstopsignatures
\sphinxAtStartPar
Bases: {\hyperref[\detokenize{deepdrr.geo:deepdrr.geo.segment.Segment}]{\sphinxcrossref{\sphinxcode{\sphinxupquote{Segment}}}}}, {\hyperref[\detokenize{deepdrr.geo:deepdrr.geo.core.Joinable}]{\sphinxcrossref{\sphinxcode{\sphinxupquote{Joinable}}}}}, {\hyperref[\detokenize{deepdrr.geo:deepdrr.geo.core.HasProjection}]{\sphinxcrossref{\sphinxcode{\sphinxupquote{HasProjection}}}}}

\sphinxAtStartPar
Represents a segment in 3D.
\index{data (deepdrr.geo.segment.Segment3D attribute)@\spxentry{data}\spxextra{deepdrr.geo.segment.Segment3D attribute}}

\begin{fulllineitems}
\phantomsection\label{\detokenize{deepdrr.geo:deepdrr.geo.segment.Segment3D.data}}
\pysigstartsignatures
\pysigline{\sphinxbfcode{\sphinxupquote{data}}\sphinxbfcode{\sphinxupquote{\DUrole{p,p}{:}\DUrole{w,w}{  }ndarray}}}
\pysigstopsignatures
\end{fulllineitems}

\index{dim (deepdrr.geo.segment.Segment3D attribute)@\spxentry{dim}\spxextra{deepdrr.geo.segment.Segment3D attribute}}

\begin{fulllineitems}
\phantomsection\label{\detokenize{deepdrr.geo:deepdrr.geo.segment.Segment3D.dim}}
\pysigstartsignatures
\pysigline{\sphinxbfcode{\sphinxupquote{dim}}\sphinxbfcode{\sphinxupquote{\DUrole{w,w}{  }\DUrole{p,p}{=}\DUrole{w,w}{  }3}}}
\pysigstopsignatures
\end{fulllineitems}

\index{join() (deepdrr.geo.segment.Segment3D method)@\spxentry{join()}\spxextra{deepdrr.geo.segment.Segment3D method}}

\begin{fulllineitems}
\phantomsection\label{\detokenize{deepdrr.geo:deepdrr.geo.segment.Segment3D.join}}
\pysigstartsignatures
\pysiglinewithargsret{\sphinxbfcode{\sphinxupquote{join}}}{\sphinxparam{\DUrole{n,n}{other}\DUrole{p,p}{:}\DUrole{w,w}{  }\DUrole{n,n}{{\hyperref[\detokenize{deepdrr.geo:deepdrr.geo.core.Point3D}]{\sphinxcrossref{Point3D}}}}}}{{ $\rightarrow$ {\hyperref[\detokenize{deepdrr.geo:deepdrr.geo.hyperplane.Plane}]{\sphinxcrossref{Plane}}}}}
\pysigstopsignatures
\sphinxAtStartPar
Join two objects.

\sphinxAtStartPar
For example, given two points, get the line that connects them.
\begin{quote}\begin{description}
\sphinxlineitem{Parameters}
\sphinxAtStartPar
\sphinxstyleliteralstrong{\sphinxupquote{other}} ({\hyperref[\detokenize{deepdrr.geo:deepdrr.geo.core.Primitive}]{\sphinxcrossref{\sphinxstyleliteralemphasis{\sphinxupquote{Primitive}}}}}) \textendash{} the other primitive.

\sphinxlineitem{Returns}
\sphinxAtStartPar
the joined primitive.

\sphinxlineitem{Return type}
\sphinxAtStartPar
{\hyperref[\detokenize{deepdrr.geo:deepdrr.geo.core.Primitive}]{\sphinxcrossref{Primitive}}}

\end{description}\end{quote}

\end{fulllineitems}

\index{meet() (deepdrr.geo.segment.Segment3D method)@\spxentry{meet()}\spxextra{deepdrr.geo.segment.Segment3D method}}

\begin{fulllineitems}
\phantomsection\label{\detokenize{deepdrr.geo:deepdrr.geo.segment.Segment3D.meet}}
\pysigstartsignatures
\pysiglinewithargsret{\sphinxbfcode{\sphinxupquote{meet}}}{\sphinxparam{\DUrole{n,n}{other}\DUrole{p,p}{:}\DUrole{w,w}{  }\DUrole{n,n}{{\hyperref[\detokenize{deepdrr.geo:deepdrr.geo.hyperplane.Plane}]{\sphinxcrossref{Plane}}}}}}{{ $\rightarrow$ {\hyperref[\detokenize{deepdrr.geo:deepdrr.geo.core.Point3D}]{\sphinxcrossref{Point3D}}}}}
\pysigstopsignatures
\sphinxAtStartPar
Get the point of intersection between this segment and a plane.

\sphinxAtStartPar
TODO: check if the intersection is on the segment.

\end{fulllineitems}

\index{projection\_type() (deepdrr.geo.segment.Segment3D class method)@\spxentry{projection\_type()}\spxextra{deepdrr.geo.segment.Segment3D class method}}

\begin{fulllineitems}
\phantomsection\label{\detokenize{deepdrr.geo:deepdrr.geo.segment.Segment3D.projection_type}}
\pysigstartsignatures
\pysiglinewithargsret{\sphinxbfcode{\sphinxupquote{classmethod\DUrole{w,w}{  }}}\sphinxbfcode{\sphinxupquote{projection\_type}}}{}{{ $\rightarrow$ Type\DUrole{p,p}{{[}}{\hyperref[\detokenize{deepdrr.geo:deepdrr.geo.segment.Segment2D}]{\sphinxcrossref{Segment2D}}}\DUrole{p,p}{{]}}}}
\pysigstopsignatures
\sphinxAtStartPar
Get the type of the projection of the object.
\begin{quote}\begin{description}
\sphinxlineitem{Returns}
\sphinxAtStartPar
the type of the projection of the object.

\sphinxlineitem{Return type}
\sphinxAtStartPar
Type{[}{\hyperref[\detokenize{deepdrr.geo:deepdrr.geo.core.Primitive}]{\sphinxcrossref{Primitive}}}{]}

\end{description}\end{quote}

\end{fulllineitems}


\end{fulllineitems}

\index{segment() (in module deepdrr.geo.segment)@\spxentry{segment()}\spxextra{in module deepdrr.geo.segment}}

\begin{fulllineitems}
\phantomsection\label{\detokenize{deepdrr.geo:deepdrr.geo.segment.segment}}
\pysigstartsignatures
\pysiglinewithargsret{\sphinxcode{\sphinxupquote{deepdrr.geo.segment.}}\sphinxbfcode{\sphinxupquote{segment}}}{\sphinxparam{\DUrole{n,n}{s}\DUrole{p,p}{:}\DUrole{w,w}{  }\DUrole{n,n}{S}}}{{ $\rightarrow$ S}}
\pysiglinewithargsret{\sphinxcode{\sphinxupquote{deepdrr.geo.segment.}}\sphinxbfcode{\sphinxupquote{segment}}}{\sphinxparam{\DUrole{n,n}{p}\DUrole{p,p}{:}\DUrole{w,w}{  }\DUrole{n,n}{{\hyperref[\detokenize{deepdrr.geo:deepdrr.geo.core.Point2D}]{\sphinxcrossref{Point2D}}}}}\sphinxparamcomma \sphinxparam{\DUrole{n,n}{q}\DUrole{p,p}{:}\DUrole{w,w}{  }\DUrole{n,n}{{\hyperref[\detokenize{deepdrr.geo:deepdrr.geo.core.Point2D}]{\sphinxcrossref{Point2D}}}}}}{{ $\rightarrow$ {\hyperref[\detokenize{deepdrr.geo:deepdrr.geo.segment.Segment2D}]{\sphinxcrossref{Segment2D}}}}}
\pysiglinewithargsret{\sphinxcode{\sphinxupquote{deepdrr.geo.segment.}}\sphinxbfcode{\sphinxupquote{segment}}}{\sphinxparam{\DUrole{n,n}{p}\DUrole{p,p}{:}\DUrole{w,w}{  }\DUrole{n,n}{{\hyperref[\detokenize{deepdrr.geo:deepdrr.geo.core.Point3D}]{\sphinxcrossref{Point3D}}}}}\sphinxparamcomma \sphinxparam{\DUrole{n,n}{q}\DUrole{p,p}{:}\DUrole{w,w}{  }\DUrole{n,n}{{\hyperref[\detokenize{deepdrr.geo:deepdrr.geo.core.Point3D}]{\sphinxcrossref{Point3D}}}}}}{{ $\rightarrow$ {\hyperref[\detokenize{deepdrr.geo:deepdrr.geo.segment.Segment3D}]{\sphinxcrossref{Segment3D}}}}}
\pysiglinewithargsret{\sphinxcode{\sphinxupquote{deepdrr.geo.segment.}}\sphinxbfcode{\sphinxupquote{segment}}}{\sphinxparam{\DUrole{n,n}{p}\DUrole{p,p}{:}\DUrole{w,w}{  }\DUrole{n,n}{{\hyperref[\detokenize{deepdrr.geo:deepdrr.geo.core.Point2D}]{\sphinxcrossref{Point2D}}}}}\sphinxparamcomma \sphinxparam{\DUrole{n,n}{n}\DUrole{p,p}{:}\DUrole{w,w}{  }\DUrole{n,n}{{\hyperref[\detokenize{deepdrr.geo:deepdrr.geo.core.Vector2D}]{\sphinxcrossref{Vector2D}}}}}}{{ $\rightarrow$ {\hyperref[\detokenize{deepdrr.geo:deepdrr.geo.segment.Segment2D}]{\sphinxcrossref{Segment2D}}}}}
\pysiglinewithargsret{\sphinxcode{\sphinxupquote{deepdrr.geo.segment.}}\sphinxbfcode{\sphinxupquote{segment}}}{\sphinxparam{\DUrole{n,n}{p}\DUrole{p,p}{:}\DUrole{w,w}{  }\DUrole{n,n}{{\hyperref[\detokenize{deepdrr.geo:deepdrr.geo.core.Point3D}]{\sphinxcrossref{Point3D}}}}}\sphinxparamcomma \sphinxparam{\DUrole{n,n}{n}\DUrole{p,p}{:}\DUrole{w,w}{  }\DUrole{n,n}{{\hyperref[\detokenize{deepdrr.geo:deepdrr.geo.core.Vector3D}]{\sphinxcrossref{Vector3D}}}}}}{{ $\rightarrow$ {\hyperref[\detokenize{deepdrr.geo:deepdrr.geo.segment.Segment3D}]{\sphinxcrossref{Segment3D}}}}}
\pysiglinewithargsret{\sphinxcode{\sphinxupquote{deepdrr.geo.segment.}}\sphinxbfcode{\sphinxupquote{segment}}}{\sphinxparam{\DUrole{n,n}{a}\DUrole{p,p}{:}\DUrole{w,w}{  }\DUrole{n,n}{float}}\sphinxparamcomma \sphinxparam{\DUrole{n,n}{b}\DUrole{p,p}{:}\DUrole{w,w}{  }\DUrole{n,n}{float}}\sphinxparamcomma \sphinxparam{\DUrole{n,n}{c}\DUrole{p,p}{:}\DUrole{w,w}{  }\DUrole{n,n}{float}}\sphinxparamcomma \sphinxparam{\DUrole{n,n}{d}\DUrole{p,p}{:}\DUrole{w,w}{  }\DUrole{n,n}{float}}}{{ $\rightarrow$ {\hyperref[\detokenize{deepdrr.geo:deepdrr.geo.segment.Segment2D}]{\sphinxcrossref{Segment2D}}}}}
\pysiglinewithargsret{\sphinxcode{\sphinxupquote{deepdrr.geo.segment.}}\sphinxbfcode{\sphinxupquote{segment}}}{\sphinxparam{\DUrole{n,n}{a}\DUrole{p,p}{:}\DUrole{w,w}{  }\DUrole{n,n}{float}}\sphinxparamcomma \sphinxparam{\DUrole{n,n}{b}\DUrole{p,p}{:}\DUrole{w,w}{  }\DUrole{n,n}{float}}\sphinxparamcomma \sphinxparam{\DUrole{n,n}{c}\DUrole{p,p}{:}\DUrole{w,w}{  }\DUrole{n,n}{float}}\sphinxparamcomma \sphinxparam{\DUrole{n,n}{d}\DUrole{p,p}{:}\DUrole{w,w}{  }\DUrole{n,n}{float}}\sphinxparamcomma \sphinxparam{\DUrole{n,n}{e}\DUrole{p,p}{:}\DUrole{w,w}{  }\DUrole{n,n}{float}}\sphinxparamcomma \sphinxparam{\DUrole{n,n}{f}\DUrole{p,p}{:}\DUrole{w,w}{  }\DUrole{n,n}{float}}}{{ $\rightarrow$ {\hyperref[\detokenize{deepdrr.geo:deepdrr.geo.segment.Segment3D}]{\sphinxcrossref{Segment3D}}}}}
\pysiglinewithargsret{\sphinxcode{\sphinxupquote{deepdrr.geo.segment.}}\sphinxbfcode{\sphinxupquote{segment}}}{\sphinxparam{\DUrole{n,n}{x}\DUrole{p,p}{:}\DUrole{w,w}{  }\DUrole{n,n}{ndarray}}}{{ $\rightarrow$ {\hyperref[\detokenize{deepdrr.geo:deepdrr.geo.segment.Segment}]{\sphinxcrossref{Segment}}}}}
\pysigstopsignatures
\sphinxAtStartPar
More flexible method for creating a segment.

\end{fulllineitems}



\subsection{deepdrr.geo.typing}
\label{\detokenize{deepdrr.geo:module-deepdrr.geo.typing}}\label{\detokenize{deepdrr.geo:deepdrr-geo-typing}}\index{module@\spxentry{module}!deepdrr.geo.typing@\spxentry{deepdrr.geo.typing}}\index{deepdrr.geo.typing@\spxentry{deepdrr.geo.typing}!module@\spxentry{module}}

\subsection{deepdrr.geo.utils}
\label{\detokenize{deepdrr.geo:module-deepdrr.geo.utils}}\label{\detokenize{deepdrr.geo:deepdrr-geo-utils}}\index{module@\spxentry{module}!deepdrr.geo.utils@\spxentry{deepdrr.geo.utils}}\index{deepdrr.geo.utils@\spxentry{deepdrr.geo.utils}!module@\spxentry{module}}

\subsection{Module contents}
\label{\detokenize{deepdrr.geo:module-deepdrr.geo}}\label{\detokenize{deepdrr.geo:module-contents}}\index{module@\spxentry{module}!deepdrr.geo@\spxentry{deepdrr.geo}}\index{deepdrr.geo@\spxentry{deepdrr.geo}!module@\spxentry{module}}
\sphinxAtStartPar
This file is part of DeepDRR.
Copyright (c) 2020 Benjamin D. Killeen.

\sphinxAtStartPar
DeepDRR is free software: you can redistribute it and/or modify
it under the terms of the GNU General Public License as published by
the Free Software Foundation, either version 3 of the License, or
(at your option) any later version.

\sphinxAtStartPar
DEEPDRR is distributed in the hope that it will be useful,
but WITHOUT ANY WARRANTY; without even the implied warranty of
MERCHANTABILITY or FITNESS FOR A PARTICULAR PURPOSE.  See the
GNU General Public License for more details.

\sphinxAtStartPar
You should have received a copy of the GNU General Public License
along with DeepDRR.  If not, see \textless{}\sphinxurl{https://www.gnu.org/licenses/}\textgreater{}.
\index{CameraIntrinsicTransform (class in deepdrr.geo)@\spxentry{CameraIntrinsicTransform}\spxextra{class in deepdrr.geo}}

\begin{fulllineitems}
\phantomsection\label{\detokenize{deepdrr.geo:deepdrr.geo.CameraIntrinsicTransform}}
\pysigstartsignatures
\pysiglinewithargsret{\sphinxbfcode{\sphinxupquote{class\DUrole{w,w}{  }}}\sphinxcode{\sphinxupquote{deepdrr.geo.}}\sphinxbfcode{\sphinxupquote{CameraIntrinsicTransform}}}{\sphinxparam{\DUrole{n,n}{data}\DUrole{p,p}{:}\DUrole{w,w}{  }\DUrole{n,n}{ndarray}}\sphinxparamcomma \sphinxparam{\DUrole{n,n}{sensor\_height}\DUrole{p,p}{:}\DUrole{w,w}{  }\DUrole{n,n}{int\DUrole{w,w}{  }\DUrole{p,p}{|}\DUrole{w,w}{  }None}\DUrole{w,w}{  }\DUrole{o,o}{=}\DUrole{w,w}{  }\DUrole{default_value}{None}}\sphinxparamcomma \sphinxparam{\DUrole{n,n}{sensor\_width}\DUrole{p,p}{:}\DUrole{w,w}{  }\DUrole{n,n}{int\DUrole{w,w}{  }\DUrole{p,p}{|}\DUrole{w,w}{  }None}\DUrole{w,w}{  }\DUrole{o,o}{=}\DUrole{w,w}{  }\DUrole{default_value}{None}}}{}
\pysigstopsignatures
\sphinxAtStartPar
Bases: {\hyperref[\detokenize{deepdrr.geo:deepdrr.geo.core.FrameTransform}]{\sphinxcrossref{\sphinxcode{\sphinxupquote{FrameTransform}}}}}
\index{aspect\_ratio (deepdrr.geo.CameraIntrinsicTransform property)@\spxentry{aspect\_ratio}\spxextra{deepdrr.geo.CameraIntrinsicTransform property}}

\begin{fulllineitems}
\phantomsection\label{\detokenize{deepdrr.geo:deepdrr.geo.CameraIntrinsicTransform.aspect_ratio}}
\pysigstartsignatures
\pysigline{\sphinxbfcode{\sphinxupquote{property\DUrole{w,w}{  }}}\sphinxbfcode{\sphinxupquote{aspect\_ratio}}\sphinxbfcode{\sphinxupquote{\DUrole{p,p}{:}\DUrole{w,w}{  }float}}}
\pysigstopsignatures
\sphinxAtStartPar
Image aspect ratio.

\end{fulllineitems}

\index{cx (deepdrr.geo.CameraIntrinsicTransform property)@\spxentry{cx}\spxextra{deepdrr.geo.CameraIntrinsicTransform property}}

\begin{fulllineitems}
\phantomsection\label{\detokenize{deepdrr.geo:deepdrr.geo.CameraIntrinsicTransform.cx}}
\pysigstartsignatures
\pysigline{\sphinxbfcode{\sphinxupquote{property\DUrole{w,w}{  }}}\sphinxbfcode{\sphinxupquote{cx}}\sphinxbfcode{\sphinxupquote{\DUrole{p,p}{:}\DUrole{w,w}{  }float}}}
\pysigstopsignatures
\end{fulllineitems}

\index{cy (deepdrr.geo.CameraIntrinsicTransform property)@\spxentry{cy}\spxextra{deepdrr.geo.CameraIntrinsicTransform property}}

\begin{fulllineitems}
\phantomsection\label{\detokenize{deepdrr.geo:deepdrr.geo.CameraIntrinsicTransform.cy}}
\pysigstartsignatures
\pysigline{\sphinxbfcode{\sphinxupquote{property\DUrole{w,w}{  }}}\sphinxbfcode{\sphinxupquote{cy}}\sphinxbfcode{\sphinxupquote{\DUrole{p,p}{:}\DUrole{w,w}{  }float}}}
\pysigstopsignatures
\end{fulllineitems}

\index{data (deepdrr.geo.CameraIntrinsicTransform attribute)@\spxentry{data}\spxextra{deepdrr.geo.CameraIntrinsicTransform attribute}}

\begin{fulllineitems}
\phantomsection\label{\detokenize{deepdrr.geo:deepdrr.geo.CameraIntrinsicTransform.data}}
\pysigstartsignatures
\pysigline{\sphinxbfcode{\sphinxupquote{data}}\sphinxbfcode{\sphinxupquote{\DUrole{p,p}{:}\DUrole{w,w}{  }ndarray}}}
\pysigstopsignatures
\end{fulllineitems}

\index{dim (deepdrr.geo.CameraIntrinsicTransform attribute)@\spxentry{dim}\spxextra{deepdrr.geo.CameraIntrinsicTransform attribute}}

\begin{fulllineitems}
\phantomsection\label{\detokenize{deepdrr.geo:deepdrr.geo.CameraIntrinsicTransform.dim}}
\pysigstartsignatures
\pysigline{\sphinxbfcode{\sphinxupquote{dim}}\sphinxbfcode{\sphinxupquote{\DUrole{p,p}{:}\DUrole{w,w}{  }int}}\sphinxbfcode{\sphinxupquote{\DUrole{w,w}{  }\DUrole{p,p}{=}\DUrole{w,w}{  }2}}}
\pysigstopsignatures
\end{fulllineitems}

\index{focal\_length (deepdrr.geo.CameraIntrinsicTransform property)@\spxentry{focal\_length}\spxextra{deepdrr.geo.CameraIntrinsicTransform property}}

\begin{fulllineitems}
\phantomsection\label{\detokenize{deepdrr.geo:deepdrr.geo.CameraIntrinsicTransform.focal_length}}
\pysigstartsignatures
\pysigline{\sphinxbfcode{\sphinxupquote{property\DUrole{w,w}{  }}}\sphinxbfcode{\sphinxupquote{focal\_length}}\sphinxbfcode{\sphinxupquote{\DUrole{p,p}{:}\DUrole{w,w}{  }float}}}
\pysigstopsignatures
\sphinxAtStartPar
Focal length in the matrix units.

\end{fulllineitems}

\index{from\_parameters() (deepdrr.geo.CameraIntrinsicTransform class method)@\spxentry{from\_parameters()}\spxextra{deepdrr.geo.CameraIntrinsicTransform class method}}

\begin{fulllineitems}
\phantomsection\label{\detokenize{deepdrr.geo:deepdrr.geo.CameraIntrinsicTransform.from_parameters}}
\pysigstartsignatures
\pysiglinewithargsret{\sphinxbfcode{\sphinxupquote{classmethod\DUrole{w,w}{  }}}\sphinxbfcode{\sphinxupquote{from\_parameters}}}{\sphinxparam{\DUrole{n,n}{optical\_center}\DUrole{p,p}{:}\DUrole{w,w}{  }\DUrole{n,n}{{\hyperref[\detokenize{deepdrr.geo:deepdrr.geo.core.Point2D}]{\sphinxcrossref{Point2D}}}}}\sphinxparamcomma \sphinxparam{\DUrole{n,n}{focal\_length}\DUrole{p,p}{:}\DUrole{w,w}{  }\DUrole{n,n}{float\DUrole{w,w}{  }\DUrole{p,p}{|}\DUrole{w,w}{  }Tuple\DUrole{p,p}{{[}}float\DUrole{p,p}{,}\DUrole{w,w}{  }float\DUrole{p,p}{{]}}}\DUrole{w,w}{  }\DUrole{o,o}{=}\DUrole{w,w}{  }\DUrole{default_value}{1}}\sphinxparamcomma \sphinxparam{\DUrole{n,n}{shear}\DUrole{p,p}{:}\DUrole{w,w}{  }\DUrole{n,n}{float}\DUrole{w,w}{  }\DUrole{o,o}{=}\DUrole{w,w}{  }\DUrole{default_value}{0}}\sphinxparamcomma \sphinxparam{\DUrole{n,n}{aspect\_ratio}\DUrole{p,p}{:}\DUrole{w,w}{  }\DUrole{n,n}{float\DUrole{w,w}{  }\DUrole{p,p}{|}\DUrole{w,w}{  }None}\DUrole{w,w}{  }\DUrole{o,o}{=}\DUrole{w,w}{  }\DUrole{default_value}{None}}}{{ $\rightarrow$ {\hyperref[\detokenize{deepdrr.geo:deepdrr.geo.core.CameraIntrinsicTransform}]{\sphinxcrossref{CameraIntrinsicTransform}}}}}
\pysigstopsignatures
\sphinxAtStartPar
The camera intrinsic matrix.

\sphinxAtStartPar
The intrinsic matrix is fundamentally a FrameTransform in 2D, namely \sphinxtitleref{index\_from\_camera2d}.
It transforms to the index\sphinxhyphen{}space of the image (as mapped on the sensor)
from the index\sphinxhyphen{}space centered on the principle ray.

\begin{sphinxadmonition}{note}{Note:}
\sphinxAtStartPar
Focal lengths are usually measured in world units (e.g. millimeters.). This function handles the conversion.
\end{sphinxadmonition}

\sphinxAtStartPar
Useful references include Szeliski’s “Computer Vision”
\sphinxhyphen{} \sphinxurl{https://ksimek.github.io/2013/08/13/intrinsic/}
\begin{quote}\begin{description}
\sphinxlineitem{Parameters}\begin{itemize}
\item {} 
\sphinxAtStartPar
\sphinxstyleliteralstrong{\sphinxupquote{optical\_center}} ({\hyperref[\detokenize{deepdrr.geo:deepdrr.geo.Point2D}]{\sphinxcrossref{\sphinxstyleliteralemphasis{\sphinxupquote{Point2D}}}}}) \textendash{} the index\sphinxhyphen{}space point where the isocenter (or pinhole) is centered.

\item {} 
\sphinxAtStartPar
\sphinxstyleliteralstrong{\sphinxupquote{focal\_length}} (\sphinxstyleliteralemphasis{\sphinxupquote{Union}}\sphinxstyleliteralemphasis{\sphinxupquote{{[}}}\sphinxstyleliteralemphasis{\sphinxupquote{float}}\sphinxstyleliteralemphasis{\sphinxupquote{, }}\sphinxstyleliteralemphasis{\sphinxupquote{Tuple}}\sphinxstyleliteralemphasis{\sphinxupquote{{[}}}\sphinxstyleliteralemphasis{\sphinxupquote{float}}\sphinxstyleliteralemphasis{\sphinxupquote{, }}\sphinxstyleliteralemphasis{\sphinxupquote{float}}\sphinxstyleliteralemphasis{\sphinxupquote{{]}}}\sphinxstyleliteralemphasis{\sphinxupquote{{]}}}) \textendash{} the focal length in index units. Can be a tubple (f\_x, f\_y),
or a scalar used for both, or a scalar modified by aspect\_ratio, in index units.

\item {} 
\sphinxAtStartPar
\sphinxstyleliteralstrong{\sphinxupquote{shear}} (\sphinxstyleliteralemphasis{\sphinxupquote{float}}) \textendash{} the shear \sphinxtitleref{s} of the camera.

\item {} 
\sphinxAtStartPar
\sphinxstyleliteralstrong{\sphinxupquote{aspect\_ratio}} (\sphinxstyleliteralemphasis{\sphinxupquote{Optional}}\sphinxstyleliteralemphasis{\sphinxupquote{{[}}}\sphinxstyleliteralemphasis{\sphinxupquote{float}}\sphinxstyleliteralemphasis{\sphinxupquote{{]}}}\sphinxstyleliteralemphasis{\sphinxupquote{, }}\sphinxstyleliteralemphasis{\sphinxupquote{optional}}) \textendash{} the aspect ratio \sphinxtitleref{a} (for use with one focal length). If not provided, aspect
ratio is 1. Defaults to None.

\end{itemize}

\sphinxlineitem{Returns}
\sphinxAtStartPar
The camera intrinsic matrix.

\sphinxlineitem{Return type}
\sphinxAtStartPar
{\hyperref[\detokenize{deepdrr.geo:deepdrr.geo.CameraIntrinsicTransform}]{\sphinxcrossref{CameraIntrinsicTransform}}}

\end{description}\end{quote}

\end{fulllineitems}

\index{from\_sizes() (deepdrr.geo.CameraIntrinsicTransform class method)@\spxentry{from\_sizes()}\spxextra{deepdrr.geo.CameraIntrinsicTransform class method}}

\begin{fulllineitems}
\phantomsection\label{\detokenize{deepdrr.geo:deepdrr.geo.CameraIntrinsicTransform.from_sizes}}
\pysigstartsignatures
\pysiglinewithargsret{\sphinxbfcode{\sphinxupquote{classmethod\DUrole{w,w}{  }}}\sphinxbfcode{\sphinxupquote{from\_sizes}}}{\sphinxparam{\DUrole{n,n}{sensor\_size}\DUrole{p,p}{:}\DUrole{w,w}{  }\DUrole{n,n}{int\DUrole{w,w}{  }\DUrole{p,p}{|}\DUrole{w,w}{  }Tuple\DUrole{p,p}{{[}}int\DUrole{p,p}{,}\DUrole{w,w}{  }int\DUrole{p,p}{{]}}}}\sphinxparamcomma \sphinxparam{\DUrole{n,n}{pixel\_size}\DUrole{p,p}{:}\DUrole{w,w}{  }\DUrole{n,n}{float\DUrole{w,w}{  }\DUrole{p,p}{|}\DUrole{w,w}{  }Tuple\DUrole{p,p}{{[}}float\DUrole{p,p}{,}\DUrole{w,w}{  }float\DUrole{p,p}{{]}}}}\sphinxparamcomma \sphinxparam{\DUrole{n,n}{source\_to\_detector\_distance}\DUrole{p,p}{:}\DUrole{w,w}{  }\DUrole{n,n}{float}}}{{ $\rightarrow$ {\hyperref[\detokenize{deepdrr.geo:deepdrr.geo.core.CameraIntrinsicTransform}]{\sphinxcrossref{CameraIntrinsicTransform}}}}}
\pysigstopsignatures
\sphinxAtStartPar
Generate the camera from human\sphinxhyphen{}readable parameters.

\sphinxAtStartPar
This is the recommended way to create the camera. Note that although pixel\_size and source\_to\_detector distance are measured in world units,
the camera intrinsic matrix contains no information about the world, as these are merely used to compute the focal length in pixels.
\begin{quote}\begin{description}
\sphinxlineitem{Parameters}\begin{itemize}
\item {} 
\sphinxAtStartPar
\sphinxstyleliteralstrong{\sphinxupquote{sensor\_size}} (\sphinxstyleliteralemphasis{\sphinxupquote{Union}}\sphinxstyleliteralemphasis{\sphinxupquote{{[}}}\sphinxstyleliteralemphasis{\sphinxupquote{float}}\sphinxstyleliteralemphasis{\sphinxupquote{, }}\sphinxstyleliteralemphasis{\sphinxupquote{Tuple}}\sphinxstyleliteralemphasis{\sphinxupquote{{[}}}\sphinxstyleliteralemphasis{\sphinxupquote{float}}\sphinxstyleliteralemphasis{\sphinxupquote{, }}\sphinxstyleliteralemphasis{\sphinxupquote{float}}\sphinxstyleliteralemphasis{\sphinxupquote{{]}}}\sphinxstyleliteralemphasis{\sphinxupquote{{]}}}) \textendash{} (width, height) of the sensor, or a single value for both, in pixels.

\item {} 
\sphinxAtStartPar
\sphinxstyleliteralstrong{\sphinxupquote{pixel\_size}} (\sphinxstyleliteralemphasis{\sphinxupquote{Union}}\sphinxstyleliteralemphasis{\sphinxupquote{{[}}}\sphinxstyleliteralemphasis{\sphinxupquote{float}}\sphinxstyleliteralemphasis{\sphinxupquote{, }}\sphinxstyleliteralemphasis{\sphinxupquote{Tuple}}\sphinxstyleliteralemphasis{\sphinxupquote{{[}}}\sphinxstyleliteralemphasis{\sphinxupquote{float}}\sphinxstyleliteralemphasis{\sphinxupquote{, }}\sphinxstyleliteralemphasis{\sphinxupquote{float}}\sphinxstyleliteralemphasis{\sphinxupquote{{]}}}\sphinxstyleliteralemphasis{\sphinxupquote{{]}}}) \textendash{} (width, height) of a pixel, or a single value for both, in world units (e.g. mm).

\item {} 
\sphinxAtStartPar
\sphinxstyleliteralstrong{\sphinxupquote{source\_to\_detector\_distance}} (\sphinxstyleliteralemphasis{\sphinxupquote{float}}) \textendash{} distance from source to detector in world units.

\end{itemize}

\end{description}\end{quote}

\sphinxAtStartPar
Returns:

\end{fulllineitems}

\index{fx (deepdrr.geo.CameraIntrinsicTransform property)@\spxentry{fx}\spxextra{deepdrr.geo.CameraIntrinsicTransform property}}

\begin{fulllineitems}
\phantomsection\label{\detokenize{deepdrr.geo:deepdrr.geo.CameraIntrinsicTransform.fx}}
\pysigstartsignatures
\pysigline{\sphinxbfcode{\sphinxupquote{property\DUrole{w,w}{  }}}\sphinxbfcode{\sphinxupquote{fx}}\sphinxbfcode{\sphinxupquote{\DUrole{p,p}{:}\DUrole{w,w}{  }float}}}
\pysigstopsignatures
\end{fulllineitems}

\index{fy (deepdrr.geo.CameraIntrinsicTransform property)@\spxentry{fy}\spxextra{deepdrr.geo.CameraIntrinsicTransform property}}

\begin{fulllineitems}
\phantomsection\label{\detokenize{deepdrr.geo:deepdrr.geo.CameraIntrinsicTransform.fy}}
\pysigstartsignatures
\pysigline{\sphinxbfcode{\sphinxupquote{property\DUrole{w,w}{  }}}\sphinxbfcode{\sphinxupquote{fy}}\sphinxbfcode{\sphinxupquote{\DUrole{p,p}{:}\DUrole{w,w}{  }float}}}
\pysigstopsignatures
\end{fulllineitems}

\index{get\_config() (deepdrr.geo.CameraIntrinsicTransform method)@\spxentry{get\_config()}\spxextra{deepdrr.geo.CameraIntrinsicTransform method}}

\begin{fulllineitems}
\phantomsection\label{\detokenize{deepdrr.geo:deepdrr.geo.CameraIntrinsicTransform.get_config}}
\pysigstartsignatures
\pysiglinewithargsret{\sphinxbfcode{\sphinxupquote{get\_config}}}{}{}
\pysigstopsignatures
\sphinxAtStartPar
Get a config dict with the data in this object.

\end{fulllineitems}

\index{input\_dim (deepdrr.geo.CameraIntrinsicTransform attribute)@\spxentry{input\_dim}\spxextra{deepdrr.geo.CameraIntrinsicTransform attribute}}

\begin{fulllineitems}
\phantomsection\label{\detokenize{deepdrr.geo:deepdrr.geo.CameraIntrinsicTransform.input_dim}}
\pysigstartsignatures
\pysigline{\sphinxbfcode{\sphinxupquote{input\_dim}}\sphinxbfcode{\sphinxupquote{\DUrole{p,p}{:}\DUrole{w,w}{  }int}}\sphinxbfcode{\sphinxupquote{\DUrole{w,w}{  }\DUrole{p,p}{=}\DUrole{w,w}{  }2}}}
\pysigstopsignatures
\sphinxAtStartPar
The intrinsic camera transform.

\sphinxAtStartPar
It should be scaled such that the units of the matrix (including the focal length) are in pixels

\end{fulllineitems}

\index{optical\_center (deepdrr.geo.CameraIntrinsicTransform property)@\spxentry{optical\_center}\spxextra{deepdrr.geo.CameraIntrinsicTransform property}}

\begin{fulllineitems}
\phantomsection\label{\detokenize{deepdrr.geo:deepdrr.geo.CameraIntrinsicTransform.optical_center}}
\pysigstartsignatures
\pysigline{\sphinxbfcode{\sphinxupquote{property\DUrole{w,w}{  }}}\sphinxbfcode{\sphinxupquote{optical\_center}}\sphinxbfcode{\sphinxupquote{\DUrole{p,p}{:}\DUrole{w,w}{  }{\hyperref[\detokenize{deepdrr.geo:deepdrr.geo.core.Point2D}]{\sphinxcrossref{Point2D}}}}}}
\pysigstopsignatures
\end{fulllineitems}

\index{sensor\_height (deepdrr.geo.CameraIntrinsicTransform property)@\spxentry{sensor\_height}\spxextra{deepdrr.geo.CameraIntrinsicTransform property}}

\begin{fulllineitems}
\phantomsection\label{\detokenize{deepdrr.geo:deepdrr.geo.CameraIntrinsicTransform.sensor_height}}
\pysigstartsignatures
\pysigline{\sphinxbfcode{\sphinxupquote{property\DUrole{w,w}{  }}}\sphinxbfcode{\sphinxupquote{sensor\_height}}\sphinxbfcode{\sphinxupquote{\DUrole{p,p}{:}\DUrole{w,w}{  }int}}}
\pysigstopsignatures
\sphinxAtStartPar
Get the sensor height in pixels.

\sphinxAtStartPar
Assumes optical center is at the center of the sensor.

\sphinxAtStartPar
Based on the convention of origin in top left, with x pointing to the right and y pointing down.

\end{fulllineitems}

\index{sensor\_size (deepdrr.geo.CameraIntrinsicTransform property)@\spxentry{sensor\_size}\spxextra{deepdrr.geo.CameraIntrinsicTransform property}}

\begin{fulllineitems}
\phantomsection\label{\detokenize{deepdrr.geo:deepdrr.geo.CameraIntrinsicTransform.sensor_size}}
\pysigstartsignatures
\pysigline{\sphinxbfcode{\sphinxupquote{property\DUrole{w,w}{  }}}\sphinxbfcode{\sphinxupquote{sensor\_size}}\sphinxbfcode{\sphinxupquote{\DUrole{p,p}{:}\DUrole{w,w}{  }Tuple\DUrole{p,p}{{[}}int\DUrole{p,p}{,}\DUrole{w,w}{  }int\DUrole{p,p}{{]}}}}}
\pysigstopsignatures
\sphinxAtStartPar
Tuple with the (width, height) of the sense/image, in matrix units.

\end{fulllineitems}

\index{sensor\_width (deepdrr.geo.CameraIntrinsicTransform property)@\spxentry{sensor\_width}\spxextra{deepdrr.geo.CameraIntrinsicTransform property}}

\begin{fulllineitems}
\phantomsection\label{\detokenize{deepdrr.geo:deepdrr.geo.CameraIntrinsicTransform.sensor_width}}
\pysigstartsignatures
\pysigline{\sphinxbfcode{\sphinxupquote{property\DUrole{w,w}{  }}}\sphinxbfcode{\sphinxupquote{sensor\_width}}\sphinxbfcode{\sphinxupquote{\DUrole{p,p}{:}\DUrole{w,w}{  }int}}}
\pysigstopsignatures
\sphinxAtStartPar
Get the sensor width in the matrix units.

\sphinxAtStartPar
Assumes optical center is at the center of the sensor.

\sphinxAtStartPar
Based on the convention of origin in top left, with x pointing to the right and y pointing down.

\end{fulllineitems}


\end{fulllineitems}

\index{CameraProjection (class in deepdrr.geo)@\spxentry{CameraProjection}\spxextra{class in deepdrr.geo}}

\begin{fulllineitems}
\phantomsection\label{\detokenize{deepdrr.geo:deepdrr.geo.CameraProjection}}
\pysigstartsignatures
\pysiglinewithargsret{\sphinxbfcode{\sphinxupquote{class\DUrole{w,w}{  }}}\sphinxcode{\sphinxupquote{deepdrr.geo.}}\sphinxbfcode{\sphinxupquote{CameraProjection}}}{\sphinxparam{\DUrole{n,n}{intrinsic}\DUrole{p,p}{:}\DUrole{w,w}{  }\DUrole{n,n}{{\hyperref[\detokenize{deepdrr.geo:deepdrr.geo.core.CameraIntrinsicTransform}]{\sphinxcrossref{CameraIntrinsicTransform}}}\DUrole{w,w}{  }\DUrole{p,p}{|}\DUrole{w,w}{  }ndarray}}\sphinxparamcomma \sphinxparam{\DUrole{n,n}{extrinsic}\DUrole{p,p}{:}\DUrole{w,w}{  }\DUrole{n,n}{{\hyperref[\detokenize{deepdrr.geo:deepdrr.geo.core.FrameTransform}]{\sphinxcrossref{FrameTransform}}}\DUrole{w,w}{  }\DUrole{p,p}{|}\DUrole{w,w}{  }ndarray}}}{}
\pysigstopsignatures
\sphinxAtStartPar
Bases: {\hyperref[\detokenize{deepdrr.geo:deepdrr.geo.core.Transform}]{\sphinxcrossref{\sphinxcode{\sphinxupquote{Transform}}}}}
\index{K (deepdrr.geo.CameraProjection property)@\spxentry{K}\spxextra{deepdrr.geo.CameraProjection property}}

\begin{fulllineitems}
\phantomsection\label{\detokenize{deepdrr.geo:deepdrr.geo.CameraProjection.K}}
\pysigstartsignatures
\pysigline{\sphinxbfcode{\sphinxupquote{property\DUrole{w,w}{  }}}\sphinxbfcode{\sphinxupquote{K}}}
\pysigstopsignatures
\end{fulllineitems}

\index{R (deepdrr.geo.CameraProjection property)@\spxentry{R}\spxextra{deepdrr.geo.CameraProjection property}}

\begin{fulllineitems}
\phantomsection\label{\detokenize{deepdrr.geo:deepdrr.geo.CameraProjection.R}}
\pysigstartsignatures
\pysigline{\sphinxbfcode{\sphinxupquote{property\DUrole{w,w}{  }}}\sphinxbfcode{\sphinxupquote{R}}}
\pysigstopsignatures
\end{fulllineitems}

\index{camera3d\_from\_index (deepdrr.geo.CameraProjection property)@\spxentry{camera3d\_from\_index}\spxextra{deepdrr.geo.CameraProjection property}}

\begin{fulllineitems}
\phantomsection\label{\detokenize{deepdrr.geo:deepdrr.geo.CameraProjection.camera3d_from_index}}
\pysigstartsignatures
\pysigline{\sphinxbfcode{\sphinxupquote{property\DUrole{w,w}{  }}}\sphinxbfcode{\sphinxupquote{camera3d\_from\_index}}\sphinxbfcode{\sphinxupquote{\DUrole{p,p}{:}\DUrole{w,w}{  }{\hyperref[\detokenize{deepdrr.geo:deepdrr.geo.core.Transform}]{\sphinxcrossref{Transform}}}}}}
\pysigstopsignatures
\end{fulllineitems}

\index{camera3d\_from\_world (deepdrr.geo.CameraProjection attribute)@\spxentry{camera3d\_from\_world}\spxextra{deepdrr.geo.CameraProjection attribute}}

\begin{fulllineitems}
\phantomsection\label{\detokenize{deepdrr.geo:deepdrr.geo.CameraProjection.camera3d_from_world}}
\pysigstartsignatures
\pysigline{\sphinxbfcode{\sphinxupquote{camera3d\_from\_world}}\sphinxbfcode{\sphinxupquote{\DUrole{p,p}{:}\DUrole{w,w}{  }{\hyperref[\detokenize{deepdrr.geo:deepdrr.geo.core.FrameTransform}]{\sphinxcrossref{FrameTransform}}}}}}
\pysigstopsignatures
\end{fulllineitems}

\index{center\_in\_world (deepdrr.geo.CameraProjection property)@\spxentry{center\_in\_world}\spxextra{deepdrr.geo.CameraProjection property}}

\begin{fulllineitems}
\phantomsection\label{\detokenize{deepdrr.geo:deepdrr.geo.CameraProjection.center_in_world}}
\pysigstartsignatures
\pysigline{\sphinxbfcode{\sphinxupquote{property\DUrole{w,w}{  }}}\sphinxbfcode{\sphinxupquote{center\_in\_world}}\sphinxbfcode{\sphinxupquote{\DUrole{p,p}{:}\DUrole{w,w}{  }{\hyperref[\detokenize{deepdrr.geo:deepdrr.geo.core.Point3D}]{\sphinxcrossref{Point3D}}}}}}
\pysigstopsignatures
\end{fulllineitems}

\index{data (deepdrr.geo.CameraProjection attribute)@\spxentry{data}\spxextra{deepdrr.geo.CameraProjection attribute}}

\begin{fulllineitems}
\phantomsection\label{\detokenize{deepdrr.geo:deepdrr.geo.CameraProjection.data}}
\pysigstartsignatures
\pysigline{\sphinxbfcode{\sphinxupquote{data}}\sphinxbfcode{\sphinxupquote{\DUrole{p,p}{:}\DUrole{w,w}{  }ndarray}}}
\pysigstopsignatures
\end{fulllineitems}

\index{dim (deepdrr.geo.CameraProjection attribute)@\spxentry{dim}\spxextra{deepdrr.geo.CameraProjection attribute}}

\begin{fulllineitems}
\phantomsection\label{\detokenize{deepdrr.geo:deepdrr.geo.CameraProjection.dim}}
\pysigstartsignatures
\pysigline{\sphinxbfcode{\sphinxupquote{dim}}\sphinxbfcode{\sphinxupquote{\DUrole{w,w}{  }\DUrole{p,p}{=}\DUrole{w,w}{  }3}}}
\pysigstopsignatures
\end{fulllineitems}

\index{extrinsic (deepdrr.geo.CameraProjection property)@\spxentry{extrinsic}\spxextra{deepdrr.geo.CameraProjection property}}

\begin{fulllineitems}
\phantomsection\label{\detokenize{deepdrr.geo:deepdrr.geo.CameraProjection.extrinsic}}
\pysigstartsignatures
\pysigline{\sphinxbfcode{\sphinxupquote{property\DUrole{w,w}{  }}}\sphinxbfcode{\sphinxupquote{extrinsic}}\sphinxbfcode{\sphinxupquote{\DUrole{p,p}{:}\DUrole{w,w}{  }{\hyperref[\detokenize{deepdrr.geo:deepdrr.geo.core.FrameTransform}]{\sphinxcrossref{FrameTransform}}}}}}
\pysigstopsignatures
\end{fulllineitems}

\index{from\_krt() (deepdrr.geo.CameraProjection class method)@\spxentry{from\_krt()}\spxextra{deepdrr.geo.CameraProjection class method}}

\begin{fulllineitems}
\phantomsection\label{\detokenize{deepdrr.geo:deepdrr.geo.CameraProjection.from_krt}}
\pysigstartsignatures
\pysiglinewithargsret{\sphinxbfcode{\sphinxupquote{classmethod\DUrole{w,w}{  }}}\sphinxbfcode{\sphinxupquote{from\_krt}}}{\sphinxparam{\DUrole{n,n}{K}\DUrole{p,p}{:}\DUrole{w,w}{  }\DUrole{n,n}{ndarray}}\sphinxparamcomma \sphinxparam{\DUrole{n,n}{R}\DUrole{p,p}{:}\DUrole{w,w}{  }\DUrole{n,n}{ndarray}}\sphinxparamcomma \sphinxparam{\DUrole{n,n}{t}\DUrole{p,p}{:}\DUrole{w,w}{  }\DUrole{n,n}{ndarray}}}{{ $\rightarrow$ {\hyperref[\detokenize{deepdrr.geo:deepdrr.geo.CameraProjection}]{\sphinxcrossref{CameraProjection}}}}}
\pysigstopsignatures
\sphinxAtStartPar
Create a CameraProjection from a camera intrinsic matrix and extrinsic matrix.
\begin{quote}\begin{description}
\sphinxlineitem{Parameters}\begin{itemize}
\item {} 
\sphinxAtStartPar
\sphinxstyleliteralstrong{\sphinxupquote{K}} (\sphinxstyleliteralemphasis{\sphinxupquote{np.ndarray}}) \textendash{} the camera intrinsic matrix.

\item {} 
\sphinxAtStartPar
\sphinxstyleliteralstrong{\sphinxupquote{R}} (\sphinxstyleliteralemphasis{\sphinxupquote{np.ndarray}}) \textendash{} the camera extrinsic matrix.

\item {} 
\sphinxAtStartPar
\sphinxstyleliteralstrong{\sphinxupquote{t}} (\sphinxstyleliteralemphasis{\sphinxupquote{np.ndarray}}) \textendash{} the camera extrinsic translation vector.

\end{itemize}

\sphinxlineitem{Returns}
\sphinxAtStartPar
the camera projection.

\sphinxlineitem{Return type}
\sphinxAtStartPar
{\hyperref[\detokenize{deepdrr.geo:deepdrr.geo.CameraProjection}]{\sphinxcrossref{CameraProjection}}}

\end{description}\end{quote}

\end{fulllineitems}

\index{from\_rtk() (deepdrr.geo.CameraProjection class method)@\spxentry{from\_rtk()}\spxextra{deepdrr.geo.CameraProjection class method}}

\begin{fulllineitems}
\phantomsection\label{\detokenize{deepdrr.geo:deepdrr.geo.CameraProjection.from_rtk}}
\pysigstartsignatures
\pysiglinewithargsret{\sphinxbfcode{\sphinxupquote{classmethod\DUrole{w,w}{  }}}\sphinxbfcode{\sphinxupquote{from\_rtk}}}{\sphinxparam{\DUrole{n,n}{R}\DUrole{p,p}{:}\DUrole{w,w}{  }\DUrole{n,n}{ndarray}}\sphinxparamcomma \sphinxparam{\DUrole{n,n}{t}\DUrole{p,p}{:}\DUrole{w,w}{  }\DUrole{n,n}{{\hyperref[\detokenize{deepdrr.geo:deepdrr.geo.core.Point3D}]{\sphinxcrossref{Point3D}}}}}\sphinxparamcomma \sphinxparam{\DUrole{n,n}{K}\DUrole{p,p}{:}\DUrole{w,w}{  }\DUrole{n,n}{{\hyperref[\detokenize{deepdrr.geo:deepdrr.geo.core.CameraIntrinsicTransform}]{\sphinxcrossref{CameraIntrinsicTransform}}}\DUrole{w,w}{  }\DUrole{p,p}{|}\DUrole{w,w}{  }ndarray}}}{}
\pysigstopsignatures
\end{fulllineitems}

\index{get\_center\_in\_volume() (deepdrr.geo.CameraProjection method)@\spxentry{get\_center\_in\_volume()}\spxextra{deepdrr.geo.CameraProjection method}}

\begin{fulllineitems}
\phantomsection\label{\detokenize{deepdrr.geo:deepdrr.geo.CameraProjection.get_center_in_volume}}
\pysigstartsignatures
\pysiglinewithargsret{\sphinxbfcode{\sphinxupquote{get\_center\_in\_volume}}}{\sphinxparam{\DUrole{n,n}{volume}\DUrole{p,p}{:}\DUrole{w,w}{  }\DUrole{n,n}{{\hyperref[\detokenize{deepdrr:deepdrr.Volume}]{\sphinxcrossref{Volume}}}}}}{{ $\rightarrow$ {\hyperref[\detokenize{deepdrr.geo:deepdrr.geo.Point3D}]{\sphinxcrossref{Point3D}}}}}
\pysigstopsignatures
\sphinxAtStartPar
Get the camera center in IJK\sphinxhyphen{}space.

\sphinxAtStartPar
In original deepdrr, this is the \sphinxtitleref{source\_point} of \sphinxtitleref{get\_canonical\_proj\_matrix()}
\begin{quote}\begin{description}
\sphinxlineitem{Parameters}
\sphinxAtStartPar
\sphinxstyleliteralstrong{\sphinxupquote{volume}} (\sphinxstyleliteralemphasis{\sphinxupquote{AnyVolume}}) \textendash{} the volume to get the camera center in.

\sphinxlineitem{Returns}
\sphinxAtStartPar
the camera center in the volume’s IJK\sphinxhyphen{}space.

\sphinxlineitem{Return type}
\sphinxAtStartPar
{\hyperref[\detokenize{deepdrr.geo:deepdrr.geo.Point3D}]{\sphinxcrossref{Point3D}}}

\end{description}\end{quote}

\end{fulllineitems}

\index{get\_center\_in\_world() (deepdrr.geo.CameraProjection method)@\spxentry{get\_center\_in\_world()}\spxextra{deepdrr.geo.CameraProjection method}}

\begin{fulllineitems}
\phantomsection\label{\detokenize{deepdrr.geo:deepdrr.geo.CameraProjection.get_center_in_world}}
\pysigstartsignatures
\pysiglinewithargsret{\sphinxbfcode{\sphinxupquote{get\_center\_in\_world}}}{}{{ $\rightarrow$ {\hyperref[\detokenize{deepdrr.geo:deepdrr.geo.core.Point3D}]{\sphinxcrossref{Point3D}}}}}
\pysigstopsignatures
\sphinxAtStartPar
Get the center of the camera (origin of camera3d frame) in world coordinates.

\sphinxAtStartPar
That is, get the translation vector of the world\_from\_camera3d FrameTransform

\sphinxAtStartPar
This is comparable to the function get\_camera\_center() in DeepDRR.
\begin{quote}\begin{description}
\sphinxlineitem{Returns}
\sphinxAtStartPar
the center of the camera in center.

\sphinxlineitem{Return type}
\sphinxAtStartPar
{\hyperref[\detokenize{deepdrr.geo:deepdrr.geo.Point3D}]{\sphinxcrossref{Point3D}}}

\end{description}\end{quote}

\end{fulllineitems}

\index{get\_config() (deepdrr.geo.CameraProjection method)@\spxentry{get\_config()}\spxextra{deepdrr.geo.CameraProjection method}}

\begin{fulllineitems}
\phantomsection\label{\detokenize{deepdrr.geo:deepdrr.geo.CameraProjection.get_config}}
\pysigstartsignatures
\pysiglinewithargsret{\sphinxbfcode{\sphinxupquote{get\_config}}}{}{{ $\rightarrow$ dict\DUrole{p,p}{{[}}str\DUrole{p,p}{,}\DUrole{w,w}{  }Any\DUrole{p,p}{{]}}}}
\pysigstopsignatures
\sphinxAtStartPar
Get the configuration of the camera projection.
\begin{quote}\begin{description}
\sphinxlineitem{Returns}
\sphinxAtStartPar
the configuration of the camera projection.

\sphinxlineitem{Return type}
\sphinxAtStartPar
dict{[}str, Any{]}

\end{description}\end{quote}

\end{fulllineitems}

\index{get\_ray\_transform() (deepdrr.geo.CameraProjection method)@\spxentry{get\_ray\_transform()}\spxextra{deepdrr.geo.CameraProjection method}}

\begin{fulllineitems}
\phantomsection\label{\detokenize{deepdrr.geo:deepdrr.geo.CameraProjection.get_ray_transform}}
\pysigstartsignatures
\pysiglinewithargsret{\sphinxbfcode{\sphinxupquote{get\_ray\_transform}}}{\sphinxparam{\DUrole{n,n}{volume}\DUrole{p,p}{:}\DUrole{w,w}{  }\DUrole{n,n}{{\hyperref[\detokenize{deepdrr:deepdrr.Volume}]{\sphinxcrossref{Volume}}}}}}{{ $\rightarrow$ {\hyperref[\detokenize{deepdrr.geo:deepdrr.geo.Transform}]{\sphinxcrossref{Transform}}}}}
\pysigstopsignatures
\sphinxAtStartPar
Get the ray transform for the camera, in IJK\sphinxhyphen{}space.

\sphinxAtStartPar
ijk\_from\_index transformation that goes from Point2D to Vector3D, with the vector in the
Point2D frame.

\sphinxAtStartPar
The ray transform takes a Point2D and converts it to a Vector3D. This is the vector in
the direction pointing between the camera center (or source) and a given index\sphinxhyphen{}space
point on the detector.
\begin{quote}\begin{description}
\sphinxlineitem{Parameters}
\sphinxAtStartPar
\sphinxstyleliteralstrong{\sphinxupquote{volume}} (\sphinxstyleliteralemphasis{\sphinxupquote{AnyVolume}}) \textendash{} the volume to get get the ray transfrom through.

\sphinxlineitem{Returns}
\sphinxAtStartPar
the \sphinxtitleref{ijk\_from\_index} transform.

\sphinxlineitem{Return type}
\sphinxAtStartPar
{\hyperref[\detokenize{deepdrr.geo:deepdrr.geo.Transform}]{\sphinxcrossref{Transform}}}

\end{description}\end{quote}

\end{fulllineitems}

\index{index\_from\_camera2d (deepdrr.geo.CameraProjection attribute)@\spxentry{index\_from\_camera2d}\spxextra{deepdrr.geo.CameraProjection attribute}}

\begin{fulllineitems}
\phantomsection\label{\detokenize{deepdrr.geo:deepdrr.geo.CameraProjection.index_from_camera2d}}
\pysigstartsignatures
\pysigline{\sphinxbfcode{\sphinxupquote{index\_from\_camera2d}}\sphinxbfcode{\sphinxupquote{\DUrole{p,p}{:}\DUrole{w,w}{  }{\hyperref[\detokenize{deepdrr.geo:deepdrr.geo.core.CameraIntrinsicTransform}]{\sphinxcrossref{CameraIntrinsicTransform}}}}}}
\pysigstopsignatures
\end{fulllineitems}

\index{index\_from\_camera3d (deepdrr.geo.CameraProjection property)@\spxentry{index\_from\_camera3d}\spxextra{deepdrr.geo.CameraProjection property}}

\begin{fulllineitems}
\phantomsection\label{\detokenize{deepdrr.geo:deepdrr.geo.CameraProjection.index_from_camera3d}}
\pysigstartsignatures
\pysigline{\sphinxbfcode{\sphinxupquote{property\DUrole{w,w}{  }}}\sphinxbfcode{\sphinxupquote{index\_from\_camera3d}}\sphinxbfcode{\sphinxupquote{\DUrole{p,p}{:}\DUrole{w,w}{  }{\hyperref[\detokenize{deepdrr.geo:deepdrr.geo.core.Transform}]{\sphinxcrossref{Transform}}}}}}
\pysigstopsignatures
\end{fulllineitems}

\index{index\_from\_world (deepdrr.geo.CameraProjection property)@\spxentry{index\_from\_world}\spxextra{deepdrr.geo.CameraProjection property}}

\begin{fulllineitems}
\phantomsection\label{\detokenize{deepdrr.geo:deepdrr.geo.CameraProjection.index_from_world}}
\pysigstartsignatures
\pysigline{\sphinxbfcode{\sphinxupquote{property\DUrole{w,w}{  }}}\sphinxbfcode{\sphinxupquote{index\_from\_world}}\sphinxbfcode{\sphinxupquote{\DUrole{p,p}{:}\DUrole{w,w}{  }{\hyperref[\detokenize{deepdrr.geo:deepdrr.geo.core.Transform}]{\sphinxcrossref{Transform}}}}}}
\pysigstopsignatures
\end{fulllineitems}

\index{intrinsic (deepdrr.geo.CameraProjection property)@\spxentry{intrinsic}\spxextra{deepdrr.geo.CameraProjection property}}

\begin{fulllineitems}
\phantomsection\label{\detokenize{deepdrr.geo:deepdrr.geo.CameraProjection.intrinsic}}
\pysigstartsignatures
\pysigline{\sphinxbfcode{\sphinxupquote{property\DUrole{w,w}{  }}}\sphinxbfcode{\sphinxupquote{intrinsic}}\sphinxbfcode{\sphinxupquote{\DUrole{p,p}{:}\DUrole{w,w}{  }{\hyperref[\detokenize{deepdrr.geo:deepdrr.geo.core.CameraIntrinsicTransform}]{\sphinxcrossref{CameraIntrinsicTransform}}}}}}
\pysigstopsignatures
\end{fulllineitems}

\index{principle\_ray\_in\_world (deepdrr.geo.CameraProjection property)@\spxentry{principle\_ray\_in\_world}\spxextra{deepdrr.geo.CameraProjection property}}

\begin{fulllineitems}
\phantomsection\label{\detokenize{deepdrr.geo:deepdrr.geo.CameraProjection.principle_ray_in_world}}
\pysigstartsignatures
\pysigline{\sphinxbfcode{\sphinxupquote{property\DUrole{w,w}{  }}}\sphinxbfcode{\sphinxupquote{principle\_ray\_in\_world}}\sphinxbfcode{\sphinxupquote{\DUrole{p,p}{:}\DUrole{w,w}{  }{\hyperref[\detokenize{deepdrr.geo:deepdrr.geo.core.Vector3D}]{\sphinxcrossref{Vector3D}}}}}}
\pysigstopsignatures
\sphinxAtStartPar
Get the principle ray in world coordinates.

\end{fulllineitems}

\index{sensor\_height (deepdrr.geo.CameraProjection property)@\spxentry{sensor\_height}\spxextra{deepdrr.geo.CameraProjection property}}

\begin{fulllineitems}
\phantomsection\label{\detokenize{deepdrr.geo:deepdrr.geo.CameraProjection.sensor_height}}
\pysigstartsignatures
\pysigline{\sphinxbfcode{\sphinxupquote{property\DUrole{w,w}{  }}}\sphinxbfcode{\sphinxupquote{sensor\_height}}\sphinxbfcode{\sphinxupquote{\DUrole{p,p}{:}\DUrole{w,w}{  }int}}}
\pysigstopsignatures
\end{fulllineitems}

\index{sensor\_width (deepdrr.geo.CameraProjection property)@\spxentry{sensor\_width}\spxextra{deepdrr.geo.CameraProjection property}}

\begin{fulllineitems}
\phantomsection\label{\detokenize{deepdrr.geo:deepdrr.geo.CameraProjection.sensor_width}}
\pysigstartsignatures
\pysigline{\sphinxbfcode{\sphinxupquote{property\DUrole{w,w}{  }}}\sphinxbfcode{\sphinxupquote{sensor\_width}}\sphinxbfcode{\sphinxupquote{\DUrole{p,p}{:}\DUrole{w,w}{  }int}}}
\pysigstopsignatures
\end{fulllineitems}

\index{t (deepdrr.geo.CameraProjection property)@\spxentry{t}\spxextra{deepdrr.geo.CameraProjection property}}

\begin{fulllineitems}
\phantomsection\label{\detokenize{deepdrr.geo:deepdrr.geo.CameraProjection.t}}
\pysigstartsignatures
\pysigline{\sphinxbfcode{\sphinxupquote{property\DUrole{w,w}{  }}}\sphinxbfcode{\sphinxupquote{t}}}
\pysigstopsignatures
\end{fulllineitems}

\index{world\_from\_camera3d (deepdrr.geo.CameraProjection property)@\spxentry{world\_from\_camera3d}\spxextra{deepdrr.geo.CameraProjection property}}

\begin{fulllineitems}
\phantomsection\label{\detokenize{deepdrr.geo:deepdrr.geo.CameraProjection.world_from_camera3d}}
\pysigstartsignatures
\pysigline{\sphinxbfcode{\sphinxupquote{property\DUrole{w,w}{  }}}\sphinxbfcode{\sphinxupquote{world\_from\_camera3d}}\sphinxbfcode{\sphinxupquote{\DUrole{p,p}{:}\DUrole{w,w}{  }{\hyperref[\detokenize{deepdrr.geo:deepdrr.geo.core.FrameTransform}]{\sphinxcrossref{FrameTransform}}}}}}
\pysigstopsignatures
\end{fulllineitems}

\index{world\_from\_index (deepdrr.geo.CameraProjection property)@\spxentry{world\_from\_index}\spxextra{deepdrr.geo.CameraProjection property}}

\begin{fulllineitems}
\phantomsection\label{\detokenize{deepdrr.geo:deepdrr.geo.CameraProjection.world_from_index}}
\pysigstartsignatures
\pysigline{\sphinxbfcode{\sphinxupquote{property\DUrole{w,w}{  }}}\sphinxbfcode{\sphinxupquote{world\_from\_index}}\sphinxbfcode{\sphinxupquote{\DUrole{p,p}{:}\DUrole{w,w}{  }{\hyperref[\detokenize{deepdrr.geo:deepdrr.geo.core.Transform}]{\sphinxcrossref{Transform}}}}}}
\pysigstopsignatures
\sphinxAtStartPar
Gets the world\sphinxhyphen{}space vector between the source in world and the given point in index space.

\end{fulllineitems}

\index{world\_from\_index\_on\_image\_plane (deepdrr.geo.CameraProjection property)@\spxentry{world\_from\_index\_on\_image\_plane}\spxextra{deepdrr.geo.CameraProjection property}}

\begin{fulllineitems}
\phantomsection\label{\detokenize{deepdrr.geo:deepdrr.geo.CameraProjection.world_from_index_on_image_plane}}
\pysigstartsignatures
\pysigline{\sphinxbfcode{\sphinxupquote{property\DUrole{w,w}{  }}}\sphinxbfcode{\sphinxupquote{world\_from\_index\_on\_image\_plane}}\sphinxbfcode{\sphinxupquote{\DUrole{p,p}{:}\DUrole{w,w}{  }{\hyperref[\detokenize{deepdrr.geo:deepdrr.geo.core.FrameTransform}]{\sphinxcrossref{FrameTransform}}}}}}
\pysigstopsignatures
\sphinxAtStartPar
Get the transform to points in world on the image (detector) plane from image indices.

\sphinxAtStartPar
The point input point should still be 3D, with a 0 in the z coordinate.

\end{fulllineitems}


\end{fulllineitems}

\index{F (class in deepdrr.geo)@\spxentry{F}\spxextra{class in deepdrr.geo}}

\begin{fulllineitems}
\phantomsection\label{\detokenize{deepdrr.geo:deepdrr.geo.F}}
\pysigstartsignatures
\pysiglinewithargsret{\sphinxbfcode{\sphinxupquote{class\DUrole{w,w}{  }}}\sphinxcode{\sphinxupquote{deepdrr.geo.}}\sphinxbfcode{\sphinxupquote{F}}}{\sphinxparam{\DUrole{n,n}{data}\DUrole{p,p}{:}\DUrole{w,w}{  }\DUrole{n,n}{ndarray}}}{}
\pysigstopsignatures
\sphinxAtStartPar
Bases: {\hyperref[\detokenize{deepdrr.geo:deepdrr.geo.core.FrameTransform}]{\sphinxcrossref{\sphinxcode{\sphinxupquote{FrameTransform}}}}}

\sphinxAtStartPar
Alias for FrameTransform.
\index{data (deepdrr.geo.F attribute)@\spxentry{data}\spxextra{deepdrr.geo.F attribute}}

\begin{fulllineitems}
\phantomsection\label{\detokenize{deepdrr.geo:deepdrr.geo.F.data}}
\pysigstartsignatures
\pysigline{\sphinxbfcode{\sphinxupquote{data}}\sphinxbfcode{\sphinxupquote{\DUrole{p,p}{:}\DUrole{w,w}{  }ndarray}}}
\pysigstopsignatures
\end{fulllineitems}


\end{fulllineitems}

\index{FrameTransform (class in deepdrr.geo)@\spxentry{FrameTransform}\spxextra{class in deepdrr.geo}}

\begin{fulllineitems}
\phantomsection\label{\detokenize{deepdrr.geo:deepdrr.geo.FrameTransform}}
\pysigstartsignatures
\pysiglinewithargsret{\sphinxbfcode{\sphinxupquote{class\DUrole{w,w}{  }}}\sphinxcode{\sphinxupquote{deepdrr.geo.}}\sphinxbfcode{\sphinxupquote{FrameTransform}}}{\sphinxparam{\DUrole{n,n}{data}\DUrole{p,p}{:}\DUrole{w,w}{  }\DUrole{n,n}{ndarray}}}{}
\pysigstopsignatures
\sphinxAtStartPar
Bases: {\hyperref[\detokenize{deepdrr.geo:deepdrr.geo.core.Transform}]{\sphinxcrossref{\sphinxcode{\sphinxupquote{Transform}}}}}, {\hyperref[\detokenize{deepdrr.geo:deepdrr.geo.core.HasLocation}]{\sphinxcrossref{\sphinxcode{\sphinxupquote{HasLocation}}}}}
\index{R (deepdrr.geo.FrameTransform property)@\spxentry{R}\spxextra{deepdrr.geo.FrameTransform property}}

\begin{fulllineitems}
\phantomsection\label{\detokenize{deepdrr.geo:deepdrr.geo.FrameTransform.R}}
\pysigstartsignatures
\pysigline{\sphinxbfcode{\sphinxupquote{property\DUrole{w,w}{  }}}\sphinxbfcode{\sphinxupquote{R}}}
\pysigstopsignatures
\end{fulllineitems}

\index{as\_quatpos() (deepdrr.geo.FrameTransform method)@\spxentry{as\_quatpos()}\spxextra{deepdrr.geo.FrameTransform method}}

\begin{fulllineitems}
\phantomsection\label{\detokenize{deepdrr.geo:deepdrr.geo.FrameTransform.as_quatpos}}
\pysigstartsignatures
\pysiglinewithargsret{\sphinxbfcode{\sphinxupquote{as\_quatpos}}}{}{{ $\rightarrow$ ndarray}}
\pysigstopsignatures
\sphinxAtStartPar
Return the transform as a quaternion and position.
\begin{quote}\begin{description}
\sphinxlineitem{Returns}
\sphinxAtStartPar
A 7\sphinxhyphen{}element array, with the first 4 elements being the quaternion, and the last 3 being the position.

\sphinxlineitem{Return type}
\sphinxAtStartPar
np.ndarray

\end{description}\end{quote}

\end{fulllineitems}

\index{data (deepdrr.geo.FrameTransform attribute)@\spxentry{data}\spxextra{deepdrr.geo.FrameTransform attribute}}

\begin{fulllineitems}
\phantomsection\label{\detokenize{deepdrr.geo:deepdrr.geo.FrameTransform.data}}
\pysigstartsignatures
\pysigline{\sphinxbfcode{\sphinxupquote{data}}\sphinxbfcode{\sphinxupquote{\DUrole{p,p}{:}\DUrole{w,w}{  }ndarray}}}
\pysigstopsignatures
\end{fulllineitems}

\index{dim (deepdrr.geo.FrameTransform property)@\spxentry{dim}\spxextra{deepdrr.geo.FrameTransform property}}

\begin{fulllineitems}
\phantomsection\label{\detokenize{deepdrr.geo:deepdrr.geo.FrameTransform.dim}}
\pysigstartsignatures
\pysigline{\sphinxbfcode{\sphinxupquote{property\DUrole{w,w}{  }}}\sphinxbfcode{\sphinxupquote{dim}}}
\pysigstopsignatures
\sphinxAtStartPar
The output dimension of the transformation.

\end{fulllineitems}

\index{from\_line\_segments() (deepdrr.geo.FrameTransform class method)@\spxentry{from\_line\_segments()}\spxextra{deepdrr.geo.FrameTransform class method}}

\begin{fulllineitems}
\phantomsection\label{\detokenize{deepdrr.geo:deepdrr.geo.FrameTransform.from_line_segments}}
\pysigstartsignatures
\pysiglinewithargsret{\sphinxbfcode{\sphinxupquote{classmethod\DUrole{w,w}{  }}}\sphinxbfcode{\sphinxupquote{from\_line\_segments}}}{\sphinxparam{\DUrole{n,n}{x\_B}\DUrole{p,p}{:}\DUrole{w,w}{  }\DUrole{n,n}{{\hyperref[\detokenize{deepdrr.geo:deepdrr.geo.core.Point3D}]{\sphinxcrossref{Point3D}}}}}\sphinxparamcomma \sphinxparam{\DUrole{n,n}{y\_B}\DUrole{p,p}{:}\DUrole{w,w}{  }\DUrole{n,n}{{\hyperref[\detokenize{deepdrr.geo:deepdrr.geo.core.Point3D}]{\sphinxcrossref{Point3D}}}}}\sphinxparamcomma \sphinxparam{\DUrole{n,n}{x\_A}\DUrole{p,p}{:}\DUrole{w,w}{  }\DUrole{n,n}{{\hyperref[\detokenize{deepdrr.geo:deepdrr.geo.core.Point3D}]{\sphinxcrossref{Point3D}}}}}\sphinxparamcomma \sphinxparam{\DUrole{n,n}{y\_A}\DUrole{p,p}{:}\DUrole{w,w}{  }\DUrole{n,n}{{\hyperref[\detokenize{deepdrr.geo:deepdrr.geo.core.Point3D}]{\sphinxcrossref{Point3D}}}}}}{{ $\rightarrow$ {\hyperref[\detokenize{deepdrr.geo:deepdrr.geo.core.FrameTransform}]{\sphinxcrossref{FrameTransform}}}}}
\pysigstopsignatures
\sphinxAtStartPar
Get the \sphinxtitleref{B\_from\_A} frame transform that aligns the line segments, given by endpoints.

\sphinxAtStartPar
Perfectly aligns the two line segments, so there is possibly some scaling.
\begin{quote}\begin{description}
\sphinxlineitem{Parameters}\begin{itemize}
\item {} 
\sphinxAtStartPar
\sphinxstyleliteralstrong{\sphinxupquote{x\_B}} ({\hyperref[\detokenize{deepdrr.geo:deepdrr.geo.Point3D}]{\sphinxcrossref{\sphinxstyleliteralemphasis{\sphinxupquote{Point3D}}}}}) \textendash{} The first endpoint, in frame B.

\item {} 
\sphinxAtStartPar
\sphinxstyleliteralstrong{\sphinxupquote{y\_B}} ({\hyperref[\detokenize{deepdrr.geo:deepdrr.geo.Point3D}]{\sphinxcrossref{\sphinxstyleliteralemphasis{\sphinxupquote{Point3D}}}}}) \textendash{} The second endpoint, in frame B.

\item {} 
\sphinxAtStartPar
\sphinxstyleliteralstrong{\sphinxupquote{x\_A}} ({\hyperref[\detokenize{deepdrr.geo:deepdrr.geo.Point3D}]{\sphinxcrossref{\sphinxstyleliteralemphasis{\sphinxupquote{Point3D}}}}}) \textendash{} The first endpoint, in frame A.

\item {} 
\sphinxAtStartPar
\sphinxstyleliteralstrong{\sphinxupquote{y\_A}} ({\hyperref[\detokenize{deepdrr.geo:deepdrr.geo.Point3D}]{\sphinxcrossref{\sphinxstyleliteralemphasis{\sphinxupquote{Point3D}}}}}) \textendash{} The second endpoint, in frame A.

\end{itemize}

\sphinxlineitem{Returns}
\sphinxAtStartPar
\begin{description}
\sphinxlineitem{A \sphinxtitleref{B\_from\_A} transform that aligns the points.}
\sphinxAtStartPar
Note that this is not unique, due to rotation about the axis between the points.

\end{description}


\sphinxlineitem{Return type}
\sphinxAtStartPar
{\hyperref[\detokenize{deepdrr.geo:deepdrr.geo.FrameTransform}]{\sphinxcrossref{FrameTransform}}}

\end{description}\end{quote}

\end{fulllineitems}

\index{from\_origin() (deepdrr.geo.FrameTransform class method)@\spxentry{from\_origin()}\spxextra{deepdrr.geo.FrameTransform class method}}

\begin{fulllineitems}
\phantomsection\label{\detokenize{deepdrr.geo:deepdrr.geo.FrameTransform.from_origin}}
\pysigstartsignatures
\pysiglinewithargsret{\sphinxbfcode{\sphinxupquote{classmethod\DUrole{w,w}{  }}}\sphinxbfcode{\sphinxupquote{from\_origin}}}{\sphinxparam{\DUrole{n,n}{origin}\DUrole{p,p}{:}\DUrole{w,w}{  }\DUrole{n,n}{{\hyperref[\detokenize{deepdrr.geo:deepdrr.geo.core.Point}]{\sphinxcrossref{Point}}}}}}{{ $\rightarrow$ {\hyperref[\detokenize{deepdrr.geo:deepdrr.geo.core.FrameTransform}]{\sphinxcrossref{FrameTransform}}}}}
\pysigstopsignatures
\sphinxAtStartPar
Make a transfrom to a frame knowing the origin.

\sphinxAtStartPar
Suppose \sphinxtitleref{origin} is point where frame \sphinxtitleref{B} has its origin, as a point
in frame \sphinxtitleref{A}. Make the \sphinxtitleref{B\_from\_A} transform.
This just negates \sphinxtitleref{origin}, but this is often counterintuitive.
\begin{quote}\begin{description}
\sphinxlineitem{Parameters}
\sphinxAtStartPar
\sphinxstyleliteralstrong{\sphinxupquote{origin}} ({\hyperref[\detokenize{deepdrr.geo:deepdrr.geo.Point}]{\sphinxcrossref{\sphinxstyleliteralemphasis{\sphinxupquote{Point}}}}}) \textendash{} origin of the target frame in the world frame

\sphinxlineitem{Returns}
\sphinxAtStartPar
the B\_from\_A transform.

\sphinxlineitem{Return type}
\sphinxAtStartPar
{\hyperref[\detokenize{deepdrr.geo:deepdrr.geo.FrameTransform}]{\sphinxcrossref{FrameTransform}}}

\end{description}\end{quote}

\end{fulllineitems}

\index{from\_pd() (deepdrr.geo.FrameTransform class method)@\spxentry{from\_pd()}\spxextra{deepdrr.geo.FrameTransform class method}}

\begin{fulllineitems}
\phantomsection\label{\detokenize{deepdrr.geo:deepdrr.geo.FrameTransform.from_pd}}
\pysigstartsignatures
\pysiglinewithargsret{\sphinxbfcode{\sphinxupquote{classmethod\DUrole{w,w}{  }}}\sphinxbfcode{\sphinxupquote{from\_pd}}}{\sphinxparam{\DUrole{n,n}{origin}\DUrole{p,p}{:}\DUrole{w,w}{  }\DUrole{n,n}{{\hyperref[\detokenize{deepdrr.geo:deepdrr.geo.core.Point3D}]{\sphinxcrossref{Point3D}}}}}\sphinxparamcomma \sphinxparam{\DUrole{n,n}{direction}\DUrole{p,p}{:}\DUrole{w,w}{  }\DUrole{n,n}{{\hyperref[\detokenize{deepdrr.geo:deepdrr.geo.core.Vector3D}]{\sphinxcrossref{Vector3D}}}}}\sphinxparamcomma \sphinxparam{\DUrole{n,n}{axis}\DUrole{p,p}{:}\DUrole{w,w}{  }\DUrole{n,n}{str\DUrole{w,w}{  }\DUrole{p,p}{|}\DUrole{w,w}{  }{\hyperref[\detokenize{deepdrr.geo:deepdrr.geo.core.Vector3D}]{\sphinxcrossref{Vector3D}}}}\DUrole{w,w}{  }\DUrole{o,o}{=}\DUrole{w,w}{  }\DUrole{default_value}{\textquotesingle{}z\textquotesingle{}}}}{{ $\rightarrow$ {\hyperref[\detokenize{deepdrr.geo:deepdrr.geo.core.FrameTransform}]{\sphinxcrossref{FrameTransform}}}}}
\pysigstopsignatures
\sphinxAtStartPar
Get the pose of the coordinate frame given its origin and direction of the given axis.
\begin{quote}\begin{description}
\sphinxlineitem{Parameters}\begin{itemize}
\item {} 
\sphinxAtStartPar
\sphinxstyleliteralstrong{\sphinxupquote{origin}} ({\hyperref[\detokenize{deepdrr.geo:deepdrr.geo.Point3D}]{\sphinxcrossref{\sphinxstyleliteralemphasis{\sphinxupquote{Point3D}}}}}) \textendash{} The origin of the frame.

\item {} 
\sphinxAtStartPar
\sphinxstyleliteralstrong{\sphinxupquote{direction}} ({\hyperref[\detokenize{deepdrr.geo:deepdrr.geo.Vector3D}]{\sphinxcrossref{\sphinxstyleliteralemphasis{\sphinxupquote{Vector3D}}}}}) \textendash{} The direction of the axis.

\item {} 
\sphinxAtStartPar
\sphinxstyleliteralstrong{\sphinxupquote{axis}} (\sphinxstyleliteralemphasis{\sphinxupquote{Union}}\sphinxstyleliteralemphasis{\sphinxupquote{{[}}}\sphinxstyleliteralemphasis{\sphinxupquote{str}}\sphinxstyleliteralemphasis{\sphinxupquote{, }}{\hyperref[\detokenize{deepdrr.geo:deepdrr.geo.Vector3D}]{\sphinxcrossref{\sphinxstyleliteralemphasis{\sphinxupquote{Vector3D}}}}}\sphinxstyleliteralemphasis{\sphinxupquote{{]}}}\sphinxstyleliteralemphasis{\sphinxupquote{, }}\sphinxstyleliteralemphasis{\sphinxupquote{optional}}) \textendash{} The axis to align with the direction. Defaults to “z”.

\end{itemize}

\end{description}\end{quote}

\end{fulllineitems}

\index{from\_point\_correspondence() (deepdrr.geo.FrameTransform class method)@\spxentry{from\_point\_correspondence()}\spxextra{deepdrr.geo.FrameTransform class method}}

\begin{fulllineitems}
\phantomsection\label{\detokenize{deepdrr.geo:deepdrr.geo.FrameTransform.from_point_correspondence}}
\pysigstartsignatures
\pysiglinewithargsret{\sphinxbfcode{\sphinxupquote{classmethod\DUrole{w,w}{  }}}\sphinxbfcode{\sphinxupquote{from\_point\_correspondence}}}{\sphinxparam{\DUrole{n,n}{points\_B}\DUrole{p,p}{:}\DUrole{w,w}{  }\DUrole{n,n}{List\DUrole{p,p}{{[}}{\hyperref[\detokenize{deepdrr.geo:deepdrr.geo.core.Point}]{\sphinxcrossref{Point}}}\DUrole{p,p}{{]}}\DUrole{w,w}{  }\DUrole{p,p}{|}\DUrole{w,w}{  }ndarray}}\sphinxparamcomma \sphinxparam{\DUrole{n,n}{points\_A}\DUrole{p,p}{:}\DUrole{w,w}{  }\DUrole{n,n}{List\DUrole{p,p}{{[}}{\hyperref[\detokenize{deepdrr.geo:deepdrr.geo.core.Point}]{\sphinxcrossref{Point}}}\DUrole{p,p}{{]}}\DUrole{w,w}{  }\DUrole{p,p}{|}\DUrole{w,w}{  }ndarray}}}{}
\pysigstopsignatures
\sphinxAtStartPar
Create a (rigid) frame transform from a known point correspondence.
\begin{quote}\begin{description}
\sphinxlineitem{Parameters}\begin{itemize}
\item {} 
\sphinxAtStartPar
\sphinxstyleliteralstrong{\sphinxupquote{points\_B}} \textendash{} a list of N corresponding points in the B frame.

\item {} 
\sphinxAtStartPar
\sphinxstyleliteralstrong{\sphinxupquote{points\_A}} \textendash{} a list of N points in the A frame (or an array with shape {[}N, 3{]}).

\end{itemize}

\sphinxlineitem{Returns}
\sphinxAtStartPar
\begin{description}
\sphinxlineitem{the \sphinxtitleref{B\_from\_A} transform that minimizes the mean squared distance}
\sphinxAtStartPar
between matching points.

\end{description}


\sphinxlineitem{Return type}
\sphinxAtStartPar
{\hyperref[\detokenize{deepdrr.geo:deepdrr.geo.FrameTransform}]{\sphinxcrossref{FrameTransform}}}

\end{description}\end{quote}

\end{fulllineitems}

\index{from\_pointdir() (deepdrr.geo.FrameTransform class method)@\spxentry{from\_pointdir()}\spxextra{deepdrr.geo.FrameTransform class method}}

\begin{fulllineitems}
\phantomsection\label{\detokenize{deepdrr.geo:deepdrr.geo.FrameTransform.from_pointdir}}
\pysigstartsignatures
\pysiglinewithargsret{\sphinxbfcode{\sphinxupquote{classmethod\DUrole{w,w}{  }}}\sphinxbfcode{\sphinxupquote{from\_pointdir}}}{\sphinxparam{\DUrole{o,o}{*}\DUrole{n,n}{args}}\sphinxparamcomma \sphinxparam{\DUrole{o,o}{**}\DUrole{n,n}{kwargs}}}{{ $\rightarrow$ {\hyperref[\detokenize{deepdrr.geo:deepdrr.geo.core.FrameTransform}]{\sphinxcrossref{FrameTransform}}}}}
\pysigstopsignatures
\sphinxAtStartPar
Alias for \sphinxtitleref{from\_pd}.

\end{fulllineitems}

\index{from\_points() (deepdrr.geo.FrameTransform class method)@\spxentry{from\_points()}\spxextra{deepdrr.geo.FrameTransform class method}}

\begin{fulllineitems}
\phantomsection\label{\detokenize{deepdrr.geo:deepdrr.geo.FrameTransform.from_points}}
\pysigstartsignatures
\pysiglinewithargsret{\sphinxbfcode{\sphinxupquote{classmethod\DUrole{w,w}{  }}}\sphinxbfcode{\sphinxupquote{from\_points}}}{\sphinxparam{\DUrole{n,n}{points\_B}\DUrole{p,p}{:}\DUrole{w,w}{  }\DUrole{n,n}{List\DUrole{p,p}{{[}}{\hyperref[\detokenize{deepdrr.geo:deepdrr.geo.core.Point3D}]{\sphinxcrossref{Point3D}}}\DUrole{p,p}{{]}}\DUrole{w,w}{  }\DUrole{p,p}{|}\DUrole{w,w}{  }ndarray}}\sphinxparamcomma \sphinxparam{\DUrole{n,n}{points\_A}\DUrole{p,p}{:}\DUrole{w,w}{  }\DUrole{n,n}{List\DUrole{p,p}{{[}}{\hyperref[\detokenize{deepdrr.geo:deepdrr.geo.core.Point3D}]{\sphinxcrossref{Point3D}}}\DUrole{p,p}{{]}}\DUrole{w,w}{  }\DUrole{p,p}{|}\DUrole{w,w}{  }ndarray}}\sphinxparamcomma \sphinxparam{\DUrole{n,n}{max\_iterations}\DUrole{p,p}{:}\DUrole{w,w}{  }\DUrole{n,n}{int}\DUrole{w,w}{  }\DUrole{o,o}{=}\DUrole{w,w}{  }\DUrole{default_value}{1000}}}{}
\pysigstopsignatures
\sphinxAtStartPar
Create a (rigid) frame transform from ICP between the two point clouds.
\begin{quote}\begin{description}
\sphinxlineitem{Parameters}\begin{itemize}
\item {} 
\sphinxAtStartPar
\sphinxstyleliteralstrong{\sphinxupquote{points\_B}} \textendash{} a list of M points in the B frame.

\item {} 
\sphinxAtStartPar
\sphinxstyleliteralstrong{\sphinxupquote{points\_A}} \textendash{} a list of N points in the A frame, corresponding to the same shape as \sphinxtitleref{points\_B}.

\end{itemize}

\sphinxlineitem{Returns}
\sphinxAtStartPar
\begin{description}
\sphinxlineitem{the \sphinxtitleref{B\_from\_A} transform that minimizes the mean squared distance}
\sphinxAtStartPar
between matching points.

\end{description}


\sphinxlineitem{Return type}
\sphinxAtStartPar
{\hyperref[\detokenize{deepdrr.geo:deepdrr.geo.FrameTransform}]{\sphinxcrossref{FrameTransform}}}

\end{description}\end{quote}

\end{fulllineitems}

\index{from\_quatpos() (deepdrr.geo.FrameTransform class method)@\spxentry{from\_quatpos()}\spxextra{deepdrr.geo.FrameTransform class method}}

\begin{fulllineitems}
\phantomsection\label{\detokenize{deepdrr.geo:deepdrr.geo.FrameTransform.from_quatpos}}
\pysigstartsignatures
\pysiglinewithargsret{\sphinxbfcode{\sphinxupquote{classmethod\DUrole{w,w}{  }}}\sphinxbfcode{\sphinxupquote{from\_quatpos}}}{\sphinxparam{\DUrole{n,n}{quatpos}\DUrole{p,p}{:}\DUrole{w,w}{  }\DUrole{n,n}{ndarray}}}{{ $\rightarrow$ {\hyperref[\detokenize{deepdrr.geo:deepdrr.geo.core.FrameTransform}]{\sphinxcrossref{FrameTransform}}}}}
\pysigstopsignatures
\sphinxAtStartPar
Create a transform from a quaternion and position.
\begin{quote}\begin{description}
\sphinxlineitem{Parameters}
\sphinxAtStartPar
\sphinxstyleliteralstrong{\sphinxupquote{quatpos}} (\sphinxstyleliteralemphasis{\sphinxupquote{np.ndarray}}) \textendash{} A 7\sphinxhyphen{}element array, with the first 4 elements being the quaternion, and the last 3 being the position.

\sphinxlineitem{Returns}
\sphinxAtStartPar
The transform.

\sphinxlineitem{Return type}
\sphinxAtStartPar
{\hyperref[\detokenize{deepdrr.geo:deepdrr.geo.FrameTransform}]{\sphinxcrossref{FrameTransform}}}

\end{description}\end{quote}

\end{fulllineitems}

\index{from\_rotation() (deepdrr.geo.FrameTransform class method)@\spxentry{from\_rotation()}\spxextra{deepdrr.geo.FrameTransform class method}}

\begin{fulllineitems}
\phantomsection\label{\detokenize{deepdrr.geo:deepdrr.geo.FrameTransform.from_rotation}}
\pysigstartsignatures
\pysiglinewithargsret{\sphinxbfcode{\sphinxupquote{classmethod\DUrole{w,w}{  }}}\sphinxbfcode{\sphinxupquote{from\_rotation}}}{\sphinxparam{\DUrole{n,n}{rotation}\DUrole{p,p}{:}\DUrole{w,w}{  }\DUrole{n,n}{{\hyperref[\detokenize{deepdrr.geo:deepdrr.geo.Rotation}]{\sphinxcrossref{Rotation}}}\DUrole{w,w}{  }\DUrole{p,p}{|}\DUrole{w,w}{  }ndarray}}}{{ $\rightarrow$ {\hyperref[\detokenize{deepdrr.geo:deepdrr.geo.core.FrameTransform}]{\sphinxcrossref{FrameTransform}}}}}
\pysigstopsignatures
\sphinxAtStartPar
Wrapper around from\_rt.

\end{fulllineitems}

\index{from\_rt() (deepdrr.geo.FrameTransform class method)@\spxentry{from\_rt()}\spxextra{deepdrr.geo.FrameTransform class method}}

\begin{fulllineitems}
\phantomsection\label{\detokenize{deepdrr.geo:deepdrr.geo.FrameTransform.from_rt}}
\pysigstartsignatures
\pysiglinewithargsret{\sphinxbfcode{\sphinxupquote{classmethod\DUrole{w,w}{  }}}\sphinxbfcode{\sphinxupquote{from\_rt}}}{\sphinxparam{\DUrole{n,n}{rotation}\DUrole{p,p}{:}\DUrole{w,w}{  }\DUrole{n,n}{{\hyperref[\detokenize{deepdrr.geo:deepdrr.geo.Rotation}]{\sphinxcrossref{Rotation}}}\DUrole{w,w}{  }\DUrole{p,p}{|}\DUrole{w,w}{  }ndarray\DUrole{w,w}{  }\DUrole{p,p}{|}\DUrole{w,w}{  }None}\DUrole{w,w}{  }\DUrole{o,o}{=}\DUrole{w,w}{  }\DUrole{default_value}{None}}\sphinxparamcomma \sphinxparam{\DUrole{n,n}{translation}\DUrole{p,p}{:}\DUrole{w,w}{  }\DUrole{n,n}{{\hyperref[\detokenize{deepdrr.geo:deepdrr.geo.core.Point3D}]{\sphinxcrossref{Point3D}}}\DUrole{w,w}{  }\DUrole{p,p}{|}\DUrole{w,w}{  }{\hyperref[\detokenize{deepdrr.geo:deepdrr.geo.core.Vector3D}]{\sphinxcrossref{Vector3D}}}\DUrole{w,w}{  }\DUrole{p,p}{|}\DUrole{w,w}{  }ndarray\DUrole{w,w}{  }\DUrole{p,p}{|}\DUrole{w,w}{  }None}\DUrole{w,w}{  }\DUrole{o,o}{=}\DUrole{w,w}{  }\DUrole{default_value}{None}}\sphinxparamcomma \sphinxparam{\DUrole{n,n}{dim}\DUrole{p,p}{:}\DUrole{w,w}{  }\DUrole{n,n}{int\DUrole{w,w}{  }\DUrole{p,p}{|}\DUrole{w,w}{  }None}\DUrole{w,w}{  }\DUrole{o,o}{=}\DUrole{w,w}{  }\DUrole{default_value}{None}}}{{ $\rightarrow$ {\hyperref[\detokenize{deepdrr.geo:deepdrr.geo.core.FrameTransform}]{\sphinxcrossref{FrameTransform}}}}}
\pysigstopsignatures
\sphinxAtStartPar
Make a frame translation from a rotation and translation, as {[}R,t{]}, where x’ = Rx + t.
\begin{quote}\begin{description}
\sphinxlineitem{Parameters}\begin{itemize}
\item {} 
\sphinxAtStartPar
\sphinxstyleliteralstrong{\sphinxupquote{rotation}} (\sphinxstyleliteralemphasis{\sphinxupquote{Optional}}\sphinxstyleliteralemphasis{\sphinxupquote{{[}}}\sphinxstyleliteralemphasis{\sphinxupquote{np.ndarray}}\sphinxstyleliteralemphasis{\sphinxupquote{{]}}}\sphinxstyleliteralemphasis{\sphinxupquote{, }}\sphinxstyleliteralemphasis{\sphinxupquote{optional}}) \textendash{} Rotation matrix. If None, uses the identity. Defaults to None.

\item {} 
\sphinxAtStartPar
\sphinxstyleliteralstrong{\sphinxupquote{translation}} \textendash{} Optional{[}Union{[}Point3D, np.ndarray{]}{]}: Translation of the transformation. If None, no translation. Defaults to None.

\item {} 
\sphinxAtStartPar
\sphinxstyleliteralstrong{\sphinxupquote{dim}} (\sphinxstyleliteralemphasis{\sphinxupquote{Optional}}\sphinxstyleliteralemphasis{\sphinxupquote{{[}}}\sphinxstyleliteralemphasis{\sphinxupquote{int}}\sphinxstyleliteralemphasis{\sphinxupquote{{]}}}\sphinxstyleliteralemphasis{\sphinxupquote{, }}\sphinxstyleliteralemphasis{\sphinxupquote{optional}}) \textendash{} Must be provided if both  Defaults to None.

\end{itemize}

\end{description}\end{quote}

\sphinxAtStartPar
If both args are None,
\begin{quote}\begin{description}
\sphinxlineitem{Returns}
\sphinxAtStartPar
The transformation \sphinxtitleref{F} such that \sphinxtitleref{F(x) = rotation @ x + translation}

\sphinxlineitem{Return type}
\sphinxAtStartPar
{\hyperref[\detokenize{deepdrr.geo:deepdrr.geo.FrameTransform}]{\sphinxcrossref{FrameTransform}}}

\end{description}\end{quote}

\end{fulllineitems}

\index{from\_scaling() (deepdrr.geo.FrameTransform class method)@\spxentry{from\_scaling()}\spxextra{deepdrr.geo.FrameTransform class method}}

\begin{fulllineitems}
\phantomsection\label{\detokenize{deepdrr.geo:deepdrr.geo.FrameTransform.from_scaling}}
\pysigstartsignatures
\pysiglinewithargsret{\sphinxbfcode{\sphinxupquote{classmethod\DUrole{w,w}{  }}}\sphinxbfcode{\sphinxupquote{from\_scaling}}}{\sphinxparam{\DUrole{n,n}{scaling}\DUrole{p,p}{:}\DUrole{w,w}{  }\DUrole{n,n}{int\DUrole{w,w}{  }\DUrole{p,p}{|}\DUrole{w,w}{  }float}}\sphinxparamcomma \sphinxparam{\DUrole{n,n}{dim}\DUrole{p,p}{:}\DUrole{w,w}{  }\DUrole{n,n}{int}\DUrole{w,w}{  }\DUrole{o,o}{=}\DUrole{w,w}{  }\DUrole{default_value}{3}}}{{ $\rightarrow$ {\hyperref[\detokenize{deepdrr.geo:deepdrr.geo.core.FrameTransform}]{\sphinxcrossref{FrameTransform}}}}}
\pysigstopsignatures
\sphinxAtStartPar
Create a frame based on scaling dimensions uniformly.
\begin{quote}\begin{description}
\sphinxlineitem{Parameters}\begin{itemize}
\item {} 
\sphinxAtStartPar
\sphinxstyleliteralstrong{\sphinxupquote{cls}} (\sphinxstyleliteralemphasis{\sphinxupquote{Type}}\sphinxstyleliteralemphasis{\sphinxupquote{{[}}}{\hyperref[\detokenize{deepdrr.geo:deepdrr.geo.FrameTransform}]{\sphinxcrossref{\sphinxstyleliteralemphasis{\sphinxupquote{FrameTransform}}}}}\sphinxstyleliteralemphasis{\sphinxupquote{{]}}}) \textendash{} the class.

\item {} 
\sphinxAtStartPar
\sphinxstyleliteralstrong{\sphinxupquote{dim}} (\sphinxstyleliteralemphasis{\sphinxupquote{int}}\sphinxstyleliteralemphasis{\sphinxupquote{, }}\sphinxstyleliteralemphasis{\sphinxupquote{optional}}) \textendash{} the dimension of the frame. Defaults to 3.

\end{itemize}

\sphinxlineitem{Return type}
\sphinxAtStartPar
{\hyperref[\detokenize{deepdrr.geo:deepdrr.geo.FrameTransform}]{\sphinxcrossref{FrameTransform}}}

\end{description}\end{quote}

\end{fulllineitems}

\index{from\_translation() (deepdrr.geo.FrameTransform class method)@\spxentry{from\_translation()}\spxextra{deepdrr.geo.FrameTransform class method}}

\begin{fulllineitems}
\phantomsection\label{\detokenize{deepdrr.geo:deepdrr.geo.FrameTransform.from_translation}}
\pysigstartsignatures
\pysiglinewithargsret{\sphinxbfcode{\sphinxupquote{classmethod\DUrole{w,w}{  }}}\sphinxbfcode{\sphinxupquote{from\_translation}}}{\sphinxparam{\DUrole{n,n}{translation}\DUrole{p,p}{:}\DUrole{w,w}{  }\DUrole{n,n}{ndarray}}}{{ $\rightarrow$ {\hyperref[\detokenize{deepdrr.geo:deepdrr.geo.core.FrameTransform}]{\sphinxcrossref{FrameTransform}}}}}
\pysigstopsignatures
\sphinxAtStartPar
Wrapper around from\_rt.

\end{fulllineitems}

\index{get\_point() (deepdrr.geo.FrameTransform method)@\spxentry{get\_point()}\spxextra{deepdrr.geo.FrameTransform method}}

\begin{fulllineitems}
\phantomsection\label{\detokenize{deepdrr.geo:deepdrr.geo.FrameTransform.get_point}}
\pysigstartsignatures
\pysiglinewithargsret{\sphinxbfcode{\sphinxupquote{get\_point}}}{\sphinxparam{\DUrole{n,n}{point}\DUrole{p,p}{:}\DUrole{w,w}{  }\DUrole{n,n}{{\hyperref[\detokenize{deepdrr.geo:deepdrr.geo.core.Point}]{\sphinxcrossref{Point}}}}}}{{ $\rightarrow$ {\hyperref[\detokenize{deepdrr.geo:deepdrr.geo.core.Point}]{\sphinxcrossref{Point}}}}}
\pysigstopsignatures
\sphinxAtStartPar
Transform a point.
\begin{quote}\begin{description}
\sphinxlineitem{Parameters}
\sphinxAtStartPar
\sphinxstyleliteralstrong{\sphinxupquote{point}} ({\hyperref[\detokenize{deepdrr.geo:deepdrr.geo.Point}]{\sphinxcrossref{\sphinxstyleliteralemphasis{\sphinxupquote{Point}}}}}) \textendash{} The point to transform.

\sphinxlineitem{Returns}
\sphinxAtStartPar
The transformed point.

\sphinxlineitem{Return type}
\sphinxAtStartPar
{\hyperref[\detokenize{deepdrr.geo:deepdrr.geo.Point}]{\sphinxcrossref{Point}}}

\end{description}\end{quote}

\end{fulllineitems}

\index{i (deepdrr.geo.FrameTransform property)@\spxentry{i}\spxextra{deepdrr.geo.FrameTransform property}}

\begin{fulllineitems}
\phantomsection\label{\detokenize{deepdrr.geo:deepdrr.geo.FrameTransform.i}}
\pysigstartsignatures
\pysigline{\sphinxbfcode{\sphinxupquote{property\DUrole{w,w}{  }}}\sphinxbfcode{\sphinxupquote{i}}\sphinxbfcode{\sphinxupquote{\DUrole{p,p}{:}\DUrole{w,w}{  }{\hyperref[\detokenize{deepdrr.geo:deepdrr.geo.core.Vector}]{\sphinxcrossref{Vector}}}}}}
\pysigstopsignatures
\end{fulllineitems}

\index{identity() (deepdrr.geo.FrameTransform class method)@\spxentry{identity()}\spxextra{deepdrr.geo.FrameTransform class method}}

\begin{fulllineitems}
\phantomsection\label{\detokenize{deepdrr.geo:deepdrr.geo.FrameTransform.identity}}
\pysigstartsignatures
\pysiglinewithargsret{\sphinxbfcode{\sphinxupquote{classmethod\DUrole{w,w}{  }}}\sphinxbfcode{\sphinxupquote{identity}}}{\sphinxparam{\DUrole{n,n}{dim}\DUrole{p,p}{:}\DUrole{w,w}{  }\DUrole{n,n}{int}\DUrole{w,w}{  }\DUrole{o,o}{=}\DUrole{w,w}{  }\DUrole{default_value}{3}}}{{ $\rightarrow$ {\hyperref[\detokenize{deepdrr.geo:deepdrr.geo.core.FrameTransform}]{\sphinxcrossref{FrameTransform}}}}}
\pysigstopsignatures
\sphinxAtStartPar
Get the identity FrameTransform.

\end{fulllineitems}

\index{inv (deepdrr.geo.FrameTransform property)@\spxentry{inv}\spxextra{deepdrr.geo.FrameTransform property}}

\begin{fulllineitems}
\phantomsection\label{\detokenize{deepdrr.geo:deepdrr.geo.FrameTransform.inv}}
\pysigstartsignatures
\pysigline{\sphinxbfcode{\sphinxupquote{property\DUrole{w,w}{  }}}\sphinxbfcode{\sphinxupquote{inv}}}
\pysigstopsignatures
\sphinxAtStartPar
Get the inverse of the Transform.
\begin{quote}\begin{description}
\sphinxlineitem{Returns}
\sphinxAtStartPar
a Transform (or subclass) that is well\sphinxhyphen{}defined as the inverse of this transform.

\sphinxlineitem{Return type}
\sphinxAtStartPar
({\hyperref[\detokenize{deepdrr.geo:deepdrr.geo.Transform}]{\sphinxcrossref{Transform}}})

\sphinxlineitem{Raises}
\sphinxAtStartPar
\sphinxstyleliteralstrong{\sphinxupquote{NotImplementedError}} \textendash{} if \_inv is None and method is not overriden.

\end{description}\end{quote}

\end{fulllineitems}

\index{j (deepdrr.geo.FrameTransform property)@\spxentry{j}\spxextra{deepdrr.geo.FrameTransform property}}

\begin{fulllineitems}
\phantomsection\label{\detokenize{deepdrr.geo:deepdrr.geo.FrameTransform.j}}
\pysigstartsignatures
\pysigline{\sphinxbfcode{\sphinxupquote{property\DUrole{w,w}{  }}}\sphinxbfcode{\sphinxupquote{j}}\sphinxbfcode{\sphinxupquote{\DUrole{p,p}{:}\DUrole{w,w}{  }{\hyperref[\detokenize{deepdrr.geo:deepdrr.geo.core.Vector}]{\sphinxcrossref{Vector}}}}}}
\pysigstopsignatures
\end{fulllineitems}

\index{k (deepdrr.geo.FrameTransform property)@\spxentry{k}\spxextra{deepdrr.geo.FrameTransform property}}

\begin{fulllineitems}
\phantomsection\label{\detokenize{deepdrr.geo:deepdrr.geo.FrameTransform.k}}
\pysigstartsignatures
\pysigline{\sphinxbfcode{\sphinxupquote{property\DUrole{w,w}{  }}}\sphinxbfcode{\sphinxupquote{k}}\sphinxbfcode{\sphinxupquote{\DUrole{p,p}{:}\DUrole{w,w}{  }{\hyperref[\detokenize{deepdrr.geo:deepdrr.geo.core.Vector}]{\sphinxcrossref{Vector}}}}}}
\pysigstopsignatures
\end{fulllineitems}

\index{load() (deepdrr.geo.FrameTransform class method)@\spxentry{load()}\spxextra{deepdrr.geo.FrameTransform class method}}

\begin{fulllineitems}
\phantomsection\label{\detokenize{deepdrr.geo:deepdrr.geo.FrameTransform.load}}
\pysigstartsignatures
\pysiglinewithargsret{\sphinxbfcode{\sphinxupquote{classmethod\DUrole{w,w}{  }}}\sphinxbfcode{\sphinxupquote{load}}}{\sphinxparam{\DUrole{n,n}{path}\DUrole{p,p}{:}\DUrole{w,w}{  }\DUrole{n,n}{str\DUrole{w,w}{  }\DUrole{p,p}{|}\DUrole{w,w}{  }Path}}}{{ $\rightarrow$ None}}
\pysigstopsignatures
\sphinxAtStartPar
Load the transform from a file.
\begin{quote}\begin{description}
\sphinxlineitem{Parameters}
\sphinxAtStartPar
\sphinxstyleliteralstrong{\sphinxupquote{path}} (\sphinxstyleliteralemphasis{\sphinxupquote{Union}}\sphinxstyleliteralemphasis{\sphinxupquote{{[}}}\sphinxstyleliteralemphasis{\sphinxupquote{str}}\sphinxstyleliteralemphasis{\sphinxupquote{, }}\sphinxstyleliteralemphasis{\sphinxupquote{Path}}\sphinxstyleliteralemphasis{\sphinxupquote{{]}}}) \textendash{} path to load the transform from.

\end{description}\end{quote}

\end{fulllineitems}

\index{load\_txt() (deepdrr.geo.FrameTransform class method)@\spxentry{load\_txt()}\spxextra{deepdrr.geo.FrameTransform class method}}

\begin{fulllineitems}
\phantomsection\label{\detokenize{deepdrr.geo:deepdrr.geo.FrameTransform.load_txt}}
\pysigstartsignatures
\pysiglinewithargsret{\sphinxbfcode{\sphinxupquote{classmethod\DUrole{w,w}{  }}}\sphinxbfcode{\sphinxupquote{load\_txt}}}{\sphinxparam{\DUrole{n,n}{path}\DUrole{p,p}{:}\DUrole{w,w}{  }\DUrole{n,n}{str\DUrole{w,w}{  }\DUrole{p,p}{|}\DUrole{w,w}{  }Path}}}{{ $\rightarrow$ {\hyperref[\detokenize{deepdrr.geo:deepdrr.geo.core.FrameTransform}]{\sphinxcrossref{FrameTransform}}}}}
\pysigstopsignatures
\sphinxAtStartPar
Load a transform from a text file.
\begin{quote}\begin{description}
\sphinxlineitem{Parameters}
\sphinxAtStartPar
\sphinxstyleliteralstrong{\sphinxupquote{path}} (\sphinxstyleliteralemphasis{\sphinxupquote{Union}}\sphinxstyleliteralemphasis{\sphinxupquote{{[}}}\sphinxstyleliteralemphasis{\sphinxupquote{str}}\sphinxstyleliteralemphasis{\sphinxupquote{, }}\sphinxstyleliteralemphasis{\sphinxupquote{Path}}\sphinxstyleliteralemphasis{\sphinxupquote{{]}}}) \textendash{} path to load the transform from.

\sphinxlineitem{Returns}
\sphinxAtStartPar
the loaded transform.

\sphinxlineitem{Return type}
\sphinxAtStartPar
{\hyperref[\detokenize{deepdrr.geo:deepdrr.geo.FrameTransform}]{\sphinxcrossref{FrameTransform}}}

\end{description}\end{quote}

\end{fulllineitems}

\index{o (deepdrr.geo.FrameTransform property)@\spxentry{o}\spxextra{deepdrr.geo.FrameTransform property}}

\begin{fulllineitems}
\phantomsection\label{\detokenize{deepdrr.geo:deepdrr.geo.FrameTransform.o}}
\pysigstartsignatures
\pysigline{\sphinxbfcode{\sphinxupquote{property\DUrole{w,w}{  }}}\sphinxbfcode{\sphinxupquote{o}}}
\pysigstopsignatures
\sphinxAtStartPar
If this is the A\_from\_B transform, return the origin of frame B in frame A.

\end{fulllineitems}

\index{save() (deepdrr.geo.FrameTransform method)@\spxentry{save()}\spxextra{deepdrr.geo.FrameTransform method}}

\begin{fulllineitems}
\phantomsection\label{\detokenize{deepdrr.geo:deepdrr.geo.FrameTransform.save}}
\pysigstartsignatures
\pysiglinewithargsret{\sphinxbfcode{\sphinxupquote{save}}}{\sphinxparam{\DUrole{n,n}{path}\DUrole{p,p}{:}\DUrole{w,w}{  }\DUrole{n,n}{str\DUrole{w,w}{  }\DUrole{p,p}{|}\DUrole{w,w}{  }Path}}}{{ $\rightarrow$ None}}
\pysigstopsignatures
\sphinxAtStartPar
Save the transform to a file.
\begin{quote}\begin{description}
\sphinxlineitem{Parameters}
\sphinxAtStartPar
\sphinxstyleliteralstrong{\sphinxupquote{path}} (\sphinxstyleliteralemphasis{\sphinxupquote{Union}}\sphinxstyleliteralemphasis{\sphinxupquote{{[}}}\sphinxstyleliteralemphasis{\sphinxupquote{str}}\sphinxstyleliteralemphasis{\sphinxupquote{, }}\sphinxstyleliteralemphasis{\sphinxupquote{Path}}\sphinxstyleliteralemphasis{\sphinxupquote{{]}}}) \textendash{} path to save the transform to.

\end{description}\end{quote}

\end{fulllineitems}

\index{save\_txt() (deepdrr.geo.FrameTransform method)@\spxentry{save\_txt()}\spxextra{deepdrr.geo.FrameTransform method}}

\begin{fulllineitems}
\phantomsection\label{\detokenize{deepdrr.geo:deepdrr.geo.FrameTransform.save_txt}}
\pysigstartsignatures
\pysiglinewithargsret{\sphinxbfcode{\sphinxupquote{save\_txt}}}{\sphinxparam{\DUrole{n,n}{path}\DUrole{p,p}{:}\DUrole{w,w}{  }\DUrole{n,n}{str\DUrole{w,w}{  }\DUrole{p,p}{|}\DUrole{w,w}{  }Path}}}{{ $\rightarrow$ None}}
\pysigstopsignatures
\sphinxAtStartPar
Save the transform to a text file.
\begin{quote}\begin{description}
\sphinxlineitem{Parameters}
\sphinxAtStartPar
\sphinxstyleliteralstrong{\sphinxupquote{path}} (\sphinxstyleliteralemphasis{\sphinxupquote{Union}}\sphinxstyleliteralemphasis{\sphinxupquote{{[}}}\sphinxstyleliteralemphasis{\sphinxupquote{str}}\sphinxstyleliteralemphasis{\sphinxupquote{, }}\sphinxstyleliteralemphasis{\sphinxupquote{Path}}\sphinxstyleliteralemphasis{\sphinxupquote{{]}}}) \textendash{} path to save the transform to.

\end{description}\end{quote}

\end{fulllineitems}

\index{t (deepdrr.geo.FrameTransform property)@\spxentry{t}\spxextra{deepdrr.geo.FrameTransform property}}

\begin{fulllineitems}
\phantomsection\label{\detokenize{deepdrr.geo:deepdrr.geo.FrameTransform.t}}
\pysigstartsignatures
\pysigline{\sphinxbfcode{\sphinxupquote{property\DUrole{w,w}{  }}}\sphinxbfcode{\sphinxupquote{t}}}
\pysigstopsignatures
\end{fulllineitems}

\index{toarray() (deepdrr.geo.FrameTransform method)@\spxentry{toarray()}\spxextra{deepdrr.geo.FrameTransform method}}

\begin{fulllineitems}
\phantomsection\label{\detokenize{deepdrr.geo:deepdrr.geo.FrameTransform.toarray}}
\pysigstartsignatures
\pysiglinewithargsret{\sphinxbfcode{\sphinxupquote{toarray}}}{}{}
\pysigstopsignatures
\sphinxAtStartPar
Return the transform as a 3x4 numpy array.

\sphinxAtStartPar
This is different from calling np.array() on the transform, which returns a 4x4 array.

\end{fulllineitems}

\index{tostring() (deepdrr.geo.FrameTransform method)@\spxentry{tostring()}\spxextra{deepdrr.geo.FrameTransform method}}

\begin{fulllineitems}
\phantomsection\label{\detokenize{deepdrr.geo:deepdrr.geo.FrameTransform.tostring}}
\pysigstartsignatures
\pysiglinewithargsret{\sphinxbfcode{\sphinxupquote{tostring}}}{}{}
\pysigstopsignatures
\end{fulllineitems}

\index{transform\_points() (deepdrr.geo.FrameTransform method)@\spxentry{transform\_points()}\spxextra{deepdrr.geo.FrameTransform method}}

\begin{fulllineitems}
\phantomsection\label{\detokenize{deepdrr.geo:deepdrr.geo.FrameTransform.transform_points}}
\pysigstartsignatures
\pysiglinewithargsret{\sphinxbfcode{\sphinxupquote{transform\_points}}}{\sphinxparam{\DUrole{n,n}{points}\DUrole{p,p}{:}\DUrole{w,w}{  }\DUrole{n,n}{ndarray}}}{{ $\rightarrow$ ndarray}}
\pysigstopsignatures
\sphinxAtStartPar
Transform a set of points.
\begin{quote}\begin{description}
\sphinxlineitem{Parameters}
\sphinxAtStartPar
\sphinxstyleliteralstrong{\sphinxupquote{points}} (\sphinxstyleliteralemphasis{\sphinxupquote{np.ndarray}}) \textendash{} {[}N, D{]} array of nonhomogeneous points.

\sphinxlineitem{Returns}
\sphinxAtStartPar
The transformed points.

\sphinxlineitem{Return type}
\sphinxAtStartPar
np.ndarray

\end{description}\end{quote}

\end{fulllineitems}


\end{fulllineitems}

\index{HomogeneousObject (class in deepdrr.geo)@\spxentry{HomogeneousObject}\spxextra{class in deepdrr.geo}}

\begin{fulllineitems}
\phantomsection\label{\detokenize{deepdrr.geo:deepdrr.geo.HomogeneousObject}}
\pysigstartsignatures
\pysiglinewithargsret{\sphinxbfcode{\sphinxupquote{class\DUrole{w,w}{  }}}\sphinxcode{\sphinxupquote{deepdrr.geo.}}\sphinxbfcode{\sphinxupquote{HomogeneousObject}}}{\sphinxparam{\DUrole{n,n}{data}\DUrole{p,p}{:}\DUrole{w,w}{  }\DUrole{n,n}{ndarray}}}{}
\pysigstopsignatures
\sphinxAtStartPar
Bases: \sphinxcode{\sphinxupquote{ABC}}

\sphinxAtStartPar
Any of the objects that rely on homogeneous transforms, all of which wrap a single array called \sphinxtitleref{data}.
\index{copy() (deepdrr.geo.HomogeneousObject method)@\spxentry{copy()}\spxextra{deepdrr.geo.HomogeneousObject method}}

\begin{fulllineitems}
\phantomsection\label{\detokenize{deepdrr.geo:deepdrr.geo.HomogeneousObject.copy}}
\pysigstartsignatures
\pysiglinewithargsret{\sphinxbfcode{\sphinxupquote{copy}}}{}{{ $\rightarrow$ Self}}
\pysigstopsignatures
\end{fulllineitems}

\index{data (deepdrr.geo.HomogeneousObject attribute)@\spxentry{data}\spxextra{deepdrr.geo.HomogeneousObject attribute}}

\begin{fulllineitems}
\phantomsection\label{\detokenize{deepdrr.geo:deepdrr.geo.HomogeneousObject.data}}
\pysigstartsignatures
\pysigline{\sphinxbfcode{\sphinxupquote{data}}\sphinxbfcode{\sphinxupquote{\DUrole{p,p}{:}\DUrole{w,w}{  }ndarray}}}
\pysigstopsignatures
\end{fulllineitems}

\index{dim (deepdrr.geo.HomogeneousObject property)@\spxentry{dim}\spxextra{deepdrr.geo.HomogeneousObject property}}

\begin{fulllineitems}
\phantomsection\label{\detokenize{deepdrr.geo:deepdrr.geo.HomogeneousObject.dim}}
\pysigstartsignatures
\pysigline{\sphinxbfcode{\sphinxupquote{abstract\DUrole{w,w}{  }property\DUrole{w,w}{  }}}\sphinxbfcode{\sphinxupquote{dim}}\sphinxbfcode{\sphinxupquote{\DUrole{p,p}{:}\DUrole{w,w}{  }int}}}
\pysigstopsignatures
\sphinxAtStartPar
Get the dimension of the space the object lives in. For transforms, this is the OUTPUT dim.

\end{fulllineitems}

\index{dtype (deepdrr.geo.HomogeneousObject attribute)@\spxentry{dtype}\spxextra{deepdrr.geo.HomogeneousObject attribute}}

\begin{fulllineitems}
\phantomsection\label{\detokenize{deepdrr.geo:deepdrr.geo.HomogeneousObject.dtype}}
\pysigstartsignatures
\pysigline{\sphinxbfcode{\sphinxupquote{dtype}}}
\pysigstopsignatures
\sphinxAtStartPar
alias of \sphinxcode{\sphinxupquote{float32}}

\end{fulllineitems}

\index{from\_array() (deepdrr.geo.HomogeneousObject class method)@\spxentry{from\_array()}\spxextra{deepdrr.geo.HomogeneousObject class method}}

\begin{fulllineitems}
\phantomsection\label{\detokenize{deepdrr.geo:deepdrr.geo.HomogeneousObject.from_array}}
\pysigstartsignatures
\pysiglinewithargsret{\sphinxbfcode{\sphinxupquote{classmethod\DUrole{w,w}{  }}}\sphinxbfcode{\sphinxupquote{from\_array}}}{\sphinxparam{\DUrole{n,n}{x}\DUrole{p,p}{:}\DUrole{w,w}{  }\DUrole{n,n}{ndarray}}}{{ $\rightarrow$ T}}
\pysigstopsignatures
\sphinxAtStartPar
Create a homogeneous object from its non\sphinxhyphen{}homogeous representation as an array.

\end{fulllineitems}

\index{get\_config() (deepdrr.geo.HomogeneousObject method)@\spxentry{get\_config()}\spxextra{deepdrr.geo.HomogeneousObject method}}

\begin{fulllineitems}
\phantomsection\label{\detokenize{deepdrr.geo:deepdrr.geo.HomogeneousObject.get_config}}
\pysigstartsignatures
\pysiglinewithargsret{\sphinxbfcode{\sphinxupquote{get\_config}}}{}{{ $\rightarrow$ dict}}
\pysigstopsignatures
\sphinxAtStartPar
Get a config dict with the data in this object.

\end{fulllineitems}

\index{get\_data() (deepdrr.geo.HomogeneousObject method)@\spxentry{get\_data()}\spxextra{deepdrr.geo.HomogeneousObject method}}

\begin{fulllineitems}
\phantomsection\label{\detokenize{deepdrr.geo:deepdrr.geo.HomogeneousObject.get_data}}
\pysigstartsignatures
\pysiglinewithargsret{\sphinxbfcode{\sphinxupquote{get\_data}}}{}{{ $\rightarrow$ ndarray}}
\pysigstopsignatures
\end{fulllineitems}

\index{shape (deepdrr.geo.HomogeneousObject property)@\spxentry{shape}\spxextra{deepdrr.geo.HomogeneousObject property}}

\begin{fulllineitems}
\phantomsection\label{\detokenize{deepdrr.geo:deepdrr.geo.HomogeneousObject.shape}}
\pysigstartsignatures
\pysigline{\sphinxbfcode{\sphinxupquote{property\DUrole{w,w}{  }}}\sphinxbfcode{\sphinxupquote{shape}}\sphinxbfcode{\sphinxupquote{\DUrole{p,p}{:}\DUrole{w,w}{  }Tuple\DUrole{p,p}{{[}}int\DUrole{p,p}{,}\DUrole{w,w}{  }\DUrole{p,p}{...}\DUrole{p,p}{{]}}}}}
\pysigstopsignatures
\end{fulllineitems}

\index{tolist() (deepdrr.geo.HomogeneousObject method)@\spxentry{tolist()}\spxextra{deepdrr.geo.HomogeneousObject method}}

\begin{fulllineitems}
\phantomsection\label{\detokenize{deepdrr.geo:deepdrr.geo.HomogeneousObject.tolist}}
\pysigstartsignatures
\pysiglinewithargsret{\sphinxbfcode{\sphinxupquote{tolist}}}{}{{ $\rightarrow$ List}}
\pysigstopsignatures
\sphinxAtStartPar
Get a json\sphinxhyphen{}save list with the data in this object.

\end{fulllineitems}


\end{fulllineitems}

\index{HyperPlane (class in deepdrr.geo)@\spxentry{HyperPlane}\spxextra{class in deepdrr.geo}}

\begin{fulllineitems}
\phantomsection\label{\detokenize{deepdrr.geo:deepdrr.geo.HyperPlane}}
\pysigstartsignatures
\pysiglinewithargsret{\sphinxbfcode{\sphinxupquote{class\DUrole{w,w}{  }}}\sphinxcode{\sphinxupquote{deepdrr.geo.}}\sphinxbfcode{\sphinxupquote{HyperPlane}}}{\sphinxparam{\DUrole{n,n}{data}\DUrole{p,p}{:}\DUrole{w,w}{  }\DUrole{n,n}{ndarray}}}{}
\pysigstopsignatures
\sphinxAtStartPar
Bases: {\hyperref[\detokenize{deepdrr.geo:deepdrr.geo.core.HasLocationAndDirection}]{\sphinxcrossref{\sphinxcode{\sphinxupquote{HasLocationAndDirection}}}}}, {\hyperref[\detokenize{deepdrr.geo:deepdrr.geo.core.Meetable}]{\sphinxcrossref{\sphinxcode{\sphinxupquote{Meetable}}}}}

\sphinxAtStartPar
Represents a hyperplane in 2D (a line) or 3D (a plane).

\sphinxAtStartPar
Hyperplanes can be intersected with other hyperplanes or lower dimensional objects, but they are
not joinable.
\index{a (deepdrr.geo.HyperPlane property)@\spxentry{a}\spxextra{deepdrr.geo.HyperPlane property}}

\begin{fulllineitems}
\phantomsection\label{\detokenize{deepdrr.geo:deepdrr.geo.HyperPlane.a}}
\pysigstartsignatures
\pysigline{\sphinxbfcode{\sphinxupquote{property\DUrole{w,w}{  }}}\sphinxbfcode{\sphinxupquote{a}}\sphinxbfcode{\sphinxupquote{\DUrole{p,p}{:}\DUrole{w,w}{  }float}}}
\pysigstopsignatures
\sphinxAtStartPar
Get the coefficient of the first variable.
\begin{quote}\begin{description}
\sphinxlineitem{Returns}
\sphinxAtStartPar
The coefficient of the first variable.

\sphinxlineitem{Return type}
\sphinxAtStartPar
float

\end{description}\end{quote}

\end{fulllineitems}

\index{b (deepdrr.geo.HyperPlane property)@\spxentry{b}\spxextra{deepdrr.geo.HyperPlane property}}

\begin{fulllineitems}
\phantomsection\label{\detokenize{deepdrr.geo:deepdrr.geo.HyperPlane.b}}
\pysigstartsignatures
\pysigline{\sphinxbfcode{\sphinxupquote{property\DUrole{w,w}{  }}}\sphinxbfcode{\sphinxupquote{b}}\sphinxbfcode{\sphinxupquote{\DUrole{p,p}{:}\DUrole{w,w}{  }float}}}
\pysigstopsignatures
\sphinxAtStartPar
Get the coefficient of the second variable.
\begin{quote}\begin{description}
\sphinxlineitem{Returns}
\sphinxAtStartPar
The coefficient of the second variable.

\sphinxlineitem{Return type}
\sphinxAtStartPar
float

\end{description}\end{quote}

\end{fulllineitems}

\index{c (deepdrr.geo.HyperPlane property)@\spxentry{c}\spxextra{deepdrr.geo.HyperPlane property}}

\begin{fulllineitems}
\phantomsection\label{\detokenize{deepdrr.geo:deepdrr.geo.HyperPlane.c}}
\pysigstartsignatures
\pysigline{\sphinxbfcode{\sphinxupquote{property\DUrole{w,w}{  }}}\sphinxbfcode{\sphinxupquote{c}}\sphinxbfcode{\sphinxupquote{\DUrole{p,p}{:}\DUrole{w,w}{  }float}}}
\pysigstopsignatures
\sphinxAtStartPar
Get the coefficient of the third variable.
\begin{quote}\begin{description}
\sphinxlineitem{Returns}
\sphinxAtStartPar
The coefficient of the third variable.

\sphinxlineitem{Return type}
\sphinxAtStartPar
float

\end{description}\end{quote}

\end{fulllineitems}

\index{d (deepdrr.geo.HyperPlane property)@\spxentry{d}\spxextra{deepdrr.geo.HyperPlane property}}

\begin{fulllineitems}
\phantomsection\label{\detokenize{deepdrr.geo:deepdrr.geo.HyperPlane.d}}
\pysigstartsignatures
\pysigline{\sphinxbfcode{\sphinxupquote{property\DUrole{w,w}{  }}}\sphinxbfcode{\sphinxupquote{d}}\sphinxbfcode{\sphinxupquote{\DUrole{p,p}{:}\DUrole{w,w}{  }float}}}
\pysigstopsignatures
\sphinxAtStartPar
Get the constant term.
\begin{quote}\begin{description}
\sphinxlineitem{Returns}
\sphinxAtStartPar
The constant term.

\sphinxlineitem{Return type}
\sphinxAtStartPar
float

\end{description}\end{quote}

\end{fulllineitems}

\index{data (deepdrr.geo.HyperPlane attribute)@\spxentry{data}\spxextra{deepdrr.geo.HyperPlane attribute}}

\begin{fulllineitems}
\phantomsection\label{\detokenize{deepdrr.geo:deepdrr.geo.HyperPlane.data}}
\pysigstartsignatures
\pysigline{\sphinxbfcode{\sphinxupquote{data}}\sphinxbfcode{\sphinxupquote{\DUrole{p,p}{:}\DUrole{w,w}{  }ndarray}}}
\pysigstopsignatures
\end{fulllineitems}

\index{distance() (deepdrr.geo.HyperPlane method)@\spxentry{distance()}\spxextra{deepdrr.geo.HyperPlane method}}

\begin{fulllineitems}
\phantomsection\label{\detokenize{deepdrr.geo:deepdrr.geo.HyperPlane.distance}}
\pysigstartsignatures
\pysiglinewithargsret{\sphinxbfcode{\sphinxupquote{distance}}}{\sphinxparam{\DUrole{n,n}{p}\DUrole{p,p}{:}\DUrole{w,w}{  }\DUrole{n,n}{{\hyperref[\detokenize{deepdrr.geo:deepdrr.geo.core.Point}]{\sphinxcrossref{Point}}}}}}{{ $\rightarrow$ float}}
\pysigstopsignatures
\sphinxAtStartPar
Get the distance of the point to the hyperplane.
\begin{quote}\begin{description}
\sphinxlineitem{Parameters}
\sphinxAtStartPar
\sphinxstyleliteralstrong{\sphinxupquote{p}} ({\hyperref[\detokenize{deepdrr.geo:deepdrr.geo.Point}]{\sphinxcrossref{\sphinxstyleliteralemphasis{\sphinxupquote{Point}}}}}) \textendash{} the point to evaluate at.

\sphinxlineitem{Returns}
\sphinxAtStartPar
the distance of the point to the hyperplane.

\sphinxlineitem{Return type}
\sphinxAtStartPar
float

\end{description}\end{quote}

\end{fulllineitems}

\index{evaluate() (deepdrr.geo.HyperPlane method)@\spxentry{evaluate()}\spxextra{deepdrr.geo.HyperPlane method}}

\begin{fulllineitems}
\phantomsection\label{\detokenize{deepdrr.geo:deepdrr.geo.HyperPlane.evaluate}}
\pysigstartsignatures
\pysiglinewithargsret{\sphinxbfcode{\sphinxupquote{evaluate}}}{\sphinxparam{\DUrole{n,n}{p}\DUrole{p,p}{:}\DUrole{w,w}{  }\DUrole{n,n}{{\hyperref[\detokenize{deepdrr.geo:deepdrr.geo.core.Point}]{\sphinxcrossref{Point}}}}}}{{ $\rightarrow$ float}}
\pysigstopsignatures
\sphinxAtStartPar
Evaluate the hyperplane at the given point.

\sphinxAtStartPar
The sign of this value tells you which side of the hyperplane the point is on.
\begin{quote}\begin{description}
\sphinxlineitem{Parameters}
\sphinxAtStartPar
\sphinxstyleliteralstrong{\sphinxupquote{p}} ({\hyperref[\detokenize{deepdrr.geo:deepdrr.geo.Point}]{\sphinxcrossref{\sphinxstyleliteralemphasis{\sphinxupquote{Point}}}}}) \textendash{} the point to evaluate at.

\sphinxlineitem{Returns}
\sphinxAtStartPar
the value of the hyperplane at the given point.

\sphinxlineitem{Return type}
\sphinxAtStartPar
float

\end{description}\end{quote}

\end{fulllineitems}

\index{get\_normal() (deepdrr.geo.HyperPlane method)@\spxentry{get\_normal()}\spxextra{deepdrr.geo.HyperPlane method}}

\begin{fulllineitems}
\phantomsection\label{\detokenize{deepdrr.geo:deepdrr.geo.HyperPlane.get_normal}}
\pysigstartsignatures
\pysiglinewithargsret{\sphinxbfcode{\sphinxupquote{get\_normal}}}{}{{ $\rightarrow$ {\hyperref[\detokenize{deepdrr.geo:deepdrr.geo.core.Vector3D}]{\sphinxcrossref{Vector3D}}}}}
\pysigstopsignatures
\sphinxAtStartPar
Get the normal vector of the plane.
\begin{quote}\begin{description}
\sphinxlineitem{Returns}
\sphinxAtStartPar
The normal vector of the plane.

\sphinxlineitem{Return type}
\sphinxAtStartPar
{\hyperref[\detokenize{deepdrr.geo:deepdrr.geo.Vector3D}]{\sphinxcrossref{Vector3D}}}

\end{description}\end{quote}

\end{fulllineitems}

\index{n (deepdrr.geo.HyperPlane property)@\spxentry{n}\spxextra{deepdrr.geo.HyperPlane property}}

\begin{fulllineitems}
\phantomsection\label{\detokenize{deepdrr.geo:deepdrr.geo.HyperPlane.n}}
\pysigstartsignatures
\pysigline{\sphinxbfcode{\sphinxupquote{property\DUrole{w,w}{  }}}\sphinxbfcode{\sphinxupquote{n}}\sphinxbfcode{\sphinxupquote{\DUrole{p,p}{:}\DUrole{w,w}{  }{\hyperref[\detokenize{deepdrr.geo:deepdrr.geo.core.Vector3D}]{\sphinxcrossref{Vector3D}}}}}}
\pysigstopsignatures
\end{fulllineitems}

\index{normal() (deepdrr.geo.HyperPlane method)@\spxentry{normal()}\spxextra{deepdrr.geo.HyperPlane method}}

\begin{fulllineitems}
\phantomsection\label{\detokenize{deepdrr.geo:deepdrr.geo.HyperPlane.normal}}
\pysigstartsignatures
\pysiglinewithargsret{\sphinxbfcode{\sphinxupquote{normal}}}{}{{ $\rightarrow$ {\hyperref[\detokenize{deepdrr.geo:deepdrr.geo.core.Vector3D}]{\sphinxcrossref{Vector3D}}}}}
\pysigstopsignatures
\end{fulllineitems}

\index{project() (deepdrr.geo.HyperPlane method)@\spxentry{project()}\spxextra{deepdrr.geo.HyperPlane method}}

\begin{fulllineitems}
\phantomsection\label{\detokenize{deepdrr.geo:deepdrr.geo.HyperPlane.project}}
\pysigstartsignatures
\pysiglinewithargsret{\sphinxbfcode{\sphinxupquote{project}}}{\sphinxparam{\DUrole{n,n}{p}\DUrole{p,p}{:}\DUrole{w,w}{  }\DUrole{n,n}{P}}}{{ $\rightarrow$ P}}
\pysigstopsignatures
\sphinxAtStartPar
Get the closest point on the hyperplane to p.
\begin{quote}\begin{description}
\sphinxlineitem{Parameters}
\sphinxAtStartPar
\sphinxstyleliteralstrong{\sphinxupquote{p}} ({\hyperref[\detokenize{deepdrr.geo:deepdrr.geo.Point}]{\sphinxcrossref{\sphinxstyleliteralemphasis{\sphinxupquote{Point}}}}}) \textendash{} The point to project.

\sphinxlineitem{Returns}
\sphinxAtStartPar
The closest point on the hyperplane to p.

\sphinxlineitem{Return type}
\sphinxAtStartPar
{\hyperref[\detokenize{deepdrr.geo:deepdrr.geo.Point}]{\sphinxcrossref{Point}}}

\end{description}\end{quote}

\end{fulllineitems}

\index{signed\_distance() (deepdrr.geo.HyperPlane method)@\spxentry{signed\_distance()}\spxextra{deepdrr.geo.HyperPlane method}}

\begin{fulllineitems}
\phantomsection\label{\detokenize{deepdrr.geo:deepdrr.geo.HyperPlane.signed_distance}}
\pysigstartsignatures
\pysiglinewithargsret{\sphinxbfcode{\sphinxupquote{signed\_distance}}}{\sphinxparam{\DUrole{n,n}{p}\DUrole{p,p}{:}\DUrole{w,w}{  }\DUrole{n,n}{{\hyperref[\detokenize{deepdrr.geo:deepdrr.geo.core.Point}]{\sphinxcrossref{Point}}}}}}{{ $\rightarrow$ float}}
\pysigstopsignatures
\sphinxAtStartPar
Get the signed distance from the given point to the hyperplane.
\begin{quote}\begin{description}
\sphinxlineitem{Parameters}
\sphinxAtStartPar
\sphinxstyleliteralstrong{\sphinxupquote{p}} ({\hyperref[\detokenize{deepdrr.geo:deepdrr.geo.Point}]{\sphinxcrossref{\sphinxstyleliteralemphasis{\sphinxupquote{Point}}}}}) \textendash{} the point to measure the distance from.

\sphinxlineitem{Returns}
\sphinxAtStartPar
the signed distance from the point to the hyperplane.

\sphinxlineitem{Return type}
\sphinxAtStartPar
float

\end{description}\end{quote}

\end{fulllineitems}


\end{fulllineitems}

\index{JoinError@\spxentry{JoinError}}

\begin{fulllineitems}
\phantomsection\label{\detokenize{deepdrr.geo:deepdrr.geo.JoinError}}
\pysigstartsignatures
\pysigline{\sphinxbfcode{\sphinxupquote{exception\DUrole{w,w}{  }}}\sphinxcode{\sphinxupquote{deepdrr.geo.}}\sphinxbfcode{\sphinxupquote{JoinError}}}
\pysigstopsignatures
\sphinxAtStartPar
Bases: \sphinxcode{\sphinxupquote{Exception}}

\sphinxAtStartPar
Represents an error when joining two primitives.

\end{fulllineitems}

\index{Line (class in deepdrr.geo)@\spxentry{Line}\spxextra{class in deepdrr.geo}}

\begin{fulllineitems}
\phantomsection\label{\detokenize{deepdrr.geo:deepdrr.geo.Line}}
\pysigstartsignatures
\pysiglinewithargsret{\sphinxbfcode{\sphinxupquote{class\DUrole{w,w}{  }}}\sphinxcode{\sphinxupquote{deepdrr.geo.}}\sphinxbfcode{\sphinxupquote{Line}}}{\sphinxparam{\DUrole{n,n}{data}\DUrole{p,p}{:}\DUrole{w,w}{  }\DUrole{n,n}{ndarray}}}{}
\pysigstopsignatures
\sphinxAtStartPar
Bases: {\hyperref[\detokenize{deepdrr.geo:deepdrr.geo.core.HasLocationAndDirection}]{\sphinxcrossref{\sphinxcode{\sphinxupquote{HasLocationAndDirection}}}}}, {\hyperref[\detokenize{deepdrr.geo:deepdrr.geo.core.Meetable}]{\sphinxcrossref{\sphinxcode{\sphinxupquote{Meetable}}}}}

\sphinxAtStartPar
Abstract parent class for lines and line\sphinxhyphen{}like objects.
\index{angle() (deepdrr.geo.Line method)@\spxentry{angle()}\spxextra{deepdrr.geo.Line method}}

\begin{fulllineitems}
\phantomsection\label{\detokenize{deepdrr.geo:deepdrr.geo.Line.angle}}
\pysigstartsignatures
\pysiglinewithargsret{\sphinxbfcode{\sphinxupquote{angle}}}{\sphinxparam{\DUrole{n,n}{other}\DUrole{p,p}{:}\DUrole{w,w}{  }\DUrole{n,n}{{\hyperref[\detokenize{deepdrr.geo:deepdrr.geo.hyperplane.Line}]{\sphinxcrossref{Line}}}\DUrole{w,w}{  }\DUrole{p,p}{|}\DUrole{w,w}{  }{\hyperref[\detokenize{deepdrr.geo:deepdrr.geo.core.Vector}]{\sphinxcrossref{Vector}}}}}}{{ $\rightarrow$ float}}
\pysigstopsignatures
\sphinxAtStartPar
Get the acute angle between the two lines.

\end{fulllineitems}

\index{as\_points() (deepdrr.geo.Line method)@\spxentry{as\_points()}\spxextra{deepdrr.geo.Line method}}

\begin{fulllineitems}
\phantomsection\label{\detokenize{deepdrr.geo:deepdrr.geo.Line.as_points}}
\pysigstartsignatures
\pysiglinewithargsret{\sphinxbfcode{\sphinxupquote{as\_points}}}{}{{ $\rightarrow$ Tuple\DUrole{p,p}{{[}}{\hyperref[\detokenize{deepdrr.geo:deepdrr.geo.core.Point2D}]{\sphinxcrossref{Point2D}}}\DUrole{p,p}{,}\DUrole{w,w}{  }{\hyperref[\detokenize{deepdrr.geo:deepdrr.geo.core.Point2D}]{\sphinxcrossref{Point2D}}}\DUrole{p,p}{{]}}}}
\pysiglinewithargsret{\sphinxbfcode{\sphinxupquote{as\_points}}}{}{{ $\rightarrow$ Tuple\DUrole{p,p}{{[}}{\hyperref[\detokenize{deepdrr.geo:deepdrr.geo.core.Point3D}]{\sphinxcrossref{Point3D}}}\DUrole{p,p}{,}\DUrole{w,w}{  }{\hyperref[\detokenize{deepdrr.geo:deepdrr.geo.core.Point3D}]{\sphinxcrossref{Point3D}}}\DUrole{p,p}{{]}}}}
\pysigstopsignatures
\sphinxAtStartPar
Get two points on the line.
\begin{quote}\begin{description}
\sphinxlineitem{Returns}
\sphinxAtStartPar
Two points on the line.

\sphinxlineitem{Return type}
\sphinxAtStartPar
Tuple{[}{\hyperref[\detokenize{deepdrr.geo:deepdrr.geo.Point}]{\sphinxcrossref{Point}}}, {\hyperref[\detokenize{deepdrr.geo:deepdrr.geo.Point}]{\sphinxcrossref{Point}}}{]}

\end{description}\end{quote}

\end{fulllineitems}

\index{data (deepdrr.geo.Line attribute)@\spxentry{data}\spxextra{deepdrr.geo.Line attribute}}

\begin{fulllineitems}
\phantomsection\label{\detokenize{deepdrr.geo:deepdrr.geo.Line.data}}
\pysigstartsignatures
\pysigline{\sphinxbfcode{\sphinxupquote{data}}\sphinxbfcode{\sphinxupquote{\DUrole{p,p}{:}\DUrole{w,w}{  }ndarray}}}
\pysigstopsignatures
\end{fulllineitems}

\index{distance() (deepdrr.geo.Line method)@\spxentry{distance()}\spxextra{deepdrr.geo.Line method}}

\begin{fulllineitems}
\phantomsection\label{\detokenize{deepdrr.geo:deepdrr.geo.Line.distance}}
\pysigstartsignatures
\pysiglinewithargsret{\sphinxbfcode{\sphinxupquote{distance}}}{\sphinxparam{\DUrole{n,n}{other}\DUrole{p,p}{:}\DUrole{w,w}{  }\DUrole{n,n}{{\hyperref[\detokenize{deepdrr.geo:deepdrr.geo.core.Point}]{\sphinxcrossref{Point}}}}}}{{ $\rightarrow$ float}}
\pysigstopsignatures
\sphinxAtStartPar
Get the distance from the line to another point.
\begin{quote}\begin{description}
\sphinxlineitem{Parameters}
\sphinxAtStartPar
\sphinxstyleliteralstrong{\sphinxupquote{other}} ({\hyperref[\detokenize{deepdrr.geo:deepdrr.geo.Point}]{\sphinxcrossref{\sphinxstyleliteralemphasis{\sphinxupquote{Point}}}}}) \textendash{} The point to which the distance is sought.

\sphinxlineitem{Returns}
\sphinxAtStartPar
The distance from the line to the other point.

\sphinxlineitem{Return type}
\sphinxAtStartPar
float

\end{description}\end{quote}

\end{fulllineitems}

\index{from\_point\_direction() (deepdrr.geo.Line class method)@\spxentry{from\_point\_direction()}\spxextra{deepdrr.geo.Line class method}}

\begin{fulllineitems}
\phantomsection\label{\detokenize{deepdrr.geo:deepdrr.geo.Line.from_point_direction}}
\pysigstartsignatures
\pysiglinewithargsret{\sphinxbfcode{\sphinxupquote{classmethod\DUrole{w,w}{  }}}\sphinxbfcode{\sphinxupquote{from\_point\_direction}}}{\sphinxparam{\DUrole{n,n}{p}\DUrole{p,p}{:}\DUrole{w,w}{  }\DUrole{n,n}{{\hyperref[\detokenize{deepdrr.geo:deepdrr.geo.core.Point}]{\sphinxcrossref{Point}}}}}\sphinxparamcomma \sphinxparam{\DUrole{n,n}{v}\DUrole{p,p}{:}\DUrole{w,w}{  }\DUrole{n,n}{{\hyperref[\detokenize{deepdrr.geo:deepdrr.geo.core.Vector}]{\sphinxcrossref{Vector}}}}}}{{ $\rightarrow$ {\hyperref[\detokenize{deepdrr.geo:deepdrr.geo.hyperplane.Line}]{\sphinxcrossref{Line}}}}}
\pysigstopsignatures
\sphinxAtStartPar
Construct a line from a point and a direction vector.
\begin{quote}\begin{description}
\sphinxlineitem{Parameters}\begin{itemize}
\item {} 
\sphinxAtStartPar
\sphinxstyleliteralstrong{\sphinxupquote{p}} ({\hyperref[\detokenize{deepdrr.geo:deepdrr.geo.Point}]{\sphinxcrossref{\sphinxstyleliteralemphasis{\sphinxupquote{Point}}}}}) \textendash{} The point on the line.

\item {} 
\sphinxAtStartPar
\sphinxstyleliteralstrong{\sphinxupquote{v}} ({\hyperref[\detokenize{deepdrr.geo:deepdrr.geo.Vector}]{\sphinxcrossref{\sphinxstyleliteralemphasis{\sphinxupquote{Vector}}}}}) \textendash{} The direction vector.

\end{itemize}

\sphinxlineitem{Returns}
\sphinxAtStartPar
The line through the point in the direction of the vector.

\sphinxlineitem{Return type}
\sphinxAtStartPar
{\hyperref[\detokenize{deepdrr.geo:deepdrr.geo.Line}]{\sphinxcrossref{Line}}}

\end{description}\end{quote}

\end{fulllineitems}

\index{project() (deepdrr.geo.Line method)@\spxentry{project()}\spxextra{deepdrr.geo.Line method}}

\begin{fulllineitems}
\phantomsection\label{\detokenize{deepdrr.geo:deepdrr.geo.Line.project}}
\pysigstartsignatures
\pysiglinewithargsret{\sphinxbfcode{\sphinxupquote{project}}}{\sphinxparam{\DUrole{n,n}{other}\DUrole{p,p}{:}\DUrole{w,w}{  }\DUrole{n,n}{{\hyperref[\detokenize{deepdrr.geo:deepdrr.geo.core.Point2D}]{\sphinxcrossref{Point2D}}}}}}{{ $\rightarrow$ {\hyperref[\detokenize{deepdrr.geo:deepdrr.geo.core.Point2D}]{\sphinxcrossref{Point2D}}}}}
\pysiglinewithargsret{\sphinxbfcode{\sphinxupquote{project}}}{\sphinxparam{\DUrole{n,n}{other}\DUrole{p,p}{:}\DUrole{w,w}{  }\DUrole{n,n}{{\hyperref[\detokenize{deepdrr.geo:deepdrr.geo.core.Point3D}]{\sphinxcrossref{Point3D}}}}}}{{ $\rightarrow$ {\hyperref[\detokenize{deepdrr.geo:deepdrr.geo.core.Point3D}]{\sphinxcrossref{Point3D}}}}}
\pysigstopsignatures
\sphinxAtStartPar
Get the closest point on the line to another point.
\begin{quote}\begin{description}
\sphinxlineitem{Parameters}
\sphinxAtStartPar
\sphinxstyleliteralstrong{\sphinxupquote{other}} ({\hyperref[\detokenize{deepdrr.geo:deepdrr.geo.Point}]{\sphinxcrossref{\sphinxstyleliteralemphasis{\sphinxupquote{Point}}}}}) \textendash{} The point to which the closest point is sought.

\sphinxlineitem{Returns}
\sphinxAtStartPar
The closest point on the line to the other point.

\sphinxlineitem{Return type}
\sphinxAtStartPar
{\hyperref[\detokenize{deepdrr.geo:deepdrr.geo.Point}]{\sphinxcrossref{Point}}}

\end{description}\end{quote}

\end{fulllineitems}


\end{fulllineitems}

\index{Line2D (class in deepdrr.geo)@\spxentry{Line2D}\spxextra{class in deepdrr.geo}}

\begin{fulllineitems}
\phantomsection\label{\detokenize{deepdrr.geo:deepdrr.geo.Line2D}}
\pysigstartsignatures
\pysiglinewithargsret{\sphinxbfcode{\sphinxupquote{class\DUrole{w,w}{  }}}\sphinxcode{\sphinxupquote{deepdrr.geo.}}\sphinxbfcode{\sphinxupquote{Line2D}}}{\sphinxparam{\DUrole{n,n}{data}\DUrole{p,p}{:}\DUrole{w,w}{  }\DUrole{n,n}{ndarray}}}{}
\pysigstopsignatures
\sphinxAtStartPar
Bases: {\hyperref[\detokenize{deepdrr.geo:deepdrr.geo.hyperplane.Line}]{\sphinxcrossref{\sphinxcode{\sphinxupquote{Line}}}}}, {\hyperref[\detokenize{deepdrr.geo:deepdrr.geo.hyperplane.HyperPlane}]{\sphinxcrossref{\sphinxcode{\sphinxupquote{HyperPlane}}}}}

\sphinxAtStartPar
Represents a line in 2D.

\sphinxAtStartPar
Consists of a 3\sphinxhyphen{}vector \(\mathbf{p} = [a, b, c]\) such that the line is all the points (x,y)
such that \(ax + by + c = 0\) or, alternatively, all the homogeneous points
\(\mathbf{x} = [x,y,w]\) such that \(p^T x = 0\).
\index{backproject() (deepdrr.geo.Line2D method)@\spxentry{backproject()}\spxextra{deepdrr.geo.Line2D method}}

\begin{fulllineitems}
\phantomsection\label{\detokenize{deepdrr.geo:deepdrr.geo.Line2D.backproject}}
\pysigstartsignatures
\pysiglinewithargsret{\sphinxbfcode{\sphinxupquote{backproject}}}{\sphinxparam{\DUrole{n,n}{index\_from\_world}\DUrole{p,p}{:}\DUrole{w,w}{  }\DUrole{n,n}{{\hyperref[\detokenize{deepdrr.geo:deepdrr.geo.CameraProjection}]{\sphinxcrossref{CameraProjection}}}}}}{{ $\rightarrow$ {\hyperref[\detokenize{deepdrr.geo:deepdrr.geo.Plane}]{\sphinxcrossref{Plane}}}}}
\pysigstopsignatures
\sphinxAtStartPar
Get the plane containing all the points that \sphinxtitleref{P} projects onto this line.
\begin{quote}\begin{description}
\sphinxlineitem{Parameters}
\sphinxAtStartPar
\sphinxstyleliteralstrong{\sphinxupquote{P}} ({\hyperref[\detokenize{deepdrr.geo:deepdrr.geo.Transform}]{\sphinxcrossref{\sphinxstyleliteralemphasis{\sphinxupquote{Transform}}}}}) \textendash{} A so\sphinxhyphen{}called \sphinxtitleref{index\_from\_world} projection transform.

\sphinxlineitem{Return type}
\sphinxAtStartPar
{\hyperref[\detokenize{deepdrr.geo:deepdrr.geo.Plane}]{\sphinxcrossref{Plane}}}

\end{description}\end{quote}

\end{fulllineitems}

\index{data (deepdrr.geo.Line2D attribute)@\spxentry{data}\spxextra{deepdrr.geo.Line2D attribute}}

\begin{fulllineitems}
\phantomsection\label{\detokenize{deepdrr.geo:deepdrr.geo.Line2D.data}}
\pysigstartsignatures
\pysigline{\sphinxbfcode{\sphinxupquote{data}}\sphinxbfcode{\sphinxupquote{\DUrole{p,p}{:}\DUrole{w,w}{  }ndarray}}}
\pysigstopsignatures
\end{fulllineitems}

\index{dim (deepdrr.geo.Line2D attribute)@\spxentry{dim}\spxextra{deepdrr.geo.Line2D attribute}}

\begin{fulllineitems}
\phantomsection\label{\detokenize{deepdrr.geo:deepdrr.geo.Line2D.dim}}
\pysigstartsignatures
\pysigline{\sphinxbfcode{\sphinxupquote{dim}}\sphinxbfcode{\sphinxupquote{\DUrole{w,w}{  }\DUrole{p,p}{=}\DUrole{w,w}{  }2}}}
\pysigstopsignatures
\end{fulllineitems}

\index{get\_direction() (deepdrr.geo.Line2D method)@\spxentry{get\_direction()}\spxextra{deepdrr.geo.Line2D method}}

\begin{fulllineitems}
\phantomsection\label{\detokenize{deepdrr.geo:deepdrr.geo.Line2D.get_direction}}
\pysigstartsignatures
\pysiglinewithargsret{\sphinxbfcode{\sphinxupquote{get\_direction}}}{}{{ $\rightarrow$ {\hyperref[\detokenize{deepdrr.geo:deepdrr.geo.core.Vector2D}]{\sphinxcrossref{Vector2D}}}}}
\pysigstopsignatures
\sphinxAtStartPar
Get the direction of the line.
\begin{quote}\begin{description}
\sphinxlineitem{Returns}
\sphinxAtStartPar
The unit\sphinxhyphen{}length direction of the line.

\sphinxlineitem{Return type}
\sphinxAtStartPar
{\hyperref[\detokenize{deepdrr.geo:deepdrr.geo.Vector2D}]{\sphinxcrossref{Vector2D}}}

\end{description}\end{quote}

\end{fulllineitems}

\index{get\_point() (deepdrr.geo.Line2D method)@\spxentry{get\_point()}\spxextra{deepdrr.geo.Line2D method}}

\begin{fulllineitems}
\phantomsection\label{\detokenize{deepdrr.geo:deepdrr.geo.Line2D.get_point}}
\pysigstartsignatures
\pysiglinewithargsret{\sphinxbfcode{\sphinxupquote{get\_point}}}{}{{ $\rightarrow$ {\hyperref[\detokenize{deepdrr.geo:deepdrr.geo.core.Point}]{\sphinxcrossref{Point}}}}}
\pysigstopsignatures
\sphinxAtStartPar
Get an arbitrary point on the line.
\begin{quote}\begin{description}
\sphinxlineitem{Returns}
\sphinxAtStartPar
A point on the line.

\sphinxlineitem{Return type}
\sphinxAtStartPar
{\hyperref[\detokenize{deepdrr.geo:deepdrr.geo.Point}]{\sphinxcrossref{Point}}}

\end{description}\end{quote}

\end{fulllineitems}

\index{meet() (deepdrr.geo.Line2D method)@\spxentry{meet()}\spxextra{deepdrr.geo.Line2D method}}

\begin{fulllineitems}
\phantomsection\label{\detokenize{deepdrr.geo:deepdrr.geo.Line2D.meet}}
\pysigstartsignatures
\pysiglinewithargsret{\sphinxbfcode{\sphinxupquote{meet}}}{\sphinxparam{\DUrole{n,n}{other}\DUrole{p,p}{:}\DUrole{w,w}{  }\DUrole{n,n}{{\hyperref[\detokenize{deepdrr.geo:deepdrr.geo.hyperplane.Line2D}]{\sphinxcrossref{Line2D}}}}}}{{ $\rightarrow$ {\hyperref[\detokenize{deepdrr.geo:deepdrr.geo.core.Point2D}]{\sphinxcrossref{Point2D}}}}}
\pysiglinewithargsret{\sphinxbfcode{\sphinxupquote{meet}}}{\sphinxparam{\DUrole{n,n}{other}\DUrole{p,p}{:}\DUrole{w,w}{  }\DUrole{n,n}{{\hyperref[\detokenize{deepdrr.geo:deepdrr.geo.segment.Segment2D}]{\sphinxcrossref{Segment2D}}}}}}{{ $\rightarrow$ {\hyperref[\detokenize{deepdrr.geo:deepdrr.geo.core.Point2D}]{\sphinxcrossref{Point2D}}}}}
\pysigstopsignatures
\sphinxAtStartPar
Get the intersection of two objects.

\sphinxAtStartPar
For example, given two lines, get the line that is the intersection of them.
\begin{quote}\begin{description}
\sphinxlineitem{Parameters}
\sphinxAtStartPar
\sphinxstyleliteralstrong{\sphinxupquote{other}} ({\hyperref[\detokenize{deepdrr.geo:deepdrr.geo.core.Primitive}]{\sphinxcrossref{\sphinxstyleliteralemphasis{\sphinxupquote{Primitive}}}}}) \textendash{} the other primitive.

\sphinxlineitem{Returns}
\sphinxAtStartPar
the intersection of \sphinxtitleref{self} and \sphinxtitleref{other}.

\sphinxlineitem{Return type}
\sphinxAtStartPar
{\hyperref[\detokenize{deepdrr.geo:deepdrr.geo.core.Primitive}]{\sphinxcrossref{Primitive}}}

\sphinxlineitem{Raises}
\sphinxAtStartPar
{\hyperref[\detokenize{deepdrr.geo:deepdrr.geo.MeetError}]{\sphinxcrossref{\sphinxstyleliteralstrong{\sphinxupquote{MeetError}}}}} \textendash{} if the objects cannot be intersected.

\end{description}\end{quote}

\end{fulllineitems}


\end{fulllineitems}

\index{Line3D (class in deepdrr.geo)@\spxentry{Line3D}\spxextra{class in deepdrr.geo}}

\begin{fulllineitems}
\phantomsection\label{\detokenize{deepdrr.geo:deepdrr.geo.Line3D}}
\pysigstartsignatures
\pysiglinewithargsret{\sphinxbfcode{\sphinxupquote{class\DUrole{w,w}{  }}}\sphinxcode{\sphinxupquote{deepdrr.geo.}}\sphinxbfcode{\sphinxupquote{Line3D}}}{\sphinxparam{\DUrole{n,n}{data}\DUrole{p,p}{:}\DUrole{w,w}{  }\DUrole{n,n}{ndarray}}}{}
\pysigstopsignatures
\sphinxAtStartPar
Bases: {\hyperref[\detokenize{deepdrr.geo:deepdrr.geo.hyperplane.Line}]{\sphinxcrossref{\sphinxcode{\sphinxupquote{Line}}}}}, {\hyperref[\detokenize{deepdrr.geo:deepdrr.geo.core.Primitive}]{\sphinxcrossref{\sphinxcode{\sphinxupquote{Primitive}}}}}, {\hyperref[\detokenize{deepdrr.geo:deepdrr.geo.core.Joinable}]{\sphinxcrossref{\sphinxcode{\sphinxupquote{Joinable}}}}}, {\hyperref[\detokenize{deepdrr.geo:deepdrr.geo.core.Meetable}]{\sphinxcrossref{\sphinxcode{\sphinxupquote{Meetable}}}}}, {\hyperref[\detokenize{deepdrr.geo:deepdrr.geo.core.HasProjection}]{\sphinxcrossref{\sphinxcode{\sphinxupquote{HasProjection}}}}}

\sphinxAtStartPar
Represents a line in 3D as a 6\sphinxhyphen{}vector (p,q,r,s,t,u).

\sphinxAtStartPar
Based on \sphinxurl{https://dl.acm.org/doi/pdf/10.1145/965141.563900}.
\index{K (deepdrr.geo.Line3D property)@\spxentry{K}\spxextra{deepdrr.geo.Line3D property}}

\begin{fulllineitems}
\phantomsection\label{\detokenize{deepdrr.geo:deepdrr.geo.Line3D.K}}
\pysigstartsignatures
\pysigline{\sphinxbfcode{\sphinxupquote{property\DUrole{w,w}{  }}}\sphinxbfcode{\sphinxupquote{K}}\sphinxbfcode{\sphinxupquote{\DUrole{p,p}{:}\DUrole{w,w}{  }ndarray}}}
\pysigstopsignatures
\sphinxAtStartPar
Get the dual form of the line.

\end{fulllineitems}

\index{L (deepdrr.geo.Line3D property)@\spxentry{L}\spxextra{deepdrr.geo.Line3D property}}

\begin{fulllineitems}
\phantomsection\label{\detokenize{deepdrr.geo:deepdrr.geo.Line3D.L}}
\pysigstartsignatures
\pysigline{\sphinxbfcode{\sphinxupquote{property\DUrole{w,w}{  }}}\sphinxbfcode{\sphinxupquote{L}}\sphinxbfcode{\sphinxupquote{\DUrole{p,p}{:}\DUrole{w,w}{  }ndarray}}}
\pysigstopsignatures
\sphinxAtStartPar
Get the primal matrix of the line.

\end{fulllineitems}

\index{data (deepdrr.geo.Line3D attribute)@\spxentry{data}\spxextra{deepdrr.geo.Line3D attribute}}

\begin{fulllineitems}
\phantomsection\label{\detokenize{deepdrr.geo:deepdrr.geo.Line3D.data}}
\pysigstartsignatures
\pysigline{\sphinxbfcode{\sphinxupquote{data}}\sphinxbfcode{\sphinxupquote{\DUrole{p,p}{:}\DUrole{w,w}{  }ndarray}}}
\pysigstopsignatures
\end{fulllineitems}

\index{dim (deepdrr.geo.Line3D attribute)@\spxentry{dim}\spxextra{deepdrr.geo.Line3D attribute}}

\begin{fulllineitems}
\phantomsection\label{\detokenize{deepdrr.geo:deepdrr.geo.Line3D.dim}}
\pysigstartsignatures
\pysigline{\sphinxbfcode{\sphinxupquote{dim}}\sphinxbfcode{\sphinxupquote{\DUrole{w,w}{  }\DUrole{p,p}{=}\DUrole{w,w}{  }3}}}
\pysigstopsignatures
\end{fulllineitems}

\index{dual() (deepdrr.geo.Line3D method)@\spxentry{dual()}\spxextra{deepdrr.geo.Line3D method}}

\begin{fulllineitems}
\phantomsection\label{\detokenize{deepdrr.geo:deepdrr.geo.Line3D.dual}}
\pysigstartsignatures
\pysiglinewithargsret{\sphinxbfcode{\sphinxupquote{dual}}}{}{{ $\rightarrow$ ndarray}}
\pysigstopsignatures
\sphinxAtStartPar
Get the dual form of the line.

\end{fulllineitems}

\index{from\_dual() (deepdrr.geo.Line3D class method)@\spxentry{from\_dual()}\spxextra{deepdrr.geo.Line3D class method}}

\begin{fulllineitems}
\phantomsection\label{\detokenize{deepdrr.geo:deepdrr.geo.Line3D.from_dual}}
\pysigstartsignatures
\pysiglinewithargsret{\sphinxbfcode{\sphinxupquote{classmethod\DUrole{w,w}{  }}}\sphinxbfcode{\sphinxupquote{from\_dual}}}{\sphinxparam{\DUrole{n,n}{lk}\DUrole{p,p}{:}\DUrole{w,w}{  }\DUrole{n,n}{ndarray}}}{{ $\rightarrow$ {\hyperref[\detokenize{deepdrr.geo:deepdrr.geo.hyperplane.Line3D}]{\sphinxcrossref{Line3D}}}}}
\pysigstopsignatures
\end{fulllineitems}

\index{from\_primal() (deepdrr.geo.Line3D class method)@\spxentry{from\_primal()}\spxextra{deepdrr.geo.Line3D class method}}

\begin{fulllineitems}
\phantomsection\label{\detokenize{deepdrr.geo:deepdrr.geo.Line3D.from_primal}}
\pysigstartsignatures
\pysiglinewithargsret{\sphinxbfcode{\sphinxupquote{classmethod\DUrole{w,w}{  }}}\sphinxbfcode{\sphinxupquote{from\_primal}}}{\sphinxparam{\DUrole{n,n}{lp}\DUrole{p,p}{:}\DUrole{w,w}{  }\DUrole{n,n}{ndarray}}}{{ $\rightarrow$ {\hyperref[\detokenize{deepdrr.geo:deepdrr.geo.hyperplane.Line3D}]{\sphinxcrossref{Line3D}}}}}
\pysigstopsignatures
\end{fulllineitems}

\index{get\_direction() (deepdrr.geo.Line3D method)@\spxentry{get\_direction()}\spxextra{deepdrr.geo.Line3D method}}

\begin{fulllineitems}
\phantomsection\label{\detokenize{deepdrr.geo:deepdrr.geo.Line3D.get_direction}}
\pysigstartsignatures
\pysiglinewithargsret{\sphinxbfcode{\sphinxupquote{get\_direction}}}{}{{ $\rightarrow$ {\hyperref[\detokenize{deepdrr.geo:deepdrr.geo.core.Vector3D}]{\sphinxcrossref{Vector3D}}}}}
\pysigstopsignatures
\sphinxAtStartPar
Get the direction of the line.

\end{fulllineitems}

\index{get\_point() (deepdrr.geo.Line3D method)@\spxentry{get\_point()}\spxextra{deepdrr.geo.Line3D method}}

\begin{fulllineitems}
\phantomsection\label{\detokenize{deepdrr.geo:deepdrr.geo.Line3D.get_point}}
\pysigstartsignatures
\pysiglinewithargsret{\sphinxbfcode{\sphinxupquote{get\_point}}}{}{{ $\rightarrow$ {\hyperref[\detokenize{deepdrr.geo:deepdrr.geo.core.Point3D}]{\sphinxcrossref{Point3D}}}}}
\pysigstopsignatures
\sphinxAtStartPar
Get a point on the line.

\end{fulllineitems}

\index{join() (deepdrr.geo.Line3D method)@\spxentry{join()}\spxextra{deepdrr.geo.Line3D method}}

\begin{fulllineitems}
\phantomsection\label{\detokenize{deepdrr.geo:deepdrr.geo.Line3D.join}}
\pysigstartsignatures
\pysiglinewithargsret{\sphinxbfcode{\sphinxupquote{join}}}{\sphinxparam{\DUrole{n,n}{other}\DUrole{p,p}{:}\DUrole{w,w}{  }\DUrole{n,n}{{\hyperref[\detokenize{deepdrr.geo:deepdrr.geo.core.Point3D}]{\sphinxcrossref{Point3D}}}}}}{{ $\rightarrow$ {\hyperref[\detokenize{deepdrr.geo:deepdrr.geo.hyperplane.Plane}]{\sphinxcrossref{Plane}}}}}
\pysigstopsignatures
\sphinxAtStartPar
Join two objects.

\sphinxAtStartPar
For example, given two points, get the line that connects them.
\begin{quote}\begin{description}
\sphinxlineitem{Parameters}
\sphinxAtStartPar
\sphinxstyleliteralstrong{\sphinxupquote{other}} ({\hyperref[\detokenize{deepdrr.geo:deepdrr.geo.core.Primitive}]{\sphinxcrossref{\sphinxstyleliteralemphasis{\sphinxupquote{Primitive}}}}}) \textendash{} the other primitive.

\sphinxlineitem{Returns}
\sphinxAtStartPar
the joined primitive.

\sphinxlineitem{Return type}
\sphinxAtStartPar
{\hyperref[\detokenize{deepdrr.geo:deepdrr.geo.core.Primitive}]{\sphinxcrossref{Primitive}}}

\end{description}\end{quote}

\end{fulllineitems}

\index{meet() (deepdrr.geo.Line3D method)@\spxentry{meet()}\spxextra{deepdrr.geo.Line3D method}}

\begin{fulllineitems}
\phantomsection\label{\detokenize{deepdrr.geo:deepdrr.geo.Line3D.meet}}
\pysigstartsignatures
\pysiglinewithargsret{\sphinxbfcode{\sphinxupquote{meet}}}{\sphinxparam{\DUrole{n,n}{other}\DUrole{p,p}{:}\DUrole{w,w}{  }\DUrole{n,n}{{\hyperref[\detokenize{deepdrr.geo:deepdrr.geo.hyperplane.Plane}]{\sphinxcrossref{Plane}}}}}}{{ $\rightarrow$ {\hyperref[\detokenize{deepdrr.geo:deepdrr.geo.core.Point3D}]{\sphinxcrossref{Point3D}}}}}
\pysigstopsignatures
\sphinxAtStartPar
Get the intersection of two objects.

\sphinxAtStartPar
For example, given two lines, get the line that is the intersection of them.
\begin{quote}\begin{description}
\sphinxlineitem{Parameters}
\sphinxAtStartPar
\sphinxstyleliteralstrong{\sphinxupquote{other}} ({\hyperref[\detokenize{deepdrr.geo:deepdrr.geo.core.Primitive}]{\sphinxcrossref{\sphinxstyleliteralemphasis{\sphinxupquote{Primitive}}}}}) \textendash{} the other primitive.

\sphinxlineitem{Returns}
\sphinxAtStartPar
the intersection of \sphinxtitleref{self} and \sphinxtitleref{other}.

\sphinxlineitem{Return type}
\sphinxAtStartPar
{\hyperref[\detokenize{deepdrr.geo:deepdrr.geo.core.Primitive}]{\sphinxcrossref{Primitive}}}

\sphinxlineitem{Raises}
\sphinxAtStartPar
{\hyperref[\detokenize{deepdrr.geo:deepdrr.geo.MeetError}]{\sphinxcrossref{\sphinxstyleliteralstrong{\sphinxupquote{MeetError}}}}} \textendash{} if the objects cannot be intersected.

\end{description}\end{quote}

\end{fulllineitems}

\index{p (deepdrr.geo.Line3D property)@\spxentry{p}\spxextra{deepdrr.geo.Line3D property}}

\begin{fulllineitems}
\phantomsection\label{\detokenize{deepdrr.geo:deepdrr.geo.Line3D.p}}
\pysigstartsignatures
\pysigline{\sphinxbfcode{\sphinxupquote{property\DUrole{w,w}{  }}}\sphinxbfcode{\sphinxupquote{p}}\sphinxbfcode{\sphinxupquote{\DUrole{p,p}{:}\DUrole{w,w}{  }float}}}
\pysigstopsignatures
\sphinxAtStartPar
Get the first parameter of the line.

\end{fulllineitems}

\index{primal() (deepdrr.geo.Line3D method)@\spxentry{primal()}\spxextra{deepdrr.geo.Line3D method}}

\begin{fulllineitems}
\phantomsection\label{\detokenize{deepdrr.geo:deepdrr.geo.Line3D.primal}}
\pysigstartsignatures
\pysiglinewithargsret{\sphinxbfcode{\sphinxupquote{primal}}}{}{{ $\rightarrow$ ndarray}}
\pysigstopsignatures
\sphinxAtStartPar
Get the primal matrix of the line.

\end{fulllineitems}

\index{projection\_type() (deepdrr.geo.Line3D class method)@\spxentry{projection\_type()}\spxextra{deepdrr.geo.Line3D class method}}

\begin{fulllineitems}
\phantomsection\label{\detokenize{deepdrr.geo:deepdrr.geo.Line3D.projection_type}}
\pysigstartsignatures
\pysiglinewithargsret{\sphinxbfcode{\sphinxupquote{classmethod\DUrole{w,w}{  }}}\sphinxbfcode{\sphinxupquote{projection\_type}}}{}{{ $\rightarrow$ Type\DUrole{p,p}{{[}}{\hyperref[\detokenize{deepdrr.geo:deepdrr.geo.hyperplane.Line2D}]{\sphinxcrossref{Line2D}}}\DUrole{p,p}{{]}}}}
\pysigstopsignatures
\sphinxAtStartPar
Get the type of the projection of the object.
\begin{quote}\begin{description}
\sphinxlineitem{Returns}
\sphinxAtStartPar
the type of the projection of the object.

\sphinxlineitem{Return type}
\sphinxAtStartPar
Type{[}{\hyperref[\detokenize{deepdrr.geo:deepdrr.geo.core.Primitive}]{\sphinxcrossref{Primitive}}}{]}

\end{description}\end{quote}

\end{fulllineitems}

\index{q (deepdrr.geo.Line3D property)@\spxentry{q}\spxextra{deepdrr.geo.Line3D property}}

\begin{fulllineitems}
\phantomsection\label{\detokenize{deepdrr.geo:deepdrr.geo.Line3D.q}}
\pysigstartsignatures
\pysigline{\sphinxbfcode{\sphinxupquote{property\DUrole{w,w}{  }}}\sphinxbfcode{\sphinxupquote{q}}\sphinxbfcode{\sphinxupquote{\DUrole{p,p}{:}\DUrole{w,w}{  }float}}}
\pysigstopsignatures
\sphinxAtStartPar
Get the second parameter of the line.

\end{fulllineitems}

\index{r (deepdrr.geo.Line3D property)@\spxentry{r}\spxextra{deepdrr.geo.Line3D property}}

\begin{fulllineitems}
\phantomsection\label{\detokenize{deepdrr.geo:deepdrr.geo.Line3D.r}}
\pysigstartsignatures
\pysigline{\sphinxbfcode{\sphinxupquote{property\DUrole{w,w}{  }}}\sphinxbfcode{\sphinxupquote{r}}\sphinxbfcode{\sphinxupquote{\DUrole{p,p}{:}\DUrole{w,w}{  }float}}}
\pysigstopsignatures
\sphinxAtStartPar
Get the third parameter of the line.

\end{fulllineitems}

\index{s (deepdrr.geo.Line3D property)@\spxentry{s}\spxextra{deepdrr.geo.Line3D property}}

\begin{fulllineitems}
\phantomsection\label{\detokenize{deepdrr.geo:deepdrr.geo.Line3D.s}}
\pysigstartsignatures
\pysigline{\sphinxbfcode{\sphinxupquote{property\DUrole{w,w}{  }}}\sphinxbfcode{\sphinxupquote{s}}\sphinxbfcode{\sphinxupquote{\DUrole{p,p}{:}\DUrole{w,w}{  }float}}}
\pysigstopsignatures
\sphinxAtStartPar
Get the fourth parameter of the line.

\end{fulllineitems}

\index{t (deepdrr.geo.Line3D property)@\spxentry{t}\spxextra{deepdrr.geo.Line3D property}}

\begin{fulllineitems}
\phantomsection\label{\detokenize{deepdrr.geo:deepdrr.geo.Line3D.t}}
\pysigstartsignatures
\pysigline{\sphinxbfcode{\sphinxupquote{property\DUrole{w,w}{  }}}\sphinxbfcode{\sphinxupquote{t}}\sphinxbfcode{\sphinxupquote{\DUrole{p,p}{:}\DUrole{w,w}{  }float}}}
\pysigstopsignatures
\sphinxAtStartPar
Get the fifth parameter of the line.

\end{fulllineitems}

\index{u (deepdrr.geo.Line3D property)@\spxentry{u}\spxextra{deepdrr.geo.Line3D property}}

\begin{fulllineitems}
\phantomsection\label{\detokenize{deepdrr.geo:deepdrr.geo.Line3D.u}}
\pysigstartsignatures
\pysigline{\sphinxbfcode{\sphinxupquote{property\DUrole{w,w}{  }}}\sphinxbfcode{\sphinxupquote{u}}\sphinxbfcode{\sphinxupquote{\DUrole{p,p}{:}\DUrole{w,w}{  }float}}}
\pysigstopsignatures
\sphinxAtStartPar
Get the sixth parameter of the line.

\end{fulllineitems}


\end{fulllineitems}

\index{MeetError@\spxentry{MeetError}}

\begin{fulllineitems}
\phantomsection\label{\detokenize{deepdrr.geo:deepdrr.geo.MeetError}}
\pysigstartsignatures
\pysigline{\sphinxbfcode{\sphinxupquote{exception\DUrole{w,w}{  }}}\sphinxcode{\sphinxupquote{deepdrr.geo.}}\sphinxbfcode{\sphinxupquote{MeetError}}}
\pysigstopsignatures
\sphinxAtStartPar
Bases: \sphinxcode{\sphinxupquote{Exception}}

\sphinxAtStartPar
Represents an error when finding the intersection of two primitives.

\end{fulllineitems}

\index{Plane (class in deepdrr.geo)@\spxentry{Plane}\spxextra{class in deepdrr.geo}}

\begin{fulllineitems}
\phantomsection\label{\detokenize{deepdrr.geo:deepdrr.geo.Plane}}
\pysigstartsignatures
\pysiglinewithargsret{\sphinxbfcode{\sphinxupquote{class\DUrole{w,w}{  }}}\sphinxcode{\sphinxupquote{deepdrr.geo.}}\sphinxbfcode{\sphinxupquote{Plane}}}{\sphinxparam{\DUrole{n,n}{data}\DUrole{p,p}{:}\DUrole{w,w}{  }\DUrole{n,n}{ndarray}}}{}
\pysigstopsignatures
\sphinxAtStartPar
Bases: {\hyperref[\detokenize{deepdrr.geo:deepdrr.geo.hyperplane.HyperPlane}]{\sphinxcrossref{\sphinxcode{\sphinxupquote{HyperPlane}}}}}

\sphinxAtStartPar
Represents a plane in 3D
\index{data (deepdrr.geo.Plane attribute)@\spxentry{data}\spxextra{deepdrr.geo.Plane attribute}}

\begin{fulllineitems}
\phantomsection\label{\detokenize{deepdrr.geo:deepdrr.geo.Plane.data}}
\pysigstartsignatures
\pysigline{\sphinxbfcode{\sphinxupquote{data}}\sphinxbfcode{\sphinxupquote{\DUrole{p,p}{:}\DUrole{w,w}{  }ndarray}}}
\pysigstopsignatures
\end{fulllineitems}

\index{dim (deepdrr.geo.Plane attribute)@\spxentry{dim}\spxextra{deepdrr.geo.Plane attribute}}

\begin{fulllineitems}
\phantomsection\label{\detokenize{deepdrr.geo:deepdrr.geo.Plane.dim}}
\pysigstartsignatures
\pysigline{\sphinxbfcode{\sphinxupquote{dim}}\sphinxbfcode{\sphinxupquote{\DUrole{w,w}{  }\DUrole{p,p}{=}\DUrole{w,w}{  }3}}}
\pysigstopsignatures
\end{fulllineitems}

\index{from\_point\_direction() (deepdrr.geo.Plane class method)@\spxentry{from\_point\_direction()}\spxextra{deepdrr.geo.Plane class method}}

\begin{fulllineitems}
\phantomsection\label{\detokenize{deepdrr.geo:deepdrr.geo.Plane.from_point_direction}}
\pysigstartsignatures
\pysiglinewithargsret{\sphinxbfcode{\sphinxupquote{classmethod\DUrole{w,w}{  }}}\sphinxbfcode{\sphinxupquote{from\_point\_direction}}}{\sphinxparam{\DUrole{n,n}{r}\DUrole{p,p}{:}\DUrole{w,w}{  }\DUrole{n,n}{{\hyperref[\detokenize{deepdrr.geo:deepdrr.geo.core.Point3D}]{\sphinxcrossref{Point3D}}}}}\sphinxparamcomma \sphinxparam{\DUrole{n,n}{d}\DUrole{p,p}{:}\DUrole{w,w}{  }\DUrole{n,n}{{\hyperref[\detokenize{deepdrr.geo:deepdrr.geo.core.Vector3D}]{\sphinxcrossref{Vector3D}}}}}}{}
\pysigstopsignatures
\sphinxAtStartPar
Make a plane from a point and a direction vector.
\begin{quote}\begin{description}
\sphinxlineitem{Parameters}\begin{itemize}
\item {} 
\sphinxAtStartPar
\sphinxstyleliteralstrong{\sphinxupquote{r}} ({\hyperref[\detokenize{deepdrr.geo:deepdrr.geo.Point3D}]{\sphinxcrossref{\sphinxstyleliteralemphasis{\sphinxupquote{Point3D}}}}}) \textendash{} The point on the plane.

\item {} 
\sphinxAtStartPar
\sphinxstyleliteralstrong{\sphinxupquote{d}} ({\hyperref[\detokenize{deepdrr.geo:deepdrr.geo.Vector3D}]{\sphinxcrossref{\sphinxstyleliteralemphasis{\sphinxupquote{Vector3D}}}}}) \textendash{} The direction vector of the plane.

\end{itemize}

\sphinxlineitem{Returns}
\sphinxAtStartPar
The plane.

\sphinxlineitem{Return type}
\sphinxAtStartPar
{\hyperref[\detokenize{deepdrr.geo:deepdrr.geo.Plane}]{\sphinxcrossref{Plane}}}

\end{description}\end{quote}

\end{fulllineitems}

\index{from\_point\_normal() (deepdrr.geo.Plane class method)@\spxentry{from\_point\_normal()}\spxextra{deepdrr.geo.Plane class method}}

\begin{fulllineitems}
\phantomsection\label{\detokenize{deepdrr.geo:deepdrr.geo.Plane.from_point_normal}}
\pysigstartsignatures
\pysiglinewithargsret{\sphinxbfcode{\sphinxupquote{classmethod\DUrole{w,w}{  }}}\sphinxbfcode{\sphinxupquote{from\_point\_normal}}}{\sphinxparam{\DUrole{n,n}{r}\DUrole{p,p}{:}\DUrole{w,w}{  }\DUrole{n,n}{{\hyperref[\detokenize{deepdrr.geo:deepdrr.geo.core.Point3D}]{\sphinxcrossref{Point3D}}}}}\sphinxparamcomma \sphinxparam{\DUrole{n,n}{n}\DUrole{p,p}{:}\DUrole{w,w}{  }\DUrole{n,n}{{\hyperref[\detokenize{deepdrr.geo:deepdrr.geo.core.Vector3D}]{\sphinxcrossref{Vector3D}}}}}}{}
\pysigstopsignatures
\sphinxAtStartPar
Make a plane from a point and a normal vector.
\begin{quote}\begin{description}
\sphinxlineitem{Parameters}\begin{itemize}
\item {} 
\sphinxAtStartPar
\sphinxstyleliteralstrong{\sphinxupquote{r}} ({\hyperref[\detokenize{deepdrr.geo:deepdrr.geo.Point3D}]{\sphinxcrossref{\sphinxstyleliteralemphasis{\sphinxupquote{Point3D}}}}}) \textendash{} The point on the plane.

\item {} 
\sphinxAtStartPar
\sphinxstyleliteralstrong{\sphinxupquote{n}} ({\hyperref[\detokenize{deepdrr.geo:deepdrr.geo.Vector3D}]{\sphinxcrossref{\sphinxstyleliteralemphasis{\sphinxupquote{Vector3D}}}}}) \textendash{} The normal vector of the plane.

\end{itemize}

\sphinxlineitem{Returns}
\sphinxAtStartPar
The plane.

\sphinxlineitem{Return type}
\sphinxAtStartPar
{\hyperref[\detokenize{deepdrr.geo:deepdrr.geo.Plane}]{\sphinxcrossref{Plane}}}

\end{description}\end{quote}

\end{fulllineitems}

\index{from\_points() (deepdrr.geo.Plane class method)@\spxentry{from\_points()}\spxextra{deepdrr.geo.Plane class method}}

\begin{fulllineitems}
\phantomsection\label{\detokenize{deepdrr.geo:deepdrr.geo.Plane.from_points}}
\pysigstartsignatures
\pysiglinewithargsret{\sphinxbfcode{\sphinxupquote{classmethod\DUrole{w,w}{  }}}\sphinxbfcode{\sphinxupquote{from\_points}}}{\sphinxparam{\DUrole{n,n}{a}\DUrole{p,p}{:}\DUrole{w,w}{  }\DUrole{n,n}{{\hyperref[\detokenize{deepdrr.geo:deepdrr.geo.core.Point3D}]{\sphinxcrossref{Point3D}}}}}\sphinxparamcomma \sphinxparam{\DUrole{n,n}{b}\DUrole{p,p}{:}\DUrole{w,w}{  }\DUrole{n,n}{{\hyperref[\detokenize{deepdrr.geo:deepdrr.geo.core.Point3D}]{\sphinxcrossref{Point3D}}}}}\sphinxparamcomma \sphinxparam{\DUrole{n,n}{c}\DUrole{p,p}{:}\DUrole{w,w}{  }\DUrole{n,n}{{\hyperref[\detokenize{deepdrr.geo:deepdrr.geo.core.Point3D}]{\sphinxcrossref{Point3D}}}}}}{{ $\rightarrow$ None}}
\pysigstopsignatures
\sphinxAtStartPar
Initialize the plane containing three points.
\begin{quote}\begin{description}
\sphinxlineitem{Parameters}\begin{itemize}
\item {} 
\sphinxAtStartPar
\sphinxstyleliteralstrong{\sphinxupquote{a}} ({\hyperref[\detokenize{deepdrr.geo:deepdrr.geo.Point3D}]{\sphinxcrossref{\sphinxstyleliteralemphasis{\sphinxupquote{Point3D}}}}}) \textendash{} a point on the plane.

\item {} 
\sphinxAtStartPar
\sphinxstyleliteralstrong{\sphinxupquote{b}} ({\hyperref[\detokenize{deepdrr.geo:deepdrr.geo.Point3D}]{\sphinxcrossref{\sphinxstyleliteralemphasis{\sphinxupquote{Point3D}}}}}) \textendash{} a point on the plane.

\item {} 
\sphinxAtStartPar
\sphinxstyleliteralstrong{\sphinxupquote{c}} ({\hyperref[\detokenize{deepdrr.geo:deepdrr.geo.Point3D}]{\sphinxcrossref{\sphinxstyleliteralemphasis{\sphinxupquote{Point3D}}}}}) \textendash{} a point on the plane.

\end{itemize}

\sphinxlineitem{Returns}
\sphinxAtStartPar
The plane.

\sphinxlineitem{Return type}
\sphinxAtStartPar
{\hyperref[\detokenize{deepdrr.geo:deepdrr.geo.Plane}]{\sphinxcrossref{Plane}}}

\end{description}\end{quote}

\end{fulllineitems}

\index{get\_direction() (deepdrr.geo.Plane method)@\spxentry{get\_direction()}\spxextra{deepdrr.geo.Plane method}}

\begin{fulllineitems}
\phantomsection\label{\detokenize{deepdrr.geo:deepdrr.geo.Plane.get_direction}}
\pysigstartsignatures
\pysiglinewithargsret{\sphinxbfcode{\sphinxupquote{get\_direction}}}{}{{ $\rightarrow$ {\hyperref[\detokenize{deepdrr.geo:deepdrr.geo.core.Vector3D}]{\sphinxcrossref{Vector3D}}}}}
\pysigstopsignatures
\sphinxAtStartPar
Get the direction of the plane.
\begin{quote}\begin{description}
\sphinxlineitem{Returns}
\sphinxAtStartPar
The unit\sphinxhyphen{}length direction of the plane.

\sphinxlineitem{Return type}
\sphinxAtStartPar
{\hyperref[\detokenize{deepdrr.geo:deepdrr.geo.Vector3D}]{\sphinxcrossref{Vector3D}}}

\end{description}\end{quote}

\end{fulllineitems}

\index{get\_point() (deepdrr.geo.Plane method)@\spxentry{get\_point()}\spxextra{deepdrr.geo.Plane method}}

\begin{fulllineitems}
\phantomsection\label{\detokenize{deepdrr.geo:deepdrr.geo.Plane.get_point}}
\pysigstartsignatures
\pysiglinewithargsret{\sphinxbfcode{\sphinxupquote{get\_point}}}{}{{ $\rightarrow$ {\hyperref[\detokenize{deepdrr.geo:deepdrr.geo.core.Point3D}]{\sphinxcrossref{Point3D}}}}}
\pysigstopsignatures
\sphinxAtStartPar
Get an arbitrary point on the plane.
\begin{quote}\begin{description}
\sphinxlineitem{Returns}
\sphinxAtStartPar
A point on the plane.

\sphinxlineitem{Return type}
\sphinxAtStartPar
{\hyperref[\detokenize{deepdrr.geo:deepdrr.geo.Point3D}]{\sphinxcrossref{Point3D}}}

\end{description}\end{quote}

\end{fulllineitems}

\index{meet() (deepdrr.geo.Plane method)@\spxentry{meet()}\spxextra{deepdrr.geo.Plane method}}

\begin{fulllineitems}
\phantomsection\label{\detokenize{deepdrr.geo:deepdrr.geo.Plane.meet}}
\pysigstartsignatures
\pysiglinewithargsret{\sphinxbfcode{\sphinxupquote{meet}}}{\sphinxparam{\DUrole{n,n}{other}\DUrole{p,p}{:}\DUrole{w,w}{  }\DUrole{n,n}{{\hyperref[\detokenize{deepdrr.geo:deepdrr.geo.hyperplane.Plane}]{\sphinxcrossref{Plane}}}}}}{{ $\rightarrow$ {\hyperref[\detokenize{deepdrr.geo:deepdrr.geo.hyperplane.Line3D}]{\sphinxcrossref{Line3D}}}}}
\pysiglinewithargsret{\sphinxbfcode{\sphinxupquote{meet}}}{\sphinxparam{\DUrole{n,n}{other}\DUrole{p,p}{:}\DUrole{w,w}{  }\DUrole{n,n}{{\hyperref[\detokenize{deepdrr.geo:deepdrr.geo.hyperplane.Line3D}]{\sphinxcrossref{Line3D}}}}}}{{ $\rightarrow$ {\hyperref[\detokenize{deepdrr.geo:deepdrr.geo.core.Point3D}]{\sphinxcrossref{Point3D}}}}}
\pysigstopsignatures
\sphinxAtStartPar
Get the intersection of two objects.

\sphinxAtStartPar
For example, given two lines, get the line that is the intersection of them.
\begin{quote}\begin{description}
\sphinxlineitem{Parameters}
\sphinxAtStartPar
\sphinxstyleliteralstrong{\sphinxupquote{other}} ({\hyperref[\detokenize{deepdrr.geo:deepdrr.geo.core.Primitive}]{\sphinxcrossref{\sphinxstyleliteralemphasis{\sphinxupquote{Primitive}}}}}) \textendash{} the other primitive.

\sphinxlineitem{Returns}
\sphinxAtStartPar
the intersection of \sphinxtitleref{self} and \sphinxtitleref{other}.

\sphinxlineitem{Return type}
\sphinxAtStartPar
{\hyperref[\detokenize{deepdrr.geo:deepdrr.geo.core.Primitive}]{\sphinxcrossref{Primitive}}}

\sphinxlineitem{Raises}
\sphinxAtStartPar
{\hyperref[\detokenize{deepdrr.geo:deepdrr.geo.MeetError}]{\sphinxcrossref{\sphinxstyleliteralstrong{\sphinxupquote{MeetError}}}}} \textendash{} if the objects cannot be intersected.

\end{description}\end{quote}

\end{fulllineitems}


\end{fulllineitems}

\index{Point (class in deepdrr.geo)@\spxentry{Point}\spxextra{class in deepdrr.geo}}

\begin{fulllineitems}
\phantomsection\label{\detokenize{deepdrr.geo:deepdrr.geo.Point}}
\pysigstartsignatures
\pysiglinewithargsret{\sphinxbfcode{\sphinxupquote{class\DUrole{w,w}{  }}}\sphinxcode{\sphinxupquote{deepdrr.geo.}}\sphinxbfcode{\sphinxupquote{Point}}}{\sphinxparam{\DUrole{n,n}{data}\DUrole{p,p}{:}\DUrole{w,w}{  }\DUrole{n,n}{ndarray}}}{}
\pysigstopsignatures
\sphinxAtStartPar
Bases: {\hyperref[\detokenize{deepdrr.geo:deepdrr.geo.core.PointOrVector}]{\sphinxcrossref{\sphinxcode{\sphinxupquote{PointOrVector}}}}}, {\hyperref[\detokenize{deepdrr.geo:deepdrr.geo.core.Joinable}]{\sphinxcrossref{\sphinxcode{\sphinxupquote{Joinable}}}}}, {\hyperref[\detokenize{deepdrr.geo:deepdrr.geo.core.HasLocation}]{\sphinxcrossref{\sphinxcode{\sphinxupquote{HasLocation}}}}}
\index{as\_vector() (deepdrr.geo.Point method)@\spxentry{as\_vector()}\spxextra{deepdrr.geo.Point method}}

\begin{fulllineitems}
\phantomsection\label{\detokenize{deepdrr.geo:deepdrr.geo.Point.as_vector}}
\pysigstartsignatures
\pysiglinewithargsret{\sphinxbfcode{\sphinxupquote{as\_vector}}}{}{{ $\rightarrow$ {\hyperref[\detokenize{deepdrr.geo:deepdrr.geo.core.Vector}]{\sphinxcrossref{Vector}}}}}
\pysigstopsignatures
\sphinxAtStartPar
Get the vector with the same numerical representation as this point.

\end{fulllineitems}

\index{data (deepdrr.geo.Point attribute)@\spxentry{data}\spxextra{deepdrr.geo.Point attribute}}

\begin{fulllineitems}
\phantomsection\label{\detokenize{deepdrr.geo:deepdrr.geo.Point.data}}
\pysigstartsignatures
\pysigline{\sphinxbfcode{\sphinxupquote{data}}\sphinxbfcode{\sphinxupquote{\DUrole{p,p}{:}\DUrole{w,w}{  }ndarray}}}
\pysigstopsignatures
\end{fulllineitems}

\index{from\_any() (deepdrr.geo.Point class method)@\spxentry{from\_any()}\spxextra{deepdrr.geo.Point class method}}

\begin{fulllineitems}
\phantomsection\label{\detokenize{deepdrr.geo:deepdrr.geo.Point.from_any}}
\pysigstartsignatures
\pysiglinewithargsret{\sphinxbfcode{\sphinxupquote{classmethod\DUrole{w,w}{  }}}\sphinxbfcode{\sphinxupquote{from\_any}}}{\sphinxparam{\DUrole{n,n}{other}\DUrole{p,p}{:}\DUrole{w,w}{  }\DUrole{n,n}{ndarray\DUrole{w,w}{  }\DUrole{p,p}{|}\DUrole{w,w}{  }{\hyperref[\detokenize{deepdrr.geo:deepdrr.geo.core.Point}]{\sphinxcrossref{Point}}}}}}{}
\pysigstopsignatures
\sphinxAtStartPar
If other is not a point, make it one.

\end{fulllineitems}

\index{from\_array() (deepdrr.geo.Point class method)@\spxentry{from\_array()}\spxextra{deepdrr.geo.Point class method}}

\begin{fulllineitems}
\phantomsection\label{\detokenize{deepdrr.geo:deepdrr.geo.Point.from_array}}
\pysigstartsignatures
\pysiglinewithargsret{\sphinxbfcode{\sphinxupquote{classmethod\DUrole{w,w}{  }}}\sphinxbfcode{\sphinxupquote{from\_array}}}{\sphinxparam{\DUrole{n,n}{x}\DUrole{p,p}{:}\DUrole{w,w}{  }\DUrole{n,n}{ndarray}}}{{ $\rightarrow$ T}}
\pysigstopsignatures
\sphinxAtStartPar
Create a homogeneous object from its non\sphinxhyphen{}homogeous representation as an array.

\end{fulllineitems}

\index{get\_point() (deepdrr.geo.Point method)@\spxentry{get\_point()}\spxextra{deepdrr.geo.Point method}}

\begin{fulllineitems}
\phantomsection\label{\detokenize{deepdrr.geo:deepdrr.geo.Point.get_point}}
\pysigstartsignatures
\pysiglinewithargsret{\sphinxbfcode{\sphinxupquote{get\_point}}}{}{{ $\rightarrow$ Self}}
\pysigstopsignatures
\sphinxAtStartPar
Get the point with the same numerical representation as this point.

\end{fulllineitems}

\index{lerp() (deepdrr.geo.Point method)@\spxentry{lerp()}\spxextra{deepdrr.geo.Point method}}

\begin{fulllineitems}
\phantomsection\label{\detokenize{deepdrr.geo:deepdrr.geo.Point.lerp}}
\pysigstartsignatures
\pysiglinewithargsret{\sphinxbfcode{\sphinxupquote{lerp}}}{\sphinxparam{\DUrole{n,n}{other}\DUrole{p,p}{:}\DUrole{w,w}{  }\DUrole{n,n}{{\hyperref[\detokenize{deepdrr.geo:deepdrr.geo.core.Point}]{\sphinxcrossref{Point}}}}}\sphinxparamcomma \sphinxparam{\DUrole{n,n}{alpha}\DUrole{p,p}{:}\DUrole{w,w}{  }\DUrole{n,n}{float}\DUrole{w,w}{  }\DUrole{o,o}{=}\DUrole{w,w}{  }\DUrole{default_value}{0.5}}}{{ $\rightarrow$ Self}}
\pysigstopsignatures
\sphinxAtStartPar
Linearly interpolate between one point and another.
\begin{quote}\begin{description}
\sphinxlineitem{Parameters}\begin{itemize}
\item {} 
\sphinxAtStartPar
\sphinxstyleliteralstrong{\sphinxupquote{other}} ({\hyperref[\detokenize{deepdrr.geo:deepdrr.geo.Point}]{\sphinxcrossref{\sphinxstyleliteralemphasis{\sphinxupquote{Point}}}}}) \textendash{} other point.

\item {} 
\sphinxAtStartPar
\sphinxstyleliteralstrong{\sphinxupquote{alpha}} (\sphinxstyleliteralemphasis{\sphinxupquote{float}}) \textendash{} fraction of the distance from self to other to travel. Defaults to 0.5 (the midpoint).

\end{itemize}

\sphinxlineitem{Returns}
\sphinxAtStartPar
the point that is \sphinxtitleref{alpha} of the way between self and other.

\sphinxlineitem{Return type}
\sphinxAtStartPar
{\hyperref[\detokenize{deepdrr.geo:deepdrr.geo.Point}]{\sphinxcrossref{Point}}}

\end{description}\end{quote}

\end{fulllineitems}


\end{fulllineitems}

\index{Point2D (class in deepdrr.geo)@\spxentry{Point2D}\spxextra{class in deepdrr.geo}}

\begin{fulllineitems}
\phantomsection\label{\detokenize{deepdrr.geo:deepdrr.geo.Point2D}}
\pysigstartsignatures
\pysiglinewithargsret{\sphinxbfcode{\sphinxupquote{class\DUrole{w,w}{  }}}\sphinxcode{\sphinxupquote{deepdrr.geo.}}\sphinxbfcode{\sphinxupquote{Point2D}}}{\sphinxparam{\DUrole{n,n}{data}\DUrole{p,p}{:}\DUrole{w,w}{  }\DUrole{n,n}{ndarray}}}{}
\pysigstopsignatures
\sphinxAtStartPar
Bases: {\hyperref[\detokenize{deepdrr.geo:deepdrr.geo.core.Point}]{\sphinxcrossref{\sphinxcode{\sphinxupquote{Point}}}}}

\sphinxAtStartPar
Homogeneous point in 2D, represented as an array with {[}x, y, 1{]}
\index{backproject() (deepdrr.geo.Point2D method)@\spxentry{backproject()}\spxextra{deepdrr.geo.Point2D method}}

\begin{fulllineitems}
\phantomsection\label{\detokenize{deepdrr.geo:deepdrr.geo.Point2D.backproject}}
\pysigstartsignatures
\pysiglinewithargsret{\sphinxbfcode{\sphinxupquote{backproject}}}{\sphinxparam{\DUrole{n,n}{index\_from\_world}\DUrole{p,p}{:}\DUrole{w,w}{  }\DUrole{n,n}{{\hyperref[\detokenize{deepdrr.geo:deepdrr.geo.CameraProjection}]{\sphinxcrossref{CameraProjection}}}}}}{{ $\rightarrow$ {\hyperref[\detokenize{deepdrr.geo:deepdrr.geo.Line3D}]{\sphinxcrossref{Line3D}}}}}
\pysigstopsignatures
\sphinxAtStartPar
Backproject this point into a line.
\begin{quote}\begin{description}
\sphinxlineitem{Parameters}
\sphinxAtStartPar
\sphinxstyleliteralstrong{\sphinxupquote{index\_from\_world}} ({\hyperref[\detokenize{deepdrr.geo:deepdrr.geo.Transform}]{\sphinxcrossref{\sphinxstyleliteralemphasis{\sphinxupquote{Transform}}}}}) \textendash{} The transform from the world to the index.

\sphinxlineitem{Returns}
\sphinxAtStartPar
The line in 3D space through the source of \sphinxtitleref{index\_from\_world} and self.

\sphinxlineitem{Return type}
\sphinxAtStartPar
{\hyperref[\detokenize{deepdrr.geo:deepdrr.geo.Line3D}]{\sphinxcrossref{Line3D}}}

\end{description}\end{quote}

\end{fulllineitems}

\index{data (deepdrr.geo.Point2D attribute)@\spxentry{data}\spxextra{deepdrr.geo.Point2D attribute}}

\begin{fulllineitems}
\phantomsection\label{\detokenize{deepdrr.geo:deepdrr.geo.Point2D.data}}
\pysigstartsignatures
\pysigline{\sphinxbfcode{\sphinxupquote{data}}\sphinxbfcode{\sphinxupquote{\DUrole{p,p}{:}\DUrole{w,w}{  }ndarray}}}
\pysigstopsignatures
\end{fulllineitems}

\index{dim (deepdrr.geo.Point2D attribute)@\spxentry{dim}\spxextra{deepdrr.geo.Point2D attribute}}

\begin{fulllineitems}
\phantomsection\label{\detokenize{deepdrr.geo:deepdrr.geo.Point2D.dim}}
\pysigstartsignatures
\pysigline{\sphinxbfcode{\sphinxupquote{dim}}\sphinxbfcode{\sphinxupquote{\DUrole{w,w}{  }\DUrole{p,p}{=}\DUrole{w,w}{  }2}}}
\pysigstopsignatures
\end{fulllineitems}

\index{join() (deepdrr.geo.Point2D method)@\spxentry{join()}\spxextra{deepdrr.geo.Point2D method}}

\begin{fulllineitems}
\phantomsection\label{\detokenize{deepdrr.geo:deepdrr.geo.Point2D.join}}
\pysigstartsignatures
\pysiglinewithargsret{\sphinxbfcode{\sphinxupquote{join}}}{\sphinxparam{\DUrole{n,n}{other}\DUrole{p,p}{:}\DUrole{w,w}{  }\DUrole{n,n}{{\hyperref[\detokenize{deepdrr.geo:deepdrr.geo.core.Point2D}]{\sphinxcrossref{Point2D}}}}}}{{ $\rightarrow$ {\hyperref[\detokenize{deepdrr.geo:deepdrr.geo.Line2D}]{\sphinxcrossref{Line2D}}}}}
\pysiglinewithargsret{\sphinxbfcode{\sphinxupquote{join}}}{\sphinxparam{\DUrole{n,n}{other}\DUrole{p,p}{:}\DUrole{w,w}{  }\DUrole{n,n}{{\hyperref[\detokenize{deepdrr.geo:deepdrr.geo.Line2D}]{\sphinxcrossref{Line2D}}}}}}{{ $\rightarrow$ {\hyperref[\detokenize{deepdrr.geo:deepdrr.geo.core.Vector2D}]{\sphinxcrossref{Vector2D}}}}}
\pysigstopsignatures
\sphinxAtStartPar
Join two objects.

\sphinxAtStartPar
For example, given two points, get the line that connects them.
\begin{quote}\begin{description}
\sphinxlineitem{Parameters}
\sphinxAtStartPar
\sphinxstyleliteralstrong{\sphinxupquote{other}} ({\hyperref[\detokenize{deepdrr.geo:deepdrr.geo.core.Primitive}]{\sphinxcrossref{\sphinxstyleliteralemphasis{\sphinxupquote{Primitive}}}}}) \textendash{} the other primitive.

\sphinxlineitem{Returns}
\sphinxAtStartPar
the joined primitive.

\sphinxlineitem{Return type}
\sphinxAtStartPar
{\hyperref[\detokenize{deepdrr.geo:deepdrr.geo.core.Primitive}]{\sphinxcrossref{Primitive}}}

\end{description}\end{quote}

\end{fulllineitems}


\end{fulllineitems}

\index{Point3D (class in deepdrr.geo)@\spxentry{Point3D}\spxextra{class in deepdrr.geo}}

\begin{fulllineitems}
\phantomsection\label{\detokenize{deepdrr.geo:deepdrr.geo.Point3D}}
\pysigstartsignatures
\pysiglinewithargsret{\sphinxbfcode{\sphinxupquote{class\DUrole{w,w}{  }}}\sphinxcode{\sphinxupquote{deepdrr.geo.}}\sphinxbfcode{\sphinxupquote{Point3D}}}{\sphinxparam{\DUrole{n,n}{data}\DUrole{p,p}{:}\DUrole{w,w}{  }\DUrole{n,n}{ndarray}}}{}
\pysigstopsignatures
\sphinxAtStartPar
Bases: {\hyperref[\detokenize{deepdrr.geo:deepdrr.geo.core.Point}]{\sphinxcrossref{\sphinxcode{\sphinxupquote{Point}}}}}

\sphinxAtStartPar
Homogeneous point in 3D, represented as an array with {[}x, y, z, 1{]}
\index{data (deepdrr.geo.Point3D attribute)@\spxentry{data}\spxextra{deepdrr.geo.Point3D attribute}}

\begin{fulllineitems}
\phantomsection\label{\detokenize{deepdrr.geo:deepdrr.geo.Point3D.data}}
\pysigstartsignatures
\pysigline{\sphinxbfcode{\sphinxupquote{data}}\sphinxbfcode{\sphinxupquote{\DUrole{p,p}{:}\DUrole{w,w}{  }ndarray}}}
\pysigstopsignatures
\end{fulllineitems}

\index{dim (deepdrr.geo.Point3D attribute)@\spxentry{dim}\spxextra{deepdrr.geo.Point3D attribute}}

\begin{fulllineitems}
\phantomsection\label{\detokenize{deepdrr.geo:deepdrr.geo.Point3D.dim}}
\pysigstartsignatures
\pysigline{\sphinxbfcode{\sphinxupquote{dim}}\sphinxbfcode{\sphinxupquote{\DUrole{w,w}{  }\DUrole{p,p}{=}\DUrole{w,w}{  }3}}}
\pysigstopsignatures
\end{fulllineitems}

\index{join() (deepdrr.geo.Point3D method)@\spxentry{join()}\spxextra{deepdrr.geo.Point3D method}}

\begin{fulllineitems}
\phantomsection\label{\detokenize{deepdrr.geo:deepdrr.geo.Point3D.join}}
\pysigstartsignatures
\pysiglinewithargsret{\sphinxbfcode{\sphinxupquote{join}}}{\sphinxparam{\DUrole{n,n}{other}\DUrole{p,p}{:}\DUrole{w,w}{  }\DUrole{n,n}{{\hyperref[\detokenize{deepdrr.geo:deepdrr.geo.core.Point3D}]{\sphinxcrossref{Point3D}}}}}}{{ $\rightarrow$ {\hyperref[\detokenize{deepdrr.geo:deepdrr.geo.Line3D}]{\sphinxcrossref{Line3D}}}}}
\pysiglinewithargsret{\sphinxbfcode{\sphinxupquote{join}}}{\sphinxparam{\DUrole{n,n}{other}\DUrole{p,p}{:}\DUrole{w,w}{  }\DUrole{n,n}{{\hyperref[\detokenize{deepdrr.geo:deepdrr.geo.Line3D}]{\sphinxcrossref{Line3D}}}}}}{{ $\rightarrow$ {\hyperref[\detokenize{deepdrr.geo:deepdrr.geo.Plane}]{\sphinxcrossref{Plane}}}}}
\pysigstopsignatures
\sphinxAtStartPar
Join two objects.

\sphinxAtStartPar
For example, given two points, get the line that connects them.
\begin{quote}\begin{description}
\sphinxlineitem{Parameters}
\sphinxAtStartPar
\sphinxstyleliteralstrong{\sphinxupquote{other}} ({\hyperref[\detokenize{deepdrr.geo:deepdrr.geo.core.Primitive}]{\sphinxcrossref{\sphinxstyleliteralemphasis{\sphinxupquote{Primitive}}}}}) \textendash{} the other primitive.

\sphinxlineitem{Returns}
\sphinxAtStartPar
the joined primitive.

\sphinxlineitem{Return type}
\sphinxAtStartPar
{\hyperref[\detokenize{deepdrr.geo:deepdrr.geo.core.Primitive}]{\sphinxcrossref{Primitive}}}

\end{description}\end{quote}

\end{fulllineitems}

\index{projection\_type() (deepdrr.geo.Point3D class method)@\spxentry{projection\_type()}\spxextra{deepdrr.geo.Point3D class method}}

\begin{fulllineitems}
\phantomsection\label{\detokenize{deepdrr.geo:deepdrr.geo.Point3D.projection_type}}
\pysigstartsignatures
\pysiglinewithargsret{\sphinxbfcode{\sphinxupquote{classmethod\DUrole{w,w}{  }}}\sphinxbfcode{\sphinxupquote{projection\_type}}}{}{{ $\rightarrow$ Type\DUrole{p,p}{{[}}{\hyperref[\detokenize{deepdrr.geo:deepdrr.geo.core.Point2D}]{\sphinxcrossref{Point2D}}}\DUrole{p,p}{{]}}}}
\pysigstopsignatures
\end{fulllineitems}


\end{fulllineitems}

\index{PointOrVector (class in deepdrr.geo)@\spxentry{PointOrVector}\spxextra{class in deepdrr.geo}}

\begin{fulllineitems}
\phantomsection\label{\detokenize{deepdrr.geo:deepdrr.geo.PointOrVector}}
\pysigstartsignatures
\pysiglinewithargsret{\sphinxbfcode{\sphinxupquote{class\DUrole{w,w}{  }}}\sphinxcode{\sphinxupquote{deepdrr.geo.}}\sphinxbfcode{\sphinxupquote{PointOrVector}}}{\sphinxparam{\DUrole{n,n}{data}\DUrole{p,p}{:}\DUrole{w,w}{  }\DUrole{n,n}{ndarray}}}{}
\pysigstopsignatures
\sphinxAtStartPar
Bases: {\hyperref[\detokenize{deepdrr.geo:deepdrr.geo.core.Primitive}]{\sphinxcrossref{\sphinxcode{\sphinxupquote{Primitive}}}}}

\sphinxAtStartPar
A Homogeneous point or vector in any dimension.
\index{data (deepdrr.geo.PointOrVector attribute)@\spxentry{data}\spxextra{deepdrr.geo.PointOrVector attribute}}

\begin{fulllineitems}
\phantomsection\label{\detokenize{deepdrr.geo:deepdrr.geo.PointOrVector.data}}
\pysigstartsignatures
\pysigline{\sphinxbfcode{\sphinxupquote{data}}\sphinxbfcode{\sphinxupquote{\DUrole{p,p}{:}\DUrole{w,w}{  }ndarray}}}
\pysigstopsignatures
\end{fulllineitems}

\index{norm() (deepdrr.geo.PointOrVector method)@\spxentry{norm()}\spxextra{deepdrr.geo.PointOrVector method}}

\begin{fulllineitems}
\phantomsection\label{\detokenize{deepdrr.geo:deepdrr.geo.PointOrVector.norm}}
\pysigstartsignatures
\pysiglinewithargsret{\sphinxbfcode{\sphinxupquote{norm}}}{\sphinxparam{\DUrole{o,o}{*}\DUrole{n,n}{args}}\sphinxparamcomma \sphinxparam{\DUrole{o,o}{**}\DUrole{n,n}{kwargs}}}{{ $\rightarrow$ float}}
\pysigstopsignatures
\sphinxAtStartPar
Get the norm of the vector. Pass any arguments to \sphinxtitleref{np.linalg.norm}.

\end{fulllineitems}

\index{normsqr() (deepdrr.geo.PointOrVector method)@\spxentry{normsqr()}\spxextra{deepdrr.geo.PointOrVector method}}

\begin{fulllineitems}
\phantomsection\label{\detokenize{deepdrr.geo:deepdrr.geo.PointOrVector.normsqr}}
\pysigstartsignatures
\pysiglinewithargsret{\sphinxbfcode{\sphinxupquote{normsqr}}}{\sphinxparam{\DUrole{n,n}{order}\DUrole{p,p}{:}\DUrole{w,w}{  }\DUrole{n,n}{int}\DUrole{w,w}{  }\DUrole{o,o}{=}\DUrole{w,w}{  }\DUrole{default_value}{2}}}{{ $\rightarrow$ float}}
\pysigstopsignatures
\sphinxAtStartPar
Get the squared L\sphinxhyphen{}order norm of the vector.

\end{fulllineitems}

\index{w (deepdrr.geo.PointOrVector property)@\spxentry{w}\spxextra{deepdrr.geo.PointOrVector property}}

\begin{fulllineitems}
\phantomsection\label{\detokenize{deepdrr.geo:deepdrr.geo.PointOrVector.w}}
\pysigstartsignatures
\pysigline{\sphinxbfcode{\sphinxupquote{property\DUrole{w,w}{  }}}\sphinxbfcode{\sphinxupquote{w}}\sphinxbfcode{\sphinxupquote{\DUrole{p,p}{:}\DUrole{w,w}{  }float}}}
\pysigstopsignatures
\end{fulllineitems}

\index{x (deepdrr.geo.PointOrVector property)@\spxentry{x}\spxextra{deepdrr.geo.PointOrVector property}}

\begin{fulllineitems}
\phantomsection\label{\detokenize{deepdrr.geo:deepdrr.geo.PointOrVector.x}}
\pysigstartsignatures
\pysigline{\sphinxbfcode{\sphinxupquote{property\DUrole{w,w}{  }}}\sphinxbfcode{\sphinxupquote{x}}\sphinxbfcode{\sphinxupquote{\DUrole{p,p}{:}\DUrole{w,w}{  }float}}}
\pysigstopsignatures
\end{fulllineitems}

\index{y (deepdrr.geo.PointOrVector property)@\spxentry{y}\spxextra{deepdrr.geo.PointOrVector property}}

\begin{fulllineitems}
\phantomsection\label{\detokenize{deepdrr.geo:deepdrr.geo.PointOrVector.y}}
\pysigstartsignatures
\pysigline{\sphinxbfcode{\sphinxupquote{property\DUrole{w,w}{  }}}\sphinxbfcode{\sphinxupquote{y}}\sphinxbfcode{\sphinxupquote{\DUrole{p,p}{:}\DUrole{w,w}{  }float}}}
\pysigstopsignatures
\end{fulllineitems}

\index{z (deepdrr.geo.PointOrVector property)@\spxentry{z}\spxextra{deepdrr.geo.PointOrVector property}}

\begin{fulllineitems}
\phantomsection\label{\detokenize{deepdrr.geo:deepdrr.geo.PointOrVector.z}}
\pysigstartsignatures
\pysigline{\sphinxbfcode{\sphinxupquote{property\DUrole{w,w}{  }}}\sphinxbfcode{\sphinxupquote{z}}\sphinxbfcode{\sphinxupquote{\DUrole{p,p}{:}\DUrole{w,w}{  }float}}}
\pysigstopsignatures
\end{fulllineitems}


\end{fulllineitems}

\index{Ray (class in deepdrr.geo)@\spxentry{Ray}\spxextra{class in deepdrr.geo}}

\begin{fulllineitems}
\phantomsection\label{\detokenize{deepdrr.geo:deepdrr.geo.Ray}}
\pysigstartsignatures
\pysiglinewithargsret{\sphinxbfcode{\sphinxupquote{class\DUrole{w,w}{  }}}\sphinxcode{\sphinxupquote{deepdrr.geo.}}\sphinxbfcode{\sphinxupquote{Ray}}}{\sphinxparam{\DUrole{n,n}{data}\DUrole{p,p}{:}\DUrole{w,w}{  }\DUrole{n,n}{ndarray}}}{}
\pysigstopsignatures
\sphinxAtStartPar
Bases: {\hyperref[\detokenize{deepdrr.geo:deepdrr.geo.core.HasLocationAndDirection}]{\sphinxcrossref{\sphinxcode{\sphinxupquote{HasLocationAndDirection}}}}}, {\hyperref[\detokenize{deepdrr.geo:deepdrr.geo.core.Meetable}]{\sphinxcrossref{\sphinxcode{\sphinxupquote{Meetable}}}}}
\index{data (deepdrr.geo.Ray attribute)@\spxentry{data}\spxextra{deepdrr.geo.Ray attribute}}

\begin{fulllineitems}
\phantomsection\label{\detokenize{deepdrr.geo:deepdrr.geo.Ray.data}}
\pysigstartsignatures
\pysigline{\sphinxbfcode{\sphinxupquote{data}}\sphinxbfcode{\sphinxupquote{\DUrole{p,p}{:}\DUrole{w,w}{  }ndarray}}}
\pysigstopsignatures
\end{fulllineitems}

\index{from\_pn() (deepdrr.geo.Ray class method)@\spxentry{from\_pn()}\spxextra{deepdrr.geo.Ray class method}}

\begin{fulllineitems}
\phantomsection\label{\detokenize{deepdrr.geo:deepdrr.geo.Ray.from_pn}}
\pysigstartsignatures
\pysiglinewithargsret{\sphinxbfcode{\sphinxupquote{classmethod\DUrole{w,w}{  }}}\sphinxbfcode{\sphinxupquote{from\_pn}}}{\sphinxparam{\DUrole{n,n}{p}\DUrole{p,p}{:}\DUrole{w,w}{  }\DUrole{n,n}{{\hyperref[\detokenize{deepdrr.geo:deepdrr.geo.core.Point}]{\sphinxcrossref{Point}}}}}\sphinxparamcomma \sphinxparam{\DUrole{n,n}{d}\DUrole{p,p}{:}\DUrole{w,w}{  }\DUrole{n,n}{{\hyperref[\detokenize{deepdrr.geo:deepdrr.geo.core.Vector}]{\sphinxcrossref{Vector}}}}}}{{ $\rightarrow$ R}}
\pysigstopsignatures
\sphinxAtStartPar
Create a ray from a point and a direction.

\end{fulllineitems}

\index{from\_point\_direction() (deepdrr.geo.Ray class method)@\spxentry{from\_point\_direction()}\spxextra{deepdrr.geo.Ray class method}}

\begin{fulllineitems}
\phantomsection\label{\detokenize{deepdrr.geo:deepdrr.geo.Ray.from_point_direction}}
\pysigstartsignatures
\pysiglinewithargsret{\sphinxbfcode{\sphinxupquote{classmethod\DUrole{w,w}{  }}}\sphinxbfcode{\sphinxupquote{from\_point\_direction}}}{\sphinxparam{\DUrole{n,n}{p}\DUrole{p,p}{:}\DUrole{w,w}{  }\DUrole{n,n}{{\hyperref[\detokenize{deepdrr.geo:deepdrr.geo.core.Point}]{\sphinxcrossref{Point}}}}}\sphinxparamcomma \sphinxparam{\DUrole{n,n}{d}\DUrole{p,p}{:}\DUrole{w,w}{  }\DUrole{n,n}{{\hyperref[\detokenize{deepdrr.geo:deepdrr.geo.core.Vector}]{\sphinxcrossref{Vector}}}}}}{{ $\rightarrow$ {\hyperref[\detokenize{deepdrr.geo:deepdrr.geo.ray.Ray}]{\sphinxcrossref{Ray}}}}}
\pysigstopsignatures
\sphinxAtStartPar
Create a ray from a point and a direction.

\end{fulllineitems}

\index{from\_pq() (deepdrr.geo.Ray class method)@\spxentry{from\_pq()}\spxextra{deepdrr.geo.Ray class method}}

\begin{fulllineitems}
\phantomsection\label{\detokenize{deepdrr.geo:deepdrr.geo.Ray.from_pq}}
\pysigstartsignatures
\pysiglinewithargsret{\sphinxbfcode{\sphinxupquote{classmethod\DUrole{w,w}{  }}}\sphinxbfcode{\sphinxupquote{from\_pq}}}{\sphinxparam{\DUrole{n,n}{p}\DUrole{p,p}{:}\DUrole{w,w}{  }\DUrole{n,n}{{\hyperref[\detokenize{deepdrr.geo:deepdrr.geo.core.Point}]{\sphinxcrossref{Point}}}}}\sphinxparamcomma \sphinxparam{\DUrole{n,n}{q}\DUrole{p,p}{:}\DUrole{w,w}{  }\DUrole{n,n}{{\hyperref[\detokenize{deepdrr.geo:deepdrr.geo.core.Point}]{\sphinxcrossref{Point}}}}}}{{ $\rightarrow$ R}}
\pysigstopsignatures
\sphinxAtStartPar
Create a ray from two points.

\sphinxAtStartPar
The point q is not preserved in the ray.
\begin{quote}\begin{description}
\sphinxlineitem{Parameters}\begin{itemize}
\item {} 
\sphinxAtStartPar
\sphinxstyleliteralstrong{\sphinxupquote{p}} ({\hyperref[\detokenize{deepdrr.geo:deepdrr.geo.Point3D}]{\sphinxcrossref{\sphinxstyleliteralemphasis{\sphinxupquote{Point3D}}}}}) \textendash{} The origin of the ray.

\item {} 
\sphinxAtStartPar
\sphinxstyleliteralstrong{\sphinxupquote{q}} ({\hyperref[\detokenize{deepdrr.geo:deepdrr.geo.Point3D}]{\sphinxcrossref{\sphinxstyleliteralemphasis{\sphinxupquote{Point3D}}}}}) \textendash{} A point on the ray.

\end{itemize}

\end{description}\end{quote}

\end{fulllineitems}

\index{get\_direction() (deepdrr.geo.Ray method)@\spxentry{get\_direction()}\spxextra{deepdrr.geo.Ray method}}

\begin{fulllineitems}
\phantomsection\label{\detokenize{deepdrr.geo:deepdrr.geo.Ray.get_direction}}
\pysigstartsignatures
\pysiglinewithargsret{\sphinxbfcode{\sphinxupquote{get\_direction}}}{}{{ $\rightarrow$ {\hyperref[\detokenize{deepdrr.geo:deepdrr.geo.core.Vector3D}]{\sphinxcrossref{Vector3D}}}}}
\pysigstopsignatures
\sphinxAtStartPar
Get the direction associated with the object.
\begin{quote}\begin{description}
\sphinxlineitem{Returns}
\sphinxAtStartPar
the direction of the object.

\sphinxlineitem{Return type}
\sphinxAtStartPar
{\hyperref[\detokenize{deepdrr.geo:deepdrr.geo.Vector}]{\sphinxcrossref{Vector}}}

\end{description}\end{quote}

\end{fulllineitems}

\index{get\_point() (deepdrr.geo.Ray method)@\spxentry{get\_point()}\spxextra{deepdrr.geo.Ray method}}

\begin{fulllineitems}
\phantomsection\label{\detokenize{deepdrr.geo:deepdrr.geo.Ray.get_point}}
\pysigstartsignatures
\pysiglinewithargsret{\sphinxbfcode{\sphinxupquote{get\_point}}}{}{{ $\rightarrow$ {\hyperref[\detokenize{deepdrr.geo:deepdrr.geo.core.Point3D}]{\sphinxcrossref{Point3D}}}}}
\pysigstopsignatures
\sphinxAtStartPar
Get the location of the object.
\begin{quote}\begin{description}
\sphinxlineitem{Returns}
\sphinxAtStartPar
the location of the object.

\sphinxlineitem{Return type}
\sphinxAtStartPar
{\hyperref[\detokenize{deepdrr.geo:deepdrr.geo.Point}]{\sphinxcrossref{Point}}}

\end{description}\end{quote}

\end{fulllineitems}

\index{n (deepdrr.geo.Ray property)@\spxentry{n}\spxextra{deepdrr.geo.Ray property}}

\begin{fulllineitems}
\phantomsection\label{\detokenize{deepdrr.geo:deepdrr.geo.Ray.n}}
\pysigstartsignatures
\pysigline{\sphinxbfcode{\sphinxupquote{property\DUrole{w,w}{  }}}\sphinxbfcode{\sphinxupquote{n}}\sphinxbfcode{\sphinxupquote{\DUrole{p,p}{:}\DUrole{w,w}{  }{\hyperref[\detokenize{deepdrr.geo:deepdrr.geo.core.Vector}]{\sphinxcrossref{Vector}}}}}}
\pysigstopsignatures
\end{fulllineitems}

\index{p (deepdrr.geo.Ray property)@\spxentry{p}\spxextra{deepdrr.geo.Ray property}}

\begin{fulllineitems}
\phantomsection\label{\detokenize{deepdrr.geo:deepdrr.geo.Ray.p}}
\pysigstartsignatures
\pysigline{\sphinxbfcode{\sphinxupquote{property\DUrole{w,w}{  }}}\sphinxbfcode{\sphinxupquote{p}}\sphinxbfcode{\sphinxupquote{\DUrole{p,p}{:}\DUrole{w,w}{  }{\hyperref[\detokenize{deepdrr.geo:deepdrr.geo.core.Point3D}]{\sphinxcrossref{Point3D}}}}}}
\pysigstopsignatures
\end{fulllineitems}


\end{fulllineitems}

\index{Ray3D (class in deepdrr.geo)@\spxentry{Ray3D}\spxextra{class in deepdrr.geo}}

\begin{fulllineitems}
\phantomsection\label{\detokenize{deepdrr.geo:deepdrr.geo.Ray3D}}
\pysigstartsignatures
\pysiglinewithargsret{\sphinxbfcode{\sphinxupquote{class\DUrole{w,w}{  }}}\sphinxcode{\sphinxupquote{deepdrr.geo.}}\sphinxbfcode{\sphinxupquote{Ray3D}}}{\sphinxparam{\DUrole{n,n}{data}\DUrole{p,p}{:}\DUrole{w,w}{  }\DUrole{n,n}{ndarray}}}{}
\pysigstopsignatures
\sphinxAtStartPar
Bases: {\hyperref[\detokenize{deepdrr.geo:deepdrr.geo.ray.Ray}]{\sphinxcrossref{\sphinxcode{\sphinxupquote{Ray}}}}}, {\hyperref[\detokenize{deepdrr.geo:deepdrr.geo.core.Joinable}]{\sphinxcrossref{\sphinxcode{\sphinxupquote{Joinable}}}}}, {\hyperref[\detokenize{deepdrr.geo:deepdrr.geo.core.HasProjection}]{\sphinxcrossref{\sphinxcode{\sphinxupquote{HasProjection}}}}}

\sphinxAtStartPar
A homogeneous representation of a ray.

\sphinxAtStartPar
This is just a (4,2) array with the homogeneous coordinates of the
origin and the direction, respectively.
\index{data (deepdrr.geo.Ray3D attribute)@\spxentry{data}\spxextra{deepdrr.geo.Ray3D attribute}}

\begin{fulllineitems}
\phantomsection\label{\detokenize{deepdrr.geo:deepdrr.geo.Ray3D.data}}
\pysigstartsignatures
\pysigline{\sphinxbfcode{\sphinxupquote{data}}\sphinxbfcode{\sphinxupquote{\DUrole{p,p}{:}\DUrole{w,w}{  }ndarray}}}
\pysigstopsignatures
\end{fulllineitems}

\index{dim (deepdrr.geo.Ray3D attribute)@\spxentry{dim}\spxextra{deepdrr.geo.Ray3D attribute}}

\begin{fulllineitems}
\phantomsection\label{\detokenize{deepdrr.geo:deepdrr.geo.Ray3D.dim}}
\pysigstartsignatures
\pysigline{\sphinxbfcode{\sphinxupquote{dim}}\sphinxbfcode{\sphinxupquote{\DUrole{w,w}{  }\DUrole{p,p}{=}\DUrole{w,w}{  }3}}}
\pysigstopsignatures
\end{fulllineitems}

\index{join() (deepdrr.geo.Ray3D method)@\spxentry{join()}\spxextra{deepdrr.geo.Ray3D method}}

\begin{fulllineitems}
\phantomsection\label{\detokenize{deepdrr.geo:deepdrr.geo.Ray3D.join}}
\pysigstartsignatures
\pysiglinewithargsret{\sphinxbfcode{\sphinxupquote{join}}}{\sphinxparam{\DUrole{n,n}{other}\DUrole{p,p}{:}\DUrole{w,w}{  }\DUrole{n,n}{{\hyperref[\detokenize{deepdrr.geo:deepdrr.geo.Point3D}]{\sphinxcrossref{Point3D}}}}}}{{ $\rightarrow$ {\hyperref[\detokenize{deepdrr.geo:deepdrr.geo.Plane}]{\sphinxcrossref{Plane}}}}}
\pysigstopsignatures
\sphinxAtStartPar
Join two objects.

\sphinxAtStartPar
For example, given two points, get the line that connects them.
\begin{quote}\begin{description}
\sphinxlineitem{Parameters}
\sphinxAtStartPar
\sphinxstyleliteralstrong{\sphinxupquote{other}} ({\hyperref[\detokenize{deepdrr.geo:deepdrr.geo.core.Primitive}]{\sphinxcrossref{\sphinxstyleliteralemphasis{\sphinxupquote{Primitive}}}}}) \textendash{} the other primitive.

\sphinxlineitem{Returns}
\sphinxAtStartPar
the joined primitive.

\sphinxlineitem{Return type}
\sphinxAtStartPar
{\hyperref[\detokenize{deepdrr.geo:deepdrr.geo.core.Primitive}]{\sphinxcrossref{Primitive}}}

\end{description}\end{quote}

\end{fulllineitems}

\index{meet() (deepdrr.geo.Ray3D method)@\spxentry{meet()}\spxextra{deepdrr.geo.Ray3D method}}

\begin{fulllineitems}
\phantomsection\label{\detokenize{deepdrr.geo:deepdrr.geo.Ray3D.meet}}
\pysigstartsignatures
\pysiglinewithargsret{\sphinxbfcode{\sphinxupquote{meet}}}{\sphinxparam{\DUrole{n,n}{other}\DUrole{p,p}{:}\DUrole{w,w}{  }\DUrole{n,n}{{\hyperref[\detokenize{deepdrr.geo:deepdrr.geo.Plane}]{\sphinxcrossref{Plane}}}}}}{{ $\rightarrow$ {\hyperref[\detokenize{deepdrr.geo:deepdrr.geo.Point3D}]{\sphinxcrossref{Point3D}}}}}
\pysigstopsignatures
\sphinxAtStartPar
Get the intersection of two objects.

\sphinxAtStartPar
For example, given two lines, get the line that is the intersection of them.
\begin{quote}\begin{description}
\sphinxlineitem{Parameters}
\sphinxAtStartPar
\sphinxstyleliteralstrong{\sphinxupquote{other}} ({\hyperref[\detokenize{deepdrr.geo:deepdrr.geo.core.Primitive}]{\sphinxcrossref{\sphinxstyleliteralemphasis{\sphinxupquote{Primitive}}}}}) \textendash{} the other primitive.

\sphinxlineitem{Returns}
\sphinxAtStartPar
the intersection of \sphinxtitleref{self} and \sphinxtitleref{other}.

\sphinxlineitem{Return type}
\sphinxAtStartPar
{\hyperref[\detokenize{deepdrr.geo:deepdrr.geo.core.Primitive}]{\sphinxcrossref{Primitive}}}

\sphinxlineitem{Raises}
\sphinxAtStartPar
{\hyperref[\detokenize{deepdrr.geo:deepdrr.geo.MeetError}]{\sphinxcrossref{\sphinxstyleliteralstrong{\sphinxupquote{MeetError}}}}} \textendash{} if the objects cannot be intersected.

\end{description}\end{quote}

\end{fulllineitems}

\index{projection\_type() (deepdrr.geo.Ray3D class method)@\spxentry{projection\_type()}\spxextra{deepdrr.geo.Ray3D class method}}

\begin{fulllineitems}
\phantomsection\label{\detokenize{deepdrr.geo:deepdrr.geo.Ray3D.projection_type}}
\pysigstartsignatures
\pysiglinewithargsret{\sphinxbfcode{\sphinxupquote{classmethod\DUrole{w,w}{  }}}\sphinxbfcode{\sphinxupquote{projection\_type}}}{}{{ $\rightarrow$ Type\DUrole{p,p}{{[}}{\hyperref[\detokenize{deepdrr.geo:deepdrr.geo.ray.Ray2D}]{\sphinxcrossref{Ray2D}}}\DUrole{p,p}{{]}}}}
\pysigstopsignatures
\sphinxAtStartPar
Get the type of the projection of the object.
\begin{quote}\begin{description}
\sphinxlineitem{Returns}
\sphinxAtStartPar
the type of the projection of the object.

\sphinxlineitem{Return type}
\sphinxAtStartPar
Type{[}{\hyperref[\detokenize{deepdrr.geo:deepdrr.geo.core.Primitive}]{\sphinxcrossref{Primitive}}}{]}

\end{description}\end{quote}

\end{fulllineitems}


\end{fulllineitems}

\index{Rotation (class in deepdrr.geo)@\spxentry{Rotation}\spxextra{class in deepdrr.geo}}

\begin{fulllineitems}
\phantomsection\label{\detokenize{deepdrr.geo:deepdrr.geo.Rotation}}
\pysigstartsignatures
\pysigline{\sphinxbfcode{\sphinxupquote{class\DUrole{w,w}{  }}}\sphinxcode{\sphinxupquote{deepdrr.geo.}}\sphinxbfcode{\sphinxupquote{Rotation}}}
\pysigstopsignatures
\sphinxAtStartPar
Bases: \sphinxcode{\sphinxupquote{object}}

\sphinxAtStartPar
Rotation in 3 dimensions.

\sphinxAtStartPar
This class provides an interface to initialize from and represent rotations
with:
\begin{itemize}
\item {} 
\sphinxAtStartPar
Quaternions

\item {} 
\sphinxAtStartPar
Rotation Matrices

\item {} 
\sphinxAtStartPar
Rotation Vectors

\item {} 
\sphinxAtStartPar
Modified Rodrigues Parameters

\item {} 
\sphinxAtStartPar
Euler Angles

\end{itemize}

\sphinxAtStartPar
The following operations on rotations are supported:
\begin{itemize}
\item {} 
\sphinxAtStartPar
Application on vectors

\item {} 
\sphinxAtStartPar
Rotation Composition

\item {} 
\sphinxAtStartPar
Rotation Inversion

\item {} 
\sphinxAtStartPar
Rotation Indexing

\end{itemize}

\sphinxAtStartPar
Indexing within a rotation is supported since multiple rotation transforms
can be stored within a single \sphinxtitleref{Rotation} instance.

\sphinxAtStartPar
To create \sphinxtitleref{Rotation} objects use \sphinxcode{\sphinxupquote{from\_...}} methods (see examples below).
\sphinxcode{\sphinxupquote{Rotation(...)}} is not supposed to be instantiated directly.
\index{single (deepdrr.geo.Rotation attribute)@\spxentry{single}\spxextra{deepdrr.geo.Rotation attribute}}

\begin{fulllineitems}
\phantomsection\label{\detokenize{deepdrr.geo:deepdrr.geo.Rotation.single}}
\pysigstartsignatures
\pysigline{\sphinxbfcode{\sphinxupquote{single}}}
\pysigstopsignatures
\end{fulllineitems}

\index{\_\_len\_\_() (deepdrr.geo.Rotation method)@\spxentry{\_\_len\_\_()}\spxextra{deepdrr.geo.Rotation method}}

\begin{fulllineitems}
\phantomsection\label{\detokenize{deepdrr.geo:deepdrr.geo.Rotation.__len__}}
\pysigstartsignatures
\pysiglinewithargsret{\sphinxbfcode{\sphinxupquote{\_\_len\_\_}}}{}{}
\pysigstopsignatures
\end{fulllineitems}

\index{from\_quat() (deepdrr.geo.Rotation method)@\spxentry{from\_quat()}\spxextra{deepdrr.geo.Rotation method}}

\begin{fulllineitems}
\phantomsection\label{\detokenize{deepdrr.geo:deepdrr.geo.Rotation.from_quat}}
\pysigstartsignatures
\pysiglinewithargsret{\sphinxbfcode{\sphinxupquote{from\_quat}}}{}{}
\pysigstopsignatures
\end{fulllineitems}

\index{from\_matrix() (deepdrr.geo.Rotation method)@\spxentry{from\_matrix()}\spxextra{deepdrr.geo.Rotation method}}

\begin{fulllineitems}
\phantomsection\label{\detokenize{deepdrr.geo:deepdrr.geo.Rotation.from_matrix}}
\pysigstartsignatures
\pysiglinewithargsret{\sphinxbfcode{\sphinxupquote{from\_matrix}}}{}{}
\pysigstopsignatures
\end{fulllineitems}

\index{from\_rotvec() (deepdrr.geo.Rotation method)@\spxentry{from\_rotvec()}\spxextra{deepdrr.geo.Rotation method}}

\begin{fulllineitems}
\phantomsection\label{\detokenize{deepdrr.geo:deepdrr.geo.Rotation.from_rotvec}}
\pysigstartsignatures
\pysiglinewithargsret{\sphinxbfcode{\sphinxupquote{from\_rotvec}}}{}{}
\pysigstopsignatures
\end{fulllineitems}

\index{from\_mrp() (deepdrr.geo.Rotation method)@\spxentry{from\_mrp()}\spxextra{deepdrr.geo.Rotation method}}

\begin{fulllineitems}
\phantomsection\label{\detokenize{deepdrr.geo:deepdrr.geo.Rotation.from_mrp}}
\pysigstartsignatures
\pysiglinewithargsret{\sphinxbfcode{\sphinxupquote{from\_mrp}}}{}{}
\pysigstopsignatures
\end{fulllineitems}

\index{from\_euler() (deepdrr.geo.Rotation method)@\spxentry{from\_euler()}\spxextra{deepdrr.geo.Rotation method}}

\begin{fulllineitems}
\phantomsection\label{\detokenize{deepdrr.geo:deepdrr.geo.Rotation.from_euler}}
\pysigstartsignatures
\pysiglinewithargsret{\sphinxbfcode{\sphinxupquote{from\_euler}}}{}{}
\pysigstopsignatures
\end{fulllineitems}

\index{as\_quat() (deepdrr.geo.Rotation method)@\spxentry{as\_quat()}\spxextra{deepdrr.geo.Rotation method}}

\begin{fulllineitems}
\phantomsection\label{\detokenize{deepdrr.geo:deepdrr.geo.Rotation.as_quat}}
\pysigstartsignatures
\pysiglinewithargsret{\sphinxbfcode{\sphinxupquote{as\_quat}}}{}{}
\pysigstopsignatures
\end{fulllineitems}

\index{as\_matrix() (deepdrr.geo.Rotation method)@\spxentry{as\_matrix()}\spxextra{deepdrr.geo.Rotation method}}

\begin{fulllineitems}
\phantomsection\label{\detokenize{deepdrr.geo:deepdrr.geo.Rotation.as_matrix}}
\pysigstartsignatures
\pysiglinewithargsret{\sphinxbfcode{\sphinxupquote{as\_matrix}}}{}{}
\pysigstopsignatures
\end{fulllineitems}

\index{as\_rotvec() (deepdrr.geo.Rotation method)@\spxentry{as\_rotvec()}\spxextra{deepdrr.geo.Rotation method}}

\begin{fulllineitems}
\phantomsection\label{\detokenize{deepdrr.geo:deepdrr.geo.Rotation.as_rotvec}}
\pysigstartsignatures
\pysiglinewithargsret{\sphinxbfcode{\sphinxupquote{as\_rotvec}}}{}{}
\pysigstopsignatures
\end{fulllineitems}

\index{as\_mrp() (deepdrr.geo.Rotation method)@\spxentry{as\_mrp()}\spxextra{deepdrr.geo.Rotation method}}

\begin{fulllineitems}
\phantomsection\label{\detokenize{deepdrr.geo:deepdrr.geo.Rotation.as_mrp}}
\pysigstartsignatures
\pysiglinewithargsret{\sphinxbfcode{\sphinxupquote{as\_mrp}}}{}{}
\pysigstopsignatures
\end{fulllineitems}

\index{as\_euler() (deepdrr.geo.Rotation method)@\spxentry{as\_euler()}\spxextra{deepdrr.geo.Rotation method}}

\begin{fulllineitems}
\phantomsection\label{\detokenize{deepdrr.geo:deepdrr.geo.Rotation.as_euler}}
\pysigstartsignatures
\pysiglinewithargsret{\sphinxbfcode{\sphinxupquote{as\_euler}}}{}{}
\pysigstopsignatures
\end{fulllineitems}

\index{concatenate() (deepdrr.geo.Rotation method)@\spxentry{concatenate()}\spxextra{deepdrr.geo.Rotation method}}

\begin{fulllineitems}
\phantomsection\label{\detokenize{deepdrr.geo:deepdrr.geo.Rotation.concatenate}}
\pysigstartsignatures
\pysiglinewithargsret{\sphinxbfcode{\sphinxupquote{concatenate}}}{}{}
\pysigstopsignatures
\end{fulllineitems}

\index{apply() (deepdrr.geo.Rotation method)@\spxentry{apply()}\spxextra{deepdrr.geo.Rotation method}}

\begin{fulllineitems}
\phantomsection\label{\detokenize{deepdrr.geo:deepdrr.geo.Rotation.apply}}
\pysigstartsignatures
\pysiglinewithargsret{\sphinxbfcode{\sphinxupquote{apply}}}{}{}
\pysigstopsignatures
\end{fulllineitems}

\index{\_\_mul\_\_() (deepdrr.geo.Rotation method)@\spxentry{\_\_mul\_\_()}\spxextra{deepdrr.geo.Rotation method}}

\begin{fulllineitems}
\phantomsection\label{\detokenize{deepdrr.geo:deepdrr.geo.Rotation.__mul__}}
\pysigstartsignatures
\pysiglinewithargsret{\sphinxbfcode{\sphinxupquote{\_\_mul\_\_}}}{}{}
\pysigstopsignatures
\end{fulllineitems}

\index{inv() (deepdrr.geo.Rotation method)@\spxentry{inv()}\spxextra{deepdrr.geo.Rotation method}}

\begin{fulllineitems}
\phantomsection\label{\detokenize{deepdrr.geo:deepdrr.geo.Rotation.inv}}
\pysigstartsignatures
\pysiglinewithargsret{\sphinxbfcode{\sphinxupquote{inv}}}{}{}
\pysigstopsignatures
\end{fulllineitems}

\index{magnitude() (deepdrr.geo.Rotation method)@\spxentry{magnitude()}\spxextra{deepdrr.geo.Rotation method}}

\begin{fulllineitems}
\phantomsection\label{\detokenize{deepdrr.geo:deepdrr.geo.Rotation.magnitude}}
\pysigstartsignatures
\pysiglinewithargsret{\sphinxbfcode{\sphinxupquote{magnitude}}}{}{}
\pysigstopsignatures
\end{fulllineitems}

\index{mean() (deepdrr.geo.Rotation method)@\spxentry{mean()}\spxextra{deepdrr.geo.Rotation method}}

\begin{fulllineitems}
\phantomsection\label{\detokenize{deepdrr.geo:deepdrr.geo.Rotation.mean}}
\pysigstartsignatures
\pysiglinewithargsret{\sphinxbfcode{\sphinxupquote{mean}}}{}{}
\pysigstopsignatures
\end{fulllineitems}

\index{reduce() (deepdrr.geo.Rotation method)@\spxentry{reduce()}\spxextra{deepdrr.geo.Rotation method}}

\begin{fulllineitems}
\phantomsection\label{\detokenize{deepdrr.geo:deepdrr.geo.Rotation.reduce}}
\pysigstartsignatures
\pysiglinewithargsret{\sphinxbfcode{\sphinxupquote{reduce}}}{}{}
\pysigstopsignatures
\end{fulllineitems}

\index{create\_group() (deepdrr.geo.Rotation method)@\spxentry{create\_group()}\spxextra{deepdrr.geo.Rotation method}}

\begin{fulllineitems}
\phantomsection\label{\detokenize{deepdrr.geo:deepdrr.geo.Rotation.create_group}}
\pysigstartsignatures
\pysiglinewithargsret{\sphinxbfcode{\sphinxupquote{create\_group}}}{}{}
\pysigstopsignatures
\end{fulllineitems}

\index{\_\_getitem\_\_() (deepdrr.geo.Rotation method)@\spxentry{\_\_getitem\_\_()}\spxextra{deepdrr.geo.Rotation method}}

\begin{fulllineitems}
\phantomsection\label{\detokenize{deepdrr.geo:deepdrr.geo.Rotation.__getitem__}}
\pysigstartsignatures
\pysiglinewithargsret{\sphinxbfcode{\sphinxupquote{\_\_getitem\_\_}}}{}{}
\pysigstopsignatures
\end{fulllineitems}

\index{identity() (deepdrr.geo.Rotation method)@\spxentry{identity()}\spxextra{deepdrr.geo.Rotation method}}

\begin{fulllineitems}
\phantomsection\label{\detokenize{deepdrr.geo:deepdrr.geo.Rotation.identity}}
\pysigstartsignatures
\pysiglinewithargsret{\sphinxbfcode{\sphinxupquote{identity}}}{}{}
\pysigstopsignatures
\end{fulllineitems}

\index{random() (deepdrr.geo.Rotation method)@\spxentry{random()}\spxextra{deepdrr.geo.Rotation method}}

\begin{fulllineitems}
\phantomsection\label{\detokenize{deepdrr.geo:deepdrr.geo.Rotation.random}}
\pysigstartsignatures
\pysiglinewithargsret{\sphinxbfcode{\sphinxupquote{random}}}{}{}
\pysigstopsignatures
\end{fulllineitems}

\index{align\_vectors() (deepdrr.geo.Rotation method)@\spxentry{align\_vectors()}\spxextra{deepdrr.geo.Rotation method}}

\begin{fulllineitems}
\phantomsection\label{\detokenize{deepdrr.geo:deepdrr.geo.Rotation.align_vectors}}
\pysigstartsignatures
\pysiglinewithargsret{\sphinxbfcode{\sphinxupquote{align\_vectors}}}{}{}
\pysigstopsignatures
\end{fulllineitems}



\begin{sphinxseealso}{See also:}

\sphinxAtStartPar
\sphinxcode{\sphinxupquote{Slerp}}


\end{sphinxseealso}

\subsubsection*{Notes}
\subsubsection*{Examples}

\begin{sphinxVerbatim}[commandchars=\\\{\}]
\PYG{g+gp}{\PYGZgt{}\PYGZgt{}\PYGZgt{} }\PYG{k+kn}{from} \PYG{n+nn}{scipy}\PYG{n+nn}{.}\PYG{n+nn}{spatial}\PYG{n+nn}{.}\PYG{n+nn}{transform} \PYG{k+kn}{import} \PYG{n}{Rotation} \PYG{k}{as} \PYG{n}{R}
\PYG{g+gp}{\PYGZgt{}\PYGZgt{}\PYGZgt{} }\PYG{k+kn}{import} \PYG{n+nn}{numpy} \PYG{k}{as} \PYG{n+nn}{np}
\end{sphinxVerbatim}

\sphinxAtStartPar
A \sphinxtitleref{Rotation} instance can be initialized in any of the above formats and
converted to any of the others. The underlying object is independent of the
representation used for initialization.

\sphinxAtStartPar
Consider a counter\sphinxhyphen{}clockwise rotation of 90 degrees about the z\sphinxhyphen{}axis. This
corresponds to the following quaternion (in scalar\sphinxhyphen{}last format):

\begin{sphinxVerbatim}[commandchars=\\\{\}]
\PYG{g+gp}{\PYGZgt{}\PYGZgt{}\PYGZgt{} }\PYG{n}{r} \PYG{o}{=} \PYG{n}{R}\PYG{o}{.}\PYG{n}{from\PYGZus{}quat}\PYG{p}{(}\PYG{p}{[}\PYG{l+m+mi}{0}\PYG{p}{,} \PYG{l+m+mi}{0}\PYG{p}{,} \PYG{n}{np}\PYG{o}{.}\PYG{n}{sin}\PYG{p}{(}\PYG{n}{np}\PYG{o}{.}\PYG{n}{pi}\PYG{o}{/}\PYG{l+m+mi}{4}\PYG{p}{)}\PYG{p}{,} \PYG{n}{np}\PYG{o}{.}\PYG{n}{cos}\PYG{p}{(}\PYG{n}{np}\PYG{o}{.}\PYG{n}{pi}\PYG{o}{/}\PYG{l+m+mi}{4}\PYG{p}{)}\PYG{p}{]}\PYG{p}{)}
\end{sphinxVerbatim}

\sphinxAtStartPar
The rotation can be expressed in any of the other formats:

\begin{sphinxVerbatim}[commandchars=\\\{\}]
\PYG{g+gp}{\PYGZgt{}\PYGZgt{}\PYGZgt{} }\PYG{n}{r}\PYG{o}{.}\PYG{n}{as\PYGZus{}matrix}\PYG{p}{(}\PYG{p}{)}
\PYG{g+go}{array([[ 2.22044605e\PYGZhy{}16, \PYGZhy{}1.00000000e+00,  0.00000000e+00],}
\PYG{g+go}{[ 1.00000000e+00,  2.22044605e\PYGZhy{}16,  0.00000000e+00],}
\PYG{g+go}{[ 0.00000000e+00,  0.00000000e+00,  1.00000000e+00]])}
\PYG{g+gp}{\PYGZgt{}\PYGZgt{}\PYGZgt{} }\PYG{n}{r}\PYG{o}{.}\PYG{n}{as\PYGZus{}rotvec}\PYG{p}{(}\PYG{p}{)}
\PYG{g+go}{array([0.        , 0.        , 1.57079633])}
\PYG{g+gp}{\PYGZgt{}\PYGZgt{}\PYGZgt{} }\PYG{n}{r}\PYG{o}{.}\PYG{n}{as\PYGZus{}euler}\PYG{p}{(}\PYG{l+s+s1}{\PYGZsq{}}\PYG{l+s+s1}{zyx}\PYG{l+s+s1}{\PYGZsq{}}\PYG{p}{,} \PYG{n}{degrees}\PYG{o}{=}\PYG{k+kc}{True}\PYG{p}{)}
\PYG{g+go}{array([90.,  0.,  0.])}
\end{sphinxVerbatim}

\sphinxAtStartPar
The same rotation can be initialized using a rotation matrix:

\begin{sphinxVerbatim}[commandchars=\\\{\}]
\PYG{g+gp}{\PYGZgt{}\PYGZgt{}\PYGZgt{} }\PYG{n}{r} \PYG{o}{=} \PYG{n}{R}\PYG{o}{.}\PYG{n}{from\PYGZus{}matrix}\PYG{p}{(}\PYG{p}{[}\PYG{p}{[}\PYG{l+m+mi}{0}\PYG{p}{,} \PYG{o}{\PYGZhy{}}\PYG{l+m+mi}{1}\PYG{p}{,} \PYG{l+m+mi}{0}\PYG{p}{]}\PYG{p}{,}
\PYG{g+gp}{... }                   \PYG{p}{[}\PYG{l+m+mi}{1}\PYG{p}{,} \PYG{l+m+mi}{0}\PYG{p}{,} \PYG{l+m+mi}{0}\PYG{p}{]}\PYG{p}{,}
\PYG{g+gp}{... }                   \PYG{p}{[}\PYG{l+m+mi}{0}\PYG{p}{,} \PYG{l+m+mi}{0}\PYG{p}{,} \PYG{l+m+mi}{1}\PYG{p}{]}\PYG{p}{]}\PYG{p}{)}
\end{sphinxVerbatim}

\sphinxAtStartPar
Representation in other formats:

\begin{sphinxVerbatim}[commandchars=\\\{\}]
\PYG{g+gp}{\PYGZgt{}\PYGZgt{}\PYGZgt{} }\PYG{n}{r}\PYG{o}{.}\PYG{n}{as\PYGZus{}quat}\PYG{p}{(}\PYG{p}{)}
\PYG{g+go}{array([0.        , 0.        , 0.70710678, 0.70710678])}
\PYG{g+gp}{\PYGZgt{}\PYGZgt{}\PYGZgt{} }\PYG{n}{r}\PYG{o}{.}\PYG{n}{as\PYGZus{}rotvec}\PYG{p}{(}\PYG{p}{)}
\PYG{g+go}{array([0.        , 0.        , 1.57079633])}
\PYG{g+gp}{\PYGZgt{}\PYGZgt{}\PYGZgt{} }\PYG{n}{r}\PYG{o}{.}\PYG{n}{as\PYGZus{}euler}\PYG{p}{(}\PYG{l+s+s1}{\PYGZsq{}}\PYG{l+s+s1}{zyx}\PYG{l+s+s1}{\PYGZsq{}}\PYG{p}{,} \PYG{n}{degrees}\PYG{o}{=}\PYG{k+kc}{True}\PYG{p}{)}
\PYG{g+go}{array([90.,  0.,  0.])}
\end{sphinxVerbatim}

\sphinxAtStartPar
The rotation vector corresponding to this rotation is given by:

\begin{sphinxVerbatim}[commandchars=\\\{\}]
\PYG{g+gp}{\PYGZgt{}\PYGZgt{}\PYGZgt{} }\PYG{n}{r} \PYG{o}{=} \PYG{n}{R}\PYG{o}{.}\PYG{n}{from\PYGZus{}rotvec}\PYG{p}{(}\PYG{n}{np}\PYG{o}{.}\PYG{n}{pi}\PYG{o}{/}\PYG{l+m+mi}{2} \PYG{o}{*} \PYG{n}{np}\PYG{o}{.}\PYG{n}{array}\PYG{p}{(}\PYG{p}{[}\PYG{l+m+mi}{0}\PYG{p}{,} \PYG{l+m+mi}{0}\PYG{p}{,} \PYG{l+m+mi}{1}\PYG{p}{]}\PYG{p}{)}\PYG{p}{)}
\end{sphinxVerbatim}

\sphinxAtStartPar
Representation in other formats:

\begin{sphinxVerbatim}[commandchars=\\\{\}]
\PYG{g+gp}{\PYGZgt{}\PYGZgt{}\PYGZgt{} }\PYG{n}{r}\PYG{o}{.}\PYG{n}{as\PYGZus{}quat}\PYG{p}{(}\PYG{p}{)}
\PYG{g+go}{array([0.        , 0.        , 0.70710678, 0.70710678])}
\PYG{g+gp}{\PYGZgt{}\PYGZgt{}\PYGZgt{} }\PYG{n}{r}\PYG{o}{.}\PYG{n}{as\PYGZus{}matrix}\PYG{p}{(}\PYG{p}{)}
\PYG{g+go}{array([[ 2.22044605e\PYGZhy{}16, \PYGZhy{}1.00000000e+00,  0.00000000e+00],}
\PYG{g+go}{       [ 1.00000000e+00,  2.22044605e\PYGZhy{}16,  0.00000000e+00],}
\PYG{g+go}{       [ 0.00000000e+00,  0.00000000e+00,  1.00000000e+00]])}
\PYG{g+gp}{\PYGZgt{}\PYGZgt{}\PYGZgt{} }\PYG{n}{r}\PYG{o}{.}\PYG{n}{as\PYGZus{}euler}\PYG{p}{(}\PYG{l+s+s1}{\PYGZsq{}}\PYG{l+s+s1}{zyx}\PYG{l+s+s1}{\PYGZsq{}}\PYG{p}{,} \PYG{n}{degrees}\PYG{o}{=}\PYG{k+kc}{True}\PYG{p}{)}
\PYG{g+go}{array([90.,  0.,  0.])}
\end{sphinxVerbatim}

\sphinxAtStartPar
The \sphinxcode{\sphinxupquote{from\_euler}} method is quite flexible in the range of input formats
it supports. Here we initialize a single rotation about a single axis:

\begin{sphinxVerbatim}[commandchars=\\\{\}]
\PYG{g+gp}{\PYGZgt{}\PYGZgt{}\PYGZgt{} }\PYG{n}{r} \PYG{o}{=} \PYG{n}{R}\PYG{o}{.}\PYG{n}{from\PYGZus{}euler}\PYG{p}{(}\PYG{l+s+s1}{\PYGZsq{}}\PYG{l+s+s1}{z}\PYG{l+s+s1}{\PYGZsq{}}\PYG{p}{,} \PYG{l+m+mi}{90}\PYG{p}{,} \PYG{n}{degrees}\PYG{o}{=}\PYG{k+kc}{True}\PYG{p}{)}
\end{sphinxVerbatim}

\sphinxAtStartPar
Again, the object is representation independent and can be converted to any
other format:

\begin{sphinxVerbatim}[commandchars=\\\{\}]
\PYG{g+gp}{\PYGZgt{}\PYGZgt{}\PYGZgt{} }\PYG{n}{r}\PYG{o}{.}\PYG{n}{as\PYGZus{}quat}\PYG{p}{(}\PYG{p}{)}
\PYG{g+go}{array([0.        , 0.        , 0.70710678, 0.70710678])}
\PYG{g+gp}{\PYGZgt{}\PYGZgt{}\PYGZgt{} }\PYG{n}{r}\PYG{o}{.}\PYG{n}{as\PYGZus{}matrix}\PYG{p}{(}\PYG{p}{)}
\PYG{g+go}{array([[ 2.22044605e\PYGZhy{}16, \PYGZhy{}1.00000000e+00,  0.00000000e+00],}
\PYG{g+go}{       [ 1.00000000e+00,  2.22044605e\PYGZhy{}16,  0.00000000e+00],}
\PYG{g+go}{       [ 0.00000000e+00,  0.00000000e+00,  1.00000000e+00]])}
\PYG{g+gp}{\PYGZgt{}\PYGZgt{}\PYGZgt{} }\PYG{n}{r}\PYG{o}{.}\PYG{n}{as\PYGZus{}rotvec}\PYG{p}{(}\PYG{p}{)}
\PYG{g+go}{array([0.        , 0.        , 1.57079633])}
\end{sphinxVerbatim}

\sphinxAtStartPar
It is also possible to initialize multiple rotations in a single instance
using any of the \sphinxcode{\sphinxupquote{from\_...}} functions. Here we initialize a stack of 3
rotations using the \sphinxcode{\sphinxupquote{from\_euler}} method:

\begin{sphinxVerbatim}[commandchars=\\\{\}]
\PYG{g+gp}{\PYGZgt{}\PYGZgt{}\PYGZgt{} }\PYG{n}{r} \PYG{o}{=} \PYG{n}{R}\PYG{o}{.}\PYG{n}{from\PYGZus{}euler}\PYG{p}{(}\PYG{l+s+s1}{\PYGZsq{}}\PYG{l+s+s1}{zyx}\PYG{l+s+s1}{\PYGZsq{}}\PYG{p}{,} \PYG{p}{[}
\PYG{g+gp}{... }\PYG{p}{[}\PYG{l+m+mi}{90}\PYG{p}{,} \PYG{l+m+mi}{0}\PYG{p}{,} \PYG{l+m+mi}{0}\PYG{p}{]}\PYG{p}{,}
\PYG{g+gp}{... }\PYG{p}{[}\PYG{l+m+mi}{0}\PYG{p}{,} \PYG{l+m+mi}{45}\PYG{p}{,} \PYG{l+m+mi}{0}\PYG{p}{]}\PYG{p}{,}
\PYG{g+gp}{... }\PYG{p}{[}\PYG{l+m+mi}{45}\PYG{p}{,} \PYG{l+m+mi}{60}\PYG{p}{,} \PYG{l+m+mi}{30}\PYG{p}{]}\PYG{p}{]}\PYG{p}{,} \PYG{n}{degrees}\PYG{o}{=}\PYG{k+kc}{True}\PYG{p}{)}
\end{sphinxVerbatim}

\sphinxAtStartPar
The other representations also now return a stack of 3 rotations. For
example:

\begin{sphinxVerbatim}[commandchars=\\\{\}]
\PYG{g+gp}{\PYGZgt{}\PYGZgt{}\PYGZgt{} }\PYG{n}{r}\PYG{o}{.}\PYG{n}{as\PYGZus{}quat}\PYG{p}{(}\PYG{p}{)}
\PYG{g+go}{array([[0.        , 0.        , 0.70710678, 0.70710678],}
\PYG{g+go}{       [0.        , 0.38268343, 0.        , 0.92387953],}
\PYG{g+go}{       [0.39190384, 0.36042341, 0.43967974, 0.72331741]])}
\end{sphinxVerbatim}

\sphinxAtStartPar
Applying the above rotations onto a vector:

\begin{sphinxVerbatim}[commandchars=\\\{\}]
\PYG{g+gp}{\PYGZgt{}\PYGZgt{}\PYGZgt{} }\PYG{n}{v} \PYG{o}{=} \PYG{p}{[}\PYG{l+m+mi}{1}\PYG{p}{,} \PYG{l+m+mi}{2}\PYG{p}{,} \PYG{l+m+mi}{3}\PYG{p}{]}
\PYG{g+gp}{\PYGZgt{}\PYGZgt{}\PYGZgt{} }\PYG{n}{r}\PYG{o}{.}\PYG{n}{apply}\PYG{p}{(}\PYG{n}{v}\PYG{p}{)}
\PYG{g+go}{array([[\PYGZhy{}2.        ,  1.        ,  3.        ],}
\PYG{g+go}{       [ 2.82842712,  2.        ,  1.41421356],}
\PYG{g+go}{       [ 2.24452282,  0.78093109,  2.89002836]])}
\end{sphinxVerbatim}

\sphinxAtStartPar
A \sphinxtitleref{Rotation} instance can be indexed and sliced as if it were a single
1D array or list:

\begin{sphinxVerbatim}[commandchars=\\\{\}]
\PYG{g+gp}{\PYGZgt{}\PYGZgt{}\PYGZgt{} }\PYG{n}{r}\PYG{o}{.}\PYG{n}{as\PYGZus{}quat}\PYG{p}{(}\PYG{p}{)}
\PYG{g+go}{array([[0.        , 0.        , 0.70710678, 0.70710678],}
\PYG{g+go}{       [0.        , 0.38268343, 0.        , 0.92387953],}
\PYG{g+go}{       [0.39190384, 0.36042341, 0.43967974, 0.72331741]])}
\PYG{g+gp}{\PYGZgt{}\PYGZgt{}\PYGZgt{} }\PYG{n}{p} \PYG{o}{=} \PYG{n}{r}\PYG{p}{[}\PYG{l+m+mi}{0}\PYG{p}{]}
\PYG{g+gp}{\PYGZgt{}\PYGZgt{}\PYGZgt{} }\PYG{n}{p}\PYG{o}{.}\PYG{n}{as\PYGZus{}matrix}\PYG{p}{(}\PYG{p}{)}
\PYG{g+go}{array([[ 2.22044605e\PYGZhy{}16, \PYGZhy{}1.00000000e+00,  0.00000000e+00],}
\PYG{g+go}{       [ 1.00000000e+00,  2.22044605e\PYGZhy{}16,  0.00000000e+00],}
\PYG{g+go}{       [ 0.00000000e+00,  0.00000000e+00,  1.00000000e+00]])}
\PYG{g+gp}{\PYGZgt{}\PYGZgt{}\PYGZgt{} }\PYG{n}{q} \PYG{o}{=} \PYG{n}{r}\PYG{p}{[}\PYG{l+m+mi}{1}\PYG{p}{:}\PYG{l+m+mi}{3}\PYG{p}{]}
\PYG{g+gp}{\PYGZgt{}\PYGZgt{}\PYGZgt{} }\PYG{n}{q}\PYG{o}{.}\PYG{n}{as\PYGZus{}quat}\PYG{p}{(}\PYG{p}{)}
\PYG{g+go}{array([[0.        , 0.38268343, 0.        , 0.92387953],}
\PYG{g+go}{       [0.39190384, 0.36042341, 0.43967974, 0.72331741]])}
\end{sphinxVerbatim}

\sphinxAtStartPar
In fact it can be converted to numpy.array:

\begin{sphinxVerbatim}[commandchars=\\\{\}]
\PYG{g+gp}{\PYGZgt{}\PYGZgt{}\PYGZgt{} }\PYG{n}{r\PYGZus{}array} \PYG{o}{=} \PYG{n}{np}\PYG{o}{.}\PYG{n}{asarray}\PYG{p}{(}\PYG{n}{r}\PYG{p}{)}
\PYG{g+gp}{\PYGZgt{}\PYGZgt{}\PYGZgt{} }\PYG{n}{r\PYGZus{}array}\PYG{o}{.}\PYG{n}{shape}
\PYG{g+go}{(3,)}
\PYG{g+gp}{\PYGZgt{}\PYGZgt{}\PYGZgt{} }\PYG{n}{r\PYGZus{}array}\PYG{p}{[}\PYG{l+m+mi}{0}\PYG{p}{]}\PYG{o}{.}\PYG{n}{as\PYGZus{}matrix}\PYG{p}{(}\PYG{p}{)}
\PYG{g+go}{array([[ 2.22044605e\PYGZhy{}16, \PYGZhy{}1.00000000e+00,  0.00000000e+00],}
\PYG{g+go}{       [ 1.00000000e+00,  2.22044605e\PYGZhy{}16,  0.00000000e+00],}
\PYG{g+go}{       [ 0.00000000e+00,  0.00000000e+00,  1.00000000e+00]])}
\end{sphinxVerbatim}

\sphinxAtStartPar
Multiple rotations can be composed using the \sphinxcode{\sphinxupquote{*}} operator:

\begin{sphinxVerbatim}[commandchars=\\\{\}]
\PYG{g+gp}{\PYGZgt{}\PYGZgt{}\PYGZgt{} }\PYG{n}{r1} \PYG{o}{=} \PYG{n}{R}\PYG{o}{.}\PYG{n}{from\PYGZus{}euler}\PYG{p}{(}\PYG{l+s+s1}{\PYGZsq{}}\PYG{l+s+s1}{z}\PYG{l+s+s1}{\PYGZsq{}}\PYG{p}{,} \PYG{l+m+mi}{90}\PYG{p}{,} \PYG{n}{degrees}\PYG{o}{=}\PYG{k+kc}{True}\PYG{p}{)}
\PYG{g+gp}{\PYGZgt{}\PYGZgt{}\PYGZgt{} }\PYG{n}{r2} \PYG{o}{=} \PYG{n}{R}\PYG{o}{.}\PYG{n}{from\PYGZus{}rotvec}\PYG{p}{(}\PYG{p}{[}\PYG{n}{np}\PYG{o}{.}\PYG{n}{pi}\PYG{o}{/}\PYG{l+m+mi}{4}\PYG{p}{,} \PYG{l+m+mi}{0}\PYG{p}{,} \PYG{l+m+mi}{0}\PYG{p}{]}\PYG{p}{)}
\PYG{g+gp}{\PYGZgt{}\PYGZgt{}\PYGZgt{} }\PYG{n}{v} \PYG{o}{=} \PYG{p}{[}\PYG{l+m+mi}{1}\PYG{p}{,} \PYG{l+m+mi}{2}\PYG{p}{,} \PYG{l+m+mi}{3}\PYG{p}{]}
\PYG{g+gp}{\PYGZgt{}\PYGZgt{}\PYGZgt{} }\PYG{n}{r2}\PYG{o}{.}\PYG{n}{apply}\PYG{p}{(}\PYG{n}{r1}\PYG{o}{.}\PYG{n}{apply}\PYG{p}{(}\PYG{n}{v}\PYG{p}{)}\PYG{p}{)}
\PYG{g+go}{array([\PYGZhy{}2.        , \PYGZhy{}1.41421356,  2.82842712])}
\PYG{g+gp}{\PYGZgt{}\PYGZgt{}\PYGZgt{} }\PYG{n}{r3} \PYG{o}{=} \PYG{n}{r2} \PYG{o}{*} \PYG{n}{r1} \PYG{c+c1}{\PYGZsh{} Note the order}
\PYG{g+gp}{\PYGZgt{}\PYGZgt{}\PYGZgt{} }\PYG{n}{r3}\PYG{o}{.}\PYG{n}{apply}\PYG{p}{(}\PYG{n}{v}\PYG{p}{)}
\PYG{g+go}{array([\PYGZhy{}2.        , \PYGZhy{}1.41421356,  2.82842712])}
\end{sphinxVerbatim}

\sphinxAtStartPar
Finally, it is also possible to invert rotations:

\begin{sphinxVerbatim}[commandchars=\\\{\}]
\PYG{g+gp}{\PYGZgt{}\PYGZgt{}\PYGZgt{} }\PYG{n}{r1} \PYG{o}{=} \PYG{n}{R}\PYG{o}{.}\PYG{n}{from\PYGZus{}euler}\PYG{p}{(}\PYG{l+s+s1}{\PYGZsq{}}\PYG{l+s+s1}{z}\PYG{l+s+s1}{\PYGZsq{}}\PYG{p}{,} \PYG{p}{[}\PYG{l+m+mi}{90}\PYG{p}{,} \PYG{l+m+mi}{45}\PYG{p}{]}\PYG{p}{,} \PYG{n}{degrees}\PYG{o}{=}\PYG{k+kc}{True}\PYG{p}{)}
\PYG{g+gp}{\PYGZgt{}\PYGZgt{}\PYGZgt{} }\PYG{n}{r2} \PYG{o}{=} \PYG{n}{r1}\PYG{o}{.}\PYG{n}{inv}\PYG{p}{(}\PYG{p}{)}
\PYG{g+gp}{\PYGZgt{}\PYGZgt{}\PYGZgt{} }\PYG{n}{r2}\PYG{o}{.}\PYG{n}{as\PYGZus{}euler}\PYG{p}{(}\PYG{l+s+s1}{\PYGZsq{}}\PYG{l+s+s1}{zyx}\PYG{l+s+s1}{\PYGZsq{}}\PYG{p}{,} \PYG{n}{degrees}\PYG{o}{=}\PYG{k+kc}{True}\PYG{p}{)}
\PYG{g+go}{array([[\PYGZhy{}90.,   0.,   0.],}
\PYG{g+go}{       [\PYGZhy{}45.,   0.,   0.]])}
\end{sphinxVerbatim}

\sphinxAtStartPar
These examples serve as an overview into the \sphinxtitleref{Rotation} class and highlight
major functionalities. For more thorough examples of the range of input and
output formats supported, consult the individual method’s examples.
\index{align\_vectors() (deepdrr.geo.Rotation method)@\spxentry{align\_vectors()}\spxextra{deepdrr.geo.Rotation method}}

\begin{fulllineitems}
\phantomsection\label{\detokenize{deepdrr.geo:id0}}
\pysigstartsignatures
\pysiglinewithargsret{\sphinxbfcode{\sphinxupquote{align\_vectors}}}{\sphinxparam{\DUrole{n,n}{type cls}}\sphinxparamcomma \sphinxparam{\DUrole{n,n}{a}}\sphinxparamcomma \sphinxparam{\DUrole{n,n}{b}}\sphinxparamcomma \sphinxparam{\DUrole{n,n}{weights=None}}\sphinxparamcomma \sphinxparam{\DUrole{n,n}{return\_sensitivity=False}}}{}
\pysigstopsignatures
\sphinxAtStartPar
Estimate a rotation to optimally align two sets of vectors.

\sphinxAtStartPar
Find a rotation between frames A and B which best aligns a set of
vectors \sphinxtitleref{a} and \sphinxtitleref{b} observed in these frames. The following loss
function is minimized to solve for the rotation matrix
\(C\):
\begin{equation*}
\begin{split}L(C) = \frac{1}{2} \sum_{i = 1}^{n} w_i \lVert \mathbf{a}_i -
C \mathbf{b}_i \rVert^2 ,\end{split}
\end{equation*}
\sphinxAtStartPar
where \(w_i\)’s are the \sphinxtitleref{weights} corresponding to each vector.

\sphinxAtStartPar
The rotation is estimated with Kabsch algorithm {\color{red}\bfseries{}{[}1{]}\_}.
\begin{quote}\begin{description}
\sphinxlineitem{Parameters}\begin{itemize}
\item {} 
\sphinxAtStartPar
\sphinxstyleliteralstrong{\sphinxupquote{a}} (\sphinxstyleliteralemphasis{\sphinxupquote{array\_like}}\sphinxstyleliteralemphasis{\sphinxupquote{, }}\sphinxstyleliteralemphasis{\sphinxupquote{shape}}\sphinxstyleliteralemphasis{\sphinxupquote{ (}}\sphinxstyleliteralemphasis{\sphinxupquote{N}}\sphinxstyleliteralemphasis{\sphinxupquote{, }}\sphinxstyleliteralemphasis{\sphinxupquote{3}}\sphinxstyleliteralemphasis{\sphinxupquote{)}}) \textendash{} Vector components observed in initial frame A. Each row of \sphinxtitleref{a}
denotes a vector.

\item {} 
\sphinxAtStartPar
\sphinxstyleliteralstrong{\sphinxupquote{b}} (\sphinxstyleliteralemphasis{\sphinxupquote{array\_like}}\sphinxstyleliteralemphasis{\sphinxupquote{, }}\sphinxstyleliteralemphasis{\sphinxupquote{shape}}\sphinxstyleliteralemphasis{\sphinxupquote{ (}}\sphinxstyleliteralemphasis{\sphinxupquote{N}}\sphinxstyleliteralemphasis{\sphinxupquote{, }}\sphinxstyleliteralemphasis{\sphinxupquote{3}}\sphinxstyleliteralemphasis{\sphinxupquote{)}}) \textendash{} Vector components observed in another frame B. Each row of \sphinxtitleref{b}
denotes a vector.

\item {} 
\sphinxAtStartPar
\sphinxstyleliteralstrong{\sphinxupquote{weights}} (\sphinxstyleliteralemphasis{\sphinxupquote{array\_like shape}}\sphinxstyleliteralemphasis{\sphinxupquote{ (}}\sphinxstyleliteralemphasis{\sphinxupquote{N}}\sphinxstyleliteralemphasis{\sphinxupquote{,}}\sphinxstyleliteralemphasis{\sphinxupquote{)}}\sphinxstyleliteralemphasis{\sphinxupquote{, }}\sphinxstyleliteralemphasis{\sphinxupquote{optional}}) \textendash{} Weights describing the relative importance of the vector
observations. If None (default), then all values in \sphinxtitleref{weights} are
assumed to be 1.

\item {} 
\sphinxAtStartPar
\sphinxstyleliteralstrong{\sphinxupquote{return\_sensitivity}} (\sphinxstyleliteralemphasis{\sphinxupquote{bool}}\sphinxstyleliteralemphasis{\sphinxupquote{, }}\sphinxstyleliteralemphasis{\sphinxupquote{optional}}) \textendash{} Whether to return the sensitivity matrix. See Notes for details.
Default is False.

\end{itemize}

\sphinxlineitem{Returns}
\sphinxAtStartPar
\begin{itemize}
\item {} 
\sphinxAtStartPar
\sphinxstylestrong{estimated\_rotation} (\sphinxtitleref{Rotation} instance) \textendash{} Best estimate of the rotation that transforms \sphinxtitleref{b} to \sphinxtitleref{a}.

\item {} 
\sphinxAtStartPar
\sphinxstylestrong{rssd} (\sphinxstyleemphasis{float}) \textendash{} Square root of the weighted sum of the squared distances between
the given sets of vectors after alignment. It is equal to
\sphinxcode{\sphinxupquote{sqrt(2 * minimum\_loss)}}, where \sphinxcode{\sphinxupquote{minimum\_loss}} is the loss
function evaluated for the found optimal rotation.

\item {} 
\sphinxAtStartPar
\sphinxstylestrong{sensitivity\_matrix} (\sphinxstyleemphasis{ndarray, shape (3, 3)}) \textendash{} Sensitivity matrix of the estimated rotation estimate as explained
in Notes. Returned only when \sphinxtitleref{return\_sensitivity} is True.

\end{itemize}


\end{description}\end{quote}
\subsubsection*{Notes}

\sphinxAtStartPar
This method can also compute the sensitivity of the estimated rotation
to small perturbations of the vector measurements. Specifically we
consider the rotation estimate error as a small rotation vector of
frame A. The sensitivity matrix is proportional to the covariance of
this rotation vector assuming that the vectors in \sphinxtitleref{a} was measured with
errors significantly less than their lengths. To get the true
covariance matrix, the returned sensitivity matrix must be multiplied
by harmonic mean {\color{red}\bfseries{}{[}3{]}\_} of variance in each observation. Note that
\sphinxtitleref{weights} are supposed to be inversely proportional to the observation
variances to get consistent results. For example, if all vectors are
measured with the same accuracy of 0.01 (\sphinxtitleref{weights} must be all equal),
then you should multiple the sensitivity matrix by 0.01**2 to get the
covariance.

\sphinxAtStartPar
Refer to {\color{red}\bfseries{}{[}2{]}\_} for more rigorous discussion of the covariance
estimation.
\subsubsection*{References}

\end{fulllineitems}

\index{apply() (deepdrr.geo.Rotation method)@\spxentry{apply()}\spxextra{deepdrr.geo.Rotation method}}

\begin{fulllineitems}
\phantomsection\label{\detokenize{deepdrr.geo:id7}}
\pysigstartsignatures
\pysiglinewithargsret{\sphinxbfcode{\sphinxupquote{apply}}}{\sphinxparam{\DUrole{n,n}{self}}\sphinxparamcomma \sphinxparam{\DUrole{n,n}{vectors}}\sphinxparamcomma \sphinxparam{\DUrole{n,n}{inverse}\DUrole{o,o}{=}\DUrole{default_value}{False}}}{}
\pysigstopsignatures
\sphinxAtStartPar
Apply this rotation to a set of vectors.

\sphinxAtStartPar
If the original frame rotates to the final frame by this rotation, then
its application to a vector can be seen in two ways:
\begin{itemize}
\item {} 
\sphinxAtStartPar
As a projection of vector components expressed in the final frame
to the original frame.

\item {} 
\sphinxAtStartPar
As the physical rotation of a vector being glued to the original
frame as it rotates. In this case the vector components are
expressed in the original frame before and after the rotation.

\end{itemize}

\sphinxAtStartPar
In terms of rotation matricies, this application is the same as
\sphinxcode{\sphinxupquote{self.as\_matrix().dot(vectors)}}.
\begin{quote}\begin{description}
\sphinxlineitem{Parameters}\begin{itemize}
\item {} 
\sphinxAtStartPar
\sphinxstyleliteralstrong{\sphinxupquote{vectors}} (\sphinxstyleliteralemphasis{\sphinxupquote{array\_like}}\sphinxstyleliteralemphasis{\sphinxupquote{, }}\sphinxstyleliteralemphasis{\sphinxupquote{shape}}\sphinxstyleliteralemphasis{\sphinxupquote{ (}}\sphinxstyleliteralemphasis{\sphinxupquote{3}}\sphinxstyleliteralemphasis{\sphinxupquote{,}}\sphinxstyleliteralemphasis{\sphinxupquote{) or }}\sphinxstyleliteralemphasis{\sphinxupquote{(}}\sphinxstyleliteralemphasis{\sphinxupquote{N}}\sphinxstyleliteralemphasis{\sphinxupquote{, }}\sphinxstyleliteralemphasis{\sphinxupquote{3}}\sphinxstyleliteralemphasis{\sphinxupquote{)}}) \textendash{} Each \sphinxtitleref{vectors{[}i{]}} represents a vector in 3D space. A single vector
can either be specified with shape \sphinxtitleref{(3, )} or \sphinxtitleref{(1, 3)}. The number
of rotations and number of vectors given must follow standard numpy
broadcasting rules: either one of them equals unity or they both
equal each other.

\item {} 
\sphinxAtStartPar
\sphinxstyleliteralstrong{\sphinxupquote{inverse}} (\sphinxstyleliteralemphasis{\sphinxupquote{boolean}}\sphinxstyleliteralemphasis{\sphinxupquote{, }}\sphinxstyleliteralemphasis{\sphinxupquote{optional}}) \textendash{} If True then the inverse of the rotation(s) is applied to the input
vectors. Default is False.

\end{itemize}

\sphinxlineitem{Returns}
\sphinxAtStartPar

\sphinxAtStartPar
\sphinxstylestrong{rotated\_vectors} \textendash{} Result of applying rotation on input vectors.
Shape depends on the following cases:
\begin{itemize}
\item {} 
\sphinxAtStartPar
If object contains a single rotation (as opposed to a stack
with a single rotation) and a single vector is specified with
shape \sphinxcode{\sphinxupquote{(3,)}}, then \sphinxtitleref{rotated\_vectors} has shape \sphinxcode{\sphinxupquote{(3,)}}.

\item {} 
\sphinxAtStartPar
In all other cases, \sphinxtitleref{rotated\_vectors} has shape \sphinxcode{\sphinxupquote{(N, 3)}},
where \sphinxcode{\sphinxupquote{N}} is either the number of rotations or vectors.

\end{itemize}


\sphinxlineitem{Return type}
\sphinxAtStartPar
ndarray, shape (3,) or (N, 3)

\end{description}\end{quote}
\subsubsection*{Examples}

\begin{sphinxVerbatim}[commandchars=\\\{\}]
\PYG{g+gp}{\PYGZgt{}\PYGZgt{}\PYGZgt{} }\PYG{k+kn}{from} \PYG{n+nn}{scipy}\PYG{n+nn}{.}\PYG{n+nn}{spatial}\PYG{n+nn}{.}\PYG{n+nn}{transform} \PYG{k+kn}{import} \PYG{n}{Rotation} \PYG{k}{as} \PYG{n}{R}
\PYG{g+gp}{\PYGZgt{}\PYGZgt{}\PYGZgt{} }\PYG{k+kn}{import} \PYG{n+nn}{numpy} \PYG{k}{as} \PYG{n+nn}{np}
\end{sphinxVerbatim}

\sphinxAtStartPar
Single rotation applied on a single vector:

\begin{sphinxVerbatim}[commandchars=\\\{\}]
\PYG{g+gp}{\PYGZgt{}\PYGZgt{}\PYGZgt{} }\PYG{n}{vector} \PYG{o}{=} \PYG{n}{np}\PYG{o}{.}\PYG{n}{array}\PYG{p}{(}\PYG{p}{[}\PYG{l+m+mi}{1}\PYG{p}{,} \PYG{l+m+mi}{0}\PYG{p}{,} \PYG{l+m+mi}{0}\PYG{p}{]}\PYG{p}{)}
\PYG{g+gp}{\PYGZgt{}\PYGZgt{}\PYGZgt{} }\PYG{n}{r} \PYG{o}{=} \PYG{n}{R}\PYG{o}{.}\PYG{n}{from\PYGZus{}rotvec}\PYG{p}{(}\PYG{p}{[}\PYG{l+m+mi}{0}\PYG{p}{,} \PYG{l+m+mi}{0}\PYG{p}{,} \PYG{n}{np}\PYG{o}{.}\PYG{n}{pi}\PYG{o}{/}\PYG{l+m+mi}{2}\PYG{p}{]}\PYG{p}{)}
\PYG{g+gp}{\PYGZgt{}\PYGZgt{}\PYGZgt{} }\PYG{n}{r}\PYG{o}{.}\PYG{n}{as\PYGZus{}matrix}\PYG{p}{(}\PYG{p}{)}
\PYG{g+go}{array([[ 2.22044605e\PYGZhy{}16, \PYGZhy{}1.00000000e+00,  0.00000000e+00],}
\PYG{g+go}{       [ 1.00000000e+00,  2.22044605e\PYGZhy{}16,  0.00000000e+00],}
\PYG{g+go}{       [ 0.00000000e+00,  0.00000000e+00,  1.00000000e+00]])}
\PYG{g+gp}{\PYGZgt{}\PYGZgt{}\PYGZgt{} }\PYG{n}{r}\PYG{o}{.}\PYG{n}{apply}\PYG{p}{(}\PYG{n}{vector}\PYG{p}{)}
\PYG{g+go}{array([2.22044605e\PYGZhy{}16, 1.00000000e+00, 0.00000000e+00])}
\PYG{g+gp}{\PYGZgt{}\PYGZgt{}\PYGZgt{} }\PYG{n}{r}\PYG{o}{.}\PYG{n}{apply}\PYG{p}{(}\PYG{n}{vector}\PYG{p}{)}\PYG{o}{.}\PYG{n}{shape}
\PYG{g+go}{(3,)}
\end{sphinxVerbatim}

\sphinxAtStartPar
Single rotation applied on multiple vectors:

\begin{sphinxVerbatim}[commandchars=\\\{\}]
\PYG{g+gp}{\PYGZgt{}\PYGZgt{}\PYGZgt{} }\PYG{n}{vectors} \PYG{o}{=} \PYG{n}{np}\PYG{o}{.}\PYG{n}{array}\PYG{p}{(}\PYG{p}{[}
\PYG{g+gp}{... }\PYG{p}{[}\PYG{l+m+mi}{1}\PYG{p}{,} \PYG{l+m+mi}{0}\PYG{p}{,} \PYG{l+m+mi}{0}\PYG{p}{]}\PYG{p}{,}
\PYG{g+gp}{... }\PYG{p}{[}\PYG{l+m+mi}{1}\PYG{p}{,} \PYG{l+m+mi}{2}\PYG{p}{,} \PYG{l+m+mi}{3}\PYG{p}{]}\PYG{p}{]}\PYG{p}{)}
\PYG{g+gp}{\PYGZgt{}\PYGZgt{}\PYGZgt{} }\PYG{n}{r} \PYG{o}{=} \PYG{n}{R}\PYG{o}{.}\PYG{n}{from\PYGZus{}rotvec}\PYG{p}{(}\PYG{p}{[}\PYG{l+m+mi}{0}\PYG{p}{,} \PYG{l+m+mi}{0}\PYG{p}{,} \PYG{n}{np}\PYG{o}{.}\PYG{n}{pi}\PYG{o}{/}\PYG{l+m+mi}{4}\PYG{p}{]}\PYG{p}{)}
\PYG{g+gp}{\PYGZgt{}\PYGZgt{}\PYGZgt{} }\PYG{n}{r}\PYG{o}{.}\PYG{n}{as\PYGZus{}matrix}\PYG{p}{(}\PYG{p}{)}
\PYG{g+go}{array([[ 0.70710678, \PYGZhy{}0.70710678,  0.        ],}
\PYG{g+go}{       [ 0.70710678,  0.70710678,  0.        ],}
\PYG{g+go}{       [ 0.        ,  0.        ,  1.        ]])}
\PYG{g+gp}{\PYGZgt{}\PYGZgt{}\PYGZgt{} }\PYG{n}{r}\PYG{o}{.}\PYG{n}{apply}\PYG{p}{(}\PYG{n}{vectors}\PYG{p}{)}
\PYG{g+go}{array([[ 0.70710678,  0.70710678,  0.        ],}
\PYG{g+go}{       [\PYGZhy{}0.70710678,  2.12132034,  3.        ]])}
\PYG{g+gp}{\PYGZgt{}\PYGZgt{}\PYGZgt{} }\PYG{n}{r}\PYG{o}{.}\PYG{n}{apply}\PYG{p}{(}\PYG{n}{vectors}\PYG{p}{)}\PYG{o}{.}\PYG{n}{shape}
\PYG{g+go}{(2, 3)}
\end{sphinxVerbatim}

\sphinxAtStartPar
Multiple rotations on a single vector:

\begin{sphinxVerbatim}[commandchars=\\\{\}]
\PYG{g+gp}{\PYGZgt{}\PYGZgt{}\PYGZgt{} }\PYG{n}{r} \PYG{o}{=} \PYG{n}{R}\PYG{o}{.}\PYG{n}{from\PYGZus{}rotvec}\PYG{p}{(}\PYG{p}{[}\PYG{p}{[}\PYG{l+m+mi}{0}\PYG{p}{,} \PYG{l+m+mi}{0}\PYG{p}{,} \PYG{n}{np}\PYG{o}{.}\PYG{n}{pi}\PYG{o}{/}\PYG{l+m+mi}{4}\PYG{p}{]}\PYG{p}{,} \PYG{p}{[}\PYG{n}{np}\PYG{o}{.}\PYG{n}{pi}\PYG{o}{/}\PYG{l+m+mi}{2}\PYG{p}{,} \PYG{l+m+mi}{0}\PYG{p}{,} \PYG{l+m+mi}{0}\PYG{p}{]}\PYG{p}{]}\PYG{p}{)}
\PYG{g+gp}{\PYGZgt{}\PYGZgt{}\PYGZgt{} }\PYG{n}{vector} \PYG{o}{=} \PYG{n}{np}\PYG{o}{.}\PYG{n}{array}\PYG{p}{(}\PYG{p}{[}\PYG{l+m+mi}{1}\PYG{p}{,}\PYG{l+m+mi}{2}\PYG{p}{,}\PYG{l+m+mi}{3}\PYG{p}{]}\PYG{p}{)}
\PYG{g+gp}{\PYGZgt{}\PYGZgt{}\PYGZgt{} }\PYG{n}{r}\PYG{o}{.}\PYG{n}{as\PYGZus{}matrix}\PYG{p}{(}\PYG{p}{)}
\PYG{g+go}{array([[[ 7.07106781e\PYGZhy{}01, \PYGZhy{}7.07106781e\PYGZhy{}01,  0.00000000e+00],}
\PYG{g+go}{        [ 7.07106781e\PYGZhy{}01,  7.07106781e\PYGZhy{}01,  0.00000000e+00],}
\PYG{g+go}{        [ 0.00000000e+00,  0.00000000e+00,  1.00000000e+00]],}
\PYG{g+go}{       [[ 1.00000000e+00,  0.00000000e+00,  0.00000000e+00],}
\PYG{g+go}{        [ 0.00000000e+00,  2.22044605e\PYGZhy{}16, \PYGZhy{}1.00000000e+00],}
\PYG{g+go}{        [ 0.00000000e+00,  1.00000000e+00,  2.22044605e\PYGZhy{}16]]])}
\PYG{g+gp}{\PYGZgt{}\PYGZgt{}\PYGZgt{} }\PYG{n}{r}\PYG{o}{.}\PYG{n}{apply}\PYG{p}{(}\PYG{n}{vector}\PYG{p}{)}
\PYG{g+go}{array([[\PYGZhy{}0.70710678,  2.12132034,  3.        ],}
\PYG{g+go}{       [ 1.        , \PYGZhy{}3.        ,  2.        ]])}
\PYG{g+gp}{\PYGZgt{}\PYGZgt{}\PYGZgt{} }\PYG{n}{r}\PYG{o}{.}\PYG{n}{apply}\PYG{p}{(}\PYG{n}{vector}\PYG{p}{)}\PYG{o}{.}\PYG{n}{shape}
\PYG{g+go}{(2, 3)}
\end{sphinxVerbatim}

\sphinxAtStartPar
Multiple rotations on multiple vectors. Each rotation is applied on the
corresponding vector:

\begin{sphinxVerbatim}[commandchars=\\\{\}]
\PYG{g+gp}{\PYGZgt{}\PYGZgt{}\PYGZgt{} }\PYG{n}{r} \PYG{o}{=} \PYG{n}{R}\PYG{o}{.}\PYG{n}{from\PYGZus{}euler}\PYG{p}{(}\PYG{l+s+s1}{\PYGZsq{}}\PYG{l+s+s1}{zxy}\PYG{l+s+s1}{\PYGZsq{}}\PYG{p}{,} \PYG{p}{[}
\PYG{g+gp}{... }\PYG{p}{[}\PYG{l+m+mi}{0}\PYG{p}{,} \PYG{l+m+mi}{0}\PYG{p}{,} \PYG{l+m+mi}{90}\PYG{p}{]}\PYG{p}{,}
\PYG{g+gp}{... }\PYG{p}{[}\PYG{l+m+mi}{45}\PYG{p}{,} \PYG{l+m+mi}{30}\PYG{p}{,} \PYG{l+m+mi}{60}\PYG{p}{]}\PYG{p}{]}\PYG{p}{,} \PYG{n}{degrees}\PYG{o}{=}\PYG{k+kc}{True}\PYG{p}{)}
\PYG{g+gp}{\PYGZgt{}\PYGZgt{}\PYGZgt{} }\PYG{n}{vectors} \PYG{o}{=} \PYG{p}{[}
\PYG{g+gp}{... }\PYG{p}{[}\PYG{l+m+mi}{1}\PYG{p}{,} \PYG{l+m+mi}{2}\PYG{p}{,} \PYG{l+m+mi}{3}\PYG{p}{]}\PYG{p}{,}
\PYG{g+gp}{... }\PYG{p}{[}\PYG{l+m+mi}{1}\PYG{p}{,} \PYG{l+m+mi}{0}\PYG{p}{,} \PYG{o}{\PYGZhy{}}\PYG{l+m+mi}{1}\PYG{p}{]}\PYG{p}{]}
\PYG{g+gp}{\PYGZgt{}\PYGZgt{}\PYGZgt{} }\PYG{n}{r}\PYG{o}{.}\PYG{n}{apply}\PYG{p}{(}\PYG{n}{vectors}\PYG{p}{)}
\PYG{g+go}{array([[ 3.        ,  2.        , \PYGZhy{}1.        ],}
\PYG{g+go}{       [\PYGZhy{}0.09026039,  1.11237244, \PYGZhy{}0.86860844]])}
\PYG{g+gp}{\PYGZgt{}\PYGZgt{}\PYGZgt{} }\PYG{n}{r}\PYG{o}{.}\PYG{n}{apply}\PYG{p}{(}\PYG{n}{vectors}\PYG{p}{)}\PYG{o}{.}\PYG{n}{shape}
\PYG{g+go}{(2, 3)}
\end{sphinxVerbatim}

\sphinxAtStartPar
It is also possible to apply the inverse rotation:

\begin{sphinxVerbatim}[commandchars=\\\{\}]
\PYG{g+gp}{\PYGZgt{}\PYGZgt{}\PYGZgt{} }\PYG{n}{r} \PYG{o}{=} \PYG{n}{R}\PYG{o}{.}\PYG{n}{from\PYGZus{}euler}\PYG{p}{(}\PYG{l+s+s1}{\PYGZsq{}}\PYG{l+s+s1}{zxy}\PYG{l+s+s1}{\PYGZsq{}}\PYG{p}{,} \PYG{p}{[}
\PYG{g+gp}{... }\PYG{p}{[}\PYG{l+m+mi}{0}\PYG{p}{,} \PYG{l+m+mi}{0}\PYG{p}{,} \PYG{l+m+mi}{90}\PYG{p}{]}\PYG{p}{,}
\PYG{g+gp}{... }\PYG{p}{[}\PYG{l+m+mi}{45}\PYG{p}{,} \PYG{l+m+mi}{30}\PYG{p}{,} \PYG{l+m+mi}{60}\PYG{p}{]}\PYG{p}{]}\PYG{p}{,} \PYG{n}{degrees}\PYG{o}{=}\PYG{k+kc}{True}\PYG{p}{)}
\PYG{g+gp}{\PYGZgt{}\PYGZgt{}\PYGZgt{} }\PYG{n}{vectors} \PYG{o}{=} \PYG{p}{[}
\PYG{g+gp}{... }\PYG{p}{[}\PYG{l+m+mi}{1}\PYG{p}{,} \PYG{l+m+mi}{2}\PYG{p}{,} \PYG{l+m+mi}{3}\PYG{p}{]}\PYG{p}{,}
\PYG{g+gp}{... }\PYG{p}{[}\PYG{l+m+mi}{1}\PYG{p}{,} \PYG{l+m+mi}{0}\PYG{p}{,} \PYG{o}{\PYGZhy{}}\PYG{l+m+mi}{1}\PYG{p}{]}\PYG{p}{]}
\PYG{g+gp}{\PYGZgt{}\PYGZgt{}\PYGZgt{} }\PYG{n}{r}\PYG{o}{.}\PYG{n}{apply}\PYG{p}{(}\PYG{n}{vectors}\PYG{p}{,} \PYG{n}{inverse}\PYG{o}{=}\PYG{k+kc}{True}\PYG{p}{)}
\PYG{g+go}{array([[\PYGZhy{}3.        ,  2.        ,  1.        ],}
\PYG{g+go}{       [ 1.09533535, \PYGZhy{}0.8365163 ,  0.3169873 ]])}
\end{sphinxVerbatim}

\end{fulllineitems}

\index{as\_euler() (deepdrr.geo.Rotation method)@\spxentry{as\_euler()}\spxextra{deepdrr.geo.Rotation method}}

\begin{fulllineitems}
\phantomsection\label{\detokenize{deepdrr.geo:id8}}
\pysigstartsignatures
\pysiglinewithargsret{\sphinxbfcode{\sphinxupquote{as\_euler}}}{\sphinxparam{\DUrole{n,n}{self}}\sphinxparamcomma \sphinxparam{\DUrole{n,n}{seq}}\sphinxparamcomma \sphinxparam{\DUrole{n,n}{degrees}\DUrole{o,o}{=}\DUrole{default_value}{False}}}{}
\pysigstopsignatures
\sphinxAtStartPar
Represent as Euler angles.

\sphinxAtStartPar
Any orientation can be expressed as a composition of 3 elementary
rotations. Once the axis sequence has been chosen, Euler angles define
the angle of rotation around each respective axis {\color{red}\bfseries{}{[}1{]}\_}.

\sphinxAtStartPar
The algorithm from {\color{red}\bfseries{}{[}2{]}\_} has been used to calculate Euler angles for the
rotation about a given sequence of axes.

\sphinxAtStartPar
Euler angles suffer from the problem of gimbal lock {\color{red}\bfseries{}{[}3{]}\_}, where the
representation loses a degree of freedom and it is not possible to
determine the first and third angles uniquely. In this case,
a warning is raised, and the third angle is set to zero. Note however
that the returned angles still represent the correct rotation.
\begin{quote}\begin{description}
\sphinxlineitem{Parameters}\begin{itemize}
\item {} 
\sphinxAtStartPar
\sphinxstyleliteralstrong{\sphinxupquote{seq}} (\sphinxstyleliteralemphasis{\sphinxupquote{string}}\sphinxstyleliteralemphasis{\sphinxupquote{, }}\sphinxstyleliteralemphasis{\sphinxupquote{length 3}}) \textendash{} 3 characters belonging to the set \{‘X’, ‘Y’, ‘Z’\} for intrinsic
rotations, or \{‘x’, ‘y’, ‘z’\} for extrinsic rotations {\color{red}\bfseries{}{[}1{]}\_}.
Adjacent axes cannot be the same.
Extrinsic and intrinsic rotations cannot be mixed in one function
call.

\item {} 
\sphinxAtStartPar
\sphinxstyleliteralstrong{\sphinxupquote{degrees}} (\sphinxstyleliteralemphasis{\sphinxupquote{boolean}}\sphinxstyleliteralemphasis{\sphinxupquote{, }}\sphinxstyleliteralemphasis{\sphinxupquote{optional}}) \textendash{} Returned angles are in degrees if this flag is True, else they are
in radians. Default is False.

\end{itemize}

\sphinxlineitem{Returns}
\sphinxAtStartPar

\sphinxAtStartPar
\sphinxstylestrong{angles} \textendash{} Shape depends on shape of inputs used to initialize object.
The returned angles are in the range:
\begin{itemize}
\item {} 
\sphinxAtStartPar
First angle belongs to {[}\sphinxhyphen{}180, 180{]} degrees (both inclusive)

\item {} 
\sphinxAtStartPar
Third angle belongs to {[}\sphinxhyphen{}180, 180{]} degrees (both inclusive)

\item {} 
\sphinxAtStartPar
Second angle belongs to:
\begin{itemize}
\item {} 
\sphinxAtStartPar
{[}\sphinxhyphen{}90, 90{]} degrees if all axes are different (like xyz)

\item {} 
\sphinxAtStartPar
{[}0, 180{]} degrees if first and third axes are the same
(like zxz)

\end{itemize}

\end{itemize}


\sphinxlineitem{Return type}
\sphinxAtStartPar
ndarray, shape (3,) or (N, 3)

\end{description}\end{quote}
\subsubsection*{References}
\subsubsection*{Examples}

\begin{sphinxVerbatim}[commandchars=\\\{\}]
\PYG{g+gp}{\PYGZgt{}\PYGZgt{}\PYGZgt{} }\PYG{k+kn}{from} \PYG{n+nn}{scipy}\PYG{n+nn}{.}\PYG{n+nn}{spatial}\PYG{n+nn}{.}\PYG{n+nn}{transform} \PYG{k+kn}{import} \PYG{n}{Rotation} \PYG{k}{as} \PYG{n}{R}
\PYG{g+gp}{\PYGZgt{}\PYGZgt{}\PYGZgt{} }\PYG{k+kn}{import} \PYG{n+nn}{numpy} \PYG{k}{as} \PYG{n+nn}{np}
\end{sphinxVerbatim}

\sphinxAtStartPar
Represent a single rotation:

\begin{sphinxVerbatim}[commandchars=\\\{\}]
\PYG{g+gp}{\PYGZgt{}\PYGZgt{}\PYGZgt{} }\PYG{n}{r} \PYG{o}{=} \PYG{n}{R}\PYG{o}{.}\PYG{n}{from\PYGZus{}rotvec}\PYG{p}{(}\PYG{p}{[}\PYG{l+m+mi}{0}\PYG{p}{,} \PYG{l+m+mi}{0}\PYG{p}{,} \PYG{n}{np}\PYG{o}{.}\PYG{n}{pi}\PYG{o}{/}\PYG{l+m+mi}{2}\PYG{p}{]}\PYG{p}{)}
\PYG{g+gp}{\PYGZgt{}\PYGZgt{}\PYGZgt{} }\PYG{n}{r}\PYG{o}{.}\PYG{n}{as\PYGZus{}euler}\PYG{p}{(}\PYG{l+s+s1}{\PYGZsq{}}\PYG{l+s+s1}{zxy}\PYG{l+s+s1}{\PYGZsq{}}\PYG{p}{,} \PYG{n}{degrees}\PYG{o}{=}\PYG{k+kc}{True}\PYG{p}{)}
\PYG{g+go}{array([90.,  0.,  0.])}
\PYG{g+gp}{\PYGZgt{}\PYGZgt{}\PYGZgt{} }\PYG{n}{r}\PYG{o}{.}\PYG{n}{as\PYGZus{}euler}\PYG{p}{(}\PYG{l+s+s1}{\PYGZsq{}}\PYG{l+s+s1}{zxy}\PYG{l+s+s1}{\PYGZsq{}}\PYG{p}{,} \PYG{n}{degrees}\PYG{o}{=}\PYG{k+kc}{True}\PYG{p}{)}\PYG{o}{.}\PYG{n}{shape}
\PYG{g+go}{(3,)}
\end{sphinxVerbatim}

\sphinxAtStartPar
Represent a stack of single rotation:

\begin{sphinxVerbatim}[commandchars=\\\{\}]
\PYG{g+gp}{\PYGZgt{}\PYGZgt{}\PYGZgt{} }\PYG{n}{r} \PYG{o}{=} \PYG{n}{R}\PYG{o}{.}\PYG{n}{from\PYGZus{}rotvec}\PYG{p}{(}\PYG{p}{[}\PYG{p}{[}\PYG{l+m+mi}{0}\PYG{p}{,} \PYG{l+m+mi}{0}\PYG{p}{,} \PYG{n}{np}\PYG{o}{.}\PYG{n}{pi}\PYG{o}{/}\PYG{l+m+mi}{2}\PYG{p}{]}\PYG{p}{]}\PYG{p}{)}
\PYG{g+gp}{\PYGZgt{}\PYGZgt{}\PYGZgt{} }\PYG{n}{r}\PYG{o}{.}\PYG{n}{as\PYGZus{}euler}\PYG{p}{(}\PYG{l+s+s1}{\PYGZsq{}}\PYG{l+s+s1}{zxy}\PYG{l+s+s1}{\PYGZsq{}}\PYG{p}{,} \PYG{n}{degrees}\PYG{o}{=}\PYG{k+kc}{True}\PYG{p}{)}
\PYG{g+go}{array([[90.,  0.,  0.]])}
\PYG{g+gp}{\PYGZgt{}\PYGZgt{}\PYGZgt{} }\PYG{n}{r}\PYG{o}{.}\PYG{n}{as\PYGZus{}euler}\PYG{p}{(}\PYG{l+s+s1}{\PYGZsq{}}\PYG{l+s+s1}{zxy}\PYG{l+s+s1}{\PYGZsq{}}\PYG{p}{,} \PYG{n}{degrees}\PYG{o}{=}\PYG{k+kc}{True}\PYG{p}{)}\PYG{o}{.}\PYG{n}{shape}
\PYG{g+go}{(1, 3)}
\end{sphinxVerbatim}

\sphinxAtStartPar
Represent multiple rotations in a single object:

\begin{sphinxVerbatim}[commandchars=\\\{\}]
\PYG{g+gp}{\PYGZgt{}\PYGZgt{}\PYGZgt{} }\PYG{n}{r} \PYG{o}{=} \PYG{n}{R}\PYG{o}{.}\PYG{n}{from\PYGZus{}rotvec}\PYG{p}{(}\PYG{p}{[}
\PYG{g+gp}{... }\PYG{p}{[}\PYG{l+m+mi}{0}\PYG{p}{,} \PYG{l+m+mi}{0}\PYG{p}{,} \PYG{n}{np}\PYG{o}{.}\PYG{n}{pi}\PYG{o}{/}\PYG{l+m+mi}{2}\PYG{p}{]}\PYG{p}{,}
\PYG{g+gp}{... }\PYG{p}{[}\PYG{l+m+mi}{0}\PYG{p}{,} \PYG{o}{\PYGZhy{}}\PYG{n}{np}\PYG{o}{.}\PYG{n}{pi}\PYG{o}{/}\PYG{l+m+mi}{3}\PYG{p}{,} \PYG{l+m+mi}{0}\PYG{p}{]}\PYG{p}{,}
\PYG{g+gp}{... }\PYG{p}{[}\PYG{n}{np}\PYG{o}{.}\PYG{n}{pi}\PYG{o}{/}\PYG{l+m+mi}{4}\PYG{p}{,} \PYG{l+m+mi}{0}\PYG{p}{,} \PYG{l+m+mi}{0}\PYG{p}{]}\PYG{p}{]}\PYG{p}{)}
\PYG{g+gp}{\PYGZgt{}\PYGZgt{}\PYGZgt{} }\PYG{n}{r}\PYG{o}{.}\PYG{n}{as\PYGZus{}euler}\PYG{p}{(}\PYG{l+s+s1}{\PYGZsq{}}\PYG{l+s+s1}{zxy}\PYG{l+s+s1}{\PYGZsq{}}\PYG{p}{,} \PYG{n}{degrees}\PYG{o}{=}\PYG{k+kc}{True}\PYG{p}{)}
\PYG{g+go}{array([[ 90.,   0.,   0.],}
\PYG{g+go}{       [  0.,   0., \PYGZhy{}60.],}
\PYG{g+go}{       [  0.,  45.,   0.]])}
\PYG{g+gp}{\PYGZgt{}\PYGZgt{}\PYGZgt{} }\PYG{n}{r}\PYG{o}{.}\PYG{n}{as\PYGZus{}euler}\PYG{p}{(}\PYG{l+s+s1}{\PYGZsq{}}\PYG{l+s+s1}{zxy}\PYG{l+s+s1}{\PYGZsq{}}\PYG{p}{,} \PYG{n}{degrees}\PYG{o}{=}\PYG{k+kc}{True}\PYG{p}{)}\PYG{o}{.}\PYG{n}{shape}
\PYG{g+go}{(3, 3)}
\end{sphinxVerbatim}

\end{fulllineitems}

\index{as\_matrix() (deepdrr.geo.Rotation method)@\spxentry{as\_matrix()}\spxextra{deepdrr.geo.Rotation method}}

\begin{fulllineitems}
\phantomsection\label{\detokenize{deepdrr.geo:id16}}
\pysigstartsignatures
\pysiglinewithargsret{\sphinxbfcode{\sphinxupquote{as\_matrix}}}{\sphinxparam{\DUrole{n,n}{self}}}{}
\pysigstopsignatures
\sphinxAtStartPar
Represent as rotation matrix.

\sphinxAtStartPar
3D rotations can be represented using rotation matrices, which
are 3 x 3 real orthogonal matrices with determinant equal to +1 {\color{red}\bfseries{}{[}1{]}\_}.
\begin{quote}\begin{description}
\sphinxlineitem{Returns}
\sphinxAtStartPar
\sphinxstylestrong{matrix} \textendash{} Shape depends on shape of inputs used for initialization.

\sphinxlineitem{Return type}
\sphinxAtStartPar
ndarray, shape (3, 3) or (N, 3, 3)

\end{description}\end{quote}
\subsubsection*{References}
\subsubsection*{Examples}

\begin{sphinxVerbatim}[commandchars=\\\{\}]
\PYG{g+gp}{\PYGZgt{}\PYGZgt{}\PYGZgt{} }\PYG{k+kn}{from} \PYG{n+nn}{scipy}\PYG{n+nn}{.}\PYG{n+nn}{spatial}\PYG{n+nn}{.}\PYG{n+nn}{transform} \PYG{k+kn}{import} \PYG{n}{Rotation} \PYG{k}{as} \PYG{n}{R}
\PYG{g+gp}{\PYGZgt{}\PYGZgt{}\PYGZgt{} }\PYG{k+kn}{import} \PYG{n+nn}{numpy} \PYG{k}{as} \PYG{n+nn}{np}
\end{sphinxVerbatim}

\sphinxAtStartPar
Represent a single rotation:

\begin{sphinxVerbatim}[commandchars=\\\{\}]
\PYG{g+gp}{\PYGZgt{}\PYGZgt{}\PYGZgt{} }\PYG{n}{r} \PYG{o}{=} \PYG{n}{R}\PYG{o}{.}\PYG{n}{from\PYGZus{}rotvec}\PYG{p}{(}\PYG{p}{[}\PYG{l+m+mi}{0}\PYG{p}{,} \PYG{l+m+mi}{0}\PYG{p}{,} \PYG{n}{np}\PYG{o}{.}\PYG{n}{pi}\PYG{o}{/}\PYG{l+m+mi}{2}\PYG{p}{]}\PYG{p}{)}
\PYG{g+gp}{\PYGZgt{}\PYGZgt{}\PYGZgt{} }\PYG{n}{r}\PYG{o}{.}\PYG{n}{as\PYGZus{}matrix}\PYG{p}{(}\PYG{p}{)}
\PYG{g+go}{array([[ 2.22044605e\PYGZhy{}16, \PYGZhy{}1.00000000e+00,  0.00000000e+00],}
\PYG{g+go}{       [ 1.00000000e+00,  2.22044605e\PYGZhy{}16,  0.00000000e+00],}
\PYG{g+go}{       [ 0.00000000e+00,  0.00000000e+00,  1.00000000e+00]])}
\PYG{g+gp}{\PYGZgt{}\PYGZgt{}\PYGZgt{} }\PYG{n}{r}\PYG{o}{.}\PYG{n}{as\PYGZus{}matrix}\PYG{p}{(}\PYG{p}{)}\PYG{o}{.}\PYG{n}{shape}
\PYG{g+go}{(3, 3)}
\end{sphinxVerbatim}

\sphinxAtStartPar
Represent a stack with a single rotation:

\begin{sphinxVerbatim}[commandchars=\\\{\}]
\PYG{g+gp}{\PYGZgt{}\PYGZgt{}\PYGZgt{} }\PYG{n}{r} \PYG{o}{=} \PYG{n}{R}\PYG{o}{.}\PYG{n}{from\PYGZus{}quat}\PYG{p}{(}\PYG{p}{[}\PYG{p}{[}\PYG{l+m+mi}{1}\PYG{p}{,} \PYG{l+m+mi}{1}\PYG{p}{,} \PYG{l+m+mi}{0}\PYG{p}{,} \PYG{l+m+mi}{0}\PYG{p}{]}\PYG{p}{]}\PYG{p}{)}
\PYG{g+gp}{\PYGZgt{}\PYGZgt{}\PYGZgt{} }\PYG{n}{r}\PYG{o}{.}\PYG{n}{as\PYGZus{}matrix}\PYG{p}{(}\PYG{p}{)}
\PYG{g+go}{array([[[ 0.,  1.,  0.],}
\PYG{g+go}{        [ 1.,  0.,  0.],}
\PYG{g+go}{        [ 0.,  0., \PYGZhy{}1.]]])}
\PYG{g+gp}{\PYGZgt{}\PYGZgt{}\PYGZgt{} }\PYG{n}{r}\PYG{o}{.}\PYG{n}{as\PYGZus{}matrix}\PYG{p}{(}\PYG{p}{)}\PYG{o}{.}\PYG{n}{shape}
\PYG{g+go}{(1, 3, 3)}
\end{sphinxVerbatim}

\sphinxAtStartPar
Represent multiple rotations:

\begin{sphinxVerbatim}[commandchars=\\\{\}]
\PYG{g+gp}{\PYGZgt{}\PYGZgt{}\PYGZgt{} }\PYG{n}{r} \PYG{o}{=} \PYG{n}{R}\PYG{o}{.}\PYG{n}{from\PYGZus{}rotvec}\PYG{p}{(}\PYG{p}{[}\PYG{p}{[}\PYG{n}{np}\PYG{o}{.}\PYG{n}{pi}\PYG{o}{/}\PYG{l+m+mi}{2}\PYG{p}{,} \PYG{l+m+mi}{0}\PYG{p}{,} \PYG{l+m+mi}{0}\PYG{p}{]}\PYG{p}{,} \PYG{p}{[}\PYG{l+m+mi}{0}\PYG{p}{,} \PYG{l+m+mi}{0}\PYG{p}{,} \PYG{n}{np}\PYG{o}{.}\PYG{n}{pi}\PYG{o}{/}\PYG{l+m+mi}{2}\PYG{p}{]}\PYG{p}{]}\PYG{p}{)}
\PYG{g+gp}{\PYGZgt{}\PYGZgt{}\PYGZgt{} }\PYG{n}{r}\PYG{o}{.}\PYG{n}{as\PYGZus{}matrix}\PYG{p}{(}\PYG{p}{)}
\PYG{g+go}{array([[[ 1.00000000e+00,  0.00000000e+00,  0.00000000e+00],}
\PYG{g+go}{        [ 0.00000000e+00,  2.22044605e\PYGZhy{}16, \PYGZhy{}1.00000000e+00],}
\PYG{g+go}{        [ 0.00000000e+00,  1.00000000e+00,  2.22044605e\PYGZhy{}16]],}
\PYG{g+go}{       [[ 2.22044605e\PYGZhy{}16, \PYGZhy{}1.00000000e+00,  0.00000000e+00],}
\PYG{g+go}{        [ 1.00000000e+00,  2.22044605e\PYGZhy{}16,  0.00000000e+00],}
\PYG{g+go}{        [ 0.00000000e+00,  0.00000000e+00,  1.00000000e+00]]])}
\PYG{g+gp}{\PYGZgt{}\PYGZgt{}\PYGZgt{} }\PYG{n}{r}\PYG{o}{.}\PYG{n}{as\PYGZus{}matrix}\PYG{p}{(}\PYG{p}{)}\PYG{o}{.}\PYG{n}{shape}
\PYG{g+go}{(2, 3, 3)}
\end{sphinxVerbatim}
\subsubsection*{Notes}

\sphinxAtStartPar
This function was called as\_dcm before.

\sphinxAtStartPar
\DUrole{versionmodified,added}{New in version 1.4.0.}

\end{fulllineitems}

\index{as\_mrp() (deepdrr.geo.Rotation method)@\spxentry{as\_mrp()}\spxextra{deepdrr.geo.Rotation method}}

\begin{fulllineitems}
\phantomsection\label{\detokenize{deepdrr.geo:id19}}
\pysigstartsignatures
\pysiglinewithargsret{\sphinxbfcode{\sphinxupquote{as\_mrp}}}{\sphinxparam{\DUrole{n,n}{self}}}{}
\pysigstopsignatures
\sphinxAtStartPar
Represent as Modified Rodrigues Parameters (MRPs).

\sphinxAtStartPar
MRPs are a 3 dimensional vector co\sphinxhyphen{}directional to the axis of rotation and whose
magnitude is equal to \sphinxcode{\sphinxupquote{tan(theta / 4)}}, where \sphinxcode{\sphinxupquote{theta}} is the angle of rotation
(in radians) {\color{red}\bfseries{}{[}1{]}\_}.

\sphinxAtStartPar
MRPs have a singuarity at 360 degrees which can be avoided by ensuring the angle of
rotation does not exceed 180 degrees, i.e. switching the direction of the rotation when
it is past 180 degrees. This function will always return MRPs corresponding to a rotation
of less than or equal to 180 degrees.
\begin{quote}\begin{description}
\sphinxlineitem{Returns}
\sphinxAtStartPar
\sphinxstylestrong{mrps} \textendash{} Shape depends on shape of inputs used for initialization.

\sphinxlineitem{Return type}
\sphinxAtStartPar
ndarray, shape (3,) or (N, 3)

\end{description}\end{quote}
\subsubsection*{References}
\subsubsection*{Examples}

\begin{sphinxVerbatim}[commandchars=\\\{\}]
\PYG{g+gp}{\PYGZgt{}\PYGZgt{}\PYGZgt{} }\PYG{k+kn}{from} \PYG{n+nn}{scipy}\PYG{n+nn}{.}\PYG{n+nn}{spatial}\PYG{n+nn}{.}\PYG{n+nn}{transform} \PYG{k+kn}{import} \PYG{n}{Rotation} \PYG{k}{as} \PYG{n}{R}
\PYG{g+gp}{\PYGZgt{}\PYGZgt{}\PYGZgt{} }\PYG{k+kn}{import} \PYG{n+nn}{numpy} \PYG{k}{as} \PYG{n+nn}{np}
\end{sphinxVerbatim}

\sphinxAtStartPar
Represent a single rotation:

\begin{sphinxVerbatim}[commandchars=\\\{\}]
\PYG{g+gp}{\PYGZgt{}\PYGZgt{}\PYGZgt{} }\PYG{n}{r} \PYG{o}{=} \PYG{n}{R}\PYG{o}{.}\PYG{n}{from\PYGZus{}rotvec}\PYG{p}{(}\PYG{p}{[}\PYG{l+m+mi}{0}\PYG{p}{,} \PYG{l+m+mi}{0}\PYG{p}{,} \PYG{n}{np}\PYG{o}{.}\PYG{n}{pi}\PYG{p}{]}\PYG{p}{)}
\PYG{g+gp}{\PYGZgt{}\PYGZgt{}\PYGZgt{} }\PYG{n}{r}\PYG{o}{.}\PYG{n}{as\PYGZus{}mrp}\PYG{p}{(}\PYG{p}{)}
\PYG{g+go}{array([0.        , 0.        , 1.         ])}
\PYG{g+gp}{\PYGZgt{}\PYGZgt{}\PYGZgt{} }\PYG{n}{r}\PYG{o}{.}\PYG{n}{as\PYGZus{}mrp}\PYG{p}{(}\PYG{p}{)}\PYG{o}{.}\PYG{n}{shape}
\PYG{g+go}{(3,)}
\end{sphinxVerbatim}

\sphinxAtStartPar
Represent a stack with a single rotation:

\begin{sphinxVerbatim}[commandchars=\\\{\}]
\PYG{g+gp}{\PYGZgt{}\PYGZgt{}\PYGZgt{} }\PYG{n}{r} \PYG{o}{=} \PYG{n}{R}\PYG{o}{.}\PYG{n}{from\PYGZus{}euler}\PYG{p}{(}\PYG{l+s+s1}{\PYGZsq{}}\PYG{l+s+s1}{xyz}\PYG{l+s+s1}{\PYGZsq{}}\PYG{p}{,} \PYG{p}{[}\PYG{p}{[}\PYG{l+m+mi}{180}\PYG{p}{,} \PYG{l+m+mi}{0}\PYG{p}{,} \PYG{l+m+mi}{0}\PYG{p}{]}\PYG{p}{]}\PYG{p}{,} \PYG{n}{degrees}\PYG{o}{=}\PYG{k+kc}{True}\PYG{p}{)}
\PYG{g+gp}{\PYGZgt{}\PYGZgt{}\PYGZgt{} }\PYG{n}{r}\PYG{o}{.}\PYG{n}{as\PYGZus{}mrp}\PYG{p}{(}\PYG{p}{)}
\PYG{g+go}{array([[1.       , 0.        , 0.         ]])}
\PYG{g+gp}{\PYGZgt{}\PYGZgt{}\PYGZgt{} }\PYG{n}{r}\PYG{o}{.}\PYG{n}{as\PYGZus{}mrp}\PYG{p}{(}\PYG{p}{)}\PYG{o}{.}\PYG{n}{shape}
\PYG{g+go}{(1, 3)}
\end{sphinxVerbatim}

\sphinxAtStartPar
Represent multiple rotations:

\begin{sphinxVerbatim}[commandchars=\\\{\}]
\PYG{g+gp}{\PYGZgt{}\PYGZgt{}\PYGZgt{} }\PYG{n}{r} \PYG{o}{=} \PYG{n}{R}\PYG{o}{.}\PYG{n}{from\PYGZus{}rotvec}\PYG{p}{(}\PYG{p}{[}\PYG{p}{[}\PYG{n}{np}\PYG{o}{.}\PYG{n}{pi}\PYG{o}{/}\PYG{l+m+mi}{2}\PYG{p}{,} \PYG{l+m+mi}{0}\PYG{p}{,} \PYG{l+m+mi}{0}\PYG{p}{]}\PYG{p}{,} \PYG{p}{[}\PYG{l+m+mi}{0}\PYG{p}{,} \PYG{l+m+mi}{0}\PYG{p}{,} \PYG{n}{np}\PYG{o}{.}\PYG{n}{pi}\PYG{o}{/}\PYG{l+m+mi}{2}\PYG{p}{]}\PYG{p}{]}\PYG{p}{)}
\PYG{g+gp}{\PYGZgt{}\PYGZgt{}\PYGZgt{} }\PYG{n}{r}\PYG{o}{.}\PYG{n}{as\PYGZus{}mrp}\PYG{p}{(}\PYG{p}{)}
\PYG{g+go}{array([[0.41421356, 0.        , 0.        ],}
\PYG{g+go}{       [0.        , 0.        , 0.41421356]])}
\PYG{g+gp}{\PYGZgt{}\PYGZgt{}\PYGZgt{} }\PYG{n}{r}\PYG{o}{.}\PYG{n}{as\PYGZus{}mrp}\PYG{p}{(}\PYG{p}{)}\PYG{o}{.}\PYG{n}{shape}
\PYG{g+go}{(2, 3)}
\end{sphinxVerbatim}
\subsubsection*{Notes}

\sphinxAtStartPar
\DUrole{versionmodified,added}{New in version 1.6.0.}

\end{fulllineitems}

\index{as\_quat() (deepdrr.geo.Rotation method)@\spxentry{as\_quat()}\spxextra{deepdrr.geo.Rotation method}}

\begin{fulllineitems}
\phantomsection\label{\detokenize{deepdrr.geo:id22}}
\pysigstartsignatures
\pysiglinewithargsret{\sphinxbfcode{\sphinxupquote{as\_quat}}}{\sphinxparam{\DUrole{n,n}{self}}}{}
\pysigstopsignatures
\sphinxAtStartPar
Represent as quaternions.

\sphinxAtStartPar
Rotations in 3 dimensions can be represented using unit norm
quaternions {\color{red}\bfseries{}{[}1{]}\_}. The mapping from quaternions to rotations is
two\sphinxhyphen{}to\sphinxhyphen{}one, i.e. quaternions \sphinxcode{\sphinxupquote{q}} and \sphinxcode{\sphinxupquote{\sphinxhyphen{}q}}, where \sphinxcode{\sphinxupquote{\sphinxhyphen{}q}} simply
reverses the sign of each component, represent the same spatial
rotation. The returned value is in scalar\sphinxhyphen{}last (x, y, z, w) format.
\begin{quote}\begin{description}
\sphinxlineitem{Returns}
\sphinxAtStartPar
\sphinxstylestrong{quat} \textendash{} Shape depends on shape of inputs used for initialization.

\sphinxlineitem{Return type}
\sphinxAtStartPar
\sphinxtitleref{numpy.ndarray}, shape (4,) or (N, 4)

\end{description}\end{quote}
\subsubsection*{References}
\subsubsection*{Examples}

\begin{sphinxVerbatim}[commandchars=\\\{\}]
\PYG{g+gp}{\PYGZgt{}\PYGZgt{}\PYGZgt{} }\PYG{k+kn}{from} \PYG{n+nn}{scipy}\PYG{n+nn}{.}\PYG{n+nn}{spatial}\PYG{n+nn}{.}\PYG{n+nn}{transform} \PYG{k+kn}{import} \PYG{n}{Rotation} \PYG{k}{as} \PYG{n}{R}
\PYG{g+gp}{\PYGZgt{}\PYGZgt{}\PYGZgt{} }\PYG{k+kn}{import} \PYG{n+nn}{numpy} \PYG{k}{as} \PYG{n+nn}{np}
\end{sphinxVerbatim}

\sphinxAtStartPar
Represent a single rotation:

\begin{sphinxVerbatim}[commandchars=\\\{\}]
\PYG{g+gp}{\PYGZgt{}\PYGZgt{}\PYGZgt{} }\PYG{n}{r} \PYG{o}{=} \PYG{n}{R}\PYG{o}{.}\PYG{n}{from\PYGZus{}matrix}\PYG{p}{(}\PYG{p}{[}\PYG{p}{[}\PYG{l+m+mi}{0}\PYG{p}{,} \PYG{o}{\PYGZhy{}}\PYG{l+m+mi}{1}\PYG{p}{,} \PYG{l+m+mi}{0}\PYG{p}{]}\PYG{p}{,}
\PYG{g+gp}{... }                   \PYG{p}{[}\PYG{l+m+mi}{1}\PYG{p}{,} \PYG{l+m+mi}{0}\PYG{p}{,} \PYG{l+m+mi}{0}\PYG{p}{]}\PYG{p}{,}
\PYG{g+gp}{... }                   \PYG{p}{[}\PYG{l+m+mi}{0}\PYG{p}{,} \PYG{l+m+mi}{0}\PYG{p}{,} \PYG{l+m+mi}{1}\PYG{p}{]}\PYG{p}{]}\PYG{p}{)}
\PYG{g+gp}{\PYGZgt{}\PYGZgt{}\PYGZgt{} }\PYG{n}{r}\PYG{o}{.}\PYG{n}{as\PYGZus{}quat}\PYG{p}{(}\PYG{p}{)}
\PYG{g+go}{array([0.        , 0.        , 0.70710678, 0.70710678])}
\PYG{g+gp}{\PYGZgt{}\PYGZgt{}\PYGZgt{} }\PYG{n}{r}\PYG{o}{.}\PYG{n}{as\PYGZus{}quat}\PYG{p}{(}\PYG{p}{)}\PYG{o}{.}\PYG{n}{shape}
\PYG{g+go}{(4,)}
\end{sphinxVerbatim}

\sphinxAtStartPar
Represent a stack with a single rotation:

\begin{sphinxVerbatim}[commandchars=\\\{\}]
\PYG{g+gp}{\PYGZgt{}\PYGZgt{}\PYGZgt{} }\PYG{n}{r} \PYG{o}{=} \PYG{n}{R}\PYG{o}{.}\PYG{n}{from\PYGZus{}quat}\PYG{p}{(}\PYG{p}{[}\PYG{p}{[}\PYG{l+m+mi}{0}\PYG{p}{,} \PYG{l+m+mi}{0}\PYG{p}{,} \PYG{l+m+mi}{0}\PYG{p}{,} \PYG{l+m+mi}{1}\PYG{p}{]}\PYG{p}{]}\PYG{p}{)}
\PYG{g+gp}{\PYGZgt{}\PYGZgt{}\PYGZgt{} }\PYG{n}{r}\PYG{o}{.}\PYG{n}{as\PYGZus{}quat}\PYG{p}{(}\PYG{p}{)}\PYG{o}{.}\PYG{n}{shape}
\PYG{g+go}{(1, 4)}
\end{sphinxVerbatim}

\sphinxAtStartPar
Represent multiple rotations in a single object:

\begin{sphinxVerbatim}[commandchars=\\\{\}]
\PYG{g+gp}{\PYGZgt{}\PYGZgt{}\PYGZgt{} }\PYG{n}{r} \PYG{o}{=} \PYG{n}{R}\PYG{o}{.}\PYG{n}{from\PYGZus{}rotvec}\PYG{p}{(}\PYG{p}{[}\PYG{p}{[}\PYG{n}{np}\PYG{o}{.}\PYG{n}{pi}\PYG{p}{,} \PYG{l+m+mi}{0}\PYG{p}{,} \PYG{l+m+mi}{0}\PYG{p}{]}\PYG{p}{,} \PYG{p}{[}\PYG{l+m+mi}{0}\PYG{p}{,} \PYG{l+m+mi}{0}\PYG{p}{,} \PYG{n}{np}\PYG{o}{.}\PYG{n}{pi}\PYG{o}{/}\PYG{l+m+mi}{2}\PYG{p}{]}\PYG{p}{]}\PYG{p}{)}
\PYG{g+gp}{\PYGZgt{}\PYGZgt{}\PYGZgt{} }\PYG{n}{r}\PYG{o}{.}\PYG{n}{as\PYGZus{}quat}\PYG{p}{(}\PYG{p}{)}\PYG{o}{.}\PYG{n}{shape}
\PYG{g+go}{(2, 4)}
\end{sphinxVerbatim}

\end{fulllineitems}

\index{as\_rotvec() (deepdrr.geo.Rotation method)@\spxentry{as\_rotvec()}\spxextra{deepdrr.geo.Rotation method}}

\begin{fulllineitems}
\phantomsection\label{\detokenize{deepdrr.geo:id25}}
\pysigstartsignatures
\pysiglinewithargsret{\sphinxbfcode{\sphinxupquote{as\_rotvec}}}{\sphinxparam{\DUrole{n,n}{self}}\sphinxparamcomma \sphinxparam{\DUrole{n,n}{degrees}\DUrole{o,o}{=}\DUrole{default_value}{False}}}{}
\pysigstopsignatures
\sphinxAtStartPar
Represent as rotation vectors.

\sphinxAtStartPar
A rotation vector is a 3 dimensional vector which is co\sphinxhyphen{}directional to
the axis of rotation and whose norm gives the angle of rotation {\color{red}\bfseries{}{[}1{]}\_}.
\begin{quote}\begin{description}
\sphinxlineitem{Parameters}
\sphinxAtStartPar
\sphinxstyleliteralstrong{\sphinxupquote{degrees}} (\sphinxstyleliteralemphasis{\sphinxupquote{boolean}}\sphinxstyleliteralemphasis{\sphinxupquote{, }}\sphinxstyleliteralemphasis{\sphinxupquote{optional}}) \textendash{} 
\sphinxAtStartPar
Returned magnitudes are in degrees if this flag is True, else they are
in radians. Default is False.

\sphinxAtStartPar
\DUrole{versionmodified,added}{New in version 1.7.0.}


\sphinxlineitem{Returns}
\sphinxAtStartPar
\sphinxstylestrong{rotvec} \textendash{} Shape depends on shape of inputs used for initialization.

\sphinxlineitem{Return type}
\sphinxAtStartPar
ndarray, shape (3,) or (N, 3)

\end{description}\end{quote}
\subsubsection*{References}
\subsubsection*{Examples}

\begin{sphinxVerbatim}[commandchars=\\\{\}]
\PYG{g+gp}{\PYGZgt{}\PYGZgt{}\PYGZgt{} }\PYG{k+kn}{from} \PYG{n+nn}{scipy}\PYG{n+nn}{.}\PYG{n+nn}{spatial}\PYG{n+nn}{.}\PYG{n+nn}{transform} \PYG{k+kn}{import} \PYG{n}{Rotation} \PYG{k}{as} \PYG{n}{R}
\PYG{g+gp}{\PYGZgt{}\PYGZgt{}\PYGZgt{} }\PYG{k+kn}{import} \PYG{n+nn}{numpy} \PYG{k}{as} \PYG{n+nn}{np}
\end{sphinxVerbatim}

\sphinxAtStartPar
Represent a single rotation:

\begin{sphinxVerbatim}[commandchars=\\\{\}]
\PYG{g+gp}{\PYGZgt{}\PYGZgt{}\PYGZgt{} }\PYG{n}{r} \PYG{o}{=} \PYG{n}{R}\PYG{o}{.}\PYG{n}{from\PYGZus{}euler}\PYG{p}{(}\PYG{l+s+s1}{\PYGZsq{}}\PYG{l+s+s1}{z}\PYG{l+s+s1}{\PYGZsq{}}\PYG{p}{,} \PYG{l+m+mi}{90}\PYG{p}{,} \PYG{n}{degrees}\PYG{o}{=}\PYG{k+kc}{True}\PYG{p}{)}
\PYG{g+gp}{\PYGZgt{}\PYGZgt{}\PYGZgt{} }\PYG{n}{r}\PYG{o}{.}\PYG{n}{as\PYGZus{}rotvec}\PYG{p}{(}\PYG{p}{)}
\PYG{g+go}{array([0.        , 0.        , 1.57079633])}
\PYG{g+gp}{\PYGZgt{}\PYGZgt{}\PYGZgt{} }\PYG{n}{r}\PYG{o}{.}\PYG{n}{as\PYGZus{}rotvec}\PYG{p}{(}\PYG{p}{)}\PYG{o}{.}\PYG{n}{shape}
\PYG{g+go}{(3,)}
\end{sphinxVerbatim}

\sphinxAtStartPar
Represent a rotation in degrees:

\begin{sphinxVerbatim}[commandchars=\\\{\}]
\PYG{g+gp}{\PYGZgt{}\PYGZgt{}\PYGZgt{} }\PYG{n}{r} \PYG{o}{=} \PYG{n}{R}\PYG{o}{.}\PYG{n}{from\PYGZus{}euler}\PYG{p}{(}\PYG{l+s+s1}{\PYGZsq{}}\PYG{l+s+s1}{YX}\PYG{l+s+s1}{\PYGZsq{}}\PYG{p}{,} \PYG{p}{(}\PYG{o}{\PYGZhy{}}\PYG{l+m+mi}{90}\PYG{p}{,} \PYG{o}{\PYGZhy{}}\PYG{l+m+mi}{90}\PYG{p}{)}\PYG{p}{,} \PYG{n}{degrees}\PYG{o}{=}\PYG{k+kc}{True}\PYG{p}{)}
\PYG{g+gp}{\PYGZgt{}\PYGZgt{}\PYGZgt{} }\PYG{n}{s} \PYG{o}{=} \PYG{n}{r}\PYG{o}{.}\PYG{n}{as\PYGZus{}rotvec}\PYG{p}{(}\PYG{n}{degrees}\PYG{o}{=}\PYG{k+kc}{True}\PYG{p}{)}
\PYG{g+gp}{\PYGZgt{}\PYGZgt{}\PYGZgt{} }\PYG{n}{s}
\PYG{g+go}{array([\PYGZhy{}69.2820323, \PYGZhy{}69.2820323, \PYGZhy{}69.2820323])}
\PYG{g+gp}{\PYGZgt{}\PYGZgt{}\PYGZgt{} }\PYG{n}{np}\PYG{o}{.}\PYG{n}{linalg}\PYG{o}{.}\PYG{n}{norm}\PYG{p}{(}\PYG{n}{s}\PYG{p}{)}
\PYG{g+go}{120.00000000000001}
\end{sphinxVerbatim}

\sphinxAtStartPar
Represent a stack with a single rotation:

\begin{sphinxVerbatim}[commandchars=\\\{\}]
\PYG{g+gp}{\PYGZgt{}\PYGZgt{}\PYGZgt{} }\PYG{n}{r} \PYG{o}{=} \PYG{n}{R}\PYG{o}{.}\PYG{n}{from\PYGZus{}quat}\PYG{p}{(}\PYG{p}{[}\PYG{p}{[}\PYG{l+m+mi}{0}\PYG{p}{,} \PYG{l+m+mi}{0}\PYG{p}{,} \PYG{l+m+mi}{1}\PYG{p}{,} \PYG{l+m+mi}{1}\PYG{p}{]}\PYG{p}{]}\PYG{p}{)}
\PYG{g+gp}{\PYGZgt{}\PYGZgt{}\PYGZgt{} }\PYG{n}{r}\PYG{o}{.}\PYG{n}{as\PYGZus{}rotvec}\PYG{p}{(}\PYG{p}{)}
\PYG{g+go}{array([[0.        , 0.        , 1.57079633]])}
\PYG{g+gp}{\PYGZgt{}\PYGZgt{}\PYGZgt{} }\PYG{n}{r}\PYG{o}{.}\PYG{n}{as\PYGZus{}rotvec}\PYG{p}{(}\PYG{p}{)}\PYG{o}{.}\PYG{n}{shape}
\PYG{g+go}{(1, 3)}
\end{sphinxVerbatim}

\sphinxAtStartPar
Represent multiple rotations in a single object:

\begin{sphinxVerbatim}[commandchars=\\\{\}]
\PYG{g+gp}{\PYGZgt{}\PYGZgt{}\PYGZgt{} }\PYG{n}{r} \PYG{o}{=} \PYG{n}{R}\PYG{o}{.}\PYG{n}{from\PYGZus{}quat}\PYG{p}{(}\PYG{p}{[}\PYG{p}{[}\PYG{l+m+mi}{0}\PYG{p}{,} \PYG{l+m+mi}{0}\PYG{p}{,} \PYG{l+m+mi}{1}\PYG{p}{,} \PYG{l+m+mi}{1}\PYG{p}{]}\PYG{p}{,} \PYG{p}{[}\PYG{l+m+mi}{1}\PYG{p}{,} \PYG{l+m+mi}{1}\PYG{p}{,} \PYG{l+m+mi}{0}\PYG{p}{,} \PYG{l+m+mi}{1}\PYG{p}{]}\PYG{p}{]}\PYG{p}{)}
\PYG{g+gp}{\PYGZgt{}\PYGZgt{}\PYGZgt{} }\PYG{n}{r}\PYG{o}{.}\PYG{n}{as\PYGZus{}rotvec}\PYG{p}{(}\PYG{p}{)}
\PYG{g+go}{array([[0.        , 0.        , 1.57079633],}
\PYG{g+go}{       [1.35102172, 1.35102172, 0.        ]])}
\PYG{g+gp}{\PYGZgt{}\PYGZgt{}\PYGZgt{} }\PYG{n}{r}\PYG{o}{.}\PYG{n}{as\PYGZus{}rotvec}\PYG{p}{(}\PYG{p}{)}\PYG{o}{.}\PYG{n}{shape}
\PYG{g+go}{(2, 3)}
\end{sphinxVerbatim}

\end{fulllineitems}

\index{concatenate() (deepdrr.geo.Rotation method)@\spxentry{concatenate()}\spxextra{deepdrr.geo.Rotation method}}

\begin{fulllineitems}
\phantomsection\label{\detokenize{deepdrr.geo:id28}}
\pysigstartsignatures
\pysiglinewithargsret{\sphinxbfcode{\sphinxupquote{concatenate}}}{\sphinxparam{\DUrole{n,n}{type cls}}\sphinxparamcomma \sphinxparam{\DUrole{n,n}{rotations}}}{}
\pysigstopsignatures
\sphinxAtStartPar
Concatenate a sequence of \sphinxtitleref{Rotation} objects.
\begin{quote}\begin{description}
\sphinxlineitem{Parameters}
\sphinxAtStartPar
\sphinxstyleliteralstrong{\sphinxupquote{rotations}} (sequence of \sphinxtitleref{Rotation} objects) \textendash{} The rotations to concatenate.

\sphinxlineitem{Returns}
\sphinxAtStartPar
\sphinxstylestrong{concatenated} \textendash{} The concatenated rotations.

\sphinxlineitem{Return type}
\sphinxAtStartPar
\sphinxtitleref{Rotation} instance

\end{description}\end{quote}
\subsubsection*{Notes}

\sphinxAtStartPar
\DUrole{versionmodified,added}{New in version 1.8.0.}

\end{fulllineitems}

\index{create\_group() (deepdrr.geo.Rotation method)@\spxentry{create\_group()}\spxextra{deepdrr.geo.Rotation method}}

\begin{fulllineitems}
\phantomsection\label{\detokenize{deepdrr.geo:id29}}
\pysigstartsignatures
\pysiglinewithargsret{\sphinxbfcode{\sphinxupquote{create\_group}}}{\sphinxparam{\DUrole{n,n}{type cls}}\sphinxparamcomma \sphinxparam{\DUrole{n,n}{group}}\sphinxparamcomma \sphinxparam{\DUrole{n,n}{axis=u\textquotesingle{}Z\textquotesingle{}}}}{}
\pysigstopsignatures
\sphinxAtStartPar
Create a 3D rotation group.
\begin{quote}\begin{description}
\sphinxlineitem{Parameters}\begin{itemize}
\item {} 
\sphinxAtStartPar
\sphinxstyleliteralstrong{\sphinxupquote{group}} (\sphinxstyleliteralemphasis{\sphinxupquote{string}}) \textendash{} 
\sphinxAtStartPar
The name of the group. Must be one of ‘I’, ‘O’, ‘T’, ‘Dn’, ‘Cn’,
where \sphinxtitleref{n} is a positive integer. The groups are:
\begin{itemize}
\item {} 
\sphinxAtStartPar
I: Icosahedral group

\item {} 
\sphinxAtStartPar
O: Octahedral group

\item {} 
\sphinxAtStartPar
T: Tetrahedral group

\item {} 
\sphinxAtStartPar
D: Dicyclic group

\item {} 
\sphinxAtStartPar
C: Cyclic group

\end{itemize}


\item {} 
\sphinxAtStartPar
\sphinxstyleliteralstrong{\sphinxupquote{axis}} (\sphinxstyleliteralemphasis{\sphinxupquote{integer}}) \textendash{} The cyclic rotation axis. Must be one of {[}‘X’, ‘Y’, ‘Z’{]} (or
lowercase). Default is ‘Z’. Ignored for groups ‘I’, ‘O’, and ‘T’.

\end{itemize}

\sphinxlineitem{Returns}
\sphinxAtStartPar
\sphinxstylestrong{rotation} \textendash{} Object containing the elements of the rotation group.

\sphinxlineitem{Return type}
\sphinxAtStartPar
\sphinxtitleref{Rotation} instance

\end{description}\end{quote}
\subsubsection*{Notes}

\sphinxAtStartPar
This method generates rotation groups only. The full 3\sphinxhyphen{}dimensional
point groups \sphinxcite{deepdrr.geo:pointgroups} also contain reflections.
\subsubsection*{References}

\end{fulllineitems}

\index{from\_euler() (deepdrr.geo.Rotation method)@\spxentry{from\_euler()}\spxextra{deepdrr.geo.Rotation method}}

\begin{fulllineitems}
\phantomsection\label{\detokenize{deepdrr.geo:id31}}
\pysigstartsignatures
\pysiglinewithargsret{\sphinxbfcode{\sphinxupquote{from\_euler}}}{\sphinxparam{\DUrole{n,n}{type cls}}\sphinxparamcomma \sphinxparam{\DUrole{n,n}{seq}}\sphinxparamcomma \sphinxparam{\DUrole{n,n}{angles}}\sphinxparamcomma \sphinxparam{\DUrole{n,n}{degrees=False}}}{}
\pysigstopsignatures
\sphinxAtStartPar
Initialize from Euler angles.

\sphinxAtStartPar
Rotations in 3\sphinxhyphen{}D can be represented by a sequence of 3
rotations around a sequence of axes. In theory, any three axes spanning
the 3\sphinxhyphen{}D Euclidean space are enough. In practice, the axes of rotation are
chosen to be the basis vectors.

\sphinxAtStartPar
The three rotations can either be in a global frame of reference
(extrinsic) or in a body centred frame of reference (intrinsic), which
is attached to, and moves with, the object under rotation {\color{red}\bfseries{}{[}1{]}\_}.
\begin{quote}\begin{description}
\sphinxlineitem{Parameters}\begin{itemize}
\item {} 
\sphinxAtStartPar
\sphinxstyleliteralstrong{\sphinxupquote{seq}} (\sphinxstyleliteralemphasis{\sphinxupquote{string}}) \textendash{} Specifies sequence of axes for rotations. Up to 3 characters
belonging to the set \{‘X’, ‘Y’, ‘Z’\} for intrinsic rotations, or
\{‘x’, ‘y’, ‘z’\} for extrinsic rotations. Extrinsic and intrinsic
rotations cannot be mixed in one function call.

\item {} 
\sphinxAtStartPar
\sphinxstyleliteralstrong{\sphinxupquote{angles}} (\sphinxstyleliteralemphasis{\sphinxupquote{float}}\sphinxstyleliteralemphasis{\sphinxupquote{ or }}\sphinxstyleliteralemphasis{\sphinxupquote{array\_like}}\sphinxstyleliteralemphasis{\sphinxupquote{, }}\sphinxstyleliteralemphasis{\sphinxupquote{shape}}\sphinxstyleliteralemphasis{\sphinxupquote{ (}}\sphinxstyleliteralemphasis{\sphinxupquote{N}}\sphinxstyleliteralemphasis{\sphinxupquote{,}}\sphinxstyleliteralemphasis{\sphinxupquote{) or }}\sphinxstyleliteralemphasis{\sphinxupquote{(}}\sphinxstyleliteralemphasis{\sphinxupquote{N}}\sphinxstyleliteralemphasis{\sphinxupquote{, }}\sphinxstyleliteralemphasis{\sphinxupquote{{[}}}\sphinxstyleliteralemphasis{\sphinxupquote{1}}\sphinxstyleliteralemphasis{\sphinxupquote{ or }}\sphinxstyleliteralemphasis{\sphinxupquote{2}}\sphinxstyleliteralemphasis{\sphinxupquote{ or }}\sphinxstyleliteralemphasis{\sphinxupquote{3}}\sphinxstyleliteralemphasis{\sphinxupquote{{]}}}\sphinxstyleliteralemphasis{\sphinxupquote{)}}) \textendash{} 
\sphinxAtStartPar
Euler angles specified in radians (\sphinxtitleref{degrees} is False) or degrees
(\sphinxtitleref{degrees} is True).
For a single character \sphinxtitleref{seq}, \sphinxtitleref{angles} can be:
\begin{itemize}
\item {} 
\sphinxAtStartPar
a single value

\item {} 
\sphinxAtStartPar
array\_like with shape (N,), where each \sphinxtitleref{angle{[}i{]}}
corresponds to a single rotation

\item {} 
\sphinxAtStartPar
array\_like with shape (N, 1), where each \sphinxtitleref{angle{[}i, 0{]}}
corresponds to a single rotation

\end{itemize}

\sphinxAtStartPar
For 2\sphinxhyphen{} and 3\sphinxhyphen{}character wide \sphinxtitleref{seq}, \sphinxtitleref{angles} can be:
\begin{itemize}
\item {} 
\sphinxAtStartPar
array\_like with shape (W,) where \sphinxtitleref{W} is the width of
\sphinxtitleref{seq}, which corresponds to a single rotation with \sphinxtitleref{W} axes

\item {} 
\sphinxAtStartPar
array\_like with shape (N, W) where each \sphinxtitleref{angle{[}i{]}}
corresponds to a sequence of Euler angles describing a single
rotation

\end{itemize}


\item {} 
\sphinxAtStartPar
\sphinxstyleliteralstrong{\sphinxupquote{degrees}} (\sphinxstyleliteralemphasis{\sphinxupquote{bool}}\sphinxstyleliteralemphasis{\sphinxupquote{, }}\sphinxstyleliteralemphasis{\sphinxupquote{optional}}) \textendash{} If True, then the given angles are assumed to be in degrees.
Default is False.

\end{itemize}

\sphinxlineitem{Returns}
\sphinxAtStartPar
\sphinxstylestrong{rotation} \textendash{} Object containing the rotation represented by the sequence of
rotations around given axes with given angles.

\sphinxlineitem{Return type}
\sphinxAtStartPar
\sphinxtitleref{Rotation} instance

\end{description}\end{quote}
\subsubsection*{References}
\subsubsection*{Examples}

\begin{sphinxVerbatim}[commandchars=\\\{\}]
\PYG{g+gp}{\PYGZgt{}\PYGZgt{}\PYGZgt{} }\PYG{k+kn}{from} \PYG{n+nn}{scipy}\PYG{n+nn}{.}\PYG{n+nn}{spatial}\PYG{n+nn}{.}\PYG{n+nn}{transform} \PYG{k+kn}{import} \PYG{n}{Rotation} \PYG{k}{as} \PYG{n}{R}
\end{sphinxVerbatim}

\sphinxAtStartPar
Initialize a single rotation along a single axis:

\begin{sphinxVerbatim}[commandchars=\\\{\}]
\PYG{g+gp}{\PYGZgt{}\PYGZgt{}\PYGZgt{} }\PYG{n}{r} \PYG{o}{=} \PYG{n}{R}\PYG{o}{.}\PYG{n}{from\PYGZus{}euler}\PYG{p}{(}\PYG{l+s+s1}{\PYGZsq{}}\PYG{l+s+s1}{x}\PYG{l+s+s1}{\PYGZsq{}}\PYG{p}{,} \PYG{l+m+mi}{90}\PYG{p}{,} \PYG{n}{degrees}\PYG{o}{=}\PYG{k+kc}{True}\PYG{p}{)}
\PYG{g+gp}{\PYGZgt{}\PYGZgt{}\PYGZgt{} }\PYG{n}{r}\PYG{o}{.}\PYG{n}{as\PYGZus{}quat}\PYG{p}{(}\PYG{p}{)}\PYG{o}{.}\PYG{n}{shape}
\PYG{g+go}{(4,)}
\end{sphinxVerbatim}

\sphinxAtStartPar
Initialize a single rotation with a given axis sequence:

\begin{sphinxVerbatim}[commandchars=\\\{\}]
\PYG{g+gp}{\PYGZgt{}\PYGZgt{}\PYGZgt{} }\PYG{n}{r} \PYG{o}{=} \PYG{n}{R}\PYG{o}{.}\PYG{n}{from\PYGZus{}euler}\PYG{p}{(}\PYG{l+s+s1}{\PYGZsq{}}\PYG{l+s+s1}{zyx}\PYG{l+s+s1}{\PYGZsq{}}\PYG{p}{,} \PYG{p}{[}\PYG{l+m+mi}{90}\PYG{p}{,} \PYG{l+m+mi}{45}\PYG{p}{,} \PYG{l+m+mi}{30}\PYG{p}{]}\PYG{p}{,} \PYG{n}{degrees}\PYG{o}{=}\PYG{k+kc}{True}\PYG{p}{)}
\PYG{g+gp}{\PYGZgt{}\PYGZgt{}\PYGZgt{} }\PYG{n}{r}\PYG{o}{.}\PYG{n}{as\PYGZus{}quat}\PYG{p}{(}\PYG{p}{)}\PYG{o}{.}\PYG{n}{shape}
\PYG{g+go}{(4,)}
\end{sphinxVerbatim}

\sphinxAtStartPar
Initialize a stack with a single rotation around a single axis:

\begin{sphinxVerbatim}[commandchars=\\\{\}]
\PYG{g+gp}{\PYGZgt{}\PYGZgt{}\PYGZgt{} }\PYG{n}{r} \PYG{o}{=} \PYG{n}{R}\PYG{o}{.}\PYG{n}{from\PYGZus{}euler}\PYG{p}{(}\PYG{l+s+s1}{\PYGZsq{}}\PYG{l+s+s1}{x}\PYG{l+s+s1}{\PYGZsq{}}\PYG{p}{,} \PYG{p}{[}\PYG{l+m+mi}{90}\PYG{p}{]}\PYG{p}{,} \PYG{n}{degrees}\PYG{o}{=}\PYG{k+kc}{True}\PYG{p}{)}
\PYG{g+gp}{\PYGZgt{}\PYGZgt{}\PYGZgt{} }\PYG{n}{r}\PYG{o}{.}\PYG{n}{as\PYGZus{}quat}\PYG{p}{(}\PYG{p}{)}\PYG{o}{.}\PYG{n}{shape}
\PYG{g+go}{(1, 4)}
\end{sphinxVerbatim}

\sphinxAtStartPar
Initialize a stack with a single rotation with an axis sequence:

\begin{sphinxVerbatim}[commandchars=\\\{\}]
\PYG{g+gp}{\PYGZgt{}\PYGZgt{}\PYGZgt{} }\PYG{n}{r} \PYG{o}{=} \PYG{n}{R}\PYG{o}{.}\PYG{n}{from\PYGZus{}euler}\PYG{p}{(}\PYG{l+s+s1}{\PYGZsq{}}\PYG{l+s+s1}{zyx}\PYG{l+s+s1}{\PYGZsq{}}\PYG{p}{,} \PYG{p}{[}\PYG{p}{[}\PYG{l+m+mi}{90}\PYG{p}{,} \PYG{l+m+mi}{45}\PYG{p}{,} \PYG{l+m+mi}{30}\PYG{p}{]}\PYG{p}{]}\PYG{p}{,} \PYG{n}{degrees}\PYG{o}{=}\PYG{k+kc}{True}\PYG{p}{)}
\PYG{g+gp}{\PYGZgt{}\PYGZgt{}\PYGZgt{} }\PYG{n}{r}\PYG{o}{.}\PYG{n}{as\PYGZus{}quat}\PYG{p}{(}\PYG{p}{)}\PYG{o}{.}\PYG{n}{shape}
\PYG{g+go}{(1, 4)}
\end{sphinxVerbatim}

\sphinxAtStartPar
Initialize multiple elementary rotations in one object:

\begin{sphinxVerbatim}[commandchars=\\\{\}]
\PYG{g+gp}{\PYGZgt{}\PYGZgt{}\PYGZgt{} }\PYG{n}{r} \PYG{o}{=} \PYG{n}{R}\PYG{o}{.}\PYG{n}{from\PYGZus{}euler}\PYG{p}{(}\PYG{l+s+s1}{\PYGZsq{}}\PYG{l+s+s1}{x}\PYG{l+s+s1}{\PYGZsq{}}\PYG{p}{,} \PYG{p}{[}\PYG{l+m+mi}{90}\PYG{p}{,} \PYG{l+m+mi}{45}\PYG{p}{,} \PYG{l+m+mi}{30}\PYG{p}{]}\PYG{p}{,} \PYG{n}{degrees}\PYG{o}{=}\PYG{k+kc}{True}\PYG{p}{)}
\PYG{g+gp}{\PYGZgt{}\PYGZgt{}\PYGZgt{} }\PYG{n}{r}\PYG{o}{.}\PYG{n}{as\PYGZus{}quat}\PYG{p}{(}\PYG{p}{)}\PYG{o}{.}\PYG{n}{shape}
\PYG{g+go}{(3, 4)}
\end{sphinxVerbatim}

\sphinxAtStartPar
Initialize multiple rotations in one object:

\begin{sphinxVerbatim}[commandchars=\\\{\}]
\PYG{g+gp}{\PYGZgt{}\PYGZgt{}\PYGZgt{} }\PYG{n}{r} \PYG{o}{=} \PYG{n}{R}\PYG{o}{.}\PYG{n}{from\PYGZus{}euler}\PYG{p}{(}\PYG{l+s+s1}{\PYGZsq{}}\PYG{l+s+s1}{zyx}\PYG{l+s+s1}{\PYGZsq{}}\PYG{p}{,} \PYG{p}{[}\PYG{p}{[}\PYG{l+m+mi}{90}\PYG{p}{,} \PYG{l+m+mi}{45}\PYG{p}{,} \PYG{l+m+mi}{30}\PYG{p}{]}\PYG{p}{,} \PYG{p}{[}\PYG{l+m+mi}{35}\PYG{p}{,} \PYG{l+m+mi}{45}\PYG{p}{,} \PYG{l+m+mi}{90}\PYG{p}{]}\PYG{p}{]}\PYG{p}{,} \PYG{n}{degrees}\PYG{o}{=}\PYG{k+kc}{True}\PYG{p}{)}
\PYG{g+gp}{\PYGZgt{}\PYGZgt{}\PYGZgt{} }\PYG{n}{r}\PYG{o}{.}\PYG{n}{as\PYGZus{}quat}\PYG{p}{(}\PYG{p}{)}\PYG{o}{.}\PYG{n}{shape}
\PYG{g+go}{(2, 4)}
\end{sphinxVerbatim}

\end{fulllineitems}

\index{from\_matrix() (deepdrr.geo.Rotation method)@\spxentry{from\_matrix()}\spxextra{deepdrr.geo.Rotation method}}

\begin{fulllineitems}
\phantomsection\label{\detokenize{deepdrr.geo:id34}}
\pysigstartsignatures
\pysiglinewithargsret{\sphinxbfcode{\sphinxupquote{from\_matrix}}}{\sphinxparam{\DUrole{n,n}{type cls}}\sphinxparamcomma \sphinxparam{\DUrole{n,n}{matrix}}}{}
\pysigstopsignatures
\sphinxAtStartPar
Initialize from rotation matrix.

\sphinxAtStartPar
Rotations in 3 dimensions can be represented with 3 x 3 proper
orthogonal matrices {\color{red}\bfseries{}{[}1{]}\_}. If the input is not proper orthogonal,
an approximation is created using the method described in {\color{red}\bfseries{}{[}2{]}\_}.
\begin{quote}\begin{description}
\sphinxlineitem{Parameters}
\sphinxAtStartPar
\sphinxstyleliteralstrong{\sphinxupquote{matrix}} (\sphinxstyleliteralemphasis{\sphinxupquote{array\_like}}\sphinxstyleliteralemphasis{\sphinxupquote{, }}\sphinxstyleliteralemphasis{\sphinxupquote{shape}}\sphinxstyleliteralemphasis{\sphinxupquote{ (}}\sphinxstyleliteralemphasis{\sphinxupquote{N}}\sphinxstyleliteralemphasis{\sphinxupquote{, }}\sphinxstyleliteralemphasis{\sphinxupquote{3}}\sphinxstyleliteralemphasis{\sphinxupquote{, }}\sphinxstyleliteralemphasis{\sphinxupquote{3}}\sphinxstyleliteralemphasis{\sphinxupquote{) or }}\sphinxstyleliteralemphasis{\sphinxupquote{(}}\sphinxstyleliteralemphasis{\sphinxupquote{3}}\sphinxstyleliteralemphasis{\sphinxupquote{, }}\sphinxstyleliteralemphasis{\sphinxupquote{3}}\sphinxstyleliteralemphasis{\sphinxupquote{)}}) \textendash{} A single matrix or a stack of matrices, where \sphinxcode{\sphinxupquote{matrix{[}i{]}}} is
the i\sphinxhyphen{}th matrix.

\sphinxlineitem{Returns}
\sphinxAtStartPar
\sphinxstylestrong{rotation} \textendash{} Object containing the rotations represented by the rotation
matrices.

\sphinxlineitem{Return type}
\sphinxAtStartPar
\sphinxtitleref{Rotation} instance

\end{description}\end{quote}
\subsubsection*{References}
\subsubsection*{Examples}

\begin{sphinxVerbatim}[commandchars=\\\{\}]
\PYG{g+gp}{\PYGZgt{}\PYGZgt{}\PYGZgt{} }\PYG{k+kn}{from} \PYG{n+nn}{scipy}\PYG{n+nn}{.}\PYG{n+nn}{spatial}\PYG{n+nn}{.}\PYG{n+nn}{transform} \PYG{k+kn}{import} \PYG{n}{Rotation} \PYG{k}{as} \PYG{n}{R}
\PYG{g+gp}{\PYGZgt{}\PYGZgt{}\PYGZgt{} }\PYG{k+kn}{import} \PYG{n+nn}{numpy} \PYG{k}{as} \PYG{n+nn}{np}
\end{sphinxVerbatim}

\sphinxAtStartPar
Initialize a single rotation:

\begin{sphinxVerbatim}[commandchars=\\\{\}]
\PYG{g+gp}{\PYGZgt{}\PYGZgt{}\PYGZgt{} }\PYG{n}{r} \PYG{o}{=} \PYG{n}{R}\PYG{o}{.}\PYG{n}{from\PYGZus{}matrix}\PYG{p}{(}\PYG{p}{[}
\PYG{g+gp}{... }\PYG{p}{[}\PYG{l+m+mi}{0}\PYG{p}{,} \PYG{o}{\PYGZhy{}}\PYG{l+m+mi}{1}\PYG{p}{,} \PYG{l+m+mi}{0}\PYG{p}{]}\PYG{p}{,}
\PYG{g+gp}{... }\PYG{p}{[}\PYG{l+m+mi}{1}\PYG{p}{,} \PYG{l+m+mi}{0}\PYG{p}{,} \PYG{l+m+mi}{0}\PYG{p}{]}\PYG{p}{,}
\PYG{g+gp}{... }\PYG{p}{[}\PYG{l+m+mi}{0}\PYG{p}{,} \PYG{l+m+mi}{0}\PYG{p}{,} \PYG{l+m+mi}{1}\PYG{p}{]}\PYG{p}{]}\PYG{p}{)}
\PYG{g+gp}{\PYGZgt{}\PYGZgt{}\PYGZgt{} }\PYG{n}{r}\PYG{o}{.}\PYG{n}{as\PYGZus{}matrix}\PYG{p}{(}\PYG{p}{)}\PYG{o}{.}\PYG{n}{shape}
\PYG{g+go}{(3, 3)}
\end{sphinxVerbatim}

\sphinxAtStartPar
Initialize multiple rotations in a single object:

\begin{sphinxVerbatim}[commandchars=\\\{\}]
\PYG{g+gp}{\PYGZgt{}\PYGZgt{}\PYGZgt{} }\PYG{n}{r} \PYG{o}{=} \PYG{n}{R}\PYG{o}{.}\PYG{n}{from\PYGZus{}matrix}\PYG{p}{(}\PYG{p}{[}
\PYG{g+gp}{... }\PYG{p}{[}
\PYG{g+gp}{... }    \PYG{p}{[}\PYG{l+m+mi}{0}\PYG{p}{,} \PYG{o}{\PYGZhy{}}\PYG{l+m+mi}{1}\PYG{p}{,} \PYG{l+m+mi}{0}\PYG{p}{]}\PYG{p}{,}
\PYG{g+gp}{... }    \PYG{p}{[}\PYG{l+m+mi}{1}\PYG{p}{,} \PYG{l+m+mi}{0}\PYG{p}{,} \PYG{l+m+mi}{0}\PYG{p}{]}\PYG{p}{,}
\PYG{g+gp}{... }    \PYG{p}{[}\PYG{l+m+mi}{0}\PYG{p}{,} \PYG{l+m+mi}{0}\PYG{p}{,} \PYG{l+m+mi}{1}\PYG{p}{]}\PYG{p}{,}
\PYG{g+gp}{... }\PYG{p}{]}\PYG{p}{,}
\PYG{g+gp}{... }\PYG{p}{[}
\PYG{g+gp}{... }    \PYG{p}{[}\PYG{l+m+mi}{1}\PYG{p}{,} \PYG{l+m+mi}{0}\PYG{p}{,} \PYG{l+m+mi}{0}\PYG{p}{]}\PYG{p}{,}
\PYG{g+gp}{... }    \PYG{p}{[}\PYG{l+m+mi}{0}\PYG{p}{,} \PYG{l+m+mi}{0}\PYG{p}{,} \PYG{o}{\PYGZhy{}}\PYG{l+m+mi}{1}\PYG{p}{]}\PYG{p}{,}
\PYG{g+gp}{... }    \PYG{p}{[}\PYG{l+m+mi}{0}\PYG{p}{,} \PYG{l+m+mi}{1}\PYG{p}{,} \PYG{l+m+mi}{0}\PYG{p}{]}\PYG{p}{,}
\PYG{g+gp}{... }\PYG{p}{]}\PYG{p}{]}\PYG{p}{)}
\PYG{g+gp}{\PYGZgt{}\PYGZgt{}\PYGZgt{} }\PYG{n}{r}\PYG{o}{.}\PYG{n}{as\PYGZus{}matrix}\PYG{p}{(}\PYG{p}{)}\PYG{o}{.}\PYG{n}{shape}
\PYG{g+go}{(2, 3, 3)}
\end{sphinxVerbatim}

\sphinxAtStartPar
If input matrices are not special orthogonal (orthogonal with
determinant equal to +1), then a special orthogonal estimate is stored:

\begin{sphinxVerbatim}[commandchars=\\\{\}]
\PYG{g+gp}{\PYGZgt{}\PYGZgt{}\PYGZgt{} }\PYG{n}{a} \PYG{o}{=} \PYG{n}{np}\PYG{o}{.}\PYG{n}{array}\PYG{p}{(}\PYG{p}{[}
\PYG{g+gp}{... }\PYG{p}{[}\PYG{l+m+mi}{0}\PYG{p}{,} \PYG{o}{\PYGZhy{}}\PYG{l+m+mf}{0.5}\PYG{p}{,} \PYG{l+m+mi}{0}\PYG{p}{]}\PYG{p}{,}
\PYG{g+gp}{... }\PYG{p}{[}\PYG{l+m+mf}{0.5}\PYG{p}{,} \PYG{l+m+mi}{0}\PYG{p}{,} \PYG{l+m+mi}{0}\PYG{p}{]}\PYG{p}{,}
\PYG{g+gp}{... }\PYG{p}{[}\PYG{l+m+mi}{0}\PYG{p}{,} \PYG{l+m+mi}{0}\PYG{p}{,} \PYG{l+m+mf}{0.5}\PYG{p}{]}\PYG{p}{]}\PYG{p}{)}
\PYG{g+gp}{\PYGZgt{}\PYGZgt{}\PYGZgt{} }\PYG{n}{np}\PYG{o}{.}\PYG{n}{linalg}\PYG{o}{.}\PYG{n}{det}\PYG{p}{(}\PYG{n}{a}\PYG{p}{)}
\PYG{g+go}{0.12500000000000003}
\PYG{g+gp}{\PYGZgt{}\PYGZgt{}\PYGZgt{} }\PYG{n}{r} \PYG{o}{=} \PYG{n}{R}\PYG{o}{.}\PYG{n}{from\PYGZus{}matrix}\PYG{p}{(}\PYG{n}{a}\PYG{p}{)}
\PYG{g+gp}{\PYGZgt{}\PYGZgt{}\PYGZgt{} }\PYG{n}{matrix} \PYG{o}{=} \PYG{n}{r}\PYG{o}{.}\PYG{n}{as\PYGZus{}matrix}\PYG{p}{(}\PYG{p}{)}
\PYG{g+gp}{\PYGZgt{}\PYGZgt{}\PYGZgt{} }\PYG{n}{matrix}
\PYG{g+go}{array([[\PYGZhy{}0.38461538, \PYGZhy{}0.92307692,  0.        ],}
\PYG{g+go}{       [ 0.92307692, \PYGZhy{}0.38461538,  0.        ],}
\PYG{g+go}{       [ 0.        ,  0.        ,  1.        ]])}
\PYG{g+gp}{\PYGZgt{}\PYGZgt{}\PYGZgt{} }\PYG{n}{np}\PYG{o}{.}\PYG{n}{linalg}\PYG{o}{.}\PYG{n}{det}\PYG{p}{(}\PYG{n}{matrix}\PYG{p}{)}
\PYG{g+go}{1.0000000000000002}
\end{sphinxVerbatim}

\sphinxAtStartPar
It is also possible to have a stack containing a single rotation:

\begin{sphinxVerbatim}[commandchars=\\\{\}]
\PYG{g+gp}{\PYGZgt{}\PYGZgt{}\PYGZgt{} }\PYG{n}{r} \PYG{o}{=} \PYG{n}{R}\PYG{o}{.}\PYG{n}{from\PYGZus{}matrix}\PYG{p}{(}\PYG{p}{[}\PYG{p}{[}
\PYG{g+gp}{... }\PYG{p}{[}\PYG{l+m+mi}{0}\PYG{p}{,} \PYG{o}{\PYGZhy{}}\PYG{l+m+mi}{1}\PYG{p}{,} \PYG{l+m+mi}{0}\PYG{p}{]}\PYG{p}{,}
\PYG{g+gp}{... }\PYG{p}{[}\PYG{l+m+mi}{1}\PYG{p}{,} \PYG{l+m+mi}{0}\PYG{p}{,} \PYG{l+m+mi}{0}\PYG{p}{]}\PYG{p}{,}
\PYG{g+gp}{... }\PYG{p}{[}\PYG{l+m+mi}{0}\PYG{p}{,} \PYG{l+m+mi}{0}\PYG{p}{,} \PYG{l+m+mi}{1}\PYG{p}{]}\PYG{p}{]}\PYG{p}{]}\PYG{p}{)}
\PYG{g+gp}{\PYGZgt{}\PYGZgt{}\PYGZgt{} }\PYG{n}{r}\PYG{o}{.}\PYG{n}{as\PYGZus{}matrix}\PYG{p}{(}\PYG{p}{)}
\PYG{g+go}{array([[[ 0., \PYGZhy{}1.,  0.],}
\PYG{g+go}{        [ 1.,  0.,  0.],}
\PYG{g+go}{        [ 0.,  0.,  1.]]])}
\PYG{g+gp}{\PYGZgt{}\PYGZgt{}\PYGZgt{} }\PYG{n}{r}\PYG{o}{.}\PYG{n}{as\PYGZus{}matrix}\PYG{p}{(}\PYG{p}{)}\PYG{o}{.}\PYG{n}{shape}
\PYG{g+go}{(1, 3, 3)}
\end{sphinxVerbatim}
\subsubsection*{Notes}

\sphinxAtStartPar
This function was called from\_dcm before.

\sphinxAtStartPar
\DUrole{versionmodified,added}{New in version 1.4.0.}

\end{fulllineitems}

\index{from\_mrp() (deepdrr.geo.Rotation method)@\spxentry{from\_mrp()}\spxextra{deepdrr.geo.Rotation method}}

\begin{fulllineitems}
\phantomsection\label{\detokenize{deepdrr.geo:id39}}
\pysigstartsignatures
\pysiglinewithargsret{\sphinxbfcode{\sphinxupquote{from\_mrp}}}{\sphinxparam{\DUrole{n,n}{type cls}}\sphinxparamcomma \sphinxparam{\DUrole{n,n}{mrp}}}{}
\pysigstopsignatures
\sphinxAtStartPar
Initialize from Modified Rodrigues Parameters (MRPs).

\sphinxAtStartPar
MRPs are a 3 dimensional vector co\sphinxhyphen{}directional to the axis of rotation and whose
magnitude is equal to \sphinxcode{\sphinxupquote{tan(theta / 4)}}, where \sphinxcode{\sphinxupquote{theta}} is the angle of rotation
(in radians) {\color{red}\bfseries{}{[}1{]}\_}.

\sphinxAtStartPar
MRPs have a singuarity at 360 degrees which can be avoided by ensuring the angle of
rotation does not exceed 180 degrees, i.e. switching the direction of the rotation when
it is past 180 degrees.
\begin{quote}\begin{description}
\sphinxlineitem{Parameters}
\sphinxAtStartPar
\sphinxstyleliteralstrong{\sphinxupquote{mrp}} (\sphinxstyleliteralemphasis{\sphinxupquote{array\_like}}\sphinxstyleliteralemphasis{\sphinxupquote{, }}\sphinxstyleliteralemphasis{\sphinxupquote{shape}}\sphinxstyleliteralemphasis{\sphinxupquote{ (}}\sphinxstyleliteralemphasis{\sphinxupquote{N}}\sphinxstyleliteralemphasis{\sphinxupquote{, }}\sphinxstyleliteralemphasis{\sphinxupquote{3}}\sphinxstyleliteralemphasis{\sphinxupquote{) or }}\sphinxstyleliteralemphasis{\sphinxupquote{(}}\sphinxstyleliteralemphasis{\sphinxupquote{3}}\sphinxstyleliteralemphasis{\sphinxupquote{,}}\sphinxstyleliteralemphasis{\sphinxupquote{)}}) \textendash{} A single vector or a stack of vectors, where \sphinxtitleref{mrp{[}i{]}} gives
the ith set of MRPs.

\sphinxlineitem{Returns}
\sphinxAtStartPar
\sphinxstylestrong{rotation} \textendash{} Object containing the rotations represented by input MRPs.

\sphinxlineitem{Return type}
\sphinxAtStartPar
\sphinxtitleref{Rotation} instance

\end{description}\end{quote}
\subsubsection*{References}
\subsubsection*{Notes}

\sphinxAtStartPar
\DUrole{versionmodified,added}{New in version 1.6.0.}
\subsubsection*{Examples}

\begin{sphinxVerbatim}[commandchars=\\\{\}]
\PYG{g+gp}{\PYGZgt{}\PYGZgt{}\PYGZgt{} }\PYG{k+kn}{from} \PYG{n+nn}{scipy}\PYG{n+nn}{.}\PYG{n+nn}{spatial}\PYG{n+nn}{.}\PYG{n+nn}{transform} \PYG{k+kn}{import} \PYG{n}{Rotation} \PYG{k}{as} \PYG{n}{R}
\PYG{g+gp}{\PYGZgt{}\PYGZgt{}\PYGZgt{} }\PYG{k+kn}{import} \PYG{n+nn}{numpy} \PYG{k}{as} \PYG{n+nn}{np}
\end{sphinxVerbatim}

\sphinxAtStartPar
Initialize a single rotation:

\begin{sphinxVerbatim}[commandchars=\\\{\}]
\PYG{g+gp}{\PYGZgt{}\PYGZgt{}\PYGZgt{} }\PYG{n}{r} \PYG{o}{=} \PYG{n}{R}\PYG{o}{.}\PYG{n}{from\PYGZus{}mrp}\PYG{p}{(}\PYG{p}{[}\PYG{l+m+mi}{0}\PYG{p}{,} \PYG{l+m+mi}{0}\PYG{p}{,} \PYG{l+m+mi}{1}\PYG{p}{]}\PYG{p}{)}
\PYG{g+gp}{\PYGZgt{}\PYGZgt{}\PYGZgt{} }\PYG{n}{r}\PYG{o}{.}\PYG{n}{as\PYGZus{}euler}\PYG{p}{(}\PYG{l+s+s1}{\PYGZsq{}}\PYG{l+s+s1}{xyz}\PYG{l+s+s1}{\PYGZsq{}}\PYG{p}{,} \PYG{n}{degrees}\PYG{o}{=}\PYG{k+kc}{True}\PYG{p}{)}
\PYG{g+go}{array([0.        , 0.        , 180.      ])}
\PYG{g+gp}{\PYGZgt{}\PYGZgt{}\PYGZgt{} }\PYG{n}{r}\PYG{o}{.}\PYG{n}{as\PYGZus{}euler}\PYG{p}{(}\PYG{l+s+s1}{\PYGZsq{}}\PYG{l+s+s1}{xyz}\PYG{l+s+s1}{\PYGZsq{}}\PYG{p}{)}\PYG{o}{.}\PYG{n}{shape}
\PYG{g+go}{(3,)}
\end{sphinxVerbatim}

\sphinxAtStartPar
Initialize multiple rotations in one object:

\begin{sphinxVerbatim}[commandchars=\\\{\}]
\PYG{g+gp}{\PYGZgt{}\PYGZgt{}\PYGZgt{} }\PYG{n}{r} \PYG{o}{=} \PYG{n}{R}\PYG{o}{.}\PYG{n}{from\PYGZus{}mrp}\PYG{p}{(}\PYG{p}{[}
\PYG{g+gp}{... }\PYG{p}{[}\PYG{l+m+mi}{0}\PYG{p}{,} \PYG{l+m+mi}{0}\PYG{p}{,} \PYG{l+m+mi}{1}\PYG{p}{]}\PYG{p}{,}
\PYG{g+gp}{... }\PYG{p}{[}\PYG{l+m+mi}{1}\PYG{p}{,} \PYG{l+m+mi}{0}\PYG{p}{,} \PYG{l+m+mi}{0}\PYG{p}{]}\PYG{p}{]}\PYG{p}{)}
\PYG{g+gp}{\PYGZgt{}\PYGZgt{}\PYGZgt{} }\PYG{n}{r}\PYG{o}{.}\PYG{n}{as\PYGZus{}euler}\PYG{p}{(}\PYG{l+s+s1}{\PYGZsq{}}\PYG{l+s+s1}{xyz}\PYG{l+s+s1}{\PYGZsq{}}\PYG{p}{,} \PYG{n}{degrees}\PYG{o}{=}\PYG{k+kc}{True}\PYG{p}{)}
\PYG{g+go}{array([[0.        , 0.        , 180.      ],}
\PYG{g+go}{       [180.0     , 0.        , 0.        ]])}
\PYG{g+gp}{\PYGZgt{}\PYGZgt{}\PYGZgt{} }\PYG{n}{r}\PYG{o}{.}\PYG{n}{as\PYGZus{}euler}\PYG{p}{(}\PYG{l+s+s1}{\PYGZsq{}}\PYG{l+s+s1}{xyz}\PYG{l+s+s1}{\PYGZsq{}}\PYG{p}{)}\PYG{o}{.}\PYG{n}{shape}
\PYG{g+go}{(2, 3)}
\end{sphinxVerbatim}

\sphinxAtStartPar
It is also possible to have a stack of a single rotation:

\begin{sphinxVerbatim}[commandchars=\\\{\}]
\PYG{g+gp}{\PYGZgt{}\PYGZgt{}\PYGZgt{} }\PYG{n}{r} \PYG{o}{=} \PYG{n}{R}\PYG{o}{.}\PYG{n}{from\PYGZus{}mrp}\PYG{p}{(}\PYG{p}{[}\PYG{p}{[}\PYG{l+m+mi}{0}\PYG{p}{,} \PYG{l+m+mi}{0}\PYG{p}{,} \PYG{n}{np}\PYG{o}{.}\PYG{n}{pi}\PYG{o}{/}\PYG{l+m+mi}{2}\PYG{p}{]}\PYG{p}{]}\PYG{p}{)}
\PYG{g+gp}{\PYGZgt{}\PYGZgt{}\PYGZgt{} }\PYG{n}{r}\PYG{o}{.}\PYG{n}{as\PYGZus{}euler}\PYG{p}{(}\PYG{l+s+s1}{\PYGZsq{}}\PYG{l+s+s1}{xyz}\PYG{l+s+s1}{\PYGZsq{}}\PYG{p}{)}\PYG{o}{.}\PYG{n}{shape}
\PYG{g+go}{(1, 3)}
\end{sphinxVerbatim}

\end{fulllineitems}

\index{from\_quat() (deepdrr.geo.Rotation method)@\spxentry{from\_quat()}\spxextra{deepdrr.geo.Rotation method}}

\begin{fulllineitems}
\phantomsection\label{\detokenize{deepdrr.geo:id42}}
\pysigstartsignatures
\pysiglinewithargsret{\sphinxbfcode{\sphinxupquote{from\_quat}}}{\sphinxparam{\DUrole{n,n}{type cls}}\sphinxparamcomma \sphinxparam{\DUrole{n,n}{quat}}}{}
\pysigstopsignatures
\sphinxAtStartPar
Initialize from quaternions.

\sphinxAtStartPar
3D rotations can be represented using unit\sphinxhyphen{}norm quaternions {\color{red}\bfseries{}{[}1{]}\_}.
\begin{quote}\begin{description}
\sphinxlineitem{Parameters}
\sphinxAtStartPar
\sphinxstyleliteralstrong{\sphinxupquote{quat}} (\sphinxstyleliteralemphasis{\sphinxupquote{array\_like}}\sphinxstyleliteralemphasis{\sphinxupquote{, }}\sphinxstyleliteralemphasis{\sphinxupquote{shape}}\sphinxstyleliteralemphasis{\sphinxupquote{ (}}\sphinxstyleliteralemphasis{\sphinxupquote{N}}\sphinxstyleliteralemphasis{\sphinxupquote{, }}\sphinxstyleliteralemphasis{\sphinxupquote{4}}\sphinxstyleliteralemphasis{\sphinxupquote{) or }}\sphinxstyleliteralemphasis{\sphinxupquote{(}}\sphinxstyleliteralemphasis{\sphinxupquote{4}}\sphinxstyleliteralemphasis{\sphinxupquote{,}}\sphinxstyleliteralemphasis{\sphinxupquote{)}}) \textendash{} Each row is a (possibly non\sphinxhyphen{}unit norm) quaternion in scalar\sphinxhyphen{}last
(x, y, z, w) format. Each quaternion will be normalized to unit
norm.

\sphinxlineitem{Returns}
\sphinxAtStartPar
\sphinxstylestrong{rotation} \textendash{} Object containing the rotations represented by input quaternions.

\sphinxlineitem{Return type}
\sphinxAtStartPar
\sphinxtitleref{Rotation} instance

\end{description}\end{quote}
\subsubsection*{References}
\subsubsection*{Examples}

\begin{sphinxVerbatim}[commandchars=\\\{\}]
\PYG{g+gp}{\PYGZgt{}\PYGZgt{}\PYGZgt{} }\PYG{k+kn}{from} \PYG{n+nn}{scipy}\PYG{n+nn}{.}\PYG{n+nn}{spatial}\PYG{n+nn}{.}\PYG{n+nn}{transform} \PYG{k+kn}{import} \PYG{n}{Rotation} \PYG{k}{as} \PYG{n}{R}
\end{sphinxVerbatim}

\sphinxAtStartPar
Initialize a single rotation:

\begin{sphinxVerbatim}[commandchars=\\\{\}]
\PYG{g+gp}{\PYGZgt{}\PYGZgt{}\PYGZgt{} }\PYG{n}{r} \PYG{o}{=} \PYG{n}{R}\PYG{o}{.}\PYG{n}{from\PYGZus{}quat}\PYG{p}{(}\PYG{p}{[}\PYG{l+m+mi}{1}\PYG{p}{,} \PYG{l+m+mi}{0}\PYG{p}{,} \PYG{l+m+mi}{0}\PYG{p}{,} \PYG{l+m+mi}{0}\PYG{p}{]}\PYG{p}{)}
\PYG{g+gp}{\PYGZgt{}\PYGZgt{}\PYGZgt{} }\PYG{n}{r}\PYG{o}{.}\PYG{n}{as\PYGZus{}quat}\PYG{p}{(}\PYG{p}{)}
\PYG{g+go}{array([1., 0., 0., 0.])}
\PYG{g+gp}{\PYGZgt{}\PYGZgt{}\PYGZgt{} }\PYG{n}{r}\PYG{o}{.}\PYG{n}{as\PYGZus{}quat}\PYG{p}{(}\PYG{p}{)}\PYG{o}{.}\PYG{n}{shape}
\PYG{g+go}{(4,)}
\end{sphinxVerbatim}

\sphinxAtStartPar
Initialize multiple rotations in a single object:

\begin{sphinxVerbatim}[commandchars=\\\{\}]
\PYG{g+gp}{\PYGZgt{}\PYGZgt{}\PYGZgt{} }\PYG{n}{r} \PYG{o}{=} \PYG{n}{R}\PYG{o}{.}\PYG{n}{from\PYGZus{}quat}\PYG{p}{(}\PYG{p}{[}
\PYG{g+gp}{... }\PYG{p}{[}\PYG{l+m+mi}{1}\PYG{p}{,} \PYG{l+m+mi}{0}\PYG{p}{,} \PYG{l+m+mi}{0}\PYG{p}{,} \PYG{l+m+mi}{0}\PYG{p}{]}\PYG{p}{,}
\PYG{g+gp}{... }\PYG{p}{[}\PYG{l+m+mi}{0}\PYG{p}{,} \PYG{l+m+mi}{0}\PYG{p}{,} \PYG{l+m+mi}{0}\PYG{p}{,} \PYG{l+m+mi}{1}\PYG{p}{]}
\PYG{g+gp}{... }\PYG{p}{]}\PYG{p}{)}
\PYG{g+gp}{\PYGZgt{}\PYGZgt{}\PYGZgt{} }\PYG{n}{r}\PYG{o}{.}\PYG{n}{as\PYGZus{}quat}\PYG{p}{(}\PYG{p}{)}
\PYG{g+go}{array([[1., 0., 0., 0.],}
\PYG{g+go}{       [0., 0., 0., 1.]])}
\PYG{g+gp}{\PYGZgt{}\PYGZgt{}\PYGZgt{} }\PYG{n}{r}\PYG{o}{.}\PYG{n}{as\PYGZus{}quat}\PYG{p}{(}\PYG{p}{)}\PYG{o}{.}\PYG{n}{shape}
\PYG{g+go}{(2, 4)}
\end{sphinxVerbatim}

\sphinxAtStartPar
It is also possible to have a stack of a single rotation:

\begin{sphinxVerbatim}[commandchars=\\\{\}]
\PYG{g+gp}{\PYGZgt{}\PYGZgt{}\PYGZgt{} }\PYG{n}{r} \PYG{o}{=} \PYG{n}{R}\PYG{o}{.}\PYG{n}{from\PYGZus{}quat}\PYG{p}{(}\PYG{p}{[}\PYG{p}{[}\PYG{l+m+mi}{0}\PYG{p}{,} \PYG{l+m+mi}{0}\PYG{p}{,} \PYG{l+m+mi}{0}\PYG{p}{,} \PYG{l+m+mi}{1}\PYG{p}{]}\PYG{p}{]}\PYG{p}{)}
\PYG{g+gp}{\PYGZgt{}\PYGZgt{}\PYGZgt{} }\PYG{n}{r}\PYG{o}{.}\PYG{n}{as\PYGZus{}quat}\PYG{p}{(}\PYG{p}{)}
\PYG{g+go}{array([[0., 0., 0., 1.]])}
\PYG{g+gp}{\PYGZgt{}\PYGZgt{}\PYGZgt{} }\PYG{n}{r}\PYG{o}{.}\PYG{n}{as\PYGZus{}quat}\PYG{p}{(}\PYG{p}{)}\PYG{o}{.}\PYG{n}{shape}
\PYG{g+go}{(1, 4)}
\end{sphinxVerbatim}

\sphinxAtStartPar
Quaternions are normalized before initialization.

\begin{sphinxVerbatim}[commandchars=\\\{\}]
\PYG{g+gp}{\PYGZgt{}\PYGZgt{}\PYGZgt{} }\PYG{n}{r} \PYG{o}{=} \PYG{n}{R}\PYG{o}{.}\PYG{n}{from\PYGZus{}quat}\PYG{p}{(}\PYG{p}{[}\PYG{l+m+mi}{0}\PYG{p}{,} \PYG{l+m+mi}{0}\PYG{p}{,} \PYG{l+m+mi}{1}\PYG{p}{,} \PYG{l+m+mi}{1}\PYG{p}{]}\PYG{p}{)}
\PYG{g+gp}{\PYGZgt{}\PYGZgt{}\PYGZgt{} }\PYG{n}{r}\PYG{o}{.}\PYG{n}{as\PYGZus{}quat}\PYG{p}{(}\PYG{p}{)}
\PYG{g+go}{array([0.        , 0.        , 0.70710678, 0.70710678])}
\end{sphinxVerbatim}

\end{fulllineitems}

\index{from\_rotvec() (deepdrr.geo.Rotation method)@\spxentry{from\_rotvec()}\spxextra{deepdrr.geo.Rotation method}}

\begin{fulllineitems}
\phantomsection\label{\detokenize{deepdrr.geo:id45}}
\pysigstartsignatures
\pysiglinewithargsret{\sphinxbfcode{\sphinxupquote{from\_rotvec}}}{\sphinxparam{\DUrole{n,n}{type cls}}\sphinxparamcomma \sphinxparam{\DUrole{n,n}{rotvec}}\sphinxparamcomma \sphinxparam{\DUrole{n,n}{degrees=False}}}{}
\pysigstopsignatures
\sphinxAtStartPar
Initialize from rotation vectors.

\sphinxAtStartPar
A rotation vector is a 3 dimensional vector which is co\sphinxhyphen{}directional to
the axis of rotation and whose norm gives the angle of rotation {\color{red}\bfseries{}{[}1{]}\_}.
\begin{quote}\begin{description}
\sphinxlineitem{Parameters}\begin{itemize}
\item {} 
\sphinxAtStartPar
\sphinxstyleliteralstrong{\sphinxupquote{rotvec}} (\sphinxstyleliteralemphasis{\sphinxupquote{array\_like}}\sphinxstyleliteralemphasis{\sphinxupquote{, }}\sphinxstyleliteralemphasis{\sphinxupquote{shape}}\sphinxstyleliteralemphasis{\sphinxupquote{ (}}\sphinxstyleliteralemphasis{\sphinxupquote{N}}\sphinxstyleliteralemphasis{\sphinxupquote{, }}\sphinxstyleliteralemphasis{\sphinxupquote{3}}\sphinxstyleliteralemphasis{\sphinxupquote{) or }}\sphinxstyleliteralemphasis{\sphinxupquote{(}}\sphinxstyleliteralemphasis{\sphinxupquote{3}}\sphinxstyleliteralemphasis{\sphinxupquote{,}}\sphinxstyleliteralemphasis{\sphinxupquote{)}}) \textendash{} A single vector or a stack of vectors, where \sphinxtitleref{rot\_vec{[}i{]}} gives
the ith rotation vector.

\item {} 
\sphinxAtStartPar
\sphinxstyleliteralstrong{\sphinxupquote{degrees}} (\sphinxstyleliteralemphasis{\sphinxupquote{bool}}\sphinxstyleliteralemphasis{\sphinxupquote{, }}\sphinxstyleliteralemphasis{\sphinxupquote{optional}}) \textendash{} 
\sphinxAtStartPar
If True, then the given magnitudes are assumed to be in degrees.
Default is False.

\sphinxAtStartPar
\DUrole{versionmodified,added}{New in version 1.7.0.}


\end{itemize}

\sphinxlineitem{Returns}
\sphinxAtStartPar
\sphinxstylestrong{rotation} \textendash{} Object containing the rotations represented by input rotation
vectors.

\sphinxlineitem{Return type}
\sphinxAtStartPar
\sphinxtitleref{Rotation} instance

\end{description}\end{quote}
\subsubsection*{References}
\subsubsection*{Examples}

\begin{sphinxVerbatim}[commandchars=\\\{\}]
\PYG{g+gp}{\PYGZgt{}\PYGZgt{}\PYGZgt{} }\PYG{k+kn}{from} \PYG{n+nn}{scipy}\PYG{n+nn}{.}\PYG{n+nn}{spatial}\PYG{n+nn}{.}\PYG{n+nn}{transform} \PYG{k+kn}{import} \PYG{n}{Rotation} \PYG{k}{as} \PYG{n}{R}
\PYG{g+gp}{\PYGZgt{}\PYGZgt{}\PYGZgt{} }\PYG{k+kn}{import} \PYG{n+nn}{numpy} \PYG{k}{as} \PYG{n+nn}{np}
\end{sphinxVerbatim}

\sphinxAtStartPar
Initialize a single rotation:

\begin{sphinxVerbatim}[commandchars=\\\{\}]
\PYG{g+gp}{\PYGZgt{}\PYGZgt{}\PYGZgt{} }\PYG{n}{r} \PYG{o}{=} \PYG{n}{R}\PYG{o}{.}\PYG{n}{from\PYGZus{}rotvec}\PYG{p}{(}\PYG{n}{np}\PYG{o}{.}\PYG{n}{pi}\PYG{o}{/}\PYG{l+m+mi}{2} \PYG{o}{*} \PYG{n}{np}\PYG{o}{.}\PYG{n}{array}\PYG{p}{(}\PYG{p}{[}\PYG{l+m+mi}{0}\PYG{p}{,} \PYG{l+m+mi}{0}\PYG{p}{,} \PYG{l+m+mi}{1}\PYG{p}{]}\PYG{p}{)}\PYG{p}{)}
\PYG{g+gp}{\PYGZgt{}\PYGZgt{}\PYGZgt{} }\PYG{n}{r}\PYG{o}{.}\PYG{n}{as\PYGZus{}rotvec}\PYG{p}{(}\PYG{p}{)}
\PYG{g+go}{array([0.        , 0.        , 1.57079633])}
\PYG{g+gp}{\PYGZgt{}\PYGZgt{}\PYGZgt{} }\PYG{n}{r}\PYG{o}{.}\PYG{n}{as\PYGZus{}rotvec}\PYG{p}{(}\PYG{p}{)}\PYG{o}{.}\PYG{n}{shape}
\PYG{g+go}{(3,)}
\end{sphinxVerbatim}

\sphinxAtStartPar
Initialize a rotation in degrees, and view it in degrees:

\begin{sphinxVerbatim}[commandchars=\\\{\}]
\PYG{g+gp}{\PYGZgt{}\PYGZgt{}\PYGZgt{} }\PYG{n}{r} \PYG{o}{=} \PYG{n}{R}\PYG{o}{.}\PYG{n}{from\PYGZus{}rotvec}\PYG{p}{(}\PYG{l+m+mi}{45} \PYG{o}{*} \PYG{n}{np}\PYG{o}{.}\PYG{n}{array}\PYG{p}{(}\PYG{p}{[}\PYG{l+m+mi}{0}\PYG{p}{,} \PYG{l+m+mi}{1}\PYG{p}{,} \PYG{l+m+mi}{0}\PYG{p}{]}\PYG{p}{)}\PYG{p}{,} \PYG{n}{degrees}\PYG{o}{=}\PYG{k+kc}{True}\PYG{p}{)}
\PYG{g+gp}{\PYGZgt{}\PYGZgt{}\PYGZgt{} }\PYG{n}{r}\PYG{o}{.}\PYG{n}{as\PYGZus{}rotvec}\PYG{p}{(}\PYG{n}{degrees}\PYG{o}{=}\PYG{k+kc}{True}\PYG{p}{)}
\PYG{g+go}{array([ 0., 45.,  0.])}
\end{sphinxVerbatim}

\sphinxAtStartPar
Initialize multiple rotations in one object:

\begin{sphinxVerbatim}[commandchars=\\\{\}]
\PYG{g+gp}{\PYGZgt{}\PYGZgt{}\PYGZgt{} }\PYG{n}{r} \PYG{o}{=} \PYG{n}{R}\PYG{o}{.}\PYG{n}{from\PYGZus{}rotvec}\PYG{p}{(}\PYG{p}{[}
\PYG{g+gp}{... }\PYG{p}{[}\PYG{l+m+mi}{0}\PYG{p}{,} \PYG{l+m+mi}{0}\PYG{p}{,} \PYG{n}{np}\PYG{o}{.}\PYG{n}{pi}\PYG{o}{/}\PYG{l+m+mi}{2}\PYG{p}{]}\PYG{p}{,}
\PYG{g+gp}{... }\PYG{p}{[}\PYG{n}{np}\PYG{o}{.}\PYG{n}{pi}\PYG{o}{/}\PYG{l+m+mi}{2}\PYG{p}{,} \PYG{l+m+mi}{0}\PYG{p}{,} \PYG{l+m+mi}{0}\PYG{p}{]}\PYG{p}{]}\PYG{p}{)}
\PYG{g+gp}{\PYGZgt{}\PYGZgt{}\PYGZgt{} }\PYG{n}{r}\PYG{o}{.}\PYG{n}{as\PYGZus{}rotvec}\PYG{p}{(}\PYG{p}{)}
\PYG{g+go}{array([[0.        , 0.        , 1.57079633],}
\PYG{g+go}{       [1.57079633, 0.        , 0.        ]])}
\PYG{g+gp}{\PYGZgt{}\PYGZgt{}\PYGZgt{} }\PYG{n}{r}\PYG{o}{.}\PYG{n}{as\PYGZus{}rotvec}\PYG{p}{(}\PYG{p}{)}\PYG{o}{.}\PYG{n}{shape}
\PYG{g+go}{(2, 3)}
\end{sphinxVerbatim}

\sphinxAtStartPar
It is also possible to have a stack of a single rotaton:

\begin{sphinxVerbatim}[commandchars=\\\{\}]
\PYG{g+gp}{\PYGZgt{}\PYGZgt{}\PYGZgt{} }\PYG{n}{r} \PYG{o}{=} \PYG{n}{R}\PYG{o}{.}\PYG{n}{from\PYGZus{}rotvec}\PYG{p}{(}\PYG{p}{[}\PYG{p}{[}\PYG{l+m+mi}{0}\PYG{p}{,} \PYG{l+m+mi}{0}\PYG{p}{,} \PYG{n}{np}\PYG{o}{.}\PYG{n}{pi}\PYG{o}{/}\PYG{l+m+mi}{2}\PYG{p}{]}\PYG{p}{]}\PYG{p}{)}
\PYG{g+gp}{\PYGZgt{}\PYGZgt{}\PYGZgt{} }\PYG{n}{r}\PYG{o}{.}\PYG{n}{as\PYGZus{}rotvec}\PYG{p}{(}\PYG{p}{)}\PYG{o}{.}\PYG{n}{shape}
\PYG{g+go}{(1, 3)}
\end{sphinxVerbatim}

\end{fulllineitems}

\index{identity() (deepdrr.geo.Rotation method)@\spxentry{identity()}\spxextra{deepdrr.geo.Rotation method}}

\begin{fulllineitems}
\phantomsection\label{\detokenize{deepdrr.geo:id48}}
\pysigstartsignatures
\pysiglinewithargsret{\sphinxbfcode{\sphinxupquote{identity}}}{\sphinxparam{\DUrole{n,n}{type cls}}\sphinxparamcomma \sphinxparam{\DUrole{n,n}{num=None}}}{}
\pysigstopsignatures
\sphinxAtStartPar
Get identity rotation(s).

\sphinxAtStartPar
Composition with the identity rotation has no effect.
\begin{quote}\begin{description}
\sphinxlineitem{Parameters}
\sphinxAtStartPar
\sphinxstyleliteralstrong{\sphinxupquote{num}} (\sphinxstyleliteralemphasis{\sphinxupquote{int}}\sphinxstyleliteralemphasis{\sphinxupquote{ or }}\sphinxstyleliteralemphasis{\sphinxupquote{None}}\sphinxstyleliteralemphasis{\sphinxupquote{, }}\sphinxstyleliteralemphasis{\sphinxupquote{optional}}) \textendash{} Number of identity rotations to generate. If None (default), then a
single rotation is generated.

\sphinxlineitem{Returns}
\sphinxAtStartPar
\sphinxstylestrong{identity} \textendash{} The identity rotation.

\sphinxlineitem{Return type}
\sphinxAtStartPar
Rotation object

\end{description}\end{quote}

\end{fulllineitems}

\index{inv() (deepdrr.geo.Rotation method)@\spxentry{inv()}\spxextra{deepdrr.geo.Rotation method}}

\begin{fulllineitems}
\phantomsection\label{\detokenize{deepdrr.geo:id49}}
\pysigstartsignatures
\pysiglinewithargsret{\sphinxbfcode{\sphinxupquote{inv}}}{\sphinxparam{\DUrole{n,n}{self}}}{}
\pysigstopsignatures
\sphinxAtStartPar
Invert this rotation.

\sphinxAtStartPar
Composition of a rotation with its inverse results in an identity
transformation.
\begin{quote}\begin{description}
\sphinxlineitem{Returns}
\sphinxAtStartPar
\sphinxstylestrong{inverse} \textendash{} Object containing inverse of the rotations in the current instance.

\sphinxlineitem{Return type}
\sphinxAtStartPar
\sphinxtitleref{Rotation} instance

\end{description}\end{quote}
\subsubsection*{Examples}

\begin{sphinxVerbatim}[commandchars=\\\{\}]
\PYG{g+gp}{\PYGZgt{}\PYGZgt{}\PYGZgt{} }\PYG{k+kn}{from} \PYG{n+nn}{scipy}\PYG{n+nn}{.}\PYG{n+nn}{spatial}\PYG{n+nn}{.}\PYG{n+nn}{transform} \PYG{k+kn}{import} \PYG{n}{Rotation} \PYG{k}{as} \PYG{n}{R}
\PYG{g+gp}{\PYGZgt{}\PYGZgt{}\PYGZgt{} }\PYG{k+kn}{import} \PYG{n+nn}{numpy} \PYG{k}{as} \PYG{n+nn}{np}
\end{sphinxVerbatim}

\sphinxAtStartPar
Inverting a single rotation:

\begin{sphinxVerbatim}[commandchars=\\\{\}]
\PYG{g+gp}{\PYGZgt{}\PYGZgt{}\PYGZgt{} }\PYG{n}{p} \PYG{o}{=} \PYG{n}{R}\PYG{o}{.}\PYG{n}{from\PYGZus{}euler}\PYG{p}{(}\PYG{l+s+s1}{\PYGZsq{}}\PYG{l+s+s1}{z}\PYG{l+s+s1}{\PYGZsq{}}\PYG{p}{,} \PYG{l+m+mi}{45}\PYG{p}{,} \PYG{n}{degrees}\PYG{o}{=}\PYG{k+kc}{True}\PYG{p}{)}
\PYG{g+gp}{\PYGZgt{}\PYGZgt{}\PYGZgt{} }\PYG{n}{q} \PYG{o}{=} \PYG{n}{p}\PYG{o}{.}\PYG{n}{inv}\PYG{p}{(}\PYG{p}{)}
\PYG{g+gp}{\PYGZgt{}\PYGZgt{}\PYGZgt{} }\PYG{n}{q}\PYG{o}{.}\PYG{n}{as\PYGZus{}euler}\PYG{p}{(}\PYG{l+s+s1}{\PYGZsq{}}\PYG{l+s+s1}{zyx}\PYG{l+s+s1}{\PYGZsq{}}\PYG{p}{,} \PYG{n}{degrees}\PYG{o}{=}\PYG{k+kc}{True}\PYG{p}{)}
\PYG{g+go}{array([\PYGZhy{}45.,   0.,   0.])}
\end{sphinxVerbatim}

\sphinxAtStartPar
Inverting multiple rotations:

\begin{sphinxVerbatim}[commandchars=\\\{\}]
\PYG{g+gp}{\PYGZgt{}\PYGZgt{}\PYGZgt{} }\PYG{n}{p} \PYG{o}{=} \PYG{n}{R}\PYG{o}{.}\PYG{n}{from\PYGZus{}rotvec}\PYG{p}{(}\PYG{p}{[}\PYG{p}{[}\PYG{l+m+mi}{0}\PYG{p}{,} \PYG{l+m+mi}{0}\PYG{p}{,} \PYG{n}{np}\PYG{o}{.}\PYG{n}{pi}\PYG{o}{/}\PYG{l+m+mi}{3}\PYG{p}{]}\PYG{p}{,} \PYG{p}{[}\PYG{o}{\PYGZhy{}}\PYG{n}{np}\PYG{o}{.}\PYG{n}{pi}\PYG{o}{/}\PYG{l+m+mi}{4}\PYG{p}{,} \PYG{l+m+mi}{0}\PYG{p}{,} \PYG{l+m+mi}{0}\PYG{p}{]}\PYG{p}{]}\PYG{p}{)}
\PYG{g+gp}{\PYGZgt{}\PYGZgt{}\PYGZgt{} }\PYG{n}{q} \PYG{o}{=} \PYG{n}{p}\PYG{o}{.}\PYG{n}{inv}\PYG{p}{(}\PYG{p}{)}
\PYG{g+gp}{\PYGZgt{}\PYGZgt{}\PYGZgt{} }\PYG{n}{q}\PYG{o}{.}\PYG{n}{as\PYGZus{}rotvec}\PYG{p}{(}\PYG{p}{)}
\PYG{g+go}{array([[\PYGZhy{}0.        , \PYGZhy{}0.        , \PYGZhy{}1.04719755],}
\PYG{g+go}{       [ 0.78539816, \PYGZhy{}0.        , \PYGZhy{}0.        ]])}
\end{sphinxVerbatim}

\end{fulllineitems}

\index{magnitude() (deepdrr.geo.Rotation method)@\spxentry{magnitude()}\spxextra{deepdrr.geo.Rotation method}}

\begin{fulllineitems}
\phantomsection\label{\detokenize{deepdrr.geo:id50}}
\pysigstartsignatures
\pysiglinewithargsret{\sphinxbfcode{\sphinxupquote{magnitude}}}{\sphinxparam{\DUrole{n,n}{self}}}{}
\pysigstopsignatures
\sphinxAtStartPar
Get the magnitude(s) of the rotation(s).
\begin{quote}\begin{description}
\sphinxlineitem{Returns}
\sphinxAtStartPar
\sphinxstylestrong{magnitude} \textendash{} Angle(s) in radians, float if object contains a single rotation
and ndarray if object contains multiple rotations.

\sphinxlineitem{Return type}
\sphinxAtStartPar
ndarray or float

\end{description}\end{quote}
\subsubsection*{Examples}

\begin{sphinxVerbatim}[commandchars=\\\{\}]
\PYG{g+gp}{\PYGZgt{}\PYGZgt{}\PYGZgt{} }\PYG{k+kn}{from} \PYG{n+nn}{scipy}\PYG{n+nn}{.}\PYG{n+nn}{spatial}\PYG{n+nn}{.}\PYG{n+nn}{transform} \PYG{k+kn}{import} \PYG{n}{Rotation} \PYG{k}{as} \PYG{n}{R}
\PYG{g+gp}{\PYGZgt{}\PYGZgt{}\PYGZgt{} }\PYG{k+kn}{import} \PYG{n+nn}{numpy} \PYG{k}{as} \PYG{n+nn}{np}
\PYG{g+gp}{\PYGZgt{}\PYGZgt{}\PYGZgt{} }\PYG{n}{r} \PYG{o}{=} \PYG{n}{R}\PYG{o}{.}\PYG{n}{from\PYGZus{}quat}\PYG{p}{(}\PYG{n}{np}\PYG{o}{.}\PYG{n}{eye}\PYG{p}{(}\PYG{l+m+mi}{4}\PYG{p}{)}\PYG{p}{)}
\PYG{g+gp}{\PYGZgt{}\PYGZgt{}\PYGZgt{} }\PYG{n}{r}\PYG{o}{.}\PYG{n}{magnitude}\PYG{p}{(}\PYG{p}{)}
\PYG{g+go}{array([3.14159265, 3.14159265, 3.14159265, 0.        ])}
\end{sphinxVerbatim}

\sphinxAtStartPar
Magnitude of a single rotation:

\begin{sphinxVerbatim}[commandchars=\\\{\}]
\PYG{g+gp}{\PYGZgt{}\PYGZgt{}\PYGZgt{} }\PYG{n}{r}\PYG{p}{[}\PYG{l+m+mi}{0}\PYG{p}{]}\PYG{o}{.}\PYG{n}{magnitude}\PYG{p}{(}\PYG{p}{)}
\PYG{g+go}{3.141592653589793}
\end{sphinxVerbatim}

\end{fulllineitems}

\index{mean() (deepdrr.geo.Rotation method)@\spxentry{mean()}\spxextra{deepdrr.geo.Rotation method}}

\begin{fulllineitems}
\phantomsection\label{\detokenize{deepdrr.geo:id51}}
\pysigstartsignatures
\pysiglinewithargsret{\sphinxbfcode{\sphinxupquote{mean}}}{\sphinxparam{\DUrole{n,n}{self}}\sphinxparamcomma \sphinxparam{\DUrole{n,n}{weights}\DUrole{o,o}{=}\DUrole{default_value}{None}}}{}
\pysigstopsignatures
\sphinxAtStartPar
Get the mean of the rotations.
\begin{quote}\begin{description}
\sphinxlineitem{Parameters}
\sphinxAtStartPar
\sphinxstyleliteralstrong{\sphinxupquote{weights}} (\sphinxstyleliteralemphasis{\sphinxupquote{array\_like shape}}\sphinxstyleliteralemphasis{\sphinxupquote{ (}}\sphinxstyleliteralemphasis{\sphinxupquote{N}}\sphinxstyleliteralemphasis{\sphinxupquote{,}}\sphinxstyleliteralemphasis{\sphinxupquote{)}}\sphinxstyleliteralemphasis{\sphinxupquote{, }}\sphinxstyleliteralemphasis{\sphinxupquote{optional}}) \textendash{} Weights describing the relative importance of the rotations. If
None (default), then all values in \sphinxtitleref{weights} are assumed to be
equal.

\sphinxlineitem{Returns}
\sphinxAtStartPar
\sphinxstylestrong{mean} \textendash{} Object containing the mean of the rotations in the current
instance.

\sphinxlineitem{Return type}
\sphinxAtStartPar
\sphinxtitleref{Rotation} instance

\end{description}\end{quote}
\subsubsection*{Notes}

\sphinxAtStartPar
The mean used is the chordal L2 mean (also called the projected or
induced arithmetic mean). If \sphinxcode{\sphinxupquote{p}} is a set of rotations with mean
\sphinxcode{\sphinxupquote{m}}, then \sphinxcode{\sphinxupquote{m}} is the rotation which minimizes
\sphinxcode{\sphinxupquote{(weights{[}:, None, None{]} * (p.as\_matrix() \sphinxhyphen{} m.as\_matrix())**2).sum()}}.
\subsubsection*{Examples}

\begin{sphinxVerbatim}[commandchars=\\\{\}]
\PYG{g+gp}{\PYGZgt{}\PYGZgt{}\PYGZgt{} }\PYG{k+kn}{from} \PYG{n+nn}{scipy}\PYG{n+nn}{.}\PYG{n+nn}{spatial}\PYG{n+nn}{.}\PYG{n+nn}{transform} \PYG{k+kn}{import} \PYG{n}{Rotation} \PYG{k}{as} \PYG{n}{R}
\PYG{g+gp}{\PYGZgt{}\PYGZgt{}\PYGZgt{} }\PYG{n}{r} \PYG{o}{=} \PYG{n}{R}\PYG{o}{.}\PYG{n}{from\PYGZus{}euler}\PYG{p}{(}\PYG{l+s+s1}{\PYGZsq{}}\PYG{l+s+s1}{zyx}\PYG{l+s+s1}{\PYGZsq{}}\PYG{p}{,} \PYG{p}{[}\PYG{p}{[}\PYG{l+m+mi}{0}\PYG{p}{,} \PYG{l+m+mi}{0}\PYG{p}{,} \PYG{l+m+mi}{0}\PYG{p}{]}\PYG{p}{,}
\PYG{g+gp}{... }                         \PYG{p}{[}\PYG{l+m+mi}{1}\PYG{p}{,} \PYG{l+m+mi}{0}\PYG{p}{,} \PYG{l+m+mi}{0}\PYG{p}{]}\PYG{p}{,}
\PYG{g+gp}{... }                         \PYG{p}{[}\PYG{l+m+mi}{0}\PYG{p}{,} \PYG{l+m+mi}{1}\PYG{p}{,} \PYG{l+m+mi}{0}\PYG{p}{]}\PYG{p}{,}
\PYG{g+gp}{... }                         \PYG{p}{[}\PYG{l+m+mi}{0}\PYG{p}{,} \PYG{l+m+mi}{0}\PYG{p}{,} \PYG{l+m+mi}{1}\PYG{p}{]}\PYG{p}{]}\PYG{p}{,} \PYG{n}{degrees}\PYG{o}{=}\PYG{k+kc}{True}\PYG{p}{)}
\PYG{g+gp}{\PYGZgt{}\PYGZgt{}\PYGZgt{} }\PYG{n}{r}\PYG{o}{.}\PYG{n}{mean}\PYG{p}{(}\PYG{p}{)}\PYG{o}{.}\PYG{n}{as\PYGZus{}euler}\PYG{p}{(}\PYG{l+s+s1}{\PYGZsq{}}\PYG{l+s+s1}{zyx}\PYG{l+s+s1}{\PYGZsq{}}\PYG{p}{,} \PYG{n}{degrees}\PYG{o}{=}\PYG{k+kc}{True}\PYG{p}{)}
\PYG{g+go}{array([0.24945696, 0.25054542, 0.24945696])}
\end{sphinxVerbatim}

\end{fulllineitems}

\index{random() (deepdrr.geo.Rotation method)@\spxentry{random()}\spxextra{deepdrr.geo.Rotation method}}

\begin{fulllineitems}
\phantomsection\label{\detokenize{deepdrr.geo:id52}}
\pysigstartsignatures
\pysiglinewithargsret{\sphinxbfcode{\sphinxupquote{random}}}{\sphinxparam{\DUrole{n,n}{type cls}}\sphinxparamcomma \sphinxparam{\DUrole{n,n}{num=None}}\sphinxparamcomma \sphinxparam{\DUrole{n,n}{random\_state=None}}}{}
\pysigstopsignatures
\sphinxAtStartPar
Generate uniformly distributed rotations.
\begin{quote}\begin{description}
\sphinxlineitem{Parameters}\begin{itemize}
\item {} 
\sphinxAtStartPar
\sphinxstyleliteralstrong{\sphinxupquote{num}} (\sphinxstyleliteralemphasis{\sphinxupquote{int}}\sphinxstyleliteralemphasis{\sphinxupquote{ or }}\sphinxstyleliteralemphasis{\sphinxupquote{None}}\sphinxstyleliteralemphasis{\sphinxupquote{, }}\sphinxstyleliteralemphasis{\sphinxupquote{optional}}) \textendash{} Number of random rotations to generate. If None (default), then a
single rotation is generated.

\item {} 
\sphinxAtStartPar
\sphinxstyleliteralstrong{\sphinxupquote{random\_state}} (\{None, int, \sphinxtitleref{numpy.random.Generator},) \textendash{} 
\sphinxAtStartPar
\sphinxtitleref{numpy.random.RandomState}\}, optional

\sphinxAtStartPar
If \sphinxtitleref{seed} is None (or \sphinxtitleref{np.random}), the \sphinxtitleref{numpy.random.RandomState}
singleton is used.
If \sphinxtitleref{seed} is an int, a new \sphinxcode{\sphinxupquote{RandomState}} instance is used,
seeded with \sphinxtitleref{seed}.
If \sphinxtitleref{seed} is already a \sphinxcode{\sphinxupquote{Generator}} or \sphinxcode{\sphinxupquote{RandomState}} instance
then that instance is used.


\end{itemize}

\sphinxlineitem{Returns}
\sphinxAtStartPar
\sphinxstylestrong{random\_rotation} \textendash{} Contains a single rotation if \sphinxtitleref{num} is None. Otherwise contains a
stack of \sphinxtitleref{num} rotations.

\sphinxlineitem{Return type}
\sphinxAtStartPar
\sphinxtitleref{Rotation} instance

\end{description}\end{quote}
\subsubsection*{Notes}

\sphinxAtStartPar
This function is optimized for efficiently sampling random rotation
matrices in three dimensions. For generating random rotation matrices
in higher dimensions, see \sphinxtitleref{scipy.stats.special\_ortho\_group}.
\subsubsection*{Examples}

\begin{sphinxVerbatim}[commandchars=\\\{\}]
\PYG{g+gp}{\PYGZgt{}\PYGZgt{}\PYGZgt{} }\PYG{k+kn}{from} \PYG{n+nn}{scipy}\PYG{n+nn}{.}\PYG{n+nn}{spatial}\PYG{n+nn}{.}\PYG{n+nn}{transform} \PYG{k+kn}{import} \PYG{n}{Rotation} \PYG{k}{as} \PYG{n}{R}
\end{sphinxVerbatim}

\sphinxAtStartPar
Sample a single rotation:

\begin{sphinxVerbatim}[commandchars=\\\{\}]
\PYG{g+gp}{\PYGZgt{}\PYGZgt{}\PYGZgt{} }\PYG{n}{R}\PYG{o}{.}\PYG{n}{random}\PYG{p}{(}\PYG{p}{)}\PYG{o}{.}\PYG{n}{as\PYGZus{}euler}\PYG{p}{(}\PYG{l+s+s1}{\PYGZsq{}}\PYG{l+s+s1}{zxy}\PYG{l+s+s1}{\PYGZsq{}}\PYG{p}{,} \PYG{n}{degrees}\PYG{o}{=}\PYG{k+kc}{True}\PYG{p}{)}
\PYG{g+go}{array([\PYGZhy{}110.5976185 ,   55.32758512,   76.3289269 ])  \PYGZsh{} random}
\end{sphinxVerbatim}

\sphinxAtStartPar
Sample a stack of rotations:

\begin{sphinxVerbatim}[commandchars=\\\{\}]
\PYG{g+gp}{\PYGZgt{}\PYGZgt{}\PYGZgt{} }\PYG{n}{R}\PYG{o}{.}\PYG{n}{random}\PYG{p}{(}\PYG{l+m+mi}{5}\PYG{p}{)}\PYG{o}{.}\PYG{n}{as\PYGZus{}euler}\PYG{p}{(}\PYG{l+s+s1}{\PYGZsq{}}\PYG{l+s+s1}{zxy}\PYG{l+s+s1}{\PYGZsq{}}\PYG{p}{,} \PYG{n}{degrees}\PYG{o}{=}\PYG{k+kc}{True}\PYG{p}{)}
\PYG{g+go}{array([[\PYGZhy{}110.5976185 ,   55.32758512,   76.3289269 ],  \PYGZsh{} random}
\PYG{g+go}{       [ \PYGZhy{}91.59132005,  \PYGZhy{}14.3629884 ,  \PYGZhy{}93.91933182],}
\PYG{g+go}{       [  25.23835501,   45.02035145, \PYGZhy{}121.67867086],}
\PYG{g+go}{       [ \PYGZhy{}51.51414184,  \PYGZhy{}15.29022692, \PYGZhy{}172.46870023],}
\PYG{g+go}{       [ \PYGZhy{}81.63376847,  \PYGZhy{}27.39521579,    2.60408416]])}
\end{sphinxVerbatim}


\begin{sphinxseealso}{See also:}

\sphinxAtStartPar
\sphinxcode{\sphinxupquote{scipy.stats.special\_ortho\_group}}


\end{sphinxseealso}


\end{fulllineitems}

\index{reduce() (deepdrr.geo.Rotation method)@\spxentry{reduce()}\spxextra{deepdrr.geo.Rotation method}}

\begin{fulllineitems}
\phantomsection\label{\detokenize{deepdrr.geo:id53}}
\pysigstartsignatures
\pysiglinewithargsret{\sphinxbfcode{\sphinxupquote{reduce}}}{\sphinxparam{\DUrole{n,n}{self}}\sphinxparamcomma \sphinxparam{\DUrole{n,n}{left}\DUrole{o,o}{=}\DUrole{default_value}{None}}\sphinxparamcomma \sphinxparam{\DUrole{n,n}{right}\DUrole{o,o}{=}\DUrole{default_value}{None}}\sphinxparamcomma \sphinxparam{\DUrole{n,n}{return\_indices}\DUrole{o,o}{=}\DUrole{default_value}{False}}}{}
\pysigstopsignatures
\sphinxAtStartPar
Reduce this rotation with the provided rotation groups.

\sphinxAtStartPar
Reduction of a rotation \sphinxcode{\sphinxupquote{p}} is a transformation of the form
\sphinxcode{\sphinxupquote{q = l * p * r}}, where \sphinxcode{\sphinxupquote{l}} and \sphinxcode{\sphinxupquote{r}} are chosen from \sphinxtitleref{left} and
\sphinxtitleref{right} respectively, such that rotation \sphinxcode{\sphinxupquote{q}} has the smallest
magnitude.

\sphinxAtStartPar
If \sphinxtitleref{left} and \sphinxtitleref{right} are rotation groups representing symmetries of
two objects rotated by \sphinxcode{\sphinxupquote{p}}, then \sphinxcode{\sphinxupquote{q}} is the rotation of the
smallest magnitude to align these objects considering their symmetries.
\begin{quote}\begin{description}
\sphinxlineitem{Parameters}\begin{itemize}
\item {} 
\sphinxAtStartPar
\sphinxstyleliteralstrong{\sphinxupquote{left}} (\sphinxtitleref{Rotation} instance, optional) \textendash{} Object containing the left rotation(s). Default value (None)
corresponds to the identity rotation.

\item {} 
\sphinxAtStartPar
\sphinxstyleliteralstrong{\sphinxupquote{right}} (\sphinxtitleref{Rotation} instance, optional) \textendash{} Object containing the right rotation(s). Default value (None)
corresponds to the identity rotation.

\item {} 
\sphinxAtStartPar
\sphinxstyleliteralstrong{\sphinxupquote{return\_indices}} (\sphinxstyleliteralemphasis{\sphinxupquote{bool}}\sphinxstyleliteralemphasis{\sphinxupquote{, }}\sphinxstyleliteralemphasis{\sphinxupquote{optional}}) \textendash{} Whether to return the indices of the rotations from \sphinxtitleref{left} and
\sphinxtitleref{right} used for reduction.

\end{itemize}

\sphinxlineitem{Returns}
\sphinxAtStartPar
\begin{itemize}
\item {} 
\sphinxAtStartPar
\sphinxstylestrong{reduced} (\sphinxtitleref{Rotation} instance) \textendash{} Object containing reduced rotations.

\item {} 
\sphinxAtStartPar
\sphinxstylestrong{left\_best, right\_best} (\sphinxstyleemphasis{integer ndarray}) \textendash{} Indices of elements from \sphinxtitleref{left} and \sphinxtitleref{right} used for reduction.

\end{itemize}


\end{description}\end{quote}

\end{fulllineitems}

\index{single (deepdrr.geo.Rotation attribute)@\spxentry{single}\spxextra{deepdrr.geo.Rotation attribute}}

\begin{fulllineitems}
\phantomsection\label{\detokenize{deepdrr.geo:id54}}
\pysigstartsignatures
\pysigline{\sphinxbfcode{\sphinxupquote{single}}}
\pysigstopsignatures
\sphinxAtStartPar
Whether this instance represents a single rotation.

\end{fulllineitems}


\end{fulllineitems}

\index{Segment (class in deepdrr.geo)@\spxentry{Segment}\spxextra{class in deepdrr.geo}}

\begin{fulllineitems}
\phantomsection\label{\detokenize{deepdrr.geo:deepdrr.geo.Segment}}
\pysigstartsignatures
\pysiglinewithargsret{\sphinxbfcode{\sphinxupquote{class\DUrole{w,w}{  }}}\sphinxcode{\sphinxupquote{deepdrr.geo.}}\sphinxbfcode{\sphinxupquote{Segment}}}{\sphinxparam{\DUrole{n,n}{data}\DUrole{p,p}{:}\DUrole{w,w}{  }\DUrole{n,n}{ndarray}}}{}
\pysigstopsignatures
\sphinxAtStartPar
Bases: {\hyperref[\detokenize{deepdrr.geo:deepdrr.geo.core.HasLocationAndDirection}]{\sphinxcrossref{\sphinxcode{\sphinxupquote{HasLocationAndDirection}}}}}, {\hyperref[\detokenize{deepdrr.geo:deepdrr.geo.core.Meetable}]{\sphinxcrossref{\sphinxcode{\sphinxupquote{Meetable}}}}}
\index{data (deepdrr.geo.Segment attribute)@\spxentry{data}\spxextra{deepdrr.geo.Segment attribute}}

\begin{fulllineitems}
\phantomsection\label{\detokenize{deepdrr.geo:deepdrr.geo.Segment.data}}
\pysigstartsignatures
\pysigline{\sphinxbfcode{\sphinxupquote{data}}\sphinxbfcode{\sphinxupquote{\DUrole{p,p}{:}\DUrole{w,w}{  }ndarray}}}
\pysigstopsignatures
\end{fulllineitems}

\index{from\_pn() (deepdrr.geo.Segment class method)@\spxentry{from\_pn()}\spxextra{deepdrr.geo.Segment class method}}

\begin{fulllineitems}
\phantomsection\label{\detokenize{deepdrr.geo:deepdrr.geo.Segment.from_pn}}
\pysigstartsignatures
\pysiglinewithargsret{\sphinxbfcode{\sphinxupquote{classmethod\DUrole{w,w}{  }}}\sphinxbfcode{\sphinxupquote{from\_pn}}}{\sphinxparam{\DUrole{n,n}{p}\DUrole{p,p}{:}\DUrole{w,w}{  }\DUrole{n,n}{{\hyperref[\detokenize{deepdrr.geo:deepdrr.geo.core.Point}]{\sphinxcrossref{Point}}}}}\sphinxparamcomma \sphinxparam{\DUrole{n,n}{n}\DUrole{p,p}{:}\DUrole{w,w}{  }\DUrole{n,n}{{\hyperref[\detokenize{deepdrr.geo:deepdrr.geo.core.Vector}]{\sphinxcrossref{Vector}}}}}}{{ $\rightarrow$ S}}
\pysigstopsignatures
\sphinxAtStartPar
Initialize the segment with a poind and a direction.
\begin{quote}\begin{description}
\sphinxlineitem{Parameters}\begin{itemize}
\item {} 
\sphinxAtStartPar
\sphinxstyleliteralstrong{\sphinxupquote{p}} ({\hyperref[\detokenize{deepdrr.geo:deepdrr.geo.Point}]{\sphinxcrossref{\sphinxstyleliteralemphasis{\sphinxupquote{Point}}}}}) \textendash{} The first point.

\item {} 
\sphinxAtStartPar
\sphinxstyleliteralstrong{\sphinxupquote{n}} ({\hyperref[\detokenize{deepdrr.geo:deepdrr.geo.Vector}]{\sphinxcrossref{\sphinxstyleliteralemphasis{\sphinxupquote{Vector}}}}}) \textendash{} The direction vector.

\end{itemize}

\sphinxlineitem{Returns}
\sphinxAtStartPar
The segment.

\sphinxlineitem{Return type}
\sphinxAtStartPar
{\hyperref[\detokenize{deepdrr.geo:deepdrr.geo.Segment}]{\sphinxcrossref{Segment}}}

\end{description}\end{quote}

\end{fulllineitems}

\index{from\_point\_direction() (deepdrr.geo.Segment class method)@\spxentry{from\_point\_direction()}\spxextra{deepdrr.geo.Segment class method}}

\begin{fulllineitems}
\phantomsection\label{\detokenize{deepdrr.geo:deepdrr.geo.Segment.from_point_direction}}
\pysigstartsignatures
\pysiglinewithargsret{\sphinxbfcode{\sphinxupquote{classmethod\DUrole{w,w}{  }}}\sphinxbfcode{\sphinxupquote{from\_point\_direction}}}{\sphinxparam{\DUrole{n,n}{p}\DUrole{p,p}{:}\DUrole{w,w}{  }\DUrole{n,n}{{\hyperref[\detokenize{deepdrr.geo:deepdrr.geo.core.Point}]{\sphinxcrossref{Point}}}}}\sphinxparamcomma \sphinxparam{\DUrole{n,n}{n}\DUrole{p,p}{:}\DUrole{w,w}{  }\DUrole{n,n}{{\hyperref[\detokenize{deepdrr.geo:deepdrr.geo.core.Vector}]{\sphinxcrossref{Vector}}}}}}{{ $\rightarrow$ S}}
\pysigstopsignatures
\sphinxAtStartPar
Initialize the segment with a poind and a direction.
\begin{quote}\begin{description}
\sphinxlineitem{Parameters}\begin{itemize}
\item {} 
\sphinxAtStartPar
\sphinxstyleliteralstrong{\sphinxupquote{p}} ({\hyperref[\detokenize{deepdrr.geo:deepdrr.geo.Point}]{\sphinxcrossref{\sphinxstyleliteralemphasis{\sphinxupquote{Point}}}}}) \textendash{} The first point.

\item {} 
\sphinxAtStartPar
\sphinxstyleliteralstrong{\sphinxupquote{n}} ({\hyperref[\detokenize{deepdrr.geo:deepdrr.geo.Vector}]{\sphinxcrossref{\sphinxstyleliteralemphasis{\sphinxupquote{Vector}}}}}) \textendash{} The direction vector.

\end{itemize}

\sphinxlineitem{Returns}
\sphinxAtStartPar
The segment.

\sphinxlineitem{Return type}
\sphinxAtStartPar
{\hyperref[\detokenize{deepdrr.geo:deepdrr.geo.Segment}]{\sphinxcrossref{Segment}}}

\end{description}\end{quote}

\end{fulllineitems}

\index{from\_pq() (deepdrr.geo.Segment class method)@\spxentry{from\_pq()}\spxextra{deepdrr.geo.Segment class method}}

\begin{fulllineitems}
\phantomsection\label{\detokenize{deepdrr.geo:deepdrr.geo.Segment.from_pq}}
\pysigstartsignatures
\pysiglinewithargsret{\sphinxbfcode{\sphinxupquote{classmethod\DUrole{w,w}{  }}}\sphinxbfcode{\sphinxupquote{from\_pq}}}{\sphinxparam{\DUrole{n,n}{p}\DUrole{p,p}{:}\DUrole{w,w}{  }\DUrole{n,n}{{\hyperref[\detokenize{deepdrr.geo:deepdrr.geo.core.Point}]{\sphinxcrossref{Point}}}}}\sphinxparamcomma \sphinxparam{\DUrole{n,n}{q}\DUrole{p,p}{:}\DUrole{w,w}{  }\DUrole{n,n}{{\hyperref[\detokenize{deepdrr.geo:deepdrr.geo.core.Point}]{\sphinxcrossref{Point}}}}}}{{ $\rightarrow$ S}}
\pysigstopsignatures
\sphinxAtStartPar
Initialize the segment containing two points.
\begin{quote}\begin{description}
\sphinxlineitem{Parameters}\begin{itemize}
\item {} 
\sphinxAtStartPar
\sphinxstyleliteralstrong{\sphinxupquote{p}} ({\hyperref[\detokenize{deepdrr.geo:deepdrr.geo.Point}]{\sphinxcrossref{\sphinxstyleliteralemphasis{\sphinxupquote{Point}}}}}) \textendash{} The first point.

\item {} 
\sphinxAtStartPar
\sphinxstyleliteralstrong{\sphinxupquote{q}} ({\hyperref[\detokenize{deepdrr.geo:deepdrr.geo.Point}]{\sphinxcrossref{\sphinxstyleliteralemphasis{\sphinxupquote{Point}}}}}) \textendash{} The second point.

\end{itemize}

\sphinxlineitem{Returns}
\sphinxAtStartPar
The segment.

\sphinxlineitem{Return type}
\sphinxAtStartPar
{\hyperref[\detokenize{deepdrr.geo:deepdrr.geo.Segment}]{\sphinxcrossref{Segment}}}

\end{description}\end{quote}

\end{fulllineitems}

\index{get\_direction() (deepdrr.geo.Segment method)@\spxentry{get\_direction()}\spxextra{deepdrr.geo.Segment method}}

\begin{fulllineitems}
\phantomsection\label{\detokenize{deepdrr.geo:deepdrr.geo.Segment.get_direction}}
\pysigstartsignatures
\pysiglinewithargsret{\sphinxbfcode{\sphinxupquote{get\_direction}}}{}{{ $\rightarrow$ {\hyperref[\detokenize{deepdrr.geo:deepdrr.geo.core.Vector}]{\sphinxcrossref{Vector}}}}}
\pysigstopsignatures
\sphinxAtStartPar
Get the direction associated with the object.
\begin{quote}\begin{description}
\sphinxlineitem{Returns}
\sphinxAtStartPar
the direction of the object.

\sphinxlineitem{Return type}
\sphinxAtStartPar
{\hyperref[\detokenize{deepdrr.geo:deepdrr.geo.Vector}]{\sphinxcrossref{Vector}}}

\end{description}\end{quote}

\end{fulllineitems}

\index{get\_point() (deepdrr.geo.Segment method)@\spxentry{get\_point()}\spxextra{deepdrr.geo.Segment method}}

\begin{fulllineitems}
\phantomsection\label{\detokenize{deepdrr.geo:deepdrr.geo.Segment.get_point}}
\pysigstartsignatures
\pysiglinewithargsret{\sphinxbfcode{\sphinxupquote{get\_point}}}{}{{ $\rightarrow$ {\hyperref[\detokenize{deepdrr.geo:deepdrr.geo.core.Point}]{\sphinxcrossref{Point}}}}}
\pysigstopsignatures
\sphinxAtStartPar
Get the location of the object.
\begin{quote}\begin{description}
\sphinxlineitem{Returns}
\sphinxAtStartPar
the location of the object.

\sphinxlineitem{Return type}
\sphinxAtStartPar
{\hyperref[\detokenize{deepdrr.geo:deepdrr.geo.Point}]{\sphinxcrossref{Point}}}

\end{description}\end{quote}

\end{fulllineitems}

\index{length() (deepdrr.geo.Segment method)@\spxentry{length()}\spxextra{deepdrr.geo.Segment method}}

\begin{fulllineitems}
\phantomsection\label{\detokenize{deepdrr.geo:deepdrr.geo.Segment.length}}
\pysigstartsignatures
\pysiglinewithargsret{\sphinxbfcode{\sphinxupquote{length}}}{}{{ $\rightarrow$ float}}
\pysigstopsignatures
\sphinxAtStartPar
Get the length of the segment.
\begin{quote}\begin{description}
\sphinxlineitem{Returns}
\sphinxAtStartPar
The length of the segment.

\sphinxlineitem{Return type}
\sphinxAtStartPar
float

\end{description}\end{quote}

\end{fulllineitems}

\index{line() (deepdrr.geo.Segment method)@\spxentry{line()}\spxextra{deepdrr.geo.Segment method}}

\begin{fulllineitems}
\phantomsection\label{\detokenize{deepdrr.geo:deepdrr.geo.Segment.line}}
\pysigstartsignatures
\pysiglinewithargsret{\sphinxbfcode{\sphinxupquote{line}}}{}{{ $\rightarrow$ {\hyperref[\detokenize{deepdrr.geo:deepdrr.geo.Line2D}]{\sphinxcrossref{Line2D}}}}}
\pysiglinewithargsret{\sphinxbfcode{\sphinxupquote{line}}}{}{{ $\rightarrow$ {\hyperref[\detokenize{deepdrr.geo:deepdrr.geo.Line3D}]{\sphinxcrossref{Line3D}}}}}
\pysigstopsignatures
\end{fulllineitems}

\index{midpoint() (deepdrr.geo.Segment method)@\spxentry{midpoint()}\spxextra{deepdrr.geo.Segment method}}

\begin{fulllineitems}
\phantomsection\label{\detokenize{deepdrr.geo:deepdrr.geo.Segment.midpoint}}
\pysigstartsignatures
\pysiglinewithargsret{\sphinxbfcode{\sphinxupquote{midpoint}}}{}{{ $\rightarrow$ {\hyperref[\detokenize{deepdrr.geo:deepdrr.geo.core.Point}]{\sphinxcrossref{Point}}}}}
\pysigstopsignatures
\end{fulllineitems}

\index{p (deepdrr.geo.Segment property)@\spxentry{p}\spxextra{deepdrr.geo.Segment property}}

\begin{fulllineitems}
\phantomsection\label{\detokenize{deepdrr.geo:deepdrr.geo.Segment.p}}
\pysigstartsignatures
\pysigline{\sphinxbfcode{\sphinxupquote{property\DUrole{w,w}{  }}}\sphinxbfcode{\sphinxupquote{p}}\sphinxbfcode{\sphinxupquote{\DUrole{p,p}{:}\DUrole{w,w}{  }{\hyperref[\detokenize{deepdrr.geo:deepdrr.geo.core.Point}]{\sphinxcrossref{Point}}}}}}
\pysigstopsignatures
\sphinxAtStartPar
Get the first point of the segment.
\begin{quote}\begin{description}
\sphinxlineitem{Returns}
\sphinxAtStartPar
The first point of the segment.

\sphinxlineitem{Return type}
\sphinxAtStartPar
{\hyperref[\detokenize{deepdrr.geo:deepdrr.geo.Point2D}]{\sphinxcrossref{Point2D}}}

\end{description}\end{quote}

\end{fulllineitems}

\index{q (deepdrr.geo.Segment property)@\spxentry{q}\spxextra{deepdrr.geo.Segment property}}

\begin{fulllineitems}
\phantomsection\label{\detokenize{deepdrr.geo:deepdrr.geo.Segment.q}}
\pysigstartsignatures
\pysigline{\sphinxbfcode{\sphinxupquote{property\DUrole{w,w}{  }}}\sphinxbfcode{\sphinxupquote{q}}\sphinxbfcode{\sphinxupquote{\DUrole{p,p}{:}\DUrole{w,w}{  }{\hyperref[\detokenize{deepdrr.geo:deepdrr.geo.core.Point}]{\sphinxcrossref{Point}}}}}}
\pysigstopsignatures
\sphinxAtStartPar
Get the second point of the segment.
\begin{quote}\begin{description}
\sphinxlineitem{Returns}
\sphinxAtStartPar
The second point of the segment.

\sphinxlineitem{Return type}
\sphinxAtStartPar
{\hyperref[\detokenize{deepdrr.geo:deepdrr.geo.Point2D}]{\sphinxcrossref{Point2D}}}

\end{description}\end{quote}

\end{fulllineitems}


\end{fulllineitems}

\index{Segment2D (class in deepdrr.geo)@\spxentry{Segment2D}\spxextra{class in deepdrr.geo}}

\begin{fulllineitems}
\phantomsection\label{\detokenize{deepdrr.geo:deepdrr.geo.Segment2D}}
\pysigstartsignatures
\pysiglinewithargsret{\sphinxbfcode{\sphinxupquote{class\DUrole{w,w}{  }}}\sphinxcode{\sphinxupquote{deepdrr.geo.}}\sphinxbfcode{\sphinxupquote{Segment2D}}}{\sphinxparam{\DUrole{n,n}{data}\DUrole{p,p}{:}\DUrole{w,w}{  }\DUrole{n,n}{ndarray}}}{}
\pysigstopsignatures
\sphinxAtStartPar
Bases: {\hyperref[\detokenize{deepdrr.geo:deepdrr.geo.segment.Segment}]{\sphinxcrossref{\sphinxcode{\sphinxupquote{Segment}}}}}

\sphinxAtStartPar
Represents a line segment in 2D.
\index{data (deepdrr.geo.Segment2D attribute)@\spxentry{data}\spxextra{deepdrr.geo.Segment2D attribute}}

\begin{fulllineitems}
\phantomsection\label{\detokenize{deepdrr.geo:deepdrr.geo.Segment2D.data}}
\pysigstartsignatures
\pysigline{\sphinxbfcode{\sphinxupquote{data}}\sphinxbfcode{\sphinxupquote{\DUrole{p,p}{:}\DUrole{w,w}{  }ndarray}}}
\pysigstopsignatures
\end{fulllineitems}

\index{dim (deepdrr.geo.Segment2D attribute)@\spxentry{dim}\spxextra{deepdrr.geo.Segment2D attribute}}

\begin{fulllineitems}
\phantomsection\label{\detokenize{deepdrr.geo:deepdrr.geo.Segment2D.dim}}
\pysigstartsignatures
\pysigline{\sphinxbfcode{\sphinxupquote{dim}}\sphinxbfcode{\sphinxupquote{\DUrole{w,w}{  }\DUrole{p,p}{=}\DUrole{w,w}{  }2}}}
\pysigstopsignatures
\end{fulllineitems}

\index{meet() (deepdrr.geo.Segment2D method)@\spxentry{meet()}\spxextra{deepdrr.geo.Segment2D method}}

\begin{fulllineitems}
\phantomsection\label{\detokenize{deepdrr.geo:deepdrr.geo.Segment2D.meet}}
\pysigstartsignatures
\pysiglinewithargsret{\sphinxbfcode{\sphinxupquote{meet}}}{\sphinxparam{\DUrole{n,n}{other}\DUrole{p,p}{:}\DUrole{w,w}{  }\DUrole{n,n}{{\hyperref[\detokenize{deepdrr.geo:deepdrr.geo.Line2D}]{\sphinxcrossref{Line2D}}}\DUrole{w,w}{  }\DUrole{p,p}{|}\DUrole{w,w}{  }{\hyperref[\detokenize{deepdrr.geo:deepdrr.geo.Segment2D}]{\sphinxcrossref{Segment2D}}}}}}{{ $\rightarrow$ {\hyperref[\detokenize{deepdrr.geo:deepdrr.geo.Point2D}]{\sphinxcrossref{Point2D}}}}}
\pysigstopsignatures
\sphinxAtStartPar
Get the point of intersection between this segment and another line.
\begin{quote}\begin{description}
\sphinxlineitem{Parameters}
\sphinxAtStartPar
\sphinxstyleliteralstrong{\sphinxupquote{other}} ({\hyperref[\detokenize{deepdrr.geo:deepdrr.geo.Line2D}]{\sphinxcrossref{\sphinxstyleliteralemphasis{\sphinxupquote{Line2D}}}}}) \textendash{} The other line.

\sphinxlineitem{Returns}
\sphinxAtStartPar
The point of intersection.

\sphinxlineitem{Return type}
\sphinxAtStartPar
{\hyperref[\detokenize{deepdrr.geo:deepdrr.geo.Point2D}]{\sphinxcrossref{Point2D}}}

\end{description}\end{quote}

\end{fulllineitems}


\end{fulllineitems}

\index{Segment3D (class in deepdrr.geo)@\spxentry{Segment3D}\spxextra{class in deepdrr.geo}}

\begin{fulllineitems}
\phantomsection\label{\detokenize{deepdrr.geo:deepdrr.geo.Segment3D}}
\pysigstartsignatures
\pysiglinewithargsret{\sphinxbfcode{\sphinxupquote{class\DUrole{w,w}{  }}}\sphinxcode{\sphinxupquote{deepdrr.geo.}}\sphinxbfcode{\sphinxupquote{Segment3D}}}{\sphinxparam{\DUrole{n,n}{data}\DUrole{p,p}{:}\DUrole{w,w}{  }\DUrole{n,n}{ndarray}}}{}
\pysigstopsignatures
\sphinxAtStartPar
Bases: {\hyperref[\detokenize{deepdrr.geo:deepdrr.geo.segment.Segment}]{\sphinxcrossref{\sphinxcode{\sphinxupquote{Segment}}}}}, {\hyperref[\detokenize{deepdrr.geo:deepdrr.geo.core.Joinable}]{\sphinxcrossref{\sphinxcode{\sphinxupquote{Joinable}}}}}, {\hyperref[\detokenize{deepdrr.geo:deepdrr.geo.core.HasProjection}]{\sphinxcrossref{\sphinxcode{\sphinxupquote{HasProjection}}}}}

\sphinxAtStartPar
Represents a segment in 3D.
\index{data (deepdrr.geo.Segment3D attribute)@\spxentry{data}\spxextra{deepdrr.geo.Segment3D attribute}}

\begin{fulllineitems}
\phantomsection\label{\detokenize{deepdrr.geo:deepdrr.geo.Segment3D.data}}
\pysigstartsignatures
\pysigline{\sphinxbfcode{\sphinxupquote{data}}\sphinxbfcode{\sphinxupquote{\DUrole{p,p}{:}\DUrole{w,w}{  }ndarray}}}
\pysigstopsignatures
\end{fulllineitems}

\index{dim (deepdrr.geo.Segment3D attribute)@\spxentry{dim}\spxextra{deepdrr.geo.Segment3D attribute}}

\begin{fulllineitems}
\phantomsection\label{\detokenize{deepdrr.geo:deepdrr.geo.Segment3D.dim}}
\pysigstartsignatures
\pysigline{\sphinxbfcode{\sphinxupquote{dim}}\sphinxbfcode{\sphinxupquote{\DUrole{w,w}{  }\DUrole{p,p}{=}\DUrole{w,w}{  }3}}}
\pysigstopsignatures
\end{fulllineitems}

\index{join() (deepdrr.geo.Segment3D method)@\spxentry{join()}\spxextra{deepdrr.geo.Segment3D method}}

\begin{fulllineitems}
\phantomsection\label{\detokenize{deepdrr.geo:deepdrr.geo.Segment3D.join}}
\pysigstartsignatures
\pysiglinewithargsret{\sphinxbfcode{\sphinxupquote{join}}}{\sphinxparam{\DUrole{n,n}{other}\DUrole{p,p}{:}\DUrole{w,w}{  }\DUrole{n,n}{{\hyperref[\detokenize{deepdrr.geo:deepdrr.geo.Point3D}]{\sphinxcrossref{Point3D}}}}}}{{ $\rightarrow$ {\hyperref[\detokenize{deepdrr.geo:deepdrr.geo.Plane}]{\sphinxcrossref{Plane}}}}}
\pysigstopsignatures
\sphinxAtStartPar
Join two objects.

\sphinxAtStartPar
For example, given two points, get the line that connects them.
\begin{quote}\begin{description}
\sphinxlineitem{Parameters}
\sphinxAtStartPar
\sphinxstyleliteralstrong{\sphinxupquote{other}} ({\hyperref[\detokenize{deepdrr.geo:deepdrr.geo.core.Primitive}]{\sphinxcrossref{\sphinxstyleliteralemphasis{\sphinxupquote{Primitive}}}}}) \textendash{} the other primitive.

\sphinxlineitem{Returns}
\sphinxAtStartPar
the joined primitive.

\sphinxlineitem{Return type}
\sphinxAtStartPar
{\hyperref[\detokenize{deepdrr.geo:deepdrr.geo.core.Primitive}]{\sphinxcrossref{Primitive}}}

\end{description}\end{quote}

\end{fulllineitems}

\index{meet() (deepdrr.geo.Segment3D method)@\spxentry{meet()}\spxextra{deepdrr.geo.Segment3D method}}

\begin{fulllineitems}
\phantomsection\label{\detokenize{deepdrr.geo:deepdrr.geo.Segment3D.meet}}
\pysigstartsignatures
\pysiglinewithargsret{\sphinxbfcode{\sphinxupquote{meet}}}{\sphinxparam{\DUrole{n,n}{other}\DUrole{p,p}{:}\DUrole{w,w}{  }\DUrole{n,n}{{\hyperref[\detokenize{deepdrr.geo:deepdrr.geo.Plane}]{\sphinxcrossref{Plane}}}}}}{{ $\rightarrow$ {\hyperref[\detokenize{deepdrr.geo:deepdrr.geo.Point3D}]{\sphinxcrossref{Point3D}}}}}
\pysigstopsignatures
\sphinxAtStartPar
Get the point of intersection between this segment and a plane.

\sphinxAtStartPar
TODO: check if the intersection is on the segment.

\end{fulllineitems}

\index{projection\_type() (deepdrr.geo.Segment3D class method)@\spxentry{projection\_type()}\spxextra{deepdrr.geo.Segment3D class method}}

\begin{fulllineitems}
\phantomsection\label{\detokenize{deepdrr.geo:deepdrr.geo.Segment3D.projection_type}}
\pysigstartsignatures
\pysiglinewithargsret{\sphinxbfcode{\sphinxupquote{classmethod\DUrole{w,w}{  }}}\sphinxbfcode{\sphinxupquote{projection\_type}}}{}{{ $\rightarrow$ Type\DUrole{p,p}{{[}}{\hyperref[\detokenize{deepdrr.geo:deepdrr.geo.segment.Segment2D}]{\sphinxcrossref{Segment2D}}}\DUrole{p,p}{{]}}}}
\pysigstopsignatures
\sphinxAtStartPar
Get the type of the projection of the object.
\begin{quote}\begin{description}
\sphinxlineitem{Returns}
\sphinxAtStartPar
the type of the projection of the object.

\sphinxlineitem{Return type}
\sphinxAtStartPar
Type{[}{\hyperref[\detokenize{deepdrr.geo:deepdrr.geo.core.Primitive}]{\sphinxcrossref{Primitive}}}{]}

\end{description}\end{quote}

\end{fulllineitems}


\end{fulllineitems}

\index{Transform (class in deepdrr.geo)@\spxentry{Transform}\spxextra{class in deepdrr.geo}}

\begin{fulllineitems}
\phantomsection\label{\detokenize{deepdrr.geo:deepdrr.geo.Transform}}
\pysigstartsignatures
\pysiglinewithargsret{\sphinxbfcode{\sphinxupquote{class\DUrole{w,w}{  }}}\sphinxcode{\sphinxupquote{deepdrr.geo.}}\sphinxbfcode{\sphinxupquote{Transform}}}{\sphinxparam{\DUrole{n,n}{data}\DUrole{p,p}{:}\DUrole{w,w}{  }\DUrole{n,n}{ndarray}}\sphinxparamcomma \sphinxparam{\DUrole{n,n}{\_inv}\DUrole{p,p}{:}\DUrole{w,w}{  }\DUrole{n,n}{ndarray\DUrole{w,w}{  }\DUrole{p,p}{|}\DUrole{w,w}{  }None}\DUrole{w,w}{  }\DUrole{o,o}{=}\DUrole{w,w}{  }\DUrole{default_value}{None}}}{}
\pysigstopsignatures
\sphinxAtStartPar
Bases: {\hyperref[\detokenize{deepdrr.geo:deepdrr.geo.core.HomogeneousObject}]{\sphinxcrossref{\sphinxcode{\sphinxupquote{HomogeneousObject}}}}}
\index{data (deepdrr.geo.Transform attribute)@\spxentry{data}\spxextra{deepdrr.geo.Transform attribute}}

\begin{fulllineitems}
\phantomsection\label{\detokenize{deepdrr.geo:deepdrr.geo.Transform.data}}
\pysigstartsignatures
\pysigline{\sphinxbfcode{\sphinxupquote{data}}\sphinxbfcode{\sphinxupquote{\DUrole{p,p}{:}\DUrole{w,w}{  }ndarray}}}
\pysigstopsignatures
\end{fulllineitems}

\index{dim (deepdrr.geo.Transform property)@\spxentry{dim}\spxextra{deepdrr.geo.Transform property}}

\begin{fulllineitems}
\phantomsection\label{\detokenize{deepdrr.geo:deepdrr.geo.Transform.dim}}
\pysigstartsignatures
\pysigline{\sphinxbfcode{\sphinxupquote{property\DUrole{w,w}{  }}}\sphinxbfcode{\sphinxupquote{dim}}}
\pysigstopsignatures
\sphinxAtStartPar
The output dimension of the transformation.

\end{fulllineitems}

\index{from\_array() (deepdrr.geo.Transform class method)@\spxentry{from\_array()}\spxextra{deepdrr.geo.Transform class method}}

\begin{fulllineitems}
\phantomsection\label{\detokenize{deepdrr.geo:deepdrr.geo.Transform.from_array}}
\pysigstartsignatures
\pysiglinewithargsret{\sphinxbfcode{\sphinxupquote{classmethod\DUrole{w,w}{  }}}\sphinxbfcode{\sphinxupquote{from\_array}}}{\sphinxparam{\DUrole{n,n}{array}\DUrole{p,p}{:}\DUrole{w,w}{  }\DUrole{n,n}{ndarray}}}{{ $\rightarrow$ {\hyperref[\detokenize{deepdrr.geo:deepdrr.geo.core.Transform}]{\sphinxcrossref{Transform}}}}}
\pysigstopsignatures
\sphinxAtStartPar
Convert non\sphinxhyphen{}homogeneous matrix to homogeneous transform.

\sphinxAtStartPar
Usually, one would instantiate Transforms directly from the homogeneous matrix \sphinxtitleref{data} or using one of the other classmethods.
\begin{quote}\begin{description}
\sphinxlineitem{Parameters}
\sphinxAtStartPar
\sphinxstyleliteralstrong{\sphinxupquote{array}} (\sphinxstyleliteralemphasis{\sphinxupquote{np.ndarray}}) \textendash{} transformation matrix.

\sphinxlineitem{Returns}
\sphinxAtStartPar
the transform.

\sphinxlineitem{Return type}
\sphinxAtStartPar
{\hyperref[\detokenize{deepdrr.geo:deepdrr.geo.Transform}]{\sphinxcrossref{Transform}}}

\end{description}\end{quote}

\end{fulllineitems}

\index{get\_center() (deepdrr.geo.Transform method)@\spxentry{get\_center()}\spxextra{deepdrr.geo.Transform method}}

\begin{fulllineitems}
\phantomsection\label{\detokenize{deepdrr.geo:deepdrr.geo.Transform.get_center}}
\pysigstartsignatures
\pysiglinewithargsret{\sphinxbfcode{\sphinxupquote{get\_center}}}{}{{ $\rightarrow$ {\hyperref[\detokenize{deepdrr.geo:deepdrr.geo.core.Point3D}]{\sphinxcrossref{Point3D}}}}}
\pysigstopsignatures
\sphinxAtStartPar
If the transform is a projection, get the center of the projection.
\begin{quote}\begin{description}
\sphinxlineitem{Returns}
\sphinxAtStartPar
the center of the projection.

\sphinxlineitem{Return type}
\sphinxAtStartPar
({\hyperref[\detokenize{deepdrr.geo:deepdrr.geo.Point3D}]{\sphinxcrossref{Point3D}}})

\sphinxlineitem{Raises}
\sphinxAtStartPar
\sphinxstyleliteralstrong{\sphinxupquote{ValueError}} \textendash{} if the transform is not a projection.

\end{description}\end{quote}

\end{fulllineitems}

\index{input\_dim (deepdrr.geo.Transform property)@\spxentry{input\_dim}\spxextra{deepdrr.geo.Transform property}}

\begin{fulllineitems}
\phantomsection\label{\detokenize{deepdrr.geo:deepdrr.geo.Transform.input_dim}}
\pysigstartsignatures
\pysigline{\sphinxbfcode{\sphinxupquote{property\DUrole{w,w}{  }}}\sphinxbfcode{\sphinxupquote{input\_dim}}}
\pysigstopsignatures
\sphinxAtStartPar
The input dimension of the transformation.

\end{fulllineitems}

\index{inv (deepdrr.geo.Transform property)@\spxentry{inv}\spxextra{deepdrr.geo.Transform property}}

\begin{fulllineitems}
\phantomsection\label{\detokenize{deepdrr.geo:deepdrr.geo.Transform.inv}}
\pysigstartsignatures
\pysigline{\sphinxbfcode{\sphinxupquote{property\DUrole{w,w}{  }}}\sphinxbfcode{\sphinxupquote{inv}}\sphinxbfcode{\sphinxupquote{\DUrole{p,p}{:}\DUrole{w,w}{  }Self}}}
\pysigstopsignatures
\sphinxAtStartPar
Get the inverse of the Transform.
\begin{quote}\begin{description}
\sphinxlineitem{Returns}
\sphinxAtStartPar
a Transform (or subclass) that is well\sphinxhyphen{}defined as the inverse of this transform.

\sphinxlineitem{Return type}
\sphinxAtStartPar
({\hyperref[\detokenize{deepdrr.geo:deepdrr.geo.Transform}]{\sphinxcrossref{Transform}}})

\sphinxlineitem{Raises}
\sphinxAtStartPar
\sphinxstyleliteralstrong{\sphinxupquote{NotImplementedError}} \textendash{} if \_inv is None and method is not overriden.

\end{description}\end{quote}

\end{fulllineitems}

\index{inverse() (deepdrr.geo.Transform method)@\spxentry{inverse()}\spxextra{deepdrr.geo.Transform method}}

\begin{fulllineitems}
\phantomsection\label{\detokenize{deepdrr.geo:deepdrr.geo.Transform.inverse}}
\pysigstartsignatures
\pysiglinewithargsret{\sphinxbfcode{\sphinxupquote{inverse}}}{}{{ $\rightarrow$ {\hyperref[\detokenize{deepdrr.geo:deepdrr.geo.core.FrameTransform}]{\sphinxcrossref{FrameTransform}}}}}
\pysiglinewithargsret{\sphinxbfcode{\sphinxupquote{inverse}}}{}{{ $\rightarrow$ {\hyperref[\detokenize{deepdrr.geo:deepdrr.geo.core.Transform}]{\sphinxcrossref{Transform}}}}}
\pysigstopsignatures
\sphinxAtStartPar
Get the inverse of the Transform.
\begin{quote}\begin{description}
\sphinxlineitem{Returns}
\sphinxAtStartPar
a Transform (or subclass) that is well\sphinxhyphen{}defined as the inverse of this transform.

\sphinxlineitem{Return type}
\sphinxAtStartPar
({\hyperref[\detokenize{deepdrr.geo:deepdrr.geo.Transform}]{\sphinxcrossref{Transform}}})

\sphinxlineitem{Raises}
\sphinxAtStartPar
\sphinxstyleliteralstrong{\sphinxupquote{NotImplementedError}} \textendash{} if \_inv is None and method is not overriden.

\end{description}\end{quote}

\end{fulllineitems}


\end{fulllineitems}

\index{Vector (class in deepdrr.geo)@\spxentry{Vector}\spxextra{class in deepdrr.geo}}

\begin{fulllineitems}
\phantomsection\label{\detokenize{deepdrr.geo:deepdrr.geo.Vector}}
\pysigstartsignatures
\pysiglinewithargsret{\sphinxbfcode{\sphinxupquote{class\DUrole{w,w}{  }}}\sphinxcode{\sphinxupquote{deepdrr.geo.}}\sphinxbfcode{\sphinxupquote{Vector}}}{\sphinxparam{\DUrole{n,n}{data}\DUrole{p,p}{:}\DUrole{w,w}{  }\DUrole{n,n}{ndarray}}}{}
\pysigstopsignatures
\sphinxAtStartPar
Bases: {\hyperref[\detokenize{deepdrr.geo:deepdrr.geo.core.PointOrVector}]{\sphinxcrossref{\sphinxcode{\sphinxupquote{PointOrVector}}}}}, {\hyperref[\detokenize{deepdrr.geo:deepdrr.geo.core.HasDirection}]{\sphinxcrossref{\sphinxcode{\sphinxupquote{HasDirection}}}}}
\index{as\_point() (deepdrr.geo.Vector method)@\spxentry{as\_point()}\spxextra{deepdrr.geo.Vector method}}

\begin{fulllineitems}
\phantomsection\label{\detokenize{deepdrr.geo:deepdrr.geo.Vector.as_point}}
\pysigstartsignatures
\pysiglinewithargsret{\sphinxbfcode{\sphinxupquote{as\_point}}}{}{{ $\rightarrow$ {\hyperref[\detokenize{deepdrr.geo:deepdrr.geo.core.Point}]{\sphinxcrossref{Point}}}}}
\pysigstopsignatures
\sphinxAtStartPar
Gets the point with the same numerical representation as this vector.

\end{fulllineitems}

\index{cross() (deepdrr.geo.Vector method)@\spxentry{cross()}\spxextra{deepdrr.geo.Vector method}}

\begin{fulllineitems}
\phantomsection\label{\detokenize{deepdrr.geo:deepdrr.geo.Vector.cross}}
\pysigstartsignatures
\pysiglinewithargsret{\sphinxbfcode{\sphinxupquote{cross}}}{\sphinxparam{\DUrole{n,n}{other}\DUrole{p,p}{:}\DUrole{w,w}{  }\DUrole{n,n}{{\hyperref[\detokenize{deepdrr.geo:deepdrr.geo.core.Vector}]{\sphinxcrossref{Vector}}}}}}{{ $\rightarrow$ {\hyperref[\detokenize{deepdrr.geo:deepdrr.geo.core.Vector3D}]{\sphinxcrossref{Vector3D}}}}}
\pysigstopsignatures
\end{fulllineitems}

\index{data (deepdrr.geo.Vector attribute)@\spxentry{data}\spxextra{deepdrr.geo.Vector attribute}}

\begin{fulllineitems}
\phantomsection\label{\detokenize{deepdrr.geo:deepdrr.geo.Vector.data}}
\pysigstartsignatures
\pysigline{\sphinxbfcode{\sphinxupquote{data}}\sphinxbfcode{\sphinxupquote{\DUrole{p,p}{:}\DUrole{w,w}{  }ndarray}}}
\pysigstopsignatures
\end{fulllineitems}

\index{dot() (deepdrr.geo.Vector method)@\spxentry{dot()}\spxextra{deepdrr.geo.Vector method}}

\begin{fulllineitems}
\phantomsection\label{\detokenize{deepdrr.geo:deepdrr.geo.Vector.dot}}
\pysigstartsignatures
\pysiglinewithargsret{\sphinxbfcode{\sphinxupquote{dot}}}{\sphinxparam{\DUrole{n,n}{other}}}{{ $\rightarrow$ float}}
\pysigstopsignatures
\end{fulllineitems}

\index{from\_any() (deepdrr.geo.Vector class method)@\spxentry{from\_any()}\spxextra{deepdrr.geo.Vector class method}}

\begin{fulllineitems}
\phantomsection\label{\detokenize{deepdrr.geo:deepdrr.geo.Vector.from_any}}
\pysigstartsignatures
\pysiglinewithargsret{\sphinxbfcode{\sphinxupquote{classmethod\DUrole{w,w}{  }}}\sphinxbfcode{\sphinxupquote{from\_any}}}{\sphinxparam{\DUrole{n,n}{other}\DUrole{p,p}{:}\DUrole{w,w}{  }\DUrole{n,n}{ndarray\DUrole{w,w}{  }\DUrole{p,p}{|}\DUrole{w,w}{  }{\hyperref[\detokenize{deepdrr.geo:deepdrr.geo.core.Vector}]{\sphinxcrossref{Vector}}}}}}{}
\pysigstopsignatures
\sphinxAtStartPar
If other is not a Vector, make it one.

\end{fulllineitems}

\index{from\_array() (deepdrr.geo.Vector class method)@\spxentry{from\_array()}\spxextra{deepdrr.geo.Vector class method}}

\begin{fulllineitems}
\phantomsection\label{\detokenize{deepdrr.geo:deepdrr.geo.Vector.from_array}}
\pysigstartsignatures
\pysiglinewithargsret{\sphinxbfcode{\sphinxupquote{classmethod\DUrole{w,w}{  }}}\sphinxbfcode{\sphinxupquote{from\_array}}}{\sphinxparam{\DUrole{n,n}{v}\DUrole{p,p}{:}\DUrole{w,w}{  }\DUrole{n,n}{ndarray}}}{{ $\rightarrow$ T}}
\pysigstopsignatures
\sphinxAtStartPar
Create a homogeneous object from its non\sphinxhyphen{}homogeous representation as an array.

\end{fulllineitems}

\index{get\_direction() (deepdrr.geo.Vector method)@\spxentry{get\_direction()}\spxextra{deepdrr.geo.Vector method}}

\begin{fulllineitems}
\phantomsection\label{\detokenize{deepdrr.geo:deepdrr.geo.Vector.get_direction}}
\pysigstartsignatures
\pysiglinewithargsret{\sphinxbfcode{\sphinxupquote{get\_direction}}}{}{{ $\rightarrow$ Self}}
\pysigstopsignatures
\sphinxAtStartPar
Gets the vector with the same numerical representation as this vector.

\end{fulllineitems}

\index{hat() (deepdrr.geo.Vector method)@\spxentry{hat()}\spxextra{deepdrr.geo.Vector method}}

\begin{fulllineitems}
\phantomsection\label{\detokenize{deepdrr.geo:deepdrr.geo.Vector.hat}}
\pysigstartsignatures
\pysiglinewithargsret{\sphinxbfcode{\sphinxupquote{hat}}}{}{{ $\rightarrow$ Self}}
\pysigstopsignatures
\end{fulllineitems}

\index{normalized() (deepdrr.geo.Vector method)@\spxentry{normalized()}\spxextra{deepdrr.geo.Vector method}}

\begin{fulllineitems}
\phantomsection\label{\detokenize{deepdrr.geo:deepdrr.geo.Vector.normalized}}
\pysigstartsignatures
\pysiglinewithargsret{\sphinxbfcode{\sphinxupquote{normalized}}}{}{{ $\rightarrow$ Self}}
\pysigstopsignatures
\end{fulllineitems}


\end{fulllineitems}

\index{Vector2D (class in deepdrr.geo)@\spxentry{Vector2D}\spxextra{class in deepdrr.geo}}

\begin{fulllineitems}
\phantomsection\label{\detokenize{deepdrr.geo:deepdrr.geo.Vector2D}}
\pysigstartsignatures
\pysiglinewithargsret{\sphinxbfcode{\sphinxupquote{class\DUrole{w,w}{  }}}\sphinxcode{\sphinxupquote{deepdrr.geo.}}\sphinxbfcode{\sphinxupquote{Vector2D}}}{\sphinxparam{\DUrole{n,n}{data}\DUrole{p,p}{:}\DUrole{w,w}{  }\DUrole{n,n}{ndarray}}}{}
\pysigstopsignatures
\sphinxAtStartPar
Bases: {\hyperref[\detokenize{deepdrr.geo:deepdrr.geo.core.Vector}]{\sphinxcrossref{\sphinxcode{\sphinxupquote{Vector}}}}}

\sphinxAtStartPar
Homogeneous vector in 2D, represented as an array with {[}x, y, 0{]}
\index{data (deepdrr.geo.Vector2D attribute)@\spxentry{data}\spxextra{deepdrr.geo.Vector2D attribute}}

\begin{fulllineitems}
\phantomsection\label{\detokenize{deepdrr.geo:deepdrr.geo.Vector2D.data}}
\pysigstartsignatures
\pysigline{\sphinxbfcode{\sphinxupquote{data}}\sphinxbfcode{\sphinxupquote{\DUrole{p,p}{:}\DUrole{w,w}{  }ndarray}}}
\pysigstopsignatures
\end{fulllineitems}

\index{dim (deepdrr.geo.Vector2D attribute)@\spxentry{dim}\spxextra{deepdrr.geo.Vector2D attribute}}

\begin{fulllineitems}
\phantomsection\label{\detokenize{deepdrr.geo:deepdrr.geo.Vector2D.dim}}
\pysigstartsignatures
\pysigline{\sphinxbfcode{\sphinxupquote{dim}}\sphinxbfcode{\sphinxupquote{\DUrole{w,w}{  }\DUrole{p,p}{=}\DUrole{w,w}{  }2}}}
\pysigstopsignatures
\end{fulllineitems}

\index{perpendicular() (deepdrr.geo.Vector2D method)@\spxentry{perpendicular()}\spxextra{deepdrr.geo.Vector2D method}}

\begin{fulllineitems}
\phantomsection\label{\detokenize{deepdrr.geo:deepdrr.geo.Vector2D.perpendicular}}
\pysigstartsignatures
\pysiglinewithargsret{\sphinxbfcode{\sphinxupquote{perpendicular}}}{\sphinxparam{\DUrole{n,n}{random}\DUrole{p,p}{:}\DUrole{w,w}{  }\DUrole{n,n}{bool}\DUrole{w,w}{  }\DUrole{o,o}{=}\DUrole{w,w}{  }\DUrole{default_value}{False}}}{{ $\rightarrow$ {\hyperref[\detokenize{deepdrr.geo:deepdrr.geo.core.Vector2D}]{\sphinxcrossref{Vector2D}}}}}
\pysigstopsignatures
\sphinxAtStartPar
Find an arbitrary perpendicular vector to self.
\begin{quote}\begin{description}
\sphinxlineitem{Parameters}
\sphinxAtStartPar
\sphinxstyleliteralstrong{\sphinxupquote{random}} \textendash{} Whether to randomize the vector’s direction in
the perpendicular plane, drawing from {[}0, 2pi).
Defaults to False.

\sphinxlineitem{Returns}
\sphinxAtStartPar
\begin{description}
\sphinxlineitem{A vector in 3D space, perpendicular}
\sphinxAtStartPar
to the original.

\end{description}


\sphinxlineitem{Return type}
\sphinxAtStartPar
{\hyperref[\detokenize{deepdrr.geo:deepdrr.geo.Vector3D}]{\sphinxcrossref{Vector3D}}}

\end{description}\end{quote}

\end{fulllineitems}


\end{fulllineitems}

\index{Vector3D (class in deepdrr.geo)@\spxentry{Vector3D}\spxextra{class in deepdrr.geo}}

\begin{fulllineitems}
\phantomsection\label{\detokenize{deepdrr.geo:deepdrr.geo.Vector3D}}
\pysigstartsignatures
\pysiglinewithargsret{\sphinxbfcode{\sphinxupquote{class\DUrole{w,w}{  }}}\sphinxcode{\sphinxupquote{deepdrr.geo.}}\sphinxbfcode{\sphinxupquote{Vector3D}}}{\sphinxparam{\DUrole{n,n}{data}\DUrole{p,p}{:}\DUrole{w,w}{  }\DUrole{n,n}{ndarray}}}{}
\pysigstopsignatures
\sphinxAtStartPar
Bases: {\hyperref[\detokenize{deepdrr.geo:deepdrr.geo.core.Vector}]{\sphinxcrossref{\sphinxcode{\sphinxupquote{Vector}}}}}

\sphinxAtStartPar
Homogeneous vector in 3D, represented as an array with {[}x, y, z, 0{]}.

\sphinxAtStartPar
A 3d vector still cannot be projected by a camera, because it doesn’t have a location.
\index{as\_plane() (deepdrr.geo.Vector3D method)@\spxentry{as\_plane()}\spxextra{deepdrr.geo.Vector3D method}}

\begin{fulllineitems}
\phantomsection\label{\detokenize{deepdrr.geo:deepdrr.geo.Vector3D.as_plane}}
\pysigstartsignatures
\pysiglinewithargsret{\sphinxbfcode{\sphinxupquote{as\_plane}}}{}{{ $\rightarrow$ {\hyperref[\detokenize{deepdrr.geo:deepdrr.geo.Plane}]{\sphinxcrossref{Plane}}}}}
\pysigstopsignatures
\sphinxAtStartPar
Get the plane through the origin with this vector as its normal.

\end{fulllineitems}

\index{data (deepdrr.geo.Vector3D attribute)@\spxentry{data}\spxextra{deepdrr.geo.Vector3D attribute}}

\begin{fulllineitems}
\phantomsection\label{\detokenize{deepdrr.geo:deepdrr.geo.Vector3D.data}}
\pysigstartsignatures
\pysigline{\sphinxbfcode{\sphinxupquote{data}}\sphinxbfcode{\sphinxupquote{\DUrole{p,p}{:}\DUrole{w,w}{  }ndarray}}}
\pysigstopsignatures
\end{fulllineitems}

\index{dim (deepdrr.geo.Vector3D attribute)@\spxentry{dim}\spxextra{deepdrr.geo.Vector3D attribute}}

\begin{fulllineitems}
\phantomsection\label{\detokenize{deepdrr.geo:deepdrr.geo.Vector3D.dim}}
\pysigstartsignatures
\pysigline{\sphinxbfcode{\sphinxupquote{dim}}\sphinxbfcode{\sphinxupquote{\DUrole{w,w}{  }\DUrole{p,p}{=}\DUrole{w,w}{  }3}}}
\pysigstopsignatures
\end{fulllineitems}

\index{perpendicular() (deepdrr.geo.Vector3D method)@\spxentry{perpendicular()}\spxextra{deepdrr.geo.Vector3D method}}

\begin{fulllineitems}
\phantomsection\label{\detokenize{deepdrr.geo:deepdrr.geo.Vector3D.perpendicular}}
\pysigstartsignatures
\pysiglinewithargsret{\sphinxbfcode{\sphinxupquote{perpendicular}}}{\sphinxparam{\DUrole{n,n}{random}\DUrole{p,p}{:}\DUrole{w,w}{  }\DUrole{n,n}{bool}\DUrole{w,w}{  }\DUrole{o,o}{=}\DUrole{w,w}{  }\DUrole{default_value}{False}}}{{ $\rightarrow$ {\hyperref[\detokenize{deepdrr.geo:deepdrr.geo.core.Vector3D}]{\sphinxcrossref{Vector3D}}}}}
\pysigstopsignatures
\sphinxAtStartPar
Find an arbitrary perpendicular vector to self.
\begin{quote}\begin{description}
\sphinxlineitem{Parameters}
\sphinxAtStartPar
\sphinxstyleliteralstrong{\sphinxupquote{random}} \textendash{} Whether to randomize the vector’s direction in
the perpendicular plane, drawing from {[}0, 2pi).
Defaults to False.

\sphinxlineitem{Returns}
\sphinxAtStartPar
\begin{description}
\sphinxlineitem{A vector in 3D space, perpendicular}
\sphinxAtStartPar
to the original.

\end{description}


\sphinxlineitem{Return type}
\sphinxAtStartPar
{\hyperref[\detokenize{deepdrr.geo:deepdrr.geo.Vector3D}]{\sphinxcrossref{Vector3D}}}

\end{description}\end{quote}

\end{fulllineitems}

\index{rotate() (deepdrr.geo.Vector3D method)@\spxentry{rotate()}\spxextra{deepdrr.geo.Vector3D method}}

\begin{fulllineitems}
\phantomsection\label{\detokenize{deepdrr.geo:deepdrr.geo.Vector3D.rotate}}
\pysigstartsignatures
\pysiglinewithargsret{\sphinxbfcode{\sphinxupquote{rotate}}}{\sphinxparam{\DUrole{n,n}{n}\DUrole{p,p}{:}\DUrole{w,w}{  }\DUrole{n,n}{{\hyperref[\detokenize{deepdrr.geo:deepdrr.geo.core.Vector3D}]{\sphinxcrossref{Vector3D}}}}}\sphinxparamcomma \sphinxparam{\DUrole{n,n}{theta}\DUrole{p,p}{:}\DUrole{w,w}{  }\DUrole{n,n}{float\DUrole{w,w}{  }\DUrole{p,p}{|}\DUrole{w,w}{  }None}\DUrole{w,w}{  }\DUrole{o,o}{=}\DUrole{w,w}{  }\DUrole{default_value}{None}}}{{ $\rightarrow$ {\hyperref[\detokenize{deepdrr.geo:deepdrr.geo.core.Vector3D}]{\sphinxcrossref{Vector3D}}}}}
\pysigstopsignatures
\sphinxAtStartPar
Rotate self by the given vector.
\begin{quote}\begin{description}
\sphinxlineitem{Parameters}\begin{itemize}
\item {} 
\sphinxAtStartPar
\sphinxstyleliteralstrong{\sphinxupquote{n}} ({\hyperref[\detokenize{deepdrr.geo:deepdrr.geo.Vector}]{\sphinxcrossref{\sphinxstyleliteralemphasis{\sphinxupquote{Vector}}}}}) \textendash{} the axis of rotation. If theta is None, the magnitude of this vector is
used. Otherwise, it is ignored.

\item {} 
\sphinxAtStartPar
\sphinxstyleliteralstrong{\sphinxupquote{theta}} (\sphinxstyleliteralemphasis{\sphinxupquote{float}}\sphinxstyleliteralemphasis{\sphinxupquote{, }}\sphinxstyleliteralemphasis{\sphinxupquote{optional}}) \textendash{} the angle of rotation. Defaults to None.

\end{itemize}

\sphinxlineitem{Returns}
\sphinxAtStartPar
the rotated vector.

\sphinxlineitem{Return type}
\sphinxAtStartPar
{\hyperref[\detokenize{deepdrr.geo:deepdrr.geo.Vector}]{\sphinxcrossref{Vector}}}

\end{description}\end{quote}

\end{fulllineitems}

\index{rotvec\_to() (deepdrr.geo.Vector3D method)@\spxentry{rotvec\_to()}\spxextra{deepdrr.geo.Vector3D method}}

\begin{fulllineitems}
\phantomsection\label{\detokenize{deepdrr.geo:deepdrr.geo.Vector3D.rotvec_to}}
\pysigstartsignatures
\pysiglinewithargsret{\sphinxbfcode{\sphinxupquote{rotvec\_to}}}{\sphinxparam{\DUrole{n,n}{other}\DUrole{p,p}{:}\DUrole{w,w}{  }\DUrole{n,n}{{\hyperref[\detokenize{deepdrr.geo:deepdrr.geo.core.Vector3D}]{\sphinxcrossref{Vector3D}}}}}}{{ $\rightarrow$ {\hyperref[\detokenize{deepdrr.geo:deepdrr.geo.core.Vector3D}]{\sphinxcrossref{Vector3D}}}}}
\pysigstopsignatures
\sphinxAtStartPar
Get the rotvec that rotates self to other.

\end{fulllineitems}


\end{fulllineitems}

\index{f() (in module deepdrr.geo)@\spxentry{f()}\spxextra{in module deepdrr.geo}}

\begin{fulllineitems}
\phantomsection\label{\detokenize{deepdrr.geo:deepdrr.geo.f}}
\pysigstartsignatures
\pysiglinewithargsret{\sphinxcode{\sphinxupquote{deepdrr.geo.}}\sphinxbfcode{\sphinxupquote{f}}}{\sphinxparam{\DUrole{o,o}{*}\DUrole{n,n}{args}}}{{ $\rightarrow$ {\hyperref[\detokenize{deepdrr.geo:deepdrr.geo.core.FrameTransform}]{\sphinxcrossref{FrameTransform}}}}}
\pysigstopsignatures
\sphinxAtStartPar
Convenience function for creating a 3D frame transform.

\sphinxAtStartPar
The output depends on how the function is called:
frame\_transform() \sphinxhyphen{}\textgreater{} 3D identity transform
frame\_transform(None) \sphinxhyphen{}\textgreater{} 3D identity transform
frame\_transform(scalar) \sphinxhyphen{}\textgreater{} FrameTransform.from\_scaling(scalar)
frame\_transform(ft: FrameTransform) \sphinxhyphen{}\textgreater{} ft
frame\_transform(data: np.ndarray{[}4,4{]}) \sphinxhyphen{}\textgreater{} FrameTransform(data)
frame\_transform(R: Rotation | np.ndarray{[}3,3{]}) \sphinxhyphen{}\textgreater{} FrameTransform.from\_rt(R)
frame\_transform(t: Point | np.ndarray{[}3{]}) \sphinxhyphen{}\textgreater{} FrameTransform.from\_translation(t)
frame\_transform((R, t)) \sphinxhyphen{}\textgreater{} FrameTransform.from\_rt(R, t)
frame\_transform(R, t) \sphinxhyphen{}\textgreater{} FrameTransform.from\_rt(R, t)
frame\_transform({[}a00, a10, a20, a01, a11, a21, a02, a12, a22, a03, a13, a23{]}) \sphinxhyphen{}\textgreater{} FrameTransform({[}
\begin{quote}

\sphinxAtStartPar
{[}a00, a01, a02, a03{]},
{[}a10, a11, a12, a13{]},
{[}a20, a21, a22, a23{]},
{[}0, 0, 0, 1{]}{]}
\end{quote}

\sphinxAtStartPar
)

\sphinxAtStartPar
R maybe be given as a (3,3) matrix or as a 9\sphinxhyphen{}vector. If provided as a 9\sphinxhyphen{}vector, column major order is assumed,
such that (a11, a21, a31, a12, a22, a32, a13, a23, a33) is the rotation matrix.
{[}{[}a11, a12, a13{]},
\begin{quote}

\sphinxAtStartPar
{[}a21, a22, a23{]},
{[}a31, a32, a33{]}{]}
\end{quote}

\sphinxAtStartPar
{[}R | t{]} may be given as a (12,) array\sphinxhyphen{}like, where the first 9 elements are the rotation in column major order,  and the last 3 are the translation.

\sphinxAtStartPar
If a string provided, it is converted to an array with whitespace separator.
\begin{quote}\begin{description}
\sphinxlineitem{Returns}
\sphinxAtStartPar
{[}description{]}

\sphinxlineitem{Return type}
\sphinxAtStartPar
{\hyperref[\detokenize{deepdrr.geo:deepdrr.geo.FrameTransform}]{\sphinxcrossref{FrameTransform}}}

\end{description}\end{quote}

\end{fulllineitems}

\index{frame\_transform() (in module deepdrr.geo)@\spxentry{frame\_transform()}\spxextra{in module deepdrr.geo}}

\begin{fulllineitems}
\phantomsection\label{\detokenize{deepdrr.geo:deepdrr.geo.frame_transform}}
\pysigstartsignatures
\pysiglinewithargsret{\sphinxcode{\sphinxupquote{deepdrr.geo.}}\sphinxbfcode{\sphinxupquote{frame\_transform}}}{\sphinxparam{\DUrole{o,o}{*}\DUrole{n,n}{args}}}{{ $\rightarrow$ {\hyperref[\detokenize{deepdrr.geo:deepdrr.geo.core.FrameTransform}]{\sphinxcrossref{FrameTransform}}}}}
\pysigstopsignatures
\sphinxAtStartPar
Convenience function for creating a 3D frame transform.

\sphinxAtStartPar
The output depends on how the function is called:
frame\_transform() \sphinxhyphen{}\textgreater{} 3D identity transform
frame\_transform(None) \sphinxhyphen{}\textgreater{} 3D identity transform
frame\_transform(scalar) \sphinxhyphen{}\textgreater{} FrameTransform.from\_scaling(scalar)
frame\_transform(ft: FrameTransform) \sphinxhyphen{}\textgreater{} ft
frame\_transform(data: np.ndarray{[}4,4{]}) \sphinxhyphen{}\textgreater{} FrameTransform(data)
frame\_transform(R: Rotation | np.ndarray{[}3,3{]}) \sphinxhyphen{}\textgreater{} FrameTransform.from\_rt(R)
frame\_transform(t: Point | np.ndarray{[}3{]}) \sphinxhyphen{}\textgreater{} FrameTransform.from\_translation(t)
frame\_transform((R, t)) \sphinxhyphen{}\textgreater{} FrameTransform.from\_rt(R, t)
frame\_transform(R, t) \sphinxhyphen{}\textgreater{} FrameTransform.from\_rt(R, t)
frame\_transform({[}a00, a10, a20, a01, a11, a21, a02, a12, a22, a03, a13, a23{]}) \sphinxhyphen{}\textgreater{} FrameTransform({[}
\begin{quote}

\sphinxAtStartPar
{[}a00, a01, a02, a03{]},
{[}a10, a11, a12, a13{]},
{[}a20, a21, a22, a23{]},
{[}0, 0, 0, 1{]}{]}
\end{quote}

\sphinxAtStartPar
)

\sphinxAtStartPar
R maybe be given as a (3,3) matrix or as a 9\sphinxhyphen{}vector. If provided as a 9\sphinxhyphen{}vector, column major order is assumed,
such that (a11, a21, a31, a12, a22, a32, a13, a23, a33) is the rotation matrix.
{[}{[}a11, a12, a13{]},
\begin{quote}

\sphinxAtStartPar
{[}a21, a22, a23{]},
{[}a31, a32, a33{]}{]}
\end{quote}

\sphinxAtStartPar
{[}R | t{]} may be given as a (12,) array\sphinxhyphen{}like, where the first 9 elements are the rotation in column major order,  and the last 3 are the translation.

\sphinxAtStartPar
If a string provided, it is converted to an array with whitespace separator.
\begin{quote}\begin{description}
\sphinxlineitem{Returns}
\sphinxAtStartPar
{[}description{]}

\sphinxlineitem{Return type}
\sphinxAtStartPar
{\hyperref[\detokenize{deepdrr.geo:deepdrr.geo.FrameTransform}]{\sphinxcrossref{FrameTransform}}}

\end{description}\end{quote}

\end{fulllineitems}

\index{get\_data() (in module deepdrr.geo)@\spxentry{get\_data()}\spxextra{in module deepdrr.geo}}

\begin{fulllineitems}
\phantomsection\label{\detokenize{deepdrr.geo:deepdrr.geo.get_data}}
\pysigstartsignatures
\pysiglinewithargsret{\sphinxcode{\sphinxupquote{deepdrr.geo.}}\sphinxbfcode{\sphinxupquote{get\_data}}}{\sphinxparam{\DUrole{n,n}{x}\DUrole{p,p}{:}\DUrole{w,w}{  }\DUrole{n,n}{{\hyperref[\detokenize{deepdrr.geo:deepdrr.geo.core.HomogeneousObject}]{\sphinxcrossref{HomogeneousObject}}}\DUrole{w,w}{  }\DUrole{p,p}{|}\DUrole{w,w}{  }List\DUrole{p,p}{{[}}{\hyperref[\detokenize{deepdrr.geo:deepdrr.geo.core.HomogeneousObject}]{\sphinxcrossref{HomogeneousObject}}}\DUrole{p,p}{{]}}}}}{{ $\rightarrow$ ndarray}}
\pysigstopsignatures
\end{fulllineitems}

\index{l() (in module deepdrr.geo)@\spxentry{l()}\spxextra{in module deepdrr.geo}}

\begin{fulllineitems}
\phantomsection\label{\detokenize{deepdrr.geo:deepdrr.geo.l}}
\pysigstartsignatures
\pysiglinewithargsret{\sphinxcode{\sphinxupquote{deepdrr.geo.}}\sphinxbfcode{\sphinxupquote{l}}}{\sphinxparam{\DUrole{o,o}{*}\DUrole{n,n}{args}}}{}
\pysigstopsignatures
\end{fulllineitems}

\index{line() (in module deepdrr.geo)@\spxentry{line()}\spxextra{in module deepdrr.geo}}

\begin{fulllineitems}
\phantomsection\label{\detokenize{deepdrr.geo:deepdrr.geo.line}}
\pysigstartsignatures
\pysiglinewithargsret{\sphinxcode{\sphinxupquote{deepdrr.geo.}}\sphinxbfcode{\sphinxupquote{line}}}{\sphinxparam{\DUrole{o,o}{*}\DUrole{n,n}{args}}}{}
\pysigstopsignatures
\sphinxAtStartPar
The preferred method for creating a line.

\sphinxAtStartPar
Can create a line using one of the following methods:
\sphinxhyphen{} Pass the coordinates as separate arguments. For instance, \sphinxtitleref{line(1, 2, 3)} returns the 2D homogeneous line \sphinxtitleref{1x + 2y + 3 = 0}.
\sphinxhyphen{} Pass a numpy array with the homogeneous coordinates (NOTE THE DIFFERENCE WITH \sphinxtitleref{point} and \sphinxtitleref{vector}).
\sphinxhyphen{} Pass a Line2D or Line3D instance, in which case \sphinxtitleref{line()} is a no\sphinxhyphen{}op.
\sphinxhyphen{} Pass two points of the same dimension, in which case \sphinxtitleref{line()} returns the line through the points.
\sphinxhyphen{} Pass two planes, in which case \sphinxtitleref{line()} returns the line of intersection of the planes.

\end{fulllineitems}

\index{p() (in module deepdrr.geo)@\spxentry{p()}\spxextra{in module deepdrr.geo}}

\begin{fulllineitems}
\phantomsection\label{\detokenize{deepdrr.geo:deepdrr.geo.p}}
\pysigstartsignatures
\pysiglinewithargsret{\sphinxcode{\sphinxupquote{deepdrr.geo.}}\sphinxbfcode{\sphinxupquote{p}}}{\sphinxparam{\DUrole{o,o}{*}\DUrole{n,n}{args}}}{}
\pysigstopsignatures
\end{fulllineitems}

\index{pl() (in module deepdrr.geo)@\spxentry{pl()}\spxextra{in module deepdrr.geo}}

\begin{fulllineitems}
\phantomsection\label{\detokenize{deepdrr.geo:deepdrr.geo.pl}}
\pysigstartsignatures
\pysiglinewithargsret{\sphinxcode{\sphinxupquote{deepdrr.geo.}}\sphinxbfcode{\sphinxupquote{pl}}}{\sphinxparam{\DUrole{o,o}{*}\DUrole{n,n}{args}}}{}
\pysigstopsignatures
\end{fulllineitems}

\index{plane() (in module deepdrr.geo)@\spxentry{plane()}\spxextra{in module deepdrr.geo}}

\begin{fulllineitems}
\phantomsection\label{\detokenize{deepdrr.geo:deepdrr.geo.plane}}
\pysigstartsignatures
\pysiglinewithargsret{\sphinxcode{\sphinxupquote{deepdrr.geo.}}\sphinxbfcode{\sphinxupquote{plane}}}{\sphinxparam{\DUrole{o,o}{*}\DUrole{n,n}{args}}}{}
\pysigstopsignatures
\sphinxAtStartPar
The preferred method for creating a plane.

\sphinxAtStartPar
Can create a plane using one of the following methods:
\sphinxhyphen{} Pass the coordinates as separate arguments. For instance, \sphinxtitleref{plane(1, 2, 3, 4)} returns the 2D homogeneous plane \sphinxtitleref{1x + 2y + 3z + 4 = 0}.
\sphinxhyphen{} Pass a numpy array with the homogeneous coordinates.
\sphinxhyphen{} Pass a Plane instance, in which case \sphinxtitleref{plane()} is a no\sphinxhyphen{}op.
\sphinxhyphen{} Pass a Point3D and Vector3D instance, in which case \sphinxtitleref{plane(p, n)} returns the plane corresponding to
\sphinxhyphen{} Pass a ray, which defines r, n as above.

\end{fulllineitems}

\index{point() (in module deepdrr.geo)@\spxentry{point()}\spxextra{in module deepdrr.geo}}

\begin{fulllineitems}
\phantomsection\label{\detokenize{deepdrr.geo:deepdrr.geo.point}}
\pysigstartsignatures
\pysiglinewithargsret{\sphinxcode{\sphinxupquote{deepdrr.geo.}}\sphinxbfcode{\sphinxupquote{point}}}{\sphinxparam{\DUrole{o,o}{*}\DUrole{n,n}{args}}}{}
\pysigstopsignatures
\sphinxAtStartPar
The preferred method for creating a point.

\sphinxAtStartPar
There are three ways to create a point using \sphinxtitleref{point()}.
\sphinxhyphen{} Pass the coordinates as separate arguments. For instance, \sphinxtitleref{point(0, 0)} returns the 2D homogeneous point for the origin \sphinxtitleref{Point2D({[}0, 0, 1{]})}.
\sphinxhyphen{} Pass a numpy array containing the non\sphinxhyphen{}homogeneous representation of the point. For example \sphinxtitleref{point(np.ndarray({[}0, 1, 2{]}))} is the 3D homogeneous point \sphinxtitleref{Point3D({[}0, 1, 2, 1{]})}.
\sphinxhyphen{} Pass a Point2D or Point3D instance, in which case \sphinxtitleref{point()} just returns the first argument.

\sphinxAtStartPar
\sphinxtitleref{point()} shoud NOT be given a numpy array containing the homogeneous data. In this case, use the \sphinxtitleref{Point2D} and \sphinxtitleref{Point3D} constructors directly.
\begin{quote}\begin{description}
\sphinxlineitem{Raises}
\sphinxAtStartPar
\sphinxstyleliteralstrong{\sphinxupquote{ValueError}} \textendash{} if arguments cannot be parsed as data for a point.

\sphinxlineitem{Returns}
\sphinxAtStartPar
Point2D or Point3D.

\sphinxlineitem{Return type}
\sphinxAtStartPar
Union{[}{\hyperref[\detokenize{deepdrr.geo:deepdrr.geo.Point2D}]{\sphinxcrossref{Point2D}}}, {\hyperref[\detokenize{deepdrr.geo:deepdrr.geo.Point3D}]{\sphinxcrossref{Point3D}}}{]}

\end{description}\end{quote}

\end{fulllineitems}

\index{ray() (in module deepdrr.geo)@\spxentry{ray()}\spxextra{in module deepdrr.geo}}

\begin{fulllineitems}
\phantomsection\label{\detokenize{deepdrr.geo:deepdrr.geo.ray}}
\pysigstartsignatures
\pysiglinewithargsret{\sphinxcode{\sphinxupquote{deepdrr.geo.}}\sphinxbfcode{\sphinxupquote{ray}}}{\sphinxparam{\DUrole{o,o}{*}\DUrole{n,n}{args}}}{}
\pysigstopsignatures
\sphinxAtStartPar
More flexible method for creating a ray.

\end{fulllineitems}

\index{segment() (in module deepdrr.geo)@\spxentry{segment()}\spxextra{in module deepdrr.geo}}

\begin{fulllineitems}
\phantomsection\label{\detokenize{deepdrr.geo:deepdrr.geo.segment}}
\pysigstartsignatures
\pysiglinewithargsret{\sphinxcode{\sphinxupquote{deepdrr.geo.}}\sphinxbfcode{\sphinxupquote{segment}}}{\sphinxparam{\DUrole{o,o}{*}\DUrole{n,n}{args}}}{}
\pysigstopsignatures
\sphinxAtStartPar
More flexible method for creating a segment.

\end{fulllineitems}

\index{spherical\_uniform() (in module deepdrr.geo)@\spxentry{spherical\_uniform()}\spxextra{in module deepdrr.geo}}

\begin{fulllineitems}
\phantomsection\label{\detokenize{deepdrr.geo:deepdrr.geo.spherical_uniform}}
\pysigstartsignatures
\pysiglinewithargsret{\sphinxcode{\sphinxupquote{deepdrr.geo.}}\sphinxbfcode{\sphinxupquote{spherical\_uniform}}}{\sphinxparam{\DUrole{n,n}{center=Vector3D({[}0. 0. 1. 0.{]})}}\sphinxparamcomma \sphinxparam{\DUrole{n,n}{d\_phi=3.141592653589793}}\sphinxparamcomma \sphinxparam{\DUrole{n,n}{n=None}}}{}
\pysigstopsignatures
\sphinxAtStartPar
Sample unit vectors on the surface of the sphere within \sphinxtitleref{d\_phi} radians of \sphinxtitleref{v}.

\end{fulllineitems}

\index{v() (in module deepdrr.geo)@\spxentry{v()}\spxextra{in module deepdrr.geo}}

\begin{fulllineitems}
\phantomsection\label{\detokenize{deepdrr.geo:deepdrr.geo.v}}
\pysigstartsignatures
\pysiglinewithargsret{\sphinxcode{\sphinxupquote{deepdrr.geo.}}\sphinxbfcode{\sphinxupquote{v}}}{\sphinxparam{\DUrole{o,o}{*}\DUrole{n,n}{args}}}{}
\pysigstopsignatures
\end{fulllineitems}

\index{vector() (in module deepdrr.geo)@\spxentry{vector()}\spxextra{in module deepdrr.geo}}

\begin{fulllineitems}
\phantomsection\label{\detokenize{deepdrr.geo:deepdrr.geo.vector}}
\pysigstartsignatures
\pysiglinewithargsret{\sphinxcode{\sphinxupquote{deepdrr.geo.}}\sphinxbfcode{\sphinxupquote{vector}}}{\sphinxparam{\DUrole{o,o}{*}\DUrole{n,n}{args}}}{}
\pysigstopsignatures
\sphinxAtStartPar
The preferred method for creating a vector.

\sphinxAtStartPar
There are three ways to create a point using \sphinxtitleref{vector()}.
\begin{itemize}
\item {} 
\sphinxAtStartPar
Pass the coordinates as separate arguments. For instance, \sphinxtitleref{vector(0, 0)} returns the 2D homogeneous vector \sphinxtitleref{Vector2D({[}0, 0, 0{]})}.

\item {} 
\sphinxAtStartPar
Pass a numpy array containing the non\sphinxhyphen{}homogeneous representation of the vector.
For example \sphinxtitleref{vector(np.ndarray({[}0, 1, 2{]}))} is the 3D homogeneous veector \sphinxtitleref{Vector3D({[}0, 1, 2, 0{]})}.

\item {} 
\sphinxAtStartPar
Pass a Vector2D or Vector3D instance, in which case \sphinxtitleref{vector()} just returns the first argument.

\end{itemize}

\sphinxAtStartPar
\sphinxtitleref{point()} should NOT be given a numpy array containing the homogeneous data. In this case, use the \sphinxtitleref{Vector2D} and \sphinxtitleref{Vector3D} constructors directly.
\begin{quote}\begin{description}
\sphinxlineitem{Raises}
\sphinxAtStartPar
\sphinxstyleliteralstrong{\sphinxupquote{ValueError}} \textendash{} if arguments cannot be parsed as data for a point.

\sphinxlineitem{Returns}
\sphinxAtStartPar
Point2D or Point3D.

\sphinxlineitem{Return type}
\sphinxAtStartPar
Union{[}{\hyperref[\detokenize{deepdrr.geo:deepdrr.geo.Point2D}]{\sphinxcrossref{Point2D}}}, {\hyperref[\detokenize{deepdrr.geo:deepdrr.geo.Point3D}]{\sphinxcrossref{Point3D}}}{]}

\end{description}\end{quote}

\end{fulllineitems}


\sphinxstepscope


\section{deepdrr.instruments package}
\label{\detokenize{deepdrr.instruments:deepdrr-instruments-package}}\label{\detokenize{deepdrr.instruments::doc}}

\subsection{deepdrr.instruments.base}
\label{\detokenize{deepdrr.instruments:module-deepdrr.instruments.base}}\label{\detokenize{deepdrr.instruments:deepdrr-instruments-base}}\index{module@\spxentry{module}!deepdrr.instruments.base@\spxentry{deepdrr.instruments.base}}\index{deepdrr.instruments.base@\spxentry{deepdrr.instruments.base}!module@\spxentry{module}}\index{Instrument (class in deepdrr.instruments.base)@\spxentry{Instrument}\spxextra{class in deepdrr.instruments.base}}

\begin{fulllineitems}
\phantomsection\label{\detokenize{deepdrr.instruments:deepdrr.instruments.base.Instrument}}
\pysigstartsignatures
\pysiglinewithargsret{\sphinxbfcode{\sphinxupquote{class\DUrole{w,w}{  }}}\sphinxcode{\sphinxupquote{deepdrr.instruments.base.}}\sphinxbfcode{\sphinxupquote{Instrument}}}{\sphinxparam{\DUrole{n,n}{density}\DUrole{p,p}{:}\DUrole{w,w}{  }\DUrole{n,n}{float}\DUrole{w,w}{  }\DUrole{o,o}{=}\DUrole{w,w}{  }\DUrole{default_value}{0.1}}\sphinxparamcomma \sphinxparam{\DUrole{n,n}{world\_from\_anatomical}\DUrole{p,p}{:}\DUrole{w,w}{  }\DUrole{n,n}{{\hyperref[\detokenize{deepdrr.geo:deepdrr.geo.core.FrameTransform}]{\sphinxcrossref{FrameTransform}}}\DUrole{w,w}{  }\DUrole{p,p}{|}\DUrole{w,w}{  }None}\DUrole{w,w}{  }\DUrole{o,o}{=}\DUrole{w,w}{  }\DUrole{default_value}{None}}\sphinxparamcomma \sphinxparam{\DUrole{n,n}{densities}\DUrole{p,p}{:}\DUrole{w,w}{  }\DUrole{n,n}{Dict\DUrole{p,p}{{[}}str\DUrole{p,p}{,}\DUrole{w,w}{  }float\DUrole{p,p}{{]}}}\DUrole{w,w}{  }\DUrole{o,o}{=}\DUrole{w,w}{  }\DUrole{default_value}{\{\}}}}{}
\pysigstopsignatures
\sphinxAtStartPar
Bases: {\hyperref[\detokenize{deepdrr.vol:deepdrr.vol.volume.Volume}]{\sphinxcrossref{\sphinxcode{\sphinxupquote{Volume}}}}}, \sphinxcode{\sphinxupquote{ABC}}

\sphinxAtStartPar
A class for representing instruments based on voxelized surface models.

\sphinxAtStartPar
In the DeepDRR\_DATA directory, place an STL file for each material you want to use, inside a
directory determined by the class name. For example, if you have a
class called \sphinxtitleref{MyTool} with steel and plastic components, place the STL files \sphinxtitleref{steel.stl} and
\sphinxtitleref{plastic.stl} in \sphinxtitleref{DeepDRR\_DATA/instruments/MyTool}.

\sphinxAtStartPar
TODO: multiple components of the same material.

\sphinxAtStartPar
DeepDRR\_DATA
└── instruments
\begin{quote}

\sphinxAtStartPar
├── ToolClassName
│   ├── material\_1.stl
│   │── material\_2.stl
│   └── material\_3.stl
└── ToolClassName2
\begin{quote}

\sphinxAtStartPar
├── material\_1.stl
│── material\_2.stl
└── material\_3.stl
\end{quote}
\end{quote}
\index{NUM\_POINTS (deepdrr.instruments.base.Instrument attribute)@\spxentry{NUM\_POINTS}\spxextra{deepdrr.instruments.base.Instrument attribute}}

\begin{fulllineitems}
\phantomsection\label{\detokenize{deepdrr.instruments:deepdrr.instruments.base.Instrument.NUM_POINTS}}
\pysigstartsignatures
\pysigline{\sphinxbfcode{\sphinxupquote{NUM\_POINTS}}\sphinxbfcode{\sphinxupquote{\DUrole{w,w}{  }\DUrole{p,p}{=}\DUrole{w,w}{  }4000}}}
\pysigstopsignatures
\end{fulllineitems}

\index{advance() (deepdrr.instruments.base.Instrument method)@\spxentry{advance()}\spxextra{deepdrr.instruments.base.Instrument method}}

\begin{fulllineitems}
\phantomsection\label{\detokenize{deepdrr.instruments:deepdrr.instruments.base.Instrument.advance}}
\pysigstartsignatures
\pysiglinewithargsret{\sphinxbfcode{\sphinxupquote{advance}}}{\sphinxparam{\DUrole{n,n}{distance}\DUrole{p,p}{:}\DUrole{w,w}{  }\DUrole{n,n}{float}}}{}
\pysigstopsignatures
\sphinxAtStartPar
Move the tool forward by the given distance.
\begin{quote}\begin{description}
\sphinxlineitem{Parameters}
\sphinxAtStartPar
\sphinxstyleliteralstrong{\sphinxupquote{distance}} (\sphinxstyleliteralemphasis{\sphinxupquote{float}}) \textendash{} The distance to move the tool forward.

\end{description}\end{quote}

\end{fulllineitems}

\index{align() (deepdrr.instruments.base.Instrument method)@\spxentry{align()}\spxextra{deepdrr.instruments.base.Instrument method}}

\begin{fulllineitems}
\phantomsection\label{\detokenize{deepdrr.instruments:deepdrr.instruments.base.Instrument.align}}
\pysigstartsignatures
\pysiglinewithargsret{\sphinxbfcode{\sphinxupquote{align}}}{\sphinxparam{\DUrole{n,n}{startpoint}\DUrole{p,p}{:}\DUrole{w,w}{  }\DUrole{n,n}{{\hyperref[\detokenize{deepdrr.geo:deepdrr.geo.core.Point3D}]{\sphinxcrossref{Point3D}}}}}\sphinxparamcomma \sphinxparam{\DUrole{n,n}{endpoint}\DUrole{p,p}{:}\DUrole{w,w}{  }\DUrole{n,n}{{\hyperref[\detokenize{deepdrr.geo:deepdrr.geo.core.Point3D}]{\sphinxcrossref{Point3D}}}}}\sphinxparamcomma \sphinxparam{\DUrole{n,n}{progress}\DUrole{p,p}{:}\DUrole{w,w}{  }\DUrole{n,n}{float}\DUrole{w,w}{  }\DUrole{o,o}{=}\DUrole{w,w}{  }\DUrole{default_value}{1}}\sphinxparamcomma \sphinxparam{\DUrole{n,n}{distance}\DUrole{p,p}{:}\DUrole{w,w}{  }\DUrole{n,n}{float\DUrole{w,w}{  }\DUrole{p,p}{|}\DUrole{w,w}{  }None}\DUrole{w,w}{  }\DUrole{o,o}{=}\DUrole{w,w}{  }\DUrole{default_value}{None}}}{}
\pysigstopsignatures
\sphinxAtStartPar
Place the tool along the line between startpoint and endpoint.
\begin{quote}\begin{description}
\sphinxlineitem{Parameters}\begin{itemize}
\item {} 
\sphinxAtStartPar
\sphinxstyleliteralstrong{\sphinxupquote{startpoint}} ({\hyperref[\detokenize{deepdrr.geo:deepdrr.geo.Point3D}]{\sphinxcrossref{\sphinxstyleliteralemphasis{\sphinxupquote{geo.Point3D}}}}}) \textendash{} Startpoint in world.

\item {} 
\sphinxAtStartPar
\sphinxstyleliteralstrong{\sphinxupquote{endpoint}} ({\hyperref[\detokenize{deepdrr.geo:deepdrr.geo.Point3D}]{\sphinxcrossref{\sphinxstyleliteralemphasis{\sphinxupquote{geo.Point3D}}}}}) \textendash{} Point in world toward which the tool points.

\item {} 
\sphinxAtStartPar
\sphinxstyleliteralstrong{\sphinxupquote{progress}} (\sphinxstyleliteralemphasis{\sphinxupquote{float}}) \textendash{} The fraction between startpoint and endpoint to place the tip of the tool. Defaults to 1.

\item {} 
\sphinxAtStartPar
\sphinxstyleliteralstrong{\sphinxupquote{distance}} (\sphinxstyleliteralemphasis{\sphinxupquote{Optional}}\sphinxstyleliteralemphasis{\sphinxupquote{{[}}}\sphinxstyleliteralemphasis{\sphinxupquote{float}}\sphinxstyleliteralemphasis{\sphinxupquote{{]}}}\sphinxstyleliteralemphasis{\sphinxupquote{, }}\sphinxstyleliteralemphasis{\sphinxupquote{optional}}) \textendash{} The distance of the tip along the trajectory. 0 corresponds
to the tip placed at the start point, {\color{red}\bfseries{}|startpoint \sphinxhyphen{} endpoint|} at the end point.
Overrides progress if provided. Defaults to None.

\end{itemize}

\end{description}\end{quote}

\end{fulllineitems}

\index{base (deepdrr.instruments.base.Instrument attribute)@\spxentry{base}\spxextra{deepdrr.instruments.base.Instrument attribute}}

\begin{fulllineitems}
\phantomsection\label{\detokenize{deepdrr.instruments:deepdrr.instruments.base.Instrument.base}}
\pysigstartsignatures
\pysigline{\sphinxbfcode{\sphinxupquote{base}}\sphinxbfcode{\sphinxupquote{\DUrole{p,p}{:}\DUrole{w,w}{  }{\hyperref[\detokenize{deepdrr.geo:deepdrr.geo.core.Point3D}]{\sphinxcrossref{Point3D}}}}}}
\pysigstopsignatures
\end{fulllineitems}

\index{base\_in\_world (deepdrr.instruments.base.Instrument property)@\spxentry{base\_in\_world}\spxextra{deepdrr.instruments.base.Instrument property}}

\begin{fulllineitems}
\phantomsection\label{\detokenize{deepdrr.instruments:deepdrr.instruments.base.Instrument.base_in_world}}
\pysigstartsignatures
\pysigline{\sphinxbfcode{\sphinxupquote{property\DUrole{w,w}{  }}}\sphinxbfcode{\sphinxupquote{base\_in\_world}}\sphinxbfcode{\sphinxupquote{\DUrole{p,p}{:}\DUrole{w,w}{  }{\hyperref[\detokenize{deepdrr.geo:deepdrr.geo.core.Point3D}]{\sphinxcrossref{Point3D}}}}}}
\pysigstopsignatures
\end{fulllineitems}

\index{center (deepdrr.instruments.base.Instrument property)@\spxentry{center}\spxextra{deepdrr.instruments.base.Instrument property}}

\begin{fulllineitems}
\phantomsection\label{\detokenize{deepdrr.instruments:deepdrr.instruments.base.Instrument.center}}
\pysigstartsignatures
\pysigline{\sphinxbfcode{\sphinxupquote{property\DUrole{w,w}{  }}}\sphinxbfcode{\sphinxupquote{center}}\sphinxbfcode{\sphinxupquote{\DUrole{p,p}{:}\DUrole{w,w}{  }{\hyperref[\detokenize{deepdrr.geo:deepdrr.geo.core.Point3D}]{\sphinxcrossref{Point3D}}}}}}
\pysigstopsignatures
\end{fulllineitems}

\index{center\_in\_world (deepdrr.instruments.base.Instrument property)@\spxentry{center\_in\_world}\spxextra{deepdrr.instruments.base.Instrument property}}

\begin{fulllineitems}
\phantomsection\label{\detokenize{deepdrr.instruments:deepdrr.instruments.base.Instrument.center_in_world}}
\pysigstartsignatures
\pysigline{\sphinxbfcode{\sphinxupquote{property\DUrole{w,w}{  }}}\sphinxbfcode{\sphinxupquote{center\_in\_world}}\sphinxbfcode{\sphinxupquote{\DUrole{p,p}{:}\DUrole{w,w}{  }{\hyperref[\detokenize{deepdrr.geo:deepdrr.geo.core.Point3D}]{\sphinxcrossref{Point3D}}}}}}
\pysigstopsignatures
\sphinxAtStartPar
The center of the volume in world coorindates. Useful for debugging.

\end{fulllineitems}

\index{centerline\_in\_world (deepdrr.instruments.base.Instrument property)@\spxentry{centerline\_in\_world}\spxextra{deepdrr.instruments.base.Instrument property}}

\begin{fulllineitems}
\phantomsection\label{\detokenize{deepdrr.instruments:deepdrr.instruments.base.Instrument.centerline_in_world}}
\pysigstartsignatures
\pysigline{\sphinxbfcode{\sphinxupquote{property\DUrole{w,w}{  }}}\sphinxbfcode{\sphinxupquote{centerline\_in\_world}}\sphinxbfcode{\sphinxupquote{\DUrole{p,p}{:}\DUrole{w,w}{  }{\hyperref[\detokenize{deepdrr.geo:deepdrr.geo.hyperplane.Line3D}]{\sphinxcrossref{Line3D}}}}}}
\pysigstopsignatures
\end{fulllineitems}

\index{get\_cache\_dir() (deepdrr.instruments.base.Instrument method)@\spxentry{get\_cache\_dir()}\spxextra{deepdrr.instruments.base.Instrument method}}

\begin{fulllineitems}
\phantomsection\label{\detokenize{deepdrr.instruments:deepdrr.instruments.base.Instrument.get_cache_dir}}
\pysigstartsignatures
\pysiglinewithargsret{\sphinxbfcode{\sphinxupquote{get\_cache\_dir}}}{}{{ $\rightarrow$ Path}}
\pysigstopsignatures
\end{fulllineitems}

\index{get\_mesh\_in\_world() (deepdrr.instruments.base.Instrument method)@\spxentry{get\_mesh\_in\_world()}\spxextra{deepdrr.instruments.base.Instrument method}}

\begin{fulllineitems}
\phantomsection\label{\detokenize{deepdrr.instruments:deepdrr.instruments.base.Instrument.get_mesh_in_world}}
\pysigstartsignatures
\pysiglinewithargsret{\sphinxbfcode{\sphinxupquote{get\_mesh\_in\_world}}}{\sphinxparam{\DUrole{n,n}{full}\DUrole{p,p}{:}\DUrole{w,w}{  }\DUrole{n,n}{bool}\DUrole{w,w}{  }\DUrole{o,o}{=}\DUrole{w,w}{  }\DUrole{default_value}{True}}\sphinxparamcomma \sphinxparam{\DUrole{n,n}{use\_cached}\DUrole{p,p}{:}\DUrole{w,w}{  }\DUrole{n,n}{bool}\DUrole{w,w}{  }\DUrole{o,o}{=}\DUrole{w,w}{  }\DUrole{default_value}{True}}}{}
\pysigstopsignatures
\sphinxAtStartPar
Get a pyvista mesh of the outline in world\sphinxhyphen{}space.
\begin{quote}\begin{description}
\sphinxlineitem{Parameters}\begin{itemize}
\item {} 
\sphinxAtStartPar
\sphinxstyleliteralstrong{\sphinxupquote{full}} (\sphinxstyleliteralemphasis{\sphinxupquote{bool}}) \textendash{} Whether to render the full volume or just a wireframe. Defaults to False.

\item {} 
\sphinxAtStartPar
\sphinxstyleliteralstrong{\sphinxupquote{cache\_dir}} (\sphinxstyleliteralemphasis{\sphinxupquote{Optional}}\sphinxstyleliteralemphasis{\sphinxupquote{{[}}}\sphinxstyleliteralemphasis{\sphinxupquote{Path}}\sphinxstyleliteralemphasis{\sphinxupquote{{]}}}\sphinxstyleliteralemphasis{\sphinxupquote{, }}\sphinxstyleliteralemphasis{\sphinxupquote{optional}}) \textendash{} a location to cache the bone surface.

\item {} 
\sphinxAtStartPar
\sphinxstyleliteralstrong{\sphinxupquote{use\_cached}} (\sphinxstyleliteralemphasis{\sphinxupquote{bool}}) \textendash{} If False, don’t use the cached bone surface but re\sphinxhyphen{}create it (expensive). Defaults to True.

\end{itemize}

\sphinxlineitem{Returns}
\sphinxAtStartPar
pyvista mesh.

\sphinxlineitem{Return type}
\sphinxAtStartPar
pv.PolyData

\end{description}\end{quote}

\end{fulllineitems}

\index{get\_model\_paths() (deepdrr.instruments.base.Instrument method)@\spxentry{get\_model\_paths()}\spxextra{deepdrr.instruments.base.Instrument method}}

\begin{fulllineitems}
\phantomsection\label{\detokenize{deepdrr.instruments:deepdrr.instruments.base.Instrument.get_model_paths}}
\pysigstartsignatures
\pysiglinewithargsret{\sphinxbfcode{\sphinxupquote{get\_model\_paths}}}{}{{ $\rightarrow$ List\DUrole{p,p}{{[}}Tuple\DUrole{p,p}{{[}}Path\DUrole{p,p}{,}\DUrole{w,w}{  }List\DUrole{p,p}{{[}}Path\DUrole{p,p}{{]}}\DUrole{p,p}{{]}}\DUrole{p,p}{{]}}}}
\pysigstopsignatures
\sphinxAtStartPar
Get the model paths associated with this Tool.
\begin{quote}\begin{description}
\sphinxlineitem{Returns}
\sphinxAtStartPar
List of tuples containing the material dir and a list of paths with STL files for that material.

\sphinxlineitem{Return type}
\sphinxAtStartPar
List{[}Tuple{[}Path, List{[}Path{]}{]}{]}

\end{description}\end{quote}

\end{fulllineitems}

\index{length\_in\_world (deepdrr.instruments.base.Instrument property)@\spxentry{length\_in\_world}\spxextra{deepdrr.instruments.base.Instrument property}}

\begin{fulllineitems}
\phantomsection\label{\detokenize{deepdrr.instruments:deepdrr.instruments.base.Instrument.length_in_world}}
\pysigstartsignatures
\pysigline{\sphinxbfcode{\sphinxupquote{property\DUrole{w,w}{  }}}\sphinxbfcode{\sphinxupquote{length\_in\_world}}}
\pysigstopsignatures
\end{fulllineitems}

\index{orient() (deepdrr.instruments.base.Instrument method)@\spxentry{orient()}\spxextra{deepdrr.instruments.base.Instrument method}}

\begin{fulllineitems}
\phantomsection\label{\detokenize{deepdrr.instruments:deepdrr.instruments.base.Instrument.orient}}
\pysigstartsignatures
\pysiglinewithargsret{\sphinxbfcode{\sphinxupquote{orient}}}{\sphinxparam{\DUrole{n,n}{startpoint}\DUrole{p,p}{:}\DUrole{w,w}{  }\DUrole{n,n}{{\hyperref[\detokenize{deepdrr.geo:deepdrr.geo.core.Point3D}]{\sphinxcrossref{Point3D}}}}}\sphinxparamcomma \sphinxparam{\DUrole{n,n}{direction}\DUrole{p,p}{:}\DUrole{w,w}{  }\DUrole{n,n}{{\hyperref[\detokenize{deepdrr.geo:deepdrr.geo.core.Vector3D}]{\sphinxcrossref{Vector3D}}}}}\sphinxparamcomma \sphinxparam{\DUrole{n,n}{distance}\DUrole{p,p}{:}\DUrole{w,w}{  }\DUrole{n,n}{float}\DUrole{w,w}{  }\DUrole{o,o}{=}\DUrole{w,w}{  }\DUrole{default_value}{0}}}{}
\pysigstopsignatures
\end{fulllineitems}

\index{radius (deepdrr.instruments.base.Instrument attribute)@\spxentry{radius}\spxextra{deepdrr.instruments.base.Instrument attribute}}

\begin{fulllineitems}
\phantomsection\label{\detokenize{deepdrr.instruments:deepdrr.instruments.base.Instrument.radius}}
\pysigstartsignatures
\pysigline{\sphinxbfcode{\sphinxupquote{radius}}\sphinxbfcode{\sphinxupquote{\DUrole{p,p}{:}\DUrole{w,w}{  }float}}}
\pysigstopsignatures
\end{fulllineitems}

\index{tip (deepdrr.instruments.base.Instrument attribute)@\spxentry{tip}\spxextra{deepdrr.instruments.base.Instrument attribute}}

\begin{fulllineitems}
\phantomsection\label{\detokenize{deepdrr.instruments:deepdrr.instruments.base.Instrument.tip}}
\pysigstartsignatures
\pysigline{\sphinxbfcode{\sphinxupquote{tip}}\sphinxbfcode{\sphinxupquote{\DUrole{p,p}{:}\DUrole{w,w}{  }{\hyperref[\detokenize{deepdrr.geo:deepdrr.geo.core.Point3D}]{\sphinxcrossref{Point3D}}}}}}
\pysigstopsignatures
\end{fulllineitems}

\index{tip\_in\_world (deepdrr.instruments.base.Instrument property)@\spxentry{tip\_in\_world}\spxextra{deepdrr.instruments.base.Instrument property}}

\begin{fulllineitems}
\phantomsection\label{\detokenize{deepdrr.instruments:deepdrr.instruments.base.Instrument.tip_in_world}}
\pysigstartsignatures
\pysigline{\sphinxbfcode{\sphinxupquote{property\DUrole{w,w}{  }}}\sphinxbfcode{\sphinxupquote{tip\_in\_world}}\sphinxbfcode{\sphinxupquote{\DUrole{p,p}{:}\DUrole{w,w}{  }{\hyperref[\detokenize{deepdrr.geo:deepdrr.geo.core.Point3D}]{\sphinxcrossref{Point3D}}}}}}
\pysigstopsignatures
\end{fulllineitems}

\index{trajectory\_in\_world (deepdrr.instruments.base.Instrument property)@\spxentry{trajectory\_in\_world}\spxextra{deepdrr.instruments.base.Instrument property}}

\begin{fulllineitems}
\phantomsection\label{\detokenize{deepdrr.instruments:deepdrr.instruments.base.Instrument.trajectory_in_world}}
\pysigstartsignatures
\pysigline{\sphinxbfcode{\sphinxupquote{property\DUrole{w,w}{  }}}\sphinxbfcode{\sphinxupquote{trajectory\_in\_world}}\sphinxbfcode{\sphinxupquote{\DUrole{p,p}{:}\DUrole{w,w}{  }{\hyperref[\detokenize{deepdrr.geo:deepdrr.geo.ray.Ray3D}]{\sphinxcrossref{Ray3D}}}}}}
\pysigstopsignatures
\end{fulllineitems}

\index{twist() (deepdrr.instruments.base.Instrument method)@\spxentry{twist()}\spxextra{deepdrr.instruments.base.Instrument method}}

\begin{fulllineitems}
\phantomsection\label{\detokenize{deepdrr.instruments:deepdrr.instruments.base.Instrument.twist}}
\pysigstartsignatures
\pysiglinewithargsret{\sphinxbfcode{\sphinxupquote{twist}}}{\sphinxparam{\DUrole{n,n}{angle}\DUrole{p,p}{:}\DUrole{w,w}{  }\DUrole{n,n}{float}}\sphinxparamcomma \sphinxparam{\DUrole{n,n}{degrees}\DUrole{p,p}{:}\DUrole{w,w}{  }\DUrole{n,n}{bool}\DUrole{w,w}{  }\DUrole{o,o}{=}\DUrole{w,w}{  }\DUrole{default_value}{True}}}{}
\pysigstopsignatures
\sphinxAtStartPar
Rotate the tool clockwise (when looking down on it) by \sphinxtitleref{angle}.
\begin{quote}\begin{description}
\sphinxlineitem{Parameters}\begin{itemize}
\item {} 
\sphinxAtStartPar
\sphinxstyleliteralstrong{\sphinxupquote{angle}} (\sphinxstyleliteralemphasis{\sphinxupquote{float}}) \textendash{} The angle.

\item {} 
\sphinxAtStartPar
\sphinxstyleliteralstrong{\sphinxupquote{degrees}} (\sphinxstyleliteralemphasis{\sphinxupquote{bool}}\sphinxstyleliteralemphasis{\sphinxupquote{, }}\sphinxstyleliteralemphasis{\sphinxupquote{optional}}) \textendash{} Whether \sphinxtitleref{angle} is in degrees. Defaults to True.

\end{itemize}

\end{description}\end{quote}

\end{fulllineitems}


\end{fulllineitems}



\subsection{Module contents}
\label{\detokenize{deepdrr.instruments:module-deepdrr.instruments}}\label{\detokenize{deepdrr.instruments:module-contents}}\index{module@\spxentry{module}!deepdrr.instruments@\spxentry{deepdrr.instruments}}\index{deepdrr.instruments@\spxentry{deepdrr.instruments}!module@\spxentry{module}}
\sphinxAtStartPar
Instruments are modeled by voxelizing the surface meshes of the instrument components.
\index{Instrument (class in deepdrr.instruments)@\spxentry{Instrument}\spxextra{class in deepdrr.instruments}}

\begin{fulllineitems}
\phantomsection\label{\detokenize{deepdrr.instruments:deepdrr.instruments.Instrument}}
\pysigstartsignatures
\pysiglinewithargsret{\sphinxbfcode{\sphinxupquote{class\DUrole{w,w}{  }}}\sphinxcode{\sphinxupquote{deepdrr.instruments.}}\sphinxbfcode{\sphinxupquote{Instrument}}}{\sphinxparam{\DUrole{n,n}{density}\DUrole{p,p}{:}\DUrole{w,w}{  }\DUrole{n,n}{float}\DUrole{w,w}{  }\DUrole{o,o}{=}\DUrole{w,w}{  }\DUrole{default_value}{0.1}}\sphinxparamcomma \sphinxparam{\DUrole{n,n}{world\_from\_anatomical}\DUrole{p,p}{:}\DUrole{w,w}{  }\DUrole{n,n}{{\hyperref[\detokenize{deepdrr.geo:deepdrr.geo.core.FrameTransform}]{\sphinxcrossref{FrameTransform}}}\DUrole{w,w}{  }\DUrole{p,p}{|}\DUrole{w,w}{  }None}\DUrole{w,w}{  }\DUrole{o,o}{=}\DUrole{w,w}{  }\DUrole{default_value}{None}}\sphinxparamcomma \sphinxparam{\DUrole{n,n}{densities}\DUrole{p,p}{:}\DUrole{w,w}{  }\DUrole{n,n}{Dict\DUrole{p,p}{{[}}str\DUrole{p,p}{,}\DUrole{w,w}{  }float\DUrole{p,p}{{]}}}\DUrole{w,w}{  }\DUrole{o,o}{=}\DUrole{w,w}{  }\DUrole{default_value}{\{\}}}}{}
\pysigstopsignatures
\sphinxAtStartPar
Bases: {\hyperref[\detokenize{deepdrr.vol:deepdrr.vol.volume.Volume}]{\sphinxcrossref{\sphinxcode{\sphinxupquote{Volume}}}}}, \sphinxcode{\sphinxupquote{ABC}}

\sphinxAtStartPar
A class for representing instruments based on voxelized surface models.

\sphinxAtStartPar
In the DeepDRR\_DATA directory, place an STL file for each material you want to use, inside a
directory determined by the class name. For example, if you have a
class called \sphinxtitleref{MyTool} with steel and plastic components, place the STL files \sphinxtitleref{steel.stl} and
\sphinxtitleref{plastic.stl} in \sphinxtitleref{DeepDRR\_DATA/instruments/MyTool}.

\sphinxAtStartPar
TODO: multiple components of the same material.

\sphinxAtStartPar
DeepDRR\_DATA
└── instruments
\begin{quote}

\sphinxAtStartPar
├── ToolClassName
│   ├── material\_1.stl
│   │── material\_2.stl
│   └── material\_3.stl
└── ToolClassName2
\begin{quote}

\sphinxAtStartPar
├── material\_1.stl
│── material\_2.stl
└── material\_3.stl
\end{quote}
\end{quote}
\index{NUM\_POINTS (deepdrr.instruments.Instrument attribute)@\spxentry{NUM\_POINTS}\spxextra{deepdrr.instruments.Instrument attribute}}

\begin{fulllineitems}
\phantomsection\label{\detokenize{deepdrr.instruments:deepdrr.instruments.Instrument.NUM_POINTS}}
\pysigstartsignatures
\pysigline{\sphinxbfcode{\sphinxupquote{NUM\_POINTS}}\sphinxbfcode{\sphinxupquote{\DUrole{w,w}{  }\DUrole{p,p}{=}\DUrole{w,w}{  }4000}}}
\pysigstopsignatures
\end{fulllineitems}

\index{advance() (deepdrr.instruments.Instrument method)@\spxentry{advance()}\spxextra{deepdrr.instruments.Instrument method}}

\begin{fulllineitems}
\phantomsection\label{\detokenize{deepdrr.instruments:deepdrr.instruments.Instrument.advance}}
\pysigstartsignatures
\pysiglinewithargsret{\sphinxbfcode{\sphinxupquote{advance}}}{\sphinxparam{\DUrole{n,n}{distance}\DUrole{p,p}{:}\DUrole{w,w}{  }\DUrole{n,n}{float}}}{}
\pysigstopsignatures
\sphinxAtStartPar
Move the tool forward by the given distance.
\begin{quote}\begin{description}
\sphinxlineitem{Parameters}
\sphinxAtStartPar
\sphinxstyleliteralstrong{\sphinxupquote{distance}} (\sphinxstyleliteralemphasis{\sphinxupquote{float}}) \textendash{} The distance to move the tool forward.

\end{description}\end{quote}

\end{fulllineitems}

\index{align() (deepdrr.instruments.Instrument method)@\spxentry{align()}\spxextra{deepdrr.instruments.Instrument method}}

\begin{fulllineitems}
\phantomsection\label{\detokenize{deepdrr.instruments:deepdrr.instruments.Instrument.align}}
\pysigstartsignatures
\pysiglinewithargsret{\sphinxbfcode{\sphinxupquote{align}}}{\sphinxparam{\DUrole{n,n}{startpoint}\DUrole{p,p}{:}\DUrole{w,w}{  }\DUrole{n,n}{{\hyperref[\detokenize{deepdrr.geo:deepdrr.geo.core.Point3D}]{\sphinxcrossref{Point3D}}}}}\sphinxparamcomma \sphinxparam{\DUrole{n,n}{endpoint}\DUrole{p,p}{:}\DUrole{w,w}{  }\DUrole{n,n}{{\hyperref[\detokenize{deepdrr.geo:deepdrr.geo.core.Point3D}]{\sphinxcrossref{Point3D}}}}}\sphinxparamcomma \sphinxparam{\DUrole{n,n}{progress}\DUrole{p,p}{:}\DUrole{w,w}{  }\DUrole{n,n}{float}\DUrole{w,w}{  }\DUrole{o,o}{=}\DUrole{w,w}{  }\DUrole{default_value}{1}}\sphinxparamcomma \sphinxparam{\DUrole{n,n}{distance}\DUrole{p,p}{:}\DUrole{w,w}{  }\DUrole{n,n}{float\DUrole{w,w}{  }\DUrole{p,p}{|}\DUrole{w,w}{  }None}\DUrole{w,w}{  }\DUrole{o,o}{=}\DUrole{w,w}{  }\DUrole{default_value}{None}}}{}
\pysigstopsignatures
\sphinxAtStartPar
Place the tool along the line between startpoint and endpoint.
\begin{quote}\begin{description}
\sphinxlineitem{Parameters}\begin{itemize}
\item {} 
\sphinxAtStartPar
\sphinxstyleliteralstrong{\sphinxupquote{startpoint}} ({\hyperref[\detokenize{deepdrr.geo:deepdrr.geo.Point3D}]{\sphinxcrossref{\sphinxstyleliteralemphasis{\sphinxupquote{geo.Point3D}}}}}) \textendash{} Startpoint in world.

\item {} 
\sphinxAtStartPar
\sphinxstyleliteralstrong{\sphinxupquote{endpoint}} ({\hyperref[\detokenize{deepdrr.geo:deepdrr.geo.Point3D}]{\sphinxcrossref{\sphinxstyleliteralemphasis{\sphinxupquote{geo.Point3D}}}}}) \textendash{} Point in world toward which the tool points.

\item {} 
\sphinxAtStartPar
\sphinxstyleliteralstrong{\sphinxupquote{progress}} (\sphinxstyleliteralemphasis{\sphinxupquote{float}}) \textendash{} The fraction between startpoint and endpoint to place the tip of the tool. Defaults to 1.

\item {} 
\sphinxAtStartPar
\sphinxstyleliteralstrong{\sphinxupquote{distance}} (\sphinxstyleliteralemphasis{\sphinxupquote{Optional}}\sphinxstyleliteralemphasis{\sphinxupquote{{[}}}\sphinxstyleliteralemphasis{\sphinxupquote{float}}\sphinxstyleliteralemphasis{\sphinxupquote{{]}}}\sphinxstyleliteralemphasis{\sphinxupquote{, }}\sphinxstyleliteralemphasis{\sphinxupquote{optional}}) \textendash{} The distance of the tip along the trajectory. 0 corresponds
to the tip placed at the start point, {\color{red}\bfseries{}|startpoint \sphinxhyphen{} endpoint|} at the end point.
Overrides progress if provided. Defaults to None.

\end{itemize}

\end{description}\end{quote}

\end{fulllineitems}

\index{anatomical\_coordinate\_system (deepdrr.instruments.Instrument attribute)@\spxentry{anatomical\_coordinate\_system}\spxextra{deepdrr.instruments.Instrument attribute}}

\begin{fulllineitems}
\phantomsection\label{\detokenize{deepdrr.instruments:deepdrr.instruments.Instrument.anatomical_coordinate_system}}
\pysigstartsignatures
\pysigline{\sphinxbfcode{\sphinxupquote{anatomical\_coordinate\_system}}\sphinxbfcode{\sphinxupquote{\DUrole{p,p}{:}\DUrole{w,w}{  }str\DUrole{w,w}{  }\DUrole{p,p}{|}\DUrole{w,w}{  }None}}}
\pysigstopsignatures
\end{fulllineitems}

\index{anatomical\_from\_IJK (deepdrr.instruments.Instrument attribute)@\spxentry{anatomical\_from\_IJK}\spxextra{deepdrr.instruments.Instrument attribute}}

\begin{fulllineitems}
\phantomsection\label{\detokenize{deepdrr.instruments:deepdrr.instruments.Instrument.anatomical_from_IJK}}
\pysigstartsignatures
\pysigline{\sphinxbfcode{\sphinxupquote{anatomical\_from\_IJK}}\sphinxbfcode{\sphinxupquote{\DUrole{p,p}{:}\DUrole{w,w}{  }{\hyperref[\detokenize{deepdrr.geo:deepdrr.geo.core.FrameTransform}]{\sphinxcrossref{FrameTransform}}}}}}
\pysigstopsignatures
\end{fulllineitems}

\index{base (deepdrr.instruments.Instrument attribute)@\spxentry{base}\spxextra{deepdrr.instruments.Instrument attribute}}

\begin{fulllineitems}
\phantomsection\label{\detokenize{deepdrr.instruments:deepdrr.instruments.Instrument.base}}
\pysigstartsignatures
\pysigline{\sphinxbfcode{\sphinxupquote{base}}\sphinxbfcode{\sphinxupquote{\DUrole{p,p}{:}\DUrole{w,w}{  }{\hyperref[\detokenize{deepdrr.geo:deepdrr.geo.core.Point3D}]{\sphinxcrossref{Point3D}}}}}}
\pysigstopsignatures
\end{fulllineitems}

\index{base\_in\_world (deepdrr.instruments.Instrument property)@\spxentry{base\_in\_world}\spxextra{deepdrr.instruments.Instrument property}}

\begin{fulllineitems}
\phantomsection\label{\detokenize{deepdrr.instruments:deepdrr.instruments.Instrument.base_in_world}}
\pysigstartsignatures
\pysigline{\sphinxbfcode{\sphinxupquote{property\DUrole{w,w}{  }}}\sphinxbfcode{\sphinxupquote{base\_in\_world}}\sphinxbfcode{\sphinxupquote{\DUrole{p,p}{:}\DUrole{w,w}{  }{\hyperref[\detokenize{deepdrr.geo:deepdrr.geo.core.Point3D}]{\sphinxcrossref{Point3D}}}}}}
\pysigstopsignatures
\end{fulllineitems}

\index{center (deepdrr.instruments.Instrument property)@\spxentry{center}\spxextra{deepdrr.instruments.Instrument property}}

\begin{fulllineitems}
\phantomsection\label{\detokenize{deepdrr.instruments:deepdrr.instruments.Instrument.center}}
\pysigstartsignatures
\pysigline{\sphinxbfcode{\sphinxupquote{property\DUrole{w,w}{  }}}\sphinxbfcode{\sphinxupquote{center}}\sphinxbfcode{\sphinxupquote{\DUrole{p,p}{:}\DUrole{w,w}{  }{\hyperref[\detokenize{deepdrr.geo:deepdrr.geo.core.Point3D}]{\sphinxcrossref{Point3D}}}}}}
\pysigstopsignatures
\end{fulllineitems}

\index{center\_in\_world (deepdrr.instruments.Instrument property)@\spxentry{center\_in\_world}\spxextra{deepdrr.instruments.Instrument property}}

\begin{fulllineitems}
\phantomsection\label{\detokenize{deepdrr.instruments:deepdrr.instruments.Instrument.center_in_world}}
\pysigstartsignatures
\pysigline{\sphinxbfcode{\sphinxupquote{property\DUrole{w,w}{  }}}\sphinxbfcode{\sphinxupquote{center\_in\_world}}\sphinxbfcode{\sphinxupquote{\DUrole{p,p}{:}\DUrole{w,w}{  }{\hyperref[\detokenize{deepdrr.geo:deepdrr.geo.core.Point3D}]{\sphinxcrossref{Point3D}}}}}}
\pysigstopsignatures
\sphinxAtStartPar
The center of the volume in world coorindates. Useful for debugging.

\end{fulllineitems}

\index{centerline\_in\_world (deepdrr.instruments.Instrument property)@\spxentry{centerline\_in\_world}\spxextra{deepdrr.instruments.Instrument property}}

\begin{fulllineitems}
\phantomsection\label{\detokenize{deepdrr.instruments:deepdrr.instruments.Instrument.centerline_in_world}}
\pysigstartsignatures
\pysigline{\sphinxbfcode{\sphinxupquote{property\DUrole{w,w}{  }}}\sphinxbfcode{\sphinxupquote{centerline\_in\_world}}\sphinxbfcode{\sphinxupquote{\DUrole{p,p}{:}\DUrole{w,w}{  }{\hyperref[\detokenize{deepdrr.geo:deepdrr.geo.hyperplane.Line3D}]{\sphinxcrossref{Line3D}}}}}}
\pysigstopsignatures
\end{fulllineitems}

\index{data (deepdrr.instruments.Instrument attribute)@\spxentry{data}\spxextra{deepdrr.instruments.Instrument attribute}}

\begin{fulllineitems}
\phantomsection\label{\detokenize{deepdrr.instruments:deepdrr.instruments.Instrument.data}}
\pysigstartsignatures
\pysigline{\sphinxbfcode{\sphinxupquote{data}}\sphinxbfcode{\sphinxupquote{\DUrole{p,p}{:}\DUrole{w,w}{  }ndarray}}}
\pysigstopsignatures
\end{fulllineitems}

\index{get\_cache\_dir() (deepdrr.instruments.Instrument method)@\spxentry{get\_cache\_dir()}\spxextra{deepdrr.instruments.Instrument method}}

\begin{fulllineitems}
\phantomsection\label{\detokenize{deepdrr.instruments:deepdrr.instruments.Instrument.get_cache_dir}}
\pysigstartsignatures
\pysiglinewithargsret{\sphinxbfcode{\sphinxupquote{get\_cache\_dir}}}{}{{ $\rightarrow$ Path}}
\pysigstopsignatures
\end{fulllineitems}

\index{get\_mesh\_in\_world() (deepdrr.instruments.Instrument method)@\spxentry{get\_mesh\_in\_world()}\spxextra{deepdrr.instruments.Instrument method}}

\begin{fulllineitems}
\phantomsection\label{\detokenize{deepdrr.instruments:deepdrr.instruments.Instrument.get_mesh_in_world}}
\pysigstartsignatures
\pysiglinewithargsret{\sphinxbfcode{\sphinxupquote{get\_mesh\_in\_world}}}{\sphinxparam{\DUrole{n,n}{full}\DUrole{p,p}{:}\DUrole{w,w}{  }\DUrole{n,n}{bool}\DUrole{w,w}{  }\DUrole{o,o}{=}\DUrole{w,w}{  }\DUrole{default_value}{True}}\sphinxparamcomma \sphinxparam{\DUrole{n,n}{use\_cached}\DUrole{p,p}{:}\DUrole{w,w}{  }\DUrole{n,n}{bool}\DUrole{w,w}{  }\DUrole{o,o}{=}\DUrole{w,w}{  }\DUrole{default_value}{True}}}{}
\pysigstopsignatures
\sphinxAtStartPar
Get a pyvista mesh of the outline in world\sphinxhyphen{}space.
\begin{quote}\begin{description}
\sphinxlineitem{Parameters}\begin{itemize}
\item {} 
\sphinxAtStartPar
\sphinxstyleliteralstrong{\sphinxupquote{full}} (\sphinxstyleliteralemphasis{\sphinxupquote{bool}}) \textendash{} Whether to render the full volume or just a wireframe. Defaults to False.

\item {} 
\sphinxAtStartPar
\sphinxstyleliteralstrong{\sphinxupquote{cache\_dir}} (\sphinxstyleliteralemphasis{\sphinxupquote{Optional}}\sphinxstyleliteralemphasis{\sphinxupquote{{[}}}\sphinxstyleliteralemphasis{\sphinxupquote{Path}}\sphinxstyleliteralemphasis{\sphinxupquote{{]}}}\sphinxstyleliteralemphasis{\sphinxupquote{, }}\sphinxstyleliteralemphasis{\sphinxupquote{optional}}) \textendash{} a location to cache the bone surface.

\item {} 
\sphinxAtStartPar
\sphinxstyleliteralstrong{\sphinxupquote{use\_cached}} (\sphinxstyleliteralemphasis{\sphinxupquote{bool}}) \textendash{} If False, don’t use the cached bone surface but re\sphinxhyphen{}create it (expensive). Defaults to True.

\end{itemize}

\sphinxlineitem{Returns}
\sphinxAtStartPar
pyvista mesh.

\sphinxlineitem{Return type}
\sphinxAtStartPar
pv.PolyData

\end{description}\end{quote}

\end{fulllineitems}

\index{get\_model\_paths() (deepdrr.instruments.Instrument method)@\spxentry{get\_model\_paths()}\spxextra{deepdrr.instruments.Instrument method}}

\begin{fulllineitems}
\phantomsection\label{\detokenize{deepdrr.instruments:deepdrr.instruments.Instrument.get_model_paths}}
\pysigstartsignatures
\pysiglinewithargsret{\sphinxbfcode{\sphinxupquote{get\_model\_paths}}}{}{{ $\rightarrow$ List\DUrole{p,p}{{[}}Tuple\DUrole{p,p}{{[}}Path\DUrole{p,p}{,}\DUrole{w,w}{  }List\DUrole{p,p}{{[}}Path\DUrole{p,p}{{]}}\DUrole{p,p}{{]}}\DUrole{p,p}{{]}}}}
\pysigstopsignatures
\sphinxAtStartPar
Get the model paths associated with this Tool.
\begin{quote}\begin{description}
\sphinxlineitem{Returns}
\sphinxAtStartPar
List of tuples containing the material dir and a list of paths with STL files for that material.

\sphinxlineitem{Return type}
\sphinxAtStartPar
List{[}Tuple{[}Path, List{[}Path{]}{]}{]}

\end{description}\end{quote}

\end{fulllineitems}

\index{length\_in\_world (deepdrr.instruments.Instrument property)@\spxentry{length\_in\_world}\spxextra{deepdrr.instruments.Instrument property}}

\begin{fulllineitems}
\phantomsection\label{\detokenize{deepdrr.instruments:deepdrr.instruments.Instrument.length_in_world}}
\pysigstartsignatures
\pysigline{\sphinxbfcode{\sphinxupquote{property\DUrole{w,w}{  }}}\sphinxbfcode{\sphinxupquote{length\_in\_world}}}
\pysigstopsignatures
\end{fulllineitems}

\index{materials (deepdrr.instruments.Instrument attribute)@\spxentry{materials}\spxextra{deepdrr.instruments.Instrument attribute}}

\begin{fulllineitems}
\phantomsection\label{\detokenize{deepdrr.instruments:deepdrr.instruments.Instrument.materials}}
\pysigstartsignatures
\pysigline{\sphinxbfcode{\sphinxupquote{materials}}\sphinxbfcode{\sphinxupquote{\DUrole{p,p}{:}\DUrole{w,w}{  }Dict\DUrole{p,p}{{[}}str\DUrole{p,p}{,}\DUrole{w,w}{  }ndarray\DUrole{p,p}{{]}}}}}
\pysigstopsignatures
\end{fulllineitems}

\index{orient() (deepdrr.instruments.Instrument method)@\spxentry{orient()}\spxextra{deepdrr.instruments.Instrument method}}

\begin{fulllineitems}
\phantomsection\label{\detokenize{deepdrr.instruments:deepdrr.instruments.Instrument.orient}}
\pysigstartsignatures
\pysiglinewithargsret{\sphinxbfcode{\sphinxupquote{orient}}}{\sphinxparam{\DUrole{n,n}{startpoint}\DUrole{p,p}{:}\DUrole{w,w}{  }\DUrole{n,n}{{\hyperref[\detokenize{deepdrr.geo:deepdrr.geo.core.Point3D}]{\sphinxcrossref{Point3D}}}}}\sphinxparamcomma \sphinxparam{\DUrole{n,n}{direction}\DUrole{p,p}{:}\DUrole{w,w}{  }\DUrole{n,n}{{\hyperref[\detokenize{deepdrr.geo:deepdrr.geo.core.Vector3D}]{\sphinxcrossref{Vector3D}}}}}\sphinxparamcomma \sphinxparam{\DUrole{n,n}{distance}\DUrole{p,p}{:}\DUrole{w,w}{  }\DUrole{n,n}{float}\DUrole{w,w}{  }\DUrole{o,o}{=}\DUrole{w,w}{  }\DUrole{default_value}{0}}}{}
\pysigstopsignatures
\end{fulllineitems}

\index{radius (deepdrr.instruments.Instrument attribute)@\spxentry{radius}\spxextra{deepdrr.instruments.Instrument attribute}}

\begin{fulllineitems}
\phantomsection\label{\detokenize{deepdrr.instruments:deepdrr.instruments.Instrument.radius}}
\pysigstartsignatures
\pysigline{\sphinxbfcode{\sphinxupquote{radius}}\sphinxbfcode{\sphinxupquote{\DUrole{p,p}{:}\DUrole{w,w}{  }float}}}
\pysigstopsignatures
\end{fulllineitems}

\index{tip (deepdrr.instruments.Instrument attribute)@\spxentry{tip}\spxextra{deepdrr.instruments.Instrument attribute}}

\begin{fulllineitems}
\phantomsection\label{\detokenize{deepdrr.instruments:deepdrr.instruments.Instrument.tip}}
\pysigstartsignatures
\pysigline{\sphinxbfcode{\sphinxupquote{tip}}\sphinxbfcode{\sphinxupquote{\DUrole{p,p}{:}\DUrole{w,w}{  }{\hyperref[\detokenize{deepdrr.geo:deepdrr.geo.core.Point3D}]{\sphinxcrossref{Point3D}}}}}}
\pysigstopsignatures
\end{fulllineitems}

\index{tip\_in\_world (deepdrr.instruments.Instrument property)@\spxentry{tip\_in\_world}\spxextra{deepdrr.instruments.Instrument property}}

\begin{fulllineitems}
\phantomsection\label{\detokenize{deepdrr.instruments:deepdrr.instruments.Instrument.tip_in_world}}
\pysigstartsignatures
\pysigline{\sphinxbfcode{\sphinxupquote{property\DUrole{w,w}{  }}}\sphinxbfcode{\sphinxupquote{tip\_in\_world}}\sphinxbfcode{\sphinxupquote{\DUrole{p,p}{:}\DUrole{w,w}{  }{\hyperref[\detokenize{deepdrr.geo:deepdrr.geo.core.Point3D}]{\sphinxcrossref{Point3D}}}}}}
\pysigstopsignatures
\end{fulllineitems}

\index{trajectory\_in\_world (deepdrr.instruments.Instrument property)@\spxentry{trajectory\_in\_world}\spxextra{deepdrr.instruments.Instrument property}}

\begin{fulllineitems}
\phantomsection\label{\detokenize{deepdrr.instruments:deepdrr.instruments.Instrument.trajectory_in_world}}
\pysigstartsignatures
\pysigline{\sphinxbfcode{\sphinxupquote{property\DUrole{w,w}{  }}}\sphinxbfcode{\sphinxupquote{trajectory\_in\_world}}\sphinxbfcode{\sphinxupquote{\DUrole{p,p}{:}\DUrole{w,w}{  }{\hyperref[\detokenize{deepdrr.geo:deepdrr.geo.ray.Ray3D}]{\sphinxcrossref{Ray3D}}}}}}
\pysigstopsignatures
\end{fulllineitems}

\index{twist() (deepdrr.instruments.Instrument method)@\spxentry{twist()}\spxextra{deepdrr.instruments.Instrument method}}

\begin{fulllineitems}
\phantomsection\label{\detokenize{deepdrr.instruments:deepdrr.instruments.Instrument.twist}}
\pysigstartsignatures
\pysiglinewithargsret{\sphinxbfcode{\sphinxupquote{twist}}}{\sphinxparam{\DUrole{n,n}{angle}\DUrole{p,p}{:}\DUrole{w,w}{  }\DUrole{n,n}{float}}\sphinxparamcomma \sphinxparam{\DUrole{n,n}{degrees}\DUrole{p,p}{:}\DUrole{w,w}{  }\DUrole{n,n}{bool}\DUrole{w,w}{  }\DUrole{o,o}{=}\DUrole{w,w}{  }\DUrole{default_value}{True}}}{}
\pysigstopsignatures
\sphinxAtStartPar
Rotate the tool clockwise (when looking down on it) by \sphinxtitleref{angle}.
\begin{quote}\begin{description}
\sphinxlineitem{Parameters}\begin{itemize}
\item {} 
\sphinxAtStartPar
\sphinxstyleliteralstrong{\sphinxupquote{angle}} (\sphinxstyleliteralemphasis{\sphinxupquote{float}}) \textendash{} The angle.

\item {} 
\sphinxAtStartPar
\sphinxstyleliteralstrong{\sphinxupquote{degrees}} (\sphinxstyleliteralemphasis{\sphinxupquote{bool}}\sphinxstyleliteralemphasis{\sphinxupquote{, }}\sphinxstyleliteralemphasis{\sphinxupquote{optional}}) \textendash{} Whether \sphinxtitleref{angle} is in degrees. Defaults to True.

\end{itemize}

\end{description}\end{quote}

\end{fulllineitems}

\index{world\_from\_anatomical (deepdrr.instruments.Instrument attribute)@\spxentry{world\_from\_anatomical}\spxextra{deepdrr.instruments.Instrument attribute}}

\begin{fulllineitems}
\phantomsection\label{\detokenize{deepdrr.instruments:deepdrr.instruments.Instrument.world_from_anatomical}}
\pysigstartsignatures
\pysigline{\sphinxbfcode{\sphinxupquote{world\_from\_anatomical}}\sphinxbfcode{\sphinxupquote{\DUrole{p,p}{:}\DUrole{w,w}{  }{\hyperref[\detokenize{deepdrr.geo:deepdrr.geo.core.FrameTransform}]{\sphinxcrossref{FrameTransform}}}}}}
\pysigstopsignatures
\end{fulllineitems}


\end{fulllineitems}


\sphinxstepscope


\section{deepdrr.projector package}
\label{\detokenize{deepdrr.projector:deepdrr-projector-package}}\label{\detokenize{deepdrr.projector::doc}}
\sphinxstepscope


\subsection{deepdrr.projector.mcgpu\_incoherent\_scatter\_data package}
\label{\detokenize{deepdrr.projector.mcgpu_incoherent_scatter_data:deepdrr-projector-mcgpu-incoherent-scatter-data-package}}\label{\detokenize{deepdrr.projector.mcgpu_incoherent_scatter_data::doc}}

\subsubsection{deepdrr.projector.mcgpu\_incoherent\_scatter\_data.PMMA\_compton\_data}
\label{\detokenize{deepdrr.projector.mcgpu_incoherent_scatter_data:module-deepdrr.projector.mcgpu_incoherent_scatter_data.PMMA_compton_data}}\label{\detokenize{deepdrr.projector.mcgpu_incoherent_scatter_data:deepdrr-projector-mcgpu-incoherent-scatter-data-pmma-compton-data}}\index{module@\spxentry{module}!deepdrr.projector.mcgpu\_incoherent\_scatter\_data.PMMA\_compton\_data@\spxentry{deepdrr.projector.mcgpu\_incoherent\_scatter\_data.PMMA\_compton\_data}}\index{deepdrr.projector.mcgpu\_incoherent\_scatter\_data.PMMA\_compton\_data@\spxentry{deepdrr.projector.mcgpu\_incoherent\_scatter\_data.PMMA\_compton\_data}!module@\spxentry{module}}

\subsubsection{deepdrr.projector.mcgpu\_incoherent\_scatter\_data.adipose\_compton\_data}
\label{\detokenize{deepdrr.projector.mcgpu_incoherent_scatter_data:module-deepdrr.projector.mcgpu_incoherent_scatter_data.adipose_compton_data}}\label{\detokenize{deepdrr.projector.mcgpu_incoherent_scatter_data:deepdrr-projector-mcgpu-incoherent-scatter-data-adipose-compton-data}}\index{module@\spxentry{module}!deepdrr.projector.mcgpu\_incoherent\_scatter\_data.adipose\_compton\_data@\spxentry{deepdrr.projector.mcgpu\_incoherent\_scatter\_data.adipose\_compton\_data}}\index{deepdrr.projector.mcgpu\_incoherent\_scatter\_data.adipose\_compton\_data@\spxentry{deepdrr.projector.mcgpu\_incoherent\_scatter\_data.adipose\_compton\_data}!module@\spxentry{module}}

\subsubsection{deepdrr.projector.mcgpu\_incoherent\_scatter\_data.air\_compton\_data}
\label{\detokenize{deepdrr.projector.mcgpu_incoherent_scatter_data:module-deepdrr.projector.mcgpu_incoherent_scatter_data.air_compton_data}}\label{\detokenize{deepdrr.projector.mcgpu_incoherent_scatter_data:deepdrr-projector-mcgpu-incoherent-scatter-data-air-compton-data}}\index{module@\spxentry{module}!deepdrr.projector.mcgpu\_incoherent\_scatter\_data.air\_compton\_data@\spxentry{deepdrr.projector.mcgpu\_incoherent\_scatter\_data.air\_compton\_data}}\index{deepdrr.projector.mcgpu\_incoherent\_scatter\_data.air\_compton\_data@\spxentry{deepdrr.projector.mcgpu\_incoherent\_scatter\_data.air\_compton\_data}!module@\spxentry{module}}

\subsubsection{deepdrr.projector.mcgpu\_incoherent\_scatter\_data.blood\_compton\_data}
\label{\detokenize{deepdrr.projector.mcgpu_incoherent_scatter_data:module-deepdrr.projector.mcgpu_incoherent_scatter_data.blood_compton_data}}\label{\detokenize{deepdrr.projector.mcgpu_incoherent_scatter_data:deepdrr-projector-mcgpu-incoherent-scatter-data-blood-compton-data}}\index{module@\spxentry{module}!deepdrr.projector.mcgpu\_incoherent\_scatter\_data.blood\_compton\_data@\spxentry{deepdrr.projector.mcgpu\_incoherent\_scatter\_data.blood\_compton\_data}}\index{deepdrr.projector.mcgpu\_incoherent\_scatter\_data.blood\_compton\_data@\spxentry{deepdrr.projector.mcgpu\_incoherent\_scatter\_data.blood\_compton\_data}!module@\spxentry{module}}

\subsubsection{deepdrr.projector.mcgpu\_incoherent\_scatter\_data.bone\_compton\_data}
\label{\detokenize{deepdrr.projector.mcgpu_incoherent_scatter_data:module-deepdrr.projector.mcgpu_incoherent_scatter_data.bone_compton_data}}\label{\detokenize{deepdrr.projector.mcgpu_incoherent_scatter_data:deepdrr-projector-mcgpu-incoherent-scatter-data-bone-compton-data}}\index{module@\spxentry{module}!deepdrr.projector.mcgpu\_incoherent\_scatter\_data.bone\_compton\_data@\spxentry{deepdrr.projector.mcgpu\_incoherent\_scatter\_data.bone\_compton\_data}}\index{deepdrr.projector.mcgpu\_incoherent\_scatter\_data.bone\_compton\_data@\spxentry{deepdrr.projector.mcgpu\_incoherent\_scatter\_data.bone\_compton\_data}!module@\spxentry{module}}

\subsubsection{deepdrr.projector.mcgpu\_incoherent\_scatter\_data.brain\_compton\_data}
\label{\detokenize{deepdrr.projector.mcgpu_incoherent_scatter_data:module-deepdrr.projector.mcgpu_incoherent_scatter_data.brain_compton_data}}\label{\detokenize{deepdrr.projector.mcgpu_incoherent_scatter_data:deepdrr-projector-mcgpu-incoherent-scatter-data-brain-compton-data}}\index{module@\spxentry{module}!deepdrr.projector.mcgpu\_incoherent\_scatter\_data.brain\_compton\_data@\spxentry{deepdrr.projector.mcgpu\_incoherent\_scatter\_data.brain\_compton\_data}}\index{deepdrr.projector.mcgpu\_incoherent\_scatter\_data.brain\_compton\_data@\spxentry{deepdrr.projector.mcgpu\_incoherent\_scatter\_data.brain\_compton\_data}!module@\spxentry{module}}

\subsubsection{deepdrr.projector.mcgpu\_incoherent\_scatter\_data.breast\_compton\_data}
\label{\detokenize{deepdrr.projector.mcgpu_incoherent_scatter_data:module-deepdrr.projector.mcgpu_incoherent_scatter_data.breast_compton_data}}\label{\detokenize{deepdrr.projector.mcgpu_incoherent_scatter_data:deepdrr-projector-mcgpu-incoherent-scatter-data-breast-compton-data}}\index{module@\spxentry{module}!deepdrr.projector.mcgpu\_incoherent\_scatter\_data.breast\_compton\_data@\spxentry{deepdrr.projector.mcgpu\_incoherent\_scatter\_data.breast\_compton\_data}}\index{deepdrr.projector.mcgpu\_incoherent\_scatter\_data.breast\_compton\_data@\spxentry{deepdrr.projector.mcgpu\_incoherent\_scatter\_data.breast\_compton\_data}!module@\spxentry{module}}

\subsubsection{deepdrr.projector.mcgpu\_incoherent\_scatter\_data.cartilage\_compton\_data}
\label{\detokenize{deepdrr.projector.mcgpu_incoherent_scatter_data:module-deepdrr.projector.mcgpu_incoherent_scatter_data.cartilage_compton_data}}\label{\detokenize{deepdrr.projector.mcgpu_incoherent_scatter_data:deepdrr-projector-mcgpu-incoherent-scatter-data-cartilage-compton-data}}\index{module@\spxentry{module}!deepdrr.projector.mcgpu\_incoherent\_scatter\_data.cartilage\_compton\_data@\spxentry{deepdrr.projector.mcgpu\_incoherent\_scatter\_data.cartilage\_compton\_data}}\index{deepdrr.projector.mcgpu\_incoherent\_scatter\_data.cartilage\_compton\_data@\spxentry{deepdrr.projector.mcgpu\_incoherent\_scatter\_data.cartilage\_compton\_data}!module@\spxentry{module}}

\subsubsection{deepdrr.projector.mcgpu\_incoherent\_scatter\_data.connective\_compton\_data}
\label{\detokenize{deepdrr.projector.mcgpu_incoherent_scatter_data:module-deepdrr.projector.mcgpu_incoherent_scatter_data.connective_compton_data}}\label{\detokenize{deepdrr.projector.mcgpu_incoherent_scatter_data:deepdrr-projector-mcgpu-incoherent-scatter-data-connective-compton-data}}\index{module@\spxentry{module}!deepdrr.projector.mcgpu\_incoherent\_scatter\_data.connective\_compton\_data@\spxentry{deepdrr.projector.mcgpu\_incoherent\_scatter\_data.connective\_compton\_data}}\index{deepdrr.projector.mcgpu\_incoherent\_scatter\_data.connective\_compton\_data@\spxentry{deepdrr.projector.mcgpu\_incoherent\_scatter\_data.connective\_compton\_data}!module@\spxentry{module}}

\subsubsection{deepdrr.projector.mcgpu\_incoherent\_scatter\_data.glands\_others\_compton\_data}
\label{\detokenize{deepdrr.projector.mcgpu_incoherent_scatter_data:module-deepdrr.projector.mcgpu_incoherent_scatter_data.glands_others_compton_data}}\label{\detokenize{deepdrr.projector.mcgpu_incoherent_scatter_data:deepdrr-projector-mcgpu-incoherent-scatter-data-glands-others-compton-data}}\index{module@\spxentry{module}!deepdrr.projector.mcgpu\_incoherent\_scatter\_data.glands\_others\_compton\_data@\spxentry{deepdrr.projector.mcgpu\_incoherent\_scatter\_data.glands\_others\_compton\_data}}\index{deepdrr.projector.mcgpu\_incoherent\_scatter\_data.glands\_others\_compton\_data@\spxentry{deepdrr.projector.mcgpu\_incoherent\_scatter\_data.glands\_others\_compton\_data}!module@\spxentry{module}}

\subsubsection{deepdrr.projector.mcgpu\_incoherent\_scatter\_data.liver\_compton\_data}
\label{\detokenize{deepdrr.projector.mcgpu_incoherent_scatter_data:module-deepdrr.projector.mcgpu_incoherent_scatter_data.liver_compton_data}}\label{\detokenize{deepdrr.projector.mcgpu_incoherent_scatter_data:deepdrr-projector-mcgpu-incoherent-scatter-data-liver-compton-data}}\index{module@\spxentry{module}!deepdrr.projector.mcgpu\_incoherent\_scatter\_data.liver\_compton\_data@\spxentry{deepdrr.projector.mcgpu\_incoherent\_scatter\_data.liver\_compton\_data}}\index{deepdrr.projector.mcgpu\_incoherent\_scatter\_data.liver\_compton\_data@\spxentry{deepdrr.projector.mcgpu\_incoherent\_scatter\_data.liver\_compton\_data}!module@\spxentry{module}}

\subsubsection{deepdrr.projector.mcgpu\_incoherent\_scatter\_data.lung\_compton\_data}
\label{\detokenize{deepdrr.projector.mcgpu_incoherent_scatter_data:module-deepdrr.projector.mcgpu_incoherent_scatter_data.lung_compton_data}}\label{\detokenize{deepdrr.projector.mcgpu_incoherent_scatter_data:deepdrr-projector-mcgpu-incoherent-scatter-data-lung-compton-data}}\index{module@\spxentry{module}!deepdrr.projector.mcgpu\_incoherent\_scatter\_data.lung\_compton\_data@\spxentry{deepdrr.projector.mcgpu\_incoherent\_scatter\_data.lung\_compton\_data}}\index{deepdrr.projector.mcgpu\_incoherent\_scatter\_data.lung\_compton\_data@\spxentry{deepdrr.projector.mcgpu\_incoherent\_scatter\_data.lung\_compton\_data}!module@\spxentry{module}}

\subsubsection{deepdrr.projector.mcgpu\_incoherent\_scatter\_data.muscle\_compton\_data}
\label{\detokenize{deepdrr.projector.mcgpu_incoherent_scatter_data:module-deepdrr.projector.mcgpu_incoherent_scatter_data.muscle_compton_data}}\label{\detokenize{deepdrr.projector.mcgpu_incoherent_scatter_data:deepdrr-projector-mcgpu-incoherent-scatter-data-muscle-compton-data}}\index{module@\spxentry{module}!deepdrr.projector.mcgpu\_incoherent\_scatter\_data.muscle\_compton\_data@\spxentry{deepdrr.projector.mcgpu\_incoherent\_scatter\_data.muscle\_compton\_data}}\index{deepdrr.projector.mcgpu\_incoherent\_scatter\_data.muscle\_compton\_data@\spxentry{deepdrr.projector.mcgpu\_incoherent\_scatter\_data.muscle\_compton\_data}!module@\spxentry{module}}

\subsubsection{deepdrr.projector.mcgpu\_incoherent\_scatter\_data.red\_marrow\_compton\_data}
\label{\detokenize{deepdrr.projector.mcgpu_incoherent_scatter_data:module-deepdrr.projector.mcgpu_incoherent_scatter_data.red_marrow_compton_data}}\label{\detokenize{deepdrr.projector.mcgpu_incoherent_scatter_data:deepdrr-projector-mcgpu-incoherent-scatter-data-red-marrow-compton-data}}\index{module@\spxentry{module}!deepdrr.projector.mcgpu\_incoherent\_scatter\_data.red\_marrow\_compton\_data@\spxentry{deepdrr.projector.mcgpu\_incoherent\_scatter\_data.red\_marrow\_compton\_data}}\index{deepdrr.projector.mcgpu\_incoherent\_scatter\_data.red\_marrow\_compton\_data@\spxentry{deepdrr.projector.mcgpu\_incoherent\_scatter\_data.red\_marrow\_compton\_data}!module@\spxentry{module}}

\subsubsection{deepdrr.projector.mcgpu\_incoherent\_scatter\_data.skin\_compton\_data}
\label{\detokenize{deepdrr.projector.mcgpu_incoherent_scatter_data:module-deepdrr.projector.mcgpu_incoherent_scatter_data.skin_compton_data}}\label{\detokenize{deepdrr.projector.mcgpu_incoherent_scatter_data:deepdrr-projector-mcgpu-incoherent-scatter-data-skin-compton-data}}\index{module@\spxentry{module}!deepdrr.projector.mcgpu\_incoherent\_scatter\_data.skin\_compton\_data@\spxentry{deepdrr.projector.mcgpu\_incoherent\_scatter\_data.skin\_compton\_data}}\index{deepdrr.projector.mcgpu\_incoherent\_scatter\_data.skin\_compton\_data@\spxentry{deepdrr.projector.mcgpu\_incoherent\_scatter\_data.skin\_compton\_data}!module@\spxentry{module}}

\subsubsection{deepdrr.projector.mcgpu\_incoherent\_scatter\_data.soft\_tissue\_compton\_data}
\label{\detokenize{deepdrr.projector.mcgpu_incoherent_scatter_data:module-deepdrr.projector.mcgpu_incoherent_scatter_data.soft_tissue_compton_data}}\label{\detokenize{deepdrr.projector.mcgpu_incoherent_scatter_data:deepdrr-projector-mcgpu-incoherent-scatter-data-soft-tissue-compton-data}}\index{module@\spxentry{module}!deepdrr.projector.mcgpu\_incoherent\_scatter\_data.soft\_tissue\_compton\_data@\spxentry{deepdrr.projector.mcgpu\_incoherent\_scatter\_data.soft\_tissue\_compton\_data}}\index{deepdrr.projector.mcgpu\_incoherent\_scatter\_data.soft\_tissue\_compton\_data@\spxentry{deepdrr.projector.mcgpu\_incoherent\_scatter\_data.soft\_tissue\_compton\_data}!module@\spxentry{module}}

\subsubsection{deepdrr.projector.mcgpu\_incoherent\_scatter\_data.stomach\_intestines\_compton\_data}
\label{\detokenize{deepdrr.projector.mcgpu_incoherent_scatter_data:module-deepdrr.projector.mcgpu_incoherent_scatter_data.stomach_intestines_compton_data}}\label{\detokenize{deepdrr.projector.mcgpu_incoherent_scatter_data:deepdrr-projector-mcgpu-incoherent-scatter-data-stomach-intestines-compton-data}}\index{module@\spxentry{module}!deepdrr.projector.mcgpu\_incoherent\_scatter\_data.stomach\_intestines\_compton\_data@\spxentry{deepdrr.projector.mcgpu\_incoherent\_scatter\_data.stomach\_intestines\_compton\_data}}\index{deepdrr.projector.mcgpu\_incoherent\_scatter\_data.stomach\_intestines\_compton\_data@\spxentry{deepdrr.projector.mcgpu\_incoherent\_scatter\_data.stomach\_intestines\_compton\_data}!module@\spxentry{module}}

\subsubsection{deepdrr.projector.mcgpu\_incoherent\_scatter\_data.titanium\_compton\_data}
\label{\detokenize{deepdrr.projector.mcgpu_incoherent_scatter_data:module-deepdrr.projector.mcgpu_incoherent_scatter_data.titanium_compton_data}}\label{\detokenize{deepdrr.projector.mcgpu_incoherent_scatter_data:deepdrr-projector-mcgpu-incoherent-scatter-data-titanium-compton-data}}\index{module@\spxentry{module}!deepdrr.projector.mcgpu\_incoherent\_scatter\_data.titanium\_compton\_data@\spxentry{deepdrr.projector.mcgpu\_incoherent\_scatter\_data.titanium\_compton\_data}}\index{deepdrr.projector.mcgpu\_incoherent\_scatter\_data.titanium\_compton\_data@\spxentry{deepdrr.projector.mcgpu\_incoherent\_scatter\_data.titanium\_compton\_data}!module@\spxentry{module}}

\subsubsection{deepdrr.projector.mcgpu\_incoherent\_scatter\_data.water\_compton\_data}
\label{\detokenize{deepdrr.projector.mcgpu_incoherent_scatter_data:module-deepdrr.projector.mcgpu_incoherent_scatter_data.water_compton_data}}\label{\detokenize{deepdrr.projector.mcgpu_incoherent_scatter_data:deepdrr-projector-mcgpu-incoherent-scatter-data-water-compton-data}}\index{module@\spxentry{module}!deepdrr.projector.mcgpu\_incoherent\_scatter\_data.water\_compton\_data@\spxentry{deepdrr.projector.mcgpu\_incoherent\_scatter\_data.water\_compton\_data}}\index{deepdrr.projector.mcgpu\_incoherent\_scatter\_data.water\_compton\_data@\spxentry{deepdrr.projector.mcgpu\_incoherent\_scatter\_data.water\_compton\_data}!module@\spxentry{module}}

\subsubsection{Module contents}
\label{\detokenize{deepdrr.projector.mcgpu_incoherent_scatter_data:module-deepdrr.projector.mcgpu_incoherent_scatter_data}}\label{\detokenize{deepdrr.projector.mcgpu_incoherent_scatter_data:module-contents}}\index{module@\spxentry{module}!deepdrr.projector.mcgpu\_incoherent\_scatter\_data@\spxentry{deepdrr.projector.mcgpu\_incoherent\_scatter\_data}}\index{deepdrr.projector.mcgpu\_incoherent\_scatter\_data@\spxentry{deepdrr.projector.mcgpu\_incoherent\_scatter\_data}!module@\spxentry{module}}
\sphinxstepscope


\subsection{deepdrr.projector.mcgpu\_mean\_free\_path\_data package}
\label{\detokenize{deepdrr.projector.mcgpu_mean_free_path_data:deepdrr-projector-mcgpu-mean-free-path-data-package}}\label{\detokenize{deepdrr.projector.mcgpu_mean_free_path_data::doc}}

\subsubsection{deepdrr.projector.mcgpu\_mean\_free\_path\_data.PMMA\_mfp}
\label{\detokenize{deepdrr.projector.mcgpu_mean_free_path_data:module-deepdrr.projector.mcgpu_mean_free_path_data.PMMA_mfp}}\label{\detokenize{deepdrr.projector.mcgpu_mean_free_path_data:deepdrr-projector-mcgpu-mean-free-path-data-pmma-mfp}}\index{module@\spxentry{module}!deepdrr.projector.mcgpu\_mean\_free\_path\_data.PMMA\_mfp@\spxentry{deepdrr.projector.mcgpu\_mean\_free\_path\_data.PMMA\_mfp}}\index{deepdrr.projector.mcgpu\_mean\_free\_path\_data.PMMA\_mfp@\spxentry{deepdrr.projector.mcgpu\_mean\_free\_path\_data.PMMA\_mfp}!module@\spxentry{module}}

\subsubsection{deepdrr.projector.mcgpu\_mean\_free\_path\_data.adipose\_mfp}
\label{\detokenize{deepdrr.projector.mcgpu_mean_free_path_data:module-deepdrr.projector.mcgpu_mean_free_path_data.adipose_mfp}}\label{\detokenize{deepdrr.projector.mcgpu_mean_free_path_data:deepdrr-projector-mcgpu-mean-free-path-data-adipose-mfp}}\index{module@\spxentry{module}!deepdrr.projector.mcgpu\_mean\_free\_path\_data.adipose\_mfp@\spxentry{deepdrr.projector.mcgpu\_mean\_free\_path\_data.adipose\_mfp}}\index{deepdrr.projector.mcgpu\_mean\_free\_path\_data.adipose\_mfp@\spxentry{deepdrr.projector.mcgpu\_mean\_free\_path\_data.adipose\_mfp}!module@\spxentry{module}}

\subsubsection{deepdrr.projector.mcgpu\_mean\_free\_path\_data.air\_mfp}
\label{\detokenize{deepdrr.projector.mcgpu_mean_free_path_data:module-deepdrr.projector.mcgpu_mean_free_path_data.air_mfp}}\label{\detokenize{deepdrr.projector.mcgpu_mean_free_path_data:deepdrr-projector-mcgpu-mean-free-path-data-air-mfp}}\index{module@\spxentry{module}!deepdrr.projector.mcgpu\_mean\_free\_path\_data.air\_mfp@\spxentry{deepdrr.projector.mcgpu\_mean\_free\_path\_data.air\_mfp}}\index{deepdrr.projector.mcgpu\_mean\_free\_path\_data.air\_mfp@\spxentry{deepdrr.projector.mcgpu\_mean\_free\_path\_data.air\_mfp}!module@\spxentry{module}}

\subsubsection{deepdrr.projector.mcgpu\_mean\_free\_path\_data.blood\_mfp}
\label{\detokenize{deepdrr.projector.mcgpu_mean_free_path_data:module-deepdrr.projector.mcgpu_mean_free_path_data.blood_mfp}}\label{\detokenize{deepdrr.projector.mcgpu_mean_free_path_data:deepdrr-projector-mcgpu-mean-free-path-data-blood-mfp}}\index{module@\spxentry{module}!deepdrr.projector.mcgpu\_mean\_free\_path\_data.blood\_mfp@\spxentry{deepdrr.projector.mcgpu\_mean\_free\_path\_data.blood\_mfp}}\index{deepdrr.projector.mcgpu\_mean\_free\_path\_data.blood\_mfp@\spxentry{deepdrr.projector.mcgpu\_mean\_free\_path\_data.blood\_mfp}!module@\spxentry{module}}

\subsubsection{deepdrr.projector.mcgpu\_mean\_free\_path\_data.bone\_mfp}
\label{\detokenize{deepdrr.projector.mcgpu_mean_free_path_data:module-deepdrr.projector.mcgpu_mean_free_path_data.bone_mfp}}\label{\detokenize{deepdrr.projector.mcgpu_mean_free_path_data:deepdrr-projector-mcgpu-mean-free-path-data-bone-mfp}}\index{module@\spxentry{module}!deepdrr.projector.mcgpu\_mean\_free\_path\_data.bone\_mfp@\spxentry{deepdrr.projector.mcgpu\_mean\_free\_path\_data.bone\_mfp}}\index{deepdrr.projector.mcgpu\_mean\_free\_path\_data.bone\_mfp@\spxentry{deepdrr.projector.mcgpu\_mean\_free\_path\_data.bone\_mfp}!module@\spxentry{module}}

\subsubsection{deepdrr.projector.mcgpu\_mean\_free\_path\_data.brain\_mfp}
\label{\detokenize{deepdrr.projector.mcgpu_mean_free_path_data:module-deepdrr.projector.mcgpu_mean_free_path_data.brain_mfp}}\label{\detokenize{deepdrr.projector.mcgpu_mean_free_path_data:deepdrr-projector-mcgpu-mean-free-path-data-brain-mfp}}\index{module@\spxentry{module}!deepdrr.projector.mcgpu\_mean\_free\_path\_data.brain\_mfp@\spxentry{deepdrr.projector.mcgpu\_mean\_free\_path\_data.brain\_mfp}}\index{deepdrr.projector.mcgpu\_mean\_free\_path\_data.brain\_mfp@\spxentry{deepdrr.projector.mcgpu\_mean\_free\_path\_data.brain\_mfp}!module@\spxentry{module}}

\subsubsection{deepdrr.projector.mcgpu\_mean\_free\_path\_data.breast\_mfp}
\label{\detokenize{deepdrr.projector.mcgpu_mean_free_path_data:module-deepdrr.projector.mcgpu_mean_free_path_data.breast_mfp}}\label{\detokenize{deepdrr.projector.mcgpu_mean_free_path_data:deepdrr-projector-mcgpu-mean-free-path-data-breast-mfp}}\index{module@\spxentry{module}!deepdrr.projector.mcgpu\_mean\_free\_path\_data.breast\_mfp@\spxentry{deepdrr.projector.mcgpu\_mean\_free\_path\_data.breast\_mfp}}\index{deepdrr.projector.mcgpu\_mean\_free\_path\_data.breast\_mfp@\spxentry{deepdrr.projector.mcgpu\_mean\_free\_path\_data.breast\_mfp}!module@\spxentry{module}}

\subsubsection{deepdrr.projector.mcgpu\_mean\_free\_path\_data.cartilage\_mfp}
\label{\detokenize{deepdrr.projector.mcgpu_mean_free_path_data:module-deepdrr.projector.mcgpu_mean_free_path_data.cartilage_mfp}}\label{\detokenize{deepdrr.projector.mcgpu_mean_free_path_data:deepdrr-projector-mcgpu-mean-free-path-data-cartilage-mfp}}\index{module@\spxentry{module}!deepdrr.projector.mcgpu\_mean\_free\_path\_data.cartilage\_mfp@\spxentry{deepdrr.projector.mcgpu\_mean\_free\_path\_data.cartilage\_mfp}}\index{deepdrr.projector.mcgpu\_mean\_free\_path\_data.cartilage\_mfp@\spxentry{deepdrr.projector.mcgpu\_mean\_free\_path\_data.cartilage\_mfp}!module@\spxentry{module}}

\subsubsection{deepdrr.projector.mcgpu\_mean\_free\_path\_data.connective\_mfp}
\label{\detokenize{deepdrr.projector.mcgpu_mean_free_path_data:module-deepdrr.projector.mcgpu_mean_free_path_data.connective_mfp}}\label{\detokenize{deepdrr.projector.mcgpu_mean_free_path_data:deepdrr-projector-mcgpu-mean-free-path-data-connective-mfp}}\index{module@\spxentry{module}!deepdrr.projector.mcgpu\_mean\_free\_path\_data.connective\_mfp@\spxentry{deepdrr.projector.mcgpu\_mean\_free\_path\_data.connective\_mfp}}\index{deepdrr.projector.mcgpu\_mean\_free\_path\_data.connective\_mfp@\spxentry{deepdrr.projector.mcgpu\_mean\_free\_path\_data.connective\_mfp}!module@\spxentry{module}}

\subsubsection{deepdrr.projector.mcgpu\_mean\_free\_path\_data.glands\_others\_mfp}
\label{\detokenize{deepdrr.projector.mcgpu_mean_free_path_data:module-deepdrr.projector.mcgpu_mean_free_path_data.glands_others_mfp}}\label{\detokenize{deepdrr.projector.mcgpu_mean_free_path_data:deepdrr-projector-mcgpu-mean-free-path-data-glands-others-mfp}}\index{module@\spxentry{module}!deepdrr.projector.mcgpu\_mean\_free\_path\_data.glands\_others\_mfp@\spxentry{deepdrr.projector.mcgpu\_mean\_free\_path\_data.glands\_others\_mfp}}\index{deepdrr.projector.mcgpu\_mean\_free\_path\_data.glands\_others\_mfp@\spxentry{deepdrr.projector.mcgpu\_mean\_free\_path\_data.glands\_others\_mfp}!module@\spxentry{module}}

\subsubsection{deepdrr.projector.mcgpu\_mean\_free\_path\_data.liver\_mfp}
\label{\detokenize{deepdrr.projector.mcgpu_mean_free_path_data:module-deepdrr.projector.mcgpu_mean_free_path_data.liver_mfp}}\label{\detokenize{deepdrr.projector.mcgpu_mean_free_path_data:deepdrr-projector-mcgpu-mean-free-path-data-liver-mfp}}\index{module@\spxentry{module}!deepdrr.projector.mcgpu\_mean\_free\_path\_data.liver\_mfp@\spxentry{deepdrr.projector.mcgpu\_mean\_free\_path\_data.liver\_mfp}}\index{deepdrr.projector.mcgpu\_mean\_free\_path\_data.liver\_mfp@\spxentry{deepdrr.projector.mcgpu\_mean\_free\_path\_data.liver\_mfp}!module@\spxentry{module}}

\subsubsection{deepdrr.projector.mcgpu\_mean\_free\_path\_data.lung\_mfp}
\label{\detokenize{deepdrr.projector.mcgpu_mean_free_path_data:module-deepdrr.projector.mcgpu_mean_free_path_data.lung_mfp}}\label{\detokenize{deepdrr.projector.mcgpu_mean_free_path_data:deepdrr-projector-mcgpu-mean-free-path-data-lung-mfp}}\index{module@\spxentry{module}!deepdrr.projector.mcgpu\_mean\_free\_path\_data.lung\_mfp@\spxentry{deepdrr.projector.mcgpu\_mean\_free\_path\_data.lung\_mfp}}\index{deepdrr.projector.mcgpu\_mean\_free\_path\_data.lung\_mfp@\spxentry{deepdrr.projector.mcgpu\_mean\_free\_path\_data.lung\_mfp}!module@\spxentry{module}}

\subsubsection{deepdrr.projector.mcgpu\_mean\_free\_path\_data.muscle\_mfp}
\label{\detokenize{deepdrr.projector.mcgpu_mean_free_path_data:module-deepdrr.projector.mcgpu_mean_free_path_data.muscle_mfp}}\label{\detokenize{deepdrr.projector.mcgpu_mean_free_path_data:deepdrr-projector-mcgpu-mean-free-path-data-muscle-mfp}}\index{module@\spxentry{module}!deepdrr.projector.mcgpu\_mean\_free\_path\_data.muscle\_mfp@\spxentry{deepdrr.projector.mcgpu\_mean\_free\_path\_data.muscle\_mfp}}\index{deepdrr.projector.mcgpu\_mean\_free\_path\_data.muscle\_mfp@\spxentry{deepdrr.projector.mcgpu\_mean\_free\_path\_data.muscle\_mfp}!module@\spxentry{module}}

\subsubsection{deepdrr.projector.mcgpu\_mean\_free\_path\_data.red\_marrow\_mfp}
\label{\detokenize{deepdrr.projector.mcgpu_mean_free_path_data:module-deepdrr.projector.mcgpu_mean_free_path_data.red_marrow_mfp}}\label{\detokenize{deepdrr.projector.mcgpu_mean_free_path_data:deepdrr-projector-mcgpu-mean-free-path-data-red-marrow-mfp}}\index{module@\spxentry{module}!deepdrr.projector.mcgpu\_mean\_free\_path\_data.red\_marrow\_mfp@\spxentry{deepdrr.projector.mcgpu\_mean\_free\_path\_data.red\_marrow\_mfp}}\index{deepdrr.projector.mcgpu\_mean\_free\_path\_data.red\_marrow\_mfp@\spxentry{deepdrr.projector.mcgpu\_mean\_free\_path\_data.red\_marrow\_mfp}!module@\spxentry{module}}

\subsubsection{deepdrr.projector.mcgpu\_mean\_free\_path\_data.skin\_mfp}
\label{\detokenize{deepdrr.projector.mcgpu_mean_free_path_data:module-deepdrr.projector.mcgpu_mean_free_path_data.skin_mfp}}\label{\detokenize{deepdrr.projector.mcgpu_mean_free_path_data:deepdrr-projector-mcgpu-mean-free-path-data-skin-mfp}}\index{module@\spxentry{module}!deepdrr.projector.mcgpu\_mean\_free\_path\_data.skin\_mfp@\spxentry{deepdrr.projector.mcgpu\_mean\_free\_path\_data.skin\_mfp}}\index{deepdrr.projector.mcgpu\_mean\_free\_path\_data.skin\_mfp@\spxentry{deepdrr.projector.mcgpu\_mean\_free\_path\_data.skin\_mfp}!module@\spxentry{module}}

\subsubsection{deepdrr.projector.mcgpu\_mean\_free\_path\_data.soft\_tissue\_mfp}
\label{\detokenize{deepdrr.projector.mcgpu_mean_free_path_data:module-deepdrr.projector.mcgpu_mean_free_path_data.soft_tissue_mfp}}\label{\detokenize{deepdrr.projector.mcgpu_mean_free_path_data:deepdrr-projector-mcgpu-mean-free-path-data-soft-tissue-mfp}}\index{module@\spxentry{module}!deepdrr.projector.mcgpu\_mean\_free\_path\_data.soft\_tissue\_mfp@\spxentry{deepdrr.projector.mcgpu\_mean\_free\_path\_data.soft\_tissue\_mfp}}\index{deepdrr.projector.mcgpu\_mean\_free\_path\_data.soft\_tissue\_mfp@\spxentry{deepdrr.projector.mcgpu\_mean\_free\_path\_data.soft\_tissue\_mfp}!module@\spxentry{module}}

\subsubsection{deepdrr.projector.mcgpu\_mean\_free\_path\_data.stomach\_intestines\_mfp}
\label{\detokenize{deepdrr.projector.mcgpu_mean_free_path_data:module-deepdrr.projector.mcgpu_mean_free_path_data.stomach_intestines_mfp}}\label{\detokenize{deepdrr.projector.mcgpu_mean_free_path_data:deepdrr-projector-mcgpu-mean-free-path-data-stomach-intestines-mfp}}\index{module@\spxentry{module}!deepdrr.projector.mcgpu\_mean\_free\_path\_data.stomach\_intestines\_mfp@\spxentry{deepdrr.projector.mcgpu\_mean\_free\_path\_data.stomach\_intestines\_mfp}}\index{deepdrr.projector.mcgpu\_mean\_free\_path\_data.stomach\_intestines\_mfp@\spxentry{deepdrr.projector.mcgpu\_mean\_free\_path\_data.stomach\_intestines\_mfp}!module@\spxentry{module}}

\subsubsection{deepdrr.projector.mcgpu\_mean\_free\_path\_data.titanium\_mfp}
\label{\detokenize{deepdrr.projector.mcgpu_mean_free_path_data:module-deepdrr.projector.mcgpu_mean_free_path_data.titanium_mfp}}\label{\detokenize{deepdrr.projector.mcgpu_mean_free_path_data:deepdrr-projector-mcgpu-mean-free-path-data-titanium-mfp}}\index{module@\spxentry{module}!deepdrr.projector.mcgpu\_mean\_free\_path\_data.titanium\_mfp@\spxentry{deepdrr.projector.mcgpu\_mean\_free\_path\_data.titanium\_mfp}}\index{deepdrr.projector.mcgpu\_mean\_free\_path\_data.titanium\_mfp@\spxentry{deepdrr.projector.mcgpu\_mean\_free\_path\_data.titanium\_mfp}!module@\spxentry{module}}

\subsubsection{deepdrr.projector.mcgpu\_mean\_free\_path\_data.water\_mfp}
\label{\detokenize{deepdrr.projector.mcgpu_mean_free_path_data:module-deepdrr.projector.mcgpu_mean_free_path_data.water_mfp}}\label{\detokenize{deepdrr.projector.mcgpu_mean_free_path_data:deepdrr-projector-mcgpu-mean-free-path-data-water-mfp}}\index{module@\spxentry{module}!deepdrr.projector.mcgpu\_mean\_free\_path\_data.water\_mfp@\spxentry{deepdrr.projector.mcgpu\_mean\_free\_path\_data.water\_mfp}}\index{deepdrr.projector.mcgpu\_mean\_free\_path\_data.water\_mfp@\spxentry{deepdrr.projector.mcgpu\_mean\_free\_path\_data.water\_mfp}!module@\spxentry{module}}

\subsubsection{Module contents}
\label{\detokenize{deepdrr.projector.mcgpu_mean_free_path_data:module-deepdrr.projector.mcgpu_mean_free_path_data}}\label{\detokenize{deepdrr.projector.mcgpu_mean_free_path_data:module-contents}}\index{module@\spxentry{module}!deepdrr.projector.mcgpu\_mean\_free\_path\_data@\spxentry{deepdrr.projector.mcgpu\_mean\_free\_path\_data}}\index{deepdrr.projector.mcgpu\_mean\_free\_path\_data@\spxentry{deepdrr.projector.mcgpu\_mean\_free\_path\_data}!module@\spxentry{module}}
\sphinxstepscope


\subsection{deepdrr.projector.mcgpu\_rita\_params package}
\label{\detokenize{deepdrr.projector.mcgpu_rita_params:deepdrr-projector-mcgpu-rita-params-package}}\label{\detokenize{deepdrr.projector.mcgpu_rita_params::doc}}

\subsubsection{deepdrr.projector.mcgpu\_rita\_params.PMMA\_rita\_params module}
\label{\detokenize{deepdrr.projector.mcgpu_rita_params:module-deepdrr.projector.mcgpu_rita_params.PMMA_rita_params}}\label{\detokenize{deepdrr.projector.mcgpu_rita_params:deepdrr-projector-mcgpu-rita-params-pmma-rita-params-module}}\index{module@\spxentry{module}!deepdrr.projector.mcgpu\_rita\_params.PMMA\_rita\_params@\spxentry{deepdrr.projector.mcgpu\_rita\_params.PMMA\_rita\_params}}\index{deepdrr.projector.mcgpu\_rita\_params.PMMA\_rita\_params@\spxentry{deepdrr.projector.mcgpu\_rita\_params.PMMA\_rita\_params}!module@\spxentry{module}}

\subsubsection{deepdrr.projector.mcgpu\_rita\_params.adipose\_rita\_params module}
\label{\detokenize{deepdrr.projector.mcgpu_rita_params:module-deepdrr.projector.mcgpu_rita_params.adipose_rita_params}}\label{\detokenize{deepdrr.projector.mcgpu_rita_params:deepdrr-projector-mcgpu-rita-params-adipose-rita-params-module}}\index{module@\spxentry{module}!deepdrr.projector.mcgpu\_rita\_params.adipose\_rita\_params@\spxentry{deepdrr.projector.mcgpu\_rita\_params.adipose\_rita\_params}}\index{deepdrr.projector.mcgpu\_rita\_params.adipose\_rita\_params@\spxentry{deepdrr.projector.mcgpu\_rita\_params.adipose\_rita\_params}!module@\spxentry{module}}

\subsubsection{deepdrr.projector.mcgpu\_rita\_params.air\_rita\_params module}
\label{\detokenize{deepdrr.projector.mcgpu_rita_params:module-deepdrr.projector.mcgpu_rita_params.air_rita_params}}\label{\detokenize{deepdrr.projector.mcgpu_rita_params:deepdrr-projector-mcgpu-rita-params-air-rita-params-module}}\index{module@\spxentry{module}!deepdrr.projector.mcgpu\_rita\_params.air\_rita\_params@\spxentry{deepdrr.projector.mcgpu\_rita\_params.air\_rita\_params}}\index{deepdrr.projector.mcgpu\_rita\_params.air\_rita\_params@\spxentry{deepdrr.projector.mcgpu\_rita\_params.air\_rita\_params}!module@\spxentry{module}}

\subsubsection{deepdrr.projector.mcgpu\_rita\_params.blood\_rita\_params module}
\label{\detokenize{deepdrr.projector.mcgpu_rita_params:module-deepdrr.projector.mcgpu_rita_params.blood_rita_params}}\label{\detokenize{deepdrr.projector.mcgpu_rita_params:deepdrr-projector-mcgpu-rita-params-blood-rita-params-module}}\index{module@\spxentry{module}!deepdrr.projector.mcgpu\_rita\_params.blood\_rita\_params@\spxentry{deepdrr.projector.mcgpu\_rita\_params.blood\_rita\_params}}\index{deepdrr.projector.mcgpu\_rita\_params.blood\_rita\_params@\spxentry{deepdrr.projector.mcgpu\_rita\_params.blood\_rita\_params}!module@\spxentry{module}}

\subsubsection{deepdrr.projector.mcgpu\_rita\_params.bone\_rita\_params module}
\label{\detokenize{deepdrr.projector.mcgpu_rita_params:module-deepdrr.projector.mcgpu_rita_params.bone_rita_params}}\label{\detokenize{deepdrr.projector.mcgpu_rita_params:deepdrr-projector-mcgpu-rita-params-bone-rita-params-module}}\index{module@\spxentry{module}!deepdrr.projector.mcgpu\_rita\_params.bone\_rita\_params@\spxentry{deepdrr.projector.mcgpu\_rita\_params.bone\_rita\_params}}\index{deepdrr.projector.mcgpu\_rita\_params.bone\_rita\_params@\spxentry{deepdrr.projector.mcgpu\_rita\_params.bone\_rita\_params}!module@\spxentry{module}}

\subsubsection{deepdrr.projector.mcgpu\_rita\_params.brain\_rita\_params module}
\label{\detokenize{deepdrr.projector.mcgpu_rita_params:module-deepdrr.projector.mcgpu_rita_params.brain_rita_params}}\label{\detokenize{deepdrr.projector.mcgpu_rita_params:deepdrr-projector-mcgpu-rita-params-brain-rita-params-module}}\index{module@\spxentry{module}!deepdrr.projector.mcgpu\_rita\_params.brain\_rita\_params@\spxentry{deepdrr.projector.mcgpu\_rita\_params.brain\_rita\_params}}\index{deepdrr.projector.mcgpu\_rita\_params.brain\_rita\_params@\spxentry{deepdrr.projector.mcgpu\_rita\_params.brain\_rita\_params}!module@\spxentry{module}}

\subsubsection{deepdrr.projector.mcgpu\_rita\_params.breast\_rita\_params module}
\label{\detokenize{deepdrr.projector.mcgpu_rita_params:module-deepdrr.projector.mcgpu_rita_params.breast_rita_params}}\label{\detokenize{deepdrr.projector.mcgpu_rita_params:deepdrr-projector-mcgpu-rita-params-breast-rita-params-module}}\index{module@\spxentry{module}!deepdrr.projector.mcgpu\_rita\_params.breast\_rita\_params@\spxentry{deepdrr.projector.mcgpu\_rita\_params.breast\_rita\_params}}\index{deepdrr.projector.mcgpu\_rita\_params.breast\_rita\_params@\spxentry{deepdrr.projector.mcgpu\_rita\_params.breast\_rita\_params}!module@\spxentry{module}}

\subsubsection{deepdrr.projector.mcgpu\_rita\_params.cartilage\_rita\_params module}
\label{\detokenize{deepdrr.projector.mcgpu_rita_params:module-deepdrr.projector.mcgpu_rita_params.cartilage_rita_params}}\label{\detokenize{deepdrr.projector.mcgpu_rita_params:deepdrr-projector-mcgpu-rita-params-cartilage-rita-params-module}}\index{module@\spxentry{module}!deepdrr.projector.mcgpu\_rita\_params.cartilage\_rita\_params@\spxentry{deepdrr.projector.mcgpu\_rita\_params.cartilage\_rita\_params}}\index{deepdrr.projector.mcgpu\_rita\_params.cartilage\_rita\_params@\spxentry{deepdrr.projector.mcgpu\_rita\_params.cartilage\_rita\_params}!module@\spxentry{module}}

\subsubsection{deepdrr.projector.mcgpu\_rita\_params.connective\_rita\_params module}
\label{\detokenize{deepdrr.projector.mcgpu_rita_params:module-deepdrr.projector.mcgpu_rita_params.connective_rita_params}}\label{\detokenize{deepdrr.projector.mcgpu_rita_params:deepdrr-projector-mcgpu-rita-params-connective-rita-params-module}}\index{module@\spxentry{module}!deepdrr.projector.mcgpu\_rita\_params.connective\_rita\_params@\spxentry{deepdrr.projector.mcgpu\_rita\_params.connective\_rita\_params}}\index{deepdrr.projector.mcgpu\_rita\_params.connective\_rita\_params@\spxentry{deepdrr.projector.mcgpu\_rita\_params.connective\_rita\_params}!module@\spxentry{module}}

\subsubsection{deepdrr.projector.mcgpu\_rita\_params.glands\_others\_rita\_params module}
\label{\detokenize{deepdrr.projector.mcgpu_rita_params:module-deepdrr.projector.mcgpu_rita_params.glands_others_rita_params}}\label{\detokenize{deepdrr.projector.mcgpu_rita_params:deepdrr-projector-mcgpu-rita-params-glands-others-rita-params-module}}\index{module@\spxentry{module}!deepdrr.projector.mcgpu\_rita\_params.glands\_others\_rita\_params@\spxentry{deepdrr.projector.mcgpu\_rita\_params.glands\_others\_rita\_params}}\index{deepdrr.projector.mcgpu\_rita\_params.glands\_others\_rita\_params@\spxentry{deepdrr.projector.mcgpu\_rita\_params.glands\_others\_rita\_params}!module@\spxentry{module}}

\subsubsection{deepdrr.projector.mcgpu\_rita\_params.liver\_rita\_params module}
\label{\detokenize{deepdrr.projector.mcgpu_rita_params:module-deepdrr.projector.mcgpu_rita_params.liver_rita_params}}\label{\detokenize{deepdrr.projector.mcgpu_rita_params:deepdrr-projector-mcgpu-rita-params-liver-rita-params-module}}\index{module@\spxentry{module}!deepdrr.projector.mcgpu\_rita\_params.liver\_rita\_params@\spxentry{deepdrr.projector.mcgpu\_rita\_params.liver\_rita\_params}}\index{deepdrr.projector.mcgpu\_rita\_params.liver\_rita\_params@\spxentry{deepdrr.projector.mcgpu\_rita\_params.liver\_rita\_params}!module@\spxentry{module}}

\subsubsection{deepdrr.projector.mcgpu\_rita\_params.lung\_rita\_params module}
\label{\detokenize{deepdrr.projector.mcgpu_rita_params:module-deepdrr.projector.mcgpu_rita_params.lung_rita_params}}\label{\detokenize{deepdrr.projector.mcgpu_rita_params:deepdrr-projector-mcgpu-rita-params-lung-rita-params-module}}\index{module@\spxentry{module}!deepdrr.projector.mcgpu\_rita\_params.lung\_rita\_params@\spxentry{deepdrr.projector.mcgpu\_rita\_params.lung\_rita\_params}}\index{deepdrr.projector.mcgpu\_rita\_params.lung\_rita\_params@\spxentry{deepdrr.projector.mcgpu\_rita\_params.lung\_rita\_params}!module@\spxentry{module}}

\subsubsection{deepdrr.projector.mcgpu\_rita\_params.muscle\_rita\_params module}
\label{\detokenize{deepdrr.projector.mcgpu_rita_params:module-deepdrr.projector.mcgpu_rita_params.muscle_rita_params}}\label{\detokenize{deepdrr.projector.mcgpu_rita_params:deepdrr-projector-mcgpu-rita-params-muscle-rita-params-module}}\index{module@\spxentry{module}!deepdrr.projector.mcgpu\_rita\_params.muscle\_rita\_params@\spxentry{deepdrr.projector.mcgpu\_rita\_params.muscle\_rita\_params}}\index{deepdrr.projector.mcgpu\_rita\_params.muscle\_rita\_params@\spxentry{deepdrr.projector.mcgpu\_rita\_params.muscle\_rita\_params}!module@\spxentry{module}}

\subsubsection{deepdrr.projector.mcgpu\_rita\_params.red\_marrow\_rita\_params module}
\label{\detokenize{deepdrr.projector.mcgpu_rita_params:module-deepdrr.projector.mcgpu_rita_params.red_marrow_rita_params}}\label{\detokenize{deepdrr.projector.mcgpu_rita_params:deepdrr-projector-mcgpu-rita-params-red-marrow-rita-params-module}}\index{module@\spxentry{module}!deepdrr.projector.mcgpu\_rita\_params.red\_marrow\_rita\_params@\spxentry{deepdrr.projector.mcgpu\_rita\_params.red\_marrow\_rita\_params}}\index{deepdrr.projector.mcgpu\_rita\_params.red\_marrow\_rita\_params@\spxentry{deepdrr.projector.mcgpu\_rita\_params.red\_marrow\_rita\_params}!module@\spxentry{module}}

\subsubsection{deepdrr.projector.mcgpu\_rita\_params.skin\_rita\_params module}
\label{\detokenize{deepdrr.projector.mcgpu_rita_params:module-deepdrr.projector.mcgpu_rita_params.skin_rita_params}}\label{\detokenize{deepdrr.projector.mcgpu_rita_params:deepdrr-projector-mcgpu-rita-params-skin-rita-params-module}}\index{module@\spxentry{module}!deepdrr.projector.mcgpu\_rita\_params.skin\_rita\_params@\spxentry{deepdrr.projector.mcgpu\_rita\_params.skin\_rita\_params}}\index{deepdrr.projector.mcgpu\_rita\_params.skin\_rita\_params@\spxentry{deepdrr.projector.mcgpu\_rita\_params.skin\_rita\_params}!module@\spxentry{module}}

\subsubsection{deepdrr.projector.mcgpu\_rita\_params.soft\_tissue\_rita\_params module}
\label{\detokenize{deepdrr.projector.mcgpu_rita_params:module-deepdrr.projector.mcgpu_rita_params.soft_tissue_rita_params}}\label{\detokenize{deepdrr.projector.mcgpu_rita_params:deepdrr-projector-mcgpu-rita-params-soft-tissue-rita-params-module}}\index{module@\spxentry{module}!deepdrr.projector.mcgpu\_rita\_params.soft\_tissue\_rita\_params@\spxentry{deepdrr.projector.mcgpu\_rita\_params.soft\_tissue\_rita\_params}}\index{deepdrr.projector.mcgpu\_rita\_params.soft\_tissue\_rita\_params@\spxentry{deepdrr.projector.mcgpu\_rita\_params.soft\_tissue\_rita\_params}!module@\spxentry{module}}

\subsubsection{deepdrr.projector.mcgpu\_rita\_params.stomach\_intestines\_rita\_params module}
\label{\detokenize{deepdrr.projector.mcgpu_rita_params:module-deepdrr.projector.mcgpu_rita_params.stomach_intestines_rita_params}}\label{\detokenize{deepdrr.projector.mcgpu_rita_params:deepdrr-projector-mcgpu-rita-params-stomach-intestines-rita-params-module}}\index{module@\spxentry{module}!deepdrr.projector.mcgpu\_rita\_params.stomach\_intestines\_rita\_params@\spxentry{deepdrr.projector.mcgpu\_rita\_params.stomach\_intestines\_rita\_params}}\index{deepdrr.projector.mcgpu\_rita\_params.stomach\_intestines\_rita\_params@\spxentry{deepdrr.projector.mcgpu\_rita\_params.stomach\_intestines\_rita\_params}!module@\spxentry{module}}

\subsubsection{deepdrr.projector.mcgpu\_rita\_params.titanium\_rita\_params module}
\label{\detokenize{deepdrr.projector.mcgpu_rita_params:module-deepdrr.projector.mcgpu_rita_params.titanium_rita_params}}\label{\detokenize{deepdrr.projector.mcgpu_rita_params:deepdrr-projector-mcgpu-rita-params-titanium-rita-params-module}}\index{module@\spxentry{module}!deepdrr.projector.mcgpu\_rita\_params.titanium\_rita\_params@\spxentry{deepdrr.projector.mcgpu\_rita\_params.titanium\_rita\_params}}\index{deepdrr.projector.mcgpu\_rita\_params.titanium\_rita\_params@\spxentry{deepdrr.projector.mcgpu\_rita\_params.titanium\_rita\_params}!module@\spxentry{module}}

\subsubsection{deepdrr.projector.mcgpu\_rita\_params.water\_rita\_params module}
\label{\detokenize{deepdrr.projector.mcgpu_rita_params:module-deepdrr.projector.mcgpu_rita_params.water_rita_params}}\label{\detokenize{deepdrr.projector.mcgpu_rita_params:deepdrr-projector-mcgpu-rita-params-water-rita-params-module}}\index{module@\spxentry{module}!deepdrr.projector.mcgpu\_rita\_params.water\_rita\_params@\spxentry{deepdrr.projector.mcgpu\_rita\_params.water\_rita\_params}}\index{deepdrr.projector.mcgpu\_rita\_params.water\_rita\_params@\spxentry{deepdrr.projector.mcgpu\_rita\_params.water\_rita\_params}!module@\spxentry{module}}

\subsubsection{Module contents}
\label{\detokenize{deepdrr.projector.mcgpu_rita_params:module-deepdrr.projector.mcgpu_rita_params}}\label{\detokenize{deepdrr.projector.mcgpu_rita_params:module-contents}}\index{module@\spxentry{module}!deepdrr.projector.mcgpu\_rita\_params@\spxentry{deepdrr.projector.mcgpu\_rita\_params}}\index{deepdrr.projector.mcgpu\_rita\_params@\spxentry{deepdrr.projector.mcgpu\_rita\_params}!module@\spxentry{module}}

\subsection{Submodules}
\label{\detokenize{deepdrr.projector:submodules}}

\subsection{deepdrr.projector.analytic\_generators module}
\label{\detokenize{deepdrr.projector:module-deepdrr.projector.analytic_generators}}\label{\detokenize{deepdrr.projector:deepdrr-projector-analytic-generators-module}}\index{module@\spxentry{module}!deepdrr.projector.analytic\_generators@\spxentry{deepdrr.projector.analytic\_generators}}\index{deepdrr.projector.analytic\_generators@\spxentry{deepdrr.projector.analytic\_generators}!module@\spxentry{module}}\index{add\_noise() (in module deepdrr.projector.analytic\_generators)@\spxentry{add\_noise()}\spxextra{in module deepdrr.projector.analytic\_generators}}

\begin{fulllineitems}
\phantomsection\label{\detokenize{deepdrr.projector:deepdrr.projector.analytic_generators.add_noise}}
\pysigstartsignatures
\pysiglinewithargsret{\sphinxcode{\sphinxupquote{deepdrr.projector.analytic\_generators.}}\sphinxbfcode{\sphinxupquote{add\_noise}}}{\sphinxparam{\DUrole{n,n}{input\_image}}\sphinxparamcomma \sphinxparam{\DUrole{n,n}{photon\_prob}}\sphinxparamcomma \sphinxparam{\DUrole{n,n}{photon\_number}}}{}
\pysigstopsignatures
\end{fulllineitems}



\subsection{deepdrr.projector.conv\_to\_mcgpu module}
\label{\detokenize{deepdrr.projector:module-deepdrr.projector.conv_to_mcgpu}}\label{\detokenize{deepdrr.projector:deepdrr-projector-conv-to-mcgpu-module}}\index{module@\spxentry{module}!deepdrr.projector.conv\_to\_mcgpu@\spxentry{deepdrr.projector.conv\_to\_mcgpu}}\index{deepdrr.projector.conv\_to\_mcgpu@\spxentry{deepdrr.projector.conv\_to\_mcgpu}!module@\spxentry{module}}\index{get\_mat\_filename() (in module deepdrr.projector.conv\_to\_mcgpu)@\spxentry{get\_mat\_filename()}\spxextra{in module deepdrr.projector.conv\_to\_mcgpu}}

\begin{fulllineitems}
\phantomsection\label{\detokenize{deepdrr.projector:deepdrr.projector.conv_to_mcgpu.get_mat_filename}}
\pysigstartsignatures
\pysiglinewithargsret{\sphinxcode{\sphinxupquote{deepdrr.projector.conv\_to\_mcgpu.}}\sphinxbfcode{\sphinxupquote{get\_mat\_filename}}}{\sphinxparam{\DUrole{n,n}{deepDRR\_mat\_name}\DUrole{p,p}{:}\DUrole{w,w}{  }\DUrole{n,n}{str}}}{{ $\rightarrow$ str}}
\pysigstopsignatures
\sphinxAtStartPar
Material names are those from the dictionary in material\_coefficients.py file

\end{fulllineitems}

\index{make\_mcgpu\_inputs() (in module deepdrr.projector.conv\_to\_mcgpu)@\spxentry{make\_mcgpu\_inputs()}\spxextra{in module deepdrr.projector.conv\_to\_mcgpu}}

\begin{fulllineitems}
\phantomsection\label{\detokenize{deepdrr.projector:deepdrr.projector.conv_to_mcgpu.make_mcgpu_inputs}}
\pysigstartsignatures
\pysiglinewithargsret{\sphinxcode{\sphinxupquote{deepdrr.projector.conv\_to\_mcgpu.}}\sphinxbfcode{\sphinxupquote{make\_mcgpu\_inputs}}}{\sphinxparam{\DUrole{n,n}{geom}\DUrole{p,p}{:}\DUrole{w,w}{  }\DUrole{n,n}{{\hyperref[\detokenize{deepdrr.vol:deepdrr.vol.volume.Volume}]{\sphinxcrossref{Volume}}}}}\sphinxparamcomma \sphinxparam{\DUrole{n,n}{filename}\DUrole{p,p}{:}\DUrole{w,w}{  }\DUrole{n,n}{str}}\sphinxparamcomma \sphinxparam{\DUrole{n,n}{target\_dir}\DUrole{p,p}{:}\DUrole{w,w}{  }\DUrole{n,n}{Path}}\sphinxparamcomma \sphinxparam{\DUrole{n,n}{histories}\DUrole{p,p}{:}\DUrole{w,w}{  }\DUrole{n,n}{int}}\sphinxparamcomma \sphinxparam{\DUrole{n,n}{seed}\DUrole{p,p}{:}\DUrole{w,w}{  }\DUrole{n,n}{int}}\sphinxparamcomma \sphinxparam{\DUrole{n,n}{threads\_per\_block}\DUrole{p,p}{:}\DUrole{w,w}{  }\DUrole{n,n}{int}}\sphinxparamcomma \sphinxparam{\DUrole{n,n}{histories\_per\_thread}\DUrole{p,p}{:}\DUrole{w,w}{  }\DUrole{n,n}{int}}\sphinxparamcomma \sphinxparam{\DUrole{n,n}{spectrum}\DUrole{p,p}{:}\DUrole{w,w}{  }\DUrole{n,n}{str}}\sphinxparamcomma \sphinxparam{\DUrole{n,n}{source\_xyz\_cm}\DUrole{p,p}{:}\DUrole{w,w}{  }\DUrole{n,n}{ndarray}}\sphinxparamcomma \sphinxparam{\DUrole{n,n}{source\_direction}\DUrole{p,p}{:}\DUrole{w,w}{  }\DUrole{n,n}{ndarray}}\sphinxparamcomma \sphinxparam{\DUrole{n,n}{detector\_pixels}\DUrole{p,p}{:}\DUrole{w,w}{  }\DUrole{n,n}{Tuple\DUrole{p,p}{{[}}int\DUrole{p,p}{,}\DUrole{w,w}{  }int\DUrole{p,p}{{]}}}}\sphinxparamcomma \sphinxparam{\DUrole{n,n}{detector\_size\_cm}\DUrole{p,p}{:}\DUrole{w,w}{  }\DUrole{n,n}{Tuple\DUrole{p,p}{{[}}float\DUrole{p,p}{,}\DUrole{w,w}{  }float\DUrole{p,p}{{]}}}}\sphinxparamcomma \sphinxparam{\DUrole{n,n}{source\_to\_detector\_distance\_cm}\DUrole{p,p}{:}\DUrole{w,w}{  }\DUrole{n,n}{float}}\sphinxparamcomma \sphinxparam{\DUrole{n,n}{source\_to\_isocenter\_distance\_cm}\DUrole{p,p}{:}\DUrole{w,w}{  }\DUrole{n,n}{float}}}{{ $\rightarrow$ None}}
\pysigstopsignatures
\sphinxAtStartPar
Creates multiple files to serve as the inputs

\end{fulllineitems}



\subsection{deepdrr.projector.cuda\_scatter\_structs module}
\label{\detokenize{deepdrr.projector:module-deepdrr.projector.cuda_scatter_structs}}\label{\detokenize{deepdrr.projector:deepdrr-projector-cuda-scatter-structs-module}}\index{module@\spxentry{module}!deepdrr.projector.cuda\_scatter\_structs@\spxentry{deepdrr.projector.cuda\_scatter\_structs}}\index{deepdrr.projector.cuda\_scatter\_structs@\spxentry{deepdrr.projector.cuda\_scatter\_structs}!module@\spxentry{module}}\index{CudaComptonStruct (class in deepdrr.projector.cuda\_scatter\_structs)@\spxentry{CudaComptonStruct}\spxextra{class in deepdrr.projector.cuda\_scatter\_structs}}

\begin{fulllineitems}
\phantomsection\label{\detokenize{deepdrr.projector:deepdrr.projector.cuda_scatter_structs.CudaComptonStruct}}
\pysigstartsignatures
\pysiglinewithargsret{\sphinxbfcode{\sphinxupquote{class\DUrole{w,w}{  }}}\sphinxcode{\sphinxupquote{deepdrr.projector.cuda\_scatter\_structs.}}\sphinxbfcode{\sphinxupquote{CudaComptonStruct}}}{\sphinxparam{\DUrole{n,n}{compton\_arr}\DUrole{p,p}{:}\DUrole{w,w}{  }\DUrole{n,n}{ndarray}}\sphinxparamcomma \sphinxparam{\DUrole{n,n}{struct\_gpu\_ptr}}}{}
\pysigstopsignatures
\sphinxAtStartPar
Bases: \sphinxcode{\sphinxupquote{object}}
\index{MEMSIZE (deepdrr.projector.cuda\_scatter\_structs.CudaComptonStruct attribute)@\spxentry{MEMSIZE}\spxextra{deepdrr.projector.cuda\_scatter\_structs.CudaComptonStruct attribute}}

\begin{fulllineitems}
\phantomsection\label{\detokenize{deepdrr.projector:deepdrr.projector.cuda_scatter_structs.CudaComptonStruct.MEMSIZE}}
\pysigstartsignatures
\pysigline{\sphinxbfcode{\sphinxupquote{MEMSIZE}}\sphinxbfcode{\sphinxupquote{\DUrole{w,w}{  }\DUrole{p,p}{=}\DUrole{w,w}{  }364}}}
\pysigstopsignatures
\end{fulllineitems}


\end{fulllineitems}

\index{CudaMatMfpStruct (class in deepdrr.projector.cuda\_scatter\_structs)@\spxentry{CudaMatMfpStruct}\spxextra{class in deepdrr.projector.cuda\_scatter\_structs}}

\begin{fulllineitems}
\phantomsection\label{\detokenize{deepdrr.projector:deepdrr.projector.cuda_scatter_structs.CudaMatMfpStruct}}
\pysigstartsignatures
\pysiglinewithargsret{\sphinxbfcode{\sphinxupquote{class\DUrole{w,w}{  }}}\sphinxcode{\sphinxupquote{deepdrr.projector.cuda\_scatter\_structs.}}\sphinxbfcode{\sphinxupquote{CudaMatMfpStruct}}}{\sphinxparam{\DUrole{n,n}{mfp\_arr}\DUrole{p,p}{:}\DUrole{w,w}{  }\DUrole{n,n}{ndarray}}\sphinxparamcomma \sphinxparam{\DUrole{n,n}{struct\_gpu\_ptr}}}{}
\pysigstopsignatures
\sphinxAtStartPar
Bases: \sphinxcode{\sphinxupquote{object}}
\index{MEMSIZE (deepdrr.projector.cuda\_scatter\_structs.CudaMatMfpStruct attribute)@\spxentry{MEMSIZE}\spxextra{deepdrr.projector.cuda\_scatter\_structs.CudaMatMfpStruct attribute}}

\begin{fulllineitems}
\phantomsection\label{\detokenize{deepdrr.projector:deepdrr.projector.cuda_scatter_structs.CudaMatMfpStruct.MEMSIZE}}
\pysigstartsignatures
\pysigline{\sphinxbfcode{\sphinxupquote{MEMSIZE}}\sphinxbfcode{\sphinxupquote{\DUrole{w,w}{  }\DUrole{p,p}{=}\DUrole{w,w}{  }400084}}}
\pysigstopsignatures
\end{fulllineitems}


\end{fulllineitems}

\index{CudaPlaneSurfaceStruct (class in deepdrr.projector.cuda\_scatter\_structs)@\spxentry{CudaPlaneSurfaceStruct}\spxextra{class in deepdrr.projector.cuda\_scatter\_structs}}

\begin{fulllineitems}
\phantomsection\label{\detokenize{deepdrr.projector:deepdrr.projector.cuda_scatter_structs.CudaPlaneSurfaceStruct}}
\pysigstartsignatures
\pysiglinewithargsret{\sphinxbfcode{\sphinxupquote{class\DUrole{w,w}{  }}}\sphinxcode{\sphinxupquote{deepdrr.projector.cuda\_scatter\_structs.}}\sphinxbfcode{\sphinxupquote{CudaPlaneSurfaceStruct}}}{\sphinxparam{\DUrole{n,n}{psurf}\DUrole{p,p}{:}\DUrole{w,w}{  }\DUrole{n,n}{{\hyperref[\detokenize{deepdrr.projector:deepdrr.projector.plane_surface.PlaneSurface}]{\sphinxcrossref{PlaneSurface}}}}}\sphinxparamcomma \sphinxparam{\DUrole{n,n}{struct\_gpu\_ptr}}}{}
\pysigstopsignatures
\sphinxAtStartPar
Bases: \sphinxcode{\sphinxupquote{object}}
\index{MEMSIZE (deepdrr.projector.cuda\_scatter\_structs.CudaPlaneSurfaceStruct attribute)@\spxentry{MEMSIZE}\spxextra{deepdrr.projector.cuda\_scatter\_structs.CudaPlaneSurfaceStruct attribute}}

\begin{fulllineitems}
\phantomsection\label{\detokenize{deepdrr.projector:deepdrr.projector.cuda_scatter_structs.CudaPlaneSurfaceStruct.MEMSIZE}}
\pysigstartsignatures
\pysigline{\sphinxbfcode{\sphinxupquote{MEMSIZE}}\sphinxbfcode{\sphinxupquote{\DUrole{w,w}{  }\DUrole{p,p}{=}\DUrole{w,w}{  }80}}}
\pysigstopsignatures
\end{fulllineitems}


\end{fulllineitems}

\index{CudaRayleighStruct (class in deepdrr.projector.cuda\_scatter\_structs)@\spxentry{CudaRayleighStruct}\spxextra{class in deepdrr.projector.cuda\_scatter\_structs}}

\begin{fulllineitems}
\phantomsection\label{\detokenize{deepdrr.projector:deepdrr.projector.cuda_scatter_structs.CudaRayleighStruct}}
\pysigstartsignatures
\pysiglinewithargsret{\sphinxbfcode{\sphinxupquote{class\DUrole{w,w}{  }}}\sphinxcode{\sphinxupquote{deepdrr.projector.cuda\_scatter\_structs.}}\sphinxbfcode{\sphinxupquote{CudaRayleighStruct}}}{\sphinxparam{\DUrole{n,n}{rita\_obj}\DUrole{p,p}{:}\DUrole{w,w}{  }\DUrole{n,n}{{\hyperref[\detokenize{deepdrr.projector:deepdrr.projector.rita.RITA}]{\sphinxcrossref{RITA}}}}}\sphinxparamcomma \sphinxparam{\DUrole{n,n}{mat\_name}\DUrole{p,p}{:}\DUrole{w,w}{  }\DUrole{n,n}{str}}\sphinxparamcomma \sphinxparam{\DUrole{n,n}{struct\_gpu\_ptr}}}{}
\pysigstopsignatures
\sphinxAtStartPar
Bases: \sphinxcode{\sphinxupquote{object}}
\index{MEMSIZE (deepdrr.projector.cuda\_scatter\_structs.CudaRayleighStruct attribute)@\spxentry{MEMSIZE}\spxextra{deepdrr.projector.cuda\_scatter\_structs.CudaRayleighStruct attribute}}

\begin{fulllineitems}
\phantomsection\label{\detokenize{deepdrr.projector:deepdrr.projector.cuda_scatter_structs.CudaRayleighStruct.MEMSIZE}}
\pysigstartsignatures
\pysigline{\sphinxbfcode{\sphinxupquote{MEMSIZE}}\sphinxbfcode{\sphinxupquote{\DUrole{w,w}{  }\DUrole{p,p}{=}\DUrole{w,w}{  }104120}}}
\pysigstopsignatures
\end{fulllineitems}


\end{fulllineitems}

\index{CudaWoodcockStruct (class in deepdrr.projector.cuda\_scatter\_structs)@\spxentry{CudaWoodcockStruct}\spxextra{class in deepdrr.projector.cuda\_scatter\_structs}}

\begin{fulllineitems}
\phantomsection\label{\detokenize{deepdrr.projector:deepdrr.projector.cuda_scatter_structs.CudaWoodcockStruct}}
\pysigstartsignatures
\pysiglinewithargsret{\sphinxbfcode{\sphinxupquote{class\DUrole{w,w}{  }}}\sphinxcode{\sphinxupquote{deepdrr.projector.cuda\_scatter\_structs.}}\sphinxbfcode{\sphinxupquote{CudaWoodcockStruct}}}{\sphinxparam{\DUrole{n,n}{mfp\_arr}\DUrole{p,p}{:}\DUrole{w,w}{  }\DUrole{n,n}{ndarray}}\sphinxparamcomma \sphinxparam{\DUrole{n,n}{struct\_gpu\_ptr}}}{}
\pysigstopsignatures
\sphinxAtStartPar
Bases: \sphinxcode{\sphinxupquote{object}}
\index{MEMSIZE (deepdrr.projector.cuda\_scatter\_structs.CudaWoodcockStruct attribute)@\spxentry{MEMSIZE}\spxextra{deepdrr.projector.cuda\_scatter\_structs.CudaWoodcockStruct attribute}}

\begin{fulllineitems}
\phantomsection\label{\detokenize{deepdrr.projector:deepdrr.projector.cuda_scatter_structs.CudaWoodcockStruct.MEMSIZE}}
\pysigstartsignatures
\pysigline{\sphinxbfcode{\sphinxupquote{MEMSIZE}}\sphinxbfcode{\sphinxupquote{\DUrole{w,w}{  }\DUrole{p,p}{=}\DUrole{w,w}{  }200044}}}
\pysigstopsignatures
\end{fulllineitems}


\end{fulllineitems}



\subsection{deepdrr.projector.mass\_attenuation module}
\label{\detokenize{deepdrr.projector:module-deepdrr.projector.mass_attenuation}}\label{\detokenize{deepdrr.projector:deepdrr-projector-mass-attenuation-module}}\index{module@\spxentry{module}!deepdrr.projector.mass\_attenuation@\spxentry{deepdrr.projector.mass\_attenuation}}\index{deepdrr.projector.mass\_attenuation@\spxentry{deepdrr.projector.mass\_attenuation}!module@\spxentry{module}}\index{calculate\_attenuation\_gpu() (in module deepdrr.projector.mass\_attenuation)@\spxentry{calculate\_attenuation\_gpu()}\spxextra{in module deepdrr.projector.mass\_attenuation}}

\begin{fulllineitems}
\phantomsection\label{\detokenize{deepdrr.projector:deepdrr.projector.mass_attenuation.calculate_attenuation_gpu}}
\pysigstartsignatures
\pysiglinewithargsret{\sphinxcode{\sphinxupquote{deepdrr.projector.mass\_attenuation.}}\sphinxbfcode{\sphinxupquote{calculate\_attenuation\_gpu}}}{\sphinxparam{\DUrole{n,n}{projections\_gpu}}\sphinxparamcomma \sphinxparam{\DUrole{n,n}{energy}}\sphinxparamcomma \sphinxparam{\DUrole{n,n}{p}}\sphinxparamcomma \sphinxparam{\DUrole{n,n}{pool}}}{}
\pysigstopsignatures
\end{fulllineitems}

\index{calculate\_intensity\_from\_spectrum() (in module deepdrr.projector.mass\_attenuation)@\spxentry{calculate\_intensity\_from\_spectrum()}\spxextra{in module deepdrr.projector.mass\_attenuation}}

\begin{fulllineitems}
\phantomsection\label{\detokenize{deepdrr.projector:deepdrr.projector.mass_attenuation.calculate_intensity_from_spectrum}}
\pysigstartsignatures
\pysiglinewithargsret{\sphinxcode{\sphinxupquote{deepdrr.projector.mass\_attenuation.}}\sphinxbfcode{\sphinxupquote{calculate\_intensity\_from\_spectrum}}}{\sphinxparam{\DUrole{n,n}{projections}}\sphinxparamcomma \sphinxparam{\DUrole{n,n}{spectrum}}\sphinxparamcomma \sphinxparam{\DUrole{n,n}{blocksize}\DUrole{o,o}{=}\DUrole{default_value}{50}}}{}
\pysigstopsignatures
\end{fulllineitems}

\index{get\_absorption\_coefs() (in module deepdrr.projector.mass\_attenuation)@\spxentry{get\_absorption\_coefs()}\spxextra{in module deepdrr.projector.mass\_attenuation}}

\begin{fulllineitems}
\phantomsection\label{\detokenize{deepdrr.projector:deepdrr.projector.mass_attenuation.get_absorption_coefs}}
\pysigstartsignatures
\pysiglinewithargsret{\sphinxcode{\sphinxupquote{deepdrr.projector.mass\_attenuation.}}\sphinxbfcode{\sphinxupquote{get\_absorption\_coefs}}}{\sphinxparam{\DUrole{n,n}{energy\_keV}}\sphinxparamcomma \sphinxparam{\DUrole{n,n}{material}}}{}
\pysigstopsignatures
\sphinxAtStartPar
Returns the absorption coefficient for the specified material at the specified energy level (in keV)
\begin{quote}\begin{description}
\sphinxlineitem{Parameters}\begin{itemize}
\item {} 
\sphinxAtStartPar
\sphinxstyleliteralstrong{\sphinxupquote{energy\_keV}} \textendash{} energy level of photon/ray (keV)

\item {} 
\sphinxAtStartPar
\sphinxstyleliteralstrong{\sphinxupquote{material}} (\sphinxstyleliteralemphasis{\sphinxupquote{str}}) \textendash{} the material

\end{itemize}

\sphinxlineitem{Returns}
\sphinxAtStartPar
the absorption coefficient (in {[}cm\textasciicircum{}2 / g{]}), interpolated from the data in material\_coefficients.py

\end{description}\end{quote}

\end{fulllineitems}

\index{log\_interp() (in module deepdrr.projector.mass\_attenuation)@\spxentry{log\_interp()}\spxextra{in module deepdrr.projector.mass\_attenuation}}

\begin{fulllineitems}
\phantomsection\label{\detokenize{deepdrr.projector:deepdrr.projector.mass_attenuation.log_interp}}
\pysigstartsignatures
\pysiglinewithargsret{\sphinxcode{\sphinxupquote{deepdrr.projector.mass\_attenuation.}}\sphinxbfcode{\sphinxupquote{log\_interp}}}{\sphinxparam{\DUrole{n,n}{xInterp}}\sphinxparamcomma \sphinxparam{\DUrole{n,n}{x}}\sphinxparamcomma \sphinxparam{\DUrole{n,n}{y}}}{}
\pysigstopsignatures
\end{fulllineitems}



\subsection{deepdrr.projector.material\_coefficients module}
\label{\detokenize{deepdrr.projector:module-deepdrr.projector.material_coefficients}}\label{\detokenize{deepdrr.projector:deepdrr-projector-material-coefficients-module}}\index{module@\spxentry{module}!deepdrr.projector.material\_coefficients@\spxentry{deepdrr.projector.material\_coefficients}}\index{deepdrr.projector.material\_coefficients@\spxentry{deepdrr.projector.material\_coefficients}!module@\spxentry{module}}

\subsection{deepdrr.projector.mcgpu\_compton\_data module}
\label{\detokenize{deepdrr.projector:module-deepdrr.projector.mcgpu_compton_data}}\label{\detokenize{deepdrr.projector:deepdrr-projector-mcgpu-compton-data-module}}\index{module@\spxentry{module}!deepdrr.projector.mcgpu\_compton\_data@\spxentry{deepdrr.projector.mcgpu\_compton\_data}}\index{deepdrr.projector.mcgpu\_compton\_data@\spxentry{deepdrr.projector.mcgpu\_compton\_data}!module@\spxentry{module}}\index{sanity\_check\_compton\_data() (in module deepdrr.projector.mcgpu\_compton\_data)@\spxentry{sanity\_check\_compton\_data()}\spxextra{in module deepdrr.projector.mcgpu\_compton\_data}}

\begin{fulllineitems}
\phantomsection\label{\detokenize{deepdrr.projector:deepdrr.projector.mcgpu_compton_data.sanity_check_compton_data}}
\pysigstartsignatures
\pysiglinewithargsret{\sphinxcode{\sphinxupquote{deepdrr.projector.mcgpu\_compton\_data.}}\sphinxbfcode{\sphinxupquote{sanity\_check\_compton\_data}}}{}{}
\pysigstopsignatures
\end{fulllineitems}



\subsection{deepdrr.projector.mcgpu\_density module}
\label{\detokenize{deepdrr.projector:module-deepdrr.projector.mcgpu_density}}\label{\detokenize{deepdrr.projector:deepdrr-projector-mcgpu-density-module}}\index{module@\spxentry{module}!deepdrr.projector.mcgpu\_density@\spxentry{deepdrr.projector.mcgpu\_density}}\index{deepdrr.projector.mcgpu\_density@\spxentry{deepdrr.projector.mcgpu\_density}!module@\spxentry{module}}

\subsection{deepdrr.projector.mcgpu\_mfp\_data module}
\label{\detokenize{deepdrr.projector:module-deepdrr.projector.mcgpu_mfp_data}}\label{\detokenize{deepdrr.projector:deepdrr-projector-mcgpu-mfp-data-module}}\index{module@\spxentry{module}!deepdrr.projector.mcgpu\_mfp\_data@\spxentry{deepdrr.projector.mcgpu\_mfp\_data}}\index{deepdrr.projector.mcgpu\_mfp\_data@\spxentry{deepdrr.projector.mcgpu\_mfp\_data}!module@\spxentry{module}}\index{sanity\_check\_mfps() (in module deepdrr.projector.mcgpu\_mfp\_data)@\spxentry{sanity\_check\_mfps()}\spxextra{in module deepdrr.projector.mcgpu\_mfp\_data}}

\begin{fulllineitems}
\phantomsection\label{\detokenize{deepdrr.projector:deepdrr.projector.mcgpu_mfp_data.sanity_check_mfps}}
\pysigstartsignatures
\pysiglinewithargsret{\sphinxcode{\sphinxupquote{deepdrr.projector.mcgpu\_mfp\_data.}}\sphinxbfcode{\sphinxupquote{sanity\_check\_mfps}}}{}{}
\pysigstopsignatures
\end{fulllineitems}



\subsection{deepdrr.projector.mcgpu\_rita\_samplers module}
\label{\detokenize{deepdrr.projector:module-deepdrr.projector.mcgpu_rita_samplers}}\label{\detokenize{deepdrr.projector:deepdrr-projector-mcgpu-rita-samplers-module}}\index{module@\spxentry{module}!deepdrr.projector.mcgpu\_rita\_samplers@\spxentry{deepdrr.projector.mcgpu\_rita\_samplers}}\index{deepdrr.projector.mcgpu\_rita\_samplers@\spxentry{deepdrr.projector.mcgpu\_rita\_samplers}!module@\spxentry{module}}\index{sanity\_check\_saved\_rita\_params() (in module deepdrr.projector.mcgpu\_rita\_samplers)@\spxentry{sanity\_check\_saved\_rita\_params()}\spxextra{in module deepdrr.projector.mcgpu\_rita\_samplers}}

\begin{fulllineitems}
\phantomsection\label{\detokenize{deepdrr.projector:deepdrr.projector.mcgpu_rita_samplers.sanity_check_saved_rita_params}}
\pysigstartsignatures
\pysiglinewithargsret{\sphinxcode{\sphinxupquote{deepdrr.projector.mcgpu\_rita\_samplers.}}\sphinxbfcode{\sphinxupquote{sanity\_check\_saved\_rita\_params}}}{}{}
\pysigstopsignatures
\end{fulllineitems}



\subsection{deepdrr.projector.plane\_surface module}
\label{\detokenize{deepdrr.projector:module-deepdrr.projector.plane_surface}}\label{\detokenize{deepdrr.projector:deepdrr-projector-plane-surface-module}}\index{module@\spxentry{module}!deepdrr.projector.plane\_surface@\spxentry{deepdrr.projector.plane\_surface}}\index{deepdrr.projector.plane\_surface@\spxentry{deepdrr.projector.plane\_surface}!module@\spxentry{module}}\index{PlaneSurface (class in deepdrr.projector.plane\_surface)@\spxentry{PlaneSurface}\spxextra{class in deepdrr.projector.plane\_surface}}

\begin{fulllineitems}
\phantomsection\label{\detokenize{deepdrr.projector:deepdrr.projector.plane_surface.PlaneSurface}}
\pysigstartsignatures
\pysiglinewithargsret{\sphinxbfcode{\sphinxupquote{class\DUrole{w,w}{  }}}\sphinxcode{\sphinxupquote{deepdrr.projector.plane\_surface.}}\sphinxbfcode{\sphinxupquote{PlaneSurface}}}{\sphinxparam{\DUrole{n,n}{plane\_vector}\DUrole{p,p}{:}\DUrole{w,w}{  }\DUrole{n,n}{ndarray}}\sphinxparamcomma \sphinxparam{\DUrole{n,n}{surface\_origin}\DUrole{p,p}{:}\DUrole{w,w}{  }\DUrole{n,n}{{\hyperref[\detokenize{deepdrr.geo:deepdrr.geo.core.Point3D}]{\sphinxcrossref{Point3D}}}}}\sphinxparamcomma \sphinxparam{\DUrole{n,n}{basis}\DUrole{p,p}{:}\DUrole{w,w}{  }\DUrole{n,n}{Tuple\DUrole{p,p}{{[}}{\hyperref[\detokenize{deepdrr.geo:deepdrr.geo.core.Vector3D}]{\sphinxcrossref{Vector3D}}}\DUrole{p,p}{,}\DUrole{w,w}{  }{\hyperref[\detokenize{deepdrr.geo:deepdrr.geo.core.Vector3D}]{\sphinxcrossref{Vector3D}}}\DUrole{p,p}{{]}}}}\sphinxparamcomma \sphinxparam{\DUrole{n,n}{bounds}\DUrole{p,p}{:}\DUrole{w,w}{  }\DUrole{n,n}{ndarray}}\sphinxparamcomma \sphinxparam{\DUrole{n,n}{orthogonal}\DUrole{p,p}{:}\DUrole{w,w}{  }\DUrole{n,n}{bool}}}{}
\pysigstopsignatures
\sphinxAtStartPar
Bases: \sphinxcode{\sphinxupquote{object}}
\index{basis\_1 (deepdrr.projector.plane\_surface.PlaneSurface attribute)@\spxentry{basis\_1}\spxextra{deepdrr.projector.plane\_surface.PlaneSurface attribute}}

\begin{fulllineitems}
\phantomsection\label{\detokenize{deepdrr.projector:deepdrr.projector.plane_surface.PlaneSurface.basis_1}}
\pysigstartsignatures
\pysigline{\sphinxbfcode{\sphinxupquote{basis\_1}}\sphinxbfcode{\sphinxupquote{\DUrole{p,p}{:}\DUrole{w,w}{  }ndarray}}}
\pysigstopsignatures
\end{fulllineitems}

\index{basis\_2 (deepdrr.projector.plane\_surface.PlaneSurface attribute)@\spxentry{basis\_2}\spxextra{deepdrr.projector.plane\_surface.PlaneSurface attribute}}

\begin{fulllineitems}
\phantomsection\label{\detokenize{deepdrr.projector:deepdrr.projector.plane_surface.PlaneSurface.basis_2}}
\pysigstartsignatures
\pysigline{\sphinxbfcode{\sphinxupquote{basis\_2}}\sphinxbfcode{\sphinxupquote{\DUrole{p,p}{:}\DUrole{w,w}{  }ndarray}}}
\pysigstopsignatures
\end{fulllineitems}

\index{bounds (deepdrr.projector.plane\_surface.PlaneSurface attribute)@\spxentry{bounds}\spxextra{deepdrr.projector.plane\_surface.PlaneSurface attribute}}

\begin{fulllineitems}
\phantomsection\label{\detokenize{deepdrr.projector:deepdrr.projector.plane_surface.PlaneSurface.bounds}}
\pysigstartsignatures
\pysigline{\sphinxbfcode{\sphinxupquote{bounds}}\sphinxbfcode{\sphinxupquote{\DUrole{p,p}{:}\DUrole{w,w}{  }ndarray}}}
\pysigstopsignatures
\end{fulllineitems}

\index{check\_ray\_intersection() (deepdrr.projector.plane\_surface.PlaneSurface method)@\spxentry{check\_ray\_intersection()}\spxextra{deepdrr.projector.plane\_surface.PlaneSurface method}}

\begin{fulllineitems}
\phantomsection\label{\detokenize{deepdrr.projector:deepdrr.projector.plane_surface.PlaneSurface.check_ray_intersection}}
\pysigstartsignatures
\pysiglinewithargsret{\sphinxbfcode{\sphinxupquote{check\_ray\_intersection}}}{\sphinxparam{\DUrole{n,n}{pos}\DUrole{p,p}{:}\DUrole{w,w}{  }\DUrole{n,n}{{\hyperref[\detokenize{deepdrr.geo:deepdrr.geo.core.Point3D}]{\sphinxcrossref{Point3D}}}}}\sphinxparamcomma \sphinxparam{\DUrole{n,n}{direction}\DUrole{p,p}{:}\DUrole{w,w}{  }\DUrole{n,n}{{\hyperref[\detokenize{deepdrr.geo:deepdrr.geo.core.Vector3D}]{\sphinxcrossref{Vector3D}}}}}}{{ $\rightarrow$ float32}}
\pysigstopsignatures
\sphinxAtStartPar
Calculates whether or not a photon at the specified position, travelling in the specified direction, will hit the plane of the PlaneSurface object.

\sphinxAtStartPar
It is imperative that all of the arguments are in the same coordinate system (unchecked).
\begin{quote}\begin{description}
\sphinxlineitem{Parameters}\begin{itemize}
\item {} 
\sphinxAtStartPar
\sphinxstyleliteralstrong{\sphinxupquote{pos}} ({\hyperref[\detokenize{deepdrr.geo:deepdrr.geo.Point3D}]{\sphinxcrossref{\sphinxstyleliteralemphasis{\sphinxupquote{geo.Point3D}}}}}) \textendash{} the position of the photon

\item {} 
\sphinxAtStartPar
\sphinxstyleliteralstrong{\sphinxupquote{dir}} ({\hyperref[\detokenize{deepdrr.geo:deepdrr.geo.Vector3D}]{\sphinxcrossref{\sphinxstyleliteralemphasis{\sphinxupquote{geo.Vector3D}}}}}) \textendash{} the direction that the photon is travelling in

\end{itemize}

\sphinxlineitem{Returns}
\sphinxAtStartPar
if there will be an intersection, the distance to the intersection.  If no intersection, returns a negative number (the negative number does not necessarily have a geometrical meaning)

\sphinxlineitem{Return type}
\sphinxAtStartPar
np.float32

\end{description}\end{quote}

\end{fulllineitems}

\index{get\_lin\_comb\_coefs() (deepdrr.projector.plane\_surface.PlaneSurface method)@\spxentry{get\_lin\_comb\_coefs()}\spxextra{deepdrr.projector.plane\_surface.PlaneSurface method}}

\begin{fulllineitems}
\phantomsection\label{\detokenize{deepdrr.projector:deepdrr.projector.plane_surface.PlaneSurface.get_lin_comb_coefs}}
\pysigstartsignatures
\pysiglinewithargsret{\sphinxbfcode{\sphinxupquote{get\_lin\_comb\_coefs}}}{\sphinxparam{\DUrole{n,n}{point}\DUrole{p,p}{:}\DUrole{w,w}{  }\DUrole{n,n}{{\hyperref[\detokenize{deepdrr.geo:deepdrr.geo.core.Point3D}]{\sphinxcrossref{Point3D}}}}}}{{ $\rightarrow$ Tuple\DUrole{p,p}{{[}}float32\DUrole{p,p}{,}\DUrole{w,w}{  }float32\DUrole{p,p}{{]}}}}
\pysigstopsignatures
\sphinxAtStartPar
Returns the ‘coordinates’ of the point in the plane, where the ‘origin’ in the plane is PlaneSurface.surface\_origin,
and where the coordinate axes correspond to the PlaneSurface basis vectors.
\begin{quote}\begin{description}
\sphinxlineitem{Parameters}
\sphinxAtStartPar
\sphinxstyleliteralstrong{\sphinxupquote{point}} ({\hyperref[\detokenize{deepdrr.geo:deepdrr.geo.Point3D}]{\sphinxcrossref{\sphinxstyleliteralemphasis{\sphinxupquote{geo.Point3D}}}}}) \textendash{} the point to check

\sphinxlineitem{Returns}
\sphinxAtStartPar
the coefficients for the two basis vectors

\sphinxlineitem{Return type}
\sphinxAtStartPar
Tuple{[}np.float32, np.float32{]}

\end{description}\end{quote}

\end{fulllineitems}

\index{orthogonal (deepdrr.projector.plane\_surface.PlaneSurface attribute)@\spxentry{orthogonal}\spxextra{deepdrr.projector.plane\_surface.PlaneSurface attribute}}

\begin{fulllineitems}
\phantomsection\label{\detokenize{deepdrr.projector:deepdrr.projector.plane_surface.PlaneSurface.orthogonal}}
\pysigstartsignatures
\pysigline{\sphinxbfcode{\sphinxupquote{orthogonal}}\sphinxbfcode{\sphinxupquote{\DUrole{p,p}{:}\DUrole{w,w}{  }bool}}}
\pysigstopsignatures
\end{fulllineitems}

\index{plane\_vector (deepdrr.projector.plane\_surface.PlaneSurface attribute)@\spxentry{plane\_vector}\spxextra{deepdrr.projector.plane\_surface.PlaneSurface attribute}}

\begin{fulllineitems}
\phantomsection\label{\detokenize{deepdrr.projector:deepdrr.projector.plane_surface.PlaneSurface.plane_vector}}
\pysigstartsignatures
\pysigline{\sphinxbfcode{\sphinxupquote{plane\_vector}}\sphinxbfcode{\sphinxupquote{\DUrole{p,p}{:}\DUrole{w,w}{  }ndarray}}}
\pysigstopsignatures
\end{fulllineitems}

\index{point\_on\_surface() (deepdrr.projector.plane\_surface.PlaneSurface method)@\spxentry{point\_on\_surface()}\spxextra{deepdrr.projector.plane\_surface.PlaneSurface method}}

\begin{fulllineitems}
\phantomsection\label{\detokenize{deepdrr.projector:deepdrr.projector.plane_surface.PlaneSurface.point_on_surface}}
\pysigstartsignatures
\pysiglinewithargsret{\sphinxbfcode{\sphinxupquote{point\_on\_surface}}}{\sphinxparam{\DUrole{n,n}{point}\DUrole{p,p}{:}\DUrole{w,w}{  }\DUrole{n,n}{{\hyperref[\detokenize{deepdrr.geo:deepdrr.geo.core.Point3D}]{\sphinxcrossref{Point3D}}}}}}{{ $\rightarrow$ bool}}
\pysigstopsignatures
\sphinxAtStartPar
Returns whether the given point, which is assumed to be on the plane of the surface, is within the bounds of the surface.
\begin{quote}\begin{description}
\sphinxlineitem{Parameters}
\sphinxAtStartPar
\sphinxstyleliteralstrong{\sphinxupquote{point}} ({\hyperref[\detokenize{deepdrr.geo:deepdrr.geo.Point3D}]{\sphinxcrossref{\sphinxstyleliteralemphasis{\sphinxupquote{geo.Point3D}}}}}) \textendash{} the point to check

\sphinxlineitem{Returns}
\sphinxAtStartPar
True if the point is in\sphinxhyphen{}bounds, False otherwise

\sphinxlineitem{Return type}
\sphinxAtStartPar
bool

\end{description}\end{quote}

\end{fulllineitems}

\index{point\_on\_surface\_checking() (deepdrr.projector.plane\_surface.PlaneSurface method)@\spxentry{point\_on\_surface\_checking()}\spxextra{deepdrr.projector.plane\_surface.PlaneSurface method}}

\begin{fulllineitems}
\phantomsection\label{\detokenize{deepdrr.projector:deepdrr.projector.plane_surface.PlaneSurface.point_on_surface_checking}}
\pysigstartsignatures
\pysiglinewithargsret{\sphinxbfcode{\sphinxupquote{point\_on\_surface\_checking}}}{\sphinxparam{\DUrole{n,n}{point}\DUrole{p,p}{:}\DUrole{w,w}{  }\DUrole{n,n}{{\hyperref[\detokenize{deepdrr.geo:deepdrr.geo.core.Point3D}]{\sphinxcrossref{Point3D}}}}}}{{ $\rightarrow$ bool}}
\pysigstopsignatures
\sphinxAtStartPar
Returns whether the given point, which is not assumed to be on the plane of the surface, is within the bounds of the surface.
\begin{quote}\begin{description}
\sphinxlineitem{Parameters}
\sphinxAtStartPar
\sphinxstyleliteralstrong{\sphinxupquote{point}} ({\hyperref[\detokenize{deepdrr.geo:deepdrr.geo.Point3D}]{\sphinxcrossref{\sphinxstyleliteralemphasis{\sphinxupquote{geo.Point3D}}}}}) \textendash{} the point to check

\sphinxlineitem{Returns}
\sphinxAtStartPar
True if the point is on the plane and in\sphinxhyphen{}bounds, False otherwise

\sphinxlineitem{Return type}
\sphinxAtStartPar
bool

\end{description}\end{quote}

\end{fulllineitems}

\index{surface\_origin (deepdrr.projector.plane\_surface.PlaneSurface attribute)@\spxentry{surface\_origin}\spxextra{deepdrr.projector.plane\_surface.PlaneSurface attribute}}

\begin{fulllineitems}
\phantomsection\label{\detokenize{deepdrr.projector:deepdrr.projector.plane_surface.PlaneSurface.surface_origin}}
\pysigstartsignatures
\pysigline{\sphinxbfcode{\sphinxupquote{surface\_origin}}\sphinxbfcode{\sphinxupquote{\DUrole{p,p}{:}\DUrole{w,w}{  }ndarray}}}
\pysigstopsignatures
\end{fulllineitems}


\end{fulllineitems}



\subsection{deepdrr.projector.projector module}
\label{\detokenize{deepdrr.projector:module-deepdrr.projector.projector}}\label{\detokenize{deepdrr.projector:deepdrr-projector-projector-module}}\index{module@\spxentry{module}!deepdrr.projector.projector@\spxentry{deepdrr.projector.projector}}\index{deepdrr.projector.projector@\spxentry{deepdrr.projector.projector}!module@\spxentry{module}}\index{Projector (class in deepdrr.projector.projector)@\spxentry{Projector}\spxextra{class in deepdrr.projector.projector}}

\begin{fulllineitems}
\phantomsection\label{\detokenize{deepdrr.projector:deepdrr.projector.projector.Projector}}
\pysigstartsignatures
\pysiglinewithargsret{\sphinxbfcode{\sphinxupquote{class\DUrole{w,w}{  }}}\sphinxcode{\sphinxupquote{deepdrr.projector.projector.}}\sphinxbfcode{\sphinxupquote{Projector}}}{\sphinxparam{\DUrole{n,n}{volume}\DUrole{p,p}{:}\DUrole{w,w}{  }\DUrole{n,n}{{\hyperref[\detokenize{deepdrr.vol:deepdrr.vol.volume.Volume}]{\sphinxcrossref{Volume}}}\DUrole{w,w}{  }\DUrole{p,p}{|}\DUrole{w,w}{  }List\DUrole{p,p}{{[}}{\hyperref[\detokenize{deepdrr.vol:deepdrr.vol.volume.Volume}]{\sphinxcrossref{Volume}}}\DUrole{p,p}{{]}}}}\sphinxparamcomma \sphinxparam{\DUrole{n,n}{priorities}\DUrole{p,p}{:}\DUrole{w,w}{  }\DUrole{n,n}{List\DUrole{p,p}{{[}}int\DUrole{p,p}{{]}}\DUrole{w,w}{  }\DUrole{p,p}{|}\DUrole{w,w}{  }None}\DUrole{w,w}{  }\DUrole{o,o}{=}\DUrole{w,w}{  }\DUrole{default_value}{None}}\sphinxparamcomma \sphinxparam{\DUrole{n,n}{camera\_intrinsics}\DUrole{p,p}{:}\DUrole{w,w}{  }\DUrole{n,n}{{\hyperref[\detokenize{deepdrr.geo:deepdrr.geo.core.CameraIntrinsicTransform}]{\sphinxcrossref{CameraIntrinsicTransform}}}\DUrole{w,w}{  }\DUrole{p,p}{|}\DUrole{w,w}{  }None}\DUrole{w,w}{  }\DUrole{o,o}{=}\DUrole{w,w}{  }\DUrole{default_value}{None}}\sphinxparamcomma \sphinxparam{\DUrole{n,n}{device}\DUrole{p,p}{:}\DUrole{w,w}{  }\DUrole{n,n}{{\hyperref[\detokenize{deepdrr.device:deepdrr.device.device.Device}]{\sphinxcrossref{Device}}}\DUrole{w,w}{  }\DUrole{p,p}{|}\DUrole{w,w}{  }None}\DUrole{w,w}{  }\DUrole{o,o}{=}\DUrole{w,w}{  }\DUrole{default_value}{None}}\sphinxparamcomma \sphinxparam{\DUrole{n,n}{step}\DUrole{p,p}{:}\DUrole{w,w}{  }\DUrole{n,n}{float}\DUrole{w,w}{  }\DUrole{o,o}{=}\DUrole{w,w}{  }\DUrole{default_value}{0.1}}\sphinxparamcomma \sphinxparam{\DUrole{n,n}{mode}\DUrole{p,p}{:}\DUrole{w,w}{  }\DUrole{n,n}{str}\DUrole{w,w}{  }\DUrole{o,o}{=}\DUrole{w,w}{  }\DUrole{default_value}{\textquotesingle{}linear\textquotesingle{}}}\sphinxparamcomma \sphinxparam{\DUrole{n,n}{spectrum}\DUrole{p,p}{:}\DUrole{w,w}{  }\DUrole{n,n}{ndarray\DUrole{w,w}{  }\DUrole{p,p}{|}\DUrole{w,w}{  }str}\DUrole{w,w}{  }\DUrole{o,o}{=}\DUrole{w,w}{  }\DUrole{default_value}{\textquotesingle{}90KV\_AL40\textquotesingle{}}}\sphinxparamcomma \sphinxparam{\DUrole{n,n}{add\_scatter}\DUrole{p,p}{:}\DUrole{w,w}{  }\DUrole{n,n}{bool\DUrole{w,w}{  }\DUrole{p,p}{|}\DUrole{w,w}{  }None}\DUrole{w,w}{  }\DUrole{o,o}{=}\DUrole{w,w}{  }\DUrole{default_value}{None}}\sphinxparamcomma \sphinxparam{\DUrole{n,n}{scatter\_num}\DUrole{p,p}{:}\DUrole{w,w}{  }\DUrole{n,n}{int}\DUrole{w,w}{  }\DUrole{o,o}{=}\DUrole{w,w}{  }\DUrole{default_value}{0}}\sphinxparamcomma \sphinxparam{\DUrole{n,n}{add\_noise}\DUrole{p,p}{:}\DUrole{w,w}{  }\DUrole{n,n}{bool}\DUrole{w,w}{  }\DUrole{o,o}{=}\DUrole{w,w}{  }\DUrole{default_value}{False}}\sphinxparamcomma \sphinxparam{\DUrole{n,n}{photon\_count}\DUrole{p,p}{:}\DUrole{w,w}{  }\DUrole{n,n}{int}\DUrole{w,w}{  }\DUrole{o,o}{=}\DUrole{w,w}{  }\DUrole{default_value}{10000}}\sphinxparamcomma \sphinxparam{\DUrole{n,n}{threads}\DUrole{p,p}{:}\DUrole{w,w}{  }\DUrole{n,n}{int}\DUrole{w,w}{  }\DUrole{o,o}{=}\DUrole{w,w}{  }\DUrole{default_value}{8}}\sphinxparamcomma \sphinxparam{\DUrole{n,n}{max\_block\_index}\DUrole{p,p}{:}\DUrole{w,w}{  }\DUrole{n,n}{int}\DUrole{w,w}{  }\DUrole{o,o}{=}\DUrole{w,w}{  }\DUrole{default_value}{1024}}\sphinxparamcomma \sphinxparam{\DUrole{n,n}{collected\_energy}\DUrole{p,p}{:}\DUrole{w,w}{  }\DUrole{n,n}{bool}\DUrole{w,w}{  }\DUrole{o,o}{=}\DUrole{w,w}{  }\DUrole{default_value}{False}}\sphinxparamcomma \sphinxparam{\DUrole{n,n}{neglog}\DUrole{p,p}{:}\DUrole{w,w}{  }\DUrole{n,n}{bool}\DUrole{w,w}{  }\DUrole{o,o}{=}\DUrole{w,w}{  }\DUrole{default_value}{True}}\sphinxparamcomma \sphinxparam{\DUrole{n,n}{intensity\_upper\_bound}\DUrole{p,p}{:}\DUrole{w,w}{  }\DUrole{n,n}{float\DUrole{w,w}{  }\DUrole{p,p}{|}\DUrole{w,w}{  }None}\DUrole{w,w}{  }\DUrole{o,o}{=}\DUrole{w,w}{  }\DUrole{default_value}{None}}\sphinxparamcomma \sphinxparam{\DUrole{n,n}{attenuate\_outside\_volume}\DUrole{p,p}{:}\DUrole{w,w}{  }\DUrole{n,n}{bool}\DUrole{w,w}{  }\DUrole{o,o}{=}\DUrole{w,w}{  }\DUrole{default_value}{False}}\sphinxparamcomma \sphinxparam{\DUrole{n,n}{source\_to\_detector\_distance}\DUrole{p,p}{:}\DUrole{w,w}{  }\DUrole{n,n}{float}\DUrole{w,w}{  }\DUrole{o,o}{=}\DUrole{w,w}{  }\DUrole{default_value}{\sphinxhyphen{}1}}\sphinxparamcomma \sphinxparam{\DUrole{n,n}{carm}\DUrole{p,p}{:}\DUrole{w,w}{  }\DUrole{n,n}{{\hyperref[\detokenize{deepdrr.device:deepdrr.device.device.Device}]{\sphinxcrossref{Device}}}\DUrole{w,w}{  }\DUrole{p,p}{|}\DUrole{w,w}{  }None}\DUrole{w,w}{  }\DUrole{o,o}{=}\DUrole{w,w}{  }\DUrole{default_value}{None}}}{}
\pysigstopsignatures
\sphinxAtStartPar
Bases: \sphinxcode{\sphinxupquote{object}}
\index{camera\_intrinsics (deepdrr.projector.projector.Projector property)@\spxentry{camera\_intrinsics}\spxextra{deepdrr.projector.projector.Projector property}}

\begin{fulllineitems}
\phantomsection\label{\detokenize{deepdrr.projector:deepdrr.projector.projector.Projector.camera_intrinsics}}
\pysigstartsignatures
\pysigline{\sphinxbfcode{\sphinxupquote{property\DUrole{w,w}{  }}}\sphinxbfcode{\sphinxupquote{camera\_intrinsics}}\sphinxbfcode{\sphinxupquote{\DUrole{p,p}{:}\DUrole{w,w}{  }{\hyperref[\detokenize{deepdrr.geo:deepdrr.geo.core.CameraIntrinsicTransform}]{\sphinxcrossref{CameraIntrinsicTransform}}}}}}
\pysigstopsignatures
\end{fulllineitems}

\index{free() (deepdrr.projector.projector.Projector method)@\spxentry{free()}\spxextra{deepdrr.projector.projector.Projector method}}

\begin{fulllineitems}
\phantomsection\label{\detokenize{deepdrr.projector:deepdrr.projector.projector.Projector.free}}
\pysigstartsignatures
\pysiglinewithargsret{\sphinxbfcode{\sphinxupquote{free}}}{}{}
\pysigstopsignatures
\sphinxAtStartPar
Free the allocated GPU memory.

\end{fulllineitems}

\index{initialize() (deepdrr.projector.projector.Projector method)@\spxentry{initialize()}\spxextra{deepdrr.projector.projector.Projector method}}

\begin{fulllineitems}
\phantomsection\label{\detokenize{deepdrr.projector:deepdrr.projector.projector.Projector.initialize}}
\pysigstartsignatures
\pysiglinewithargsret{\sphinxbfcode{\sphinxupquote{initialize}}}{}{}
\pysigstopsignatures
\sphinxAtStartPar
Allocate GPU memory and transfer the volume, segmentations to GPU.

\end{fulllineitems}

\index{initialize\_output\_arrays() (deepdrr.projector.projector.Projector method)@\spxentry{initialize\_output\_arrays()}\spxextra{deepdrr.projector.projector.Projector method}}

\begin{fulllineitems}
\phantomsection\label{\detokenize{deepdrr.projector:deepdrr.projector.projector.Projector.initialize_output_arrays}}
\pysigstartsignatures
\pysiglinewithargsret{\sphinxbfcode{\sphinxupquote{initialize\_output\_arrays}}}{\sphinxparam{\DUrole{n,n}{sensor\_size}\DUrole{p,p}{:}\DUrole{w,w}{  }\DUrole{n,n}{Tuple\DUrole{p,p}{{[}}int\DUrole{p,p}{,}\DUrole{w,w}{  }int\DUrole{p,p}{{]}}}}}{{ $\rightarrow$ None}}
\pysigstopsignatures
\sphinxAtStartPar
Allocate arrays dependent on the output size. Frees previously allocated arrays.

\sphinxAtStartPar
This may have to be called multiple times if the output size changes.

\end{fulllineitems}

\index{output\_size (deepdrr.projector.projector.Projector property)@\spxentry{output\_size}\spxextra{deepdrr.projector.projector.Projector property}}

\begin{fulllineitems}
\phantomsection\label{\detokenize{deepdrr.projector:deepdrr.projector.projector.Projector.output_size}}
\pysigstartsignatures
\pysigline{\sphinxbfcode{\sphinxupquote{property\DUrole{w,w}{  }}}\sphinxbfcode{\sphinxupquote{output\_size}}\sphinxbfcode{\sphinxupquote{\DUrole{p,p}{:}\DUrole{w,w}{  }int}}}
\pysigstopsignatures
\end{fulllineitems}

\index{project() (deepdrr.projector.projector.Projector method)@\spxentry{project()}\spxextra{deepdrr.projector.projector.Projector method}}

\begin{fulllineitems}
\phantomsection\label{\detokenize{deepdrr.projector:deepdrr.projector.projector.Projector.project}}
\pysigstartsignatures
\pysiglinewithargsret{\sphinxbfcode{\sphinxupquote{project}}}{\sphinxparam{\DUrole{o,o}{*}\DUrole{n,n}{camera\_projections}\DUrole{p,p}{:}\DUrole{w,w}{  }\DUrole{n,n}{{\hyperref[\detokenize{deepdrr.geo:deepdrr.geo.core.CameraProjection}]{\sphinxcrossref{CameraProjection}}}}}}{{ $\rightarrow$ ndarray}}
\pysigstopsignatures
\sphinxAtStartPar
Perform the projection.
\begin{quote}\begin{description}
\sphinxlineitem{Parameters}
\sphinxAtStartPar
\sphinxstyleliteralstrong{\sphinxupquote{camera\_projection}} \textendash{} any number of camera projections. If none are provided, the Projector uses the CArm device to obtain a camera projection.

\sphinxlineitem{Raises}
\sphinxAtStartPar
\sphinxstyleliteralstrong{\sphinxupquote{ValueError}} \textendash{} if no projections are provided and self.device is None.

\sphinxlineitem{Returns}
\sphinxAtStartPar
array of DRRs, after mass attenuation, etc.

\sphinxlineitem{Return type}
\sphinxAtStartPar
np.ndarray

\end{description}\end{quote}

\end{fulllineitems}

\index{project\_over\_carm\_range() (deepdrr.projector.projector.Projector method)@\spxentry{project\_over\_carm\_range()}\spxextra{deepdrr.projector.projector.Projector method}}

\begin{fulllineitems}
\phantomsection\label{\detokenize{deepdrr.projector:deepdrr.projector.projector.Projector.project_over_carm_range}}
\pysigstartsignatures
\pysiglinewithargsret{\sphinxbfcode{\sphinxupquote{project\_over\_carm\_range}}}{\sphinxparam{\DUrole{n,n}{phi\_range}\DUrole{p,p}{:}\DUrole{w,w}{  }\DUrole{n,n}{Tuple\DUrole{p,p}{{[}}float\DUrole{p,p}{,}\DUrole{w,w}{  }float\DUrole{p,p}{,}\DUrole{w,w}{  }float\DUrole{p,p}{{]}}}}\sphinxparamcomma \sphinxparam{\DUrole{n,n}{theta\_range}\DUrole{p,p}{:}\DUrole{w,w}{  }\DUrole{n,n}{Tuple\DUrole{p,p}{{[}}float\DUrole{p,p}{,}\DUrole{w,w}{  }float\DUrole{p,p}{,}\DUrole{w,w}{  }float\DUrole{p,p}{{]}}}}\sphinxparamcomma \sphinxparam{\DUrole{n,n}{degrees}\DUrole{p,p}{:}\DUrole{w,w}{  }\DUrole{n,n}{bool}\DUrole{w,w}{  }\DUrole{o,o}{=}\DUrole{w,w}{  }\DUrole{default_value}{True}}}{{ $\rightarrow$ ndarray}}
\pysigstopsignatures
\sphinxAtStartPar
Project over a range of angles using the included CArm.

\sphinxAtStartPar
Ignores the CArm’s internal pose, except for its isocenter.

\end{fulllineitems}

\index{source\_to\_detector\_distance (deepdrr.projector.projector.Projector property)@\spxentry{source\_to\_detector\_distance}\spxextra{deepdrr.projector.projector.Projector property}}

\begin{fulllineitems}
\phantomsection\label{\detokenize{deepdrr.projector:deepdrr.projector.projector.Projector.source_to_detector_distance}}
\pysigstartsignatures
\pysigline{\sphinxbfcode{\sphinxupquote{property\DUrole{w,w}{  }}}\sphinxbfcode{\sphinxupquote{source\_to\_detector\_distance}}\sphinxbfcode{\sphinxupquote{\DUrole{p,p}{:}\DUrole{w,w}{  }float}}}
\pysigstopsignatures
\end{fulllineitems}

\index{volume (deepdrr.projector.projector.Projector property)@\spxentry{volume}\spxextra{deepdrr.projector.projector.Projector property}}

\begin{fulllineitems}
\phantomsection\label{\detokenize{deepdrr.projector:deepdrr.projector.projector.Projector.volume}}
\pysigstartsignatures
\pysigline{\sphinxbfcode{\sphinxupquote{property\DUrole{w,w}{  }}}\sphinxbfcode{\sphinxupquote{volume}}}
\pysigstopsignatures
\end{fulllineitems}

\index{volumes (deepdrr.projector.projector.Projector attribute)@\spxentry{volumes}\spxextra{deepdrr.projector.projector.Projector attribute}}

\begin{fulllineitems}
\phantomsection\label{\detokenize{deepdrr.projector:deepdrr.projector.projector.Projector.volumes}}
\pysigstartsignatures
\pysigline{\sphinxbfcode{\sphinxupquote{volumes}}\sphinxbfcode{\sphinxupquote{\DUrole{p,p}{:}\DUrole{w,w}{  }List\DUrole{p,p}{{[}}{\hyperref[\detokenize{deepdrr.vol:deepdrr.vol.volume.Volume}]{\sphinxcrossref{Volume}}}\DUrole{p,p}{{]}}}}}
\pysigstopsignatures
\end{fulllineitems}


\end{fulllineitems}

\index{import\_pycuda() (in module deepdrr.projector.projector)@\spxentry{import\_pycuda()}\spxextra{in module deepdrr.projector.projector}}

\begin{fulllineitems}
\phantomsection\label{\detokenize{deepdrr.projector:deepdrr.projector.projector.import_pycuda}}
\pysigstartsignatures
\pysiglinewithargsret{\sphinxcode{\sphinxupquote{deepdrr.projector.projector.}}\sphinxbfcode{\sphinxupquote{import\_pycuda}}}{}{}
\pysigstopsignatures
\sphinxAtStartPar
Import pycuda and return the context.
\begin{quote}\begin{description}
\sphinxlineitem{Returns}
\sphinxAtStartPar
The pycuda context.

\sphinxlineitem{Return type}
\sphinxAtStartPar
pycuda.autoinit.context

\end{description}\end{quote}

\end{fulllineitems}



\subsection{deepdrr.projector.rita module}
\label{\detokenize{deepdrr.projector:module-deepdrr.projector.rita}}\label{\detokenize{deepdrr.projector:deepdrr-projector-rita-module}}\index{module@\spxentry{module}!deepdrr.projector.rita@\spxentry{deepdrr.projector.rita}}\index{deepdrr.projector.rita@\spxentry{deepdrr.projector.rita}!module@\spxentry{module}}\index{RITA (class in deepdrr.projector.rita)@\spxentry{RITA}\spxextra{class in deepdrr.projector.rita}}

\begin{fulllineitems}
\phantomsection\label{\detokenize{deepdrr.projector:deepdrr.projector.rita.RITA}}
\pysigstartsignatures
\pysiglinewithargsret{\sphinxbfcode{\sphinxupquote{class\DUrole{w,w}{  }}}\sphinxcode{\sphinxupquote{deepdrr.projector.rita.}}\sphinxbfcode{\sphinxupquote{RITA}}}{\sphinxparam{\DUrole{n,n}{x\_arr}\DUrole{p,p}{:}\DUrole{w,w}{  }\DUrole{n,n}{ndarray}}\sphinxparamcomma \sphinxparam{\DUrole{n,n}{y\_arr}\DUrole{p,p}{:}\DUrole{w,w}{  }\DUrole{n,n}{ndarray}}\sphinxparamcomma \sphinxparam{\DUrole{n,n}{a\_arr}\DUrole{p,p}{:}\DUrole{w,w}{  }\DUrole{n,n}{ndarray}}\sphinxparamcomma \sphinxparam{\DUrole{n,n}{b\_arr}\DUrole{p,p}{:}\DUrole{w,w}{  }\DUrole{n,n}{ndarray}}}{}
\pysigstopsignatures
\sphinxAtStartPar
Bases: \sphinxcode{\sphinxupquote{object}}
\index{dtype (deepdrr.projector.rita.RITA attribute)@\spxentry{dtype}\spxextra{deepdrr.projector.rita.RITA attribute}}

\begin{fulllineitems}
\phantomsection\label{\detokenize{deepdrr.projector:deepdrr.projector.rita.RITA.dtype}}
\pysigstartsignatures
\pysigline{\sphinxbfcode{\sphinxupquote{dtype}}}
\pysigstopsignatures
\sphinxAtStartPar
alias of \sphinxcode{\sphinxupquote{float64}}

\end{fulllineitems}

\index{from\_pdf() (deepdrr.projector.rita.RITA class method)@\spxentry{from\_pdf()}\spxextra{deepdrr.projector.rita.RITA class method}}

\begin{fulllineitems}
\phantomsection\label{\detokenize{deepdrr.projector:deepdrr.projector.rita.RITA.from_pdf}}
\pysigstartsignatures
\pysiglinewithargsret{\sphinxbfcode{\sphinxupquote{classmethod\DUrole{w,w}{  }}}\sphinxbfcode{\sphinxupquote{from\_pdf}}}{\sphinxparam{\DUrole{n,n}{x\_min}\DUrole{p,p}{:}\DUrole{w,w}{  }\DUrole{n,n}{float64}}\sphinxparamcomma \sphinxparam{\DUrole{n,n}{x\_max}\DUrole{p,p}{:}\DUrole{w,w}{  }\DUrole{n,n}{float64}}\sphinxparamcomma \sphinxparam{\DUrole{n,n}{pdf\_func}\DUrole{p,p}{:}\DUrole{w,w}{  }\DUrole{n,n}{Callable\DUrole{p,p}{{[}}\DUrole{p,p}{{[}}float64\DUrole{p,p}{{]}}\DUrole{p,p}{,}\DUrole{w,w}{  }float64\DUrole{p,p}{{]}}}}\sphinxparamcomma \sphinxparam{\DUrole{n,n}{n\_grid\_points}\DUrole{p,p}{:}\DUrole{w,w}{  }\DUrole{n,n}{int32\DUrole{w,w}{  }\DUrole{p,p}{|}\DUrole{w,w}{  }None}\DUrole{w,w}{  }\DUrole{o,o}{=}\DUrole{w,w}{  }\DUrole{default_value}{128}}}{}
\pysigstopsignatures
\sphinxAtStartPar
Creates and returns a RITA object from the provided PDF over the provided interval, using the specified number of gridpoints.
\begin{quote}\begin{description}
\sphinxlineitem{Parameters}\begin{itemize}
\item {} 
\sphinxAtStartPar
\sphinxstyleliteralstrong{\sphinxupquote{x\_min}} (\sphinxstyleliteralemphasis{\sphinxupquote{np.float64}}) \textendash{} the lower bound of the interval to sample from

\item {} 
\sphinxAtStartPar
\sphinxstyleliteralstrong{\sphinxupquote{x\_max}} (\sphinxstyleliteralemphasis{\sphinxupquote{np.float64}}) \textendash{} the upper bound of the interval to sample from

\item {} 
\sphinxAtStartPar
\sphinxstyleliteralstrong{\sphinxupquote{pdf\_func}} (\sphinxstyleliteralemphasis{\sphinxupquote{Callable}}\sphinxstyleliteralemphasis{\sphinxupquote{{[}}}\sphinxstyleliteralemphasis{\sphinxupquote{{[}}}\sphinxstyleliteralemphasis{\sphinxupquote{np.float64}}\sphinxstyleliteralemphasis{\sphinxupquote{{]}}}\sphinxstyleliteralemphasis{\sphinxupquote{, }}\sphinxstyleliteralemphasis{\sphinxupquote{np.float64}}\sphinxstyleliteralemphasis{\sphinxupquote{{]}}}) \textendash{} the analytical PDF (float \sphinxhyphen{}\textgreater{} float) of the function whose PDF we are sampling from

\item {} 
\sphinxAtStartPar
\sphinxstyleliteralstrong{\sphinxupquote{n\_grid\_points}} (\sphinxstyleliteralemphasis{\sphinxupquote{Optional}}\sphinxstyleliteralemphasis{\sphinxupquote{{[}}}\sphinxstyleliteralemphasis{\sphinxupquote{np.int32}}\sphinxstyleliteralemphasis{\sphinxupquote{{]}}}\sphinxstyleliteralemphasis{\sphinxupquote{, }}\sphinxstyleliteralemphasis{\sphinxupquote{optional}}) \textendash{} the number of grid points to finish with.  Must be at least 10.  Defaults to 128

\end{itemize}

\end{description}\end{quote}

\end{fulllineitems}

\index{from\_saved\_params() (deepdrr.projector.rita.RITA class method)@\spxentry{from\_saved\_params()}\spxextra{deepdrr.projector.rita.RITA class method}}

\begin{fulllineitems}
\phantomsection\label{\detokenize{deepdrr.projector:deepdrr.projector.rita.RITA.from_saved_params}}
\pysigstartsignatures
\pysiglinewithargsret{\sphinxbfcode{\sphinxupquote{classmethod\DUrole{w,w}{  }}}\sphinxbfcode{\sphinxupquote{from\_saved\_params}}}{\sphinxparam{\DUrole{n,n}{params}\DUrole{p,p}{:}\DUrole{w,w}{  }\DUrole{n,n}{ndarray}}}{}
\pysigstopsignatures
\sphinxAtStartPar
Creates and returns a RITA object based on the saved RITA parameters.
\begin{quote}\begin{description}
\sphinxlineitem{Parameters}
\sphinxAtStartPar
\sphinxstyleliteralstrong{\sphinxupquote{params}} (\sphinxstyleliteralemphasis{\sphinxupquote{np.ndarray}}) \textendash{} the saved parameters.  See mcgpu\_rita\_samplers.py:saved\_rita\_params dictionary for available options

\end{description}\end{quote}

\end{fulllineitems}

\index{sample\_rita() (deepdrr.projector.rita.RITA method)@\spxentry{sample\_rita()}\spxextra{deepdrr.projector.rita.RITA method}}

\begin{fulllineitems}
\phantomsection\label{\detokenize{deepdrr.projector:deepdrr.projector.rita.RITA.sample_rita}}
\pysigstartsignatures
\pysiglinewithargsret{\sphinxbfcode{\sphinxupquote{sample\_rita}}}{}{{ $\rightarrow$ float64}}
\pysigstopsignatures
\sphinxAtStartPar
Using the provided RITA parameters, return an x\sphinxhyphen{}value based on the RITA\sphinxhyphen{}approximated PDF
\begin{quote}\begin{description}
\sphinxlineitem{Parameters}\begin{itemize}
\item {} 
\sphinxAtStartPar
\sphinxstyleliteralstrong{\sphinxupquote{x\_arr}} (\sphinxstyleliteralemphasis{\sphinxupquote{np.ndarray}}) \textendash{} the x\_i’s (grid points) for RITA

\item {} 
\sphinxAtStartPar
\sphinxstyleliteralstrong{\sphinxupquote{y\_arr}} (\sphinxstyleliteralemphasis{\sphinxupquote{np.ndarray}}) \textendash{} the y\_i’s (Greek ‘xi’ values) for RITA

\item {} 
\sphinxAtStartPar
\sphinxstyleliteralstrong{\sphinxupquote{a\_arr}} (\sphinxstyleliteralemphasis{\sphinxupquote{np.ndarray}}) \textendash{} the a\_i’s for RITA

\item {} 
\sphinxAtStartPar
\sphinxstyleliteralstrong{\sphinxupquote{b\_arr}} (\sphinxstyleliteralemphasis{\sphinxupquote{np.ndarray}}) \textendash{} the b\_i’s for RITA

\end{itemize}

\sphinxlineitem{Returns}
\sphinxAtStartPar
a randomly sampled x\sphinxhyphen{}value

\sphinxlineitem{Return type}
\sphinxAtStartPar
np.float64

\end{description}\end{quote}

\end{fulllineitems}


\end{fulllineitems}

\index{numerically\_integrate() (in module deepdrr.projector.rita)@\spxentry{numerically\_integrate()}\spxextra{in module deepdrr.projector.rita}}

\begin{fulllineitems}
\phantomsection\label{\detokenize{deepdrr.projector:deepdrr.projector.rita.numerically_integrate}}
\pysigstartsignatures
\pysiglinewithargsret{\sphinxcode{\sphinxupquote{deepdrr.projector.rita.}}\sphinxbfcode{\sphinxupquote{numerically\_integrate}}}{\sphinxparam{\DUrole{n,n}{func}\DUrole{p,p}{:}\DUrole{w,w}{  }\DUrole{n,n}{Callable\DUrole{p,p}{{[}}\DUrole{p,p}{{[}}float64\DUrole{p,p}{{]}}\DUrole{p,p}{,}\DUrole{w,w}{  }float64\DUrole{p,p}{{]}}}}\sphinxparamcomma \sphinxparam{\DUrole{n,n}{x\_min}\DUrole{p,p}{:}\DUrole{w,w}{  }\DUrole{n,n}{float64}}\sphinxparamcomma \sphinxparam{\DUrole{n,n}{x\_max}\DUrole{p,p}{:}\DUrole{w,w}{  }\DUrole{n,n}{float64}}}{}
\pysigstopsignatures
\sphinxAtStartPar
Numerically integrates function ‘func’ on the interval {[}x\_min,x\_max{]} using the 20\sphinxhyphen{}point Gauss method (see page 261 of ‘PENELOPE\sphinxhyphen{}2006’)
\begin{quote}\begin{description}
\sphinxlineitem{Parameters}\begin{itemize}
\item {} 
\sphinxAtStartPar
\sphinxstyleliteralstrong{\sphinxupquote{func}} (\sphinxstyleliteralemphasis{\sphinxupquote{Callable}}\sphinxstyleliteralemphasis{\sphinxupquote{{[}}}\sphinxstyleliteralemphasis{\sphinxupquote{{[}}}\sphinxstyleliteralemphasis{\sphinxupquote{np.float64}}\sphinxstyleliteralemphasis{\sphinxupquote{{]}}}\sphinxstyleliteralemphasis{\sphinxupquote{, }}\sphinxstyleliteralemphasis{\sphinxupquote{np.float64}}\sphinxstyleliteralemphasis{\sphinxupquote{{]}}}) \textendash{} the function to integrate

\item {} 
\sphinxAtStartPar
\sphinxstyleliteralstrong{\sphinxupquote{x\_min}} (\sphinxstyleliteralemphasis{\sphinxupquote{np.float64}}) \textendash{} the lower integration bound

\item {} 
\sphinxAtStartPar
\sphinxstyleliteralstrong{\sphinxupquote{x\_max}} (\sphinxstyleliteralemphasis{\sphinxupquote{np.float64}}) \textendash{} the upper integration bound

\end{itemize}

\sphinxlineitem{Returns}
\sphinxAtStartPar
the result of numerically integrating ‘func’

\sphinxlineitem{Return type}
\sphinxAtStartPar
np.float64

\end{description}\end{quote}

\end{fulllineitems}



\subsection{deepdrr.projector.scatter module}
\label{\detokenize{deepdrr.projector:module-deepdrr.projector.scatter}}\label{\detokenize{deepdrr.projector:deepdrr-projector-scatter-module}}\index{module@\spxentry{module}!deepdrr.projector.scatter@\spxentry{deepdrr.projector.scatter}}\index{deepdrr.projector.scatter@\spxentry{deepdrr.projector.scatter}!module@\spxentry{module}}\index{get\_detector\_plane() (in module deepdrr.projector.scatter)@\spxentry{get\_detector\_plane()}\spxextra{in module deepdrr.projector.scatter}}

\begin{fulllineitems}
\phantomsection\label{\detokenize{deepdrr.projector:deepdrr.projector.scatter.get_detector_plane}}
\pysigstartsignatures
\pysiglinewithargsret{\sphinxcode{\sphinxupquote{deepdrr.projector.scatter.}}\sphinxbfcode{\sphinxupquote{get\_detector\_plane}}}{\sphinxparam{\DUrole{n,n}{rt\_kinv}\DUrole{p,p}{:}\DUrole{w,w}{  }\DUrole{n,n}{ndarray}}\sphinxparamcomma \sphinxparam{\DUrole{n,n}{camera\_intrinsics}\DUrole{p,p}{:}\DUrole{w,w}{  }\DUrole{n,n}{{\hyperref[\detokenize{deepdrr.geo:deepdrr.geo.core.CameraIntrinsicTransform}]{\sphinxcrossref{CameraIntrinsicTransform}}}}}\sphinxparamcomma \sphinxparam{\DUrole{n,n}{sdd}\DUrole{p,p}{:}\DUrole{w,w}{  }\DUrole{n,n}{float}}\sphinxparamcomma \sphinxparam{\DUrole{n,n}{source\_world}\DUrole{p,p}{:}\DUrole{w,w}{  }\DUrole{n,n}{{\hyperref[\detokenize{deepdrr.geo:deepdrr.geo.core.Point3D}]{\sphinxcrossref{Point3D}}}}}\sphinxparamcomma \sphinxparam{\DUrole{n,n}{sensor\_size}\DUrole{p,p}{:}\DUrole{w,w}{  }\DUrole{n,n}{Tuple\DUrole{p,p}{{[}}int\DUrole{p,p}{,}\DUrole{w,w}{  }int\DUrole{p,p}{{]}}}}}{{ $\rightarrow$ {\hyperref[\detokenize{deepdrr.projector:deepdrr.projector.plane_surface.PlaneSurface}]{\sphinxcrossref{PlaneSurface}}}}}
\pysigstopsignatures
\sphinxAtStartPar
Calculates the PlaneSurface object of the detector plane in IJK coordinates.
Note that the cosines of the plane’s normal vector (n\_x, n\_y, n\_z) are NOT normalized to be a unit vector.

\sphinxAtStartPar
The first basis vector represents moving one pixel ACROSS the image (left to right).
The second basis vector represents moving one pixel DOWN the image (top to bottom).
\begin{quote}\begin{description}
\sphinxlineitem{Parameters}\begin{itemize}
\item {} 
\sphinxAtStartPar
\sphinxstyleliteralstrong{\sphinxupquote{rt\_kinv}} (\sphinxstyleliteralemphasis{\sphinxupquote{np.ndarray}}) \textendash{} the 3x3 ray transform for the projection.  Transforms pixel indices (u,v,1) to world\sphinxhyphen{}space vector along
the ray from the X\sphinxhyphen{}Ray source to the pixel {[}u,v{]} on the detector, such that the resulting world\sphinxhyphen{}space vector
has unit projection along the vector pointing from the source to the center of the detector.

\item {} 
\sphinxAtStartPar
\sphinxstyleliteralstrong{\sphinxupquote{camera\_intrinsics}} ({\hyperref[\detokenize{deepdrr.geo:deepdrr.geo.CameraIntrinsicTransform}]{\sphinxcrossref{\sphinxstyleliteralemphasis{\sphinxupquote{geo.CameraIntrinsicTransform}}}}}) \textendash{} the 3x3 matrix that denotes the camera’s intrinsics.  Canonically represented by K.

\item {} 
\sphinxAtStartPar
\sphinxstyleliteralstrong{\sphinxupquote{sdd}} (\sphinxstyleliteralemphasis{\sphinxupquote{float}}) \textendash{} the distance from the X\sphinxhyphen{}Ray source to the detector.

\item {} 
\sphinxAtStartPar
\sphinxstyleliteralstrong{\sphinxupquote{source\_world}} ({\hyperref[\detokenize{deepdrr.geo:deepdrr.geo.Point3D}]{\sphinxcrossref{\sphinxstyleliteralemphasis{\sphinxupquote{geo.Point3D}}}}}) \textendash{} the world coordinates of the X\sphinxhyphen{}Ray source

\item {} 
\sphinxAtStartPar
\sphinxstyleliteralstrong{\sphinxupquote{sensor\_size}} (\sphinxstyleliteralemphasis{\sphinxupquote{Tuple}}\sphinxstyleliteralemphasis{\sphinxupquote{{[}}}\sphinxstyleliteralemphasis{\sphinxupquote{int}}\sphinxstyleliteralemphasis{\sphinxupquote{,}}\sphinxstyleliteralemphasis{\sphinxupquote{int}}\sphinxstyleliteralemphasis{\sphinxupquote{{]}}}) \textendash{} the sensor size \{width\}x\{height\}, in pixels, of the detector

\end{itemize}

\sphinxlineitem{Returns}
\sphinxAtStartPar
a PlaneSurface object representing the detector.

\sphinxlineitem{Return type}
\sphinxAtStartPar
{\hyperref[\detokenize{deepdrr.projector:deepdrr.projector.plane_surface.PlaneSurface}]{\sphinxcrossref{PlaneSurface}}}

\end{description}\end{quote}

\end{fulllineitems}

\index{get\_mfp\_data() (in module deepdrr.projector.scatter)@\spxentry{get\_mfp\_data()}\spxextra{in module deepdrr.projector.scatter}}

\begin{fulllineitems}
\phantomsection\label{\detokenize{deepdrr.projector:deepdrr.projector.scatter.get_mfp_data}}
\pysigstartsignatures
\pysiglinewithargsret{\sphinxcode{\sphinxupquote{deepdrr.projector.scatter.}}\sphinxbfcode{\sphinxupquote{get\_mfp\_data}}}{\sphinxparam{\DUrole{n,n}{table}\DUrole{p,p}{:}\DUrole{w,w}{  }\DUrole{n,n}{ndarray}}\sphinxparamcomma \sphinxparam{\DUrole{n,n}{E}\DUrole{p,p}{:}\DUrole{w,w}{  }\DUrole{n,n}{float32}}}{{ $\rightarrow$ Tuple\DUrole{p,p}{{[}}float32\DUrole{p,p}{,}\DUrole{w,w}{  }float32\DUrole{p,p}{,}\DUrole{w,w}{  }float32\DUrole{p,p}{{]}}}}
\pysigstopsignatures
\sphinxAtStartPar
Access the Mean Free Path data for the given material’s table at the given photon energy level.
Performs linear interpolation for any energy value that isn’t exactly a table entry.
\begin{quote}\begin{description}
\sphinxlineitem{Parameters}\begin{itemize}
\item {} 
\sphinxAtStartPar
\sphinxstyleliteralstrong{\sphinxupquote{table}} (\sphinxstyleliteralemphasis{\sphinxupquote{np.ndarray}}) \textendash{} a table of Mean Free Path data.  See mcgpu\_mean\_free\_path\_data directory for examples.

\item {} 
\sphinxAtStartPar
\sphinxstyleliteralstrong{\sphinxupquote{E}} (\sphinxstyleliteralemphasis{\sphinxupquote{np.float32}}) \textendash{} the energy of the photon

\end{itemize}

\sphinxlineitem{Returns}
\sphinxAtStartPar
the Rayleigh scatter mean free path
np.float32: the Compton scatter mean free path
np.float32: the total mean free path

\sphinxlineitem{Return type}
\sphinxAtStartPar
np.float32

\end{description}\end{quote}

\end{fulllineitems}

\index{get\_scattered\_dir() (in module deepdrr.projector.scatter)@\spxentry{get\_scattered\_dir()}\spxextra{in module deepdrr.projector.scatter}}

\begin{fulllineitems}
\phantomsection\label{\detokenize{deepdrr.projector:deepdrr.projector.scatter.get_scattered_dir}}
\pysigstartsignatures
\pysiglinewithargsret{\sphinxcode{\sphinxupquote{deepdrr.projector.scatter.}}\sphinxbfcode{\sphinxupquote{get\_scattered\_dir}}}{\sphinxparam{\DUrole{n,n}{direction}\DUrole{p,p}{:}\DUrole{w,w}{  }\DUrole{n,n}{{\hyperref[\detokenize{deepdrr.geo:deepdrr.geo.core.Vector3D}]{\sphinxcrossref{Vector3D}}}}}\sphinxparamcomma \sphinxparam{\DUrole{n,n}{cos\_theta}\DUrole{p,p}{:}\DUrole{w,w}{  }\DUrole{n,n}{float32}}\sphinxparamcomma \sphinxparam{\DUrole{n,n}{phi}\DUrole{p,p}{:}\DUrole{w,w}{  }\DUrole{n,n}{float32}}}{{ $\rightarrow$ {\hyperref[\detokenize{deepdrr.geo:deepdrr.geo.core.Vector3D}]{\sphinxcrossref{Vector3D}}}}}
\pysigstopsignatures
\sphinxAtStartPar
Determine the new direction of travel after getting scattered
\begin{quote}\begin{description}
\sphinxlineitem{Parameters}\begin{itemize}
\item {} 
\sphinxAtStartPar
\sphinxstyleliteralstrong{\sphinxupquote{dir}} ({\hyperref[\detokenize{deepdrr.geo:deepdrr.geo.Vector3D}]{\sphinxcrossref{\sphinxstyleliteralemphasis{\sphinxupquote{geo.Vector3D}}}}}) \textendash{} the incoming direction of travel

\item {} 
\sphinxAtStartPar
\sphinxstyleliteralstrong{\sphinxupquote{cos\_theta}} (\sphinxstyleliteralemphasis{\sphinxupquote{np.float32}}) \textendash{} the cosine of the polar scattering angle, i.e. the angle dir and dir\_prime

\item {} 
\sphinxAtStartPar
\sphinxstyleliteralstrong{\sphinxupquote{phi}} (\sphinxstyleliteralemphasis{\sphinxupquote{np.float32}}) \textendash{} the azimuthal angle, i.e. how dir\_prime is rotated about the axis ‘dir’.

\end{itemize}

\sphinxlineitem{Returns}
\sphinxAtStartPar
the outgoing direction of travel

\sphinxlineitem{Return type}
\sphinxAtStartPar
{\hyperref[\detokenize{deepdrr.geo:deepdrr.geo.Vector3D}]{\sphinxcrossref{geo.Vector3D}}}

\end{description}\end{quote}

\end{fulllineitems}

\index{get\_woodcock\_mfp() (in module deepdrr.projector.scatter)@\spxentry{get\_woodcock\_mfp()}\spxextra{in module deepdrr.projector.scatter}}

\begin{fulllineitems}
\phantomsection\label{\detokenize{deepdrr.projector:deepdrr.projector.scatter.get_woodcock_mfp}}
\pysigstartsignatures
\pysiglinewithargsret{\sphinxcode{\sphinxupquote{deepdrr.projector.scatter.}}\sphinxbfcode{\sphinxupquote{get\_woodcock\_mfp}}}{\sphinxparam{\DUrole{n,n}{table}\DUrole{p,p}{:}\DUrole{w,w}{  }\DUrole{n,n}{ndarray}}\sphinxparamcomma \sphinxparam{\DUrole{n,n}{E}\DUrole{p,p}{:}\DUrole{w,w}{  }\DUrole{n,n}{float32}}}{{ $\rightarrow$ float32}}
\pysigstopsignatures
\sphinxAtStartPar
Access the Woodcock Mean Free Path at the given photon energy level.
For an explanation of what the Woodcock Mean Free Path is, see mcgpu\_mfp\_data.py.
Performs linear interpolation for any energy value that isn’t exactly a table entry.
\begin{quote}\begin{description}
\sphinxlineitem{Parameters}\begin{itemize}
\item {} 
\sphinxAtStartPar
\sphinxstyleliteralstrong{\sphinxupquote{table}} (\sphinxstyleliteralemphasis{\sphinxupquote{np.ndarray}}) \textendash{} a table of Woodcock Mean Free Path data.  See make\_woodcock\_mfp(…).

\item {} 
\sphinxAtStartPar
\sphinxstyleliteralstrong{\sphinxupquote{E}} (\sphinxstyleliteralemphasis{\sphinxupquote{np.float32}}) \textendash{} the energy of the photon

\end{itemize}

\sphinxlineitem{Returns}
\sphinxAtStartPar
the inverse of the total majorant cross section.  This returned value has units of centimeters.

\sphinxlineitem{Return type}
\sphinxAtStartPar
np.float32

\end{description}\end{quote}

\end{fulllineitems}

\index{make\_woodcock\_mfp() (in module deepdrr.projector.scatter)@\spxentry{make\_woodcock\_mfp()}\spxextra{in module deepdrr.projector.scatter}}

\begin{fulllineitems}
\phantomsection\label{\detokenize{deepdrr.projector:deepdrr.projector.scatter.make_woodcock_mfp}}
\pysigstartsignatures
\pysiglinewithargsret{\sphinxcode{\sphinxupquote{deepdrr.projector.scatter.}}\sphinxbfcode{\sphinxupquote{make\_woodcock\_mfp}}}{\sphinxparam{\DUrole{n,n}{materials}\DUrole{p,p}{:}\DUrole{w,w}{  }\DUrole{n,n}{List\DUrole{p,p}{{[}}str\DUrole{p,p}{{]}}}}}{{ $\rightarrow$ ndarray}}
\pysigstopsignatures
\sphinxAtStartPar
Generates and returns a table of {[}energy, Woodcock MFP{]} for each energy level, based on the provided materials
\begin{quote}\begin{description}
\sphinxlineitem{Parameters}
\sphinxAtStartPar
\sphinxstyleliteralstrong{\sphinxupquote{materials}} (\sphinxstyleliteralemphasis{\sphinxupquote{List}}\sphinxstyleliteralemphasis{\sphinxupquote{{[}}}\sphinxstyleliteralemphasis{\sphinxupquote{str}}\sphinxstyleliteralemphasis{\sphinxupquote{{]}}}) \textendash{} list of material names to generate Woodock MFP data for.  For a list of available materials, reference mcgpu\_mfp\_data.py

\sphinxlineitem{Returns}
\sphinxAtStartPar
a table of {[}energy, Woodcock MFP{]} for each applicable energy level

\sphinxlineitem{Return type}
\sphinxAtStartPar
np.ndarray

\end{description}\end{quote}

\end{fulllineitems}

\index{move\_photon\_to\_volume() (in module deepdrr.projector.scatter)@\spxentry{move\_photon\_to\_volume()}\spxextra{in module deepdrr.projector.scatter}}

\begin{fulllineitems}
\phantomsection\label{\detokenize{deepdrr.projector:deepdrr.projector.scatter.move_photon_to_volume}}
\pysigstartsignatures
\pysiglinewithargsret{\sphinxcode{\sphinxupquote{deepdrr.projector.scatter.}}\sphinxbfcode{\sphinxupquote{move\_photon\_to\_volume}}}{\sphinxparam{\DUrole{n,n}{pos}\DUrole{p,p}{:}\DUrole{w,w}{  }\DUrole{n,n}{{\hyperref[\detokenize{deepdrr.geo:deepdrr.geo.core.Point3D}]{\sphinxcrossref{Point3D}}}}}\sphinxparamcomma \sphinxparam{\DUrole{n,n}{direction}\DUrole{p,p}{:}\DUrole{w,w}{  }\DUrole{n,n}{{\hyperref[\detokenize{deepdrr.geo:deepdrr.geo.core.Vector3D}]{\sphinxcrossref{Vector3D}}}}}\sphinxparamcomma \sphinxparam{\DUrole{n,n}{volume\_min\_bounds}\DUrole{p,p}{:}\DUrole{w,w}{  }\DUrole{n,n}{Tuple\DUrole{p,p}{{[}}float\DUrole{p,p}{,}\DUrole{w,w}{  }float\DUrole{p,p}{,}\DUrole{w,w}{  }float\DUrole{p,p}{{]}}}}\sphinxparamcomma \sphinxparam{\DUrole{n,n}{volume\_max\_bounds}\DUrole{p,p}{:}\DUrole{w,w}{  }\DUrole{n,n}{Tuple\DUrole{p,p}{{[}}float\DUrole{p,p}{,}\DUrole{w,w}{  }float\DUrole{p,p}{,}\DUrole{w,w}{  }float\DUrole{p,p}{{]}}}}}{{ $\rightarrow$ Tuple\DUrole{p,p}{{[}}bool\DUrole{p,p}{,}\DUrole{w,w}{  }{\hyperref[\detokenize{deepdrr.geo:deepdrr.geo.core.Point3D}]{\sphinxcrossref{Point3D}}}\DUrole{p,p}{{]}}}}
\pysigstopsignatures
\sphinxAtStartPar
Transports a photon at the given position, travelling in the given direction, to a rectangular\sphinxhyphen{}prism volume of the given bounds.
Assumes the volume’s surfaces are aligned with the major planes of the coordinate system
\begin{quote}\begin{description}
\sphinxlineitem{Parameters}\begin{itemize}
\item {} 
\sphinxAtStartPar
\sphinxstyleliteralstrong{\sphinxupquote{pos}} ({\hyperref[\detokenize{deepdrr.geo:deepdrr.geo.Point3D}]{\sphinxcrossref{\sphinxstyleliteralemphasis{\sphinxupquote{geo.Point3D}}}}}) \textendash{} the initial position of the photon.  Very likely to be the X\sphinxhyphen{}ray source.

\item {} 
\sphinxAtStartPar
\sphinxstyleliteralstrong{\sphinxupquote{direction}} ({\hyperref[\detokenize{deepdrr.geo:deepdrr.geo.Vector3D}]{\sphinxcrossref{\sphinxstyleliteralemphasis{\sphinxupquote{geo.Vector3D}}}}}) \textendash{} a unit vector denoting the direction in which the photon is traveling

\item {} 
\sphinxAtStartPar
\sphinxstyleliteralstrong{\sphinxupquote{volume\_min\_bounds}} (\sphinxstyleliteralemphasis{\sphinxupquote{Tuple}}\sphinxstyleliteralemphasis{\sphinxupquote{{[}}}\sphinxstyleliteralemphasis{\sphinxupquote{float}}\sphinxstyleliteralemphasis{\sphinxupquote{, }}\sphinxstyleliteralemphasis{\sphinxupquote{float}}\sphinxstyleliteralemphasis{\sphinxupquote{, }}\sphinxstyleliteralemphasis{\sphinxupquote{float}}\sphinxstyleliteralemphasis{\sphinxupquote{{]}}}) \textendash{} the minimum coordinate bound for the volume in each direction

\item {} 
\sphinxAtStartPar
\sphinxstyleliteralstrong{\sphinxupquote{volume\_max\_bounds}} (\sphinxstyleliteralemphasis{\sphinxupquote{Tuple}}\sphinxstyleliteralemphasis{\sphinxupquote{{[}}}\sphinxstyleliteralemphasis{\sphinxupquote{float}}\sphinxstyleliteralemphasis{\sphinxupquote{, }}\sphinxstyleliteralemphasis{\sphinxupquote{float}}\sphinxstyleliteralemphasis{\sphinxupquote{, }}\sphinxstyleliteralemphasis{\sphinxupquote{float}}\sphinxstyleliteralemphasis{\sphinxupquote{{]}}}) \textendash{} the minimum coordinate bound for the volume in each direction

\end{itemize}

\sphinxlineitem{Returns}
\sphinxAtStartPar
whether the photon hits the volume or not
geo.Point3D: where the photon hits the volume if it hits the volume, else the original position

\sphinxlineitem{Return type}
\sphinxAtStartPar
bool

\end{description}\end{quote}

\end{fulllineitems}

\index{sample\_Compton\_theta\_E\_prime() (in module deepdrr.projector.scatter)@\spxentry{sample\_Compton\_theta\_E\_prime()}\spxextra{in module deepdrr.projector.scatter}}

\begin{fulllineitems}
\phantomsection\label{\detokenize{deepdrr.projector:deepdrr.projector.scatter.sample_Compton_theta_E_prime}}
\pysigstartsignatures
\pysiglinewithargsret{\sphinxcode{\sphinxupquote{deepdrr.projector.scatter.}}\sphinxbfcode{\sphinxupquote{sample\_Compton\_theta\_E\_prime}}}{\sphinxparam{\DUrole{n,n}{photon\_energy}\DUrole{p,p}{:}\DUrole{w,w}{  }\DUrole{n,n}{float32}}\sphinxparamcomma \sphinxparam{\DUrole{n,n}{mat\_nshells}\DUrole{p,p}{:}\DUrole{w,w}{  }\DUrole{n,n}{int32}}\sphinxparamcomma \sphinxparam{\DUrole{n,n}{mat\_compton\_data}\DUrole{p,p}{:}\DUrole{w,w}{  }\DUrole{n,n}{ndarray}}}{{ $\rightarrow$ float32}}
\pysigstopsignatures
\sphinxAtStartPar
Randomly sample values of theta and W for a given Compton scatter interaction
\begin{quote}\begin{description}
\sphinxlineitem{Parameters}\begin{itemize}
\item {} 
\sphinxAtStartPar
\sphinxstyleliteralstrong{\sphinxupquote{photon\_energy}} (\sphinxstyleliteralemphasis{\sphinxupquote{np.float32}}) \textendash{} the energy of the incoming photon

\item {} 
\sphinxAtStartPar
\sphinxstyleliteralstrong{\sphinxupquote{mat\_nshells}} (\sphinxstyleliteralemphasis{\sphinxupquote{np.int32}}) \textendash{} the number of electron shells in the material being interacted with

\item {} 
\sphinxAtStartPar
\sphinxstyleliteralstrong{\sphinxupquote{mat\_compton\_data}} (\sphinxstyleliteralemphasis{\sphinxupquote{np.ndarray}}) \textendash{} the Compton scatter data for the material being interacted with.  See mcgpu\_compton\_data.py for more details

\end{itemize}

\sphinxlineitem{Returns}
\sphinxAtStartPar
cos\_theta, the polar scattering angle
np.float32: E\_prime, the energy of the outgoing photon

\sphinxlineitem{Return type}
\sphinxAtStartPar
np.float32

\end{description}\end{quote}

\end{fulllineitems}

\index{sample\_Rayleigh\_theta() (in module deepdrr.projector.scatter)@\spxentry{sample\_Rayleigh\_theta()}\spxextra{in module deepdrr.projector.scatter}}

\begin{fulllineitems}
\phantomsection\label{\detokenize{deepdrr.projector:deepdrr.projector.scatter.sample_Rayleigh_theta}}
\pysigstartsignatures
\pysiglinewithargsret{\sphinxcode{\sphinxupquote{deepdrr.projector.scatter.}}\sphinxbfcode{\sphinxupquote{sample\_Rayleigh\_theta}}}{\sphinxparam{\DUrole{n,n}{photon\_energy}\DUrole{p,p}{:}\DUrole{w,w}{  }\DUrole{n,n}{float32}}\sphinxparamcomma \sphinxparam{\DUrole{n,n}{rayleigh\_sampler}\DUrole{p,p}{:}\DUrole{w,w}{  }\DUrole{n,n}{{\hyperref[\detokenize{deepdrr.projector:deepdrr.projector.rita.RITA}]{\sphinxcrossref{RITA}}}}}}{{ $\rightarrow$ float32}}
\pysigstopsignatures
\sphinxAtStartPar
Randomly sample values of theta for a given Rayleigh scatter interaction
Based on page 49 of paper ‘PENELOPE\sphinxhyphen{}2006: A Code System for Monte Carlo Simulation of Electron and Photon Transport’

\sphinxAtStartPar
Note that the materials files distributed with MC\sphinxhyphen{}GPU\_v1.3 (\sphinxurl{https://code.google.com/archive/p/mcgpu/downloads}) uses
Form Factor data from PENELOPE\sphinxhyphen{}2006 files.  Accordingly, the (unnormalized) PDF is factored as:
\begin{quote}

\sphinxAtStartPar
p\_\{Ra\}(cos theta) = g(cos theta){[}F(x,Z){]}\textasciicircum{}2
\end{quote}
\begin{description}
\sphinxlineitem{not}
\sphinxAtStartPar
p\_\{Ra\}(cos theta) = g(cos theta){[}F(q,Z){]}\textasciicircum{}2

\end{description}

\sphinxAtStartPar
Accordingly, we compute cos(theta) using the x\sphinxhyphen{}values, not the q\sphinxhyphen{}values
\begin{quote}\begin{description}
\sphinxlineitem{Parameters}\begin{itemize}
\item {} 
\sphinxAtStartPar
\sphinxstyleliteralstrong{\sphinxupquote{photon\_energy}} (\sphinxstyleliteralemphasis{\sphinxupquote{np.float32}}) \textendash{} the energy of the incoming photon

\item {} 
\sphinxAtStartPar
\sphinxstyleliteralstrong{\sphinxupquote{rayleigh\_sampler}} ({\hyperref[\detokenize{deepdrr.projector:deepdrr.projector.rita.RITA}]{\sphinxcrossref{\sphinxstyleliteralemphasis{\sphinxupquote{RITA}}}}}) \textendash{} the RITA sampler object for the material at the location of the interaction

\end{itemize}

\sphinxlineitem{Returns}
\sphinxAtStartPar
cos(theta), where theta is the polar scattering angle

\sphinxlineitem{Return type}
\sphinxAtStartPar
np.float32

\end{description}\end{quote}

\end{fulllineitems}

\index{sample\_U01() (in module deepdrr.projector.scatter)@\spxentry{sample\_U01()}\spxextra{in module deepdrr.projector.scatter}}

\begin{fulllineitems}
\phantomsection\label{\detokenize{deepdrr.projector:deepdrr.projector.scatter.sample_U01}}
\pysigstartsignatures
\pysiglinewithargsret{\sphinxcode{\sphinxupquote{deepdrr.projector.scatter.}}\sphinxbfcode{\sphinxupquote{sample\_U01}}}{}{{ $\rightarrow$ float32}}
\pysigstopsignatures
\sphinxAtStartPar
Returns a value uniformly sampled from the interval {[}0,1{]}

\end{fulllineitems}

\index{sample\_initial\_direction() (in module deepdrr.projector.scatter)@\spxentry{sample\_initial\_direction()}\spxextra{in module deepdrr.projector.scatter}}

\begin{fulllineitems}
\phantomsection\label{\detokenize{deepdrr.projector:deepdrr.projector.scatter.sample_initial_direction}}
\pysigstartsignatures
\pysiglinewithargsret{\sphinxcode{\sphinxupquote{deepdrr.projector.scatter.}}\sphinxbfcode{\sphinxupquote{sample\_initial\_direction}}}{}{{ $\rightarrow$ {\hyperref[\detokenize{deepdrr.geo:deepdrr.geo.core.Vector3D}]{\sphinxcrossref{Vector3D}}}}}
\pysigstopsignatures
\sphinxAtStartPar
Returns an initial direction vector for a photon, uniformly distributed over the unit sphere.
\begin{quote}\begin{description}
\sphinxlineitem{Returns}
\sphinxAtStartPar
the initial direction unit vector (dx, dy, dz)\textasciicircum{}T

\sphinxlineitem{Return type}
\sphinxAtStartPar
{\hyperref[\detokenize{deepdrr.geo:deepdrr.geo.Vector3D}]{\sphinxcrossref{geo.Vector3D}}}

\end{description}\end{quote}

\end{fulllineitems}

\index{sample\_initial\_energy() (in module deepdrr.projector.scatter)@\spxentry{sample\_initial\_energy()}\spxextra{in module deepdrr.projector.scatter}}

\begin{fulllineitems}
\phantomsection\label{\detokenize{deepdrr.projector:deepdrr.projector.scatter.sample_initial_energy}}
\pysigstartsignatures
\pysiglinewithargsret{\sphinxcode{\sphinxupquote{deepdrr.projector.scatter.}}\sphinxbfcode{\sphinxupquote{sample\_initial\_energy}}}{\sphinxparam{\DUrole{n,n}{spectrum}\DUrole{p,p}{:}\DUrole{w,w}{  }\DUrole{n,n}{ndarray}}}{{ $\rightarrow$ float32}}
\pysigstopsignatures
\sphinxAtStartPar
Determine the energy (in eV) of a photon emitted by an X\sphinxhyphen{}Ray source with the given spectrum
\begin{quote}\begin{description}
\sphinxlineitem{Parameters}
\sphinxAtStartPar
\sphinxstyleliteralstrong{\sphinxupquote{spectrum}} (\sphinxstyleliteralemphasis{\sphinxupquote{np.ndarray}}) \textendash{} the data associated with the spectrum.  Cross\sphinxhyphen{}reference spectral\_data.py

\sphinxlineitem{Returns}
\sphinxAtStartPar
the energy of a photon, in eV

\sphinxlineitem{Return type}
\sphinxAtStartPar
np.float32

\end{description}\end{quote}

\end{fulllineitems}

\index{simulate\_scatter\_no\_vr() (in module deepdrr.projector.scatter)@\spxentry{simulate\_scatter\_no\_vr()}\spxextra{in module deepdrr.projector.scatter}}

\begin{fulllineitems}
\phantomsection\label{\detokenize{deepdrr.projector:deepdrr.projector.scatter.simulate_scatter_no_vr}}
\pysigstartsignatures
\pysiglinewithargsret{\sphinxcode{\sphinxupquote{deepdrr.projector.scatter.}}\sphinxbfcode{\sphinxupquote{simulate\_scatter\_no\_vr}}}{\sphinxparam{\DUrole{n,n}{volume}\DUrole{p,p}{:}\DUrole{w,w}{  }\DUrole{n,n}{{\hyperref[\detokenize{deepdrr.vol:deepdrr.vol.volume.Volume}]{\sphinxcrossref{Volume}}}}}\sphinxparamcomma \sphinxparam{\DUrole{n,n}{source\_ijk}\DUrole{p,p}{:}\DUrole{w,w}{  }\DUrole{n,n}{{\hyperref[\detokenize{deepdrr.geo:deepdrr.geo.core.Point3D}]{\sphinxcrossref{Point3D}}}}}\sphinxparamcomma \sphinxparam{\DUrole{n,n}{rt\_kinv}\DUrole{p,p}{:}\DUrole{w,w}{  }\DUrole{n,n}{ndarray}}\sphinxparamcomma \sphinxparam{\DUrole{n,n}{camera\_intrinsics}\DUrole{p,p}{:}\DUrole{w,w}{  }\DUrole{n,n}{{\hyperref[\detokenize{deepdrr.geo:deepdrr.geo.core.CameraIntrinsicTransform}]{\sphinxcrossref{CameraIntrinsicTransform}}}}}\sphinxparamcomma \sphinxparam{\DUrole{n,n}{source\_to\_detector\_distance}\DUrole{p,p}{:}\DUrole{w,w}{  }\DUrole{n,n}{float}}\sphinxparamcomma \sphinxparam{\DUrole{n,n}{index\_from\_ijk}\DUrole{p,p}{:}\DUrole{w,w}{  }\DUrole{n,n}{ndarray}}\sphinxparamcomma \sphinxparam{\DUrole{n,n}{sensor\_size}\DUrole{p,p}{:}\DUrole{w,w}{  }\DUrole{n,n}{Tuple\DUrole{p,p}{{[}}int\DUrole{p,p}{,}\DUrole{w,w}{  }int\DUrole{p,p}{{]}}}}\sphinxparamcomma \sphinxparam{\DUrole{n,n}{photon\_count}\DUrole{p,p}{:}\DUrole{w,w}{  }\DUrole{n,n}{int}}\sphinxparamcomma \sphinxparam{\DUrole{n,n}{mfp\_woodcock}\DUrole{p,p}{:}\DUrole{w,w}{  }\DUrole{n,n}{ndarray}}\sphinxparamcomma \sphinxparam{\DUrole{n,n}{spectrum}\DUrole{p,p}{:}\DUrole{w,w}{  }\DUrole{n,n}{ndarray\DUrole{w,w}{  }\DUrole{p,p}{|}\DUrole{w,w}{  }None}\DUrole{w,w}{  }\DUrole{o,o}{=}\DUrole{w,w}{  }\DUrole{default_value}{array({[}{[}1.50000e+04, 1.04178e+01{]}, {[}1.55000e+04, 2.39995e+01{]}, {[}1.60000e+04, 5.05336e+01{]}, {[}1.65000e+04, 9.64766e+01{]}, {[}1.70000e+04, 1.74507e+02{]}, {[}1.75000e+04, 2.91945e+02{]}, {[}1.80000e+04, 4.66386e+02{]}, {[}1.85000e+04, 7.11402e+02{]}, {[}1.90000e+04, 1.03605e+03{]}, {[}1.95000e+04, 1.46023e+03{]}, {[}2.00000e+04, 1.98617e+03{]}, {[}2.05000e+04, 2.61021e+03{]}, {[}2.10000e+04, 3.38544e+03{]}, {[}2.15000e+04, 4.24095e+03{]}, {[}2.20000e+04, 5.24354e+03{]}, {[}2.25000e+04, 6.32791e+03{]}, {[}2.30000e+04, 7.54120e+03{]}, {[}2.35000e+04, 8.77681e+03{]}, {[}2.40000e+04, 1.01946e+04{]}, {[}2.45000e+04, 1.15664e+04{]}, {[}2.50000e+04, 1.29585e+04{]}, {[}2.55000e+04, 1.44935e+04{]}, {[}2.60000e+04, 1.60105e+04{]}, {[}2.65000e+04, 1.74699e+04{]}, {[}2.70000e+04, 1.90277e+04{]}, {[}2.75000e+04, 2.04829e+04{]}, {[}2.80000e+04, 2.20221e+04{]}, {[}2.85000e+04, 2.33956e+04{]}, {[}2.90000e+04, 2.48280e+04{]}, {[}2.95000e+04, 2.60390e+04{]}, {[}3.00000e+04, 2.72837e+04{]}, {[}3.05000e+04, 2.85632e+04{]}, {[}3.10000e+04, 2.95578e+04{]}, {[}3.15000e+04, 3.06992e+04{]}, {[}3.20000e+04, 3.16468e+04{]}, {[}3.25000e+04, 3.25768e+04{]}, {[}3.30000e+04, 3.34075e+04{]}, {[}3.35000e+04, 3.41587e+04{]}, {[}3.40000e+04, 3.48395e+04{]}, {[}3.45000e+04, 3.54891e+04{]}, {[}3.50000e+04, 3.60055e+04{]}, {[}3.55000e+04, 3.65150e+04{]}, {[}3.60000e+04, 3.69445e+04{]}, {[}3.65000e+04, 3.72769e+04{]}, {[}3.70000e+04, 3.76006e+04{]}, {[}3.75000e+04, 3.78407e+04{]}, {[}3.80000e+04, 3.80723e+04{]}, {[}3.85000e+04, 3.82196e+04{]}, {[}3.90000e+04, 3.83128e+04{]}, {[}3.95000e+04, 3.83984e+04{]}, {[}4.00000e+04, 3.84307e+04{]}, {[}4.05000e+04, 3.84100e+04{]}, {[}4.10000e+04, 3.83524e+04{]}, {[}4.15000e+04, 3.82432e+04{]}, {[}4.20000e+04, 3.81287e+04{]}, {[}4.25000e+04, 3.80096e+04{]}, {[}4.30000e+04, 3.78105e+04{]}, {[}4.35000e+04, 3.76531e+04{]}, {[}4.40000e+04, 3.74023e+04{]}, {[}4.45000e+04, 3.71790e+04{]}, {[}4.50000e+04, 3.68942e+04{]}, {[}4.55000e+04, 3.66373e+04{]}, {[}4.60000e+04, 3.63352e+04{]}, {[}4.65000e+04, 3.59890e+04{]}, {[}4.70000e+04, 3.57000e+04{]}, {[}4.75000e+04, 3.53536e+04{]}, {[}4.80000e+04, 3.49795e+04{]}, {[}4.85000e+04, 3.46479e+04{]}, {[}4.90000e+04, 3.42752e+04{]}, {[}4.95000e+04, 3.38628e+04{]}, {[}5.00000e+04, 3.34899e+04{]}, {[}5.05000e+04, 3.30836e+04{]}, {[}5.10000e+04, 3.26791e+04{]}, {[}5.15000e+04, 3.22751e+04{]}, {[}5.20000e+04, 3.18398e+04{]}, {[}5.25000e+04, 3.14410e+04{]}, {[}5.30000e+04, 3.10103e+04{]}, {[}5.35000e+04, 3.05821e+04{]}, {[}5.40000e+04, 3.01563e+04{]}, {[}5.45000e+04, 2.97005e+04{]}, {[}5.50000e+04, 2.92796e+04{]}, {[}5.55000e+04, 2.88296e+04{]}, {[}5.60000e+04, 2.83824e+04{]}, {[}5.65000e+04, 2.79381e+04{]}, {[}5.70000e+04, 2.74965e+04{]}, {[}5.75000e+04, 2.70575e+04{]}, {[}5.80000e+04, 6.96222e+04{]}, {[}5.85000e+04, 2.61584e+04{]}, {[}5.90000e+04, 2.57269e+04{]}, {[}5.95000e+04, 1.01222e+05{]}, {[}6.00000e+04, 2.48164e+04{]}, {[}6.05000e+04, 2.43656e+04{]}, {[}6.10000e+04, 2.39174e+04{]}, {[}6.15000e+04, 2.34718e+04{]}, {[}6.20000e+04, 2.30288e+04{]}, {[}6.25000e+04, 2.25881e+04{]}, {[}6.30000e+04, 2.21258e+04{]}, {[}6.35000e+04, 2.16900e+04{]}, {[}6.40000e+04, 2.12564e+04{]}, {[}6.45000e+04, 2.08023e+04{]}, {[}6.50000e+04, 2.03509e+04{]}, {[}6.55000e+04, 1.99230e+04{]}, {[}6.60000e+04, 1.94765e+04{]}, {[}6.65000e+04, 1.90326e+04{]}, {[}6.70000e+04, 4.47529e+04{]}, {[}6.75000e+04, 1.81519e+04{]}, {[}6.80000e+04, 1.77151e+04{]}, {[}6.85000e+04, 1.72804e+04{]}, {[}6.90000e+04, 2.37937e+04{]}, {[}6.95000e+04, 1.61531e+04{]}, {[}7.00000e+04, 1.44526e+04{]}, {[}7.05000e+04, 1.41230e+04{]}, {[}7.10000e+04, 1.37761e+04{]}, {[}7.15000e+04, 1.34418e+04{]}, {[}7.20000e+04, 1.31051e+04{]}, {[}7.25000e+04, 1.27522e+04{]}, {[}7.30000e+04, 1.24112e+04{]}, {[}7.35000e+04, 1.20550e+04{]}, {[}7.40000e+04, 1.17100e+04{]}, {[}7.45000e+04, 1.13507e+04{]}, {[}7.50000e+04, 1.10021e+04{]}, {[}7.55000e+04, 1.06401e+04{]}, {[}7.60000e+04, 1.02771e+04{]}, {[}7.65000e+04, 9.92380e+03{]}, {[}7.70000e+04, 9.55851e+03{]}, {[}7.75000e+04, 9.19241e+03{]}, {[}7.80000e+04, 8.83513e+03{]}, {[}7.85000e+04, 8.46727e+03{]}, {[}7.90000e+04, 8.09884e+03{]}, {[}7.95000e+04, 7.72989e+03{]}, {[}8.00000e+04, 7.36053e+03{]}, {[}8.05000e+04, 6.99082e+03{]}, {[}8.10000e+04, 6.62085e+03{]}, {[}8.15000e+04, 6.25067e+03{]}, {[}8.20000e+04, 5.88674e+03{]}, {[}8.25000e+04, 5.51600e+03{]}, {[}8.30000e+04, 5.14530e+03{]}, {[}8.35000e+04, 4.77470e+03{]}, {[}8.40000e+04, 4.39953e+03{]}, {[}8.45000e+04, 4.02977e+03{]}, {[}8.50000e+04, 3.66033e+03{]}, {[}8.55000e+04, 3.29129e+03{]}, {[}8.60000e+04, 2.92273e+03{]}, {[}8.65000e+04, 2.55471e+03{]}, {[}8.70000e+04, 2.18731e+03{]}, {[}8.75000e+04, 1.82062e+03{]}, {[}8.80000e+04, 1.45468e+03{]}, {[}8.85000e+04, 1.08841e+03{]}, {[}8.90000e+04, 7.24621e+02{]}, {[}8.95000e+04, 3.61821e+02{]}, {[}9.00000e+04, \sphinxhyphen{}3.01387e+01{]}{]})}}\sphinxparamcomma \sphinxparam{\DUrole{n,n}{E\_abs}\DUrole{p,p}{:}\DUrole{w,w}{  }\DUrole{n,n}{float32\DUrole{w,w}{  }\DUrole{p,p}{|}\DUrole{w,w}{  }None}\DUrole{w,w}{  }\DUrole{o,o}{=}\DUrole{w,w}{  }\DUrole{default_value}{5000}}}{{ $\rightarrow$ ndarray}}
\pysigstopsignatures
\sphinxAtStartPar
Produce a grayscale (intensity\sphinxhyphen{}based) image representing the photon scatter during an X\sphinxhyphen{}Ray,
without using VR (variance reduction) techniques.
\begin{quote}\begin{description}
\sphinxlineitem{Parameters}\begin{itemize}
\item {} 
\sphinxAtStartPar
\sphinxstyleliteralstrong{\sphinxupquote{volume}} (\sphinxstyleliteralemphasis{\sphinxupquote{np.ndarray}}) \textendash{} the volume density data.

\item {} 
\sphinxAtStartPar
\sphinxstyleliteralstrong{\sphinxupquote{source\_ijk}} ({\hyperref[\detokenize{deepdrr.geo:deepdrr.geo.Point3D}]{\sphinxcrossref{\sphinxstyleliteralemphasis{\sphinxupquote{geo.Point3D}}}}}) \textendash{} the source point for rays in the camera’s IJK space

\item {} 
\sphinxAtStartPar
\sphinxstyleliteralstrong{\sphinxupquote{rt\_kinv}} (\sphinxstyleliteralemphasis{\sphinxupquote{np.ndarray}}) \textendash{} the ray transform for the projection.  Transforms pixel indices (u,v,1) to IJK vector along ray from from the X\sphinxhyphen{}Ray source to the detector pixel {[}u,v{]}.

\item {} 
\sphinxAtStartPar
\sphinxstyleliteralstrong{\sphinxupquote{camera\_intrinsics}} ({\hyperref[\detokenize{deepdrr.geo:deepdrr.geo.CameraIntrinsicTransform}]{\sphinxcrossref{\sphinxstyleliteralemphasis{\sphinxupquote{geo.CameraIntrinsicTransform}}}}}) \textendash{} the C\sphinxhyphen{}Arm “camera” intrinsic transform.  Used to calculate the detector plane.

\item {} 
\sphinxAtStartPar
\sphinxstyleliteralstrong{\sphinxupquote{source\_to\_detector\_distance}} (\sphinxstyleliteralemphasis{\sphinxupquote{float}}) \textendash{} distance from source to detector in millimeters.

\item {} 
\sphinxAtStartPar
\sphinxstyleliteralstrong{\sphinxupquote{index\_from\_ijk}} (\sphinxstyleliteralemphasis{\sphinxupquote{np.ndarray}}) \textendash{} the inverse transformation of ijk\_from\_index.  Takes 3D IJK coordinates and transforms to 2D pixel coordinates

\item {} 
\sphinxAtStartPar
\sphinxstyleliteralstrong{\sphinxupquote{sensor\_size}} (\sphinxstyleliteralemphasis{\sphinxupquote{Tuple}}\sphinxstyleliteralemphasis{\sphinxupquote{{[}}}\sphinxstyleliteralemphasis{\sphinxupquote{int}}\sphinxstyleliteralemphasis{\sphinxupquote{,}}\sphinxstyleliteralemphasis{\sphinxupquote{int}}\sphinxstyleliteralemphasis{\sphinxupquote{{]}}}) \textendash{} the sensor size \{width\}x\{height\}, in pixels, of the detector

\item {} 
\sphinxAtStartPar
\sphinxstyleliteralstrong{\sphinxupquote{photon\_count}} (\sphinxstyleliteralemphasis{\sphinxupquote{int}}) \textendash{} the number of photons simulated.

\item {} 
\sphinxAtStartPar
\sphinxstyleliteralstrong{\sphinxupquote{mfp\_woodcock}} (\sphinxstyleliteralemphasis{\sphinxupquote{np.ndarray}}) \textendash{} the Woodcock MFP data for the materials being simulated.  See make\_woodock\_mfp(…).

\item {} 
\sphinxAtStartPar
\sphinxstyleliteralstrong{\sphinxupquote{spectrum}} (\sphinxstyleliteralemphasis{\sphinxupquote{Optional}}\sphinxstyleliteralemphasis{\sphinxupquote{{[}}}\sphinxstyleliteralemphasis{\sphinxupquote{np.ndarray}}\sphinxstyleliteralemphasis{\sphinxupquote{{]}}}\sphinxstyleliteralemphasis{\sphinxupquote{, }}\sphinxstyleliteralemphasis{\sphinxupquote{optional}}) \textendash{} spectrum array.  Defaults to 90KV\_AL40 spectrum.

\item {} 
\sphinxAtStartPar
\sphinxstyleliteralstrong{\sphinxupquote{E\_abs}} (\sphinxstyleliteralemphasis{\sphinxupquote{Optional}}\sphinxstyleliteralemphasis{\sphinxupquote{{[}}}\sphinxstyleliteralemphasis{\sphinxupquote{np.float32}}\sphinxstyleliteralemphasis{\sphinxupquote{{]}}}\sphinxstyleliteralemphasis{\sphinxupquote{, }}\sphinxstyleliteralemphasis{\sphinxupquote{optional}}) \textendash{} the energy (in eV) at or below which photons are assumed to be absorbed by the materials.  Defaults to 5000 (eV).

\end{itemize}

\sphinxlineitem{Returns}
\sphinxAtStartPar
deposited\sphinxhyphen{}energy image of the photon scatter

\sphinxlineitem{Return type}
\sphinxAtStartPar
np.ndarray

\end{description}\end{quote}

\end{fulllineitems}

\index{track\_single\_photon\_no\_vr() (in module deepdrr.projector.scatter)@\spxentry{track\_single\_photon\_no\_vr()}\spxextra{in module deepdrr.projector.scatter}}

\begin{fulllineitems}
\phantomsection\label{\detokenize{deepdrr.projector:deepdrr.projector.scatter.track_single_photon_no_vr}}
\pysigstartsignatures
\pysiglinewithargsret{\sphinxcode{\sphinxupquote{deepdrr.projector.scatter.}}\sphinxbfcode{\sphinxupquote{track\_single\_photon\_no\_vr}}}{\sphinxparam{\DUrole{n,n}{initial\_pos}\DUrole{p,p}{:}\DUrole{w,w}{  }\DUrole{n,n}{{\hyperref[\detokenize{deepdrr.geo:deepdrr.geo.core.Point3D}]{\sphinxcrossref{Point3D}}}}}\sphinxparamcomma \sphinxparam{\DUrole{n,n}{initial\_dir}\DUrole{p,p}{:}\DUrole{w,w}{  }\DUrole{n,n}{{\hyperref[\detokenize{deepdrr.geo:deepdrr.geo.core.Vector3D}]{\sphinxcrossref{Vector3D}}}}}\sphinxparamcomma \sphinxparam{\DUrole{n,n}{initial\_E}\DUrole{p,p}{:}\DUrole{w,w}{  }\DUrole{n,n}{float32}}\sphinxparamcomma \sphinxparam{\DUrole{n,n}{E\_abs}\DUrole{p,p}{:}\DUrole{w,w}{  }\DUrole{n,n}{float32}}\sphinxparamcomma \sphinxparam{\DUrole{n,n}{labeled\_seg}\DUrole{p,p}{:}\DUrole{w,w}{  }\DUrole{n,n}{ndarray}}\sphinxparamcomma \sphinxparam{\DUrole{n,n}{volume\_shape}\DUrole{p,p}{:}\DUrole{w,w}{  }\DUrole{n,n}{Tuple\DUrole{p,p}{{[}}int\DUrole{p,p}{,}\DUrole{w,w}{  }int\DUrole{p,p}{,}\DUrole{w,w}{  }int\DUrole{p,p}{{]}}}}\sphinxparamcomma \sphinxparam{\DUrole{n,n}{detector\_plane}\DUrole{p,p}{:}\DUrole{w,w}{  }\DUrole{n,n}{{\hyperref[\detokenize{deepdrr.projector:deepdrr.projector.plane_surface.PlaneSurface}]{\sphinxcrossref{PlaneSurface}}}}}\sphinxparamcomma \sphinxparam{\DUrole{n,n}{index\_from\_ijk}\DUrole{p,p}{:}\DUrole{w,w}{  }\DUrole{n,n}{ndarray}}\sphinxparamcomma \sphinxparam{\DUrole{n,n}{source\_ijk}\DUrole{p,p}{:}\DUrole{w,w}{  }\DUrole{n,n}{{\hyperref[\detokenize{deepdrr.geo:deepdrr.geo.core.Point3D}]{\sphinxcrossref{Point3D}}}}}\sphinxparamcomma \sphinxparam{\DUrole{n,n}{source\_to\_detector\_distance}\DUrole{p,p}{:}\DUrole{w,w}{  }\DUrole{n,n}{float}}\sphinxparamcomma \sphinxparam{\DUrole{n,n}{mfp\_woodcock}\DUrole{p,p}{:}\DUrole{w,w}{  }\DUrole{n,n}{ndarray}}\sphinxparamcomma \sphinxparam{\DUrole{n,n}{material\_ids}\DUrole{p,p}{:}\DUrole{w,w}{  }\DUrole{n,n}{Dict\DUrole{p,p}{{[}}int\DUrole{p,p}{,}\DUrole{w,w}{  }str\DUrole{p,p}{{]}}}}}{{ $\rightarrow$ Tuple\DUrole{p,p}{{[}}int\DUrole{p,p}{,}\DUrole{w,w}{  }int\DUrole{p,p}{,}\DUrole{w,w}{  }float32\DUrole{p,p}{,}\DUrole{w,w}{  }int\DUrole{p,p}{{]}}}}
\pysigstopsignatures
\sphinxAtStartPar
Produce a grayscale (intensity\sphinxhyphen{}based) image representing the photon scatter of a single photon
during an X\sphinxhyphen{}Ray, without using VR (variance reduction) techniques.
\begin{quote}\begin{description}
\sphinxlineitem{Parameters}\begin{itemize}
\item {} 
\sphinxAtStartPar
\sphinxstyleliteralstrong{\sphinxupquote{initial\_pos}} ({\hyperref[\detokenize{deepdrr.geo:deepdrr.geo.Point3D}]{\sphinxcrossref{\sphinxstyleliteralemphasis{\sphinxupquote{geo.Point3D}}}}}) \textendash{} the initial position (in IJK space) of the photon once it has entered the volume.  This IS NOT the X\sphinxhyphen{}Ray source.  See function sample\_initial\_direction(…)

\item {} 
\sphinxAtStartPar
\sphinxstyleliteralstrong{\sphinxupquote{initial\_dir}} ({\hyperref[\detokenize{deepdrr.geo:deepdrr.geo.Vector3D}]{\sphinxcrossref{\sphinxstyleliteralemphasis{\sphinxupquote{geo.Vector3D}}}}}) \textendash{} the initial direction of travel of the photon, in IJK space

\item {} 
\sphinxAtStartPar
\sphinxstyleliteralstrong{\sphinxupquote{initital\_E}} (\sphinxstyleliteralemphasis{\sphinxupquote{np.float32}}) \textendash{} the initial energy of the photon

\item {} 
\sphinxAtStartPar
\sphinxstyleliteralstrong{\sphinxupquote{E\_abs}} (\sphinxstyleliteralemphasis{\sphinxupquote{np.float32}}) \textendash{} the energy (in eV) at or below which photons are assumed to be absorbed by the materials.

\item {} 
\sphinxAtStartPar
\sphinxstyleliteralstrong{\sphinxupquote{labeled\_seg}} (\sphinxstyleliteralemphasis{\sphinxupquote{np.ndarray}}) \textendash{} a {[}0..N\sphinxhyphen{}1{]}\sphinxhyphen{}labeled segmentation of the volume

\item {} 
\sphinxAtStartPar
\sphinxstyleliteralstrong{\sphinxupquote{density\_vol}} (\sphinxstyleliteralemphasis{\sphinxupquote{np.ndarray}}) \textendash{} the density information of the volume

\item {} 
\sphinxAtStartPar
\sphinxstyleliteralstrong{\sphinxupquote{detector\_plane}} (\sphinxstyleliteralemphasis{\sphinxupquote{np.ndarray}}) \textendash{} the ‘plane vector’ of the detector

\item {} 
\sphinxAtStartPar
\sphinxstyleliteralstrong{\sphinxupquote{index\_from\_ijk}} (\sphinxstyleliteralemphasis{\sphinxupquote{np.ndarray}}) \textendash{} the inverse transformation of ijk\_from\_index, the ray transform for the projection.

\item {} 
\sphinxAtStartPar
\sphinxstyleliteralstrong{\sphinxupquote{material\_ids}} (\sphinxstyleliteralemphasis{\sphinxupquote{Dict}}\sphinxstyleliteralemphasis{\sphinxupquote{{[}}}\sphinxstyleliteralemphasis{\sphinxupquote{int}}\sphinxstyleliteralemphasis{\sphinxupquote{,}}\sphinxstyleliteralemphasis{\sphinxupquote{str}}\sphinxstyleliteralemphasis{\sphinxupquote{{]}}}) \textendash{} a dictionary mapping an integer material ID\sphinxhyphen{}label to the name of the material

\end{itemize}

\sphinxlineitem{Returns}
\sphinxAtStartPar
\begin{description}
\sphinxlineitem{the pixel coord.s of the hit pixel, as well as the energy (in eV) of the photon when it hit the detector.}
\sphinxAtStartPar
The final int is the number of scatter events experienced by the photon.
Note that the returned pixel coord.s CAN BE out\sphinxhyphen{}of\sphinxhyphen{}bounds.

\end{description}


\sphinxlineitem{Return type}
\sphinxAtStartPar
Tuple{[}int, int, np.float32, int{]}

\end{description}\end{quote}

\end{fulllineitems}



\subsection{deepdrr.projector.spectral\_data module}
\label{\detokenize{deepdrr.projector:module-deepdrr.projector.spectral_data}}\label{\detokenize{deepdrr.projector:deepdrr-projector-spectral-data-module}}\index{module@\spxentry{module}!deepdrr.projector.spectral\_data@\spxentry{deepdrr.projector.spectral\_data}}\index{deepdrr.projector.spectral\_data@\spxentry{deepdrr.projector.spectral\_data}!module@\spxentry{module}}

\subsection{Module contents}
\label{\detokenize{deepdrr.projector:module-deepdrr.projector}}\label{\detokenize{deepdrr.projector:module-contents}}\index{module@\spxentry{module}!deepdrr.projector@\spxentry{deepdrr.projector}}\index{deepdrr.projector@\spxentry{deepdrr.projector}!module@\spxentry{module}}\index{Projector (class in deepdrr.projector)@\spxentry{Projector}\spxextra{class in deepdrr.projector}}

\begin{fulllineitems}
\phantomsection\label{\detokenize{deepdrr.projector:deepdrr.projector.Projector}}
\pysigstartsignatures
\pysiglinewithargsret{\sphinxbfcode{\sphinxupquote{class\DUrole{w,w}{  }}}\sphinxcode{\sphinxupquote{deepdrr.projector.}}\sphinxbfcode{\sphinxupquote{Projector}}}{\sphinxparam{\DUrole{n,n}{volume}\DUrole{p,p}{:}\DUrole{w,w}{  }\DUrole{n,n}{{\hyperref[\detokenize{deepdrr.vol:deepdrr.vol.volume.Volume}]{\sphinxcrossref{Volume}}}\DUrole{w,w}{  }\DUrole{p,p}{|}\DUrole{w,w}{  }List\DUrole{p,p}{{[}}{\hyperref[\detokenize{deepdrr.vol:deepdrr.vol.volume.Volume}]{\sphinxcrossref{Volume}}}\DUrole{p,p}{{]}}}}\sphinxparamcomma \sphinxparam{\DUrole{n,n}{priorities}\DUrole{p,p}{:}\DUrole{w,w}{  }\DUrole{n,n}{List\DUrole{p,p}{{[}}int\DUrole{p,p}{{]}}\DUrole{w,w}{  }\DUrole{p,p}{|}\DUrole{w,w}{  }None}\DUrole{w,w}{  }\DUrole{o,o}{=}\DUrole{w,w}{  }\DUrole{default_value}{None}}\sphinxparamcomma \sphinxparam{\DUrole{n,n}{camera\_intrinsics}\DUrole{p,p}{:}\DUrole{w,w}{  }\DUrole{n,n}{{\hyperref[\detokenize{deepdrr.geo:deepdrr.geo.core.CameraIntrinsicTransform}]{\sphinxcrossref{CameraIntrinsicTransform}}}\DUrole{w,w}{  }\DUrole{p,p}{|}\DUrole{w,w}{  }None}\DUrole{w,w}{  }\DUrole{o,o}{=}\DUrole{w,w}{  }\DUrole{default_value}{None}}\sphinxparamcomma \sphinxparam{\DUrole{n,n}{device}\DUrole{p,p}{:}\DUrole{w,w}{  }\DUrole{n,n}{{\hyperref[\detokenize{deepdrr.device:deepdrr.device.device.Device}]{\sphinxcrossref{Device}}}\DUrole{w,w}{  }\DUrole{p,p}{|}\DUrole{w,w}{  }None}\DUrole{w,w}{  }\DUrole{o,o}{=}\DUrole{w,w}{  }\DUrole{default_value}{None}}\sphinxparamcomma \sphinxparam{\DUrole{n,n}{step}\DUrole{p,p}{:}\DUrole{w,w}{  }\DUrole{n,n}{float}\DUrole{w,w}{  }\DUrole{o,o}{=}\DUrole{w,w}{  }\DUrole{default_value}{0.1}}\sphinxparamcomma \sphinxparam{\DUrole{n,n}{mode}\DUrole{p,p}{:}\DUrole{w,w}{  }\DUrole{n,n}{str}\DUrole{w,w}{  }\DUrole{o,o}{=}\DUrole{w,w}{  }\DUrole{default_value}{\textquotesingle{}linear\textquotesingle{}}}\sphinxparamcomma \sphinxparam{\DUrole{n,n}{spectrum}\DUrole{p,p}{:}\DUrole{w,w}{  }\DUrole{n,n}{ndarray\DUrole{w,w}{  }\DUrole{p,p}{|}\DUrole{w,w}{  }str}\DUrole{w,w}{  }\DUrole{o,o}{=}\DUrole{w,w}{  }\DUrole{default_value}{\textquotesingle{}90KV\_AL40\textquotesingle{}}}\sphinxparamcomma \sphinxparam{\DUrole{n,n}{add\_scatter}\DUrole{p,p}{:}\DUrole{w,w}{  }\DUrole{n,n}{bool\DUrole{w,w}{  }\DUrole{p,p}{|}\DUrole{w,w}{  }None}\DUrole{w,w}{  }\DUrole{o,o}{=}\DUrole{w,w}{  }\DUrole{default_value}{None}}\sphinxparamcomma \sphinxparam{\DUrole{n,n}{scatter\_num}\DUrole{p,p}{:}\DUrole{w,w}{  }\DUrole{n,n}{int}\DUrole{w,w}{  }\DUrole{o,o}{=}\DUrole{w,w}{  }\DUrole{default_value}{0}}\sphinxparamcomma \sphinxparam{\DUrole{n,n}{add\_noise}\DUrole{p,p}{:}\DUrole{w,w}{  }\DUrole{n,n}{bool}\DUrole{w,w}{  }\DUrole{o,o}{=}\DUrole{w,w}{  }\DUrole{default_value}{False}}\sphinxparamcomma \sphinxparam{\DUrole{n,n}{photon\_count}\DUrole{p,p}{:}\DUrole{w,w}{  }\DUrole{n,n}{int}\DUrole{w,w}{  }\DUrole{o,o}{=}\DUrole{w,w}{  }\DUrole{default_value}{10000}}\sphinxparamcomma \sphinxparam{\DUrole{n,n}{threads}\DUrole{p,p}{:}\DUrole{w,w}{  }\DUrole{n,n}{int}\DUrole{w,w}{  }\DUrole{o,o}{=}\DUrole{w,w}{  }\DUrole{default_value}{8}}\sphinxparamcomma \sphinxparam{\DUrole{n,n}{max\_block\_index}\DUrole{p,p}{:}\DUrole{w,w}{  }\DUrole{n,n}{int}\DUrole{w,w}{  }\DUrole{o,o}{=}\DUrole{w,w}{  }\DUrole{default_value}{1024}}\sphinxparamcomma \sphinxparam{\DUrole{n,n}{collected\_energy}\DUrole{p,p}{:}\DUrole{w,w}{  }\DUrole{n,n}{bool}\DUrole{w,w}{  }\DUrole{o,o}{=}\DUrole{w,w}{  }\DUrole{default_value}{False}}\sphinxparamcomma \sphinxparam{\DUrole{n,n}{neglog}\DUrole{p,p}{:}\DUrole{w,w}{  }\DUrole{n,n}{bool}\DUrole{w,w}{  }\DUrole{o,o}{=}\DUrole{w,w}{  }\DUrole{default_value}{True}}\sphinxparamcomma \sphinxparam{\DUrole{n,n}{intensity\_upper\_bound}\DUrole{p,p}{:}\DUrole{w,w}{  }\DUrole{n,n}{float\DUrole{w,w}{  }\DUrole{p,p}{|}\DUrole{w,w}{  }None}\DUrole{w,w}{  }\DUrole{o,o}{=}\DUrole{w,w}{  }\DUrole{default_value}{None}}\sphinxparamcomma \sphinxparam{\DUrole{n,n}{attenuate\_outside\_volume}\DUrole{p,p}{:}\DUrole{w,w}{  }\DUrole{n,n}{bool}\DUrole{w,w}{  }\DUrole{o,o}{=}\DUrole{w,w}{  }\DUrole{default_value}{False}}\sphinxparamcomma \sphinxparam{\DUrole{n,n}{source\_to\_detector\_distance}\DUrole{p,p}{:}\DUrole{w,w}{  }\DUrole{n,n}{float}\DUrole{w,w}{  }\DUrole{o,o}{=}\DUrole{w,w}{  }\DUrole{default_value}{\sphinxhyphen{}1}}\sphinxparamcomma \sphinxparam{\DUrole{n,n}{carm}\DUrole{p,p}{:}\DUrole{w,w}{  }\DUrole{n,n}{{\hyperref[\detokenize{deepdrr.device:deepdrr.device.device.Device}]{\sphinxcrossref{Device}}}\DUrole{w,w}{  }\DUrole{p,p}{|}\DUrole{w,w}{  }None}\DUrole{w,w}{  }\DUrole{o,o}{=}\DUrole{w,w}{  }\DUrole{default_value}{None}}}{}
\pysigstopsignatures
\sphinxAtStartPar
Bases: \sphinxcode{\sphinxupquote{object}}
\index{camera\_intrinsics (deepdrr.projector.Projector property)@\spxentry{camera\_intrinsics}\spxextra{deepdrr.projector.Projector property}}

\begin{fulllineitems}
\phantomsection\label{\detokenize{deepdrr.projector:deepdrr.projector.Projector.camera_intrinsics}}
\pysigstartsignatures
\pysigline{\sphinxbfcode{\sphinxupquote{property\DUrole{w,w}{  }}}\sphinxbfcode{\sphinxupquote{camera\_intrinsics}}\sphinxbfcode{\sphinxupquote{\DUrole{p,p}{:}\DUrole{w,w}{  }{\hyperref[\detokenize{deepdrr.geo:deepdrr.geo.core.CameraIntrinsicTransform}]{\sphinxcrossref{CameraIntrinsicTransform}}}}}}
\pysigstopsignatures
\end{fulllineitems}

\index{free() (deepdrr.projector.Projector method)@\spxentry{free()}\spxextra{deepdrr.projector.Projector method}}

\begin{fulllineitems}
\phantomsection\label{\detokenize{deepdrr.projector:deepdrr.projector.Projector.free}}
\pysigstartsignatures
\pysiglinewithargsret{\sphinxbfcode{\sphinxupquote{free}}}{}{}
\pysigstopsignatures
\sphinxAtStartPar
Free the allocated GPU memory.

\end{fulllineitems}

\index{initialize() (deepdrr.projector.Projector method)@\spxentry{initialize()}\spxextra{deepdrr.projector.Projector method}}

\begin{fulllineitems}
\phantomsection\label{\detokenize{deepdrr.projector:deepdrr.projector.Projector.initialize}}
\pysigstartsignatures
\pysiglinewithargsret{\sphinxbfcode{\sphinxupquote{initialize}}}{}{}
\pysigstopsignatures
\sphinxAtStartPar
Allocate GPU memory and transfer the volume, segmentations to GPU.

\end{fulllineitems}

\index{initialize\_output\_arrays() (deepdrr.projector.Projector method)@\spxentry{initialize\_output\_arrays()}\spxextra{deepdrr.projector.Projector method}}

\begin{fulllineitems}
\phantomsection\label{\detokenize{deepdrr.projector:deepdrr.projector.Projector.initialize_output_arrays}}
\pysigstartsignatures
\pysiglinewithargsret{\sphinxbfcode{\sphinxupquote{initialize\_output\_arrays}}}{\sphinxparam{\DUrole{n,n}{sensor\_size}\DUrole{p,p}{:}\DUrole{w,w}{  }\DUrole{n,n}{Tuple\DUrole{p,p}{{[}}int\DUrole{p,p}{,}\DUrole{w,w}{  }int\DUrole{p,p}{{]}}}}}{{ $\rightarrow$ None}}
\pysigstopsignatures
\sphinxAtStartPar
Allocate arrays dependent on the output size. Frees previously allocated arrays.

\sphinxAtStartPar
This may have to be called multiple times if the output size changes.

\end{fulllineitems}

\index{output\_size (deepdrr.projector.Projector property)@\spxentry{output\_size}\spxextra{deepdrr.projector.Projector property}}

\begin{fulllineitems}
\phantomsection\label{\detokenize{deepdrr.projector:deepdrr.projector.Projector.output_size}}
\pysigstartsignatures
\pysigline{\sphinxbfcode{\sphinxupquote{property\DUrole{w,w}{  }}}\sphinxbfcode{\sphinxupquote{output\_size}}\sphinxbfcode{\sphinxupquote{\DUrole{p,p}{:}\DUrole{w,w}{  }int}}}
\pysigstopsignatures
\end{fulllineitems}

\index{project() (deepdrr.projector.Projector method)@\spxentry{project()}\spxextra{deepdrr.projector.Projector method}}

\begin{fulllineitems}
\phantomsection\label{\detokenize{deepdrr.projector:deepdrr.projector.Projector.project}}
\pysigstartsignatures
\pysiglinewithargsret{\sphinxbfcode{\sphinxupquote{project}}}{\sphinxparam{\DUrole{o,o}{*}\DUrole{n,n}{camera\_projections}\DUrole{p,p}{:}\DUrole{w,w}{  }\DUrole{n,n}{{\hyperref[\detokenize{deepdrr.geo:deepdrr.geo.core.CameraProjection}]{\sphinxcrossref{CameraProjection}}}}}}{{ $\rightarrow$ ndarray}}
\pysigstopsignatures
\sphinxAtStartPar
Perform the projection.
\begin{quote}\begin{description}
\sphinxlineitem{Parameters}
\sphinxAtStartPar
\sphinxstyleliteralstrong{\sphinxupquote{camera\_projection}} \textendash{} any number of camera projections. If none are provided, the Projector uses the CArm device to obtain a camera projection.

\sphinxlineitem{Raises}
\sphinxAtStartPar
\sphinxstyleliteralstrong{\sphinxupquote{ValueError}} \textendash{} if no projections are provided and self.device is None.

\sphinxlineitem{Returns}
\sphinxAtStartPar
array of DRRs, after mass attenuation, etc.

\sphinxlineitem{Return type}
\sphinxAtStartPar
np.ndarray

\end{description}\end{quote}

\end{fulllineitems}

\index{project\_over\_carm\_range() (deepdrr.projector.Projector method)@\spxentry{project\_over\_carm\_range()}\spxextra{deepdrr.projector.Projector method}}

\begin{fulllineitems}
\phantomsection\label{\detokenize{deepdrr.projector:deepdrr.projector.Projector.project_over_carm_range}}
\pysigstartsignatures
\pysiglinewithargsret{\sphinxbfcode{\sphinxupquote{project\_over\_carm\_range}}}{\sphinxparam{\DUrole{n,n}{phi\_range}\DUrole{p,p}{:}\DUrole{w,w}{  }\DUrole{n,n}{Tuple\DUrole{p,p}{{[}}float\DUrole{p,p}{,}\DUrole{w,w}{  }float\DUrole{p,p}{,}\DUrole{w,w}{  }float\DUrole{p,p}{{]}}}}\sphinxparamcomma \sphinxparam{\DUrole{n,n}{theta\_range}\DUrole{p,p}{:}\DUrole{w,w}{  }\DUrole{n,n}{Tuple\DUrole{p,p}{{[}}float\DUrole{p,p}{,}\DUrole{w,w}{  }float\DUrole{p,p}{,}\DUrole{w,w}{  }float\DUrole{p,p}{{]}}}}\sphinxparamcomma \sphinxparam{\DUrole{n,n}{degrees}\DUrole{p,p}{:}\DUrole{w,w}{  }\DUrole{n,n}{bool}\DUrole{w,w}{  }\DUrole{o,o}{=}\DUrole{w,w}{  }\DUrole{default_value}{True}}}{{ $\rightarrow$ ndarray}}
\pysigstopsignatures
\sphinxAtStartPar
Project over a range of angles using the included CArm.

\sphinxAtStartPar
Ignores the CArm’s internal pose, except for its isocenter.

\end{fulllineitems}

\index{source\_to\_detector\_distance (deepdrr.projector.Projector property)@\spxentry{source\_to\_detector\_distance}\spxextra{deepdrr.projector.Projector property}}

\begin{fulllineitems}
\phantomsection\label{\detokenize{deepdrr.projector:deepdrr.projector.Projector.source_to_detector_distance}}
\pysigstartsignatures
\pysigline{\sphinxbfcode{\sphinxupquote{property\DUrole{w,w}{  }}}\sphinxbfcode{\sphinxupquote{source\_to\_detector\_distance}}\sphinxbfcode{\sphinxupquote{\DUrole{p,p}{:}\DUrole{w,w}{  }float}}}
\pysigstopsignatures
\end{fulllineitems}

\index{volume (deepdrr.projector.Projector property)@\spxentry{volume}\spxextra{deepdrr.projector.Projector property}}

\begin{fulllineitems}
\phantomsection\label{\detokenize{deepdrr.projector:deepdrr.projector.Projector.volume}}
\pysigstartsignatures
\pysigline{\sphinxbfcode{\sphinxupquote{property\DUrole{w,w}{  }}}\sphinxbfcode{\sphinxupquote{volume}}}
\pysigstopsignatures
\end{fulllineitems}

\index{volumes (deepdrr.projector.Projector attribute)@\spxentry{volumes}\spxextra{deepdrr.projector.Projector attribute}}

\begin{fulllineitems}
\phantomsection\label{\detokenize{deepdrr.projector:deepdrr.projector.Projector.volumes}}
\pysigstartsignatures
\pysigline{\sphinxbfcode{\sphinxupquote{volumes}}\sphinxbfcode{\sphinxupquote{\DUrole{p,p}{:}\DUrole{w,w}{  }List\DUrole{p,p}{{[}}{\hyperref[\detokenize{deepdrr.vol:deepdrr.vol.volume.Volume}]{\sphinxcrossref{Volume}}}\DUrole{p,p}{{]}}}}}
\pysigstopsignatures
\end{fulllineitems}


\end{fulllineitems}


\sphinxstepscope


\section{deepdrr.utils package}
\label{\detokenize{deepdrr.utils:deepdrr-utils-package}}\label{\detokenize{deepdrr.utils::doc}}

\subsection{deepdrr.utils.data\_utils module}
\label{\detokenize{deepdrr.utils:module-deepdrr.utils.data_utils}}\label{\detokenize{deepdrr.utils:deepdrr-utils-data-utils-module}}\index{module@\spxentry{module}!deepdrr.utils.data\_utils@\spxentry{deepdrr.utils.data\_utils}}\index{deepdrr.utils.data\_utils@\spxentry{deepdrr.utils.data\_utils}!module@\spxentry{module}}\index{deepdrr\_data\_dir() (in module deepdrr.utils.data\_utils)@\spxentry{deepdrr\_data\_dir()}\spxextra{in module deepdrr.utils.data\_utils}}

\begin{fulllineitems}
\phantomsection\label{\detokenize{deepdrr.utils:deepdrr.utils.data_utils.deepdrr_data_dir}}
\pysigstartsignatures
\pysiglinewithargsret{\sphinxcode{\sphinxupquote{deepdrr.utils.data\_utils.}}\sphinxbfcode{\sphinxupquote{deepdrr\_data\_dir}}}{}{{ $\rightarrow$ Path}}
\pysigstopsignatures
\sphinxAtStartPar
Get the data directory for DeepDRR.

\sphinxAtStartPar
The data directory is determined by the environment variable \sphinxtitleref{DEEPDRR\_DATA\_DIR} if it exists.
Otherwise, it is \sphinxtitleref{\textasciitilde{}/datasets/DeepDRR}. If the directory does not exist, it is created.
\begin{quote}\begin{description}
\sphinxlineitem{Returns}
\sphinxAtStartPar
The data directory.

\sphinxlineitem{Return type}
\sphinxAtStartPar
Path

\end{description}\end{quote}

\end{fulllineitems}

\index{download() (in module deepdrr.utils.data\_utils)@\spxentry{download()}\spxextra{in module deepdrr.utils.data\_utils}}

\begin{fulllineitems}
\phantomsection\label{\detokenize{deepdrr.utils:deepdrr.utils.data_utils.download}}
\pysigstartsignatures
\pysiglinewithargsret{\sphinxcode{\sphinxupquote{deepdrr.utils.data\_utils.}}\sphinxbfcode{\sphinxupquote{download}}}{\sphinxparam{\DUrole{n,n}{url}\DUrole{p,p}{:}\DUrole{w,w}{  }\DUrole{n,n}{str}}\sphinxparamcomma \sphinxparam{\DUrole{n,n}{filename}\DUrole{p,p}{:}\DUrole{w,w}{  }\DUrole{n,n}{str\DUrole{w,w}{  }\DUrole{p,p}{|}\DUrole{w,w}{  }None}\DUrole{w,w}{  }\DUrole{o,o}{=}\DUrole{w,w}{  }\DUrole{default_value}{None}}\sphinxparamcomma \sphinxparam{\DUrole{n,n}{root}\DUrole{p,p}{:}\DUrole{w,w}{  }\DUrole{n,n}{str\DUrole{w,w}{  }\DUrole{p,p}{|}\DUrole{w,w}{  }None}\DUrole{w,w}{  }\DUrole{o,o}{=}\DUrole{w,w}{  }\DUrole{default_value}{None}}\sphinxparamcomma \sphinxparam{\DUrole{n,n}{md5}\DUrole{p,p}{:}\DUrole{w,w}{  }\DUrole{n,n}{str\DUrole{w,w}{  }\DUrole{p,p}{|}\DUrole{w,w}{  }None}\DUrole{w,w}{  }\DUrole{o,o}{=}\DUrole{w,w}{  }\DUrole{default_value}{None}}\sphinxparamcomma \sphinxparam{\DUrole{n,n}{extract\_name}\DUrole{p,p}{:}\DUrole{w,w}{  }\DUrole{n,n}{str\DUrole{w,w}{  }\DUrole{p,p}{|}\DUrole{w,w}{  }None}\DUrole{w,w}{  }\DUrole{o,o}{=}\DUrole{w,w}{  }\DUrole{default_value}{None}}}{{ $\rightarrow$ Path}}
\pysigstopsignatures
\sphinxAtStartPar
Download a data file and place it in root.
\begin{quote}\begin{description}
\sphinxlineitem{Parameters}\begin{itemize}
\item {} 
\sphinxAtStartPar
\sphinxstyleliteralstrong{\sphinxupquote{url}} (\sphinxstyleliteralemphasis{\sphinxupquote{str}}) \textendash{} The download link.

\item {} 
\sphinxAtStartPar
\sphinxstyleliteralstrong{\sphinxupquote{filename}} (\sphinxstyleliteralemphasis{\sphinxupquote{str}}\sphinxstyleliteralemphasis{\sphinxupquote{, }}\sphinxstyleliteralemphasis{\sphinxupquote{optional}}) \textendash{} The name the save the file under. If None, uses the name from the URL. Defaults to None.

\item {} 
\sphinxAtStartPar
\sphinxstyleliteralstrong{\sphinxupquote{root}} (\sphinxstyleliteralemphasis{\sphinxupquote{str}}\sphinxstyleliteralemphasis{\sphinxupquote{, }}\sphinxstyleliteralemphasis{\sphinxupquote{optional}}) \textendash{} The directory to place downloaded data in. Can be overriden by setting the environment variable DEEPDRR\_DATA\_DIR. Defaults to “\textasciitilde{}/datasets/DeepDRR\_Data”.

\item {} 
\sphinxAtStartPar
\sphinxstyleliteralstrong{\sphinxupquote{md5}} (\sphinxstyleliteralemphasis{\sphinxupquote{str}}\sphinxstyleliteralemphasis{\sphinxupquote{, }}\sphinxstyleliteralemphasis{\sphinxupquote{optional}}) \textendash{} MD5 checksum of the download. Defaults to None.

\item {} 
\sphinxAtStartPar
\sphinxstyleliteralstrong{\sphinxupquote{extract\_name}} \textendash{} If not None, extract the downloaded file to \sphinxtitleref{root / extract\_name}.

\end{itemize}

\sphinxlineitem{Returns}
\sphinxAtStartPar
The path of the downloaded file, or the extracted directory.

\sphinxlineitem{Return type}
\sphinxAtStartPar
Path

\end{description}\end{quote}

\end{fulllineitems}

\index{jsonable() (in module deepdrr.utils.data\_utils)@\spxentry{jsonable()}\spxextra{in module deepdrr.utils.data\_utils}}

\begin{fulllineitems}
\phantomsection\label{\detokenize{deepdrr.utils:deepdrr.utils.data_utils.jsonable}}
\pysigstartsignatures
\pysiglinewithargsret{\sphinxcode{\sphinxupquote{deepdrr.utils.data\_utils.}}\sphinxbfcode{\sphinxupquote{jsonable}}}{\sphinxparam{\DUrole{n,n}{obj}\DUrole{p,p}{:}\DUrole{w,w}{  }\DUrole{n,n}{Any}}}{}
\pysigstopsignatures
\sphinxAtStartPar
Convert obj to a JSON\sphinxhyphen{}ready container or object.
:param obj:
:type obj: {[}type{]}

\end{fulllineitems}

\index{load\_fcsv() (in module deepdrr.utils.data\_utils)@\spxentry{load\_fcsv()}\spxextra{in module deepdrr.utils.data\_utils}}

\begin{fulllineitems}
\phantomsection\label{\detokenize{deepdrr.utils:deepdrr.utils.data_utils.load_fcsv}}
\pysigstartsignatures
\pysiglinewithargsret{\sphinxcode{\sphinxupquote{deepdrr.utils.data\_utils.}}\sphinxbfcode{\sphinxupquote{load\_fcsv}}}{\sphinxparam{\DUrole{n,n}{path}\DUrole{p,p}{:}\DUrole{w,w}{  }\DUrole{n,n}{str}}}{{ $\rightarrow$ Tuple\DUrole{p,p}{{[}}ndarray\DUrole{p,p}{,}\DUrole{w,w}{  }ndarray\DUrole{p,p}{{]}}}}
\pysigstopsignatures
\sphinxAtStartPar
Load a fcsv file.
\begin{quote}\begin{description}
\sphinxlineitem{Parameters}
\sphinxAtStartPar
\sphinxstyleliteralstrong{\sphinxupquote{path}} (\sphinxstyleliteralemphasis{\sphinxupquote{str}}) \textendash{} The path to the fcsv file.

\sphinxlineitem{Returns}
\sphinxAtStartPar
The points. Shape: (N, 3)
np.ndarray: The names of the points. Shape: (N,)

\sphinxlineitem{Return type}
\sphinxAtStartPar
np.ndarray

\end{description}\end{quote}

\end{fulllineitems}

\index{load\_json() (in module deepdrr.utils.data\_utils)@\spxentry{load\_json()}\spxextra{in module deepdrr.utils.data\_utils}}

\begin{fulllineitems}
\phantomsection\label{\detokenize{deepdrr.utils:deepdrr.utils.data_utils.load_json}}
\pysigstartsignatures
\pysiglinewithargsret{\sphinxcode{\sphinxupquote{deepdrr.utils.data\_utils.}}\sphinxbfcode{\sphinxupquote{load\_json}}}{\sphinxparam{\DUrole{n,n}{path}\DUrole{p,p}{:}\DUrole{w,w}{  }\DUrole{n,n}{str}}}{{ $\rightarrow$ Any}}
\pysigstopsignatures
\end{fulllineitems}

\index{save\_fcsv() (in module deepdrr.utils.data\_utils)@\spxentry{save\_fcsv()}\spxextra{in module deepdrr.utils.data\_utils}}

\begin{fulllineitems}
\phantomsection\label{\detokenize{deepdrr.utils:deepdrr.utils.data_utils.save_fcsv}}
\pysigstartsignatures
\pysiglinewithargsret{\sphinxcode{\sphinxupquote{deepdrr.utils.data\_utils.}}\sphinxbfcode{\sphinxupquote{save\_fcsv}}}{\sphinxparam{\DUrole{n,n}{path}\DUrole{p,p}{:}\DUrole{w,w}{  }\DUrole{n,n}{str}}\sphinxparamcomma \sphinxparam{\DUrole{n,n}{points}\DUrole{p,p}{:}\DUrole{w,w}{  }\DUrole{n,n}{ndarray}}\sphinxparamcomma \sphinxparam{\DUrole{n,n}{names}\DUrole{p,p}{:}\DUrole{w,w}{  }\DUrole{n,n}{List\DUrole{p,p}{{[}}str\DUrole{p,p}{{]}}\DUrole{w,w}{  }\DUrole{p,p}{|}\DUrole{w,w}{  }None}\DUrole{w,w}{  }\DUrole{o,o}{=}\DUrole{w,w}{  }\DUrole{default_value}{None}}\sphinxparamcomma \sphinxparam{\DUrole{n,n}{coordinate\_system}\DUrole{p,p}{:}\DUrole{w,w}{  }\DUrole{n,n}{str}\DUrole{w,w}{  }\DUrole{o,o}{=}\DUrole{w,w}{  }\DUrole{default_value}{\textquotesingle{}LPS\textquotesingle{}}}}{}
\pysigstopsignatures
\sphinxAtStartPar
Save a fcsv file.
\begin{quote}\begin{description}
\sphinxlineitem{Parameters}\begin{itemize}
\item {} 
\sphinxAtStartPar
\sphinxstyleliteralstrong{\sphinxupquote{path}} (\sphinxstyleliteralemphasis{\sphinxupquote{str}}) \textendash{} The path to save the file to.

\item {} 
\sphinxAtStartPar
\sphinxstyleliteralstrong{\sphinxupquote{points}} (\sphinxstyleliteralemphasis{\sphinxupquote{np.ndarray}}) \textendash{} The points to save. Shape: (N, 3)

\item {} 
\sphinxAtStartPar
\sphinxstyleliteralstrong{\sphinxupquote{names}} (\sphinxstyleliteralemphasis{\sphinxupquote{List}}\sphinxstyleliteralemphasis{\sphinxupquote{{[}}}\sphinxstyleliteralemphasis{\sphinxupquote{str}}\sphinxstyleliteralemphasis{\sphinxupquote{{]}}}) \textendash{} The names of the points. Shape: (N,)

\end{itemize}

\end{description}\end{quote}

\end{fulllineitems}

\index{save\_json() (in module deepdrr.utils.data\_utils)@\spxentry{save\_json()}\spxextra{in module deepdrr.utils.data\_utils}}

\begin{fulllineitems}
\phantomsection\label{\detokenize{deepdrr.utils:deepdrr.utils.data_utils.save_json}}
\pysigstartsignatures
\pysiglinewithargsret{\sphinxcode{\sphinxupquote{deepdrr.utils.data\_utils.}}\sphinxbfcode{\sphinxupquote{save\_json}}}{\sphinxparam{\DUrole{n,n}{path}\DUrole{p,p}{:}\DUrole{w,w}{  }\DUrole{n,n}{str}}\sphinxparamcomma \sphinxparam{\DUrole{n,n}{obj}\DUrole{p,p}{:}\DUrole{w,w}{  }\DUrole{n,n}{Any}}}{}
\pysigstopsignatures
\end{fulllineitems}



\subsection{deepdrr.utils.dicom\_utils module}
\label{\detokenize{deepdrr.utils:module-deepdrr.utils.dicom_utils}}\label{\detokenize{deepdrr.utils:deepdrr-utils-dicom-utils-module}}\index{module@\spxentry{module}!deepdrr.utils.dicom\_utils@\spxentry{deepdrr.utils.dicom\_utils}}\index{deepdrr.utils.dicom\_utils@\spxentry{deepdrr.utils.dicom\_utils}!module@\spxentry{module}}
\sphinxAtStartPar
Utils for dealing with DICOM images.
\index{find\_dicom() (in module deepdrr.utils.dicom\_utils)@\spxentry{find\_dicom()}\spxextra{in module deepdrr.utils.dicom\_utils}}

\begin{fulllineitems}
\phantomsection\label{\detokenize{deepdrr.utils:deepdrr.utils.dicom_utils.find_dicom}}
\pysigstartsignatures
\pysiglinewithargsret{\sphinxcode{\sphinxupquote{deepdrr.utils.dicom\_utils.}}\sphinxbfcode{\sphinxupquote{find\_dicom}}}{\sphinxparam{\DUrole{n,n}{image\_dir}\DUrole{p,p}{:}\DUrole{w,w}{  }\DUrole{n,n}{Path}}\sphinxparamcomma \sphinxparam{\DUrole{n,n}{device\_time}\DUrole{p,p}{:}\DUrole{w,w}{  }\DUrole{n,n}{datetime}}\sphinxparamcomma \sphinxparam{\DUrole{n,n}{max\_difference}\DUrole{p,p}{:}\DUrole{w,w}{  }\DUrole{n,n}{int\DUrole{w,w}{  }\DUrole{p,p}{|}\DUrole{w,w}{  }None}\DUrole{w,w}{  }\DUrole{o,o}{=}\DUrole{w,w}{  }\DUrole{default_value}{2}}}{{ $\rightarrow$ Path\DUrole{w,w}{  }\DUrole{p,p}{|}\DUrole{w,w}{  }None}}
\pysigstopsignatures
\sphinxAtStartPar
Get the image with the given acquisition time.

\sphinxAtStartPar
\# TODO: parallelize this to work for multiple provided device times, for efficiency.
\begin{quote}\begin{description}
\sphinxlineitem{Parameters}\begin{itemize}
\item {} 
\sphinxAtStartPar
\sphinxstyleliteralstrong{\sphinxupquote{image\_dir}} \textendash{} Path to the directory containing the images. All subtrees are searched, so make sure it’s not large.

\item {} 
\sphinxAtStartPar
\sphinxstyleliteralstrong{\sphinxupquote{device\_time}} (\sphinxstyleliteralemphasis{\sphinxupquote{str}}) \textendash{} Datetime object for the acquisition time.

\item {} 
\sphinxAtStartPar
\sphinxstyleliteralstrong{\sphinxupquote{max\_difference}} (\sphinxstyleliteralemphasis{\sphinxupquote{int}}\sphinxstyleliteralemphasis{\sphinxupquote{, }}\sphinxstyleliteralemphasis{\sphinxupquote{optional}}) \textendash{} Maximum difference in seconds between the given timestamp and the closest image. Defaults to 2.

\end{itemize}

\sphinxlineitem{Returns}
\sphinxAtStartPar
Path to the image.

\sphinxlineitem{Return type}
\sphinxAtStartPar
Path

\end{description}\end{quote}

\end{fulllineitems}

\index{get\_time() (in module deepdrr.utils.dicom\_utils)@\spxentry{get\_time()}\spxextra{in module deepdrr.utils.dicom\_utils}}

\begin{fulllineitems}
\phantomsection\label{\detokenize{deepdrr.utils:deepdrr.utils.dicom_utils.get_time}}
\pysigstartsignatures
\pysiglinewithargsret{\sphinxcode{\sphinxupquote{deepdrr.utils.dicom\_utils.}}\sphinxbfcode{\sphinxupquote{get\_time}}}{\sphinxparam{\DUrole{n,n}{path}\DUrole{p,p}{:}\DUrole{w,w}{  }\DUrole{n,n}{str}}}{{ $\rightarrow$ datetime\DUrole{w,w}{  }\DUrole{p,p}{|}\DUrole{w,w}{  }None}}
\pysigstopsignatures
\sphinxAtStartPar
Parse a dicom path to get the acquisition time.

\sphinxAtStartPar
Does not actually open the image.

\end{fulllineitems}

\index{read\_image() (in module deepdrr.utils.dicom\_utils)@\spxentry{read\_image()}\spxextra{in module deepdrr.utils.dicom\_utils}}

\begin{fulllineitems}
\phantomsection\label{\detokenize{deepdrr.utils:deepdrr.utils.dicom_utils.read_image}}
\pysigstartsignatures
\pysiglinewithargsret{\sphinxcode{\sphinxupquote{deepdrr.utils.dicom\_utils.}}\sphinxbfcode{\sphinxupquote{read\_image}}}{\sphinxparam{\DUrole{n,n}{path}\DUrole{p,p}{:}\DUrole{w,w}{  }\DUrole{n,n}{str}}}{{ $\rightarrow$ ndarray}}
\pysigstopsignatures
\sphinxAtStartPar
Load a DICOM image.
\begin{quote}\begin{description}
\sphinxlineitem{Parameters}
\sphinxAtStartPar
\sphinxstyleliteralstrong{\sphinxupquote{path}} \textendash{} Path to the DICOM file.

\sphinxlineitem{Returns}
\sphinxAtStartPar
The image as a float32 numpy array, in range {[}0, 1{]}.

\end{description}\end{quote}

\end{fulllineitems}



\subsection{deepdrr.utils.heatmap\_utils module}
\label{\detokenize{deepdrr.utils:module-deepdrr.utils.heatmap_utils}}\label{\detokenize{deepdrr.utils:deepdrr-utils-heatmap-utils-module}}\index{module@\spxentry{module}!deepdrr.utils.heatmap\_utils@\spxentry{deepdrr.utils.heatmap\_utils}}\index{deepdrr.utils.heatmap\_utils@\spxentry{deepdrr.utils.heatmap\_utils}!module@\spxentry{module}}\index{get\_threshold() (in module deepdrr.utils.heatmap\_utils)@\spxentry{get\_threshold()}\spxextra{in module deepdrr.utils.heatmap\_utils}}

\begin{fulllineitems}
\phantomsection\label{\detokenize{deepdrr.utils:deepdrr.utils.heatmap_utils.get_threshold}}
\pysigstartsignatures
\pysiglinewithargsret{\sphinxcode{\sphinxupquote{deepdrr.utils.heatmap\_utils.}}\sphinxbfcode{\sphinxupquote{get\_threshold}}}{\sphinxparam{\DUrole{n,n}{h}\DUrole{p,p}{:}\DUrole{w,w}{  }\DUrole{n,n}{ndarray}}\sphinxparamcomma \sphinxparam{\DUrole{n,n}{fraction}\DUrole{p,p}{:}\DUrole{w,w}{  }\DUrole{n,n}{float}\DUrole{w,w}{  }\DUrole{o,o}{=}\DUrole{w,w}{  }\DUrole{default_value}{0.5}}}{{ $\rightarrow$ float}}
\pysigstopsignatures
\sphinxAtStartPar
Get the threshold for a heatmap.
\begin{quote}\begin{description}
\sphinxlineitem{Parameters}\begin{itemize}
\item {} 
\sphinxAtStartPar
\sphinxstyleliteralstrong{\sphinxupquote{h}} (\sphinxstyleliteralemphasis{\sphinxupquote{np.ndarray}}) \textendash{} A 2D array

\item {} 
\sphinxAtStartPar
\sphinxstyleliteralstrong{\sphinxupquote{fraction}} (\sphinxstyleliteralemphasis{\sphinxupquote{float}}) \textendash{} Fraction of the heatmap range to set the threshold at. Higher values keeps fewer pixels.

\end{itemize}

\end{description}\end{quote}

\end{fulllineitems}



\subsection{deepdrr.utils.image\_utils module}
\label{\detokenize{deepdrr.utils:module-deepdrr.utils.image_utils}}\label{\detokenize{deepdrr.utils:deepdrr-utils-image-utils-module}}\index{module@\spxentry{module}!deepdrr.utils.image\_utils@\spxentry{deepdrr.utils.image\_utils}}\index{deepdrr.utils.image\_utils@\spxentry{deepdrr.utils.image\_utils}!module@\spxentry{module}}\index{as\_float32() (in module deepdrr.utils.image\_utils)@\spxentry{as\_float32()}\spxextra{in module deepdrr.utils.image\_utils}}

\begin{fulllineitems}
\phantomsection\label{\detokenize{deepdrr.utils:deepdrr.utils.image_utils.as_float32}}
\pysigstartsignatures
\pysiglinewithargsret{\sphinxcode{\sphinxupquote{deepdrr.utils.image\_utils.}}\sphinxbfcode{\sphinxupquote{as\_float32}}}{\sphinxparam{\DUrole{n,n}{image}\DUrole{p,p}{:}\DUrole{w,w}{  }\DUrole{n,n}{ndarray}}}{{ $\rightarrow$ ndarray}}
\pysigstopsignatures
\sphinxAtStartPar
Convert the image to float32.
\begin{quote}\begin{description}
\sphinxlineitem{Parameters}
\sphinxAtStartPar
\sphinxstyleliteralstrong{\sphinxupquote{image}} (\sphinxstyleliteralemphasis{\sphinxupquote{np.ndarray}}) \textendash{} the image to convert.

\sphinxlineitem{Returns}
\sphinxAtStartPar
the converted image.

\sphinxlineitem{Return type}
\sphinxAtStartPar
np.ndarray

\end{description}\end{quote}

\end{fulllineitems}

\index{as\_uint8() (in module deepdrr.utils.image\_utils)@\spxentry{as\_uint8()}\spxextra{in module deepdrr.utils.image\_utils}}

\begin{fulllineitems}
\phantomsection\label{\detokenize{deepdrr.utils:deepdrr.utils.image_utils.as_uint8}}
\pysigstartsignatures
\pysiglinewithargsret{\sphinxcode{\sphinxupquote{deepdrr.utils.image\_utils.}}\sphinxbfcode{\sphinxupquote{as\_uint8}}}{\sphinxparam{\DUrole{n,n}{image}\DUrole{p,p}{:}\DUrole{w,w}{  }\DUrole{n,n}{ndarray}}}{{ $\rightarrow$ ndarray}}
\pysigstopsignatures
\sphinxAtStartPar
Convert the image to uint8.
\begin{quote}\begin{description}
\sphinxlineitem{Parameters}
\sphinxAtStartPar
\sphinxstyleliteralstrong{\sphinxupquote{image}} (\sphinxstyleliteralemphasis{\sphinxupquote{np.ndarray}}) \textendash{} the image to convert.

\sphinxlineitem{Returns}
\sphinxAtStartPar
the converted image.

\sphinxlineitem{Return type}
\sphinxAtStartPar
np.ndarray

\end{description}\end{quote}

\end{fulllineitems}

\index{blend\_heatmaps() (in module deepdrr.utils.image\_utils)@\spxentry{blend\_heatmaps()}\spxextra{in module deepdrr.utils.image\_utils}}

\begin{fulllineitems}
\phantomsection\label{\detokenize{deepdrr.utils:deepdrr.utils.image_utils.blend_heatmaps}}
\pysigstartsignatures
\pysiglinewithargsret{\sphinxcode{\sphinxupquote{deepdrr.utils.image\_utils.}}\sphinxbfcode{\sphinxupquote{blend\_heatmaps}}}{\sphinxparam{\DUrole{n,n}{image}\DUrole{p,p}{:}\DUrole{w,w}{  }\DUrole{n,n}{ndarray}}\sphinxparamcomma \sphinxparam{\DUrole{n,n}{heatmaps}\DUrole{p,p}{:}\DUrole{w,w}{  }\DUrole{n,n}{ndarray}}\sphinxparamcomma \sphinxparam{\DUrole{n,n}{alpha}\DUrole{p,p}{:}\DUrole{w,w}{  }\DUrole{n,n}{float}\DUrole{w,w}{  }\DUrole{o,o}{=}\DUrole{w,w}{  }\DUrole{default_value}{0.5}}\sphinxparamcomma \sphinxparam{\DUrole{n,n}{seed}\DUrole{p,p}{:}\DUrole{w,w}{  }\DUrole{n,n}{int\DUrole{w,w}{  }\DUrole{p,p}{|}\DUrole{w,w}{  }None}\DUrole{w,w}{  }\DUrole{o,o}{=}\DUrole{w,w}{  }\DUrole{default_value}{0}}\sphinxparamcomma \sphinxparam{\DUrole{n,n}{palette}\DUrole{p,p}{:}\DUrole{w,w}{  }\DUrole{n,n}{str}\DUrole{w,w}{  }\DUrole{o,o}{=}\DUrole{w,w}{  }\DUrole{default_value}{\textquotesingle{}Spectral\textquotesingle{}}}}{{ $\rightarrow$ ndarray}}
\pysigstopsignatures
\sphinxAtStartPar
Visualize heatmaps on top of an image.
\begin{quote}\begin{description}
\sphinxlineitem{Parameters}\begin{itemize}
\item {} 
\sphinxAtStartPar
\sphinxstyleliteralstrong{\sphinxupquote{image}} (\sphinxstyleliteralemphasis{\sphinxupquote{np.ndarray}}) \textendash{} (H, W, C) Image to visualize heatmaps on top of. If float, in range {[}0, 1{]}, will be converted to uint8.

\item {} 
\sphinxAtStartPar
\sphinxstyleliteralstrong{\sphinxupquote{heatmaps}} (\sphinxstyleliteralemphasis{\sphinxupquote{np.ndarray}}) \textendash{} (H, W, num\_heatmaps) Heatmaps to visualize. If float, in range {[}0, 1{]}, will be converted to uint8.

\item {} 
\sphinxAtStartPar
\sphinxstyleliteralstrong{\sphinxupquote{alpha}} (\sphinxstyleliteralemphasis{\sphinxupquote{float}}\sphinxstyleliteralemphasis{\sphinxupquote{, }}\sphinxstyleliteralemphasis{\sphinxupquote{optional}}) \textendash{} Alpha value for the heatmaps. Defaults to 0.5.

\end{itemize}

\sphinxlineitem{Returns}
\sphinxAtStartPar
Image with heatmaps visualized, as a uint8 image.

\sphinxlineitem{Return type}
\sphinxAtStartPar
np.ndarray

\end{description}\end{quote}

\end{fulllineitems}

\index{draw\_circles() (in module deepdrr.utils.image\_utils)@\spxentry{draw\_circles()}\spxextra{in module deepdrr.utils.image\_utils}}

\begin{fulllineitems}
\phantomsection\label{\detokenize{deepdrr.utils:deepdrr.utils.image_utils.draw_circles}}
\pysigstartsignatures
\pysiglinewithargsret{\sphinxcode{\sphinxupquote{deepdrr.utils.image\_utils.}}\sphinxbfcode{\sphinxupquote{draw\_circles}}}{\sphinxparam{\DUrole{n,n}{image}\DUrole{p,p}{:}\DUrole{w,w}{  }\DUrole{n,n}{ndarray}}\sphinxparamcomma \sphinxparam{\DUrole{n,n}{circles}\DUrole{p,p}{:}\DUrole{w,w}{  }\DUrole{n,n}{ndarray}}\sphinxparamcomma \sphinxparam{\DUrole{n,n}{color}\DUrole{p,p}{:}\DUrole{w,w}{  }\DUrole{n,n}{List\DUrole{p,p}{{[}}int\DUrole{p,p}{{]}}}\DUrole{w,w}{  }\DUrole{o,o}{=}\DUrole{w,w}{  }\DUrole{default_value}{{[}255, 0, 0{]}}}\sphinxparamcomma \sphinxparam{\DUrole{n,n}{thickness}\DUrole{p,p}{:}\DUrole{w,w}{  }\DUrole{n,n}{int}\DUrole{w,w}{  }\DUrole{o,o}{=}\DUrole{w,w}{  }\DUrole{default_value}{2}}\sphinxparamcomma \sphinxparam{\DUrole{n,n}{radius}\DUrole{p,p}{:}\DUrole{w,w}{  }\DUrole{n,n}{int\DUrole{w,w}{  }\DUrole{p,p}{|}\DUrole{w,w}{  }None}\DUrole{w,w}{  }\DUrole{o,o}{=}\DUrole{w,w}{  }\DUrole{default_value}{None}}}{{ $\rightarrow$ ndarray}}
\pysigstopsignatures
\sphinxAtStartPar
Draw circles on an image.
\begin{quote}\begin{description}
\sphinxlineitem{Parameters}\begin{itemize}
\item {} 
\sphinxAtStartPar
\sphinxstyleliteralstrong{\sphinxupquote{image}} (\sphinxstyleliteralemphasis{\sphinxupquote{np.ndarray}}) \textendash{} the image to draw on.

\item {} 
\sphinxAtStartPar
\sphinxstyleliteralstrong{\sphinxupquote{circles}} (\sphinxstyleliteralemphasis{\sphinxupquote{np.ndarray}}) \textendash{} the circles to draw. {[}N, 3{]} array of {[}x, y, r{]} coordinates.

\end{itemize}

\end{description}\end{quote}

\end{fulllineitems}

\index{draw\_line() (in module deepdrr.utils.image\_utils)@\spxentry{draw\_line()}\spxextra{in module deepdrr.utils.image\_utils}}

\begin{fulllineitems}
\phantomsection\label{\detokenize{deepdrr.utils:deepdrr.utils.image_utils.draw_line}}
\pysigstartsignatures
\pysiglinewithargsret{\sphinxcode{\sphinxupquote{deepdrr.utils.image\_utils.}}\sphinxbfcode{\sphinxupquote{draw\_line}}}{\sphinxparam{\DUrole{n,n}{image}\DUrole{p,p}{:}\DUrole{w,w}{  }\DUrole{n,n}{ndarray}}\sphinxparamcomma \sphinxparam{\DUrole{n,n}{line}\DUrole{p,p}{:}\DUrole{w,w}{  }\DUrole{n,n}{{\hyperref[\detokenize{deepdrr.geo:deepdrr.geo.hyperplane.Line2D}]{\sphinxcrossref{Line2D}}}}}\sphinxparamcomma \sphinxparam{\DUrole{n,n}{color}\DUrole{p,p}{:}\DUrole{w,w}{  }\DUrole{n,n}{tuple}\DUrole{w,w}{  }\DUrole{o,o}{=}\DUrole{w,w}{  }\DUrole{default_value}{(255, 0, 0)}}\sphinxparamcomma \sphinxparam{\DUrole{n,n}{thickness}\DUrole{p,p}{:}\DUrole{w,w}{  }\DUrole{n,n}{int}\DUrole{w,w}{  }\DUrole{o,o}{=}\DUrole{w,w}{  }\DUrole{default_value}{2}}}{{ $\rightarrow$ ndarray}}
\pysigstopsignatures
\sphinxAtStartPar
Draw a line on an image.
\begin{quote}\begin{description}
\sphinxlineitem{Parameters}\begin{itemize}
\item {} 
\sphinxAtStartPar
\sphinxstyleliteralstrong{\sphinxupquote{image}} (\sphinxstyleliteralemphasis{\sphinxupquote{np.ndarray}}) \textendash{} the image to draw on.

\item {} 
\sphinxAtStartPar
\sphinxstyleliteralstrong{\sphinxupquote{line}} ({\hyperref[\detokenize{deepdrr.geo:deepdrr.geo.Line2D}]{\sphinxcrossref{\sphinxstyleliteralemphasis{\sphinxupquote{geo.Line2D}}}}}) \textendash{} the line to draw.

\item {} 
\sphinxAtStartPar
\sphinxstyleliteralstrong{\sphinxupquote{color}} (\sphinxstyleliteralemphasis{\sphinxupquote{tuple}}\sphinxstyleliteralemphasis{\sphinxupquote{, }}\sphinxstyleliteralemphasis{\sphinxupquote{optional}}) \textendash{} the color to draw the line in. Defaults to (255, 0, 0).

\item {} 
\sphinxAtStartPar
\sphinxstyleliteralstrong{\sphinxupquote{thickness}} (\sphinxstyleliteralemphasis{\sphinxupquote{int}}\sphinxstyleliteralemphasis{\sphinxupquote{, }}\sphinxstyleliteralemphasis{\sphinxupquote{optional}}) \textendash{} the thickness of the line. Defaults to 2.

\end{itemize}

\end{description}\end{quote}

\end{fulllineitems}

\index{draw\_masks() (in module deepdrr.utils.image\_utils)@\spxentry{draw\_masks()}\spxextra{in module deepdrr.utils.image\_utils}}

\begin{fulllineitems}
\phantomsection\label{\detokenize{deepdrr.utils:deepdrr.utils.image_utils.draw_masks}}
\pysigstartsignatures
\pysiglinewithargsret{\sphinxcode{\sphinxupquote{deepdrr.utils.image\_utils.}}\sphinxbfcode{\sphinxupquote{draw\_masks}}}{\sphinxparam{\DUrole{n,n}{image}\DUrole{p,p}{:}\DUrole{w,w}{  }\DUrole{n,n}{ndarray}}\sphinxparamcomma \sphinxparam{\DUrole{n,n}{masks}\DUrole{p,p}{:}\DUrole{w,w}{  }\DUrole{n,n}{ndarray}}\sphinxparamcomma \sphinxparam{\DUrole{n,n}{alpha}\DUrole{p,p}{:}\DUrole{w,w}{  }\DUrole{n,n}{float}\DUrole{w,w}{  }\DUrole{o,o}{=}\DUrole{w,w}{  }\DUrole{default_value}{0.3}}\sphinxparamcomma \sphinxparam{\DUrole{n,n}{palette}\DUrole{p,p}{:}\DUrole{w,w}{  }\DUrole{n,n}{str}\DUrole{w,w}{  }\DUrole{o,o}{=}\DUrole{w,w}{  }\DUrole{default_value}{\textquotesingle{}Spectral\textquotesingle{}}}\sphinxparamcomma \sphinxparam{\DUrole{n,n}{threshold}\DUrole{p,p}{:}\DUrole{w,w}{  }\DUrole{n,n}{float}\DUrole{w,w}{  }\DUrole{o,o}{=}\DUrole{w,w}{  }\DUrole{default_value}{0.5}}\sphinxparamcomma \sphinxparam{\DUrole{n,n}{seed}\DUrole{p,p}{:}\DUrole{w,w}{  }\DUrole{n,n}{int\DUrole{w,w}{  }\DUrole{p,p}{|}\DUrole{w,w}{  }None}\DUrole{w,w}{  }\DUrole{o,o}{=}\DUrole{w,w}{  }\DUrole{default_value}{0}}}{{ $\rightarrow$ ndarray}}
\pysigstopsignatures
\sphinxAtStartPar
Draw contours of masks on an image.
\begin{quote}\begin{description}
\sphinxlineitem{Parameters}\begin{itemize}
\item {} 
\sphinxAtStartPar
\sphinxstyleliteralstrong{\sphinxupquote{image}} (\sphinxstyleliteralemphasis{\sphinxupquote{np.ndarray}}) \textendash{} the image to draw on.

\item {} 
\sphinxAtStartPar
\sphinxstyleliteralstrong{\sphinxupquote{masks}} (\sphinxstyleliteralemphasis{\sphinxupquote{np.ndarray}}) \textendash{} the masks to draw. {[}H, W, num\_masks{]} array of masks.

\end{itemize}

\end{description}\end{quote}

\end{fulllineitems}

\index{draw\_segment() (in module deepdrr.utils.image\_utils)@\spxentry{draw\_segment()}\spxextra{in module deepdrr.utils.image\_utils}}

\begin{fulllineitems}
\phantomsection\label{\detokenize{deepdrr.utils:deepdrr.utils.image_utils.draw_segment}}
\pysigstartsignatures
\pysiglinewithargsret{\sphinxcode{\sphinxupquote{deepdrr.utils.image\_utils.}}\sphinxbfcode{\sphinxupquote{draw\_segment}}}{\sphinxparam{\DUrole{n,n}{image}\DUrole{p,p}{:}\DUrole{w,w}{  }\DUrole{n,n}{ndarray}}\sphinxparamcomma \sphinxparam{\DUrole{n,n}{segment}\DUrole{p,p}{:}\DUrole{w,w}{  }\DUrole{n,n}{{\hyperref[\detokenize{deepdrr.geo:deepdrr.geo.segment.Segment2D}]{\sphinxcrossref{Segment2D}}}}}\sphinxparamcomma \sphinxparam{\DUrole{n,n}{color}\DUrole{o,o}{=}\DUrole{default_value}{{[}255, 0, 0{]}}}\sphinxparamcomma \sphinxparam{\DUrole{n,n}{thickness}\DUrole{p,p}{:}\DUrole{w,w}{  }\DUrole{n,n}{int}\DUrole{w,w}{  }\DUrole{o,o}{=}\DUrole{w,w}{  }\DUrole{default_value}{2}}\sphinxparamcomma \sphinxparam{\DUrole{n,n}{radius}\DUrole{p,p}{:}\DUrole{w,w}{  }\DUrole{n,n}{int}\DUrole{w,w}{  }\DUrole{o,o}{=}\DUrole{w,w}{  }\DUrole{default_value}{5}}}{{ $\rightarrow$ ndarray}}
\pysigstopsignatures
\sphinxAtStartPar
Draw a segment on an image.

\end{fulllineitems}

\index{ensure\_cdim() (in module deepdrr.utils.image\_utils)@\spxentry{ensure\_cdim()}\spxextra{in module deepdrr.utils.image\_utils}}

\begin{fulllineitems}
\phantomsection\label{\detokenize{deepdrr.utils:deepdrr.utils.image_utils.ensure_cdim}}
\pysigstartsignatures
\pysiglinewithargsret{\sphinxcode{\sphinxupquote{deepdrr.utils.image\_utils.}}\sphinxbfcode{\sphinxupquote{ensure\_cdim}}}{\sphinxparam{\DUrole{n,n}{x}\DUrole{p,p}{:}\DUrole{w,w}{  }\DUrole{n,n}{ndarray}}\sphinxparamcomma \sphinxparam{\DUrole{n,n}{c}\DUrole{p,p}{:}\DUrole{w,w}{  }\DUrole{n,n}{int}\DUrole{w,w}{  }\DUrole{o,o}{=}\DUrole{w,w}{  }\DUrole{default_value}{3}}}{{ $\rightarrow$ ndarray}}
\pysigstopsignatures
\end{fulllineitems}

\index{image\_saver() (in module deepdrr.utils.image\_utils)@\spxentry{image\_saver()}\spxextra{in module deepdrr.utils.image\_utils}}

\begin{fulllineitems}
\phantomsection\label{\detokenize{deepdrr.utils:deepdrr.utils.image_utils.image_saver}}
\pysigstartsignatures
\pysiglinewithargsret{\sphinxcode{\sphinxupquote{deepdrr.utils.image\_utils.}}\sphinxbfcode{\sphinxupquote{image\_saver}}}{\sphinxparam{\DUrole{n,n}{images}\DUrole{p,p}{:}\DUrole{w,w}{  }\DUrole{n,n}{ndarray}}\sphinxparamcomma \sphinxparam{\DUrole{n,n}{prefix}\DUrole{p,p}{:}\DUrole{w,w}{  }\DUrole{n,n}{str}}\sphinxparamcomma \sphinxparam{\DUrole{n,n}{path}\DUrole{p,p}{:}\DUrole{w,w}{  }\DUrole{n,n}{str}}}{{ $\rightarrow$ bool}}
\pysigstopsignatures
\sphinxAtStartPar
Save the images as tiff
\begin{quote}\begin{description}
\sphinxlineitem{Parameters}\begin{itemize}
\item {} 
\sphinxAtStartPar
\sphinxstyleliteralstrong{\sphinxupquote{images}} (\sphinxstyleliteralemphasis{\sphinxupquote{np.ndarray}}) \textendash{} array of images

\item {} 
\sphinxAtStartPar
\sphinxstyleliteralstrong{\sphinxupquote{prefix}} (\sphinxstyleliteralemphasis{\sphinxupquote{str}}) \textendash{} prefix for each file name

\item {} 
\sphinxAtStartPar
\sphinxstyleliteralstrong{\sphinxupquote{path}} (\sphinxstyleliteralemphasis{\sphinxupquote{str}}) \textendash{} path to directory to save the files in

\end{itemize}

\sphinxlineitem{Returns}
\sphinxAtStartPar
return code.

\sphinxlineitem{Return type}
\sphinxAtStartPar
bool

\end{description}\end{quote}

\end{fulllineitems}

\index{save() (in module deepdrr.utils.image\_utils)@\spxentry{save()}\spxextra{in module deepdrr.utils.image\_utils}}

\begin{fulllineitems}
\phantomsection\label{\detokenize{deepdrr.utils:deepdrr.utils.image_utils.save}}
\pysigstartsignatures
\pysiglinewithargsret{\sphinxcode{\sphinxupquote{deepdrr.utils.image\_utils.}}\sphinxbfcode{\sphinxupquote{save}}}{\sphinxparam{\DUrole{n,n}{path}\DUrole{p,p}{:}\DUrole{w,w}{  }\DUrole{n,n}{Path}}\sphinxparamcomma \sphinxparam{\DUrole{n,n}{image}\DUrole{p,p}{:}\DUrole{w,w}{  }\DUrole{n,n}{ndarray}}\sphinxparamcomma \sphinxparam{\DUrole{n,n}{mkdir}\DUrole{p,p}{:}\DUrole{w,w}{  }\DUrole{n,n}{bool}\DUrole{w,w}{  }\DUrole{o,o}{=}\DUrole{w,w}{  }\DUrole{default_value}{True}}}{{ $\rightarrow$ Path}}
\pysigstopsignatures
\sphinxAtStartPar
Save the given image using PIL.
\begin{quote}\begin{description}
\sphinxlineitem{Parameters}\begin{itemize}
\item {} 
\sphinxAtStartPar
\sphinxstyleliteralstrong{\sphinxupquote{path}} (\sphinxstyleliteralemphasis{\sphinxupquote{Path}}) \textendash{} the path to write the image to. Also determines the type.

\item {} 
\sphinxAtStartPar
\sphinxstyleliteralstrong{\sphinxupquote{image}} (\sphinxstyleliteralemphasis{\sphinxupquote{np.ndarray}}) \textendash{} the image, in {[}C, H, W{]} or {[}H, W, C{]} order. (If the former, transposes).
If in float32, assumed to be a float image. Converted to uint8 before saving.

\end{itemize}

\end{description}\end{quote}

\end{fulllineitems}



\subsection{deepdrr.utils.mesh\_utils module}
\label{\detokenize{deepdrr.utils:module-deepdrr.utils.mesh_utils}}\label{\detokenize{deepdrr.utils:deepdrr-utils-mesh-utils-module}}\index{module@\spxentry{module}!deepdrr.utils.mesh\_utils@\spxentry{deepdrr.utils.mesh\_utils}}\index{deepdrr.utils.mesh\_utils@\spxentry{deepdrr.utils.mesh\_utils}!module@\spxentry{module}}\index{isosurface() (in module deepdrr.utils.mesh\_utils)@\spxentry{isosurface()}\spxextra{in module deepdrr.utils.mesh\_utils}}

\begin{fulllineitems}
\phantomsection\label{\detokenize{deepdrr.utils:deepdrr.utils.mesh_utils.isosurface}}
\pysigstartsignatures
\pysiglinewithargsret{\sphinxcode{\sphinxupquote{deepdrr.utils.mesh\_utils.}}\sphinxbfcode{\sphinxupquote{isosurface}}}{\sphinxparam{\DUrole{n,n}{data}\DUrole{p,p}{:}\DUrole{w,w}{  }\DUrole{n,n}{ndarray}}\sphinxparamcomma \sphinxparam{\DUrole{n,n}{value}\DUrole{p,p}{:}\DUrole{w,w}{  }\DUrole{n,n}{float}\DUrole{w,w}{  }\DUrole{o,o}{=}\DUrole{w,w}{  }\DUrole{default_value}{0.5}}\sphinxparamcomma \sphinxparam{\DUrole{n,n}{label}\DUrole{p,p}{:}\DUrole{w,w}{  }\DUrole{n,n}{int\DUrole{w,w}{  }\DUrole{p,p}{|}\DUrole{w,w}{  }None}\DUrole{w,w}{  }\DUrole{o,o}{=}\DUrole{w,w}{  }\DUrole{default_value}{None}}\sphinxparamcomma \sphinxparam{\DUrole{n,n}{node\_centered}\DUrole{p,p}{:}\DUrole{w,w}{  }\DUrole{n,n}{bool}\DUrole{w,w}{  }\DUrole{o,o}{=}\DUrole{w,w}{  }\DUrole{default_value}{True}}\sphinxparamcomma \sphinxparam{\DUrole{n,n}{smooth}\DUrole{p,p}{:}\DUrole{w,w}{  }\DUrole{n,n}{bool}\DUrole{w,w}{  }\DUrole{o,o}{=}\DUrole{w,w}{  }\DUrole{default_value}{True}}\sphinxparamcomma \sphinxparam{\DUrole{n,n}{decimation}\DUrole{p,p}{:}\DUrole{w,w}{  }\DUrole{n,n}{float}\DUrole{w,w}{  }\DUrole{o,o}{=}\DUrole{w,w}{  }\DUrole{default_value}{0.01}}\sphinxparamcomma \sphinxparam{\DUrole{n,n}{smooth\_iter}\DUrole{p,p}{:}\DUrole{w,w}{  }\DUrole{n,n}{int}\DUrole{w,w}{  }\DUrole{o,o}{=}\DUrole{w,w}{  }\DUrole{default_value}{30}}\sphinxparamcomma \sphinxparam{\DUrole{n,n}{relaxation\_factor}\DUrole{p,p}{:}\DUrole{w,w}{  }\DUrole{n,n}{float}\DUrole{w,w}{  }\DUrole{o,o}{=}\DUrole{w,w}{  }\DUrole{default_value}{0.25}}}{{ $\rightarrow$ PolyData}}
\pysigstopsignatures
\sphinxAtStartPar
Create an isosurface model using marching cubes.
\begin{quote}\begin{description}
\sphinxlineitem{Parameters}\begin{itemize}
\item {} 
\sphinxAtStartPar
\sphinxstyleliteralstrong{\sphinxupquote{data}} (\sphinxstyleliteralemphasis{\sphinxupquote{np.ndarray}}) \textendash{} A 3\sphinxhyphen{}D array with scalar or integer data.

\item {} 
\sphinxAtStartPar
\sphinxstyleliteralstrong{\sphinxupquote{value}} (\sphinxstyleliteralemphasis{\sphinxupquote{float}}\sphinxstyleliteralemphasis{\sphinxupquote{, }}\sphinxstyleliteralemphasis{\sphinxupquote{optional}}) \textendash{} The value of the surface in \sphinxtitleref{data}. Defaults to 0.5.

\item {} 
\sphinxAtStartPar
\sphinxstyleliteralstrong{\sphinxupquote{label}} (\sphinxstyleliteralemphasis{\sphinxupquote{Optional}}\sphinxstyleliteralemphasis{\sphinxupquote{{[}}}\sphinxstyleliteralemphasis{\sphinxupquote{int}}\sphinxstyleliteralemphasis{\sphinxupquote{{]}}}\sphinxstyleliteralemphasis{\sphinxupquote{, }}\sphinxstyleliteralemphasis{\sphinxupquote{optional}}) \textendash{} Get the isosurface of the \sphinxtitleref{data == label} segmentation. Defaults to None.

\item {} 
\sphinxAtStartPar
\sphinxstyleliteralstrong{\sphinxupquote{node\_centered}} (\sphinxstyleliteralemphasis{\sphinxupquote{bool}}\sphinxstyleliteralemphasis{\sphinxupquote{, }}\sphinxstyleliteralemphasis{\sphinxupquote{optional}}) \textendash{} Whether the values in the data are sampled in the node\sphinxhyphen{}centered style. Defaults to true.

\item {} 
\sphinxAtStartPar
\sphinxstyleliteralstrong{\sphinxupquote{smooth}} (\sphinxstyleliteralemphasis{\sphinxupquote{bool}}\sphinxstyleliteralemphasis{\sphinxupquote{, }}\sphinxstyleliteralemphasis{\sphinxupquote{optional}}) \textendash{} whether to apply smoothing. Defaults to True.

\item {} 
\sphinxAtStartPar
\sphinxstyleliteralstrong{\sphinxupquote{decimation}} (\sphinxstyleliteralemphasis{\sphinxupquote{float}}\sphinxstyleliteralemphasis{\sphinxupquote{, }}\sphinxstyleliteralemphasis{\sphinxupquote{optional}}) \textendash{} How much to decimate the surface. Defaults to 0.01.

\item {} 
\sphinxAtStartPar
\sphinxstyleliteralstrong{\sphinxupquote{smooth\_iter}} (\sphinxstyleliteralemphasis{\sphinxupquote{int}}\sphinxstyleliteralemphasis{\sphinxupquote{, }}\sphinxstyleliteralemphasis{\sphinxupquote{optional}}) \textendash{} number of smoothing iterations to run.

\item {} 
\sphinxAtStartPar
\sphinxstyleliteralstrong{\sphinxupquote{relaxation\_factor}} (\sphinxstyleliteralemphasis{\sphinxupquote{float}}) \textendash{} passed to surface.smooth.

\end{itemize}

\sphinxlineitem{Returns}
\sphinxAtStartPar
a Pyvista mesh.

\sphinxlineitem{Return type}
\sphinxAtStartPar
pv.PolyData

\end{description}\end{quote}

\end{fulllineitems}

\index{voxelize() (in module deepdrr.utils.mesh\_utils)@\spxentry{voxelize()}\spxextra{in module deepdrr.utils.mesh\_utils}}

\begin{fulllineitems}
\phantomsection\label{\detokenize{deepdrr.utils:deepdrr.utils.mesh_utils.voxelize}}
\pysigstartsignatures
\pysiglinewithargsret{\sphinxcode{\sphinxupquote{deepdrr.utils.mesh\_utils.}}\sphinxbfcode{\sphinxupquote{voxelize}}}{\sphinxparam{\DUrole{n,n}{surface}\DUrole{p,p}{:}\DUrole{w,w}{  }\DUrole{n,n}{PolyData}}\sphinxparamcomma \sphinxparam{\DUrole{n,n}{density}\DUrole{p,p}{:}\DUrole{w,w}{  }\DUrole{n,n}{float}\DUrole{w,w}{  }\DUrole{o,o}{=}\DUrole{w,w}{  }\DUrole{default_value}{0.2}}\sphinxparamcomma \sphinxparam{\DUrole{n,n}{bounds}\DUrole{p,p}{:}\DUrole{w,w}{  }\DUrole{n,n}{List\DUrole{p,p}{{[}}float\DUrole{p,p}{{]}}\DUrole{w,w}{  }\DUrole{p,p}{|}\DUrole{w,w}{  }None}\DUrole{w,w}{  }\DUrole{o,o}{=}\DUrole{w,w}{  }\DUrole{default_value}{None}}}{{ $\rightarrow$ Tuple\DUrole{p,p}{{[}}ndarray\DUrole{p,p}{,}\DUrole{w,w}{  }{\hyperref[\detokenize{deepdrr.geo:deepdrr.geo.core.FrameTransform}]{\sphinxcrossref{FrameTransform}}}\DUrole{p,p}{{]}}}}
\pysigstopsignatures
\sphinxAtStartPar
Voxelize the surface mesh with the given density.
\begin{quote}\begin{description}
\sphinxlineitem{Parameters}\begin{itemize}
\item {} 
\sphinxAtStartPar
\sphinxstyleliteralstrong{\sphinxupquote{surface}} (\sphinxstyleliteralemphasis{\sphinxupquote{pv.PolyData}}) \textendash{} The surface.

\item {} 
\sphinxAtStartPar
\sphinxstyleliteralstrong{\sphinxupquote{density}} (\sphinxstyleliteralemphasis{\sphinxupquote{Union}}\sphinxstyleliteralemphasis{\sphinxupquote{{[}}}\sphinxstyleliteralemphasis{\sphinxupquote{float}}\sphinxstyleliteralemphasis{\sphinxupquote{, }}\sphinxstyleliteralemphasis{\sphinxupquote{Tuple}}\sphinxstyleliteralemphasis{\sphinxupquote{{[}}}\sphinxstyleliteralemphasis{\sphinxupquote{float}}\sphinxstyleliteralemphasis{\sphinxupquote{, }}\sphinxstyleliteralemphasis{\sphinxupquote{float}}\sphinxstyleliteralemphasis{\sphinxupquote{, }}\sphinxstyleliteralemphasis{\sphinxupquote{float}}\sphinxstyleliteralemphasis{\sphinxupquote{{]}}}\sphinxstyleliteralemphasis{\sphinxupquote{{]}}}) \textendash{} Either a single float or a
list of floats giving the size of a voxel in x, y, z.
(This is really a spacing, but it’s misnamed in pyvista.)

\end{itemize}

\sphinxlineitem{Returns}
\sphinxAtStartPar
The voxelized segmentation of the surface as np.uint8 and the associated world\_from\_ijk transform.

\sphinxlineitem{Return type}
\sphinxAtStartPar
Tuple{[}np.ndarray, {\hyperref[\detokenize{deepdrr.geo:deepdrr.geo.FrameTransform}]{\sphinxcrossref{geo.FrameTransform}}}{]}

\end{description}\end{quote}

\end{fulllineitems}

\index{voxelize\_dir() (in module deepdrr.utils.mesh\_utils)@\spxentry{voxelize\_dir()}\spxextra{in module deepdrr.utils.mesh\_utils}}

\begin{fulllineitems}
\phantomsection\label{\detokenize{deepdrr.utils:deepdrr.utils.mesh_utils.voxelize_dir}}
\pysigstartsignatures
\pysiglinewithargsret{\sphinxcode{\sphinxupquote{deepdrr.utils.mesh\_utils.}}\sphinxbfcode{\sphinxupquote{voxelize\_dir}}}{\sphinxparam{\DUrole{n,n}{input\_dir}\DUrole{p,p}{:}\DUrole{w,w}{  }\DUrole{n,n}{str}}\sphinxparamcomma \sphinxparam{\DUrole{n,n}{output\_dir}\DUrole{p,p}{:}\DUrole{w,w}{  }\DUrole{n,n}{str}}\sphinxparamcomma \sphinxparam{\DUrole{n,n}{use\_cached}\DUrole{p,p}{:}\DUrole{w,w}{  }\DUrole{n,n}{bool}\DUrole{w,w}{  }\DUrole{o,o}{=}\DUrole{w,w}{  }\DUrole{default_value}{True}}\sphinxparamcomma \sphinxparam{\DUrole{o,o}{**}\DUrole{n,n}{kwargs}}}{}
\pysigstopsignatures
\end{fulllineitems}

\index{voxelize\_file() (in module deepdrr.utils.mesh\_utils)@\spxentry{voxelize\_file()}\spxextra{in module deepdrr.utils.mesh\_utils}}

\begin{fulllineitems}
\phantomsection\label{\detokenize{deepdrr.utils:deepdrr.utils.mesh_utils.voxelize_file}}
\pysigstartsignatures
\pysiglinewithargsret{\sphinxcode{\sphinxupquote{deepdrr.utils.mesh\_utils.}}\sphinxbfcode{\sphinxupquote{voxelize\_file}}}{\sphinxparam{\DUrole{n,n}{path}\DUrole{p,p}{:}\DUrole{w,w}{  }\DUrole{n,n}{str}}\sphinxparamcomma \sphinxparam{\DUrole{n,n}{output\_path}\DUrole{p,p}{:}\DUrole{w,w}{  }\DUrole{n,n}{str}}\sphinxparamcomma \sphinxparam{\DUrole{o,o}{**}\DUrole{n,n}{kwargs}}}{}
\pysigstopsignatures
\end{fulllineitems}



\subsection{deepdrr.utils.test\_utils module}
\label{\detokenize{deepdrr.utils:module-deepdrr.utils.test_utils}}\label{\detokenize{deepdrr.utils:deepdrr-utils-test-utils-module}}\index{module@\spxentry{module}!deepdrr.utils.test\_utils@\spxentry{deepdrr.utils.test\_utils}}\index{deepdrr.utils.test\_utils@\spxentry{deepdrr.utils.test\_utils}!module@\spxentry{module}}
\sphinxAtStartPar
Util functions for automated tests.
\index{download\_sampledata() (in module deepdrr.utils.test\_utils)@\spxentry{download\_sampledata()}\spxextra{in module deepdrr.utils.test\_utils}}

\begin{fulllineitems}
\phantomsection\label{\detokenize{deepdrr.utils:deepdrr.utils.test_utils.download_sampledata}}
\pysigstartsignatures
\pysiglinewithargsret{\sphinxcode{\sphinxupquote{deepdrr.utils.test\_utils.}}\sphinxbfcode{\sphinxupquote{download\_sampledata}}}{\sphinxparam{\DUrole{n,n}{name}\DUrole{p,p}{:}\DUrole{w,w}{  }\DUrole{n,n}{str}\DUrole{w,w}{  }\DUrole{o,o}{=}\DUrole{w,w}{  }\DUrole{default_value}{\textquotesingle{}CT\sphinxhyphen{}chest\textquotesingle{}}}\sphinxparamcomma \sphinxparam{\DUrole{o,o}{**}\DUrole{n,n}{kwargs}}}{{ $\rightarrow$ Path}}
\pysigstopsignatures
\sphinxAtStartPar
Download the given sample data for testing.

\sphinxAtStartPar
Options include:
* \sphinxtitleref{“CT\sphinxhyphen{}chest”}: a NRRD file containing a torso.
* \sphinxtitleref{“CTPelvic1K”}: a CT scan of the pelvis with a right superior pubic ramus fracture,
\begin{quote}

\sphinxAtStartPar
selected from the \sphinxhref{https://github.com/ICT-MIRACLE-lab/CTPelvic1K}{CTPelvik1K dataset}.
This downloads the CT, pelvis segmentation, and KWire trajectory annotations and unzips them,
returning the directory where they are located.
\end{quote}
\begin{quote}\begin{description}
\sphinxlineitem{Parameters}
\sphinxAtStartPar
\sphinxstyleliteralstrong{\sphinxupquote{name}} (\sphinxstyleliteralemphasis{\sphinxupquote{str}}\sphinxstyleliteralemphasis{\sphinxupquote{, }}\sphinxstyleliteralemphasis{\sphinxupquote{optional}}) \textendash{} The name of the volume, used as a key to the public domain downloadable data. Defaults to ‘CT\sphinxhyphen{}chest’.

\sphinxlineitem{Returns}
\sphinxAtStartPar
The path to the downloaded file or directory (if unzipped).

\sphinxlineitem{Return type}
\sphinxAtStartPar
Path

\end{description}\end{quote}

\end{fulllineitems}

\index{get\_output\_dir() (in module deepdrr.utils.test\_utils)@\spxentry{get\_output\_dir()}\spxextra{in module deepdrr.utils.test\_utils}}

\begin{fulllineitems}
\phantomsection\label{\detokenize{deepdrr.utils:deepdrr.utils.test_utils.get_output_dir}}
\pysigstartsignatures
\pysiglinewithargsret{\sphinxcode{\sphinxupquote{deepdrr.utils.test\_utils.}}\sphinxbfcode{\sphinxupquote{get\_output\_dir}}}{}{{ $\rightarrow$ Path}}
\pysigstopsignatures
\end{fulllineitems}



\subsection{Module contents}
\label{\detokenize{deepdrr.utils:module-deepdrr.utils}}\label{\detokenize{deepdrr.utils:module-contents}}\index{module@\spxentry{module}!deepdrr.utils@\spxentry{deepdrr.utils}}\index{deepdrr.utils@\spxentry{deepdrr.utils}!module@\spxentry{module}}\index{generate\_uniform\_angles() (in module deepdrr.utils)@\spxentry{generate\_uniform\_angles()}\spxextra{in module deepdrr.utils}}

\begin{fulllineitems}
\phantomsection\label{\detokenize{deepdrr.utils:deepdrr.utils.generate_uniform_angles}}
\pysigstartsignatures
\pysiglinewithargsret{\sphinxcode{\sphinxupquote{deepdrr.utils.}}\sphinxbfcode{\sphinxupquote{generate\_uniform\_angles}}}{\sphinxparam{\DUrole{n,n}{phi\_range}\DUrole{p,p}{:}\DUrole{w,w}{  }\DUrole{n,n}{Tuple\DUrole{p,p}{{[}}float\DUrole{p,p}{,}\DUrole{w,w}{  }float\DUrole{p,p}{,}\DUrole{w,w}{  }float\DUrole{p,p}{{]}}}}\sphinxparamcomma \sphinxparam{\DUrole{n,n}{theta\_range}\DUrole{p,p}{:}\DUrole{w,w}{  }\DUrole{n,n}{Tuple\DUrole{p,p}{{[}}float\DUrole{p,p}{,}\DUrole{w,w}{  }float\DUrole{p,p}{,}\DUrole{w,w}{  }float\DUrole{p,p}{{]}}}}}{{ $\rightarrow$ Tuple\DUrole{p,p}{{[}}ndarray\DUrole{p,p}{,}\DUrole{w,w}{  }ndarray\DUrole{p,p}{{]}}}}
\pysigstopsignatures
\sphinxAtStartPar
Generate a uniform sampling of angles over the given ranges.

\sphinxAtStartPar
If inputs are in degrees, so will the outputs be.
\begin{quote}\begin{description}
\sphinxlineitem{Parameters}\begin{itemize}
\item {} 
\sphinxAtStartPar
\sphinxstyleliteralstrong{\sphinxupquote{phi\_range}} (\sphinxstyleliteralemphasis{\sphinxupquote{Tuple}}\sphinxstyleliteralemphasis{\sphinxupquote{{[}}}\sphinxstyleliteralemphasis{\sphinxupquote{float}}\sphinxstyleliteralemphasis{\sphinxupquote{, }}\sphinxstyleliteralemphasis{\sphinxupquote{float}}\sphinxstyleliteralemphasis{\sphinxupquote{, }}\sphinxstyleliteralemphasis{\sphinxupquote{float}}\sphinxstyleliteralemphasis{\sphinxupquote{{]}}}) \textendash{} range of angles phi in (min, max, step) form, in degrees.

\item {} 
\sphinxAtStartPar
\sphinxstyleliteralstrong{\sphinxupquote{theta\_range}} (\sphinxstyleliteralemphasis{\sphinxupquote{Tuple}}\sphinxstyleliteralemphasis{\sphinxupquote{{[}}}\sphinxstyleliteralemphasis{\sphinxupquote{float}}\sphinxstyleliteralemphasis{\sphinxupquote{, }}\sphinxstyleliteralemphasis{\sphinxupquote{float}}\sphinxstyleliteralemphasis{\sphinxupquote{, }}\sphinxstyleliteralemphasis{\sphinxupquote{float}}\sphinxstyleliteralemphasis{\sphinxupquote{{]}}}) \textendash{} range of angles theta in (min, max, step) form, in degrees.

\end{itemize}

\sphinxlineitem{Returns}
\sphinxAtStartPar
phis, thetas over uniform angles, in radians.

\sphinxlineitem{Return type}
\sphinxAtStartPar
Tuple{[}np.ndarray, np.ndarray{]}

\end{description}\end{quote}

\end{fulllineitems}

\index{jsonable() (in module deepdrr.utils)@\spxentry{jsonable()}\spxextra{in module deepdrr.utils}}

\begin{fulllineitems}
\phantomsection\label{\detokenize{deepdrr.utils:deepdrr.utils.jsonable}}
\pysigstartsignatures
\pysiglinewithargsret{\sphinxcode{\sphinxupquote{deepdrr.utils.}}\sphinxbfcode{\sphinxupquote{jsonable}}}{\sphinxparam{\DUrole{n,n}{obj}\DUrole{p,p}{:}\DUrole{w,w}{  }\DUrole{n,n}{Any}}}{}
\pysigstopsignatures
\sphinxAtStartPar
Convert obj to a JSON\sphinxhyphen{}ready container or object.
:param obj:
:type obj: {[}type{]}

\end{fulllineitems}

\index{listify() (in module deepdrr.utils)@\spxentry{listify()}\spxextra{in module deepdrr.utils}}

\begin{fulllineitems}
\phantomsection\label{\detokenize{deepdrr.utils:deepdrr.utils.listify}}
\pysigstartsignatures
\pysiglinewithargsret{\sphinxcode{\sphinxupquote{deepdrr.utils.}}\sphinxbfcode{\sphinxupquote{listify}}}{\sphinxparam{\DUrole{n,n}{x}\DUrole{p,p}{:}\DUrole{w,w}{  }\DUrole{n,n}{List\DUrole{p,p}{{[}}T\DUrole{p,p}{{]}}\DUrole{w,w}{  }\DUrole{p,p}{|}\DUrole{w,w}{  }T}}\sphinxparamcomma \sphinxparam{\DUrole{n,n}{n}\DUrole{p,p}{:}\DUrole{w,w}{  }\DUrole{n,n}{int}\DUrole{w,w}{  }\DUrole{o,o}{=}\DUrole{w,w}{  }\DUrole{default_value}{1}}}{{ $\rightarrow$ List\DUrole{p,p}{{[}}T\DUrole{p,p}{{]}}}}
\pysigstopsignatures
\end{fulllineitems}

\index{mappable() (in module deepdrr.utils)@\spxentry{mappable()}\spxextra{in module deepdrr.utils}}

\begin{fulllineitems}
\phantomsection\label{\detokenize{deepdrr.utils:deepdrr.utils.mappable}}
\pysigstartsignatures
\pysiglinewithargsret{\sphinxcode{\sphinxupquote{deepdrr.utils.}}\sphinxbfcode{\sphinxupquote{mappable}}}{\sphinxparam{\DUrole{n,n}{ndim}\DUrole{p,p}{:}\DUrole{w,w}{  }\DUrole{n,n}{int\DUrole{w,w}{  }\DUrole{p,p}{|}\DUrole{w,w}{  }List\DUrole{p,p}{{[}}int\DUrole{p,p}{{]}}}\DUrole{w,w}{  }\DUrole{o,o}{=}\DUrole{w,w}{  }\DUrole{default_value}{1}}\sphinxparamcomma \sphinxparam{\DUrole{n,n}{every}\DUrole{p,p}{:}\DUrole{w,w}{  }\DUrole{n,n}{bool}\DUrole{w,w}{  }\DUrole{o,o}{=}\DUrole{w,w}{  }\DUrole{default_value}{False}}\sphinxparamcomma \sphinxparam{\DUrole{n,n}{method}\DUrole{p,p}{:}\DUrole{w,w}{  }\DUrole{n,n}{bool}\DUrole{w,w}{  }\DUrole{o,o}{=}\DUrole{w,w}{  }\DUrole{default_value}{False}}}{}
\pysigstopsignatures
\sphinxAtStartPar
Decorator for funcs that take a n\sphinxhyphen{}D array x.

\sphinxAtStartPar
Maps func across the last axis for arrays with ndim \textgreater{} 1. Assumes that the array \sphinxtitleref{x} is the last
positional argument to the function, allowing for methods of a class or functions that take
other parameters first, unless \sphinxtitleref{every} is \sphinxtitleref{True}. The function may also take keyword arguments,
of course, and these are passed unchanged.
\begin{quote}\begin{description}
\sphinxlineitem{Parameters}\begin{itemize}
\item {} 
\sphinxAtStartPar
\sphinxstyleliteralstrong{\sphinxupquote{ndim}} \textendash{} ndim(s) of argument(s) that the function expects. Either an int (same for each) or a
list of ints, one for each argument. 0 indicates a scalar.

\item {} 
\sphinxAtStartPar
\sphinxstyleliteralstrong{\sphinxupquote{every}} (\sphinxstyleliteralemphasis{\sphinxupquote{bool}}) \textendash{} whether every argument is mappable (of base dimension n) or just the last argument.

\item {} 
\sphinxAtStartPar
\sphinxstyleliteralstrong{\sphinxupquote{method}} (\sphinxstyleliteralemphasis{\sphinxupquote{bool}}) \textendash{} whether the function is a method of a class instance.

\end{itemize}

\end{description}\end{quote}

\end{fulllineitems}

\index{neglog() (in module deepdrr.utils)@\spxentry{neglog()}\spxextra{in module deepdrr.utils}}

\begin{fulllineitems}
\phantomsection\label{\detokenize{deepdrr.utils:deepdrr.utils.neglog}}
\pysigstartsignatures
\pysiglinewithargsret{\sphinxcode{\sphinxupquote{deepdrr.utils.}}\sphinxbfcode{\sphinxupquote{neglog}}}{\sphinxparam{\DUrole{n,n}{image}\DUrole{p,p}{:}\DUrole{w,w}{  }\DUrole{n,n}{ndarray}}\sphinxparamcomma \sphinxparam{\DUrole{n,n}{epsilon}\DUrole{p,p}{:}\DUrole{w,w}{  }\DUrole{n,n}{float}\DUrole{w,w}{  }\DUrole{o,o}{=}\DUrole{w,w}{  }\DUrole{default_value}{0.01}}}{{ $\rightarrow$ ndarray}}
\pysigstopsignatures
\sphinxAtStartPar
Take the negative log transform of an intensity image.
\begin{quote}\begin{description}
\sphinxlineitem{Parameters}\begin{itemize}
\item {} 
\sphinxAtStartPar
\sphinxstyleliteralstrong{\sphinxupquote{image}} (\sphinxstyleliteralemphasis{\sphinxupquote{np.ndarray}}) \textendash{} a single 2D image, or N such images.

\item {} 
\sphinxAtStartPar
\sphinxstyleliteralstrong{\sphinxupquote{epsilon}} (\sphinxstyleliteralemphasis{\sphinxupquote{float}}\sphinxstyleliteralemphasis{\sphinxupquote{, }}\sphinxstyleliteralemphasis{\sphinxupquote{optional}}) \textendash{} positive offset from 0 before taking the logarithm.

\end{itemize}

\sphinxlineitem{Returns}
\sphinxAtStartPar
the image or images after a negative log transform, scaled to {[}0, 1{]}

\sphinxlineitem{Return type}
\sphinxAtStartPar
np.ndarray

\end{description}\end{quote}

\end{fulllineitems}

\index{one\_hot() (in module deepdrr.utils)@\spxentry{one\_hot()}\spxextra{in module deepdrr.utils}}

\begin{fulllineitems}
\phantomsection\label{\detokenize{deepdrr.utils:deepdrr.utils.one_hot}}
\pysigstartsignatures
\pysiglinewithargsret{\sphinxcode{\sphinxupquote{deepdrr.utils.}}\sphinxbfcode{\sphinxupquote{one\_hot}}}{\sphinxparam{\DUrole{n,n}{x}\DUrole{p,p}{:}\DUrole{w,w}{  }\DUrole{n,n}{ndarray}}\sphinxparamcomma \sphinxparam{\DUrole{n,n}{num\_classes}\DUrole{p,p}{:}\DUrole{w,w}{  }\DUrole{n,n}{int\DUrole{w,w}{  }\DUrole{p,p}{|}\DUrole{w,w}{  }None}\DUrole{w,w}{  }\DUrole{o,o}{=}\DUrole{w,w}{  }\DUrole{default_value}{None}}\sphinxparamcomma \sphinxparam{\DUrole{n,n}{axis}\DUrole{p,p}{:}\DUrole{w,w}{  }\DUrole{n,n}{int}\DUrole{w,w}{  }\DUrole{o,o}{=}\DUrole{w,w}{  }\DUrole{default_value}{\sphinxhyphen{}1}}}{{ $\rightarrow$ ndarray}}
\pysigstopsignatures
\sphinxAtStartPar
One\sphinxhyphen{}hot encode the vector x along the axis.
\begin{quote}\begin{description}
\sphinxlineitem{Parameters}\begin{itemize}
\item {} 
\sphinxAtStartPar
\sphinxstyleliteralstrong{\sphinxupquote{x}} (\sphinxstyleliteralemphasis{\sphinxupquote{np.ndarray}}) \textendash{} n\sphinxhyphen{}dim array x.

\item {} 
\sphinxAtStartPar
\sphinxstyleliteralstrong{\sphinxupquote{num\_classes}} (\sphinxstyleliteralemphasis{\sphinxupquote{Optional}}\sphinxstyleliteralemphasis{\sphinxupquote{{[}}}\sphinxstyleliteralemphasis{\sphinxupquote{int}}\sphinxstyleliteralemphasis{\sphinxupquote{{]}}}) \textendash{} number of classes. Uses maximum label if not provided.

\item {} 
\sphinxAtStartPar
\sphinxstyleliteralstrong{\sphinxupquote{axis}} (\sphinxstyleliteralemphasis{\sphinxupquote{int}}) \textendash{} the axis to insert the labels along.

\end{itemize}

\sphinxlineitem{Returns}
\sphinxAtStartPar
one\sphinxhyphen{}hot encoded labels with n + 1 axes.

\sphinxlineitem{Return type}
\sphinxAtStartPar
np.ndarray

\end{description}\end{quote}

\end{fulllineitems}

\index{param\_saver() (in module deepdrr.utils)@\spxentry{param\_saver()}\spxextra{in module deepdrr.utils}}

\begin{fulllineitems}
\phantomsection\label{\detokenize{deepdrr.utils:deepdrr.utils.param_saver}}
\pysigstartsignatures
\pysiglinewithargsret{\sphinxcode{\sphinxupquote{deepdrr.utils.}}\sphinxbfcode{\sphinxupquote{param\_saver}}}{\sphinxparam{\DUrole{n,n}{thetas}}\sphinxparamcomma \sphinxparam{\DUrole{n,n}{phis}}\sphinxparamcomma \sphinxparam{\DUrole{n,n}{proj\_mats}}\sphinxparamcomma \sphinxparam{\DUrole{n,n}{camera}}\sphinxparamcomma \sphinxparam{\DUrole{n,n}{origin}}\sphinxparamcomma \sphinxparam{\DUrole{n,n}{photons}}\sphinxparamcomma \sphinxparam{\DUrole{n,n}{spectrum}}\sphinxparamcomma \sphinxparam{\DUrole{n,n}{prefix}}\sphinxparamcomma \sphinxparam{\DUrole{n,n}{save\_path}}}{}
\pysigstopsignatures
\sphinxAtStartPar
Save the paramaters.

\sphinxAtStartPar
This function may be deprecated.
\begin{quote}\begin{description}
\sphinxlineitem{Parameters}\begin{itemize}
\item {} 
\sphinxAtStartPar
\sphinxstyleliteralstrong{\sphinxupquote{thetas}} (\sphinxstyleliteralemphasis{\sphinxupquote{{[}}}\sphinxstyleliteralemphasis{\sphinxupquote{type}}\sphinxstyleliteralemphasis{\sphinxupquote{{]}}}) \textendash{} {[}description{]}

\item {} 
\sphinxAtStartPar
\sphinxstyleliteralstrong{\sphinxupquote{phis}} (\sphinxstyleliteralemphasis{\sphinxupquote{{[}}}\sphinxstyleliteralemphasis{\sphinxupquote{type}}\sphinxstyleliteralemphasis{\sphinxupquote{{]}}}) \textendash{} {[}description{]}

\item {} 
\sphinxAtStartPar
\sphinxstyleliteralstrong{\sphinxupquote{proj\_mats}} (\sphinxstyleliteralemphasis{\sphinxupquote{{[}}}\sphinxstyleliteralemphasis{\sphinxupquote{type}}\sphinxstyleliteralemphasis{\sphinxupquote{{]}}}) \textendash{} {[}description{]}

\item {} 
\sphinxAtStartPar
\sphinxstyleliteralstrong{\sphinxupquote{camera}} (\sphinxstyleliteralemphasis{\sphinxupquote{{[}}}\sphinxstyleliteralemphasis{\sphinxupquote{type}}\sphinxstyleliteralemphasis{\sphinxupquote{{]}}}) \textendash{} {[}description{]}

\item {} 
\sphinxAtStartPar
\sphinxstyleliteralstrong{\sphinxupquote{origin}} (\sphinxstyleliteralemphasis{\sphinxupquote{{[}}}\sphinxstyleliteralemphasis{\sphinxupquote{type}}\sphinxstyleliteralemphasis{\sphinxupquote{{]}}}) \textendash{} {[}description{]}

\item {} 
\sphinxAtStartPar
\sphinxstyleliteralstrong{\sphinxupquote{photons}} (\sphinxstyleliteralemphasis{\sphinxupquote{{[}}}\sphinxstyleliteralemphasis{\sphinxupquote{type}}\sphinxstyleliteralemphasis{\sphinxupquote{{]}}}) \textendash{} {[}description{]}

\item {} 
\sphinxAtStartPar
\sphinxstyleliteralstrong{\sphinxupquote{spectrum}} (\sphinxstyleliteralemphasis{\sphinxupquote{{[}}}\sphinxstyleliteralemphasis{\sphinxupquote{type}}\sphinxstyleliteralemphasis{\sphinxupquote{{]}}}) \textendash{} {[}description{]}

\item {} 
\sphinxAtStartPar
\sphinxstyleliteralstrong{\sphinxupquote{prefix}} (\sphinxstyleliteralemphasis{\sphinxupquote{{[}}}\sphinxstyleliteralemphasis{\sphinxupquote{type}}\sphinxstyleliteralemphasis{\sphinxupquote{{]}}}) \textendash{} {[}description{]}

\item {} 
\sphinxAtStartPar
\sphinxstyleliteralstrong{\sphinxupquote{save\_path}} (\sphinxstyleliteralemphasis{\sphinxupquote{{[}}}\sphinxstyleliteralemphasis{\sphinxupquote{type}}\sphinxstyleliteralemphasis{\sphinxupquote{{]}}}) \textendash{} {[}description{]}

\end{itemize}

\sphinxlineitem{Returns}
\sphinxAtStartPar
{[}description{]}

\sphinxlineitem{Return type}
\sphinxAtStartPar
{[}type{]}

\end{description}\end{quote}

\end{fulllineitems}

\index{radians() (in module deepdrr.utils)@\spxentry{radians()}\spxextra{in module deepdrr.utils}}

\begin{fulllineitems}
\phantomsection\label{\detokenize{deepdrr.utils:deepdrr.utils.radians}}
\pysigstartsignatures
\pysiglinewithargsret{\sphinxcode{\sphinxupquote{deepdrr.utils.}}\sphinxbfcode{\sphinxupquote{radians}}}{\sphinxparam{\DUrole{n,n}{t}\DUrole{p,p}{:}\DUrole{w,w}{  }\DUrole{n,n}{float}}\sphinxparamcomma \sphinxparam{\DUrole{n,n}{degrees}\DUrole{p,p}{:}\DUrole{w,w}{  }\DUrole{n,n}{bool}}}{{ $\rightarrow$ float}}
\pysiglinewithargsret{\sphinxcode{\sphinxupquote{deepdrr.utils.}}\sphinxbfcode{\sphinxupquote{radians}}}{\sphinxparam{\DUrole{n,n}{t}\DUrole{p,p}{:}\DUrole{w,w}{  }\DUrole{n,n}{ndarray}}\sphinxparamcomma \sphinxparam{\DUrole{n,n}{degrees}\DUrole{p,p}{:}\DUrole{w,w}{  }\DUrole{n,n}{bool}}}{{ $\rightarrow$ ndarray}}
\pysiglinewithargsret{\sphinxcode{\sphinxupquote{deepdrr.utils.}}\sphinxbfcode{\sphinxupquote{radians}}}{\sphinxparam{\DUrole{n,n}{ts}\DUrole{p,p}{:}\DUrole{w,w}{  }\DUrole{n,n}{List\DUrole{p,p}{{[}}T\DUrole{p,p}{{]}}}}\sphinxparamcomma \sphinxparam{\DUrole{n,n}{degrees}\DUrole{p,p}{:}\DUrole{w,w}{  }\DUrole{n,n}{bool}}}{{ $\rightarrow$ List\DUrole{p,p}{{[}}T\DUrole{p,p}{{]}}}}
\pysiglinewithargsret{\sphinxcode{\sphinxupquote{deepdrr.utils.}}\sphinxbfcode{\sphinxupquote{radians}}}{\sphinxparam{\DUrole{n,n}{ts}\DUrole{p,p}{:}\DUrole{w,w}{  }\DUrole{n,n}{Dict\DUrole{p,p}{{[}}S\DUrole{p,p}{,}\DUrole{w,w}{  }T\DUrole{p,p}{{]}}}}\sphinxparamcomma \sphinxparam{\DUrole{n,n}{degrees}\DUrole{p,p}{:}\DUrole{w,w}{  }\DUrole{n,n}{bool}}}{{ $\rightarrow$ Dict\DUrole{p,p}{{[}}S\DUrole{p,p}{,}\DUrole{w,w}{  }T\DUrole{p,p}{{]}}}}
\pysiglinewithargsret{\sphinxcode{\sphinxupquote{deepdrr.utils.}}\sphinxbfcode{\sphinxupquote{radians}}}{\sphinxparam{\DUrole{o,o}{*}\DUrole{n,n}{ts}\DUrole{p,p}{:}\DUrole{w,w}{  }\DUrole{n,n}{T}}\sphinxparamcomma \sphinxparam{\DUrole{n,n}{degrees}\DUrole{p,p}{:}\DUrole{w,w}{  }\DUrole{n,n}{bool}}}{{ $\rightarrow$ List\DUrole{p,p}{{[}}T\DUrole{p,p}{{]}}}}
\pysigstopsignatures
\sphinxAtStartPar
Convert to radians.
\begin{quote}\begin{description}
\sphinxlineitem{Parameters}\begin{itemize}
\item {} 
\sphinxAtStartPar
\sphinxstyleliteralstrong{\sphinxupquote{ts}} \textendash{} the angle or array of angles.

\item {} 
\sphinxAtStartPar
\sphinxstyleliteralstrong{\sphinxupquote{degrees}} (\sphinxstyleliteralemphasis{\sphinxupquote{bool}}\sphinxstyleliteralemphasis{\sphinxupquote{, }}\sphinxstyleliteralemphasis{\sphinxupquote{optional}}) \textendash{} whether the inputs are in degrees. If False, this is a no\sphinxhyphen{}op. Defaults to True.

\end{itemize}

\sphinxlineitem{Returns}
\sphinxAtStartPar
each argument, converted to radians.

\sphinxlineitem{Return type}
\sphinxAtStartPar
Union{[}float, List{[}float{]}{]}

\end{description}\end{quote}

\end{fulllineitems}

\index{try\_import\_pyvista() (in module deepdrr.utils)@\spxentry{try\_import\_pyvista()}\spxextra{in module deepdrr.utils}}

\begin{fulllineitems}
\phantomsection\label{\detokenize{deepdrr.utils:deepdrr.utils.try_import_pyvista}}
\pysigstartsignatures
\pysiglinewithargsret{\sphinxcode{\sphinxupquote{deepdrr.utils.}}\sphinxbfcode{\sphinxupquote{try\_import\_pyvista}}}{}{}
\pysigstopsignatures
\end{fulllineitems}

\index{try\_import\_vtk() (in module deepdrr.utils)@\spxentry{try\_import\_vtk()}\spxextra{in module deepdrr.utils}}

\begin{fulllineitems}
\phantomsection\label{\detokenize{deepdrr.utils:deepdrr.utils.try_import_vtk}}
\pysigstartsignatures
\pysiglinewithargsret{\sphinxcode{\sphinxupquote{deepdrr.utils.}}\sphinxbfcode{\sphinxupquote{try\_import\_vtk}}}{}{}
\pysigstopsignatures
\end{fulllineitems}

\index{tuplify() (in module deepdrr.utils)@\spxentry{tuplify()}\spxextra{in module deepdrr.utils}}

\begin{fulllineitems}
\phantomsection\label{\detokenize{deepdrr.utils:deepdrr.utils.tuplify}}
\pysigstartsignatures
\pysiglinewithargsret{\sphinxcode{\sphinxupquote{deepdrr.utils.}}\sphinxbfcode{\sphinxupquote{tuplify}}}{\sphinxparam{\DUrole{n,n}{t}\DUrole{p,p}{:}\DUrole{w,w}{  }\DUrole{n,n}{Tuple\DUrole{p,p}{{[}}T\DUrole{p,p}{,}\DUrole{w,w}{  }\DUrole{p,p}{...}\DUrole{p,p}{{]}}\DUrole{w,w}{  }\DUrole{p,p}{|}\DUrole{w,w}{  }T}}\sphinxparamcomma \sphinxparam{\DUrole{n,n}{n}\DUrole{p,p}{:}\DUrole{w,w}{  }\DUrole{n,n}{int}\DUrole{w,w}{  }\DUrole{o,o}{=}\DUrole{w,w}{  }\DUrole{default_value}{1}}}{{ $\rightarrow$ Tuple\DUrole{p,p}{{[}}T\DUrole{p,p}{,}\DUrole{w,w}{  }\DUrole{p,p}{...}\DUrole{p,p}{{]}}}}
\pysigstopsignatures
\sphinxAtStartPar
Create a tuple with \sphinxtitleref{n} copies of \sphinxtitleref{t},  if \sphinxtitleref{t} is not already a tuple of length \sphinxtitleref{n}.

\end{fulllineitems}


\sphinxstepscope


\section{deepdrr.vol package}
\label{\detokenize{deepdrr.vol:deepdrr-vol-package}}\label{\detokenize{deepdrr.vol::doc}}

\subsection{deepdrr.vol.kwire module}
\label{\detokenize{deepdrr.vol:module-deepdrr.vol.kwire}}\label{\detokenize{deepdrr.vol:deepdrr-vol-kwire-module}}\index{module@\spxentry{module}!deepdrr.vol.kwire@\spxentry{deepdrr.vol.kwire}}\index{deepdrr.vol.kwire@\spxentry{deepdrr.vol.kwire}!module@\spxentry{module}}\index{KWire (class in deepdrr.vol.kwire)@\spxentry{KWire}\spxextra{class in deepdrr.vol.kwire}}

\begin{fulllineitems}
\phantomsection\label{\detokenize{deepdrr.vol:deepdrr.vol.kwire.KWire}}
\pysigstartsignatures
\pysiglinewithargsret{\sphinxbfcode{\sphinxupquote{class\DUrole{w,w}{  }}}\sphinxcode{\sphinxupquote{deepdrr.vol.kwire.}}\sphinxbfcode{\sphinxupquote{KWire}}}{\sphinxparam{\DUrole{o,o}{*}\DUrole{n,n}{args}}\sphinxparamcomma \sphinxparam{\DUrole{n,n}{tip}\DUrole{p,p}{:}\DUrole{w,w}{  }\DUrole{n,n}{{\hyperref[\detokenize{deepdrr.geo:deepdrr.geo.core.Point3D}]{\sphinxcrossref{Point3D}}}\DUrole{w,w}{  }\DUrole{p,p}{|}\DUrole{w,w}{  }None}\DUrole{w,w}{  }\DUrole{o,o}{=}\DUrole{w,w}{  }\DUrole{default_value}{None}}\sphinxparamcomma \sphinxparam{\DUrole{n,n}{base}\DUrole{p,p}{:}\DUrole{w,w}{  }\DUrole{n,n}{{\hyperref[\detokenize{deepdrr.geo:deepdrr.geo.core.Point3D}]{\sphinxcrossref{Point3D}}}\DUrole{w,w}{  }\DUrole{p,p}{|}\DUrole{w,w}{  }None}\DUrole{w,w}{  }\DUrole{o,o}{=}\DUrole{w,w}{  }\DUrole{default_value}{None}}\sphinxparamcomma \sphinxparam{\DUrole{o,o}{**}\DUrole{n,n}{kwargs}}}{}
\pysigstopsignatures
\sphinxAtStartPar
Bases: {\hyperref[\detokenize{deepdrr.vol:deepdrr.vol.volume.Volume}]{\sphinxcrossref{\sphinxcode{\sphinxupquote{Volume}}}}}
\index{advance() (deepdrr.vol.kwire.KWire method)@\spxentry{advance()}\spxextra{deepdrr.vol.kwire.KWire method}}

\begin{fulllineitems}
\phantomsection\label{\detokenize{deepdrr.vol:deepdrr.vol.kwire.KWire.advance}}
\pysigstartsignatures
\pysiglinewithargsret{\sphinxbfcode{\sphinxupquote{advance}}}{\sphinxparam{\DUrole{n,n}{distance}\DUrole{p,p}{:}\DUrole{w,w}{  }\DUrole{n,n}{float}}}{}
\pysigstopsignatures
\sphinxAtStartPar
Move the tool forward by the given distance.
\begin{quote}\begin{description}
\sphinxlineitem{Parameters}
\sphinxAtStartPar
\sphinxstyleliteralstrong{\sphinxupquote{distance}} (\sphinxstyleliteralemphasis{\sphinxupquote{float}}) \textendash{} The distance to move the tool forward.

\end{description}\end{quote}

\end{fulllineitems}

\index{align() (deepdrr.vol.kwire.KWire method)@\spxentry{align()}\spxextra{deepdrr.vol.kwire.KWire method}}

\begin{fulllineitems}
\phantomsection\label{\detokenize{deepdrr.vol:deepdrr.vol.kwire.KWire.align}}
\pysigstartsignatures
\pysiglinewithargsret{\sphinxbfcode{\sphinxupquote{align}}}{\sphinxparam{\DUrole{n,n}{startpoint\_in\_world}\DUrole{p,p}{:}\DUrole{w,w}{  }\DUrole{n,n}{{\hyperref[\detokenize{deepdrr.geo:deepdrr.geo.core.Point3D}]{\sphinxcrossref{Point3D}}}}}\sphinxparamcomma \sphinxparam{\DUrole{n,n}{endpoint\_in\_world}\DUrole{p,p}{:}\DUrole{w,w}{  }\DUrole{n,n}{{\hyperref[\detokenize{deepdrr.geo:deepdrr.geo.core.Point3D}]{\sphinxcrossref{Point3D}}}}}\sphinxparamcomma \sphinxparam{\DUrole{n,n}{progress}\DUrole{p,p}{:}\DUrole{w,w}{  }\DUrole{n,n}{float}\DUrole{w,w}{  }\DUrole{o,o}{=}\DUrole{w,w}{  }\DUrole{default_value}{1.0}}\sphinxparamcomma \sphinxparam{\DUrole{n,n}{distance}\DUrole{p,p}{:}\DUrole{w,w}{  }\DUrole{n,n}{float\DUrole{w,w}{  }\DUrole{p,p}{|}\DUrole{w,w}{  }None}\DUrole{w,w}{  }\DUrole{o,o}{=}\DUrole{w,w}{  }\DUrole{default_value}{None}}}{{ $\rightarrow$ None}}
\pysigstopsignatures
\sphinxAtStartPar
Align the tool so that it lies between the two points, tip pointing toward the endpoint.
\begin{quote}\begin{description}
\sphinxlineitem{Parameters}\begin{itemize}
\item {} 
\sphinxAtStartPar
\sphinxstyleliteralstrong{\sphinxupquote{start\_point\_in\_world}} ({\hyperref[\detokenize{deepdrr.geo:deepdrr.geo.Point3D}]{\sphinxcrossref{\sphinxstyleliteralemphasis{\sphinxupquote{geo.Point3D}}}}}) \textendash{} The first point, in world space.

\item {} 
\sphinxAtStartPar
\sphinxstyleliteralstrong{\sphinxupquote{end\_point\_in\_world}} ({\hyperref[\detokenize{deepdrr.geo:deepdrr.geo.Point3D}]{\sphinxcrossref{\sphinxstyleliteralemphasis{\sphinxupquote{geo.Point3D}}}}}) \textendash{} The second point, in world space. The tip of the tool points toward this point.

\item {} 
\sphinxAtStartPar
\sphinxstyleliteralstrong{\sphinxupquote{progress}} (\sphinxstyleliteralemphasis{\sphinxupquote{float}}\sphinxstyleliteralemphasis{\sphinxupquote{, }}\sphinxstyleliteralemphasis{\sphinxupquote{optional}}) \textendash{} Where to place the tip of the tool between the start and end point,
on a scale from 0 to 1. 0 corresponds to the tip placed at the start point, 1 at the end point. Defaults to 1.0.

\item {} 
\sphinxAtStartPar
\sphinxstyleliteralstrong{\sphinxupquote{distance}} (\sphinxstyleliteralemphasis{\sphinxupquote{Optional}}\sphinxstyleliteralemphasis{\sphinxupquote{{[}}}\sphinxstyleliteralemphasis{\sphinxupquote{float}}\sphinxstyleliteralemphasis{\sphinxupquote{{]}}}\sphinxstyleliteralemphasis{\sphinxupquote{, }}\sphinxstyleliteralemphasis{\sphinxupquote{optional}}) \textendash{} The distance of the tip along the trajectory. 0 corresponds
to the tip placed at the start point, {\color{red}\bfseries{}|startpoint \sphinxhyphen{} endpoint|} at the end point.
Overrides progress if provided. Defaults to None.

\end{itemize}

\end{description}\end{quote}

\end{fulllineitems}

\index{base (deepdrr.vol.kwire.KWire property)@\spxentry{base}\spxextra{deepdrr.vol.kwire.KWire property}}

\begin{fulllineitems}
\phantomsection\label{\detokenize{deepdrr.vol:deepdrr.vol.kwire.KWire.base}}
\pysigstartsignatures
\pysigline{\sphinxbfcode{\sphinxupquote{property\DUrole{w,w}{  }}}\sphinxbfcode{\sphinxupquote{base}}\sphinxbfcode{\sphinxupquote{\DUrole{p,p}{:}\DUrole{w,w}{  }{\hyperref[\detokenize{deepdrr.geo:deepdrr.geo.core.Point3D}]{\sphinxcrossref{Point3D}}}}}}
\pysigstopsignatures
\sphinxAtStartPar
The base of the tool in world space.

\end{fulllineitems}

\index{base\_in\_IJK (deepdrr.vol.kwire.KWire attribute)@\spxentry{base\_in\_IJK}\spxextra{deepdrr.vol.kwire.KWire attribute}}

\begin{fulllineitems}
\phantomsection\label{\detokenize{deepdrr.vol:deepdrr.vol.kwire.KWire.base_in_IJK}}
\pysigstartsignatures
\pysigline{\sphinxbfcode{\sphinxupquote{base\_in\_IJK}}\sphinxbfcode{\sphinxupquote{\DUrole{p,p}{:}\DUrole{w,w}{  }{\hyperref[\detokenize{deepdrr.geo:deepdrr.geo.core.Point3D}]{\sphinxcrossref{Point3D}}}}}}
\pysigstopsignatures
\end{fulllineitems}

\index{base\_in\_anatomical (deepdrr.vol.kwire.KWire property)@\spxentry{base\_in\_anatomical}\spxextra{deepdrr.vol.kwire.KWire property}}

\begin{fulllineitems}
\phantomsection\label{\detokenize{deepdrr.vol:deepdrr.vol.kwire.KWire.base_in_anatomical}}
\pysigstartsignatures
\pysigline{\sphinxbfcode{\sphinxupquote{property\DUrole{w,w}{  }}}\sphinxbfcode{\sphinxupquote{base\_in\_anatomical}}\sphinxbfcode{\sphinxupquote{\DUrole{p,p}{:}\DUrole{w,w}{  }{\hyperref[\detokenize{deepdrr.geo:deepdrr.geo.core.Point3D}]{\sphinxcrossref{Point3D}}}}}}
\pysigstopsignatures
\sphinxAtStartPar
Get the location of the tool base in anatomical coordinates.

\end{fulllineitems}

\index{base\_in\_ijk (deepdrr.vol.kwire.KWire property)@\spxentry{base\_in\_ijk}\spxextra{deepdrr.vol.kwire.KWire property}}

\begin{fulllineitems}
\phantomsection\label{\detokenize{deepdrr.vol:deepdrr.vol.kwire.KWire.base_in_ijk}}
\pysigstartsignatures
\pysigline{\sphinxbfcode{\sphinxupquote{property\DUrole{w,w}{  }}}\sphinxbfcode{\sphinxupquote{base\_in\_ijk}}\sphinxbfcode{\sphinxupquote{\DUrole{p,p}{:}\DUrole{w,w}{  }{\hyperref[\detokenize{deepdrr.geo:deepdrr.geo.core.Point3D}]{\sphinxcrossref{Point3D}}}}}}
\pysigstopsignatures
\end{fulllineitems}

\index{base\_in\_world (deepdrr.vol.kwire.KWire property)@\spxentry{base\_in\_world}\spxextra{deepdrr.vol.kwire.KWire property}}

\begin{fulllineitems}
\phantomsection\label{\detokenize{deepdrr.vol:deepdrr.vol.kwire.KWire.base_in_world}}
\pysigstartsignatures
\pysigline{\sphinxbfcode{\sphinxupquote{property\DUrole{w,w}{  }}}\sphinxbfcode{\sphinxupquote{base\_in\_world}}\sphinxbfcode{\sphinxupquote{\DUrole{p,p}{:}\DUrole{w,w}{  }{\hyperref[\detokenize{deepdrr.geo:deepdrr.geo.core.Point3D}]{\sphinxcrossref{Point3D}}}}}}
\pysigstopsignatures
\sphinxAtStartPar
Get the location of the tool base in world coordinates.

\end{fulllineitems}

\index{centerline\_in\_world (deepdrr.vol.kwire.KWire property)@\spxentry{centerline\_in\_world}\spxextra{deepdrr.vol.kwire.KWire property}}

\begin{fulllineitems}
\phantomsection\label{\detokenize{deepdrr.vol:deepdrr.vol.kwire.KWire.centerline_in_world}}
\pysigstartsignatures
\pysigline{\sphinxbfcode{\sphinxupquote{property\DUrole{w,w}{  }}}\sphinxbfcode{\sphinxupquote{centerline\_in\_world}}\sphinxbfcode{\sphinxupquote{\DUrole{p,p}{:}\DUrole{w,w}{  }{\hyperref[\detokenize{deepdrr.geo:deepdrr.geo.hyperplane.Line3D}]{\sphinxcrossref{Line3D}}}}}}
\pysigstopsignatures
\end{fulllineitems}

\index{diameter (deepdrr.vol.kwire.KWire attribute)@\spxentry{diameter}\spxextra{deepdrr.vol.kwire.KWire attribute}}

\begin{fulllineitems}
\phantomsection\label{\detokenize{deepdrr.vol:deepdrr.vol.kwire.KWire.diameter}}
\pysigstartsignatures
\pysigline{\sphinxbfcode{\sphinxupquote{diameter}}\sphinxbfcode{\sphinxupquote{\DUrole{w,w}{  }\DUrole{p,p}{=}\DUrole{w,w}{  }2.0}}}
\pysigstopsignatures
\end{fulllineitems}

\index{from\_example() (deepdrr.vol.kwire.KWire class method)@\spxentry{from\_example()}\spxextra{deepdrr.vol.kwire.KWire class method}}

\begin{fulllineitems}
\phantomsection\label{\detokenize{deepdrr.vol:deepdrr.vol.kwire.KWire.from_example}}
\pysigstartsignatures
\pysiglinewithargsret{\sphinxbfcode{\sphinxupquote{classmethod\DUrole{w,w}{  }}}\sphinxbfcode{\sphinxupquote{from\_example}}}{\sphinxparam{\DUrole{n,n}{diameter}\DUrole{p,p}{:}\DUrole{w,w}{  }\DUrole{n,n}{float}\DUrole{w,w}{  }\DUrole{o,o}{=}\DUrole{w,w}{  }\DUrole{default_value}{2}}\sphinxparamcomma \sphinxparam{\DUrole{n,n}{density}\DUrole{p,p}{:}\DUrole{w,w}{  }\DUrole{n,n}{float}\DUrole{w,w}{  }\DUrole{o,o}{=}\DUrole{w,w}{  }\DUrole{default_value}{7.5}}\sphinxparamcomma \sphinxparam{\DUrole{n,n}{world\_from\_anatomical}\DUrole{p,p}{:}\DUrole{w,w}{  }\DUrole{n,n}{{\hyperref[\detokenize{deepdrr.geo:deepdrr.geo.core.F}]{\sphinxcrossref{F}}}\DUrole{w,w}{  }\DUrole{p,p}{|}\DUrole{w,w}{  }None}\DUrole{w,w}{  }\DUrole{o,o}{=}\DUrole{w,w}{  }\DUrole{default_value}{None}}\sphinxparamcomma \sphinxparam{\DUrole{o,o}{**}\DUrole{n,n}{kwargs}}}{}
\pysigstopsignatures
\sphinxAtStartPar
Creates a KWire from the provided download link.
\begin{quote}\begin{description}
\sphinxlineitem{Parameters}
\sphinxAtStartPar
\sphinxstyleliteralstrong{\sphinxupquote{density}} (\sphinxstyleliteralemphasis{\sphinxupquote{float}}\sphinxstyleliteralemphasis{\sphinxupquote{, }}\sphinxstyleliteralemphasis{\sphinxupquote{optional}}) \textendash{} Density of the K\sphinxhyphen{}wire metal.

\sphinxlineitem{Returns}
\sphinxAtStartPar
The example KWire built into DeepDRR.

\sphinxlineitem{Return type}
\sphinxAtStartPar
{\hyperref[\detokenize{deepdrr.vol:deepdrr.vol.kwire.KWire}]{\sphinxcrossref{KWire}}}

\end{description}\end{quote}

\end{fulllineitems}

\index{length\_in\_world (deepdrr.vol.kwire.KWire property)@\spxentry{length\_in\_world}\spxextra{deepdrr.vol.kwire.KWire property}}

\begin{fulllineitems}
\phantomsection\label{\detokenize{deepdrr.vol:deepdrr.vol.kwire.KWire.length_in_world}}
\pysigstartsignatures
\pysigline{\sphinxbfcode{\sphinxupquote{property\DUrole{w,w}{  }}}\sphinxbfcode{\sphinxupquote{length\_in\_world}}}
\pysigstopsignatures
\end{fulllineitems}

\index{orient() (deepdrr.vol.kwire.KWire method)@\spxentry{orient()}\spxextra{deepdrr.vol.kwire.KWire method}}

\begin{fulllineitems}
\phantomsection\label{\detokenize{deepdrr.vol:deepdrr.vol.kwire.KWire.orient}}
\pysigstartsignatures
\pysiglinewithargsret{\sphinxbfcode{\sphinxupquote{orient}}}{\sphinxparam{\DUrole{n,n}{startpoint}\DUrole{p,p}{:}\DUrole{w,w}{  }\DUrole{n,n}{{\hyperref[\detokenize{deepdrr.geo:deepdrr.geo.core.Point3D}]{\sphinxcrossref{Point3D}}}}}\sphinxparamcomma \sphinxparam{\DUrole{n,n}{direction}\DUrole{p,p}{:}\DUrole{w,w}{  }\DUrole{n,n}{{\hyperref[\detokenize{deepdrr.geo:deepdrr.geo.core.Vector3D}]{\sphinxcrossref{Vector3D}}}}}\sphinxparamcomma \sphinxparam{\DUrole{n,n}{distance}\DUrole{p,p}{:}\DUrole{w,w}{  }\DUrole{n,n}{float}\DUrole{w,w}{  }\DUrole{o,o}{=}\DUrole{w,w}{  }\DUrole{default_value}{0}}}{}
\pysigstopsignatures
\sphinxAtStartPar
Place the tip at startpoint and orient the tool to point toward the direction.

\end{fulllineitems}

\index{radius (deepdrr.vol.kwire.KWire property)@\spxentry{radius}\spxextra{deepdrr.vol.kwire.KWire property}}

\begin{fulllineitems}
\phantomsection\label{\detokenize{deepdrr.vol:deepdrr.vol.kwire.KWire.radius}}
\pysigstartsignatures
\pysigline{\sphinxbfcode{\sphinxupquote{property\DUrole{w,w}{  }}}\sphinxbfcode{\sphinxupquote{radius}}\sphinxbfcode{\sphinxupquote{\DUrole{p,p}{:}\DUrole{w,w}{  }float}}}
\pysigstopsignatures
\end{fulllineitems}

\index{scale() (deepdrr.vol.kwire.KWire method)@\spxentry{scale()}\spxextra{deepdrr.vol.kwire.KWire method}}

\begin{fulllineitems}
\phantomsection\label{\detokenize{deepdrr.vol:deepdrr.vol.kwire.KWire.scale}}
\pysigstartsignatures
\pysiglinewithargsret{\sphinxbfcode{\sphinxupquote{scale}}}{\sphinxparam{\DUrole{n,n}{factor}\DUrole{p,p}{:}\DUrole{w,w}{  }\DUrole{n,n}{float}}}{{ $\rightarrow$ None}}
\pysigstopsignatures
\sphinxAtStartPar
Scales the volume by the given factor.
\begin{quote}\begin{description}
\sphinxlineitem{Parameters}
\sphinxAtStartPar
\sphinxstyleliteralstrong{\sphinxupquote{factor}} (\sphinxstyleliteralemphasis{\sphinxupquote{float}}) \textendash{} The factor by which to scale the tool. 1 would be no scaling.

\end{description}\end{quote}

\end{fulllineitems}

\index{tip (deepdrr.vol.kwire.KWire property)@\spxentry{tip}\spxextra{deepdrr.vol.kwire.KWire property}}

\begin{fulllineitems}
\phantomsection\label{\detokenize{deepdrr.vol:deepdrr.vol.kwire.KWire.tip}}
\pysigstartsignatures
\pysigline{\sphinxbfcode{\sphinxupquote{property\DUrole{w,w}{  }}}\sphinxbfcode{\sphinxupquote{tip}}\sphinxbfcode{\sphinxupquote{\DUrole{p,p}{:}\DUrole{w,w}{  }{\hyperref[\detokenize{deepdrr.geo:deepdrr.geo.core.Point3D}]{\sphinxcrossref{Point3D}}}}}}
\pysigstopsignatures
\sphinxAtStartPar
The tip of the tool in world space.

\end{fulllineitems}

\index{tip\_in\_IJK (deepdrr.vol.kwire.KWire attribute)@\spxentry{tip\_in\_IJK}\spxextra{deepdrr.vol.kwire.KWire attribute}}

\begin{fulllineitems}
\phantomsection\label{\detokenize{deepdrr.vol:deepdrr.vol.kwire.KWire.tip_in_IJK}}
\pysigstartsignatures
\pysigline{\sphinxbfcode{\sphinxupquote{tip\_in\_IJK}}\sphinxbfcode{\sphinxupquote{\DUrole{p,p}{:}\DUrole{w,w}{  }{\hyperref[\detokenize{deepdrr.geo:deepdrr.geo.core.Point3D}]{\sphinxcrossref{Point3D}}}}}}
\pysigstopsignatures
\end{fulllineitems}

\index{tip\_in\_anatomical (deepdrr.vol.kwire.KWire property)@\spxentry{tip\_in\_anatomical}\spxextra{deepdrr.vol.kwire.KWire property}}

\begin{fulllineitems}
\phantomsection\label{\detokenize{deepdrr.vol:deepdrr.vol.kwire.KWire.tip_in_anatomical}}
\pysigstartsignatures
\pysigline{\sphinxbfcode{\sphinxupquote{property\DUrole{w,w}{  }}}\sphinxbfcode{\sphinxupquote{tip\_in\_anatomical}}\sphinxbfcode{\sphinxupquote{\DUrole{p,p}{:}\DUrole{w,w}{  }{\hyperref[\detokenize{deepdrr.geo:deepdrr.geo.core.Point3D}]{\sphinxcrossref{Point3D}}}}}}
\pysigstopsignatures
\sphinxAtStartPar
Get the location of the tool tip (the pointy end) in anatomical coordinates.

\end{fulllineitems}

\index{tip\_in\_ijk (deepdrr.vol.kwire.KWire property)@\spxentry{tip\_in\_ijk}\spxextra{deepdrr.vol.kwire.KWire property}}

\begin{fulllineitems}
\phantomsection\label{\detokenize{deepdrr.vol:deepdrr.vol.kwire.KWire.tip_in_ijk}}
\pysigstartsignatures
\pysigline{\sphinxbfcode{\sphinxupquote{property\DUrole{w,w}{  }}}\sphinxbfcode{\sphinxupquote{tip\_in\_ijk}}\sphinxbfcode{\sphinxupquote{\DUrole{p,p}{:}\DUrole{w,w}{  }{\hyperref[\detokenize{deepdrr.geo:deepdrr.geo.core.Point3D}]{\sphinxcrossref{Point3D}}}}}}
\pysigstopsignatures
\end{fulllineitems}

\index{tip\_in\_world (deepdrr.vol.kwire.KWire property)@\spxentry{tip\_in\_world}\spxextra{deepdrr.vol.kwire.KWire property}}

\begin{fulllineitems}
\phantomsection\label{\detokenize{deepdrr.vol:deepdrr.vol.kwire.KWire.tip_in_world}}
\pysigstartsignatures
\pysigline{\sphinxbfcode{\sphinxupquote{property\DUrole{w,w}{  }}}\sphinxbfcode{\sphinxupquote{tip\_in\_world}}\sphinxbfcode{\sphinxupquote{\DUrole{p,p}{:}\DUrole{w,w}{  }{\hyperref[\detokenize{deepdrr.geo:deepdrr.geo.core.Point3D}]{\sphinxcrossref{Point3D}}}}}}
\pysigstopsignatures
\sphinxAtStartPar
Get the location of the tool tip (the pointy end) in world coordinates.

\end{fulllineitems}

\index{trajectory\_in\_world (deepdrr.vol.kwire.KWire property)@\spxentry{trajectory\_in\_world}\spxextra{deepdrr.vol.kwire.KWire property}}

\begin{fulllineitems}
\phantomsection\label{\detokenize{deepdrr.vol:deepdrr.vol.kwire.KWire.trajectory_in_world}}
\pysigstartsignatures
\pysigline{\sphinxbfcode{\sphinxupquote{property\DUrole{w,w}{  }}}\sphinxbfcode{\sphinxupquote{trajectory\_in\_world}}\sphinxbfcode{\sphinxupquote{\DUrole{p,p}{:}\DUrole{w,w}{  }{\hyperref[\detokenize{deepdrr.geo:deepdrr.geo.ray.Ray3D}]{\sphinxcrossref{Ray3D}}}}}}
\pysigstopsignatures
\end{fulllineitems}

\index{twist() (deepdrr.vol.kwire.KWire method)@\spxentry{twist()}\spxextra{deepdrr.vol.kwire.KWire method}}

\begin{fulllineitems}
\phantomsection\label{\detokenize{deepdrr.vol:deepdrr.vol.kwire.KWire.twist}}
\pysigstartsignatures
\pysiglinewithargsret{\sphinxbfcode{\sphinxupquote{twist}}}{\sphinxparam{\DUrole{n,n}{angle}\DUrole{p,p}{:}\DUrole{w,w}{  }\DUrole{n,n}{float}}\sphinxparamcomma \sphinxparam{\DUrole{n,n}{degrees}\DUrole{p,p}{:}\DUrole{w,w}{  }\DUrole{n,n}{bool}\DUrole{w,w}{  }\DUrole{o,o}{=}\DUrole{w,w}{  }\DUrole{default_value}{True}}}{}
\pysigstopsignatures
\sphinxAtStartPar
Rotate the tool clockwise (when looking down on it) by \sphinxtitleref{angle}.
\begin{quote}\begin{description}
\sphinxlineitem{Parameters}\begin{itemize}
\item {} 
\sphinxAtStartPar
\sphinxstyleliteralstrong{\sphinxupquote{angle}} (\sphinxstyleliteralemphasis{\sphinxupquote{float}}) \textendash{} The angle.

\item {} 
\sphinxAtStartPar
\sphinxstyleliteralstrong{\sphinxupquote{degrees}} (\sphinxstyleliteralemphasis{\sphinxupquote{bool}}\sphinxstyleliteralemphasis{\sphinxupquote{, }}\sphinxstyleliteralemphasis{\sphinxupquote{optional}}) \textendash{} Whether \sphinxtitleref{angle} is in degrees. Defaults to True.

\end{itemize}

\end{description}\end{quote}

\end{fulllineitems}


\end{fulllineitems}



\subsection{deepdrr.vol.volume module}
\label{\detokenize{deepdrr.vol:module-deepdrr.vol.volume}}\label{\detokenize{deepdrr.vol:deepdrr-vol-volume-module}}\index{module@\spxentry{module}!deepdrr.vol.volume@\spxentry{deepdrr.vol.volume}}\index{deepdrr.vol.volume@\spxentry{deepdrr.vol.volume}!module@\spxentry{module}}
\sphinxAtStartPar
Volume class for CT volume.
\index{MetalVolume (class in deepdrr.vol.volume)@\spxentry{MetalVolume}\spxextra{class in deepdrr.vol.volume}}

\begin{fulllineitems}
\phantomsection\label{\detokenize{deepdrr.vol:deepdrr.vol.volume.MetalVolume}}
\pysigstartsignatures
\pysiglinewithargsret{\sphinxbfcode{\sphinxupquote{class\DUrole{w,w}{  }}}\sphinxcode{\sphinxupquote{deepdrr.vol.volume.}}\sphinxbfcode{\sphinxupquote{MetalVolume}}}{\sphinxparam{\DUrole{n,n}{data}\DUrole{p,p}{:}\DUrole{w,w}{  }\DUrole{n,n}{ndarray}}\sphinxparamcomma \sphinxparam{\DUrole{n,n}{materials}\DUrole{p,p}{:}\DUrole{w,w}{  }\DUrole{n,n}{Dict\DUrole{p,p}{{[}}str\DUrole{p,p}{,}\DUrole{w,w}{  }ndarray\DUrole{p,p}{{]}}}}\sphinxparamcomma \sphinxparam{\DUrole{n,n}{anatomical\_from\_IJK}\DUrole{p,p}{:}\DUrole{w,w}{  }\DUrole{n,n}{{\hyperref[\detokenize{deepdrr.geo:deepdrr.geo.core.FrameTransform}]{\sphinxcrossref{FrameTransform}}}\DUrole{w,w}{  }\DUrole{p,p}{|}\DUrole{w,w}{  }None}\DUrole{w,w}{  }\DUrole{o,o}{=}\DUrole{w,w}{  }\DUrole{default_value}{None}}\sphinxparamcomma \sphinxparam{\DUrole{n,n}{world\_from\_anatomical}\DUrole{p,p}{:}\DUrole{w,w}{  }\DUrole{n,n}{{\hyperref[\detokenize{deepdrr.geo:deepdrr.geo.core.FrameTransform}]{\sphinxcrossref{FrameTransform}}}\DUrole{w,w}{  }\DUrole{p,p}{|}\DUrole{w,w}{  }None}\DUrole{w,w}{  }\DUrole{o,o}{=}\DUrole{w,w}{  }\DUrole{default_value}{None}}\sphinxparamcomma \sphinxparam{\DUrole{n,n}{anatomical\_coordinate\_system}\DUrole{p,p}{:}\DUrole{w,w}{  }\DUrole{n,n}{str\DUrole{w,w}{  }\DUrole{p,p}{|}\DUrole{w,w}{  }None}\DUrole{w,w}{  }\DUrole{o,o}{=}\DUrole{w,w}{  }\DUrole{default_value}{None}}\sphinxparamcomma \sphinxparam{\DUrole{n,n}{cache\_dir}\DUrole{p,p}{:}\DUrole{w,w}{  }\DUrole{n,n}{str\DUrole{w,w}{  }\DUrole{p,p}{|}\DUrole{w,w}{  }None}\DUrole{w,w}{  }\DUrole{o,o}{=}\DUrole{w,w}{  }\DUrole{default_value}{None}}\sphinxparamcomma \sphinxparam{\DUrole{n,n}{config}\DUrole{p,p}{:}\DUrole{w,w}{  }\DUrole{n,n}{Dict\DUrole{p,p}{{[}}str\DUrole{p,p}{,}\DUrole{w,w}{  }Any\DUrole{p,p}{{]}}}\DUrole{w,w}{  }\DUrole{o,o}{=}\DUrole{w,w}{  }\DUrole{default_value}{\{\}}}\sphinxparamcomma \sphinxparam{\DUrole{n,n}{anatomical\_from\_ijk}\DUrole{p,p}{:}\DUrole{w,w}{  }\DUrole{n,n}{{\hyperref[\detokenize{deepdrr.geo:deepdrr.geo.core.FrameTransform}]{\sphinxcrossref{FrameTransform}}}\DUrole{w,w}{  }\DUrole{p,p}{|}\DUrole{w,w}{  }None}\DUrole{w,w}{  }\DUrole{o,o}{=}\DUrole{w,w}{  }\DUrole{default_value}{None}}}{}
\pysigstopsignatures
\sphinxAtStartPar
Bases: {\hyperref[\detokenize{deepdrr.vol:deepdrr.vol.volume.Volume}]{\sphinxcrossref{\sphinxcode{\sphinxupquote{Volume}}}}}

\sphinxAtStartPar
Same as a volume, but with a different segmentation for the materials.
\index{anatomical\_coordinate\_system (deepdrr.vol.volume.MetalVolume attribute)@\spxentry{anatomical\_coordinate\_system}\spxextra{deepdrr.vol.volume.MetalVolume attribute}}

\begin{fulllineitems}
\phantomsection\label{\detokenize{deepdrr.vol:deepdrr.vol.volume.MetalVolume.anatomical_coordinate_system}}
\pysigstartsignatures
\pysigline{\sphinxbfcode{\sphinxupquote{anatomical\_coordinate\_system}}\sphinxbfcode{\sphinxupquote{\DUrole{p,p}{:}\DUrole{w,w}{  }str\DUrole{w,w}{  }\DUrole{p,p}{|}\DUrole{w,w}{  }None}}}
\pysigstopsignatures
\end{fulllineitems}

\index{anatomical\_from\_IJK (deepdrr.vol.volume.MetalVolume attribute)@\spxentry{anatomical\_from\_IJK}\spxextra{deepdrr.vol.volume.MetalVolume attribute}}

\begin{fulllineitems}
\phantomsection\label{\detokenize{deepdrr.vol:deepdrr.vol.volume.MetalVolume.anatomical_from_IJK}}
\pysigstartsignatures
\pysigline{\sphinxbfcode{\sphinxupquote{anatomical\_from\_IJK}}\sphinxbfcode{\sphinxupquote{\DUrole{p,p}{:}\DUrole{w,w}{  }{\hyperref[\detokenize{deepdrr.geo:deepdrr.geo.core.FrameTransform}]{\sphinxcrossref{FrameTransform}}}}}}
\pysigstopsignatures
\end{fulllineitems}

\index{data (deepdrr.vol.volume.MetalVolume attribute)@\spxentry{data}\spxextra{deepdrr.vol.volume.MetalVolume attribute}}

\begin{fulllineitems}
\phantomsection\label{\detokenize{deepdrr.vol:deepdrr.vol.volume.MetalVolume.data}}
\pysigstartsignatures
\pysigline{\sphinxbfcode{\sphinxupquote{data}}\sphinxbfcode{\sphinxupquote{\DUrole{p,p}{:}\DUrole{w,w}{  }ndarray}}}
\pysigstopsignatures
\end{fulllineitems}

\index{materials (deepdrr.vol.volume.MetalVolume attribute)@\spxentry{materials}\spxextra{deepdrr.vol.volume.MetalVolume attribute}}

\begin{fulllineitems}
\phantomsection\label{\detokenize{deepdrr.vol:deepdrr.vol.volume.MetalVolume.materials}}
\pysigstartsignatures
\pysigline{\sphinxbfcode{\sphinxupquote{materials}}\sphinxbfcode{\sphinxupquote{\DUrole{p,p}{:}\DUrole{w,w}{  }Dict\DUrole{p,p}{{[}}str\DUrole{p,p}{,}\DUrole{w,w}{  }ndarray\DUrole{p,p}{{]}}}}}
\pysigstopsignatures
\end{fulllineitems}

\index{world\_from\_anatomical (deepdrr.vol.volume.MetalVolume attribute)@\spxentry{world\_from\_anatomical}\spxextra{deepdrr.vol.volume.MetalVolume attribute}}

\begin{fulllineitems}
\phantomsection\label{\detokenize{deepdrr.vol:deepdrr.vol.volume.MetalVolume.world_from_anatomical}}
\pysigstartsignatures
\pysigline{\sphinxbfcode{\sphinxupquote{world\_from\_anatomical}}\sphinxbfcode{\sphinxupquote{\DUrole{p,p}{:}\DUrole{w,w}{  }{\hyperref[\detokenize{deepdrr.geo:deepdrr.geo.core.FrameTransform}]{\sphinxcrossref{FrameTransform}}}}}}
\pysigstopsignatures
\end{fulllineitems}


\end{fulllineitems}

\index{Volume (class in deepdrr.vol.volume)@\spxentry{Volume}\spxextra{class in deepdrr.vol.volume}}

\begin{fulllineitems}
\phantomsection\label{\detokenize{deepdrr.vol:deepdrr.vol.volume.Volume}}
\pysigstartsignatures
\pysiglinewithargsret{\sphinxbfcode{\sphinxupquote{class\DUrole{w,w}{  }}}\sphinxcode{\sphinxupquote{deepdrr.vol.volume.}}\sphinxbfcode{\sphinxupquote{Volume}}}{\sphinxparam{\DUrole{n,n}{data}\DUrole{p,p}{:}\DUrole{w,w}{  }\DUrole{n,n}{ndarray}}\sphinxparamcomma \sphinxparam{\DUrole{n,n}{materials}\DUrole{p,p}{:}\DUrole{w,w}{  }\DUrole{n,n}{Dict\DUrole{p,p}{{[}}str\DUrole{p,p}{,}\DUrole{w,w}{  }ndarray\DUrole{p,p}{{]}}}}\sphinxparamcomma \sphinxparam{\DUrole{n,n}{anatomical\_from\_IJK}\DUrole{p,p}{:}\DUrole{w,w}{  }\DUrole{n,n}{{\hyperref[\detokenize{deepdrr.geo:deepdrr.geo.core.FrameTransform}]{\sphinxcrossref{FrameTransform}}}\DUrole{w,w}{  }\DUrole{p,p}{|}\DUrole{w,w}{  }None}\DUrole{w,w}{  }\DUrole{o,o}{=}\DUrole{w,w}{  }\DUrole{default_value}{None}}\sphinxparamcomma \sphinxparam{\DUrole{n,n}{world\_from\_anatomical}\DUrole{p,p}{:}\DUrole{w,w}{  }\DUrole{n,n}{{\hyperref[\detokenize{deepdrr.geo:deepdrr.geo.core.FrameTransform}]{\sphinxcrossref{FrameTransform}}}\DUrole{w,w}{  }\DUrole{p,p}{|}\DUrole{w,w}{  }None}\DUrole{w,w}{  }\DUrole{o,o}{=}\DUrole{w,w}{  }\DUrole{default_value}{None}}\sphinxparamcomma \sphinxparam{\DUrole{n,n}{anatomical\_coordinate\_system}\DUrole{p,p}{:}\DUrole{w,w}{  }\DUrole{n,n}{str\DUrole{w,w}{  }\DUrole{p,p}{|}\DUrole{w,w}{  }None}\DUrole{w,w}{  }\DUrole{o,o}{=}\DUrole{w,w}{  }\DUrole{default_value}{None}}\sphinxparamcomma \sphinxparam{\DUrole{n,n}{cache\_dir}\DUrole{p,p}{:}\DUrole{w,w}{  }\DUrole{n,n}{str\DUrole{w,w}{  }\DUrole{p,p}{|}\DUrole{w,w}{  }None}\DUrole{w,w}{  }\DUrole{o,o}{=}\DUrole{w,w}{  }\DUrole{default_value}{None}}\sphinxparamcomma \sphinxparam{\DUrole{n,n}{config}\DUrole{p,p}{:}\DUrole{w,w}{  }\DUrole{n,n}{Dict\DUrole{p,p}{{[}}str\DUrole{p,p}{,}\DUrole{w,w}{  }Any\DUrole{p,p}{{]}}}\DUrole{w,w}{  }\DUrole{o,o}{=}\DUrole{w,w}{  }\DUrole{default_value}{\{\}}}\sphinxparamcomma \sphinxparam{\DUrole{n,n}{anatomical\_from\_ijk}\DUrole{p,p}{:}\DUrole{w,w}{  }\DUrole{n,n}{{\hyperref[\detokenize{deepdrr.geo:deepdrr.geo.core.FrameTransform}]{\sphinxcrossref{FrameTransform}}}\DUrole{w,w}{  }\DUrole{p,p}{|}\DUrole{w,w}{  }None}\DUrole{w,w}{  }\DUrole{o,o}{=}\DUrole{w,w}{  }\DUrole{default_value}{None}}}{}
\pysigstopsignatures
\sphinxAtStartPar
Bases: \sphinxcode{\sphinxupquote{object}}
\index{IJK\_from\_LPS (deepdrr.vol.volume.Volume property)@\spxentry{IJK\_from\_LPS}\spxextra{deepdrr.vol.volume.Volume property}}

\begin{fulllineitems}
\phantomsection\label{\detokenize{deepdrr.vol:deepdrr.vol.volume.Volume.IJK_from_LPS}}
\pysigstartsignatures
\pysigline{\sphinxbfcode{\sphinxupquote{property\DUrole{w,w}{  }}}\sphinxbfcode{\sphinxupquote{IJK\_from\_LPS}}\sphinxbfcode{\sphinxupquote{\DUrole{p,p}{:}\DUrole{w,w}{  }{\hyperref[\detokenize{deepdrr.geo:deepdrr.geo.core.FrameTransform}]{\sphinxcrossref{FrameTransform}}}}}}
\pysigstopsignatures
\end{fulllineitems}

\index{IJK\_from\_RAS (deepdrr.vol.volume.Volume property)@\spxentry{IJK\_from\_RAS}\spxextra{deepdrr.vol.volume.Volume property}}

\begin{fulllineitems}
\phantomsection\label{\detokenize{deepdrr.vol:deepdrr.vol.volume.Volume.IJK_from_RAS}}
\pysigstartsignatures
\pysigline{\sphinxbfcode{\sphinxupquote{property\DUrole{w,w}{  }}}\sphinxbfcode{\sphinxupquote{IJK\_from\_RAS}}}
\pysigstopsignatures
\end{fulllineitems}

\index{IJK\_from\_anatomical (deepdrr.vol.volume.Volume property)@\spxentry{IJK\_from\_anatomical}\spxextra{deepdrr.vol.volume.Volume property}}

\begin{fulllineitems}
\phantomsection\label{\detokenize{deepdrr.vol:deepdrr.vol.volume.Volume.IJK_from_anatomical}}
\pysigstartsignatures
\pysigline{\sphinxbfcode{\sphinxupquote{property\DUrole{w,w}{  }}}\sphinxbfcode{\sphinxupquote{IJK\_from\_anatomical}}}
\pysigstopsignatures
\end{fulllineitems}

\index{IJK\_from\_world (deepdrr.vol.volume.Volume property)@\spxentry{IJK\_from\_world}\spxextra{deepdrr.vol.volume.Volume property}}

\begin{fulllineitems}
\phantomsection\label{\detokenize{deepdrr.vol:deepdrr.vol.volume.Volume.IJK_from_world}}
\pysigstartsignatures
\pysigline{\sphinxbfcode{\sphinxupquote{property\DUrole{w,w}{  }}}\sphinxbfcode{\sphinxupquote{IJK\_from\_world}}\sphinxbfcode{\sphinxupquote{\DUrole{p,p}{:}\DUrole{w,w}{  }{\hyperref[\detokenize{deepdrr.geo:deepdrr.geo.core.FrameTransform}]{\sphinxcrossref{FrameTransform}}}}}}
\pysigstopsignatures
\end{fulllineitems}

\index{LPS\_from\_IJK (deepdrr.vol.volume.Volume property)@\spxentry{LPS\_from\_IJK}\spxextra{deepdrr.vol.volume.Volume property}}

\begin{fulllineitems}
\phantomsection\label{\detokenize{deepdrr.vol:deepdrr.vol.volume.Volume.LPS_from_IJK}}
\pysigstartsignatures
\pysigline{\sphinxbfcode{\sphinxupquote{property\DUrole{w,w}{  }}}\sphinxbfcode{\sphinxupquote{LPS\_from\_IJK}}\sphinxbfcode{\sphinxupquote{\DUrole{p,p}{:}\DUrole{w,w}{  }{\hyperref[\detokenize{deepdrr.geo:deepdrr.geo.core.FrameTransform}]{\sphinxcrossref{FrameTransform}}}}}}
\pysigstopsignatures
\sphinxAtStartPar
Get the LPS\_from\_IJK transform.

\end{fulllineitems}

\index{RAS\_from\_IJK (deepdrr.vol.volume.Volume property)@\spxentry{RAS\_from\_IJK}\spxextra{deepdrr.vol.volume.Volume property}}

\begin{fulllineitems}
\phantomsection\label{\detokenize{deepdrr.vol:deepdrr.vol.volume.Volume.RAS_from_IJK}}
\pysigstartsignatures
\pysigline{\sphinxbfcode{\sphinxupquote{property\DUrole{w,w}{  }}}\sphinxbfcode{\sphinxupquote{RAS\_from\_IJK}}}
\pysigstopsignatures
\sphinxAtStartPar
Get the RAS\_from\_IJK transform.

\end{fulllineitems}

\index{anatomical\_coordinate\_system (deepdrr.vol.volume.Volume attribute)@\spxentry{anatomical\_coordinate\_system}\spxextra{deepdrr.vol.volume.Volume attribute}}

\begin{fulllineitems}
\phantomsection\label{\detokenize{deepdrr.vol:deepdrr.vol.volume.Volume.anatomical_coordinate_system}}
\pysigstartsignatures
\pysigline{\sphinxbfcode{\sphinxupquote{anatomical\_coordinate\_system}}\sphinxbfcode{\sphinxupquote{\DUrole{p,p}{:}\DUrole{w,w}{  }str\DUrole{w,w}{  }\DUrole{p,p}{|}\DUrole{w,w}{  }None}}}
\pysigstopsignatures
\end{fulllineitems}

\index{anatomical\_from\_IJK (deepdrr.vol.volume.Volume attribute)@\spxentry{anatomical\_from\_IJK}\spxextra{deepdrr.vol.volume.Volume attribute}}

\begin{fulllineitems}
\phantomsection\label{\detokenize{deepdrr.vol:deepdrr.vol.volume.Volume.anatomical_from_IJK}}
\pysigstartsignatures
\pysigline{\sphinxbfcode{\sphinxupquote{anatomical\_from\_IJK}}\sphinxbfcode{\sphinxupquote{\DUrole{p,p}{:}\DUrole{w,w}{  }{\hyperref[\detokenize{deepdrr.geo:deepdrr.geo.core.FrameTransform}]{\sphinxcrossref{FrameTransform}}}}}}
\pysigstopsignatures
\end{fulllineitems}

\index{anatomical\_from\_ijk (deepdrr.vol.volume.Volume property)@\spxentry{anatomical\_from\_ijk}\spxextra{deepdrr.vol.volume.Volume property}}

\begin{fulllineitems}
\phantomsection\label{\detokenize{deepdrr.vol:deepdrr.vol.volume.Volume.anatomical_from_ijk}}
\pysigstartsignatures
\pysigline{\sphinxbfcode{\sphinxupquote{property\DUrole{w,w}{  }}}\sphinxbfcode{\sphinxupquote{anatomical\_from\_ijk}}\sphinxbfcode{\sphinxupquote{\DUrole{p,p}{:}\DUrole{w,w}{  }{\hyperref[\detokenize{deepdrr.geo:deepdrr.geo.core.FrameTransform}]{\sphinxcrossref{FrameTransform}}}}}}
\pysigstopsignatures
\end{fulllineitems}

\index{anatomical\_from\_world (deepdrr.vol.volume.Volume property)@\spxentry{anatomical\_from\_world}\spxextra{deepdrr.vol.volume.Volume property}}

\begin{fulllineitems}
\phantomsection\label{\detokenize{deepdrr.vol:deepdrr.vol.volume.Volume.anatomical_from_world}}
\pysigstartsignatures
\pysigline{\sphinxbfcode{\sphinxupquote{property\DUrole{w,w}{  }}}\sphinxbfcode{\sphinxupquote{anatomical\_from\_world}}}
\pysigstopsignatures
\end{fulllineitems}

\index{cache\_dir (deepdrr.vol.volume.Volume attribute)@\spxentry{cache\_dir}\spxextra{deepdrr.vol.volume.Volume attribute}}

\begin{fulllineitems}
\phantomsection\label{\detokenize{deepdrr.vol:deepdrr.vol.volume.Volume.cache_dir}}
\pysigstartsignatures
\pysigline{\sphinxbfcode{\sphinxupquote{cache\_dir}}\sphinxbfcode{\sphinxupquote{\DUrole{p,p}{:}\DUrole{w,w}{  }Path\DUrole{w,w}{  }\DUrole{p,p}{|}\DUrole{w,w}{  }None}}\sphinxbfcode{\sphinxupquote{\DUrole{w,w}{  }\DUrole{p,p}{=}\DUrole{w,w}{  }None}}}
\pysigstopsignatures
\end{fulllineitems}

\index{center\_in\_world (deepdrr.vol.volume.Volume property)@\spxentry{center\_in\_world}\spxextra{deepdrr.vol.volume.Volume property}}

\begin{fulllineitems}
\phantomsection\label{\detokenize{deepdrr.vol:deepdrr.vol.volume.Volume.center_in_world}}
\pysigstartsignatures
\pysigline{\sphinxbfcode{\sphinxupquote{property\DUrole{w,w}{  }}}\sphinxbfcode{\sphinxupquote{center\_in\_world}}\sphinxbfcode{\sphinxupquote{\DUrole{p,p}{:}\DUrole{w,w}{  }{\hyperref[\detokenize{deepdrr.geo:deepdrr.geo.core.Point3D}]{\sphinxcrossref{Point3D}}}}}}
\pysigstopsignatures
\sphinxAtStartPar
The center of the volume in world coorindates. Useful for debugging.

\end{fulllineitems}

\index{copy\_pose() (deepdrr.vol.volume.Volume method)@\spxentry{copy\_pose()}\spxextra{deepdrr.vol.volume.Volume method}}

\begin{fulllineitems}
\phantomsection\label{\detokenize{deepdrr.vol:deepdrr.vol.volume.Volume.copy_pose}}
\pysigstartsignatures
\pysiglinewithargsret{\sphinxbfcode{\sphinxupquote{copy\_pose}}}{\sphinxparam{\DUrole{n,n}{other}\DUrole{p,p}{:}\DUrole{w,w}{  }\DUrole{n,n}{{\hyperref[\detokenize{deepdrr.vol:deepdrr.vol.volume.Volume}]{\sphinxcrossref{Volume}}}}}}{{ $\rightarrow$ None}}
\pysigstopsignatures
\sphinxAtStartPar
Copy the pose of another volume.

\end{fulllineitems}

\index{crop() (deepdrr.vol.volume.Volume method)@\spxentry{crop()}\spxextra{deepdrr.vol.volume.Volume method}}

\begin{fulllineitems}
\phantomsection\label{\detokenize{deepdrr.vol:deepdrr.vol.volume.Volume.crop}}
\pysigstartsignatures
\pysiglinewithargsret{\sphinxbfcode{\sphinxupquote{crop}}}{\sphinxparam{\DUrole{n,n}{crop\_box}\DUrole{p,p}{:}\DUrole{w,w}{  }\DUrole{n,n}{Tuple\DUrole{p,p}{{[}}Tuple\DUrole{p,p}{{[}}float\DUrole{p,p}{,}\DUrole{w,w}{  }float\DUrole{p,p}{{]}}\DUrole{p,p}{,}\DUrole{w,w}{  }\DUrole{p,p}{...}\DUrole{p,p}{{]}}}}}{{ $\rightarrow$ {\hyperref[\detokenize{deepdrr.vol:deepdrr.vol.volume.Volume}]{\sphinxcrossref{Volume}}}}}
\pysigstopsignatures
\sphinxAtStartPar
Crop the volume to a given bounding box.
\begin{quote}\begin{description}
\sphinxlineitem{Parameters}
\sphinxAtStartPar
\sphinxstyleliteralstrong{\sphinxupquote{crop\_box}} (\sphinxstyleliteralemphasis{\sphinxupquote{Tuple}}\sphinxstyleliteralemphasis{\sphinxupquote{{[}}}\sphinxstyleliteralemphasis{\sphinxupquote{Tuple}}\sphinxstyleliteralemphasis{\sphinxupquote{{[}}}\sphinxstyleliteralemphasis{\sphinxupquote{float}}\sphinxstyleliteralemphasis{\sphinxupquote{, }}\sphinxstyleliteralemphasis{\sphinxupquote{float}}\sphinxstyleliteralemphasis{\sphinxupquote{{]}}}\sphinxstyleliteralemphasis{\sphinxupquote{, }}\sphinxstyleliteralemphasis{\sphinxupquote{...}}\sphinxstyleliteralemphasis{\sphinxupquote{{]}}}) \textendash{} The bounding box to crop to, in IJK.

\sphinxlineitem{Returns}
\sphinxAtStartPar
The cropped volume.

\sphinxlineitem{Return type}
\sphinxAtStartPar
{\hyperref[\detokenize{deepdrr.vol:deepdrr.vol.volume.Volume}]{\sphinxcrossref{Volume}}}

\end{description}\end{quote}

\end{fulllineitems}

\index{data (deepdrr.vol.volume.Volume attribute)@\spxentry{data}\spxextra{deepdrr.vol.volume.Volume attribute}}

\begin{fulllineitems}
\phantomsection\label{\detokenize{deepdrr.vol:deepdrr.vol.volume.Volume.data}}
\pysigstartsignatures
\pysigline{\sphinxbfcode{\sphinxupquote{data}}\sphinxbfcode{\sphinxupquote{\DUrole{p,p}{:}\DUrole{w,w}{  }ndarray}}}
\pysigstopsignatures
\end{fulllineitems}

\index{facedown() (deepdrr.vol.volume.Volume method)@\spxentry{facedown()}\spxextra{deepdrr.vol.volume.Volume method}}

\begin{fulllineitems}
\phantomsection\label{\detokenize{deepdrr.vol:deepdrr.vol.volume.Volume.facedown}}
\pysigstartsignatures
\pysiglinewithargsret{\sphinxbfcode{\sphinxupquote{facedown}}}{}{}
\pysigstopsignatures
\sphinxAtStartPar
Turns the volume to be face down.

\sphinxAtStartPar
This aligns the patient so that, in world space,
the posterior side is toward +Z, inferior is toward +X,
and right is toward +Y.
\begin{quote}\begin{description}
\sphinxlineitem{Raises}
\sphinxAtStartPar
\sphinxstyleliteralstrong{\sphinxupquote{NotImplementedError}} \textendash{} If the anatomical coordinate system is not “RAS”.

\end{description}\end{quote}

\end{fulllineitems}

\index{faceup() (deepdrr.vol.volume.Volume method)@\spxentry{faceup()}\spxextra{deepdrr.vol.volume.Volume method}}

\begin{fulllineitems}
\phantomsection\label{\detokenize{deepdrr.vol:deepdrr.vol.volume.Volume.faceup}}
\pysigstartsignatures
\pysiglinewithargsret{\sphinxbfcode{\sphinxupquote{faceup}}}{}{}
\pysigstopsignatures
\sphinxAtStartPar
Turns the volume to be face up.

\sphinxAtStartPar
This aligns the patient so that, in world space,
the anterior side is toward +Z, inferior is toward +X,
and left is toward +Y.
\begin{quote}\begin{description}
\sphinxlineitem{Raises}
\sphinxAtStartPar
\sphinxstyleliteralstrong{\sphinxupquote{NotImplementedError}} \textendash{} If the anatomical coordinate system is not “RAS”.

\end{description}\end{quote}

\end{fulllineitems}

\index{from\_dicom() (deepdrr.vol.volume.Volume class method)@\spxentry{from\_dicom()}\spxextra{deepdrr.vol.volume.Volume class method}}

\begin{fulllineitems}
\phantomsection\label{\detokenize{deepdrr.vol:deepdrr.vol.volume.Volume.from_dicom}}
\pysigstartsignatures
\pysiglinewithargsret{\sphinxbfcode{\sphinxupquote{classmethod\DUrole{w,w}{  }}}\sphinxbfcode{\sphinxupquote{from\_dicom}}}{\sphinxparam{\DUrole{n,n}{path}\DUrole{p,p}{:}\DUrole{w,w}{  }\DUrole{n,n}{Path}}\sphinxparamcomma \sphinxparam{\DUrole{n,n}{use\_thresholding}\DUrole{p,p}{:}\DUrole{w,w}{  }\DUrole{n,n}{bool}\DUrole{w,w}{  }\DUrole{o,o}{=}\DUrole{w,w}{  }\DUrole{default_value}{True}}\sphinxparamcomma \sphinxparam{\DUrole{n,n}{world\_from\_anatomical}\DUrole{p,p}{:}\DUrole{w,w}{  }\DUrole{n,n}{{\hyperref[\detokenize{deepdrr.geo:deepdrr.geo.core.FrameTransform}]{\sphinxcrossref{FrameTransform}}}\DUrole{w,w}{  }\DUrole{p,p}{|}\DUrole{w,w}{  }None}\DUrole{w,w}{  }\DUrole{o,o}{=}\DUrole{w,w}{  }\DUrole{default_value}{None}}\sphinxparamcomma \sphinxparam{\DUrole{n,n}{use\_cached}\DUrole{p,p}{:}\DUrole{w,w}{  }\DUrole{n,n}{bool}\DUrole{w,w}{  }\DUrole{o,o}{=}\DUrole{w,w}{  }\DUrole{default_value}{True}}\sphinxparamcomma \sphinxparam{\DUrole{n,n}{cache\_dir}\DUrole{p,p}{:}\DUrole{w,w}{  }\DUrole{n,n}{Path\DUrole{w,w}{  }\DUrole{p,p}{|}\DUrole{w,w}{  }None}\DUrole{w,w}{  }\DUrole{o,o}{=}\DUrole{w,w}{  }\DUrole{default_value}{None}}\sphinxparamcomma \sphinxparam{\DUrole{o,o}{**}\DUrole{n,n}{kwargs}}}{}
\pysigstopsignatures
\sphinxAtStartPar
load a volume from a dicom file and compute the anatomical\_from\_ijk transform from metadata
\sphinxurl{https://www.slicer.org/wiki/Coordinate\_systems}
:param path: path\sphinxhyphen{}like to a multi\sphinxhyphen{}frame dicom file. (Currently only Multi\sphinxhyphen{}Frame from Siemens supported)
:param use\_thresholding: segment the materials using thresholding (faster but less accurate). Defaults to True.
:type use\_thresholding: bool, optional
:param world\_from\_anatomical: position the volume in world space. If None, uses identity. Defaults to None.
:type world\_from\_anatomical: Optional{[}geo.FrameTransform{]}, optional
:param use\_cached: {[}description{]}. Use a cached segmentation if available. Defaults to True.
:type use\_cached: bool, optional
:param cache\_dir: Where to load/save the cached segmentation. If None, use the parent dir of \sphinxtitleref{path}. Defaults to None.
:type cache\_dir: Optional{[}Path{]}, optional
\begin{quote}\begin{description}
\sphinxlineitem{Returns}
\sphinxAtStartPar
an instance of a deepdrr volume

\sphinxlineitem{Return type}
\sphinxAtStartPar
{\hyperref[\detokenize{deepdrr.vol:deepdrr.vol.volume.Volume}]{\sphinxcrossref{Volume}}}

\end{description}\end{quote}

\end{fulllineitems}

\index{from\_hu() (deepdrr.vol.volume.Volume class method)@\spxentry{from\_hu()}\spxextra{deepdrr.vol.volume.Volume class method}}

\begin{fulllineitems}
\phantomsection\label{\detokenize{deepdrr.vol:deepdrr.vol.volume.Volume.from_hu}}
\pysigstartsignatures
\pysiglinewithargsret{\sphinxbfcode{\sphinxupquote{classmethod\DUrole{w,w}{  }}}\sphinxbfcode{\sphinxupquote{from\_hu}}}{\sphinxparam{\DUrole{n,n}{hu\_values}\DUrole{p,p}{:}\DUrole{w,w}{  }\DUrole{n,n}{ndarray}}\sphinxparamcomma \sphinxparam{\DUrole{n,n}{origin}\DUrole{p,p}{:}\DUrole{w,w}{  }\DUrole{n,n}{{\hyperref[\detokenize{deepdrr.geo:deepdrr.geo.core.Point3D}]{\sphinxcrossref{Point3D}}}}}\sphinxparamcomma \sphinxparam{\DUrole{n,n}{use\_thresholding}\DUrole{p,p}{:}\DUrole{w,w}{  }\DUrole{n,n}{bool}\DUrole{w,w}{  }\DUrole{o,o}{=}\DUrole{w,w}{  }\DUrole{default_value}{True}}\sphinxparamcomma \sphinxparam{\DUrole{n,n}{spacing}\DUrole{p,p}{:}\DUrole{w,w}{  }\DUrole{n,n}{{\hyperref[\detokenize{deepdrr.geo:deepdrr.geo.core.Vector3D}]{\sphinxcrossref{Vector3D}}}\DUrole{w,w}{  }\DUrole{p,p}{|}\DUrole{w,w}{  }None}\DUrole{w,w}{  }\DUrole{o,o}{=}\DUrole{w,w}{  }\DUrole{default_value}{(1, 1, 1)}}\sphinxparamcomma \sphinxparam{\DUrole{n,n}{anatomical\_coordinate\_system}\DUrole{p,p}{:}\DUrole{w,w}{  }\DUrole{n,n}{str\DUrole{w,w}{  }\DUrole{p,p}{|}\DUrole{w,w}{  }None}\DUrole{w,w}{  }\DUrole{o,o}{=}\DUrole{w,w}{  }\DUrole{default_value}{None}}\sphinxparamcomma \sphinxparam{\DUrole{n,n}{world\_from\_anatomical}\DUrole{p,p}{:}\DUrole{w,w}{  }\DUrole{n,n}{{\hyperref[\detokenize{deepdrr.geo:deepdrr.geo.core.FrameTransform}]{\sphinxcrossref{FrameTransform}}}\DUrole{w,w}{  }\DUrole{p,p}{|}\DUrole{w,w}{  }None}\DUrole{w,w}{  }\DUrole{o,o}{=}\DUrole{w,w}{  }\DUrole{default_value}{None}}\sphinxparamcomma \sphinxparam{\DUrole{o,o}{**}\DUrole{n,n}{kwargs}}}{{ $\rightarrow$ None}}
\pysigstopsignatures
\end{fulllineitems}

\index{from\_nifti() (deepdrr.vol.volume.Volume class method)@\spxentry{from\_nifti()}\spxextra{deepdrr.vol.volume.Volume class method}}

\begin{fulllineitems}
\phantomsection\label{\detokenize{deepdrr.vol:deepdrr.vol.volume.Volume.from_nifti}}
\pysigstartsignatures
\pysiglinewithargsret{\sphinxbfcode{\sphinxupquote{classmethod\DUrole{w,w}{  }}}\sphinxbfcode{\sphinxupquote{from\_nifti}}}{\sphinxparam{\DUrole{n,n}{path}\DUrole{p,p}{:}\DUrole{w,w}{  }\DUrole{n,n}{Path}}\sphinxparamcomma \sphinxparam{\DUrole{n,n}{world\_from\_anatomical}\DUrole{p,p}{:}\DUrole{w,w}{  }\DUrole{n,n}{{\hyperref[\detokenize{deepdrr.geo:deepdrr.geo.core.FrameTransform}]{\sphinxcrossref{FrameTransform}}}\DUrole{w,w}{  }\DUrole{p,p}{|}\DUrole{w,w}{  }None}\DUrole{w,w}{  }\DUrole{o,o}{=}\DUrole{w,w}{  }\DUrole{default_value}{None}}\sphinxparamcomma \sphinxparam{\DUrole{n,n}{use\_thresholding}\DUrole{p,p}{:}\DUrole{w,w}{  }\DUrole{n,n}{bool}\DUrole{w,w}{  }\DUrole{o,o}{=}\DUrole{w,w}{  }\DUrole{default_value}{True}}\sphinxparamcomma \sphinxparam{\DUrole{n,n}{use\_cached}\DUrole{p,p}{:}\DUrole{w,w}{  }\DUrole{n,n}{bool}\DUrole{w,w}{  }\DUrole{o,o}{=}\DUrole{w,w}{  }\DUrole{default_value}{True}}\sphinxparamcomma \sphinxparam{\DUrole{n,n}{save\_cache}\DUrole{p,p}{:}\DUrole{w,w}{  }\DUrole{n,n}{bool}\DUrole{w,w}{  }\DUrole{o,o}{=}\DUrole{w,w}{  }\DUrole{default_value}{False}}\sphinxparamcomma \sphinxparam{\DUrole{n,n}{cache\_dir}\DUrole{p,p}{:}\DUrole{w,w}{  }\DUrole{n,n}{Path\DUrole{w,w}{  }\DUrole{p,p}{|}\DUrole{w,w}{  }None}\DUrole{w,w}{  }\DUrole{o,o}{=}\DUrole{w,w}{  }\DUrole{default_value}{None}}\sphinxparamcomma \sphinxparam{\DUrole{n,n}{materials}\DUrole{p,p}{:}\DUrole{w,w}{  }\DUrole{n,n}{Dict\DUrole{p,p}{{[}}str\DUrole{p,p}{,}\DUrole{w,w}{  }ndarray\DUrole{p,p}{{]}}\DUrole{w,w}{  }\DUrole{p,p}{|}\DUrole{w,w}{  }None}\DUrole{w,w}{  }\DUrole{o,o}{=}\DUrole{w,w}{  }\DUrole{default_value}{None}}\sphinxparamcomma \sphinxparam{\DUrole{n,n}{segmentation}\DUrole{p,p}{:}\DUrole{w,w}{  }\DUrole{n,n}{bool}\DUrole{w,w}{  }\DUrole{o,o}{=}\DUrole{w,w}{  }\DUrole{default_value}{False}}\sphinxparamcomma \sphinxparam{\DUrole{n,n}{label}\DUrole{p,p}{:}\DUrole{w,w}{  }\DUrole{n,n}{None\DUrole{w,w}{  }\DUrole{p,p}{|}\DUrole{w,w}{  }int\DUrole{w,w}{  }\DUrole{p,p}{|}\DUrole{w,w}{  }List\DUrole{p,p}{{[}}int\DUrole{p,p}{{]}}}\DUrole{w,w}{  }\DUrole{o,o}{=}\DUrole{w,w}{  }\DUrole{default_value}{None}}\sphinxparamcomma \sphinxparam{\DUrole{n,n}{density\_kwargs}\DUrole{p,p}{:}\DUrole{w,w}{  }\DUrole{n,n}{dict}\DUrole{w,w}{  }\DUrole{o,o}{=}\DUrole{w,w}{  }\DUrole{default_value}{\{\}}}\sphinxparamcomma \sphinxparam{\DUrole{o,o}{**}\DUrole{n,n}{kwargs}}}{}
\pysigstopsignatures
\sphinxAtStartPar
Load a volume from NiFti file.
\begin{quote}\begin{description}
\sphinxlineitem{Parameters}\begin{itemize}
\item {} 
\sphinxAtStartPar
\sphinxstyleliteralstrong{\sphinxupquote{path}} (\sphinxstyleliteralemphasis{\sphinxupquote{Path}}) \textendash{} path to the .nii.gz file.

\item {} 
\sphinxAtStartPar
\sphinxstyleliteralstrong{\sphinxupquote{use\_thresholding}} (\sphinxstyleliteralemphasis{\sphinxupquote{bool}}\sphinxstyleliteralemphasis{\sphinxupquote{, }}\sphinxstyleliteralemphasis{\sphinxupquote{optional}}) \textendash{} segment the materials using thresholding (faster but less accurate). Defaults to True.

\item {} 
\sphinxAtStartPar
\sphinxstyleliteralstrong{\sphinxupquote{world\_from\_anatomical}} (\sphinxstyleliteralemphasis{\sphinxupquote{Optional}}\sphinxstyleliteralemphasis{\sphinxupquote{{[}}}{\hyperref[\detokenize{deepdrr.geo:deepdrr.geo.FrameTransform}]{\sphinxcrossref{\sphinxstyleliteralemphasis{\sphinxupquote{geo.FrameTransform}}}}}\sphinxstyleliteralemphasis{\sphinxupquote{{]}}}\sphinxstyleliteralemphasis{\sphinxupquote{, }}\sphinxstyleliteralemphasis{\sphinxupquote{optional}}) \textendash{} position the volume in world space. If None, uses identity. Defaults to None.

\item {} 
\sphinxAtStartPar
\sphinxstyleliteralstrong{\sphinxupquote{use\_cached}} (\sphinxstyleliteralemphasis{\sphinxupquote{bool}}\sphinxstyleliteralemphasis{\sphinxupquote{, }}\sphinxstyleliteralemphasis{\sphinxupquote{optional}}) \textendash{} Use a cached segmentation if available. Defaults to True.

\item {} 
\sphinxAtStartPar
\sphinxstyleliteralstrong{\sphinxupquote{cache\_dir}} (\sphinxstyleliteralemphasis{\sphinxupquote{Optional}}\sphinxstyleliteralemphasis{\sphinxupquote{{[}}}\sphinxstyleliteralemphasis{\sphinxupquote{Path}}\sphinxstyleliteralemphasis{\sphinxupquote{{]}}}\sphinxstyleliteralemphasis{\sphinxupquote{, }}\sphinxstyleliteralemphasis{\sphinxupquote{optional}}) \textendash{} Where to load/save the cached segmentation. If None, use a “cache” directory
in the same location as the nifti file. Defaults to None.

\item {} 
\sphinxAtStartPar
\sphinxstyleliteralstrong{\sphinxupquote{materials}} \textendash{} Optional material segmentation, as a dictionary mapping material name to binary segmentation.
If not provided, materials are segmented from the CT. Defaults to None.
Can also provide a dictionary mapping material names to Nifti files containing the segmentations.

\item {} 
\sphinxAtStartPar
\sphinxstyleliteralstrong{\sphinxupquote{segmentation}} (\sphinxstyleliteralemphasis{\sphinxupquote{bool}}\sphinxstyleliteralemphasis{\sphinxupquote{, }}\sphinxstyleliteralemphasis{\sphinxupquote{optional}}\sphinxstyleliteralemphasis{\sphinxupquote{) }}\sphinxstyleliteralemphasis{\sphinxupquote{If the file is a segmentation file}}\sphinxstyleliteralemphasis{\sphinxupquote{, }}\sphinxstyleliteralemphasis{\sphinxupquote{then its "materials" correspond to a high density material}}\sphinxstyleliteralemphasis{\sphinxupquote{ (}}\sphinxstyleliteralemphasis{\sphinxupquote{bone}}) \textendash{} where the values are \textgreater{}0. Defaults to false. Overrides provided materials.

\item {} 
\sphinxAtStartPar
\sphinxstyleliteralstrong{\sphinxupquote{label}} \textendash{} which labels to treat as solid. If None, then all nonzero labels are treated as solid. Defaults to None.

\item {} 
\sphinxAtStartPar
\sphinxstyleliteralstrong{\sphinxupquote{density\_kwargs}} \textendash{} Additional kwargs passed to convert\_hounsfield\_to\_density.

\end{itemize}

\sphinxlineitem{Returns}
\sphinxAtStartPar
A new volume object.

\sphinxlineitem{Return type}
\sphinxAtStartPar
{\hyperref[\detokenize{deepdrr.vol:deepdrr.vol.volume.Volume}]{\sphinxcrossref{Volume}}}

\end{description}\end{quote}

\end{fulllineitems}

\index{from\_nrrd() (deepdrr.vol.volume.Volume class method)@\spxentry{from\_nrrd()}\spxextra{deepdrr.vol.volume.Volume class method}}

\begin{fulllineitems}
\phantomsection\label{\detokenize{deepdrr.vol:deepdrr.vol.volume.Volume.from_nrrd}}
\pysigstartsignatures
\pysiglinewithargsret{\sphinxbfcode{\sphinxupquote{classmethod\DUrole{w,w}{  }}}\sphinxbfcode{\sphinxupquote{from\_nrrd}}}{\sphinxparam{\DUrole{n,n}{path}\DUrole{p,p}{:}\DUrole{w,w}{  }\DUrole{n,n}{str}}\sphinxparamcomma \sphinxparam{\DUrole{n,n}{world\_from\_anatomical}\DUrole{p,p}{:}\DUrole{w,w}{  }\DUrole{n,n}{{\hyperref[\detokenize{deepdrr.geo:deepdrr.geo.core.FrameTransform}]{\sphinxcrossref{FrameTransform}}}\DUrole{w,w}{  }\DUrole{p,p}{|}\DUrole{w,w}{  }None}\DUrole{w,w}{  }\DUrole{o,o}{=}\DUrole{w,w}{  }\DUrole{default_value}{None}}\sphinxparamcomma \sphinxparam{\DUrole{n,n}{use\_thresholding}\DUrole{p,p}{:}\DUrole{w,w}{  }\DUrole{n,n}{bool}\DUrole{w,w}{  }\DUrole{o,o}{=}\DUrole{w,w}{  }\DUrole{default_value}{True}}\sphinxparamcomma \sphinxparam{\DUrole{n,n}{use\_cached}\DUrole{p,p}{:}\DUrole{w,w}{  }\DUrole{n,n}{bool}\DUrole{w,w}{  }\DUrole{o,o}{=}\DUrole{w,w}{  }\DUrole{default_value}{True}}\sphinxparamcomma \sphinxparam{\DUrole{n,n}{cache\_dir}\DUrole{p,p}{:}\DUrole{w,w}{  }\DUrole{n,n}{Path\DUrole{w,w}{  }\DUrole{p,p}{|}\DUrole{w,w}{  }None}\DUrole{w,w}{  }\DUrole{o,o}{=}\DUrole{w,w}{  }\DUrole{default_value}{None}}\sphinxparamcomma \sphinxparam{\DUrole{o,o}{**}\DUrole{n,n}{kwargs}}}{}
\pysigstopsignatures
\sphinxAtStartPar
Load a volume from a nrrd file.
\begin{quote}\begin{description}
\sphinxlineitem{Parameters}\begin{itemize}
\item {} 
\sphinxAtStartPar
\sphinxstyleliteralstrong{\sphinxupquote{path}} (\sphinxstyleliteralemphasis{\sphinxupquote{str}}) \textendash{} path to the file.

\item {} 
\sphinxAtStartPar
\sphinxstyleliteralstrong{\sphinxupquote{use\_thresholding}} (\sphinxstyleliteralemphasis{\sphinxupquote{bool}}\sphinxstyleliteralemphasis{\sphinxupquote{, }}\sphinxstyleliteralemphasis{\sphinxupquote{optional}}) \textendash{} segment the materials using thresholding (faster but less accurate). Defaults to True.

\item {} 
\sphinxAtStartPar
\sphinxstyleliteralstrong{\sphinxupquote{world\_from\_anatomical}} (\sphinxstyleliteralemphasis{\sphinxupquote{Optional}}\sphinxstyleliteralemphasis{\sphinxupquote{{[}}}{\hyperref[\detokenize{deepdrr.geo:deepdrr.geo.FrameTransform}]{\sphinxcrossref{\sphinxstyleliteralemphasis{\sphinxupquote{geo.FrameTransform}}}}}\sphinxstyleliteralemphasis{\sphinxupquote{{]}}}\sphinxstyleliteralemphasis{\sphinxupquote{, }}\sphinxstyleliteralemphasis{\sphinxupquote{optional}}) \textendash{} position the volume in world space. If None, uses identity. Defaults to None.

\item {} 
\sphinxAtStartPar
\sphinxstyleliteralstrong{\sphinxupquote{use\_cached}} (\sphinxstyleliteralemphasis{\sphinxupquote{bool}}\sphinxstyleliteralemphasis{\sphinxupquote{, }}\sphinxstyleliteralemphasis{\sphinxupquote{optional}}) \textendash{} Use a cached segmentation if available. Defaults to True.

\item {} 
\sphinxAtStartPar
\sphinxstyleliteralstrong{\sphinxupquote{cache\_dir}} (\sphinxstyleliteralemphasis{\sphinxupquote{Optional}}\sphinxstyleliteralemphasis{\sphinxupquote{{[}}}\sphinxstyleliteralemphasis{\sphinxupquote{Path}}\sphinxstyleliteralemphasis{\sphinxupquote{{]}}}\sphinxstyleliteralemphasis{\sphinxupquote{, }}\sphinxstyleliteralemphasis{\sphinxupquote{optional}}) \textendash{} Where to load/save the cached segmentation. If None, use the parent dir of \sphinxtitleref{path}. Defaults to None.

\end{itemize}

\sphinxlineitem{Returns}
\sphinxAtStartPar
A volume formed from the NRRD.

\sphinxlineitem{Return type}
\sphinxAtStartPar
{\hyperref[\detokenize{deepdrr.vol:deepdrr.vol.volume.Volume}]{\sphinxcrossref{Volume}}}

\end{description}\end{quote}

\end{fulllineitems}

\index{from\_parameters() (deepdrr.vol.volume.Volume class method)@\spxentry{from\_parameters()}\spxextra{deepdrr.vol.volume.Volume class method}}

\begin{fulllineitems}
\phantomsection\label{\detokenize{deepdrr.vol:deepdrr.vol.volume.Volume.from_parameters}}
\pysigstartsignatures
\pysiglinewithargsret{\sphinxbfcode{\sphinxupquote{classmethod\DUrole{w,w}{  }}}\sphinxbfcode{\sphinxupquote{from\_parameters}}}{\sphinxparam{\DUrole{n,n}{data}\DUrole{p,p}{:}\DUrole{w,w}{  }\DUrole{n,n}{ndarray}}\sphinxparamcomma \sphinxparam{\DUrole{n,n}{materials}\DUrole{p,p}{:}\DUrole{w,w}{  }\DUrole{n,n}{Dict\DUrole{p,p}{{[}}str\DUrole{p,p}{,}\DUrole{w,w}{  }ndarray\DUrole{p,p}{{]}}}}\sphinxparamcomma \sphinxparam{\DUrole{n,n}{origin}\DUrole{p,p}{:}\DUrole{w,w}{  }\DUrole{n,n}{{\hyperref[\detokenize{deepdrr.geo:deepdrr.geo.core.Point3D}]{\sphinxcrossref{Point3D}}}}}\sphinxparamcomma \sphinxparam{\DUrole{n,n}{spacing}\DUrole{p,p}{:}\DUrole{w,w}{  }\DUrole{n,n}{{\hyperref[\detokenize{deepdrr.geo:deepdrr.geo.core.Vector3D}]{\sphinxcrossref{Vector3D}}}\DUrole{w,w}{  }\DUrole{p,p}{|}\DUrole{w,w}{  }None}\DUrole{w,w}{  }\DUrole{o,o}{=}\DUrole{w,w}{  }\DUrole{default_value}{{[}1, 1, 1{]}}}\sphinxparamcomma \sphinxparam{\DUrole{n,n}{anatomical\_coordinate\_system}\DUrole{p,p}{:}\DUrole{w,w}{  }\DUrole{n,n}{str\DUrole{w,w}{  }\DUrole{p,p}{|}\DUrole{w,w}{  }None}\DUrole{w,w}{  }\DUrole{o,o}{=}\DUrole{w,w}{  }\DUrole{default_value}{None}}\sphinxparamcomma \sphinxparam{\DUrole{n,n}{world\_from\_anatomical}\DUrole{p,p}{:}\DUrole{w,w}{  }\DUrole{n,n}{{\hyperref[\detokenize{deepdrr.geo:deepdrr.geo.core.FrameTransform}]{\sphinxcrossref{FrameTransform}}}\DUrole{w,w}{  }\DUrole{p,p}{|}\DUrole{w,w}{  }None}\DUrole{w,w}{  }\DUrole{o,o}{=}\DUrole{w,w}{  }\DUrole{default_value}{None}}\sphinxparamcomma \sphinxparam{\DUrole{o,o}{**}\DUrole{n,n}{kwargs}}}{}
\pysigstopsignatures
\sphinxAtStartPar
Create a volume object with a segmentation of the materials, from parameters.

\sphinxAtStartPar
Note that the anatomical coordinate system is not the world coordinate system (which is cartesian).

\sphinxAtStartPar
Suggested anatomical coordinate space units is millimeters.
A helpful introduction to the geometry is can be found {[}here{]}(\sphinxurl{https://www.slicer.org/wiki/Coordinate\_systems}).
\begin{quote}\begin{description}
\sphinxlineitem{Parameters}\begin{itemize}
\item {} 
\sphinxAtStartPar
\sphinxstyleliteralstrong{\sphinxupquote{volume}} (\sphinxstyleliteralemphasis{\sphinxupquote{np.ndarray}}) \textendash{} the volume density data.

\item {} 
\sphinxAtStartPar
\sphinxstyleliteralstrong{\sphinxupquote{materials}} (\sphinxstyleliteralemphasis{\sphinxupquote{Dict}}\sphinxstyleliteralemphasis{\sphinxupquote{{[}}}\sphinxstyleliteralemphasis{\sphinxupquote{str}}\sphinxstyleliteralemphasis{\sphinxupquote{, }}\sphinxstyleliteralemphasis{\sphinxupquote{np.ndarray}}\sphinxstyleliteralemphasis{\sphinxupquote{{]}}}) \textendash{} mapping from material names to binary segmentation of that material.

\item {} 
\sphinxAtStartPar
\sphinxstyleliteralstrong{\sphinxupquote{origin}} ({\hyperref[\detokenize{deepdrr.geo:deepdrr.geo.core.Point3D}]{\sphinxcrossref{\sphinxstyleliteralemphasis{\sphinxupquote{Point3D}}}}}) \textendash{} Location of the volume’s origin in the anatomical coordinate system.

\item {} 
\sphinxAtStartPar
\sphinxstyleliteralstrong{\sphinxupquote{spacing}} (\sphinxstyleliteralemphasis{\sphinxupquote{Tuple}}\sphinxstyleliteralemphasis{\sphinxupquote{{[}}}\sphinxstyleliteralemphasis{\sphinxupquote{float}}\sphinxstyleliteralemphasis{\sphinxupquote{, }}\sphinxstyleliteralemphasis{\sphinxupquote{float}}\sphinxstyleliteralemphasis{\sphinxupquote{, }}\sphinxstyleliteralemphasis{\sphinxupquote{float}}\sphinxstyleliteralemphasis{\sphinxupquote{{]}}}\sphinxstyleliteralemphasis{\sphinxupquote{, }}\sphinxstyleliteralemphasis{\sphinxupquote{optional}}) \textendash{} Spacing of the volume in the anatomical coordinate system. Defaults to (1, 1, 1).

\item {} 
\sphinxAtStartPar
\sphinxstyleliteralstrong{\sphinxupquote{anatomical\_coordinate\_system}} (\sphinxstyleliteralemphasis{\sphinxupquote{Optional}}\sphinxstyleliteralemphasis{\sphinxupquote{{[}}}\sphinxstyleliteralemphasis{\sphinxupquote{str}}\sphinxstyleliteralemphasis{\sphinxupquote{{]}}}) \textendash{} anatomical coordinate system convention, either “RAS” or “LPS”. Defaults to None.

\item {} 
\sphinxAtStartPar
\sphinxstyleliteralstrong{\sphinxupquote{world\_from\_anatomical}} ({\hyperref[\detokenize{deepdrr.geo:deepdrr.geo.core.FrameTransform}]{\sphinxcrossref{\sphinxstyleliteralemphasis{\sphinxupquote{FrameTransform}}}}}\sphinxstyleliteralemphasis{\sphinxupquote{, }}\sphinxstyleliteralemphasis{\sphinxupquote{optional}}) \textendash{} Optional transformation from anatomical to world coordinates.
If None, then identity is used. Defaults to None.

\end{itemize}

\end{description}\end{quote}

\end{fulllineitems}

\index{get\_bbox\_IJK() (deepdrr.vol.volume.Volume method)@\spxentry{get\_bbox\_IJK()}\spxextra{deepdrr.vol.volume.Volume method}}

\begin{fulllineitems}
\phantomsection\label{\detokenize{deepdrr.vol:deepdrr.vol.volume.Volume.get_bbox_IJK}}
\pysigstartsignatures
\pysiglinewithargsret{\sphinxbfcode{\sphinxupquote{get\_bbox\_IJK}}}{}{{ $\rightarrow$ ndarray\DUrole{w,w}{  }\DUrole{p,p}{|}\DUrole{w,w}{  }None}}
\pysigstopsignatures
\sphinxAtStartPar
Get the bounding box of the materials in IJK.
\begin{quote}\begin{description}
\sphinxlineitem{Returns}
\sphinxAtStartPar
The bounding box as a {[}3, 2{]} array.
None, if the volume is empty.

\sphinxlineitem{Return type}
\sphinxAtStartPar
np.ndarray

\end{description}\end{quote}

\end{fulllineitems}

\index{get\_bounding\_box\_in\_world() (deepdrr.vol.volume.Volume method)@\spxentry{get\_bounding\_box\_in\_world()}\spxextra{deepdrr.vol.volume.Volume method}}

\begin{fulllineitems}
\phantomsection\label{\detokenize{deepdrr.vol:deepdrr.vol.volume.Volume.get_bounding_box_in_world}}
\pysigstartsignatures
\pysiglinewithargsret{\sphinxbfcode{\sphinxupquote{get\_bounding\_box\_in\_world}}}{}{{ $\rightarrow$ Tuple\DUrole{p,p}{{[}}{\hyperref[\detokenize{deepdrr.geo:deepdrr.geo.core.Point3D}]{\sphinxcrossref{Point3D}}}\DUrole{p,p}{,}\DUrole{w,w}{  }{\hyperref[\detokenize{deepdrr.geo:deepdrr.geo.core.Point3D}]{\sphinxcrossref{Point3D}}}\DUrole{p,p}{{]}}}}
\pysigstopsignatures
\sphinxAtStartPar
Get the corners of a bounding box enclosing the volume in world coordinates.

\sphinxAtStartPar
Assumes cell\sphinxhyphen{}centered sampling.
\begin{quote}\begin{description}
\sphinxlineitem{Returns}
\sphinxAtStartPar
The lower corner of the bounding box.
geo.Point3D: The upper corner of the bounding box.

\sphinxlineitem{Return type}
\sphinxAtStartPar
{\hyperref[\detokenize{deepdrr.geo:deepdrr.geo.Point3D}]{\sphinxcrossref{geo.Point3D}}}

\end{description}\end{quote}

\end{fulllineitems}

\index{get\_config() (deepdrr.vol.volume.Volume method)@\spxentry{get\_config()}\spxextra{deepdrr.vol.volume.Volume method}}

\begin{fulllineitems}
\phantomsection\label{\detokenize{deepdrr.vol:deepdrr.vol.volume.Volume.get_config}}
\pysigstartsignatures
\pysiglinewithargsret{\sphinxbfcode{\sphinxupquote{get\_config}}}{}{{ $\rightarrow$ Dict\DUrole{p,p}{{[}}str\DUrole{p,p}{,}\DUrole{w,w}{  }Any\DUrole{p,p}{{]}}}}
\pysigstopsignatures
\sphinxAtStartPar
Get the configuration of the volume. Does not include volumetric data.

\sphinxAtStartPar
Includes any info passed into \sphinxtitleref{config}.
\begin{quote}\begin{description}
\sphinxlineitem{Returns}
\sphinxAtStartPar
The configuration of the volume.

\sphinxlineitem{Return type}
\sphinxAtStartPar
Dict{[}str, Any{]}

\end{description}\end{quote}

\end{fulllineitems}

\index{get\_mesh\_in\_world() (deepdrr.vol.volume.Volume method)@\spxentry{get\_mesh\_in\_world()}\spxextra{deepdrr.vol.volume.Volume method}}

\begin{fulllineitems}
\phantomsection\label{\detokenize{deepdrr.vol:deepdrr.vol.volume.Volume.get_mesh_in_world}}
\pysigstartsignatures
\pysiglinewithargsret{\sphinxbfcode{\sphinxupquote{get\_mesh\_in\_world}}}{\sphinxparam{\DUrole{n,n}{full}\DUrole{p,p}{:}\DUrole{w,w}{  }\DUrole{n,n}{bool}\DUrole{w,w}{  }\DUrole{o,o}{=}\DUrole{w,w}{  }\DUrole{default_value}{False}}\sphinxparamcomma \sphinxparam{\DUrole{n,n}{use\_cached}\DUrole{p,p}{:}\DUrole{w,w}{  }\DUrole{n,n}{bool}\DUrole{w,w}{  }\DUrole{o,o}{=}\DUrole{w,w}{  }\DUrole{default_value}{True}}}{{ $\rightarrow$ PolyData}}
\pysigstopsignatures
\sphinxAtStartPar
Get a pyvista mesh of the outline in world\sphinxhyphen{}space.
\begin{quote}\begin{description}
\sphinxlineitem{Parameters}\begin{itemize}
\item {} 
\sphinxAtStartPar
\sphinxstyleliteralstrong{\sphinxupquote{full}} (\sphinxstyleliteralemphasis{\sphinxupquote{bool}}) \textendash{} Whether to render the full volume or just a wireframe. Defaults to False.

\item {} 
\sphinxAtStartPar
\sphinxstyleliteralstrong{\sphinxupquote{cache\_dir}} (\sphinxstyleliteralemphasis{\sphinxupquote{Optional}}\sphinxstyleliteralemphasis{\sphinxupquote{{[}}}\sphinxstyleliteralemphasis{\sphinxupquote{Path}}\sphinxstyleliteralemphasis{\sphinxupquote{{]}}}\sphinxstyleliteralemphasis{\sphinxupquote{, }}\sphinxstyleliteralemphasis{\sphinxupquote{optional}}) \textendash{} a location to cache the bone surface.

\item {} 
\sphinxAtStartPar
\sphinxstyleliteralstrong{\sphinxupquote{use\_cached}} (\sphinxstyleliteralemphasis{\sphinxupquote{bool}}) \textendash{} If False, don’t use the cached bone surface but re\sphinxhyphen{}create it (expensive). Defaults to True.

\end{itemize}

\sphinxlineitem{Returns}
\sphinxAtStartPar
pyvista mesh.

\sphinxlineitem{Return type}
\sphinxAtStartPar
pv.PolyData

\end{description}\end{quote}

\end{fulllineitems}

\index{get\_surface() (deepdrr.vol.volume.Volume method)@\spxentry{get\_surface()}\spxextra{deepdrr.vol.volume.Volume method}}

\begin{fulllineitems}
\phantomsection\label{\detokenize{deepdrr.vol:deepdrr.vol.volume.Volume.get_surface}}
\pysigstartsignatures
\pysiglinewithargsret{\sphinxbfcode{\sphinxupquote{get\_surface}}}{\sphinxparam{\DUrole{n,n}{material}\DUrole{p,p}{:}\DUrole{w,w}{  }\DUrole{n,n}{str}\DUrole{w,w}{  }\DUrole{o,o}{=}\DUrole{w,w}{  }\DUrole{default_value}{\textquotesingle{}bone\textquotesingle{}}}\sphinxparamcomma \sphinxparam{\DUrole{n,n}{use\_cached}\DUrole{p,p}{:}\DUrole{w,w}{  }\DUrole{n,n}{bool}\DUrole{w,w}{  }\DUrole{o,o}{=}\DUrole{w,w}{  }\DUrole{default_value}{True}}}{}
\pysigstopsignatures
\end{fulllineitems}

\index{ijk\_from\_anatomical (deepdrr.vol.volume.Volume property)@\spxentry{ijk\_from\_anatomical}\spxextra{deepdrr.vol.volume.Volume property}}

\begin{fulllineitems}
\phantomsection\label{\detokenize{deepdrr.vol:deepdrr.vol.volume.Volume.ijk_from_anatomical}}
\pysigstartsignatures
\pysigline{\sphinxbfcode{\sphinxupquote{property\DUrole{w,w}{  }}}\sphinxbfcode{\sphinxupquote{ijk\_from\_anatomical}}}
\pysigstopsignatures
\end{fulllineitems}

\index{ijk\_from\_world (deepdrr.vol.volume.Volume property)@\spxentry{ijk\_from\_world}\spxextra{deepdrr.vol.volume.Volume property}}

\begin{fulllineitems}
\phantomsection\label{\detokenize{deepdrr.vol:deepdrr.vol.volume.Volume.ijk_from_world}}
\pysigstartsignatures
\pysigline{\sphinxbfcode{\sphinxupquote{property\DUrole{w,w}{  }}}\sphinxbfcode{\sphinxupquote{ijk\_from\_world}}\sphinxbfcode{\sphinxupquote{\DUrole{p,p}{:}\DUrole{w,w}{  }{\hyperref[\detokenize{deepdrr.geo:deepdrr.geo.core.FrameTransform}]{\sphinxcrossref{FrameTransform}}}}}}
\pysigstopsignatures
\end{fulllineitems}

\index{interpolate() (deepdrr.vol.volume.Volume method)@\spxentry{interpolate()}\spxextra{deepdrr.vol.volume.Volume method}}

\begin{fulllineitems}
\phantomsection\label{\detokenize{deepdrr.vol:deepdrr.vol.volume.Volume.interpolate}}
\pysigstartsignatures
\pysiglinewithargsret{\sphinxbfcode{\sphinxupquote{interpolate}}}{\sphinxparam{\DUrole{o,o}{*}\DUrole{n,n}{x}\DUrole{p,p}{:}\DUrole{w,w}{  }\DUrole{n,n}{{\hyperref[\detokenize{deepdrr.geo:deepdrr.geo.core.Point3D}]{\sphinxcrossref{Point3D}}}}}\sphinxparamcomma \sphinxparam{\DUrole{n,n}{method}\DUrole{p,p}{:}\DUrole{w,w}{  }\DUrole{n,n}{str}\DUrole{w,w}{  }\DUrole{o,o}{=}\DUrole{w,w}{  }\DUrole{default_value}{\textquotesingle{}linear\textquotesingle{}}}}{{ $\rightarrow$ ndarray}}
\pysigstopsignatures
\sphinxAtStartPar
Interpolate the value of the volume at the point.

\sphinxAtStartPar
This is a \sphinxstyleemphasis{slow} version of interpolation, using scipy under the hood. DeepDRR uses cubic
spline interpolation on the GPU for rendering. This function is provided as a convenience.
\begin{quote}\begin{description}
\sphinxlineitem{Parameters}\begin{itemize}
\item {} 
\sphinxAtStartPar
\sphinxstyleliteralstrong{\sphinxupquote{x}} ({\hyperref[\detokenize{deepdrr.geo:deepdrr.geo.Point3D}]{\sphinxcrossref{\sphinxstyleliteralemphasis{\sphinxupquote{geo.Point3D}}}}}) \textendash{} The point or points in world\sphinxhyphen{}space.

\item {} 
\sphinxAtStartPar
\sphinxstyleliteralstrong{\sphinxupquote{method}} (\sphinxstyleliteralemphasis{\sphinxupquote{str}}) \textendash{} The interpolation method to be used.
Accepted values are “linear” and “nearest”.
Defaults to “linear.”

\end{itemize}

\sphinxlineitem{Returns}
\sphinxAtStartPar
\begin{description}
\sphinxlineitem{The interpolated value(s) of the point(s)}
\sphinxAtStartPar
in the Volume. If a point is outside the volume, the value is NaN.

\end{description}


\sphinxlineitem{Return type}
\sphinxAtStartPar
Union{[}float, np.ndarray{]}

\end{description}\end{quote}

\end{fulllineitems}

\index{isosurface() (deepdrr.vol.volume.Volume method)@\spxentry{isosurface()}\spxextra{deepdrr.vol.volume.Volume method}}

\begin{fulllineitems}
\phantomsection\label{\detokenize{deepdrr.vol:deepdrr.vol.volume.Volume.isosurface}}
\pysigstartsignatures
\pysiglinewithargsret{\sphinxbfcode{\sphinxupquote{isosurface}}}{\sphinxparam{\DUrole{n,n}{value}\DUrole{p,p}{:}\DUrole{w,w}{  }\DUrole{n,n}{float}\DUrole{w,w}{  }\DUrole{o,o}{=}\DUrole{w,w}{  }\DUrole{default_value}{0.5}}\sphinxparamcomma \sphinxparam{\DUrole{n,n}{label}\DUrole{p,p}{:}\DUrole{w,w}{  }\DUrole{n,n}{int\DUrole{w,w}{  }\DUrole{p,p}{|}\DUrole{w,w}{  }None}\DUrole{w,w}{  }\DUrole{o,o}{=}\DUrole{w,w}{  }\DUrole{default_value}{None}}\sphinxparamcomma \sphinxparam{\DUrole{n,n}{node\_centered}\DUrole{p,p}{:}\DUrole{w,w}{  }\DUrole{n,n}{bool}\DUrole{w,w}{  }\DUrole{o,o}{=}\DUrole{w,w}{  }\DUrole{default_value}{True}}\sphinxparamcomma \sphinxparam{\DUrole{n,n}{smooth}\DUrole{p,p}{:}\DUrole{w,w}{  }\DUrole{n,n}{bool}\DUrole{w,w}{  }\DUrole{o,o}{=}\DUrole{w,w}{  }\DUrole{default_value}{True}}\sphinxparamcomma \sphinxparam{\DUrole{n,n}{decimation}\DUrole{p,p}{:}\DUrole{w,w}{  }\DUrole{n,n}{float}\DUrole{w,w}{  }\DUrole{o,o}{=}\DUrole{w,w}{  }\DUrole{default_value}{0.01}}\sphinxparamcomma \sphinxparam{\DUrole{n,n}{decimation\_points}\DUrole{p,p}{:}\DUrole{w,w}{  }\DUrole{n,n}{int\DUrole{w,w}{  }\DUrole{p,p}{|}\DUrole{w,w}{  }None}\DUrole{w,w}{  }\DUrole{o,o}{=}\DUrole{w,w}{  }\DUrole{default_value}{None}}\sphinxparamcomma \sphinxparam{\DUrole{n,n}{smooth\_iter}\DUrole{p,p}{:}\DUrole{w,w}{  }\DUrole{n,n}{int}\DUrole{w,w}{  }\DUrole{o,o}{=}\DUrole{w,w}{  }\DUrole{default_value}{200}}\sphinxparamcomma \sphinxparam{\DUrole{n,n}{relaxation\_factor}\DUrole{p,p}{:}\DUrole{w,w}{  }\DUrole{n,n}{float}\DUrole{w,w}{  }\DUrole{o,o}{=}\DUrole{w,w}{  }\DUrole{default_value}{0.25}}\sphinxparamcomma \sphinxparam{\DUrole{n,n}{convert\_to\_LPS}\DUrole{p,p}{:}\DUrole{w,w}{  }\DUrole{n,n}{bool}\DUrole{w,w}{  }\DUrole{o,o}{=}\DUrole{w,w}{  }\DUrole{default_value}{False}}}{{ $\rightarrow$ PolyData}}
\pysigstopsignatures
\sphinxAtStartPar
Make an isosurface from the volume’s data, transforming to anatomical\_coordinates.

\sphinxAtStartPar
Accepts arguments passed to {\hyperref[\detokenize{deepdrr.utils:deepdrr.utils.mesh_utils.isosurface}]{\sphinxcrossref{\sphinxcode{\sphinxupquote{deepdrr.utils.mesh\_utils.isosurface()}}}}}.
\begin{quote}\begin{description}
\sphinxlineitem{Parameters}\begin{itemize}
\item {} 
\sphinxAtStartPar
\sphinxstyleliteralstrong{\sphinxupquote{value}} (\sphinxstyleliteralemphasis{\sphinxupquote{float}}) \textendash{} The value at which to make the isosurface.

\item {} 
\sphinxAtStartPar
\sphinxstyleliteralstrong{\sphinxupquote{label}} (\sphinxstyleliteralemphasis{\sphinxupquote{int}}) \textendash{} The label of the isosurface.

\item {} 
\sphinxAtStartPar
\sphinxstyleliteralstrong{\sphinxupquote{node\_centered}} (\sphinxstyleliteralemphasis{\sphinxupquote{bool}}) \textendash{} If True, the isosurface is centered at the node.
If False, the isosurface is centered at the cell.

\item {} 
\sphinxAtStartPar
\sphinxstyleliteralstrong{\sphinxupquote{smooth}} (\sphinxstyleliteralemphasis{\sphinxupquote{bool}}) \textendash{} If True, the isosurface is smoothed.

\item {} 
\sphinxAtStartPar
\sphinxstyleliteralstrong{\sphinxupquote{decimation}} (\sphinxstyleliteralemphasis{\sphinxupquote{float}}) \textendash{} The decimation factor (how many points to remove).

\item {} 
\sphinxAtStartPar
\sphinxstyleliteralstrong{\sphinxupquote{smooth\_iter}} (\sphinxstyleliteralemphasis{\sphinxupquote{int}}) \textendash{} The number of smoothing iterations.

\item {} 
\sphinxAtStartPar
\sphinxstyleliteralstrong{\sphinxupquote{relaxation\_factor}} (\sphinxstyleliteralemphasis{\sphinxupquote{float}}) \textendash{} The relaxation factor.

\item {} 
\sphinxAtStartPar
\sphinxstyleliteralstrong{\sphinxupquote{convert\_to\_LPS}} (\sphinxstyleliteralemphasis{\sphinxupquote{bool}}) \textendash{} If True, the isosurface is converted to LPS coordinates. (Recommended)

\end{itemize}

\sphinxlineitem{Returns}
\sphinxAtStartPar
The surface mesh in anatomical coordinates.

\sphinxlineitem{Return type}
\sphinxAtStartPar
pv.PolyData

\end{description}\end{quote}

\end{fulllineitems}

\index{load() (deepdrr.vol.volume.Volume class method)@\spxentry{load()}\spxextra{deepdrr.vol.volume.Volume class method}}

\begin{fulllineitems}
\phantomsection\label{\detokenize{deepdrr.vol:deepdrr.vol.volume.Volume.load}}
\pysigstartsignatures
\pysiglinewithargsret{\sphinxbfcode{\sphinxupquote{classmethod\DUrole{w,w}{  }}}\sphinxbfcode{\sphinxupquote{load}}}{\sphinxparam{\DUrole{n,n}{path}\DUrole{p,p}{:}\DUrole{w,w}{  }\DUrole{n,n}{Path}}\sphinxparamcomma \sphinxparam{\DUrole{n,n}{segmentation}\DUrole{p,p}{:}\DUrole{w,w}{  }\DUrole{n,n}{bool}\DUrole{w,w}{  }\DUrole{o,o}{=}\DUrole{w,w}{  }\DUrole{default_value}{False}}}{{ $\rightarrow$ {\hyperref[\detokenize{deepdrr.vol:deepdrr.vol.volume.Volume}]{\sphinxcrossref{Volume}}}}}
\pysigstopsignatures
\sphinxAtStartPar
Load a volume from disk.
\begin{quote}\begin{description}
\sphinxlineitem{Parameters}\begin{itemize}
\item {} 
\sphinxAtStartPar
\sphinxstyleliteralstrong{\sphinxupquote{path}} (\sphinxstyleliteralemphasis{\sphinxupquote{Path}}) \textendash{} a directory containing the volume and segmentations.

\item {} 
\sphinxAtStartPar
\sphinxstyleliteralstrong{\sphinxupquote{segmentation}} (\sphinxstyleliteralemphasis{\sphinxupquote{bool}}\sphinxstyleliteralemphasis{\sphinxupquote{, }}\sphinxstyleliteralemphasis{\sphinxupquote{optional}}) \textendash{} if the volume is a segmentation,
populate the materials from that.

\end{itemize}

\end{description}\end{quote}

\sphinxAtStartPar
Returns: Volume.

\end{fulllineitems}

\index{materials (deepdrr.vol.volume.Volume attribute)@\spxentry{materials}\spxextra{deepdrr.vol.volume.Volume attribute}}

\begin{fulllineitems}
\phantomsection\label{\detokenize{deepdrr.vol:deepdrr.vol.volume.Volume.materials}}
\pysigstartsignatures
\pysigline{\sphinxbfcode{\sphinxupquote{materials}}\sphinxbfcode{\sphinxupquote{\DUrole{p,p}{:}\DUrole{w,w}{  }Dict\DUrole{p,p}{{[}}str\DUrole{p,p}{,}\DUrole{w,w}{  }ndarray\DUrole{p,p}{{]}}}}}
\pysigstopsignatures
\end{fulllineitems}

\index{orient\_patient() (deepdrr.vol.volume.Volume method)@\spxentry{orient\_patient()}\spxextra{deepdrr.vol.volume.Volume method}}

\begin{fulllineitems}
\phantomsection\label{\detokenize{deepdrr.vol:deepdrr.vol.volume.Volume.orient_patient}}
\pysigstartsignatures
\pysiglinewithargsret{\sphinxbfcode{\sphinxupquote{orient\_patient}}}{\sphinxparam{\DUrole{n,n}{head\_first}\DUrole{p,p}{:}\DUrole{w,w}{  }\DUrole{n,n}{bool}\DUrole{w,w}{  }\DUrole{o,o}{=}\DUrole{w,w}{  }\DUrole{default_value}{True}}\sphinxparamcomma \sphinxparam{\DUrole{n,n}{supine}\DUrole{p,p}{:}\DUrole{w,w}{  }\DUrole{n,n}{bool}\DUrole{w,w}{  }\DUrole{o,o}{=}\DUrole{w,w}{  }\DUrole{default_value}{True}}\sphinxparamcomma \sphinxparam{\DUrole{n,n}{world\_from\_device}\DUrole{p,p}{:}\DUrole{w,w}{  }\DUrole{n,n}{{\hyperref[\detokenize{deepdrr.geo:deepdrr.geo.core.FrameTransform}]{\sphinxcrossref{FrameTransform}}}\DUrole{w,w}{  }\DUrole{p,p}{|}\DUrole{w,w}{  }None}\DUrole{w,w}{  }\DUrole{o,o}{=}\DUrole{w,w}{  }\DUrole{default_value}{None}}}{{ $\rightarrow$ None}}
\pysigstopsignatures
\sphinxAtStartPar
Orient the patient with the given orientation, aligning with the Loop\sphinxhyphen{}X coordinates.
\begin{quote}\begin{description}
\sphinxlineitem{Parameters}\begin{itemize}
\item {} 
\sphinxAtStartPar
\sphinxstyleliteralstrong{\sphinxupquote{head\_first}} \textendash{} If True, the patient is oriented with head (superior axis) pointing in the \sphinxhyphen{}Y direction. Defaults to True.

\item {} 
\sphinxAtStartPar
\sphinxstyleliteralstrong{\sphinxupquote{supine}} \textendash{} If True, the patient is oriented so that the anterior axis (stomach) points toward +Z. Defaults to True.

\end{itemize}

\end{description}\end{quote}

\end{fulllineitems}

\index{origin (deepdrr.vol.volume.Volume property)@\spxentry{origin}\spxextra{deepdrr.vol.volume.Volume property}}

\begin{fulllineitems}
\phantomsection\label{\detokenize{deepdrr.vol:deepdrr.vol.volume.Volume.origin}}
\pysigstartsignatures
\pysigline{\sphinxbfcode{\sphinxupquote{property\DUrole{w,w}{  }}}\sphinxbfcode{\sphinxupquote{origin}}\sphinxbfcode{\sphinxupquote{\DUrole{p,p}{:}\DUrole{w,w}{  }{\hyperref[\detokenize{deepdrr.geo:deepdrr.geo.core.Point3D}]{\sphinxcrossref{Point3D}}}}}}
\pysigstopsignatures
\sphinxAtStartPar
The origin of the volume in anatomical space.

\end{fulllineitems}

\index{origin\_in\_anatomical (deepdrr.vol.volume.Volume property)@\spxentry{origin\_in\_anatomical}\spxextra{deepdrr.vol.volume.Volume property}}

\begin{fulllineitems}
\phantomsection\label{\detokenize{deepdrr.vol:deepdrr.vol.volume.Volume.origin_in_anatomical}}
\pysigstartsignatures
\pysigline{\sphinxbfcode{\sphinxupquote{property\DUrole{w,w}{  }}}\sphinxbfcode{\sphinxupquote{origin\_in\_anatomical}}\sphinxbfcode{\sphinxupquote{\DUrole{p,p}{:}\DUrole{w,w}{  }{\hyperref[\detokenize{deepdrr.geo:deepdrr.geo.core.Point3D}]{\sphinxcrossref{Point3D}}}}}}
\pysigstopsignatures
\sphinxAtStartPar
The origin of the volume in anatomical space.

\end{fulllineitems}

\index{origin\_in\_world (deepdrr.vol.volume.Volume property)@\spxentry{origin\_in\_world}\spxextra{deepdrr.vol.volume.Volume property}}

\begin{fulllineitems}
\phantomsection\label{\detokenize{deepdrr.vol:deepdrr.vol.volume.Volume.origin_in_world}}
\pysigstartsignatures
\pysigline{\sphinxbfcode{\sphinxupquote{property\DUrole{w,w}{  }}}\sphinxbfcode{\sphinxupquote{origin\_in\_world}}\sphinxbfcode{\sphinxupquote{\DUrole{p,p}{:}\DUrole{w,w}{  }{\hyperref[\detokenize{deepdrr.geo:deepdrr.geo.core.Point3D}]{\sphinxcrossref{Point3D}}}}}}
\pysigstopsignatures
\sphinxAtStartPar
The origin of the volume in world space.

\end{fulllineitems}

\index{place() (deepdrr.vol.volume.Volume method)@\spxentry{place()}\spxextra{deepdrr.vol.volume.Volume method}}

\begin{fulllineitems}
\phantomsection\label{\detokenize{deepdrr.vol:deepdrr.vol.volume.Volume.place}}
\pysigstartsignatures
\pysiglinewithargsret{\sphinxbfcode{\sphinxupquote{place}}}{\sphinxparam{\DUrole{n,n}{point\_in\_anatomical}\DUrole{p,p}{:}\DUrole{w,w}{  }\DUrole{n,n}{{\hyperref[\detokenize{deepdrr.geo:deepdrr.geo.core.Point3D}]{\sphinxcrossref{Point3D}}}}}\sphinxparamcomma \sphinxparam{\DUrole{n,n}{desired\_point\_in\_world}\DUrole{p,p}{:}\DUrole{w,w}{  }\DUrole{n,n}{{\hyperref[\detokenize{deepdrr.geo:deepdrr.geo.core.Point3D}]{\sphinxcrossref{Point3D}}}}}}{{ $\rightarrow$ None}}
\pysigstopsignatures
\sphinxAtStartPar
Translate the volume so that x\_in\_anatomical corresponds to x\_in\_world.

\end{fulllineitems}

\index{place\_center() (deepdrr.vol.volume.Volume method)@\spxentry{place\_center()}\spxextra{deepdrr.vol.volume.Volume method}}

\begin{fulllineitems}
\phantomsection\label{\detokenize{deepdrr.vol:deepdrr.vol.volume.Volume.place_center}}
\pysigstartsignatures
\pysiglinewithargsret{\sphinxbfcode{\sphinxupquote{place\_center}}}{\sphinxparam{\DUrole{n,n}{x}\DUrole{p,p}{:}\DUrole{w,w}{  }\DUrole{n,n}{{\hyperref[\detokenize{deepdrr.geo:deepdrr.geo.core.Point3D}]{\sphinxcrossref{Point3D}}}}}}{{ $\rightarrow$ None}}
\pysigstopsignatures
\sphinxAtStartPar
Translate the volume so that its center is located at world\sphinxhyphen{}space point x.

\sphinxAtStartPar
Only changes the translation elements of the world\_from\_anatomical transform. Preserves the current rotation of the
\begin{quote}\begin{description}
\sphinxlineitem{Parameters}
\sphinxAtStartPar
\sphinxstyleliteralstrong{\sphinxupquote{x}} ({\hyperref[\detokenize{deepdrr.geo:deepdrr.geo.Point3D}]{\sphinxcrossref{\sphinxstyleliteralemphasis{\sphinxupquote{geo.Point3D}}}}}) \textendash{} the world\sphinxhyphen{}space point.

\end{description}\end{quote}

\end{fulllineitems}

\index{prone() (deepdrr.vol.volume.Volume method)@\spxentry{prone()}\spxextra{deepdrr.vol.volume.Volume method}}

\begin{fulllineitems}
\phantomsection\label{\detokenize{deepdrr.vol:deepdrr.vol.volume.Volume.prone}}
\pysigstartsignatures
\pysiglinewithargsret{\sphinxbfcode{\sphinxupquote{prone}}}{}{}
\pysigstopsignatures
\sphinxAtStartPar
Turns the volume to be face down.

\sphinxAtStartPar
This aligns the patient so that, in world space,
the posterior side is toward +Z, inferior is toward +X,
and right is toward +Y.
\begin{quote}\begin{description}
\sphinxlineitem{Raises}
\sphinxAtStartPar
\sphinxstyleliteralstrong{\sphinxupquote{NotImplementedError}} \textendash{} If the anatomical coordinate system is not “RAS”.

\end{description}\end{quote}

\end{fulllineitems}

\index{rotate() (deepdrr.vol.volume.Volume method)@\spxentry{rotate()}\spxextra{deepdrr.vol.volume.Volume method}}

\begin{fulllineitems}
\phantomsection\label{\detokenize{deepdrr.vol:deepdrr.vol.volume.Volume.rotate}}
\pysigstartsignatures
\pysiglinewithargsret{\sphinxbfcode{\sphinxupquote{rotate}}}{\sphinxparam{\DUrole{n,n}{rotation}\DUrole{p,p}{:}\DUrole{w,w}{  }\DUrole{n,n}{{\hyperref[\detokenize{deepdrr.geo:deepdrr.geo.core.Vector3D}]{\sphinxcrossref{Vector3D}}}\DUrole{w,w}{  }\DUrole{p,p}{|}\DUrole{w,w}{  }{\hyperref[\detokenize{deepdrr.geo:deepdrr.geo.Rotation}]{\sphinxcrossref{Rotation}}}}}\sphinxparamcomma \sphinxparam{\DUrole{n,n}{center}\DUrole{p,p}{:}\DUrole{w,w}{  }\DUrole{n,n}{{\hyperref[\detokenize{deepdrr.geo:deepdrr.geo.core.Point3D}]{\sphinxcrossref{Point3D}}}\DUrole{w,w}{  }\DUrole{p,p}{|}\DUrole{w,w}{  }None}\DUrole{w,w}{  }\DUrole{o,o}{=}\DUrole{w,w}{  }\DUrole{default_value}{None}}}{{ $\rightarrow$ {\hyperref[\detokenize{deepdrr.vol:deepdrr.vol.volume.Volume}]{\sphinxcrossref{Volume}}}}}
\pysigstopsignatures
\sphinxAtStartPar
Rotate the volume by \sphinxtitleref{rotation} about \sphinxtitleref{center}.
\begin{quote}\begin{description}
\sphinxlineitem{Parameters}\begin{itemize}
\item {} 
\sphinxAtStartPar
\sphinxstyleliteralstrong{\sphinxupquote{rotation}} (\sphinxstyleliteralemphasis{\sphinxupquote{Union}}\sphinxstyleliteralemphasis{\sphinxupquote{{[}}}{\hyperref[\detokenize{deepdrr.geo:deepdrr.geo.Vector3D}]{\sphinxcrossref{\sphinxstyleliteralemphasis{\sphinxupquote{geo.Vector3D}}}}}\sphinxstyleliteralemphasis{\sphinxupquote{, }}{\hyperref[\detokenize{deepdrr.geo:deepdrr.geo.Rotation}]{\sphinxcrossref{\sphinxstyleliteralemphasis{\sphinxupquote{Rotation}}}}}\sphinxstyleliteralemphasis{\sphinxupquote{{]}}}) \textendash{} the rotation in world\sphinxhyphen{}space. If it is a vector, \sphinxtitleref{Rotation.from\_rotvec(rotation)} is used.

\item {} 
\sphinxAtStartPar
\sphinxstyleliteralstrong{\sphinxupquote{center}} ({\hyperref[\detokenize{deepdrr.geo:deepdrr.geo.Point3D}]{\sphinxcrossref{\sphinxstyleliteralemphasis{\sphinxupquote{geo.Point3D}}}}}\sphinxstyleliteralemphasis{\sphinxupquote{, }}\sphinxstyleliteralemphasis{\sphinxupquote{optional}}) \textendash{} the center of rotation in world space coordinates. If None, the center of the volume is used.

\end{itemize}

\end{description}\end{quote}

\end{fulllineitems}

\index{save() (deepdrr.vol.volume.Volume method)@\spxentry{save()}\spxextra{deepdrr.vol.volume.Volume method}}

\begin{fulllineitems}
\phantomsection\label{\detokenize{deepdrr.vol:deepdrr.vol.volume.Volume.save}}
\pysigstartsignatures
\pysiglinewithargsret{\sphinxbfcode{\sphinxupquote{save}}}{\sphinxparam{\DUrole{n,n}{output\_dir}\DUrole{p,p}{:}\DUrole{w,w}{  }\DUrole{n,n}{Path}}\sphinxparamcomma \sphinxparam{\DUrole{n,n}{segmentation}\DUrole{p,p}{:}\DUrole{w,w}{  }\DUrole{n,n}{bool}\DUrole{w,w}{  }\DUrole{o,o}{=}\DUrole{w,w}{  }\DUrole{default_value}{False}}}{}
\pysigstopsignatures
\sphinxAtStartPar
Save the volume to disk as a nifti file.
\begin{quote}\begin{description}
\sphinxlineitem{Parameters}\begin{itemize}
\item {} 
\sphinxAtStartPar
\sphinxstyleliteralstrong{\sphinxupquote{output\_dir}} (\sphinxstyleliteralemphasis{\sphinxupquote{Path}}) \textendash{} a directory to save the volume and segmentations to.

\item {} 
\sphinxAtStartPar
\sphinxstyleliteralstrong{\sphinxupquote{segmentation}} (\sphinxstyleliteralemphasis{\sphinxupquote{bool}}\sphinxstyleliteralemphasis{\sphinxupquote{, }}\sphinxstyleliteralemphasis{\sphinxupquote{optional}}) \textendash{} if the volume is a segmentation, there’s
no need to save the materials.

\end{itemize}

\end{description}\end{quote}

\end{fulllineitems}

\index{segment\_materials() (deepdrr.vol.volume.Volume class method)@\spxentry{segment\_materials()}\spxextra{deepdrr.vol.volume.Volume class method}}

\begin{fulllineitems}
\phantomsection\label{\detokenize{deepdrr.vol:deepdrr.vol.volume.Volume.segment_materials}}
\pysigstartsignatures
\pysiglinewithargsret{\sphinxbfcode{\sphinxupquote{classmethod\DUrole{w,w}{  }}}\sphinxbfcode{\sphinxupquote{segment\_materials}}}{\sphinxparam{\DUrole{n,n}{hu\_values}\DUrole{p,p}{:}\DUrole{w,w}{  }\DUrole{n,n}{ndarray}}\sphinxparamcomma \sphinxparam{\DUrole{n,n}{anatomical\_from\_ijk}\DUrole{p,p}{:}\DUrole{w,w}{  }\DUrole{n,n}{{\hyperref[\detokenize{deepdrr.geo:deepdrr.geo.core.FrameTransform}]{\sphinxcrossref{FrameTransform}}}}}\sphinxparamcomma \sphinxparam{\DUrole{n,n}{use\_thresholding}\DUrole{p,p}{:}\DUrole{w,w}{  }\DUrole{n,n}{bool}\DUrole{w,w}{  }\DUrole{o,o}{=}\DUrole{w,w}{  }\DUrole{default_value}{True}}\sphinxparamcomma \sphinxparam{\DUrole{n,n}{use\_cached}\DUrole{p,p}{:}\DUrole{w,w}{  }\DUrole{n,n}{bool}\DUrole{w,w}{  }\DUrole{o,o}{=}\DUrole{w,w}{  }\DUrole{default_value}{True}}\sphinxparamcomma \sphinxparam{\DUrole{n,n}{save\_cache}\DUrole{p,p}{:}\DUrole{w,w}{  }\DUrole{n,n}{bool}\DUrole{w,w}{  }\DUrole{o,o}{=}\DUrole{w,w}{  }\DUrole{default_value}{False}}\sphinxparamcomma \sphinxparam{\DUrole{n,n}{cache\_dir}\DUrole{p,p}{:}\DUrole{w,w}{  }\DUrole{n,n}{Path\DUrole{w,w}{  }\DUrole{p,p}{|}\DUrole{w,w}{  }None}\DUrole{w,w}{  }\DUrole{o,o}{=}\DUrole{w,w}{  }\DUrole{default_value}{None}}\sphinxparamcomma \sphinxparam{\DUrole{n,n}{cache\_name}\DUrole{p,p}{:}\DUrole{w,w}{  }\DUrole{n,n}{str\DUrole{w,w}{  }\DUrole{p,p}{|}\DUrole{w,w}{  }None}\DUrole{w,w}{  }\DUrole{o,o}{=}\DUrole{w,w}{  }\DUrole{default_value}{None}}}{{ $\rightarrow$ Dict\DUrole{p,p}{{[}}str\DUrole{p,p}{,}\DUrole{w,w}{  }ndarray\DUrole{p,p}{{]}}}}
\pysigstopsignatures
\sphinxAtStartPar
Segment the materials in a volume, potentially caching.

\sphinxAtStartPar
If cache\_dir is None, then
\begin{quote}\begin{description}
\sphinxlineitem{Parameters}\begin{itemize}
\item {} 
\sphinxAtStartPar
\sphinxstyleliteralstrong{\sphinxupquote{hu\_values}} (\sphinxstyleliteralemphasis{\sphinxupquote{np.ndarray}}) \textendash{} volume data in Hounsfield Units.

\item {} 
\sphinxAtStartPar
\sphinxstyleliteralstrong{\sphinxupquote{use\_thretholding}} (\sphinxstyleliteralemphasis{\sphinxupquote{bool}}\sphinxstyleliteralemphasis{\sphinxupquote{, }}\sphinxstyleliteralemphasis{\sphinxupquote{optional}}) \textendash{} whether to segment with thresholding (true) or a DNN. Defaults to True.

\item {} 
\sphinxAtStartPar
\sphinxstyleliteralstrong{\sphinxupquote{use\_cached}} (\sphinxstyleliteralemphasis{\sphinxupquote{bool}}\sphinxstyleliteralemphasis{\sphinxupquote{, }}\sphinxstyleliteralemphasis{\sphinxupquote{optional}}) \textendash{} use the cached segmentation, if it exists. Defaults to True.

\item {} 
\sphinxAtStartPar
\sphinxstyleliteralstrong{\sphinxupquote{save\_cache}} (\sphinxstyleliteralemphasis{\sphinxupquote{bool}}\sphinxstyleliteralemphasis{\sphinxupquote{, }}\sphinxstyleliteralemphasis{\sphinxupquote{optional}}) \textendash{} save the segmentation to cache\_dir. Defaults to True.

\item {} 
\sphinxAtStartPar
\sphinxstyleliteralstrong{\sphinxupquote{cache\_dir}} (\sphinxstyleliteralemphasis{\sphinxupquote{Optional}}\sphinxstyleliteralemphasis{\sphinxupquote{{[}}}\sphinxstyleliteralemphasis{\sphinxupquote{Path}}\sphinxstyleliteralemphasis{\sphinxupquote{{]}}}\sphinxstyleliteralemphasis{\sphinxupquote{, }}\sphinxstyleliteralemphasis{\sphinxupquote{optional}}) \textendash{} where to look for the segmentation cache. If None, no caching performed. Defaults to None.

\item {} 
\sphinxAtStartPar
\sphinxstyleliteralstrong{\sphinxupquote{cache\_name}} (\sphinxstyleliteralemphasis{\sphinxupquote{str}}\sphinxstyleliteralemphasis{\sphinxupquote{, }}\sphinxstyleliteralemphasis{\sphinxupquote{optional}}) \textendash{} Name of cache file. Must be provided if use\_cached or cache\_dir is True. Defaults to None.

\end{itemize}

\sphinxlineitem{Returns}
\sphinxAtStartPar
materials segmentation.

\sphinxlineitem{Return type}
\sphinxAtStartPar
Dict{[}str, np.ndarray{]}

\end{description}\end{quote}

\end{fulllineitems}

\index{shape (deepdrr.vol.volume.Volume property)@\spxentry{shape}\spxextra{deepdrr.vol.volume.Volume property}}

\begin{fulllineitems}
\phantomsection\label{\detokenize{deepdrr.vol:deepdrr.vol.volume.Volume.shape}}
\pysigstartsignatures
\pysigline{\sphinxbfcode{\sphinxupquote{property\DUrole{w,w}{  }}}\sphinxbfcode{\sphinxupquote{shape}}\sphinxbfcode{\sphinxupquote{\DUrole{p,p}{:}\DUrole{w,w}{  }Tuple\DUrole{p,p}{{[}}int\DUrole{p,p}{,}\DUrole{w,w}{  }int\DUrole{p,p}{,}\DUrole{w,w}{  }int\DUrole{p,p}{{]}}}}}
\pysigstopsignatures
\end{fulllineitems}

\index{shrink() (deepdrr.vol.volume.Volume method)@\spxentry{shrink()}\spxextra{deepdrr.vol.volume.Volume method}}

\begin{fulllineitems}
\phantomsection\label{\detokenize{deepdrr.vol:deepdrr.vol.volume.Volume.shrink}}
\pysigstartsignatures
\pysiglinewithargsret{\sphinxbfcode{\sphinxupquote{shrink}}}{}{{ $\rightarrow$ {\hyperref[\detokenize{deepdrr.vol:deepdrr.vol.volume.Volume}]{\sphinxcrossref{Volume}}}}}
\pysigstopsignatures
\sphinxAtStartPar
Crop the volume to remove empty space.
\begin{quote}\begin{description}
\sphinxlineitem{Returns}
\sphinxAtStartPar
The cropped volume.

\sphinxlineitem{Return type}
\sphinxAtStartPar
{\hyperref[\detokenize{deepdrr.vol:deepdrr.vol.volume.Volume}]{\sphinxcrossref{Volume}}}

\end{description}\end{quote}

\end{fulllineitems}

\index{spacing (deepdrr.vol.volume.Volume property)@\spxentry{spacing}\spxextra{deepdrr.vol.volume.Volume property}}

\begin{fulllineitems}
\phantomsection\label{\detokenize{deepdrr.vol:deepdrr.vol.volume.Volume.spacing}}
\pysigstartsignatures
\pysigline{\sphinxbfcode{\sphinxupquote{property\DUrole{w,w}{  }}}\sphinxbfcode{\sphinxupquote{spacing}}\sphinxbfcode{\sphinxupquote{\DUrole{p,p}{:}\DUrole{w,w}{  }{\hyperref[\detokenize{deepdrr.geo:deepdrr.geo.core.Vector3D}]{\sphinxcrossref{Vector3D}}}}}}
\pysigstopsignatures
\sphinxAtStartPar
The spacing of the voxels.

\end{fulllineitems}

\index{supine() (deepdrr.vol.volume.Volume method)@\spxentry{supine()}\spxextra{deepdrr.vol.volume.Volume method}}

\begin{fulllineitems}
\phantomsection\label{\detokenize{deepdrr.vol:deepdrr.vol.volume.Volume.supine}}
\pysigstartsignatures
\pysiglinewithargsret{\sphinxbfcode{\sphinxupquote{supine}}}{}{}
\pysigstopsignatures
\sphinxAtStartPar
Turns the volume to be face up.

\sphinxAtStartPar
This aligns the patient so that, in world space,
the anterior side is toward +Z, inferior is toward +X,
and left is toward +Y.
\begin{quote}\begin{description}
\sphinxlineitem{Raises}
\sphinxAtStartPar
\sphinxstyleliteralstrong{\sphinxupquote{NotImplementedError}} \textendash{} If the anatomical coordinate system is not “RAS”.

\end{description}\end{quote}

\end{fulllineitems}

\index{translate() (deepdrr.vol.volume.Volume method)@\spxentry{translate()}\spxextra{deepdrr.vol.volume.Volume method}}

\begin{fulllineitems}
\phantomsection\label{\detokenize{deepdrr.vol:deepdrr.vol.volume.Volume.translate}}
\pysigstartsignatures
\pysiglinewithargsret{\sphinxbfcode{\sphinxupquote{translate}}}{\sphinxparam{\DUrole{n,n}{t}\DUrole{p,p}{:}\DUrole{w,w}{  }\DUrole{n,n}{{\hyperref[\detokenize{deepdrr.geo:deepdrr.geo.core.Vector3D}]{\sphinxcrossref{Vector3D}}}}}}{{ $\rightarrow$ {\hyperref[\detokenize{deepdrr.vol:deepdrr.vol.volume.Volume}]{\sphinxcrossref{Volume}}}}}
\pysigstopsignatures
\sphinxAtStartPar
Translate the volume by \sphinxtitleref{t}.
\begin{quote}\begin{description}
\sphinxlineitem{Parameters}
\sphinxAtStartPar
\sphinxstyleliteralstrong{\sphinxupquote{t}} ({\hyperref[\detokenize{deepdrr.geo:deepdrr.geo.Vector3D}]{\sphinxcrossref{\sphinxstyleliteralemphasis{\sphinxupquote{geo.Vector3D}}}}}) \textendash{} The vector to translate by, in world space.

\end{description}\end{quote}

\end{fulllineitems}

\index{translate\_center\_to() (deepdrr.vol.volume.Volume method)@\spxentry{translate\_center\_to()}\spxextra{deepdrr.vol.volume.Volume method}}

\begin{fulllineitems}
\phantomsection\label{\detokenize{deepdrr.vol:deepdrr.vol.volume.Volume.translate_center_to}}
\pysigstartsignatures
\pysiglinewithargsret{\sphinxbfcode{\sphinxupquote{translate\_center\_to}}}{\sphinxparam{\DUrole{n,n}{x}\DUrole{p,p}{:}\DUrole{w,w}{  }\DUrole{n,n}{{\hyperref[\detokenize{deepdrr.geo:deepdrr.geo.core.Point3D}]{\sphinxcrossref{Point3D}}}}}}{{ $\rightarrow$ None}}
\pysigstopsignatures
\sphinxAtStartPar
Translate the volume so that its center is located at world\sphinxhyphen{}space point x.

\sphinxAtStartPar
Only changes the translation elements of the world\_from\_anatomical transform. Preserves the current rotation of the
\begin{quote}\begin{description}
\sphinxlineitem{Parameters}
\sphinxAtStartPar
\sphinxstyleliteralstrong{\sphinxupquote{x}} ({\hyperref[\detokenize{deepdrr.geo:deepdrr.geo.Point3D}]{\sphinxcrossref{\sphinxstyleliteralemphasis{\sphinxupquote{geo.Point3D}}}}}) \textendash{} the world\sphinxhyphen{}space point.

\end{description}\end{quote}

\end{fulllineitems}

\index{world\_from\_IJK (deepdrr.vol.volume.Volume property)@\spxentry{world\_from\_IJK}\spxextra{deepdrr.vol.volume.Volume property}}

\begin{fulllineitems}
\phantomsection\label{\detokenize{deepdrr.vol:deepdrr.vol.volume.Volume.world_from_IJK}}
\pysigstartsignatures
\pysigline{\sphinxbfcode{\sphinxupquote{property\DUrole{w,w}{  }}}\sphinxbfcode{\sphinxupquote{world\_from\_IJK}}\sphinxbfcode{\sphinxupquote{\DUrole{p,p}{:}\DUrole{w,w}{  }{\hyperref[\detokenize{deepdrr.geo:deepdrr.geo.core.FrameTransform}]{\sphinxcrossref{FrameTransform}}}}}}
\pysigstopsignatures
\end{fulllineitems}

\index{world\_from\_anatomical (deepdrr.vol.volume.Volume attribute)@\spxentry{world\_from\_anatomical}\spxextra{deepdrr.vol.volume.Volume attribute}}

\begin{fulllineitems}
\phantomsection\label{\detokenize{deepdrr.vol:deepdrr.vol.volume.Volume.world_from_anatomical}}
\pysigstartsignatures
\pysigline{\sphinxbfcode{\sphinxupquote{world\_from\_anatomical}}\sphinxbfcode{\sphinxupquote{\DUrole{p,p}{:}\DUrole{w,w}{  }{\hyperref[\detokenize{deepdrr.geo:deepdrr.geo.core.FrameTransform}]{\sphinxcrossref{FrameTransform}}}}}}
\pysigstopsignatures
\end{fulllineitems}

\index{world\_from\_ijk (deepdrr.vol.volume.Volume property)@\spxentry{world\_from\_ijk}\spxextra{deepdrr.vol.volume.Volume property}}

\begin{fulllineitems}
\phantomsection\label{\detokenize{deepdrr.vol:deepdrr.vol.volume.Volume.world_from_ijk}}
\pysigstartsignatures
\pysigline{\sphinxbfcode{\sphinxupquote{property\DUrole{w,w}{  }}}\sphinxbfcode{\sphinxupquote{world\_from\_ijk}}\sphinxbfcode{\sphinxupquote{\DUrole{p,p}{:}\DUrole{w,w}{  }{\hyperref[\detokenize{deepdrr.geo:deepdrr.geo.core.FrameTransform}]{\sphinxcrossref{FrameTransform}}}}}}
\pysigstopsignatures
\end{fulllineitems}


\end{fulllineitems}



\subsection{Module contents}
\label{\detokenize{deepdrr.vol:module-deepdrr.vol}}\label{\detokenize{deepdrr.vol:module-contents}}\index{module@\spxentry{module}!deepdrr.vol@\spxentry{deepdrr.vol}}\index{deepdrr.vol@\spxentry{deepdrr.vol}!module@\spxentry{module}}\index{KWire (class in deepdrr.vol)@\spxentry{KWire}\spxextra{class in deepdrr.vol}}

\begin{fulllineitems}
\phantomsection\label{\detokenize{deepdrr.vol:deepdrr.vol.KWire}}
\pysigstartsignatures
\pysiglinewithargsret{\sphinxbfcode{\sphinxupquote{class\DUrole{w,w}{  }}}\sphinxcode{\sphinxupquote{deepdrr.vol.}}\sphinxbfcode{\sphinxupquote{KWire}}}{\sphinxparam{\DUrole{o,o}{*}\DUrole{n,n}{args}}\sphinxparamcomma \sphinxparam{\DUrole{n,n}{tip}\DUrole{p,p}{:}\DUrole{w,w}{  }\DUrole{n,n}{{\hyperref[\detokenize{deepdrr.geo:deepdrr.geo.core.Point3D}]{\sphinxcrossref{Point3D}}}\DUrole{w,w}{  }\DUrole{p,p}{|}\DUrole{w,w}{  }None}\DUrole{w,w}{  }\DUrole{o,o}{=}\DUrole{w,w}{  }\DUrole{default_value}{None}}\sphinxparamcomma \sphinxparam{\DUrole{n,n}{base}\DUrole{p,p}{:}\DUrole{w,w}{  }\DUrole{n,n}{{\hyperref[\detokenize{deepdrr.geo:deepdrr.geo.core.Point3D}]{\sphinxcrossref{Point3D}}}\DUrole{w,w}{  }\DUrole{p,p}{|}\DUrole{w,w}{  }None}\DUrole{w,w}{  }\DUrole{o,o}{=}\DUrole{w,w}{  }\DUrole{default_value}{None}}\sphinxparamcomma \sphinxparam{\DUrole{o,o}{**}\DUrole{n,n}{kwargs}}}{}
\pysigstopsignatures
\sphinxAtStartPar
Bases: {\hyperref[\detokenize{deepdrr.vol:deepdrr.vol.volume.Volume}]{\sphinxcrossref{\sphinxcode{\sphinxupquote{Volume}}}}}
\index{advance() (deepdrr.vol.KWire method)@\spxentry{advance()}\spxextra{deepdrr.vol.KWire method}}

\begin{fulllineitems}
\phantomsection\label{\detokenize{deepdrr.vol:deepdrr.vol.KWire.advance}}
\pysigstartsignatures
\pysiglinewithargsret{\sphinxbfcode{\sphinxupquote{advance}}}{\sphinxparam{\DUrole{n,n}{distance}\DUrole{p,p}{:}\DUrole{w,w}{  }\DUrole{n,n}{float}}}{}
\pysigstopsignatures
\sphinxAtStartPar
Move the tool forward by the given distance.
\begin{quote}\begin{description}
\sphinxlineitem{Parameters}
\sphinxAtStartPar
\sphinxstyleliteralstrong{\sphinxupquote{distance}} (\sphinxstyleliteralemphasis{\sphinxupquote{float}}) \textendash{} The distance to move the tool forward.

\end{description}\end{quote}

\end{fulllineitems}

\index{align() (deepdrr.vol.KWire method)@\spxentry{align()}\spxextra{deepdrr.vol.KWire method}}

\begin{fulllineitems}
\phantomsection\label{\detokenize{deepdrr.vol:deepdrr.vol.KWire.align}}
\pysigstartsignatures
\pysiglinewithargsret{\sphinxbfcode{\sphinxupquote{align}}}{\sphinxparam{\DUrole{n,n}{startpoint\_in\_world}\DUrole{p,p}{:}\DUrole{w,w}{  }\DUrole{n,n}{{\hyperref[\detokenize{deepdrr.geo:deepdrr.geo.core.Point3D}]{\sphinxcrossref{Point3D}}}}}\sphinxparamcomma \sphinxparam{\DUrole{n,n}{endpoint\_in\_world}\DUrole{p,p}{:}\DUrole{w,w}{  }\DUrole{n,n}{{\hyperref[\detokenize{deepdrr.geo:deepdrr.geo.core.Point3D}]{\sphinxcrossref{Point3D}}}}}\sphinxparamcomma \sphinxparam{\DUrole{n,n}{progress}\DUrole{p,p}{:}\DUrole{w,w}{  }\DUrole{n,n}{float}\DUrole{w,w}{  }\DUrole{o,o}{=}\DUrole{w,w}{  }\DUrole{default_value}{1.0}}\sphinxparamcomma \sphinxparam{\DUrole{n,n}{distance}\DUrole{p,p}{:}\DUrole{w,w}{  }\DUrole{n,n}{float\DUrole{w,w}{  }\DUrole{p,p}{|}\DUrole{w,w}{  }None}\DUrole{w,w}{  }\DUrole{o,o}{=}\DUrole{w,w}{  }\DUrole{default_value}{None}}}{{ $\rightarrow$ None}}
\pysigstopsignatures
\sphinxAtStartPar
Align the tool so that it lies between the two points, tip pointing toward the endpoint.
\begin{quote}\begin{description}
\sphinxlineitem{Parameters}\begin{itemize}
\item {} 
\sphinxAtStartPar
\sphinxstyleliteralstrong{\sphinxupquote{start\_point\_in\_world}} ({\hyperref[\detokenize{deepdrr.geo:deepdrr.geo.Point3D}]{\sphinxcrossref{\sphinxstyleliteralemphasis{\sphinxupquote{geo.Point3D}}}}}) \textendash{} The first point, in world space.

\item {} 
\sphinxAtStartPar
\sphinxstyleliteralstrong{\sphinxupquote{end\_point\_in\_world}} ({\hyperref[\detokenize{deepdrr.geo:deepdrr.geo.Point3D}]{\sphinxcrossref{\sphinxstyleliteralemphasis{\sphinxupquote{geo.Point3D}}}}}) \textendash{} The second point, in world space. The tip of the tool points toward this point.

\item {} 
\sphinxAtStartPar
\sphinxstyleliteralstrong{\sphinxupquote{progress}} (\sphinxstyleliteralemphasis{\sphinxupquote{float}}\sphinxstyleliteralemphasis{\sphinxupquote{, }}\sphinxstyleliteralemphasis{\sphinxupquote{optional}}) \textendash{} Where to place the tip of the tool between the start and end point,
on a scale from 0 to 1. 0 corresponds to the tip placed at the start point, 1 at the end point. Defaults to 1.0.

\item {} 
\sphinxAtStartPar
\sphinxstyleliteralstrong{\sphinxupquote{distance}} (\sphinxstyleliteralemphasis{\sphinxupquote{Optional}}\sphinxstyleliteralemphasis{\sphinxupquote{{[}}}\sphinxstyleliteralemphasis{\sphinxupquote{float}}\sphinxstyleliteralemphasis{\sphinxupquote{{]}}}\sphinxstyleliteralemphasis{\sphinxupquote{, }}\sphinxstyleliteralemphasis{\sphinxupquote{optional}}) \textendash{} The distance of the tip along the trajectory. 0 corresponds
to the tip placed at the start point, {\color{red}\bfseries{}|startpoint \sphinxhyphen{} endpoint|} at the end point.
Overrides progress if provided. Defaults to None.

\end{itemize}

\end{description}\end{quote}

\end{fulllineitems}

\index{anatomical\_coordinate\_system (deepdrr.vol.KWire attribute)@\spxentry{anatomical\_coordinate\_system}\spxextra{deepdrr.vol.KWire attribute}}

\begin{fulllineitems}
\phantomsection\label{\detokenize{deepdrr.vol:deepdrr.vol.KWire.anatomical_coordinate_system}}
\pysigstartsignatures
\pysigline{\sphinxbfcode{\sphinxupquote{anatomical\_coordinate\_system}}\sphinxbfcode{\sphinxupquote{\DUrole{p,p}{:}\DUrole{w,w}{  }str\DUrole{w,w}{  }\DUrole{p,p}{|}\DUrole{w,w}{  }None}}}
\pysigstopsignatures
\end{fulllineitems}

\index{anatomical\_from\_IJK (deepdrr.vol.KWire attribute)@\spxentry{anatomical\_from\_IJK}\spxextra{deepdrr.vol.KWire attribute}}

\begin{fulllineitems}
\phantomsection\label{\detokenize{deepdrr.vol:deepdrr.vol.KWire.anatomical_from_IJK}}
\pysigstartsignatures
\pysigline{\sphinxbfcode{\sphinxupquote{anatomical\_from\_IJK}}\sphinxbfcode{\sphinxupquote{\DUrole{p,p}{:}\DUrole{w,w}{  }{\hyperref[\detokenize{deepdrr.geo:deepdrr.geo.core.FrameTransform}]{\sphinxcrossref{FrameTransform}}}}}}
\pysigstopsignatures
\end{fulllineitems}

\index{base (deepdrr.vol.KWire property)@\spxentry{base}\spxextra{deepdrr.vol.KWire property}}

\begin{fulllineitems}
\phantomsection\label{\detokenize{deepdrr.vol:deepdrr.vol.KWire.base}}
\pysigstartsignatures
\pysigline{\sphinxbfcode{\sphinxupquote{property\DUrole{w,w}{  }}}\sphinxbfcode{\sphinxupquote{base}}\sphinxbfcode{\sphinxupquote{\DUrole{p,p}{:}\DUrole{w,w}{  }{\hyperref[\detokenize{deepdrr.geo:deepdrr.geo.core.Point3D}]{\sphinxcrossref{Point3D}}}}}}
\pysigstopsignatures
\sphinxAtStartPar
The base of the tool in world space.

\end{fulllineitems}

\index{base\_in\_IJK (deepdrr.vol.KWire attribute)@\spxentry{base\_in\_IJK}\spxextra{deepdrr.vol.KWire attribute}}

\begin{fulllineitems}
\phantomsection\label{\detokenize{deepdrr.vol:deepdrr.vol.KWire.base_in_IJK}}
\pysigstartsignatures
\pysigline{\sphinxbfcode{\sphinxupquote{base\_in\_IJK}}\sphinxbfcode{\sphinxupquote{\DUrole{p,p}{:}\DUrole{w,w}{  }{\hyperref[\detokenize{deepdrr.geo:deepdrr.geo.core.Point3D}]{\sphinxcrossref{Point3D}}}}}}
\pysigstopsignatures
\end{fulllineitems}

\index{base\_in\_anatomical (deepdrr.vol.KWire property)@\spxentry{base\_in\_anatomical}\spxextra{deepdrr.vol.KWire property}}

\begin{fulllineitems}
\phantomsection\label{\detokenize{deepdrr.vol:deepdrr.vol.KWire.base_in_anatomical}}
\pysigstartsignatures
\pysigline{\sphinxbfcode{\sphinxupquote{property\DUrole{w,w}{  }}}\sphinxbfcode{\sphinxupquote{base\_in\_anatomical}}\sphinxbfcode{\sphinxupquote{\DUrole{p,p}{:}\DUrole{w,w}{  }{\hyperref[\detokenize{deepdrr.geo:deepdrr.geo.core.Point3D}]{\sphinxcrossref{Point3D}}}}}}
\pysigstopsignatures
\sphinxAtStartPar
Get the location of the tool base in anatomical coordinates.

\end{fulllineitems}

\index{base\_in\_ijk (deepdrr.vol.KWire property)@\spxentry{base\_in\_ijk}\spxextra{deepdrr.vol.KWire property}}

\begin{fulllineitems}
\phantomsection\label{\detokenize{deepdrr.vol:deepdrr.vol.KWire.base_in_ijk}}
\pysigstartsignatures
\pysigline{\sphinxbfcode{\sphinxupquote{property\DUrole{w,w}{  }}}\sphinxbfcode{\sphinxupquote{base\_in\_ijk}}\sphinxbfcode{\sphinxupquote{\DUrole{p,p}{:}\DUrole{w,w}{  }{\hyperref[\detokenize{deepdrr.geo:deepdrr.geo.core.Point3D}]{\sphinxcrossref{Point3D}}}}}}
\pysigstopsignatures
\end{fulllineitems}

\index{base\_in\_world (deepdrr.vol.KWire property)@\spxentry{base\_in\_world}\spxextra{deepdrr.vol.KWire property}}

\begin{fulllineitems}
\phantomsection\label{\detokenize{deepdrr.vol:deepdrr.vol.KWire.base_in_world}}
\pysigstartsignatures
\pysigline{\sphinxbfcode{\sphinxupquote{property\DUrole{w,w}{  }}}\sphinxbfcode{\sphinxupquote{base\_in\_world}}\sphinxbfcode{\sphinxupquote{\DUrole{p,p}{:}\DUrole{w,w}{  }{\hyperref[\detokenize{deepdrr.geo:deepdrr.geo.core.Point3D}]{\sphinxcrossref{Point3D}}}}}}
\pysigstopsignatures
\sphinxAtStartPar
Get the location of the tool base in world coordinates.

\end{fulllineitems}

\index{centerline\_in\_world (deepdrr.vol.KWire property)@\spxentry{centerline\_in\_world}\spxextra{deepdrr.vol.KWire property}}

\begin{fulllineitems}
\phantomsection\label{\detokenize{deepdrr.vol:deepdrr.vol.KWire.centerline_in_world}}
\pysigstartsignatures
\pysigline{\sphinxbfcode{\sphinxupquote{property\DUrole{w,w}{  }}}\sphinxbfcode{\sphinxupquote{centerline\_in\_world}}\sphinxbfcode{\sphinxupquote{\DUrole{p,p}{:}\DUrole{w,w}{  }{\hyperref[\detokenize{deepdrr.geo:deepdrr.geo.hyperplane.Line3D}]{\sphinxcrossref{Line3D}}}}}}
\pysigstopsignatures
\end{fulllineitems}

\index{data (deepdrr.vol.KWire attribute)@\spxentry{data}\spxextra{deepdrr.vol.KWire attribute}}

\begin{fulllineitems}
\phantomsection\label{\detokenize{deepdrr.vol:deepdrr.vol.KWire.data}}
\pysigstartsignatures
\pysigline{\sphinxbfcode{\sphinxupquote{data}}\sphinxbfcode{\sphinxupquote{\DUrole{p,p}{:}\DUrole{w,w}{  }ndarray}}}
\pysigstopsignatures
\end{fulllineitems}

\index{diameter (deepdrr.vol.KWire attribute)@\spxentry{diameter}\spxextra{deepdrr.vol.KWire attribute}}

\begin{fulllineitems}
\phantomsection\label{\detokenize{deepdrr.vol:deepdrr.vol.KWire.diameter}}
\pysigstartsignatures
\pysigline{\sphinxbfcode{\sphinxupquote{diameter}}\sphinxbfcode{\sphinxupquote{\DUrole{w,w}{  }\DUrole{p,p}{=}\DUrole{w,w}{  }2.0}}}
\pysigstopsignatures
\end{fulllineitems}

\index{from\_example() (deepdrr.vol.KWire class method)@\spxentry{from\_example()}\spxextra{deepdrr.vol.KWire class method}}

\begin{fulllineitems}
\phantomsection\label{\detokenize{deepdrr.vol:deepdrr.vol.KWire.from_example}}
\pysigstartsignatures
\pysiglinewithargsret{\sphinxbfcode{\sphinxupquote{classmethod\DUrole{w,w}{  }}}\sphinxbfcode{\sphinxupquote{from\_example}}}{\sphinxparam{\DUrole{n,n}{diameter}\DUrole{p,p}{:}\DUrole{w,w}{  }\DUrole{n,n}{float}\DUrole{w,w}{  }\DUrole{o,o}{=}\DUrole{w,w}{  }\DUrole{default_value}{2}}\sphinxparamcomma \sphinxparam{\DUrole{n,n}{density}\DUrole{p,p}{:}\DUrole{w,w}{  }\DUrole{n,n}{float}\DUrole{w,w}{  }\DUrole{o,o}{=}\DUrole{w,w}{  }\DUrole{default_value}{7.5}}\sphinxparamcomma \sphinxparam{\DUrole{n,n}{world\_from\_anatomical}\DUrole{p,p}{:}\DUrole{w,w}{  }\DUrole{n,n}{{\hyperref[\detokenize{deepdrr.geo:deepdrr.geo.core.F}]{\sphinxcrossref{F}}}\DUrole{w,w}{  }\DUrole{p,p}{|}\DUrole{w,w}{  }None}\DUrole{w,w}{  }\DUrole{o,o}{=}\DUrole{w,w}{  }\DUrole{default_value}{None}}\sphinxparamcomma \sphinxparam{\DUrole{o,o}{**}\DUrole{n,n}{kwargs}}}{}
\pysigstopsignatures
\sphinxAtStartPar
Creates a KWire from the provided download link.
\begin{quote}\begin{description}
\sphinxlineitem{Parameters}
\sphinxAtStartPar
\sphinxstyleliteralstrong{\sphinxupquote{density}} (\sphinxstyleliteralemphasis{\sphinxupquote{float}}\sphinxstyleliteralemphasis{\sphinxupquote{, }}\sphinxstyleliteralemphasis{\sphinxupquote{optional}}) \textendash{} Density of the K\sphinxhyphen{}wire metal.

\sphinxlineitem{Returns}
\sphinxAtStartPar
The example KWire built into DeepDRR.

\sphinxlineitem{Return type}
\sphinxAtStartPar
{\hyperref[\detokenize{deepdrr.vol:deepdrr.vol.KWire}]{\sphinxcrossref{KWire}}}

\end{description}\end{quote}

\end{fulllineitems}

\index{length\_in\_world (deepdrr.vol.KWire property)@\spxentry{length\_in\_world}\spxextra{deepdrr.vol.KWire property}}

\begin{fulllineitems}
\phantomsection\label{\detokenize{deepdrr.vol:deepdrr.vol.KWire.length_in_world}}
\pysigstartsignatures
\pysigline{\sphinxbfcode{\sphinxupquote{property\DUrole{w,w}{  }}}\sphinxbfcode{\sphinxupquote{length\_in\_world}}}
\pysigstopsignatures
\end{fulllineitems}

\index{materials (deepdrr.vol.KWire attribute)@\spxentry{materials}\spxextra{deepdrr.vol.KWire attribute}}

\begin{fulllineitems}
\phantomsection\label{\detokenize{deepdrr.vol:deepdrr.vol.KWire.materials}}
\pysigstartsignatures
\pysigline{\sphinxbfcode{\sphinxupquote{materials}}\sphinxbfcode{\sphinxupquote{\DUrole{p,p}{:}\DUrole{w,w}{  }Dict\DUrole{p,p}{{[}}str\DUrole{p,p}{,}\DUrole{w,w}{  }ndarray\DUrole{p,p}{{]}}}}}
\pysigstopsignatures
\end{fulllineitems}

\index{orient() (deepdrr.vol.KWire method)@\spxentry{orient()}\spxextra{deepdrr.vol.KWire method}}

\begin{fulllineitems}
\phantomsection\label{\detokenize{deepdrr.vol:deepdrr.vol.KWire.orient}}
\pysigstartsignatures
\pysiglinewithargsret{\sphinxbfcode{\sphinxupquote{orient}}}{\sphinxparam{\DUrole{n,n}{startpoint}\DUrole{p,p}{:}\DUrole{w,w}{  }\DUrole{n,n}{{\hyperref[\detokenize{deepdrr.geo:deepdrr.geo.core.Point3D}]{\sphinxcrossref{Point3D}}}}}\sphinxparamcomma \sphinxparam{\DUrole{n,n}{direction}\DUrole{p,p}{:}\DUrole{w,w}{  }\DUrole{n,n}{{\hyperref[\detokenize{deepdrr.geo:deepdrr.geo.core.Vector3D}]{\sphinxcrossref{Vector3D}}}}}\sphinxparamcomma \sphinxparam{\DUrole{n,n}{distance}\DUrole{p,p}{:}\DUrole{w,w}{  }\DUrole{n,n}{float}\DUrole{w,w}{  }\DUrole{o,o}{=}\DUrole{w,w}{  }\DUrole{default_value}{0}}}{}
\pysigstopsignatures
\sphinxAtStartPar
Place the tip at startpoint and orient the tool to point toward the direction.

\end{fulllineitems}

\index{radius (deepdrr.vol.KWire property)@\spxentry{radius}\spxextra{deepdrr.vol.KWire property}}

\begin{fulllineitems}
\phantomsection\label{\detokenize{deepdrr.vol:deepdrr.vol.KWire.radius}}
\pysigstartsignatures
\pysigline{\sphinxbfcode{\sphinxupquote{property\DUrole{w,w}{  }}}\sphinxbfcode{\sphinxupquote{radius}}\sphinxbfcode{\sphinxupquote{\DUrole{p,p}{:}\DUrole{w,w}{  }float}}}
\pysigstopsignatures
\end{fulllineitems}

\index{scale() (deepdrr.vol.KWire method)@\spxentry{scale()}\spxextra{deepdrr.vol.KWire method}}

\begin{fulllineitems}
\phantomsection\label{\detokenize{deepdrr.vol:deepdrr.vol.KWire.scale}}
\pysigstartsignatures
\pysiglinewithargsret{\sphinxbfcode{\sphinxupquote{scale}}}{\sphinxparam{\DUrole{n,n}{factor}\DUrole{p,p}{:}\DUrole{w,w}{  }\DUrole{n,n}{float}}}{{ $\rightarrow$ None}}
\pysigstopsignatures
\sphinxAtStartPar
Scales the volume by the given factor.
\begin{quote}\begin{description}
\sphinxlineitem{Parameters}
\sphinxAtStartPar
\sphinxstyleliteralstrong{\sphinxupquote{factor}} (\sphinxstyleliteralemphasis{\sphinxupquote{float}}) \textendash{} The factor by which to scale the tool. 1 would be no scaling.

\end{description}\end{quote}

\end{fulllineitems}

\index{tip (deepdrr.vol.KWire property)@\spxentry{tip}\spxextra{deepdrr.vol.KWire property}}

\begin{fulllineitems}
\phantomsection\label{\detokenize{deepdrr.vol:deepdrr.vol.KWire.tip}}
\pysigstartsignatures
\pysigline{\sphinxbfcode{\sphinxupquote{property\DUrole{w,w}{  }}}\sphinxbfcode{\sphinxupquote{tip}}\sphinxbfcode{\sphinxupquote{\DUrole{p,p}{:}\DUrole{w,w}{  }{\hyperref[\detokenize{deepdrr.geo:deepdrr.geo.core.Point3D}]{\sphinxcrossref{Point3D}}}}}}
\pysigstopsignatures
\sphinxAtStartPar
The tip of the tool in world space.

\end{fulllineitems}

\index{tip\_in\_IJK (deepdrr.vol.KWire attribute)@\spxentry{tip\_in\_IJK}\spxextra{deepdrr.vol.KWire attribute}}

\begin{fulllineitems}
\phantomsection\label{\detokenize{deepdrr.vol:deepdrr.vol.KWire.tip_in_IJK}}
\pysigstartsignatures
\pysigline{\sphinxbfcode{\sphinxupquote{tip\_in\_IJK}}\sphinxbfcode{\sphinxupquote{\DUrole{p,p}{:}\DUrole{w,w}{  }{\hyperref[\detokenize{deepdrr.geo:deepdrr.geo.core.Point3D}]{\sphinxcrossref{Point3D}}}}}}
\pysigstopsignatures
\end{fulllineitems}

\index{tip\_in\_anatomical (deepdrr.vol.KWire property)@\spxentry{tip\_in\_anatomical}\spxextra{deepdrr.vol.KWire property}}

\begin{fulllineitems}
\phantomsection\label{\detokenize{deepdrr.vol:deepdrr.vol.KWire.tip_in_anatomical}}
\pysigstartsignatures
\pysigline{\sphinxbfcode{\sphinxupquote{property\DUrole{w,w}{  }}}\sphinxbfcode{\sphinxupquote{tip\_in\_anatomical}}\sphinxbfcode{\sphinxupquote{\DUrole{p,p}{:}\DUrole{w,w}{  }{\hyperref[\detokenize{deepdrr.geo:deepdrr.geo.core.Point3D}]{\sphinxcrossref{Point3D}}}}}}
\pysigstopsignatures
\sphinxAtStartPar
Get the location of the tool tip (the pointy end) in anatomical coordinates.

\end{fulllineitems}

\index{tip\_in\_ijk (deepdrr.vol.KWire property)@\spxentry{tip\_in\_ijk}\spxextra{deepdrr.vol.KWire property}}

\begin{fulllineitems}
\phantomsection\label{\detokenize{deepdrr.vol:deepdrr.vol.KWire.tip_in_ijk}}
\pysigstartsignatures
\pysigline{\sphinxbfcode{\sphinxupquote{property\DUrole{w,w}{  }}}\sphinxbfcode{\sphinxupquote{tip\_in\_ijk}}\sphinxbfcode{\sphinxupquote{\DUrole{p,p}{:}\DUrole{w,w}{  }{\hyperref[\detokenize{deepdrr.geo:deepdrr.geo.core.Point3D}]{\sphinxcrossref{Point3D}}}}}}
\pysigstopsignatures
\end{fulllineitems}

\index{tip\_in\_world (deepdrr.vol.KWire property)@\spxentry{tip\_in\_world}\spxextra{deepdrr.vol.KWire property}}

\begin{fulllineitems}
\phantomsection\label{\detokenize{deepdrr.vol:deepdrr.vol.KWire.tip_in_world}}
\pysigstartsignatures
\pysigline{\sphinxbfcode{\sphinxupquote{property\DUrole{w,w}{  }}}\sphinxbfcode{\sphinxupquote{tip\_in\_world}}\sphinxbfcode{\sphinxupquote{\DUrole{p,p}{:}\DUrole{w,w}{  }{\hyperref[\detokenize{deepdrr.geo:deepdrr.geo.core.Point3D}]{\sphinxcrossref{Point3D}}}}}}
\pysigstopsignatures
\sphinxAtStartPar
Get the location of the tool tip (the pointy end) in world coordinates.

\end{fulllineitems}

\index{trajectory\_in\_world (deepdrr.vol.KWire property)@\spxentry{trajectory\_in\_world}\spxextra{deepdrr.vol.KWire property}}

\begin{fulllineitems}
\phantomsection\label{\detokenize{deepdrr.vol:deepdrr.vol.KWire.trajectory_in_world}}
\pysigstartsignatures
\pysigline{\sphinxbfcode{\sphinxupquote{property\DUrole{w,w}{  }}}\sphinxbfcode{\sphinxupquote{trajectory\_in\_world}}\sphinxbfcode{\sphinxupquote{\DUrole{p,p}{:}\DUrole{w,w}{  }{\hyperref[\detokenize{deepdrr.geo:deepdrr.geo.ray.Ray3D}]{\sphinxcrossref{Ray3D}}}}}}
\pysigstopsignatures
\end{fulllineitems}

\index{twist() (deepdrr.vol.KWire method)@\spxentry{twist()}\spxextra{deepdrr.vol.KWire method}}

\begin{fulllineitems}
\phantomsection\label{\detokenize{deepdrr.vol:deepdrr.vol.KWire.twist}}
\pysigstartsignatures
\pysiglinewithargsret{\sphinxbfcode{\sphinxupquote{twist}}}{\sphinxparam{\DUrole{n,n}{angle}\DUrole{p,p}{:}\DUrole{w,w}{  }\DUrole{n,n}{float}}\sphinxparamcomma \sphinxparam{\DUrole{n,n}{degrees}\DUrole{p,p}{:}\DUrole{w,w}{  }\DUrole{n,n}{bool}\DUrole{w,w}{  }\DUrole{o,o}{=}\DUrole{w,w}{  }\DUrole{default_value}{True}}}{}
\pysigstopsignatures
\sphinxAtStartPar
Rotate the tool clockwise (when looking down on it) by \sphinxtitleref{angle}.
\begin{quote}\begin{description}
\sphinxlineitem{Parameters}\begin{itemize}
\item {} 
\sphinxAtStartPar
\sphinxstyleliteralstrong{\sphinxupquote{angle}} (\sphinxstyleliteralemphasis{\sphinxupquote{float}}) \textendash{} The angle.

\item {} 
\sphinxAtStartPar
\sphinxstyleliteralstrong{\sphinxupquote{degrees}} (\sphinxstyleliteralemphasis{\sphinxupquote{bool}}\sphinxstyleliteralemphasis{\sphinxupquote{, }}\sphinxstyleliteralemphasis{\sphinxupquote{optional}}) \textendash{} Whether \sphinxtitleref{angle} is in degrees. Defaults to True.

\end{itemize}

\end{description}\end{quote}

\end{fulllineitems}

\index{world\_from\_anatomical (deepdrr.vol.KWire attribute)@\spxentry{world\_from\_anatomical}\spxextra{deepdrr.vol.KWire attribute}}

\begin{fulllineitems}
\phantomsection\label{\detokenize{deepdrr.vol:deepdrr.vol.KWire.world_from_anatomical}}
\pysigstartsignatures
\pysigline{\sphinxbfcode{\sphinxupquote{world\_from\_anatomical}}\sphinxbfcode{\sphinxupquote{\DUrole{p,p}{:}\DUrole{w,w}{  }{\hyperref[\detokenize{deepdrr.geo:deepdrr.geo.core.FrameTransform}]{\sphinxcrossref{FrameTransform}}}}}}
\pysigstopsignatures
\end{fulllineitems}


\end{fulllineitems}

\index{MetalVolume (class in deepdrr.vol)@\spxentry{MetalVolume}\spxextra{class in deepdrr.vol}}

\begin{fulllineitems}
\phantomsection\label{\detokenize{deepdrr.vol:deepdrr.vol.MetalVolume}}
\pysigstartsignatures
\pysiglinewithargsret{\sphinxbfcode{\sphinxupquote{class\DUrole{w,w}{  }}}\sphinxcode{\sphinxupquote{deepdrr.vol.}}\sphinxbfcode{\sphinxupquote{MetalVolume}}}{\sphinxparam{\DUrole{n,n}{data}\DUrole{p,p}{:}\DUrole{w,w}{  }\DUrole{n,n}{ndarray}}\sphinxparamcomma \sphinxparam{\DUrole{n,n}{materials}\DUrole{p,p}{:}\DUrole{w,w}{  }\DUrole{n,n}{Dict\DUrole{p,p}{{[}}str\DUrole{p,p}{,}\DUrole{w,w}{  }ndarray\DUrole{p,p}{{]}}}}\sphinxparamcomma \sphinxparam{\DUrole{n,n}{anatomical\_from\_IJK}\DUrole{p,p}{:}\DUrole{w,w}{  }\DUrole{n,n}{{\hyperref[\detokenize{deepdrr.geo:deepdrr.geo.core.FrameTransform}]{\sphinxcrossref{FrameTransform}}}\DUrole{w,w}{  }\DUrole{p,p}{|}\DUrole{w,w}{  }None}\DUrole{w,w}{  }\DUrole{o,o}{=}\DUrole{w,w}{  }\DUrole{default_value}{None}}\sphinxparamcomma \sphinxparam{\DUrole{n,n}{world\_from\_anatomical}\DUrole{p,p}{:}\DUrole{w,w}{  }\DUrole{n,n}{{\hyperref[\detokenize{deepdrr.geo:deepdrr.geo.core.FrameTransform}]{\sphinxcrossref{FrameTransform}}}\DUrole{w,w}{  }\DUrole{p,p}{|}\DUrole{w,w}{  }None}\DUrole{w,w}{  }\DUrole{o,o}{=}\DUrole{w,w}{  }\DUrole{default_value}{None}}\sphinxparamcomma \sphinxparam{\DUrole{n,n}{anatomical\_coordinate\_system}\DUrole{p,p}{:}\DUrole{w,w}{  }\DUrole{n,n}{str\DUrole{w,w}{  }\DUrole{p,p}{|}\DUrole{w,w}{  }None}\DUrole{w,w}{  }\DUrole{o,o}{=}\DUrole{w,w}{  }\DUrole{default_value}{None}}\sphinxparamcomma \sphinxparam{\DUrole{n,n}{cache\_dir}\DUrole{p,p}{:}\DUrole{w,w}{  }\DUrole{n,n}{str\DUrole{w,w}{  }\DUrole{p,p}{|}\DUrole{w,w}{  }None}\DUrole{w,w}{  }\DUrole{o,o}{=}\DUrole{w,w}{  }\DUrole{default_value}{None}}\sphinxparamcomma \sphinxparam{\DUrole{n,n}{config}\DUrole{p,p}{:}\DUrole{w,w}{  }\DUrole{n,n}{Dict\DUrole{p,p}{{[}}str\DUrole{p,p}{,}\DUrole{w,w}{  }Any\DUrole{p,p}{{]}}}\DUrole{w,w}{  }\DUrole{o,o}{=}\DUrole{w,w}{  }\DUrole{default_value}{\{\}}}\sphinxparamcomma \sphinxparam{\DUrole{n,n}{anatomical\_from\_ijk}\DUrole{p,p}{:}\DUrole{w,w}{  }\DUrole{n,n}{{\hyperref[\detokenize{deepdrr.geo:deepdrr.geo.core.FrameTransform}]{\sphinxcrossref{FrameTransform}}}\DUrole{w,w}{  }\DUrole{p,p}{|}\DUrole{w,w}{  }None}\DUrole{w,w}{  }\DUrole{o,o}{=}\DUrole{w,w}{  }\DUrole{default_value}{None}}}{}
\pysigstopsignatures
\sphinxAtStartPar
Bases: {\hyperref[\detokenize{deepdrr.vol:deepdrr.vol.volume.Volume}]{\sphinxcrossref{\sphinxcode{\sphinxupquote{Volume}}}}}

\sphinxAtStartPar
Same as a volume, but with a different segmentation for the materials.
\index{anatomical\_coordinate\_system (deepdrr.vol.MetalVolume attribute)@\spxentry{anatomical\_coordinate\_system}\spxextra{deepdrr.vol.MetalVolume attribute}}

\begin{fulllineitems}
\phantomsection\label{\detokenize{deepdrr.vol:deepdrr.vol.MetalVolume.anatomical_coordinate_system}}
\pysigstartsignatures
\pysigline{\sphinxbfcode{\sphinxupquote{anatomical\_coordinate\_system}}\sphinxbfcode{\sphinxupquote{\DUrole{p,p}{:}\DUrole{w,w}{  }str\DUrole{w,w}{  }\DUrole{p,p}{|}\DUrole{w,w}{  }None}}}
\pysigstopsignatures
\end{fulllineitems}

\index{anatomical\_from\_IJK (deepdrr.vol.MetalVolume attribute)@\spxentry{anatomical\_from\_IJK}\spxextra{deepdrr.vol.MetalVolume attribute}}

\begin{fulllineitems}
\phantomsection\label{\detokenize{deepdrr.vol:deepdrr.vol.MetalVolume.anatomical_from_IJK}}
\pysigstartsignatures
\pysigline{\sphinxbfcode{\sphinxupquote{anatomical\_from\_IJK}}\sphinxbfcode{\sphinxupquote{\DUrole{p,p}{:}\DUrole{w,w}{  }{\hyperref[\detokenize{deepdrr.geo:deepdrr.geo.core.FrameTransform}]{\sphinxcrossref{FrameTransform}}}}}}
\pysigstopsignatures
\end{fulllineitems}

\index{data (deepdrr.vol.MetalVolume attribute)@\spxentry{data}\spxextra{deepdrr.vol.MetalVolume attribute}}

\begin{fulllineitems}
\phantomsection\label{\detokenize{deepdrr.vol:deepdrr.vol.MetalVolume.data}}
\pysigstartsignatures
\pysigline{\sphinxbfcode{\sphinxupquote{data}}\sphinxbfcode{\sphinxupquote{\DUrole{p,p}{:}\DUrole{w,w}{  }ndarray}}}
\pysigstopsignatures
\end{fulllineitems}

\index{materials (deepdrr.vol.MetalVolume attribute)@\spxentry{materials}\spxextra{deepdrr.vol.MetalVolume attribute}}

\begin{fulllineitems}
\phantomsection\label{\detokenize{deepdrr.vol:deepdrr.vol.MetalVolume.materials}}
\pysigstartsignatures
\pysigline{\sphinxbfcode{\sphinxupquote{materials}}\sphinxbfcode{\sphinxupquote{\DUrole{p,p}{:}\DUrole{w,w}{  }Dict\DUrole{p,p}{{[}}str\DUrole{p,p}{,}\DUrole{w,w}{  }ndarray\DUrole{p,p}{{]}}}}}
\pysigstopsignatures
\end{fulllineitems}

\index{world\_from\_anatomical (deepdrr.vol.MetalVolume attribute)@\spxentry{world\_from\_anatomical}\spxextra{deepdrr.vol.MetalVolume attribute}}

\begin{fulllineitems}
\phantomsection\label{\detokenize{deepdrr.vol:deepdrr.vol.MetalVolume.world_from_anatomical}}
\pysigstartsignatures
\pysigline{\sphinxbfcode{\sphinxupquote{world\_from\_anatomical}}\sphinxbfcode{\sphinxupquote{\DUrole{p,p}{:}\DUrole{w,w}{  }{\hyperref[\detokenize{deepdrr.geo:deepdrr.geo.core.FrameTransform}]{\sphinxcrossref{FrameTransform}}}}}}
\pysigstopsignatures
\end{fulllineitems}


\end{fulllineitems}

\index{Volume (class in deepdrr.vol)@\spxentry{Volume}\spxextra{class in deepdrr.vol}}

\begin{fulllineitems}
\phantomsection\label{\detokenize{deepdrr.vol:deepdrr.vol.Volume}}
\pysigstartsignatures
\pysiglinewithargsret{\sphinxbfcode{\sphinxupquote{class\DUrole{w,w}{  }}}\sphinxcode{\sphinxupquote{deepdrr.vol.}}\sphinxbfcode{\sphinxupquote{Volume}}}{\sphinxparam{\DUrole{n,n}{data}\DUrole{p,p}{:}\DUrole{w,w}{  }\DUrole{n,n}{ndarray}}\sphinxparamcomma \sphinxparam{\DUrole{n,n}{materials}\DUrole{p,p}{:}\DUrole{w,w}{  }\DUrole{n,n}{Dict\DUrole{p,p}{{[}}str\DUrole{p,p}{,}\DUrole{w,w}{  }ndarray\DUrole{p,p}{{]}}}}\sphinxparamcomma \sphinxparam{\DUrole{n,n}{anatomical\_from\_IJK}\DUrole{p,p}{:}\DUrole{w,w}{  }\DUrole{n,n}{{\hyperref[\detokenize{deepdrr.geo:deepdrr.geo.core.FrameTransform}]{\sphinxcrossref{FrameTransform}}}\DUrole{w,w}{  }\DUrole{p,p}{|}\DUrole{w,w}{  }None}\DUrole{w,w}{  }\DUrole{o,o}{=}\DUrole{w,w}{  }\DUrole{default_value}{None}}\sphinxparamcomma \sphinxparam{\DUrole{n,n}{world\_from\_anatomical}\DUrole{p,p}{:}\DUrole{w,w}{  }\DUrole{n,n}{{\hyperref[\detokenize{deepdrr.geo:deepdrr.geo.core.FrameTransform}]{\sphinxcrossref{FrameTransform}}}\DUrole{w,w}{  }\DUrole{p,p}{|}\DUrole{w,w}{  }None}\DUrole{w,w}{  }\DUrole{o,o}{=}\DUrole{w,w}{  }\DUrole{default_value}{None}}\sphinxparamcomma \sphinxparam{\DUrole{n,n}{anatomical\_coordinate\_system}\DUrole{p,p}{:}\DUrole{w,w}{  }\DUrole{n,n}{str\DUrole{w,w}{  }\DUrole{p,p}{|}\DUrole{w,w}{  }None}\DUrole{w,w}{  }\DUrole{o,o}{=}\DUrole{w,w}{  }\DUrole{default_value}{None}}\sphinxparamcomma \sphinxparam{\DUrole{n,n}{cache\_dir}\DUrole{p,p}{:}\DUrole{w,w}{  }\DUrole{n,n}{str\DUrole{w,w}{  }\DUrole{p,p}{|}\DUrole{w,w}{  }None}\DUrole{w,w}{  }\DUrole{o,o}{=}\DUrole{w,w}{  }\DUrole{default_value}{None}}\sphinxparamcomma \sphinxparam{\DUrole{n,n}{config}\DUrole{p,p}{:}\DUrole{w,w}{  }\DUrole{n,n}{Dict\DUrole{p,p}{{[}}str\DUrole{p,p}{,}\DUrole{w,w}{  }Any\DUrole{p,p}{{]}}}\DUrole{w,w}{  }\DUrole{o,o}{=}\DUrole{w,w}{  }\DUrole{default_value}{\{\}}}\sphinxparamcomma \sphinxparam{\DUrole{n,n}{anatomical\_from\_ijk}\DUrole{p,p}{:}\DUrole{w,w}{  }\DUrole{n,n}{{\hyperref[\detokenize{deepdrr.geo:deepdrr.geo.core.FrameTransform}]{\sphinxcrossref{FrameTransform}}}\DUrole{w,w}{  }\DUrole{p,p}{|}\DUrole{w,w}{  }None}\DUrole{w,w}{  }\DUrole{o,o}{=}\DUrole{w,w}{  }\DUrole{default_value}{None}}}{}
\pysigstopsignatures
\sphinxAtStartPar
Bases: \sphinxcode{\sphinxupquote{object}}
\index{IJK\_from\_LPS (deepdrr.vol.Volume property)@\spxentry{IJK\_from\_LPS}\spxextra{deepdrr.vol.Volume property}}

\begin{fulllineitems}
\phantomsection\label{\detokenize{deepdrr.vol:deepdrr.vol.Volume.IJK_from_LPS}}
\pysigstartsignatures
\pysigline{\sphinxbfcode{\sphinxupquote{property\DUrole{w,w}{  }}}\sphinxbfcode{\sphinxupquote{IJK\_from\_LPS}}\sphinxbfcode{\sphinxupquote{\DUrole{p,p}{:}\DUrole{w,w}{  }{\hyperref[\detokenize{deepdrr.geo:deepdrr.geo.core.FrameTransform}]{\sphinxcrossref{FrameTransform}}}}}}
\pysigstopsignatures
\end{fulllineitems}

\index{IJK\_from\_RAS (deepdrr.vol.Volume property)@\spxentry{IJK\_from\_RAS}\spxextra{deepdrr.vol.Volume property}}

\begin{fulllineitems}
\phantomsection\label{\detokenize{deepdrr.vol:deepdrr.vol.Volume.IJK_from_RAS}}
\pysigstartsignatures
\pysigline{\sphinxbfcode{\sphinxupquote{property\DUrole{w,w}{  }}}\sphinxbfcode{\sphinxupquote{IJK\_from\_RAS}}}
\pysigstopsignatures
\end{fulllineitems}

\index{IJK\_from\_anatomical (deepdrr.vol.Volume property)@\spxentry{IJK\_from\_anatomical}\spxextra{deepdrr.vol.Volume property}}

\begin{fulllineitems}
\phantomsection\label{\detokenize{deepdrr.vol:deepdrr.vol.Volume.IJK_from_anatomical}}
\pysigstartsignatures
\pysigline{\sphinxbfcode{\sphinxupquote{property\DUrole{w,w}{  }}}\sphinxbfcode{\sphinxupquote{IJK\_from\_anatomical}}}
\pysigstopsignatures
\end{fulllineitems}

\index{IJK\_from\_world (deepdrr.vol.Volume property)@\spxentry{IJK\_from\_world}\spxextra{deepdrr.vol.Volume property}}

\begin{fulllineitems}
\phantomsection\label{\detokenize{deepdrr.vol:deepdrr.vol.Volume.IJK_from_world}}
\pysigstartsignatures
\pysigline{\sphinxbfcode{\sphinxupquote{property\DUrole{w,w}{  }}}\sphinxbfcode{\sphinxupquote{IJK\_from\_world}}\sphinxbfcode{\sphinxupquote{\DUrole{p,p}{:}\DUrole{w,w}{  }{\hyperref[\detokenize{deepdrr.geo:deepdrr.geo.core.FrameTransform}]{\sphinxcrossref{FrameTransform}}}}}}
\pysigstopsignatures
\end{fulllineitems}

\index{LPS\_from\_IJK (deepdrr.vol.Volume property)@\spxentry{LPS\_from\_IJK}\spxextra{deepdrr.vol.Volume property}}

\begin{fulllineitems}
\phantomsection\label{\detokenize{deepdrr.vol:deepdrr.vol.Volume.LPS_from_IJK}}
\pysigstartsignatures
\pysigline{\sphinxbfcode{\sphinxupquote{property\DUrole{w,w}{  }}}\sphinxbfcode{\sphinxupquote{LPS\_from\_IJK}}\sphinxbfcode{\sphinxupquote{\DUrole{p,p}{:}\DUrole{w,w}{  }{\hyperref[\detokenize{deepdrr.geo:deepdrr.geo.core.FrameTransform}]{\sphinxcrossref{FrameTransform}}}}}}
\pysigstopsignatures
\sphinxAtStartPar
Get the LPS\_from\_IJK transform.

\end{fulllineitems}

\index{RAS\_from\_IJK (deepdrr.vol.Volume property)@\spxentry{RAS\_from\_IJK}\spxextra{deepdrr.vol.Volume property}}

\begin{fulllineitems}
\phantomsection\label{\detokenize{deepdrr.vol:deepdrr.vol.Volume.RAS_from_IJK}}
\pysigstartsignatures
\pysigline{\sphinxbfcode{\sphinxupquote{property\DUrole{w,w}{  }}}\sphinxbfcode{\sphinxupquote{RAS\_from\_IJK}}}
\pysigstopsignatures
\sphinxAtStartPar
Get the RAS\_from\_IJK transform.

\end{fulllineitems}

\index{anatomical\_coordinate\_system (deepdrr.vol.Volume attribute)@\spxentry{anatomical\_coordinate\_system}\spxextra{deepdrr.vol.Volume attribute}}

\begin{fulllineitems}
\phantomsection\label{\detokenize{deepdrr.vol:deepdrr.vol.Volume.anatomical_coordinate_system}}
\pysigstartsignatures
\pysigline{\sphinxbfcode{\sphinxupquote{anatomical\_coordinate\_system}}\sphinxbfcode{\sphinxupquote{\DUrole{p,p}{:}\DUrole{w,w}{  }str\DUrole{w,w}{  }\DUrole{p,p}{|}\DUrole{w,w}{  }None}}}
\pysigstopsignatures
\end{fulllineitems}

\index{anatomical\_from\_IJK (deepdrr.vol.Volume attribute)@\spxentry{anatomical\_from\_IJK}\spxextra{deepdrr.vol.Volume attribute}}

\begin{fulllineitems}
\phantomsection\label{\detokenize{deepdrr.vol:deepdrr.vol.Volume.anatomical_from_IJK}}
\pysigstartsignatures
\pysigline{\sphinxbfcode{\sphinxupquote{anatomical\_from\_IJK}}\sphinxbfcode{\sphinxupquote{\DUrole{p,p}{:}\DUrole{w,w}{  }{\hyperref[\detokenize{deepdrr.geo:deepdrr.geo.core.FrameTransform}]{\sphinxcrossref{FrameTransform}}}}}}
\pysigstopsignatures
\end{fulllineitems}

\index{anatomical\_from\_ijk (deepdrr.vol.Volume property)@\spxentry{anatomical\_from\_ijk}\spxextra{deepdrr.vol.Volume property}}

\begin{fulllineitems}
\phantomsection\label{\detokenize{deepdrr.vol:deepdrr.vol.Volume.anatomical_from_ijk}}
\pysigstartsignatures
\pysigline{\sphinxbfcode{\sphinxupquote{property\DUrole{w,w}{  }}}\sphinxbfcode{\sphinxupquote{anatomical\_from\_ijk}}\sphinxbfcode{\sphinxupquote{\DUrole{p,p}{:}\DUrole{w,w}{  }{\hyperref[\detokenize{deepdrr.geo:deepdrr.geo.core.FrameTransform}]{\sphinxcrossref{FrameTransform}}}}}}
\pysigstopsignatures
\end{fulllineitems}

\index{anatomical\_from\_world (deepdrr.vol.Volume property)@\spxentry{anatomical\_from\_world}\spxextra{deepdrr.vol.Volume property}}

\begin{fulllineitems}
\phantomsection\label{\detokenize{deepdrr.vol:deepdrr.vol.Volume.anatomical_from_world}}
\pysigstartsignatures
\pysigline{\sphinxbfcode{\sphinxupquote{property\DUrole{w,w}{  }}}\sphinxbfcode{\sphinxupquote{anatomical\_from\_world}}}
\pysigstopsignatures
\end{fulllineitems}

\index{cache\_dir (deepdrr.vol.Volume attribute)@\spxentry{cache\_dir}\spxextra{deepdrr.vol.Volume attribute}}

\begin{fulllineitems}
\phantomsection\label{\detokenize{deepdrr.vol:deepdrr.vol.Volume.cache_dir}}
\pysigstartsignatures
\pysigline{\sphinxbfcode{\sphinxupquote{cache\_dir}}\sphinxbfcode{\sphinxupquote{\DUrole{p,p}{:}\DUrole{w,w}{  }Path\DUrole{w,w}{  }\DUrole{p,p}{|}\DUrole{w,w}{  }None}}\sphinxbfcode{\sphinxupquote{\DUrole{w,w}{  }\DUrole{p,p}{=}\DUrole{w,w}{  }None}}}
\pysigstopsignatures
\end{fulllineitems}

\index{center\_in\_world (deepdrr.vol.Volume property)@\spxentry{center\_in\_world}\spxextra{deepdrr.vol.Volume property}}

\begin{fulllineitems}
\phantomsection\label{\detokenize{deepdrr.vol:deepdrr.vol.Volume.center_in_world}}
\pysigstartsignatures
\pysigline{\sphinxbfcode{\sphinxupquote{property\DUrole{w,w}{  }}}\sphinxbfcode{\sphinxupquote{center\_in\_world}}\sphinxbfcode{\sphinxupquote{\DUrole{p,p}{:}\DUrole{w,w}{  }{\hyperref[\detokenize{deepdrr.geo:deepdrr.geo.core.Point3D}]{\sphinxcrossref{Point3D}}}}}}
\pysigstopsignatures
\sphinxAtStartPar
The center of the volume in world coorindates. Useful for debugging.

\end{fulllineitems}

\index{copy\_pose() (deepdrr.vol.Volume method)@\spxentry{copy\_pose()}\spxextra{deepdrr.vol.Volume method}}

\begin{fulllineitems}
\phantomsection\label{\detokenize{deepdrr.vol:deepdrr.vol.Volume.copy_pose}}
\pysigstartsignatures
\pysiglinewithargsret{\sphinxbfcode{\sphinxupquote{copy\_pose}}}{\sphinxparam{\DUrole{n,n}{other}\DUrole{p,p}{:}\DUrole{w,w}{  }\DUrole{n,n}{{\hyperref[\detokenize{deepdrr.vol:deepdrr.vol.volume.Volume}]{\sphinxcrossref{Volume}}}}}}{{ $\rightarrow$ None}}
\pysigstopsignatures
\sphinxAtStartPar
Copy the pose of another volume.

\end{fulllineitems}

\index{crop() (deepdrr.vol.Volume method)@\spxentry{crop()}\spxextra{deepdrr.vol.Volume method}}

\begin{fulllineitems}
\phantomsection\label{\detokenize{deepdrr.vol:deepdrr.vol.Volume.crop}}
\pysigstartsignatures
\pysiglinewithargsret{\sphinxbfcode{\sphinxupquote{crop}}}{\sphinxparam{\DUrole{n,n}{crop\_box}\DUrole{p,p}{:}\DUrole{w,w}{  }\DUrole{n,n}{Tuple\DUrole{p,p}{{[}}Tuple\DUrole{p,p}{{[}}float\DUrole{p,p}{,}\DUrole{w,w}{  }float\DUrole{p,p}{{]}}\DUrole{p,p}{,}\DUrole{w,w}{  }\DUrole{p,p}{...}\DUrole{p,p}{{]}}}}}{{ $\rightarrow$ {\hyperref[\detokenize{deepdrr.vol:deepdrr.vol.volume.Volume}]{\sphinxcrossref{Volume}}}}}
\pysigstopsignatures
\sphinxAtStartPar
Crop the volume to a given bounding box.
\begin{quote}\begin{description}
\sphinxlineitem{Parameters}
\sphinxAtStartPar
\sphinxstyleliteralstrong{\sphinxupquote{crop\_box}} (\sphinxstyleliteralemphasis{\sphinxupquote{Tuple}}\sphinxstyleliteralemphasis{\sphinxupquote{{[}}}\sphinxstyleliteralemphasis{\sphinxupquote{Tuple}}\sphinxstyleliteralemphasis{\sphinxupquote{{[}}}\sphinxstyleliteralemphasis{\sphinxupquote{float}}\sphinxstyleliteralemphasis{\sphinxupquote{, }}\sphinxstyleliteralemphasis{\sphinxupquote{float}}\sphinxstyleliteralemphasis{\sphinxupquote{{]}}}\sphinxstyleliteralemphasis{\sphinxupquote{, }}\sphinxstyleliteralemphasis{\sphinxupquote{...}}\sphinxstyleliteralemphasis{\sphinxupquote{{]}}}) \textendash{} The bounding box to crop to, in IJK.

\sphinxlineitem{Returns}
\sphinxAtStartPar
The cropped volume.

\sphinxlineitem{Return type}
\sphinxAtStartPar
{\hyperref[\detokenize{deepdrr.vol:deepdrr.vol.Volume}]{\sphinxcrossref{Volume}}}

\end{description}\end{quote}

\end{fulllineitems}

\index{data (deepdrr.vol.Volume attribute)@\spxentry{data}\spxextra{deepdrr.vol.Volume attribute}}

\begin{fulllineitems}
\phantomsection\label{\detokenize{deepdrr.vol:deepdrr.vol.Volume.data}}
\pysigstartsignatures
\pysigline{\sphinxbfcode{\sphinxupquote{data}}\sphinxbfcode{\sphinxupquote{\DUrole{p,p}{:}\DUrole{w,w}{  }ndarray}}}
\pysigstopsignatures
\end{fulllineitems}

\index{facedown() (deepdrr.vol.Volume method)@\spxentry{facedown()}\spxextra{deepdrr.vol.Volume method}}

\begin{fulllineitems}
\phantomsection\label{\detokenize{deepdrr.vol:deepdrr.vol.Volume.facedown}}
\pysigstartsignatures
\pysiglinewithargsret{\sphinxbfcode{\sphinxupquote{facedown}}}{}{}
\pysigstopsignatures
\sphinxAtStartPar
Turns the volume to be face down.

\sphinxAtStartPar
This aligns the patient so that, in world space,
the posterior side is toward +Z, inferior is toward +X,
and right is toward +Y.
\begin{quote}\begin{description}
\sphinxlineitem{Raises}
\sphinxAtStartPar
\sphinxstyleliteralstrong{\sphinxupquote{NotImplementedError}} \textendash{} If the anatomical coordinate system is not “RAS”.

\end{description}\end{quote}

\end{fulllineitems}

\index{faceup() (deepdrr.vol.Volume method)@\spxentry{faceup()}\spxextra{deepdrr.vol.Volume method}}

\begin{fulllineitems}
\phantomsection\label{\detokenize{deepdrr.vol:deepdrr.vol.Volume.faceup}}
\pysigstartsignatures
\pysiglinewithargsret{\sphinxbfcode{\sphinxupquote{faceup}}}{}{}
\pysigstopsignatures
\sphinxAtStartPar
Turns the volume to be face up.

\sphinxAtStartPar
This aligns the patient so that, in world space,
the anterior side is toward +Z, inferior is toward +X,
and left is toward +Y.
\begin{quote}\begin{description}
\sphinxlineitem{Raises}
\sphinxAtStartPar
\sphinxstyleliteralstrong{\sphinxupquote{NotImplementedError}} \textendash{} If the anatomical coordinate system is not “RAS”.

\end{description}\end{quote}

\end{fulllineitems}

\index{from\_dicom() (deepdrr.vol.Volume class method)@\spxentry{from\_dicom()}\spxextra{deepdrr.vol.Volume class method}}

\begin{fulllineitems}
\phantomsection\label{\detokenize{deepdrr.vol:deepdrr.vol.Volume.from_dicom}}
\pysigstartsignatures
\pysiglinewithargsret{\sphinxbfcode{\sphinxupquote{classmethod\DUrole{w,w}{  }}}\sphinxbfcode{\sphinxupquote{from\_dicom}}}{\sphinxparam{\DUrole{n,n}{path}\DUrole{p,p}{:}\DUrole{w,w}{  }\DUrole{n,n}{Path}}\sphinxparamcomma \sphinxparam{\DUrole{n,n}{use\_thresholding}\DUrole{p,p}{:}\DUrole{w,w}{  }\DUrole{n,n}{bool}\DUrole{w,w}{  }\DUrole{o,o}{=}\DUrole{w,w}{  }\DUrole{default_value}{True}}\sphinxparamcomma \sphinxparam{\DUrole{n,n}{world\_from\_anatomical}\DUrole{p,p}{:}\DUrole{w,w}{  }\DUrole{n,n}{{\hyperref[\detokenize{deepdrr.geo:deepdrr.geo.core.FrameTransform}]{\sphinxcrossref{FrameTransform}}}\DUrole{w,w}{  }\DUrole{p,p}{|}\DUrole{w,w}{  }None}\DUrole{w,w}{  }\DUrole{o,o}{=}\DUrole{w,w}{  }\DUrole{default_value}{None}}\sphinxparamcomma \sphinxparam{\DUrole{n,n}{use\_cached}\DUrole{p,p}{:}\DUrole{w,w}{  }\DUrole{n,n}{bool}\DUrole{w,w}{  }\DUrole{o,o}{=}\DUrole{w,w}{  }\DUrole{default_value}{True}}\sphinxparamcomma \sphinxparam{\DUrole{n,n}{cache\_dir}\DUrole{p,p}{:}\DUrole{w,w}{  }\DUrole{n,n}{Path\DUrole{w,w}{  }\DUrole{p,p}{|}\DUrole{w,w}{  }None}\DUrole{w,w}{  }\DUrole{o,o}{=}\DUrole{w,w}{  }\DUrole{default_value}{None}}\sphinxparamcomma \sphinxparam{\DUrole{o,o}{**}\DUrole{n,n}{kwargs}}}{}
\pysigstopsignatures
\sphinxAtStartPar
load a volume from a dicom file and compute the anatomical\_from\_ijk transform from metadata
\sphinxurl{https://www.slicer.org/wiki/Coordinate\_systems}
:param path: path\sphinxhyphen{}like to a multi\sphinxhyphen{}frame dicom file. (Currently only Multi\sphinxhyphen{}Frame from Siemens supported)
:param use\_thresholding: segment the materials using thresholding (faster but less accurate). Defaults to True.
:type use\_thresholding: bool, optional
:param world\_from\_anatomical: position the volume in world space. If None, uses identity. Defaults to None.
:type world\_from\_anatomical: Optional{[}geo.FrameTransform{]}, optional
:param use\_cached: {[}description{]}. Use a cached segmentation if available. Defaults to True.
:type use\_cached: bool, optional
:param cache\_dir: Where to load/save the cached segmentation. If None, use the parent dir of \sphinxtitleref{path}. Defaults to None.
:type cache\_dir: Optional{[}Path{]}, optional
\begin{quote}\begin{description}
\sphinxlineitem{Returns}
\sphinxAtStartPar
an instance of a deepdrr volume

\sphinxlineitem{Return type}
\sphinxAtStartPar
{\hyperref[\detokenize{deepdrr.vol:deepdrr.vol.Volume}]{\sphinxcrossref{Volume}}}

\end{description}\end{quote}

\end{fulllineitems}

\index{from\_hu() (deepdrr.vol.Volume class method)@\spxentry{from\_hu()}\spxextra{deepdrr.vol.Volume class method}}

\begin{fulllineitems}
\phantomsection\label{\detokenize{deepdrr.vol:deepdrr.vol.Volume.from_hu}}
\pysigstartsignatures
\pysiglinewithargsret{\sphinxbfcode{\sphinxupquote{classmethod\DUrole{w,w}{  }}}\sphinxbfcode{\sphinxupquote{from\_hu}}}{\sphinxparam{\DUrole{n,n}{hu\_values}\DUrole{p,p}{:}\DUrole{w,w}{  }\DUrole{n,n}{ndarray}}\sphinxparamcomma \sphinxparam{\DUrole{n,n}{origin}\DUrole{p,p}{:}\DUrole{w,w}{  }\DUrole{n,n}{{\hyperref[\detokenize{deepdrr.geo:deepdrr.geo.core.Point3D}]{\sphinxcrossref{Point3D}}}}}\sphinxparamcomma \sphinxparam{\DUrole{n,n}{use\_thresholding}\DUrole{p,p}{:}\DUrole{w,w}{  }\DUrole{n,n}{bool}\DUrole{w,w}{  }\DUrole{o,o}{=}\DUrole{w,w}{  }\DUrole{default_value}{True}}\sphinxparamcomma \sphinxparam{\DUrole{n,n}{spacing}\DUrole{p,p}{:}\DUrole{w,w}{  }\DUrole{n,n}{{\hyperref[\detokenize{deepdrr.geo:deepdrr.geo.core.Vector3D}]{\sphinxcrossref{Vector3D}}}\DUrole{w,w}{  }\DUrole{p,p}{|}\DUrole{w,w}{  }None}\DUrole{w,w}{  }\DUrole{o,o}{=}\DUrole{w,w}{  }\DUrole{default_value}{(1, 1, 1)}}\sphinxparamcomma \sphinxparam{\DUrole{n,n}{anatomical\_coordinate\_system}\DUrole{p,p}{:}\DUrole{w,w}{  }\DUrole{n,n}{str\DUrole{w,w}{  }\DUrole{p,p}{|}\DUrole{w,w}{  }None}\DUrole{w,w}{  }\DUrole{o,o}{=}\DUrole{w,w}{  }\DUrole{default_value}{None}}\sphinxparamcomma \sphinxparam{\DUrole{n,n}{world\_from\_anatomical}\DUrole{p,p}{:}\DUrole{w,w}{  }\DUrole{n,n}{{\hyperref[\detokenize{deepdrr.geo:deepdrr.geo.core.FrameTransform}]{\sphinxcrossref{FrameTransform}}}\DUrole{w,w}{  }\DUrole{p,p}{|}\DUrole{w,w}{  }None}\DUrole{w,w}{  }\DUrole{o,o}{=}\DUrole{w,w}{  }\DUrole{default_value}{None}}\sphinxparamcomma \sphinxparam{\DUrole{o,o}{**}\DUrole{n,n}{kwargs}}}{{ $\rightarrow$ None}}
\pysigstopsignatures
\end{fulllineitems}

\index{from\_nifti() (deepdrr.vol.Volume class method)@\spxentry{from\_nifti()}\spxextra{deepdrr.vol.Volume class method}}

\begin{fulllineitems}
\phantomsection\label{\detokenize{deepdrr.vol:deepdrr.vol.Volume.from_nifti}}
\pysigstartsignatures
\pysiglinewithargsret{\sphinxbfcode{\sphinxupquote{classmethod\DUrole{w,w}{  }}}\sphinxbfcode{\sphinxupquote{from\_nifti}}}{\sphinxparam{\DUrole{n,n}{path}\DUrole{p,p}{:}\DUrole{w,w}{  }\DUrole{n,n}{Path}}\sphinxparamcomma \sphinxparam{\DUrole{n,n}{world\_from\_anatomical}\DUrole{p,p}{:}\DUrole{w,w}{  }\DUrole{n,n}{{\hyperref[\detokenize{deepdrr.geo:deepdrr.geo.core.FrameTransform}]{\sphinxcrossref{FrameTransform}}}\DUrole{w,w}{  }\DUrole{p,p}{|}\DUrole{w,w}{  }None}\DUrole{w,w}{  }\DUrole{o,o}{=}\DUrole{w,w}{  }\DUrole{default_value}{None}}\sphinxparamcomma \sphinxparam{\DUrole{n,n}{use\_thresholding}\DUrole{p,p}{:}\DUrole{w,w}{  }\DUrole{n,n}{bool}\DUrole{w,w}{  }\DUrole{o,o}{=}\DUrole{w,w}{  }\DUrole{default_value}{True}}\sphinxparamcomma \sphinxparam{\DUrole{n,n}{use\_cached}\DUrole{p,p}{:}\DUrole{w,w}{  }\DUrole{n,n}{bool}\DUrole{w,w}{  }\DUrole{o,o}{=}\DUrole{w,w}{  }\DUrole{default_value}{True}}\sphinxparamcomma \sphinxparam{\DUrole{n,n}{save\_cache}\DUrole{p,p}{:}\DUrole{w,w}{  }\DUrole{n,n}{bool}\DUrole{w,w}{  }\DUrole{o,o}{=}\DUrole{w,w}{  }\DUrole{default_value}{False}}\sphinxparamcomma \sphinxparam{\DUrole{n,n}{cache\_dir}\DUrole{p,p}{:}\DUrole{w,w}{  }\DUrole{n,n}{Path\DUrole{w,w}{  }\DUrole{p,p}{|}\DUrole{w,w}{  }None}\DUrole{w,w}{  }\DUrole{o,o}{=}\DUrole{w,w}{  }\DUrole{default_value}{None}}\sphinxparamcomma \sphinxparam{\DUrole{n,n}{materials}\DUrole{p,p}{:}\DUrole{w,w}{  }\DUrole{n,n}{Dict\DUrole{p,p}{{[}}str\DUrole{p,p}{,}\DUrole{w,w}{  }ndarray\DUrole{p,p}{{]}}\DUrole{w,w}{  }\DUrole{p,p}{|}\DUrole{w,w}{  }None}\DUrole{w,w}{  }\DUrole{o,o}{=}\DUrole{w,w}{  }\DUrole{default_value}{None}}\sphinxparamcomma \sphinxparam{\DUrole{n,n}{segmentation}\DUrole{p,p}{:}\DUrole{w,w}{  }\DUrole{n,n}{bool}\DUrole{w,w}{  }\DUrole{o,o}{=}\DUrole{w,w}{  }\DUrole{default_value}{False}}\sphinxparamcomma \sphinxparam{\DUrole{n,n}{label}\DUrole{p,p}{:}\DUrole{w,w}{  }\DUrole{n,n}{None\DUrole{w,w}{  }\DUrole{p,p}{|}\DUrole{w,w}{  }int\DUrole{w,w}{  }\DUrole{p,p}{|}\DUrole{w,w}{  }List\DUrole{p,p}{{[}}int\DUrole{p,p}{{]}}}\DUrole{w,w}{  }\DUrole{o,o}{=}\DUrole{w,w}{  }\DUrole{default_value}{None}}\sphinxparamcomma \sphinxparam{\DUrole{n,n}{density\_kwargs}\DUrole{p,p}{:}\DUrole{w,w}{  }\DUrole{n,n}{dict}\DUrole{w,w}{  }\DUrole{o,o}{=}\DUrole{w,w}{  }\DUrole{default_value}{\{\}}}\sphinxparamcomma \sphinxparam{\DUrole{o,o}{**}\DUrole{n,n}{kwargs}}}{}
\pysigstopsignatures
\sphinxAtStartPar
Load a volume from NiFti file.
\begin{quote}\begin{description}
\sphinxlineitem{Parameters}\begin{itemize}
\item {} 
\sphinxAtStartPar
\sphinxstyleliteralstrong{\sphinxupquote{path}} (\sphinxstyleliteralemphasis{\sphinxupquote{Path}}) \textendash{} path to the .nii.gz file.

\item {} 
\sphinxAtStartPar
\sphinxstyleliteralstrong{\sphinxupquote{use\_thresholding}} (\sphinxstyleliteralemphasis{\sphinxupquote{bool}}\sphinxstyleliteralemphasis{\sphinxupquote{, }}\sphinxstyleliteralemphasis{\sphinxupquote{optional}}) \textendash{} segment the materials using thresholding (faster but less accurate). Defaults to True.

\item {} 
\sphinxAtStartPar
\sphinxstyleliteralstrong{\sphinxupquote{world\_from\_anatomical}} (\sphinxstyleliteralemphasis{\sphinxupquote{Optional}}\sphinxstyleliteralemphasis{\sphinxupquote{{[}}}{\hyperref[\detokenize{deepdrr.geo:deepdrr.geo.FrameTransform}]{\sphinxcrossref{\sphinxstyleliteralemphasis{\sphinxupquote{geo.FrameTransform}}}}}\sphinxstyleliteralemphasis{\sphinxupquote{{]}}}\sphinxstyleliteralemphasis{\sphinxupquote{, }}\sphinxstyleliteralemphasis{\sphinxupquote{optional}}) \textendash{} position the volume in world space. If None, uses identity. Defaults to None.

\item {} 
\sphinxAtStartPar
\sphinxstyleliteralstrong{\sphinxupquote{use\_cached}} (\sphinxstyleliteralemphasis{\sphinxupquote{bool}}\sphinxstyleliteralemphasis{\sphinxupquote{, }}\sphinxstyleliteralemphasis{\sphinxupquote{optional}}) \textendash{} Use a cached segmentation if available. Defaults to True.

\item {} 
\sphinxAtStartPar
\sphinxstyleliteralstrong{\sphinxupquote{cache\_dir}} (\sphinxstyleliteralemphasis{\sphinxupquote{Optional}}\sphinxstyleliteralemphasis{\sphinxupquote{{[}}}\sphinxstyleliteralemphasis{\sphinxupquote{Path}}\sphinxstyleliteralemphasis{\sphinxupquote{{]}}}\sphinxstyleliteralemphasis{\sphinxupquote{, }}\sphinxstyleliteralemphasis{\sphinxupquote{optional}}) \textendash{} Where to load/save the cached segmentation. If None, use a “cache” directory
in the same location as the nifti file. Defaults to None.

\item {} 
\sphinxAtStartPar
\sphinxstyleliteralstrong{\sphinxupquote{materials}} \textendash{} Optional material segmentation, as a dictionary mapping material name to binary segmentation.
If not provided, materials are segmented from the CT. Defaults to None.
Can also provide a dictionary mapping material names to Nifti files containing the segmentations.

\item {} 
\sphinxAtStartPar
\sphinxstyleliteralstrong{\sphinxupquote{segmentation}} (\sphinxstyleliteralemphasis{\sphinxupquote{bool}}\sphinxstyleliteralemphasis{\sphinxupquote{, }}\sphinxstyleliteralemphasis{\sphinxupquote{optional}}\sphinxstyleliteralemphasis{\sphinxupquote{) }}\sphinxstyleliteralemphasis{\sphinxupquote{If the file is a segmentation file}}\sphinxstyleliteralemphasis{\sphinxupquote{, }}\sphinxstyleliteralemphasis{\sphinxupquote{then its "materials" correspond to a high density material}}\sphinxstyleliteralemphasis{\sphinxupquote{ (}}\sphinxstyleliteralemphasis{\sphinxupquote{bone}}) \textendash{} where the values are \textgreater{}0. Defaults to false. Overrides provided materials.

\item {} 
\sphinxAtStartPar
\sphinxstyleliteralstrong{\sphinxupquote{label}} \textendash{} which labels to treat as solid. If None, then all nonzero labels are treated as solid. Defaults to None.

\item {} 
\sphinxAtStartPar
\sphinxstyleliteralstrong{\sphinxupquote{density\_kwargs}} \textendash{} Additional kwargs passed to convert\_hounsfield\_to\_density.

\end{itemize}

\sphinxlineitem{Returns}
\sphinxAtStartPar
A new volume object.

\sphinxlineitem{Return type}
\sphinxAtStartPar
{\hyperref[\detokenize{deepdrr.vol:deepdrr.vol.Volume}]{\sphinxcrossref{Volume}}}

\end{description}\end{quote}

\end{fulllineitems}

\index{from\_nrrd() (deepdrr.vol.Volume class method)@\spxentry{from\_nrrd()}\spxextra{deepdrr.vol.Volume class method}}

\begin{fulllineitems}
\phantomsection\label{\detokenize{deepdrr.vol:deepdrr.vol.Volume.from_nrrd}}
\pysigstartsignatures
\pysiglinewithargsret{\sphinxbfcode{\sphinxupquote{classmethod\DUrole{w,w}{  }}}\sphinxbfcode{\sphinxupquote{from\_nrrd}}}{\sphinxparam{\DUrole{n,n}{path}\DUrole{p,p}{:}\DUrole{w,w}{  }\DUrole{n,n}{str}}\sphinxparamcomma \sphinxparam{\DUrole{n,n}{world\_from\_anatomical}\DUrole{p,p}{:}\DUrole{w,w}{  }\DUrole{n,n}{{\hyperref[\detokenize{deepdrr.geo:deepdrr.geo.core.FrameTransform}]{\sphinxcrossref{FrameTransform}}}\DUrole{w,w}{  }\DUrole{p,p}{|}\DUrole{w,w}{  }None}\DUrole{w,w}{  }\DUrole{o,o}{=}\DUrole{w,w}{  }\DUrole{default_value}{None}}\sphinxparamcomma \sphinxparam{\DUrole{n,n}{use\_thresholding}\DUrole{p,p}{:}\DUrole{w,w}{  }\DUrole{n,n}{bool}\DUrole{w,w}{  }\DUrole{o,o}{=}\DUrole{w,w}{  }\DUrole{default_value}{True}}\sphinxparamcomma \sphinxparam{\DUrole{n,n}{use\_cached}\DUrole{p,p}{:}\DUrole{w,w}{  }\DUrole{n,n}{bool}\DUrole{w,w}{  }\DUrole{o,o}{=}\DUrole{w,w}{  }\DUrole{default_value}{True}}\sphinxparamcomma \sphinxparam{\DUrole{n,n}{cache\_dir}\DUrole{p,p}{:}\DUrole{w,w}{  }\DUrole{n,n}{Path\DUrole{w,w}{  }\DUrole{p,p}{|}\DUrole{w,w}{  }None}\DUrole{w,w}{  }\DUrole{o,o}{=}\DUrole{w,w}{  }\DUrole{default_value}{None}}\sphinxparamcomma \sphinxparam{\DUrole{o,o}{**}\DUrole{n,n}{kwargs}}}{}
\pysigstopsignatures
\sphinxAtStartPar
Load a volume from a nrrd file.
\begin{quote}\begin{description}
\sphinxlineitem{Parameters}\begin{itemize}
\item {} 
\sphinxAtStartPar
\sphinxstyleliteralstrong{\sphinxupquote{path}} (\sphinxstyleliteralemphasis{\sphinxupquote{str}}) \textendash{} path to the file.

\item {} 
\sphinxAtStartPar
\sphinxstyleliteralstrong{\sphinxupquote{use\_thresholding}} (\sphinxstyleliteralemphasis{\sphinxupquote{bool}}\sphinxstyleliteralemphasis{\sphinxupquote{, }}\sphinxstyleliteralemphasis{\sphinxupquote{optional}}) \textendash{} segment the materials using thresholding (faster but less accurate). Defaults to True.

\item {} 
\sphinxAtStartPar
\sphinxstyleliteralstrong{\sphinxupquote{world\_from\_anatomical}} (\sphinxstyleliteralemphasis{\sphinxupquote{Optional}}\sphinxstyleliteralemphasis{\sphinxupquote{{[}}}{\hyperref[\detokenize{deepdrr.geo:deepdrr.geo.FrameTransform}]{\sphinxcrossref{\sphinxstyleliteralemphasis{\sphinxupquote{geo.FrameTransform}}}}}\sphinxstyleliteralemphasis{\sphinxupquote{{]}}}\sphinxstyleliteralemphasis{\sphinxupquote{, }}\sphinxstyleliteralemphasis{\sphinxupquote{optional}}) \textendash{} position the volume in world space. If None, uses identity. Defaults to None.

\item {} 
\sphinxAtStartPar
\sphinxstyleliteralstrong{\sphinxupquote{use\_cached}} (\sphinxstyleliteralemphasis{\sphinxupquote{bool}}\sphinxstyleliteralemphasis{\sphinxupquote{, }}\sphinxstyleliteralemphasis{\sphinxupquote{optional}}) \textendash{} Use a cached segmentation if available. Defaults to True.

\item {} 
\sphinxAtStartPar
\sphinxstyleliteralstrong{\sphinxupquote{cache\_dir}} (\sphinxstyleliteralemphasis{\sphinxupquote{Optional}}\sphinxstyleliteralemphasis{\sphinxupquote{{[}}}\sphinxstyleliteralemphasis{\sphinxupquote{Path}}\sphinxstyleliteralemphasis{\sphinxupquote{{]}}}\sphinxstyleliteralemphasis{\sphinxupquote{, }}\sphinxstyleliteralemphasis{\sphinxupquote{optional}}) \textendash{} Where to load/save the cached segmentation. If None, use the parent dir of \sphinxtitleref{path}. Defaults to None.

\end{itemize}

\sphinxlineitem{Returns}
\sphinxAtStartPar
A volume formed from the NRRD.

\sphinxlineitem{Return type}
\sphinxAtStartPar
{\hyperref[\detokenize{deepdrr.vol:deepdrr.vol.Volume}]{\sphinxcrossref{Volume}}}

\end{description}\end{quote}

\end{fulllineitems}

\index{from\_parameters() (deepdrr.vol.Volume class method)@\spxentry{from\_parameters()}\spxextra{deepdrr.vol.Volume class method}}

\begin{fulllineitems}
\phantomsection\label{\detokenize{deepdrr.vol:deepdrr.vol.Volume.from_parameters}}
\pysigstartsignatures
\pysiglinewithargsret{\sphinxbfcode{\sphinxupquote{classmethod\DUrole{w,w}{  }}}\sphinxbfcode{\sphinxupquote{from\_parameters}}}{\sphinxparam{\DUrole{n,n}{data}\DUrole{p,p}{:}\DUrole{w,w}{  }\DUrole{n,n}{ndarray}}\sphinxparamcomma \sphinxparam{\DUrole{n,n}{materials}\DUrole{p,p}{:}\DUrole{w,w}{  }\DUrole{n,n}{Dict\DUrole{p,p}{{[}}str\DUrole{p,p}{,}\DUrole{w,w}{  }ndarray\DUrole{p,p}{{]}}}}\sphinxparamcomma \sphinxparam{\DUrole{n,n}{origin}\DUrole{p,p}{:}\DUrole{w,w}{  }\DUrole{n,n}{{\hyperref[\detokenize{deepdrr.geo:deepdrr.geo.core.Point3D}]{\sphinxcrossref{Point3D}}}}}\sphinxparamcomma \sphinxparam{\DUrole{n,n}{spacing}\DUrole{p,p}{:}\DUrole{w,w}{  }\DUrole{n,n}{{\hyperref[\detokenize{deepdrr.geo:deepdrr.geo.core.Vector3D}]{\sphinxcrossref{Vector3D}}}\DUrole{w,w}{  }\DUrole{p,p}{|}\DUrole{w,w}{  }None}\DUrole{w,w}{  }\DUrole{o,o}{=}\DUrole{w,w}{  }\DUrole{default_value}{{[}1, 1, 1{]}}}\sphinxparamcomma \sphinxparam{\DUrole{n,n}{anatomical\_coordinate\_system}\DUrole{p,p}{:}\DUrole{w,w}{  }\DUrole{n,n}{str\DUrole{w,w}{  }\DUrole{p,p}{|}\DUrole{w,w}{  }None}\DUrole{w,w}{  }\DUrole{o,o}{=}\DUrole{w,w}{  }\DUrole{default_value}{None}}\sphinxparamcomma \sphinxparam{\DUrole{n,n}{world\_from\_anatomical}\DUrole{p,p}{:}\DUrole{w,w}{  }\DUrole{n,n}{{\hyperref[\detokenize{deepdrr.geo:deepdrr.geo.core.FrameTransform}]{\sphinxcrossref{FrameTransform}}}\DUrole{w,w}{  }\DUrole{p,p}{|}\DUrole{w,w}{  }None}\DUrole{w,w}{  }\DUrole{o,o}{=}\DUrole{w,w}{  }\DUrole{default_value}{None}}\sphinxparamcomma \sphinxparam{\DUrole{o,o}{**}\DUrole{n,n}{kwargs}}}{}
\pysigstopsignatures
\sphinxAtStartPar
Create a volume object with a segmentation of the materials, from parameters.

\sphinxAtStartPar
Note that the anatomical coordinate system is not the world coordinate system (which is cartesian).

\sphinxAtStartPar
Suggested anatomical coordinate space units is millimeters.
A helpful introduction to the geometry is can be found {[}here{]}(\sphinxurl{https://www.slicer.org/wiki/Coordinate\_systems}).
\begin{quote}\begin{description}
\sphinxlineitem{Parameters}\begin{itemize}
\item {} 
\sphinxAtStartPar
\sphinxstyleliteralstrong{\sphinxupquote{volume}} (\sphinxstyleliteralemphasis{\sphinxupquote{np.ndarray}}) \textendash{} the volume density data.

\item {} 
\sphinxAtStartPar
\sphinxstyleliteralstrong{\sphinxupquote{materials}} (\sphinxstyleliteralemphasis{\sphinxupquote{Dict}}\sphinxstyleliteralemphasis{\sphinxupquote{{[}}}\sphinxstyleliteralemphasis{\sphinxupquote{str}}\sphinxstyleliteralemphasis{\sphinxupquote{, }}\sphinxstyleliteralemphasis{\sphinxupquote{np.ndarray}}\sphinxstyleliteralemphasis{\sphinxupquote{{]}}}) \textendash{} mapping from material names to binary segmentation of that material.

\item {} 
\sphinxAtStartPar
\sphinxstyleliteralstrong{\sphinxupquote{origin}} ({\hyperref[\detokenize{deepdrr.geo:deepdrr.geo.core.Point3D}]{\sphinxcrossref{\sphinxstyleliteralemphasis{\sphinxupquote{Point3D}}}}}) \textendash{} Location of the volume’s origin in the anatomical coordinate system.

\item {} 
\sphinxAtStartPar
\sphinxstyleliteralstrong{\sphinxupquote{spacing}} (\sphinxstyleliteralemphasis{\sphinxupquote{Tuple}}\sphinxstyleliteralemphasis{\sphinxupquote{{[}}}\sphinxstyleliteralemphasis{\sphinxupquote{float}}\sphinxstyleliteralemphasis{\sphinxupquote{, }}\sphinxstyleliteralemphasis{\sphinxupquote{float}}\sphinxstyleliteralemphasis{\sphinxupquote{, }}\sphinxstyleliteralemphasis{\sphinxupquote{float}}\sphinxstyleliteralemphasis{\sphinxupquote{{]}}}\sphinxstyleliteralemphasis{\sphinxupquote{, }}\sphinxstyleliteralemphasis{\sphinxupquote{optional}}) \textendash{} Spacing of the volume in the anatomical coordinate system. Defaults to (1, 1, 1).

\item {} 
\sphinxAtStartPar
\sphinxstyleliteralstrong{\sphinxupquote{anatomical\_coordinate\_system}} (\sphinxstyleliteralemphasis{\sphinxupquote{Optional}}\sphinxstyleliteralemphasis{\sphinxupquote{{[}}}\sphinxstyleliteralemphasis{\sphinxupquote{str}}\sphinxstyleliteralemphasis{\sphinxupquote{{]}}}) \textendash{} anatomical coordinate system convention, either “RAS” or “LPS”. Defaults to None.

\item {} 
\sphinxAtStartPar
\sphinxstyleliteralstrong{\sphinxupquote{world\_from\_anatomical}} ({\hyperref[\detokenize{deepdrr.geo:deepdrr.geo.core.FrameTransform}]{\sphinxcrossref{\sphinxstyleliteralemphasis{\sphinxupquote{FrameTransform}}}}}\sphinxstyleliteralemphasis{\sphinxupquote{, }}\sphinxstyleliteralemphasis{\sphinxupquote{optional}}) \textendash{} Optional transformation from anatomical to world coordinates.
If None, then identity is used. Defaults to None.

\end{itemize}

\end{description}\end{quote}

\end{fulllineitems}

\index{get\_bbox\_IJK() (deepdrr.vol.Volume method)@\spxentry{get\_bbox\_IJK()}\spxextra{deepdrr.vol.Volume method}}

\begin{fulllineitems}
\phantomsection\label{\detokenize{deepdrr.vol:deepdrr.vol.Volume.get_bbox_IJK}}
\pysigstartsignatures
\pysiglinewithargsret{\sphinxbfcode{\sphinxupquote{get\_bbox\_IJK}}}{}{{ $\rightarrow$ ndarray\DUrole{w,w}{  }\DUrole{p,p}{|}\DUrole{w,w}{  }None}}
\pysigstopsignatures
\sphinxAtStartPar
Get the bounding box of the materials in IJK.
\begin{quote}\begin{description}
\sphinxlineitem{Returns}
\sphinxAtStartPar
The bounding box as a {[}3, 2{]} array.
None, if the volume is empty.

\sphinxlineitem{Return type}
\sphinxAtStartPar
np.ndarray

\end{description}\end{quote}

\end{fulllineitems}

\index{get\_bounding\_box\_in\_world() (deepdrr.vol.Volume method)@\spxentry{get\_bounding\_box\_in\_world()}\spxextra{deepdrr.vol.Volume method}}

\begin{fulllineitems}
\phantomsection\label{\detokenize{deepdrr.vol:deepdrr.vol.Volume.get_bounding_box_in_world}}
\pysigstartsignatures
\pysiglinewithargsret{\sphinxbfcode{\sphinxupquote{get\_bounding\_box\_in\_world}}}{}{{ $\rightarrow$ Tuple\DUrole{p,p}{{[}}{\hyperref[\detokenize{deepdrr.geo:deepdrr.geo.core.Point3D}]{\sphinxcrossref{Point3D}}}\DUrole{p,p}{,}\DUrole{w,w}{  }{\hyperref[\detokenize{deepdrr.geo:deepdrr.geo.core.Point3D}]{\sphinxcrossref{Point3D}}}\DUrole{p,p}{{]}}}}
\pysigstopsignatures
\sphinxAtStartPar
Get the corners of a bounding box enclosing the volume in world coordinates.

\sphinxAtStartPar
Assumes cell\sphinxhyphen{}centered sampling.
\begin{quote}\begin{description}
\sphinxlineitem{Returns}
\sphinxAtStartPar
The lower corner of the bounding box.
geo.Point3D: The upper corner of the bounding box.

\sphinxlineitem{Return type}
\sphinxAtStartPar
{\hyperref[\detokenize{deepdrr.geo:deepdrr.geo.Point3D}]{\sphinxcrossref{geo.Point3D}}}

\end{description}\end{quote}

\end{fulllineitems}

\index{get\_config() (deepdrr.vol.Volume method)@\spxentry{get\_config()}\spxextra{deepdrr.vol.Volume method}}

\begin{fulllineitems}
\phantomsection\label{\detokenize{deepdrr.vol:deepdrr.vol.Volume.get_config}}
\pysigstartsignatures
\pysiglinewithargsret{\sphinxbfcode{\sphinxupquote{get\_config}}}{}{{ $\rightarrow$ Dict\DUrole{p,p}{{[}}str\DUrole{p,p}{,}\DUrole{w,w}{  }Any\DUrole{p,p}{{]}}}}
\pysigstopsignatures
\sphinxAtStartPar
Get the configuration of the volume. Does not include volumetric data.

\sphinxAtStartPar
Includes any info passed into \sphinxtitleref{config}.
\begin{quote}\begin{description}
\sphinxlineitem{Returns}
\sphinxAtStartPar
The configuration of the volume.

\sphinxlineitem{Return type}
\sphinxAtStartPar
Dict{[}str, Any{]}

\end{description}\end{quote}

\end{fulllineitems}

\index{get\_mesh\_in\_world() (deepdrr.vol.Volume method)@\spxentry{get\_mesh\_in\_world()}\spxextra{deepdrr.vol.Volume method}}

\begin{fulllineitems}
\phantomsection\label{\detokenize{deepdrr.vol:deepdrr.vol.Volume.get_mesh_in_world}}
\pysigstartsignatures
\pysiglinewithargsret{\sphinxbfcode{\sphinxupquote{get\_mesh\_in\_world}}}{\sphinxparam{\DUrole{n,n}{full}\DUrole{p,p}{:}\DUrole{w,w}{  }\DUrole{n,n}{bool}\DUrole{w,w}{  }\DUrole{o,o}{=}\DUrole{w,w}{  }\DUrole{default_value}{False}}\sphinxparamcomma \sphinxparam{\DUrole{n,n}{use\_cached}\DUrole{p,p}{:}\DUrole{w,w}{  }\DUrole{n,n}{bool}\DUrole{w,w}{  }\DUrole{o,o}{=}\DUrole{w,w}{  }\DUrole{default_value}{True}}}{{ $\rightarrow$ PolyData}}
\pysigstopsignatures
\sphinxAtStartPar
Get a pyvista mesh of the outline in world\sphinxhyphen{}space.
\begin{quote}\begin{description}
\sphinxlineitem{Parameters}\begin{itemize}
\item {} 
\sphinxAtStartPar
\sphinxstyleliteralstrong{\sphinxupquote{full}} (\sphinxstyleliteralemphasis{\sphinxupquote{bool}}) \textendash{} Whether to render the full volume or just a wireframe. Defaults to False.

\item {} 
\sphinxAtStartPar
\sphinxstyleliteralstrong{\sphinxupquote{cache\_dir}} (\sphinxstyleliteralemphasis{\sphinxupquote{Optional}}\sphinxstyleliteralemphasis{\sphinxupquote{{[}}}\sphinxstyleliteralemphasis{\sphinxupquote{Path}}\sphinxstyleliteralemphasis{\sphinxupquote{{]}}}\sphinxstyleliteralemphasis{\sphinxupquote{, }}\sphinxstyleliteralemphasis{\sphinxupquote{optional}}) \textendash{} a location to cache the bone surface.

\item {} 
\sphinxAtStartPar
\sphinxstyleliteralstrong{\sphinxupquote{use\_cached}} (\sphinxstyleliteralemphasis{\sphinxupquote{bool}}) \textendash{} If False, don’t use the cached bone surface but re\sphinxhyphen{}create it (expensive). Defaults to True.

\end{itemize}

\sphinxlineitem{Returns}
\sphinxAtStartPar
pyvista mesh.

\sphinxlineitem{Return type}
\sphinxAtStartPar
pv.PolyData

\end{description}\end{quote}

\end{fulllineitems}

\index{get\_surface() (deepdrr.vol.Volume method)@\spxentry{get\_surface()}\spxextra{deepdrr.vol.Volume method}}

\begin{fulllineitems}
\phantomsection\label{\detokenize{deepdrr.vol:deepdrr.vol.Volume.get_surface}}
\pysigstartsignatures
\pysiglinewithargsret{\sphinxbfcode{\sphinxupquote{get\_surface}}}{\sphinxparam{\DUrole{n,n}{material}\DUrole{p,p}{:}\DUrole{w,w}{  }\DUrole{n,n}{str}\DUrole{w,w}{  }\DUrole{o,o}{=}\DUrole{w,w}{  }\DUrole{default_value}{\textquotesingle{}bone\textquotesingle{}}}\sphinxparamcomma \sphinxparam{\DUrole{n,n}{use\_cached}\DUrole{p,p}{:}\DUrole{w,w}{  }\DUrole{n,n}{bool}\DUrole{w,w}{  }\DUrole{o,o}{=}\DUrole{w,w}{  }\DUrole{default_value}{True}}}{}
\pysigstopsignatures
\end{fulllineitems}

\index{ijk\_from\_anatomical (deepdrr.vol.Volume property)@\spxentry{ijk\_from\_anatomical}\spxextra{deepdrr.vol.Volume property}}

\begin{fulllineitems}
\phantomsection\label{\detokenize{deepdrr.vol:deepdrr.vol.Volume.ijk_from_anatomical}}
\pysigstartsignatures
\pysigline{\sphinxbfcode{\sphinxupquote{property\DUrole{w,w}{  }}}\sphinxbfcode{\sphinxupquote{ijk\_from\_anatomical}}}
\pysigstopsignatures
\end{fulllineitems}

\index{ijk\_from\_world (deepdrr.vol.Volume property)@\spxentry{ijk\_from\_world}\spxextra{deepdrr.vol.Volume property}}

\begin{fulllineitems}
\phantomsection\label{\detokenize{deepdrr.vol:deepdrr.vol.Volume.ijk_from_world}}
\pysigstartsignatures
\pysigline{\sphinxbfcode{\sphinxupquote{property\DUrole{w,w}{  }}}\sphinxbfcode{\sphinxupquote{ijk\_from\_world}}\sphinxbfcode{\sphinxupquote{\DUrole{p,p}{:}\DUrole{w,w}{  }{\hyperref[\detokenize{deepdrr.geo:deepdrr.geo.core.FrameTransform}]{\sphinxcrossref{FrameTransform}}}}}}
\pysigstopsignatures
\end{fulllineitems}

\index{interpolate() (deepdrr.vol.Volume method)@\spxentry{interpolate()}\spxextra{deepdrr.vol.Volume method}}

\begin{fulllineitems}
\phantomsection\label{\detokenize{deepdrr.vol:deepdrr.vol.Volume.interpolate}}
\pysigstartsignatures
\pysiglinewithargsret{\sphinxbfcode{\sphinxupquote{interpolate}}}{\sphinxparam{\DUrole{o,o}{*}\DUrole{n,n}{x}\DUrole{p,p}{:}\DUrole{w,w}{  }\DUrole{n,n}{{\hyperref[\detokenize{deepdrr.geo:deepdrr.geo.core.Point3D}]{\sphinxcrossref{Point3D}}}}}\sphinxparamcomma \sphinxparam{\DUrole{n,n}{method}\DUrole{p,p}{:}\DUrole{w,w}{  }\DUrole{n,n}{str}\DUrole{w,w}{  }\DUrole{o,o}{=}\DUrole{w,w}{  }\DUrole{default_value}{\textquotesingle{}linear\textquotesingle{}}}}{{ $\rightarrow$ ndarray}}
\pysigstopsignatures
\sphinxAtStartPar
Interpolate the value of the volume at the point.

\sphinxAtStartPar
This is a \sphinxstyleemphasis{slow} version of interpolation, using scipy under the hood. DeepDRR uses cubic
spline interpolation on the GPU for rendering. This function is provided as a convenience.
\begin{quote}\begin{description}
\sphinxlineitem{Parameters}\begin{itemize}
\item {} 
\sphinxAtStartPar
\sphinxstyleliteralstrong{\sphinxupquote{x}} ({\hyperref[\detokenize{deepdrr.geo:deepdrr.geo.Point3D}]{\sphinxcrossref{\sphinxstyleliteralemphasis{\sphinxupquote{geo.Point3D}}}}}) \textendash{} The point or points in world\sphinxhyphen{}space.

\item {} 
\sphinxAtStartPar
\sphinxstyleliteralstrong{\sphinxupquote{method}} (\sphinxstyleliteralemphasis{\sphinxupquote{str}}) \textendash{} The interpolation method to be used.
Accepted values are “linear” and “nearest”.
Defaults to “linear.”

\end{itemize}

\sphinxlineitem{Returns}
\sphinxAtStartPar
\begin{description}
\sphinxlineitem{The interpolated value(s) of the point(s)}
\sphinxAtStartPar
in the Volume. If a point is outside the volume, the value is NaN.

\end{description}


\sphinxlineitem{Return type}
\sphinxAtStartPar
Union{[}float, np.ndarray{]}

\end{description}\end{quote}

\end{fulllineitems}

\index{isosurface() (deepdrr.vol.Volume method)@\spxentry{isosurface()}\spxextra{deepdrr.vol.Volume method}}

\begin{fulllineitems}
\phantomsection\label{\detokenize{deepdrr.vol:deepdrr.vol.Volume.isosurface}}
\pysigstartsignatures
\pysiglinewithargsret{\sphinxbfcode{\sphinxupquote{isosurface}}}{\sphinxparam{\DUrole{n,n}{value}\DUrole{p,p}{:}\DUrole{w,w}{  }\DUrole{n,n}{float}\DUrole{w,w}{  }\DUrole{o,o}{=}\DUrole{w,w}{  }\DUrole{default_value}{0.5}}\sphinxparamcomma \sphinxparam{\DUrole{n,n}{label}\DUrole{p,p}{:}\DUrole{w,w}{  }\DUrole{n,n}{int\DUrole{w,w}{  }\DUrole{p,p}{|}\DUrole{w,w}{  }None}\DUrole{w,w}{  }\DUrole{o,o}{=}\DUrole{w,w}{  }\DUrole{default_value}{None}}\sphinxparamcomma \sphinxparam{\DUrole{n,n}{node\_centered}\DUrole{p,p}{:}\DUrole{w,w}{  }\DUrole{n,n}{bool}\DUrole{w,w}{  }\DUrole{o,o}{=}\DUrole{w,w}{  }\DUrole{default_value}{True}}\sphinxparamcomma \sphinxparam{\DUrole{n,n}{smooth}\DUrole{p,p}{:}\DUrole{w,w}{  }\DUrole{n,n}{bool}\DUrole{w,w}{  }\DUrole{o,o}{=}\DUrole{w,w}{  }\DUrole{default_value}{True}}\sphinxparamcomma \sphinxparam{\DUrole{n,n}{decimation}\DUrole{p,p}{:}\DUrole{w,w}{  }\DUrole{n,n}{float}\DUrole{w,w}{  }\DUrole{o,o}{=}\DUrole{w,w}{  }\DUrole{default_value}{0.01}}\sphinxparamcomma \sphinxparam{\DUrole{n,n}{decimation\_points}\DUrole{p,p}{:}\DUrole{w,w}{  }\DUrole{n,n}{int\DUrole{w,w}{  }\DUrole{p,p}{|}\DUrole{w,w}{  }None}\DUrole{w,w}{  }\DUrole{o,o}{=}\DUrole{w,w}{  }\DUrole{default_value}{None}}\sphinxparamcomma \sphinxparam{\DUrole{n,n}{smooth\_iter}\DUrole{p,p}{:}\DUrole{w,w}{  }\DUrole{n,n}{int}\DUrole{w,w}{  }\DUrole{o,o}{=}\DUrole{w,w}{  }\DUrole{default_value}{200}}\sphinxparamcomma \sphinxparam{\DUrole{n,n}{relaxation\_factor}\DUrole{p,p}{:}\DUrole{w,w}{  }\DUrole{n,n}{float}\DUrole{w,w}{  }\DUrole{o,o}{=}\DUrole{w,w}{  }\DUrole{default_value}{0.25}}\sphinxparamcomma \sphinxparam{\DUrole{n,n}{convert\_to\_LPS}\DUrole{p,p}{:}\DUrole{w,w}{  }\DUrole{n,n}{bool}\DUrole{w,w}{  }\DUrole{o,o}{=}\DUrole{w,w}{  }\DUrole{default_value}{False}}}{{ $\rightarrow$ PolyData}}
\pysigstopsignatures
\sphinxAtStartPar
Make an isosurface from the volume’s data, transforming to anatomical\_coordinates.

\sphinxAtStartPar
Accepts arguments passed to {\hyperref[\detokenize{deepdrr.utils:deepdrr.utils.mesh_utils.isosurface}]{\sphinxcrossref{\sphinxcode{\sphinxupquote{deepdrr.utils.mesh\_utils.isosurface()}}}}}.
\begin{quote}\begin{description}
\sphinxlineitem{Parameters}\begin{itemize}
\item {} 
\sphinxAtStartPar
\sphinxstyleliteralstrong{\sphinxupquote{value}} (\sphinxstyleliteralemphasis{\sphinxupquote{float}}) \textendash{} The value at which to make the isosurface.

\item {} 
\sphinxAtStartPar
\sphinxstyleliteralstrong{\sphinxupquote{label}} (\sphinxstyleliteralemphasis{\sphinxupquote{int}}) \textendash{} The label of the isosurface.

\item {} 
\sphinxAtStartPar
\sphinxstyleliteralstrong{\sphinxupquote{node\_centered}} (\sphinxstyleliteralemphasis{\sphinxupquote{bool}}) \textendash{} If True, the isosurface is centered at the node.
If False, the isosurface is centered at the cell.

\item {} 
\sphinxAtStartPar
\sphinxstyleliteralstrong{\sphinxupquote{smooth}} (\sphinxstyleliteralemphasis{\sphinxupquote{bool}}) \textendash{} If True, the isosurface is smoothed.

\item {} 
\sphinxAtStartPar
\sphinxstyleliteralstrong{\sphinxupquote{decimation}} (\sphinxstyleliteralemphasis{\sphinxupquote{float}}) \textendash{} The decimation factor (how many points to remove).

\item {} 
\sphinxAtStartPar
\sphinxstyleliteralstrong{\sphinxupquote{smooth\_iter}} (\sphinxstyleliteralemphasis{\sphinxupquote{int}}) \textendash{} The number of smoothing iterations.

\item {} 
\sphinxAtStartPar
\sphinxstyleliteralstrong{\sphinxupquote{relaxation\_factor}} (\sphinxstyleliteralemphasis{\sphinxupquote{float}}) \textendash{} The relaxation factor.

\item {} 
\sphinxAtStartPar
\sphinxstyleliteralstrong{\sphinxupquote{convert\_to\_LPS}} (\sphinxstyleliteralemphasis{\sphinxupquote{bool}}) \textendash{} If True, the isosurface is converted to LPS coordinates. (Recommended)

\end{itemize}

\sphinxlineitem{Returns}
\sphinxAtStartPar
The surface mesh in anatomical coordinates.

\sphinxlineitem{Return type}
\sphinxAtStartPar
pv.PolyData

\end{description}\end{quote}

\end{fulllineitems}

\index{load() (deepdrr.vol.Volume class method)@\spxentry{load()}\spxextra{deepdrr.vol.Volume class method}}

\begin{fulllineitems}
\phantomsection\label{\detokenize{deepdrr.vol:deepdrr.vol.Volume.load}}
\pysigstartsignatures
\pysiglinewithargsret{\sphinxbfcode{\sphinxupquote{classmethod\DUrole{w,w}{  }}}\sphinxbfcode{\sphinxupquote{load}}}{\sphinxparam{\DUrole{n,n}{path}\DUrole{p,p}{:}\DUrole{w,w}{  }\DUrole{n,n}{Path}}\sphinxparamcomma \sphinxparam{\DUrole{n,n}{segmentation}\DUrole{p,p}{:}\DUrole{w,w}{  }\DUrole{n,n}{bool}\DUrole{w,w}{  }\DUrole{o,o}{=}\DUrole{w,w}{  }\DUrole{default_value}{False}}}{{ $\rightarrow$ {\hyperref[\detokenize{deepdrr.vol:deepdrr.vol.volume.Volume}]{\sphinxcrossref{Volume}}}}}
\pysigstopsignatures
\sphinxAtStartPar
Load a volume from disk.
\begin{quote}\begin{description}
\sphinxlineitem{Parameters}\begin{itemize}
\item {} 
\sphinxAtStartPar
\sphinxstyleliteralstrong{\sphinxupquote{path}} (\sphinxstyleliteralemphasis{\sphinxupquote{Path}}) \textendash{} a directory containing the volume and segmentations.

\item {} 
\sphinxAtStartPar
\sphinxstyleliteralstrong{\sphinxupquote{segmentation}} (\sphinxstyleliteralemphasis{\sphinxupquote{bool}}\sphinxstyleliteralemphasis{\sphinxupquote{, }}\sphinxstyleliteralemphasis{\sphinxupquote{optional}}) \textendash{} if the volume is a segmentation,
populate the materials from that.

\end{itemize}

\end{description}\end{quote}

\sphinxAtStartPar
Returns: Volume.

\end{fulllineitems}

\index{materials (deepdrr.vol.Volume attribute)@\spxentry{materials}\spxextra{deepdrr.vol.Volume attribute}}

\begin{fulllineitems}
\phantomsection\label{\detokenize{deepdrr.vol:deepdrr.vol.Volume.materials}}
\pysigstartsignatures
\pysigline{\sphinxbfcode{\sphinxupquote{materials}}\sphinxbfcode{\sphinxupquote{\DUrole{p,p}{:}\DUrole{w,w}{  }Dict\DUrole{p,p}{{[}}str\DUrole{p,p}{,}\DUrole{w,w}{  }ndarray\DUrole{p,p}{{]}}}}}
\pysigstopsignatures
\end{fulllineitems}

\index{orient\_patient() (deepdrr.vol.Volume method)@\spxentry{orient\_patient()}\spxextra{deepdrr.vol.Volume method}}

\begin{fulllineitems}
\phantomsection\label{\detokenize{deepdrr.vol:deepdrr.vol.Volume.orient_patient}}
\pysigstartsignatures
\pysiglinewithargsret{\sphinxbfcode{\sphinxupquote{orient\_patient}}}{\sphinxparam{\DUrole{n,n}{head\_first}\DUrole{p,p}{:}\DUrole{w,w}{  }\DUrole{n,n}{bool}\DUrole{w,w}{  }\DUrole{o,o}{=}\DUrole{w,w}{  }\DUrole{default_value}{True}}\sphinxparamcomma \sphinxparam{\DUrole{n,n}{supine}\DUrole{p,p}{:}\DUrole{w,w}{  }\DUrole{n,n}{bool}\DUrole{w,w}{  }\DUrole{o,o}{=}\DUrole{w,w}{  }\DUrole{default_value}{True}}\sphinxparamcomma \sphinxparam{\DUrole{n,n}{world\_from\_device}\DUrole{p,p}{:}\DUrole{w,w}{  }\DUrole{n,n}{{\hyperref[\detokenize{deepdrr.geo:deepdrr.geo.core.FrameTransform}]{\sphinxcrossref{FrameTransform}}}\DUrole{w,w}{  }\DUrole{p,p}{|}\DUrole{w,w}{  }None}\DUrole{w,w}{  }\DUrole{o,o}{=}\DUrole{w,w}{  }\DUrole{default_value}{None}}}{{ $\rightarrow$ None}}
\pysigstopsignatures
\sphinxAtStartPar
Orient the patient with the given orientation, aligning with the Loop\sphinxhyphen{}X coordinates.
\begin{quote}\begin{description}
\sphinxlineitem{Parameters}\begin{itemize}
\item {} 
\sphinxAtStartPar
\sphinxstyleliteralstrong{\sphinxupquote{head\_first}} \textendash{} If True, the patient is oriented with head (superior axis) pointing in the \sphinxhyphen{}Y direction. Defaults to True.

\item {} 
\sphinxAtStartPar
\sphinxstyleliteralstrong{\sphinxupquote{supine}} \textendash{} If True, the patient is oriented so that the anterior axis (stomach) points toward +Z. Defaults to True.

\end{itemize}

\end{description}\end{quote}

\end{fulllineitems}

\index{origin (deepdrr.vol.Volume property)@\spxentry{origin}\spxextra{deepdrr.vol.Volume property}}

\begin{fulllineitems}
\phantomsection\label{\detokenize{deepdrr.vol:deepdrr.vol.Volume.origin}}
\pysigstartsignatures
\pysigline{\sphinxbfcode{\sphinxupquote{property\DUrole{w,w}{  }}}\sphinxbfcode{\sphinxupquote{origin}}\sphinxbfcode{\sphinxupquote{\DUrole{p,p}{:}\DUrole{w,w}{  }{\hyperref[\detokenize{deepdrr.geo:deepdrr.geo.core.Point3D}]{\sphinxcrossref{Point3D}}}}}}
\pysigstopsignatures
\sphinxAtStartPar
The origin of the volume in anatomical space.

\end{fulllineitems}

\index{origin\_in\_anatomical (deepdrr.vol.Volume property)@\spxentry{origin\_in\_anatomical}\spxextra{deepdrr.vol.Volume property}}

\begin{fulllineitems}
\phantomsection\label{\detokenize{deepdrr.vol:deepdrr.vol.Volume.origin_in_anatomical}}
\pysigstartsignatures
\pysigline{\sphinxbfcode{\sphinxupquote{property\DUrole{w,w}{  }}}\sphinxbfcode{\sphinxupquote{origin\_in\_anatomical}}\sphinxbfcode{\sphinxupquote{\DUrole{p,p}{:}\DUrole{w,w}{  }{\hyperref[\detokenize{deepdrr.geo:deepdrr.geo.core.Point3D}]{\sphinxcrossref{Point3D}}}}}}
\pysigstopsignatures
\sphinxAtStartPar
The origin of the volume in anatomical space.

\end{fulllineitems}

\index{origin\_in\_world (deepdrr.vol.Volume property)@\spxentry{origin\_in\_world}\spxextra{deepdrr.vol.Volume property}}

\begin{fulllineitems}
\phantomsection\label{\detokenize{deepdrr.vol:deepdrr.vol.Volume.origin_in_world}}
\pysigstartsignatures
\pysigline{\sphinxbfcode{\sphinxupquote{property\DUrole{w,w}{  }}}\sphinxbfcode{\sphinxupquote{origin\_in\_world}}\sphinxbfcode{\sphinxupquote{\DUrole{p,p}{:}\DUrole{w,w}{  }{\hyperref[\detokenize{deepdrr.geo:deepdrr.geo.core.Point3D}]{\sphinxcrossref{Point3D}}}}}}
\pysigstopsignatures
\sphinxAtStartPar
The origin of the volume in world space.

\end{fulllineitems}

\index{place() (deepdrr.vol.Volume method)@\spxentry{place()}\spxextra{deepdrr.vol.Volume method}}

\begin{fulllineitems}
\phantomsection\label{\detokenize{deepdrr.vol:deepdrr.vol.Volume.place}}
\pysigstartsignatures
\pysiglinewithargsret{\sphinxbfcode{\sphinxupquote{place}}}{\sphinxparam{\DUrole{n,n}{point\_in\_anatomical}\DUrole{p,p}{:}\DUrole{w,w}{  }\DUrole{n,n}{{\hyperref[\detokenize{deepdrr.geo:deepdrr.geo.core.Point3D}]{\sphinxcrossref{Point3D}}}}}\sphinxparamcomma \sphinxparam{\DUrole{n,n}{desired\_point\_in\_world}\DUrole{p,p}{:}\DUrole{w,w}{  }\DUrole{n,n}{{\hyperref[\detokenize{deepdrr.geo:deepdrr.geo.core.Point3D}]{\sphinxcrossref{Point3D}}}}}}{{ $\rightarrow$ None}}
\pysigstopsignatures
\sphinxAtStartPar
Translate the volume so that x\_in\_anatomical corresponds to x\_in\_world.

\end{fulllineitems}

\index{place\_center() (deepdrr.vol.Volume method)@\spxentry{place\_center()}\spxextra{deepdrr.vol.Volume method}}

\begin{fulllineitems}
\phantomsection\label{\detokenize{deepdrr.vol:deepdrr.vol.Volume.place_center}}
\pysigstartsignatures
\pysiglinewithargsret{\sphinxbfcode{\sphinxupquote{place\_center}}}{\sphinxparam{\DUrole{n,n}{x}\DUrole{p,p}{:}\DUrole{w,w}{  }\DUrole{n,n}{{\hyperref[\detokenize{deepdrr.geo:deepdrr.geo.core.Point3D}]{\sphinxcrossref{Point3D}}}}}}{{ $\rightarrow$ None}}
\pysigstopsignatures
\sphinxAtStartPar
Translate the volume so that its center is located at world\sphinxhyphen{}space point x.

\sphinxAtStartPar
Only changes the translation elements of the world\_from\_anatomical transform. Preserves the current rotation of the
\begin{quote}\begin{description}
\sphinxlineitem{Parameters}
\sphinxAtStartPar
\sphinxstyleliteralstrong{\sphinxupquote{x}} ({\hyperref[\detokenize{deepdrr.geo:deepdrr.geo.Point3D}]{\sphinxcrossref{\sphinxstyleliteralemphasis{\sphinxupquote{geo.Point3D}}}}}) \textendash{} the world\sphinxhyphen{}space point.

\end{description}\end{quote}

\end{fulllineitems}

\index{prone() (deepdrr.vol.Volume method)@\spxentry{prone()}\spxextra{deepdrr.vol.Volume method}}

\begin{fulllineitems}
\phantomsection\label{\detokenize{deepdrr.vol:deepdrr.vol.Volume.prone}}
\pysigstartsignatures
\pysiglinewithargsret{\sphinxbfcode{\sphinxupquote{prone}}}{}{}
\pysigstopsignatures
\sphinxAtStartPar
Turns the volume to be face down.

\sphinxAtStartPar
This aligns the patient so that, in world space,
the posterior side is toward +Z, inferior is toward +X,
and right is toward +Y.
\begin{quote}\begin{description}
\sphinxlineitem{Raises}
\sphinxAtStartPar
\sphinxstyleliteralstrong{\sphinxupquote{NotImplementedError}} \textendash{} If the anatomical coordinate system is not “RAS”.

\end{description}\end{quote}

\end{fulllineitems}

\index{rotate() (deepdrr.vol.Volume method)@\spxentry{rotate()}\spxextra{deepdrr.vol.Volume method}}

\begin{fulllineitems}
\phantomsection\label{\detokenize{deepdrr.vol:deepdrr.vol.Volume.rotate}}
\pysigstartsignatures
\pysiglinewithargsret{\sphinxbfcode{\sphinxupquote{rotate}}}{\sphinxparam{\DUrole{n,n}{rotation}\DUrole{p,p}{:}\DUrole{w,w}{  }\DUrole{n,n}{{\hyperref[\detokenize{deepdrr.geo:deepdrr.geo.core.Vector3D}]{\sphinxcrossref{Vector3D}}}\DUrole{w,w}{  }\DUrole{p,p}{|}\DUrole{w,w}{  }{\hyperref[\detokenize{deepdrr.geo:deepdrr.geo.Rotation}]{\sphinxcrossref{Rotation}}}}}\sphinxparamcomma \sphinxparam{\DUrole{n,n}{center}\DUrole{p,p}{:}\DUrole{w,w}{  }\DUrole{n,n}{{\hyperref[\detokenize{deepdrr.geo:deepdrr.geo.core.Point3D}]{\sphinxcrossref{Point3D}}}\DUrole{w,w}{  }\DUrole{p,p}{|}\DUrole{w,w}{  }None}\DUrole{w,w}{  }\DUrole{o,o}{=}\DUrole{w,w}{  }\DUrole{default_value}{None}}}{{ $\rightarrow$ {\hyperref[\detokenize{deepdrr.vol:deepdrr.vol.volume.Volume}]{\sphinxcrossref{Volume}}}}}
\pysigstopsignatures
\sphinxAtStartPar
Rotate the volume by \sphinxtitleref{rotation} about \sphinxtitleref{center}.
\begin{quote}\begin{description}
\sphinxlineitem{Parameters}\begin{itemize}
\item {} 
\sphinxAtStartPar
\sphinxstyleliteralstrong{\sphinxupquote{rotation}} (\sphinxstyleliteralemphasis{\sphinxupquote{Union}}\sphinxstyleliteralemphasis{\sphinxupquote{{[}}}{\hyperref[\detokenize{deepdrr.geo:deepdrr.geo.Vector3D}]{\sphinxcrossref{\sphinxstyleliteralemphasis{\sphinxupquote{geo.Vector3D}}}}}\sphinxstyleliteralemphasis{\sphinxupquote{, }}{\hyperref[\detokenize{deepdrr.geo:deepdrr.geo.Rotation}]{\sphinxcrossref{\sphinxstyleliteralemphasis{\sphinxupquote{Rotation}}}}}\sphinxstyleliteralemphasis{\sphinxupquote{{]}}}) \textendash{} the rotation in world\sphinxhyphen{}space. If it is a vector, \sphinxtitleref{Rotation.from\_rotvec(rotation)} is used.

\item {} 
\sphinxAtStartPar
\sphinxstyleliteralstrong{\sphinxupquote{center}} ({\hyperref[\detokenize{deepdrr.geo:deepdrr.geo.Point3D}]{\sphinxcrossref{\sphinxstyleliteralemphasis{\sphinxupquote{geo.Point3D}}}}}\sphinxstyleliteralemphasis{\sphinxupquote{, }}\sphinxstyleliteralemphasis{\sphinxupquote{optional}}) \textendash{} the center of rotation in world space coordinates. If None, the center of the volume is used.

\end{itemize}

\end{description}\end{quote}

\end{fulllineitems}

\index{save() (deepdrr.vol.Volume method)@\spxentry{save()}\spxextra{deepdrr.vol.Volume method}}

\begin{fulllineitems}
\phantomsection\label{\detokenize{deepdrr.vol:deepdrr.vol.Volume.save}}
\pysigstartsignatures
\pysiglinewithargsret{\sphinxbfcode{\sphinxupquote{save}}}{\sphinxparam{\DUrole{n,n}{output\_dir}\DUrole{p,p}{:}\DUrole{w,w}{  }\DUrole{n,n}{Path}}\sphinxparamcomma \sphinxparam{\DUrole{n,n}{segmentation}\DUrole{p,p}{:}\DUrole{w,w}{  }\DUrole{n,n}{bool}\DUrole{w,w}{  }\DUrole{o,o}{=}\DUrole{w,w}{  }\DUrole{default_value}{False}}}{}
\pysigstopsignatures
\sphinxAtStartPar
Save the volume to disk as a nifti file.
\begin{quote}\begin{description}
\sphinxlineitem{Parameters}\begin{itemize}
\item {} 
\sphinxAtStartPar
\sphinxstyleliteralstrong{\sphinxupquote{output\_dir}} (\sphinxstyleliteralemphasis{\sphinxupquote{Path}}) \textendash{} a directory to save the volume and segmentations to.

\item {} 
\sphinxAtStartPar
\sphinxstyleliteralstrong{\sphinxupquote{segmentation}} (\sphinxstyleliteralemphasis{\sphinxupquote{bool}}\sphinxstyleliteralemphasis{\sphinxupquote{, }}\sphinxstyleliteralemphasis{\sphinxupquote{optional}}) \textendash{} if the volume is a segmentation, there’s
no need to save the materials.

\end{itemize}

\end{description}\end{quote}

\end{fulllineitems}

\index{segment\_materials() (deepdrr.vol.Volume class method)@\spxentry{segment\_materials()}\spxextra{deepdrr.vol.Volume class method}}

\begin{fulllineitems}
\phantomsection\label{\detokenize{deepdrr.vol:deepdrr.vol.Volume.segment_materials}}
\pysigstartsignatures
\pysiglinewithargsret{\sphinxbfcode{\sphinxupquote{classmethod\DUrole{w,w}{  }}}\sphinxbfcode{\sphinxupquote{segment\_materials}}}{\sphinxparam{\DUrole{n,n}{hu\_values}\DUrole{p,p}{:}\DUrole{w,w}{  }\DUrole{n,n}{ndarray}}\sphinxparamcomma \sphinxparam{\DUrole{n,n}{anatomical\_from\_ijk}\DUrole{p,p}{:}\DUrole{w,w}{  }\DUrole{n,n}{{\hyperref[\detokenize{deepdrr.geo:deepdrr.geo.core.FrameTransform}]{\sphinxcrossref{FrameTransform}}}}}\sphinxparamcomma \sphinxparam{\DUrole{n,n}{use\_thresholding}\DUrole{p,p}{:}\DUrole{w,w}{  }\DUrole{n,n}{bool}\DUrole{w,w}{  }\DUrole{o,o}{=}\DUrole{w,w}{  }\DUrole{default_value}{True}}\sphinxparamcomma \sphinxparam{\DUrole{n,n}{use\_cached}\DUrole{p,p}{:}\DUrole{w,w}{  }\DUrole{n,n}{bool}\DUrole{w,w}{  }\DUrole{o,o}{=}\DUrole{w,w}{  }\DUrole{default_value}{True}}\sphinxparamcomma \sphinxparam{\DUrole{n,n}{save\_cache}\DUrole{p,p}{:}\DUrole{w,w}{  }\DUrole{n,n}{bool}\DUrole{w,w}{  }\DUrole{o,o}{=}\DUrole{w,w}{  }\DUrole{default_value}{False}}\sphinxparamcomma \sphinxparam{\DUrole{n,n}{cache\_dir}\DUrole{p,p}{:}\DUrole{w,w}{  }\DUrole{n,n}{Path\DUrole{w,w}{  }\DUrole{p,p}{|}\DUrole{w,w}{  }None}\DUrole{w,w}{  }\DUrole{o,o}{=}\DUrole{w,w}{  }\DUrole{default_value}{None}}\sphinxparamcomma \sphinxparam{\DUrole{n,n}{cache\_name}\DUrole{p,p}{:}\DUrole{w,w}{  }\DUrole{n,n}{str\DUrole{w,w}{  }\DUrole{p,p}{|}\DUrole{w,w}{  }None}\DUrole{w,w}{  }\DUrole{o,o}{=}\DUrole{w,w}{  }\DUrole{default_value}{None}}}{{ $\rightarrow$ Dict\DUrole{p,p}{{[}}str\DUrole{p,p}{,}\DUrole{w,w}{  }ndarray\DUrole{p,p}{{]}}}}
\pysigstopsignatures
\sphinxAtStartPar
Segment the materials in a volume, potentially caching.

\sphinxAtStartPar
If cache\_dir is None, then
\begin{quote}\begin{description}
\sphinxlineitem{Parameters}\begin{itemize}
\item {} 
\sphinxAtStartPar
\sphinxstyleliteralstrong{\sphinxupquote{hu\_values}} (\sphinxstyleliteralemphasis{\sphinxupquote{np.ndarray}}) \textendash{} volume data in Hounsfield Units.

\item {} 
\sphinxAtStartPar
\sphinxstyleliteralstrong{\sphinxupquote{use\_thretholding}} (\sphinxstyleliteralemphasis{\sphinxupquote{bool}}\sphinxstyleliteralemphasis{\sphinxupquote{, }}\sphinxstyleliteralemphasis{\sphinxupquote{optional}}) \textendash{} whether to segment with thresholding (true) or a DNN. Defaults to True.

\item {} 
\sphinxAtStartPar
\sphinxstyleliteralstrong{\sphinxupquote{use\_cached}} (\sphinxstyleliteralemphasis{\sphinxupquote{bool}}\sphinxstyleliteralemphasis{\sphinxupquote{, }}\sphinxstyleliteralemphasis{\sphinxupquote{optional}}) \textendash{} use the cached segmentation, if it exists. Defaults to True.

\item {} 
\sphinxAtStartPar
\sphinxstyleliteralstrong{\sphinxupquote{save\_cache}} (\sphinxstyleliteralemphasis{\sphinxupquote{bool}}\sphinxstyleliteralemphasis{\sphinxupquote{, }}\sphinxstyleliteralemphasis{\sphinxupquote{optional}}) \textendash{} save the segmentation to cache\_dir. Defaults to True.

\item {} 
\sphinxAtStartPar
\sphinxstyleliteralstrong{\sphinxupquote{cache\_dir}} (\sphinxstyleliteralemphasis{\sphinxupquote{Optional}}\sphinxstyleliteralemphasis{\sphinxupquote{{[}}}\sphinxstyleliteralemphasis{\sphinxupquote{Path}}\sphinxstyleliteralemphasis{\sphinxupquote{{]}}}\sphinxstyleliteralemphasis{\sphinxupquote{, }}\sphinxstyleliteralemphasis{\sphinxupquote{optional}}) \textendash{} where to look for the segmentation cache. If None, no caching performed. Defaults to None.

\item {} 
\sphinxAtStartPar
\sphinxstyleliteralstrong{\sphinxupquote{cache\_name}} (\sphinxstyleliteralemphasis{\sphinxupquote{str}}\sphinxstyleliteralemphasis{\sphinxupquote{, }}\sphinxstyleliteralemphasis{\sphinxupquote{optional}}) \textendash{} Name of cache file. Must be provided if use\_cached or cache\_dir is True. Defaults to None.

\end{itemize}

\sphinxlineitem{Returns}
\sphinxAtStartPar
materials segmentation.

\sphinxlineitem{Return type}
\sphinxAtStartPar
Dict{[}str, np.ndarray{]}

\end{description}\end{quote}

\end{fulllineitems}

\index{shape (deepdrr.vol.Volume property)@\spxentry{shape}\spxextra{deepdrr.vol.Volume property}}

\begin{fulllineitems}
\phantomsection\label{\detokenize{deepdrr.vol:deepdrr.vol.Volume.shape}}
\pysigstartsignatures
\pysigline{\sphinxbfcode{\sphinxupquote{property\DUrole{w,w}{  }}}\sphinxbfcode{\sphinxupquote{shape}}\sphinxbfcode{\sphinxupquote{\DUrole{p,p}{:}\DUrole{w,w}{  }Tuple\DUrole{p,p}{{[}}int\DUrole{p,p}{,}\DUrole{w,w}{  }int\DUrole{p,p}{,}\DUrole{w,w}{  }int\DUrole{p,p}{{]}}}}}
\pysigstopsignatures
\end{fulllineitems}

\index{shrink() (deepdrr.vol.Volume method)@\spxentry{shrink()}\spxextra{deepdrr.vol.Volume method}}

\begin{fulllineitems}
\phantomsection\label{\detokenize{deepdrr.vol:deepdrr.vol.Volume.shrink}}
\pysigstartsignatures
\pysiglinewithargsret{\sphinxbfcode{\sphinxupquote{shrink}}}{}{{ $\rightarrow$ {\hyperref[\detokenize{deepdrr.vol:deepdrr.vol.volume.Volume}]{\sphinxcrossref{Volume}}}}}
\pysigstopsignatures
\sphinxAtStartPar
Crop the volume to remove empty space.
\begin{quote}\begin{description}
\sphinxlineitem{Returns}
\sphinxAtStartPar
The cropped volume.

\sphinxlineitem{Return type}
\sphinxAtStartPar
{\hyperref[\detokenize{deepdrr.vol:deepdrr.vol.Volume}]{\sphinxcrossref{Volume}}}

\end{description}\end{quote}

\end{fulllineitems}

\index{spacing (deepdrr.vol.Volume property)@\spxentry{spacing}\spxextra{deepdrr.vol.Volume property}}

\begin{fulllineitems}
\phantomsection\label{\detokenize{deepdrr.vol:deepdrr.vol.Volume.spacing}}
\pysigstartsignatures
\pysigline{\sphinxbfcode{\sphinxupquote{property\DUrole{w,w}{  }}}\sphinxbfcode{\sphinxupquote{spacing}}\sphinxbfcode{\sphinxupquote{\DUrole{p,p}{:}\DUrole{w,w}{  }{\hyperref[\detokenize{deepdrr.geo:deepdrr.geo.core.Vector3D}]{\sphinxcrossref{Vector3D}}}}}}
\pysigstopsignatures
\sphinxAtStartPar
The spacing of the voxels.

\end{fulllineitems}

\index{supine() (deepdrr.vol.Volume method)@\spxentry{supine()}\spxextra{deepdrr.vol.Volume method}}

\begin{fulllineitems}
\phantomsection\label{\detokenize{deepdrr.vol:deepdrr.vol.Volume.supine}}
\pysigstartsignatures
\pysiglinewithargsret{\sphinxbfcode{\sphinxupquote{supine}}}{}{}
\pysigstopsignatures
\sphinxAtStartPar
Turns the volume to be face up.

\sphinxAtStartPar
This aligns the patient so that, in world space,
the anterior side is toward +Z, inferior is toward +X,
and left is toward +Y.
\begin{quote}\begin{description}
\sphinxlineitem{Raises}
\sphinxAtStartPar
\sphinxstyleliteralstrong{\sphinxupquote{NotImplementedError}} \textendash{} If the anatomical coordinate system is not “RAS”.

\end{description}\end{quote}

\end{fulllineitems}

\index{translate() (deepdrr.vol.Volume method)@\spxentry{translate()}\spxextra{deepdrr.vol.Volume method}}

\begin{fulllineitems}
\phantomsection\label{\detokenize{deepdrr.vol:deepdrr.vol.Volume.translate}}
\pysigstartsignatures
\pysiglinewithargsret{\sphinxbfcode{\sphinxupquote{translate}}}{\sphinxparam{\DUrole{n,n}{t}\DUrole{p,p}{:}\DUrole{w,w}{  }\DUrole{n,n}{{\hyperref[\detokenize{deepdrr.geo:deepdrr.geo.core.Vector3D}]{\sphinxcrossref{Vector3D}}}}}}{{ $\rightarrow$ {\hyperref[\detokenize{deepdrr.vol:deepdrr.vol.volume.Volume}]{\sphinxcrossref{Volume}}}}}
\pysigstopsignatures
\sphinxAtStartPar
Translate the volume by \sphinxtitleref{t}.
\begin{quote}\begin{description}
\sphinxlineitem{Parameters}
\sphinxAtStartPar
\sphinxstyleliteralstrong{\sphinxupquote{t}} ({\hyperref[\detokenize{deepdrr.geo:deepdrr.geo.Vector3D}]{\sphinxcrossref{\sphinxstyleliteralemphasis{\sphinxupquote{geo.Vector3D}}}}}) \textendash{} The vector to translate by, in world space.

\end{description}\end{quote}

\end{fulllineitems}

\index{translate\_center\_to() (deepdrr.vol.Volume method)@\spxentry{translate\_center\_to()}\spxextra{deepdrr.vol.Volume method}}

\begin{fulllineitems}
\phantomsection\label{\detokenize{deepdrr.vol:deepdrr.vol.Volume.translate_center_to}}
\pysigstartsignatures
\pysiglinewithargsret{\sphinxbfcode{\sphinxupquote{translate\_center\_to}}}{\sphinxparam{\DUrole{n,n}{x}\DUrole{p,p}{:}\DUrole{w,w}{  }\DUrole{n,n}{{\hyperref[\detokenize{deepdrr.geo:deepdrr.geo.core.Point3D}]{\sphinxcrossref{Point3D}}}}}}{{ $\rightarrow$ None}}
\pysigstopsignatures
\sphinxAtStartPar
Translate the volume so that its center is located at world\sphinxhyphen{}space point x.

\sphinxAtStartPar
Only changes the translation elements of the world\_from\_anatomical transform. Preserves the current rotation of the
\begin{quote}\begin{description}
\sphinxlineitem{Parameters}
\sphinxAtStartPar
\sphinxstyleliteralstrong{\sphinxupquote{x}} ({\hyperref[\detokenize{deepdrr.geo:deepdrr.geo.Point3D}]{\sphinxcrossref{\sphinxstyleliteralemphasis{\sphinxupquote{geo.Point3D}}}}}) \textendash{} the world\sphinxhyphen{}space point.

\end{description}\end{quote}

\end{fulllineitems}

\index{world\_from\_IJK (deepdrr.vol.Volume property)@\spxentry{world\_from\_IJK}\spxextra{deepdrr.vol.Volume property}}

\begin{fulllineitems}
\phantomsection\label{\detokenize{deepdrr.vol:deepdrr.vol.Volume.world_from_IJK}}
\pysigstartsignatures
\pysigline{\sphinxbfcode{\sphinxupquote{property\DUrole{w,w}{  }}}\sphinxbfcode{\sphinxupquote{world\_from\_IJK}}\sphinxbfcode{\sphinxupquote{\DUrole{p,p}{:}\DUrole{w,w}{  }{\hyperref[\detokenize{deepdrr.geo:deepdrr.geo.core.FrameTransform}]{\sphinxcrossref{FrameTransform}}}}}}
\pysigstopsignatures
\end{fulllineitems}

\index{world\_from\_anatomical (deepdrr.vol.Volume attribute)@\spxentry{world\_from\_anatomical}\spxextra{deepdrr.vol.Volume attribute}}

\begin{fulllineitems}
\phantomsection\label{\detokenize{deepdrr.vol:deepdrr.vol.Volume.world_from_anatomical}}
\pysigstartsignatures
\pysigline{\sphinxbfcode{\sphinxupquote{world\_from\_anatomical}}\sphinxbfcode{\sphinxupquote{\DUrole{p,p}{:}\DUrole{w,w}{  }{\hyperref[\detokenize{deepdrr.geo:deepdrr.geo.core.FrameTransform}]{\sphinxcrossref{FrameTransform}}}}}}
\pysigstopsignatures
\end{fulllineitems}

\index{world\_from\_ijk (deepdrr.vol.Volume property)@\spxentry{world\_from\_ijk}\spxextra{deepdrr.vol.Volume property}}

\begin{fulllineitems}
\phantomsection\label{\detokenize{deepdrr.vol:deepdrr.vol.Volume.world_from_ijk}}
\pysigstartsignatures
\pysigline{\sphinxbfcode{\sphinxupquote{property\DUrole{w,w}{  }}}\sphinxbfcode{\sphinxupquote{world\_from\_ijk}}\sphinxbfcode{\sphinxupquote{\DUrole{p,p}{:}\DUrole{w,w}{  }{\hyperref[\detokenize{deepdrr.geo:deepdrr.geo.core.FrameTransform}]{\sphinxcrossref{FrameTransform}}}}}}
\pysigstopsignatures
\end{fulllineitems}


\end{fulllineitems}



\section{Submodules}
\label{\detokenize{deepdrr:submodules}}

\section{deepdrr.downsample\_tool module}
\label{\detokenize{deepdrr:module-deepdrr.downsample_tool}}\label{\detokenize{deepdrr:deepdrr-downsample-tool-module}}\index{module@\spxentry{module}!deepdrr.downsample\_tool@\spxentry{deepdrr.downsample\_tool}}\index{deepdrr.downsample\_tool@\spxentry{deepdrr.downsample\_tool}!module@\spxentry{module}}\index{downsample\_tool() (in module deepdrr.downsample\_tool)@\spxentry{downsample\_tool()}\spxextra{in module deepdrr.downsample\_tool}}

\begin{fulllineitems}
\phantomsection\label{\detokenize{deepdrr:deepdrr.downsample_tool.downsample_tool}}
\pysigstartsignatures
\pysiglinewithargsret{\sphinxcode{\sphinxupquote{deepdrr.downsample\_tool.}}\sphinxbfcode{\sphinxupquote{downsample\_tool}}}{\sphinxparam{\DUrole{n,n}{ori\_CT\_volume}}\sphinxparamcomma \sphinxparam{\DUrole{n,n}{CT\_volume}}\sphinxparamcomma \sphinxparam{\DUrole{n,n}{CT\_voxel\_size}}\sphinxparamcomma \sphinxparam{\DUrole{n,n}{tool\_volume}}\sphinxparamcomma \sphinxparam{\DUrole{n,n}{tool\_voxel\_size}}\sphinxparamcomma \sphinxparam{\DUrole{n,n}{CT\_materials}}\sphinxparamcomma \sphinxparam{\DUrole{n,n}{tool\_materials}}\sphinxparamcomma \sphinxparam{\DUrole{n,n}{origin}}\sphinxparamcomma \sphinxparam{\DUrole{n,n}{tool\_origin}}}{}
\pysigstopsignatures
\end{fulllineitems}



\section{deepdrr.load\_dicom module}
\label{\detokenize{deepdrr:module-deepdrr.load_dicom}}\label{\detokenize{deepdrr:deepdrr-load-dicom-module}}\index{module@\spxentry{module}!deepdrr.load\_dicom@\spxentry{deepdrr.load\_dicom}}\index{deepdrr.load\_dicom@\spxentry{deepdrr.load\_dicom}!module@\spxentry{module}}
\sphinxAtStartPar
Legacy code for loading DICOM files. See vol.Volume.from\_dicom.
\index{conv\_hu\_to\_density() (in module deepdrr.load\_dicom)@\spxentry{conv\_hu\_to\_density()}\spxextra{in module deepdrr.load\_dicom}}

\begin{fulllineitems}
\phantomsection\label{\detokenize{deepdrr:deepdrr.load_dicom.conv_hu_to_density}}
\pysigstartsignatures
\pysiglinewithargsret{\sphinxcode{\sphinxupquote{deepdrr.load\_dicom.}}\sphinxbfcode{\sphinxupquote{conv\_hu\_to\_density}}}{\sphinxparam{\DUrole{n,n}{hu\_values}}\sphinxparamcomma \sphinxparam{\DUrole{n,n}{smoothAir}\DUrole{o,o}{=}\DUrole{default_value}{False}}}{}
\pysigstopsignatures
\end{fulllineitems}

\index{conv\_hu\_to\_materials() (in module deepdrr.load\_dicom)@\spxentry{conv\_hu\_to\_materials()}\spxextra{in module deepdrr.load\_dicom}}

\begin{fulllineitems}
\phantomsection\label{\detokenize{deepdrr:deepdrr.load_dicom.conv_hu_to_materials}}
\pysigstartsignatures
\pysiglinewithargsret{\sphinxcode{\sphinxupquote{deepdrr.load\_dicom.}}\sphinxbfcode{\sphinxupquote{conv\_hu\_to\_materials}}}{\sphinxparam{\DUrole{n,n}{hu\_values}}}{}
\pysigstopsignatures
\end{fulllineitems}

\index{conv\_hu\_to\_materials\_thresholding() (in module deepdrr.load\_dicom)@\spxentry{conv\_hu\_to\_materials\_thresholding()}\spxextra{in module deepdrr.load\_dicom}}

\begin{fulllineitems}
\phantomsection\label{\detokenize{deepdrr:deepdrr.load_dicom.conv_hu_to_materials_thresholding}}
\pysigstartsignatures
\pysiglinewithargsret{\sphinxcode{\sphinxupquote{deepdrr.load\_dicom.}}\sphinxbfcode{\sphinxupquote{conv\_hu\_to\_materials\_thresholding}}}{\sphinxparam{\DUrole{n,n}{hu\_values}}}{}
\pysigstopsignatures
\end{fulllineitems}

\index{load\_dicom() (in module deepdrr.load\_dicom)@\spxentry{load\_dicom()}\spxextra{in module deepdrr.load\_dicom}}

\begin{fulllineitems}
\phantomsection\label{\detokenize{deepdrr:deepdrr.load_dicom.load_dicom}}
\pysigstartsignatures
\pysiglinewithargsret{\sphinxcode{\sphinxupquote{deepdrr.load\_dicom.}}\sphinxbfcode{\sphinxupquote{load\_dicom}}}{\sphinxparam{\DUrole{n,n}{source\_path}\DUrole{o,o}{=}\DUrole{default_value}{\textquotesingle{}./*/*/\textquotesingle{}}}\sphinxparamcomma \sphinxparam{\DUrole{n,n}{fixed\_slice\_thinckness}\DUrole{o,o}{=}\DUrole{default_value}{None}}\sphinxparamcomma \sphinxparam{\DUrole{n,n}{new\_resolution}\DUrole{o,o}{=}\DUrole{default_value}{None}}\sphinxparamcomma \sphinxparam{\DUrole{n,n}{truncate}\DUrole{o,o}{=}\DUrole{default_value}{None}}\sphinxparamcomma \sphinxparam{\DUrole{n,n}{smooth\_air}\DUrole{o,o}{=}\DUrole{default_value}{False}}\sphinxparamcomma \sphinxparam{\DUrole{n,n}{use\_thresholding\_segmentation}\DUrole{o,o}{=}\DUrole{default_value}{False}}\sphinxparamcomma \sphinxparam{\DUrole{n,n}{file\_extension}\DUrole{o,o}{=}\DUrole{default_value}{\textquotesingle{}.dcm\textquotesingle{}}}}{}
\pysigstopsignatures
\end{fulllineitems}

\index{upsample() (in module deepdrr.load\_dicom)@\spxentry{upsample()}\spxextra{in module deepdrr.load\_dicom}}

\begin{fulllineitems}
\phantomsection\label{\detokenize{deepdrr:deepdrr.load_dicom.upsample}}
\pysigstartsignatures
\pysiglinewithargsret{\sphinxcode{\sphinxupquote{deepdrr.load\_dicom.}}\sphinxbfcode{\sphinxupquote{upsample}}}{\sphinxparam{\DUrole{n,n}{volume}}\sphinxparamcomma \sphinxparam{\DUrole{n,n}{newResolution}}\sphinxparamcomma \sphinxparam{\DUrole{n,n}{voxelSize}}}{}
\pysigstopsignatures
\end{fulllineitems}



\section{deepdrr.load\_dicom\_tool module}
\label{\detokenize{deepdrr:module-deepdrr.load_dicom_tool}}\label{\detokenize{deepdrr:deepdrr-load-dicom-tool-module}}\index{module@\spxentry{module}!deepdrr.load\_dicom\_tool@\spxentry{deepdrr.load\_dicom\_tool}}\index{deepdrr.load\_dicom\_tool@\spxentry{deepdrr.load\_dicom\_tool}!module@\spxentry{module}}
\sphinxAtStartPar
Legacy code for loading tool volumes.
\index{conv\_hu\_to\_density() (in module deepdrr.load\_dicom\_tool)@\spxentry{conv\_hu\_to\_density()}\spxextra{in module deepdrr.load\_dicom\_tool}}

\begin{fulllineitems}
\phantomsection\label{\detokenize{deepdrr:deepdrr.load_dicom_tool.conv_hu_to_density}}
\pysigstartsignatures
\pysiglinewithargsret{\sphinxcode{\sphinxupquote{deepdrr.load\_dicom\_tool.}}\sphinxbfcode{\sphinxupquote{conv\_hu\_to\_density}}}{\sphinxparam{\DUrole{n,n}{hu\_values}}\sphinxparamcomma \sphinxparam{\DUrole{n,n}{smoothAir}\DUrole{o,o}{=}\DUrole{default_value}{False}}}{}
\pysigstopsignatures
\end{fulllineitems}

\index{conv\_hu\_to\_materials() (in module deepdrr.load\_dicom\_tool)@\spxentry{conv\_hu\_to\_materials()}\spxextra{in module deepdrr.load\_dicom\_tool}}

\begin{fulllineitems}
\phantomsection\label{\detokenize{deepdrr:deepdrr.load_dicom_tool.conv_hu_to_materials}}
\pysigstartsignatures
\pysiglinewithargsret{\sphinxcode{\sphinxupquote{deepdrr.load\_dicom\_tool.}}\sphinxbfcode{\sphinxupquote{conv\_hu\_to\_materials}}}{\sphinxparam{\DUrole{n,n}{hu\_values}}}{}
\pysigstopsignatures
\end{fulllineitems}

\index{conv\_hu\_to\_materials\_thresholding() (in module deepdrr.load\_dicom\_tool)@\spxentry{conv\_hu\_to\_materials\_thresholding()}\spxextra{in module deepdrr.load\_dicom\_tool}}

\begin{fulllineitems}
\phantomsection\label{\detokenize{deepdrr:deepdrr.load_dicom_tool.conv_hu_to_materials_thresholding}}
\pysigstartsignatures
\pysiglinewithargsret{\sphinxcode{\sphinxupquote{deepdrr.load\_dicom\_tool.}}\sphinxbfcode{\sphinxupquote{conv\_hu\_to\_materials\_thresholding}}}{\sphinxparam{\DUrole{n,n}{hu\_values}}}{}
\pysigstopsignatures
\end{fulllineitems}

\index{load\_dicom\_CT() (in module deepdrr.load\_dicom\_tool)@\spxentry{load\_dicom\_CT()}\spxextra{in module deepdrr.load\_dicom\_tool}}

\begin{fulllineitems}
\phantomsection\label{\detokenize{deepdrr:deepdrr.load_dicom_tool.load_dicom_CT}}
\pysigstartsignatures
\pysiglinewithargsret{\sphinxcode{\sphinxupquote{deepdrr.load\_dicom\_tool.}}\sphinxbfcode{\sphinxupquote{load\_dicom\_CT}}}{\sphinxparam{\DUrole{n,n}{source\_path}\DUrole{o,o}{=}\DUrole{default_value}{\textquotesingle{}./*/*/\textquotesingle{}}}\sphinxparamcomma \sphinxparam{\DUrole{n,n}{fixed\_slice\_thinckness}\DUrole{o,o}{=}\DUrole{default_value}{None}}\sphinxparamcomma \sphinxparam{\DUrole{n,n}{new\_resolution}\DUrole{o,o}{=}\DUrole{default_value}{None}}\sphinxparamcomma \sphinxparam{\DUrole{n,n}{truncate}\DUrole{o,o}{=}\DUrole{default_value}{None}}\sphinxparamcomma \sphinxparam{\DUrole{n,n}{smooth\_air}\DUrole{o,o}{=}\DUrole{default_value}{False}}\sphinxparamcomma \sphinxparam{\DUrole{n,n}{use\_thresholding\_segmentation}\DUrole{o,o}{=}\DUrole{default_value}{False}}\sphinxparamcomma \sphinxparam{\DUrole{n,n}{file\_extension}\DUrole{o,o}{=}\DUrole{default_value}{\textquotesingle{}.dcm\textquotesingle{}}}}{}
\pysigstopsignatures
\end{fulllineitems}

\index{load\_dicom\_metal() (in module deepdrr.load\_dicom\_tool)@\spxentry{load\_dicom\_metal()}\spxextra{in module deepdrr.load\_dicom\_tool}}

\begin{fulllineitems}
\phantomsection\label{\detokenize{deepdrr:deepdrr.load_dicom_tool.load_dicom_metal}}
\pysigstartsignatures
\pysiglinewithargsret{\sphinxcode{\sphinxupquote{deepdrr.load\_dicom\_tool.}}\sphinxbfcode{\sphinxupquote{load\_dicom\_metal}}}{\sphinxparam{\DUrole{n,n}{source\_path}\DUrole{o,o}{=}\DUrole{default_value}{\textquotesingle{}./*/*/\textquotesingle{}}}\sphinxparamcomma \sphinxparam{\DUrole{n,n}{sortBy}\DUrole{o,o}{=}\DUrole{default_value}{\textquotesingle{}SliceLocation\textquotesingle{}}}\sphinxparamcomma \sphinxparam{\DUrole{n,n}{fixed\_slice\_thinkness}\DUrole{o,o}{=}\DUrole{default_value}{None}}\sphinxparamcomma \sphinxparam{\DUrole{n,n}{new\_resolution}\DUrole{o,o}{=}\DUrole{default_value}{None}}\sphinxparamcomma \sphinxparam{\DUrole{n,n}{truncate}\DUrole{o,o}{=}\DUrole{default_value}{None}}\sphinxparamcomma \sphinxparam{\DUrole{n,n}{smooth\_air}\DUrole{o,o}{=}\DUrole{default_value}{False}}\sphinxparamcomma \sphinxparam{\DUrole{n,n}{use\_thresholding\_segmentation}\DUrole{o,o}{=}\DUrole{default_value}{False}}\sphinxparamcomma \sphinxparam{\DUrole{n,n}{flip}\DUrole{o,o}{=}\DUrole{default_value}{False}}}{}
\pysigstopsignatures
\end{fulllineitems}

\index{replace\_material() (in module deepdrr.load\_dicom\_tool)@\spxentry{replace\_material()}\spxextra{in module deepdrr.load\_dicom\_tool}}

\begin{fulllineitems}
\phantomsection\label{\detokenize{deepdrr:deepdrr.load_dicom_tool.replace_material}}
\pysigstartsignatures
\pysiglinewithargsret{\sphinxcode{\sphinxupquote{deepdrr.load\_dicom\_tool.}}\sphinxbfcode{\sphinxupquote{replace\_material}}}{\sphinxparam{\DUrole{n,n}{metal\_volume\_m\_ori}}\sphinxparamcomma \sphinxparam{\DUrole{n,n}{smooth\_air}\DUrole{o,o}{=}\DUrole{default_value}{False}}\sphinxparamcomma \sphinxparam{\DUrole{n,n}{use\_thresholding\_segmentation}\DUrole{o,o}{=}\DUrole{default_value}{True}}}{}
\pysigstopsignatures
\end{fulllineitems}

\index{upsample() (in module deepdrr.load\_dicom\_tool)@\spxentry{upsample()}\spxextra{in module deepdrr.load\_dicom\_tool}}

\begin{fulllineitems}
\phantomsection\label{\detokenize{deepdrr:deepdrr.load_dicom_tool.upsample}}
\pysigstartsignatures
\pysiglinewithargsret{\sphinxcode{\sphinxupquote{deepdrr.load\_dicom\_tool.}}\sphinxbfcode{\sphinxupquote{upsample}}}{\sphinxparam{\DUrole{n,n}{volume}}\sphinxparamcomma \sphinxparam{\DUrole{n,n}{newResolution}}\sphinxparamcomma \sphinxparam{\DUrole{n,n}{voxelSize}}}{}
\pysigstopsignatures
\end{fulllineitems}



\section{deepdrr.logging module}
\label{\detokenize{deepdrr:module-deepdrr.logging}}\label{\detokenize{deepdrr:deepdrr-logging-module}}\index{module@\spxentry{module}!deepdrr.logging@\spxentry{deepdrr.logging}}\index{deepdrr.logging@\spxentry{deepdrr.logging}!module@\spxentry{module}}\index{setup\_log() (in module deepdrr.logging)@\spxentry{setup\_log()}\spxextra{in module deepdrr.logging}}

\begin{fulllineitems}
\phantomsection\label{\detokenize{deepdrr:deepdrr.logging.setup_log}}
\pysigstartsignatures
\pysiglinewithargsret{\sphinxcode{\sphinxupquote{deepdrr.logging.}}\sphinxbfcode{\sphinxupquote{setup\_log}}}{}{}
\pysigstopsignatures
\end{fulllineitems}



\section{deepdrr.network\_segmentation module}
\label{\detokenize{deepdrr:module-deepdrr.network_segmentation}}\label{\detokenize{deepdrr:deepdrr-network-segmentation-module}}\index{module@\spxentry{module}!deepdrr.network\_segmentation@\spxentry{deepdrr.network\_segmentation}}\index{deepdrr.network\_segmentation@\spxentry{deepdrr.network\_segmentation}!module@\spxentry{module}}
\sphinxAtStartPar
Copyright (c) Matthew Macy 2017, All rights reserved.

\sphinxAtStartPar
Redistribution and use in source and binary forms, with or without
modification, are permitted provided that the following conditions are met:
\begin{itemize}
\item {} 
\sphinxAtStartPar
Redistributions of source code must retain the above copyright notice, this
list of conditions and the following disclaimer.

\item {} 
\sphinxAtStartPar
Redistributions in binary form must reproduce the above copyright notice,
this list of conditions and the following disclaimer in the documentation
and/or other materials provided with the distribution.

\item {} 
\sphinxAtStartPar
Neither the name of the copyright holder nor the names of its
contributors may be used to endorse or promote products derived from
this software without specific prior written permission.

\end{itemize}

\sphinxAtStartPar
THIS SOFTWARE IS PROVIDED BY THE COPYRIGHT HOLDERS AND CONTRIBUTORS “AS IS”
AND ANY EXPRESS OR IMPLIED WARRANTIES, INCLUDING, BUT NOT LIMITED TO, THE
IMPLIED WARRANTIES OF MERCHANTABILITY AND FITNESS FOR A PARTICULAR PURPOSE ARE
DISCLAIMED. IN NO EVENT SHALL THE COPYRIGHT HOLDER OR CONTRIBUTORS BE LIABLE
FOR ANY DIRECT, INDIRECT, INCIDENTAL, SPECIAL, EXEMPLARY, OR CONSEQUENTIAL
DAMAGES (INCLUDING, BUT NOT LIMITED TO, PROCUREMENT OF SUBSTITUTE GOODS OR
SERVICES; LOSS OF USE, DATA, OR PROFITS; OR BUSINESS INTERRUPTION) HOWEVER
CAUSED AND ON ANY THEORY OF LIABILITY, WHETHER IN CONTRACT, STRICT LIABILITY,
OR TORT (INCLUDING NEGLIGENCE OR OTHERWISE) ARISING IN ANY WAY OUT OF THE USE
OF THIS SOFTWARE, EVEN IF ADVISED OF THE POSSIBILITY OF SUCH DAMAGE.
\index{ContBatchNorm3d (class in deepdrr.network\_segmentation)@\spxentry{ContBatchNorm3d}\spxextra{class in deepdrr.network\_segmentation}}

\begin{fulllineitems}
\phantomsection\label{\detokenize{deepdrr:deepdrr.network_segmentation.ContBatchNorm3d}}
\pysigstartsignatures
\pysiglinewithargsret{\sphinxbfcode{\sphinxupquote{class\DUrole{w,w}{  }}}\sphinxcode{\sphinxupquote{deepdrr.network\_segmentation.}}\sphinxbfcode{\sphinxupquote{ContBatchNorm3d}}}{\sphinxparam{\DUrole{n,n}{num\_features}\DUrole{p,p}{:}\DUrole{w,w}{  }\DUrole{n,n}{int}}\sphinxparamcomma \sphinxparam{\DUrole{n,n}{eps}\DUrole{p,p}{:}\DUrole{w,w}{  }\DUrole{n,n}{float}\DUrole{w,w}{  }\DUrole{o,o}{=}\DUrole{w,w}{  }\DUrole{default_value}{1e\sphinxhyphen{}05}}\sphinxparamcomma \sphinxparam{\DUrole{n,n}{momentum}\DUrole{p,p}{:}\DUrole{w,w}{  }\DUrole{n,n}{float}\DUrole{w,w}{  }\DUrole{o,o}{=}\DUrole{w,w}{  }\DUrole{default_value}{0.1}}\sphinxparamcomma \sphinxparam{\DUrole{n,n}{affine}\DUrole{p,p}{:}\DUrole{w,w}{  }\DUrole{n,n}{bool}\DUrole{w,w}{  }\DUrole{o,o}{=}\DUrole{w,w}{  }\DUrole{default_value}{True}}\sphinxparamcomma \sphinxparam{\DUrole{n,n}{track\_running\_stats}\DUrole{p,p}{:}\DUrole{w,w}{  }\DUrole{n,n}{bool}\DUrole{w,w}{  }\DUrole{o,o}{=}\DUrole{w,w}{  }\DUrole{default_value}{True}}\sphinxparamcomma \sphinxparam{\DUrole{n,n}{device}\DUrole{o,o}{=}\DUrole{default_value}{None}}\sphinxparamcomma \sphinxparam{\DUrole{n,n}{dtype}\DUrole{o,o}{=}\DUrole{default_value}{None}}}{}
\pysigstopsignatures
\sphinxAtStartPar
Bases: \sphinxcode{\sphinxupquote{\_BatchNorm}}
\index{affine (deepdrr.network\_segmentation.ContBatchNorm3d attribute)@\spxentry{affine}\spxextra{deepdrr.network\_segmentation.ContBatchNorm3d attribute}}

\begin{fulllineitems}
\phantomsection\label{\detokenize{deepdrr:deepdrr.network_segmentation.ContBatchNorm3d.affine}}
\pysigstartsignatures
\pysigline{\sphinxbfcode{\sphinxupquote{affine}}\sphinxbfcode{\sphinxupquote{\DUrole{p,p}{:}\DUrole{w,w}{  }bool}}}
\pysigstopsignatures
\end{fulllineitems}

\index{eps (deepdrr.network\_segmentation.ContBatchNorm3d attribute)@\spxentry{eps}\spxextra{deepdrr.network\_segmentation.ContBatchNorm3d attribute}}

\begin{fulllineitems}
\phantomsection\label{\detokenize{deepdrr:deepdrr.network_segmentation.ContBatchNorm3d.eps}}
\pysigstartsignatures
\pysigline{\sphinxbfcode{\sphinxupquote{eps}}\sphinxbfcode{\sphinxupquote{\DUrole{p,p}{:}\DUrole{w,w}{  }float}}}
\pysigstopsignatures
\end{fulllineitems}

\index{forward() (deepdrr.network\_segmentation.ContBatchNorm3d method)@\spxentry{forward()}\spxextra{deepdrr.network\_segmentation.ContBatchNorm3d method}}

\begin{fulllineitems}
\phantomsection\label{\detokenize{deepdrr:deepdrr.network_segmentation.ContBatchNorm3d.forward}}
\pysigstartsignatures
\pysiglinewithargsret{\sphinxbfcode{\sphinxupquote{forward}}}{\sphinxparam{\DUrole{n,n}{input}}}{}
\pysigstopsignatures
\sphinxAtStartPar
Defines the computation performed at every call.

\sphinxAtStartPar
Should be overridden by all subclasses.

\begin{sphinxadmonition}{note}{Note:}
\sphinxAtStartPar
Although the recipe for forward pass needs to be defined within
this function, one should call the \sphinxcode{\sphinxupquote{Module}} instance afterwards
instead of this since the former takes care of running the
registered hooks while the latter silently ignores them.
\end{sphinxadmonition}

\end{fulllineitems}

\index{momentum (deepdrr.network\_segmentation.ContBatchNorm3d attribute)@\spxentry{momentum}\spxextra{deepdrr.network\_segmentation.ContBatchNorm3d attribute}}

\begin{fulllineitems}
\phantomsection\label{\detokenize{deepdrr:deepdrr.network_segmentation.ContBatchNorm3d.momentum}}
\pysigstartsignatures
\pysigline{\sphinxbfcode{\sphinxupquote{momentum}}\sphinxbfcode{\sphinxupquote{\DUrole{p,p}{:}\DUrole{w,w}{  }float}}}
\pysigstopsignatures
\end{fulllineitems}

\index{num\_features (deepdrr.network\_segmentation.ContBatchNorm3d attribute)@\spxentry{num\_features}\spxextra{deepdrr.network\_segmentation.ContBatchNorm3d attribute}}

\begin{fulllineitems}
\phantomsection\label{\detokenize{deepdrr:deepdrr.network_segmentation.ContBatchNorm3d.num_features}}
\pysigstartsignatures
\pysigline{\sphinxbfcode{\sphinxupquote{num\_features}}\sphinxbfcode{\sphinxupquote{\DUrole{p,p}{:}\DUrole{w,w}{  }int}}}
\pysigstopsignatures
\end{fulllineitems}

\index{track\_running\_stats (deepdrr.network\_segmentation.ContBatchNorm3d attribute)@\spxentry{track\_running\_stats}\spxextra{deepdrr.network\_segmentation.ContBatchNorm3d attribute}}

\begin{fulllineitems}
\phantomsection\label{\detokenize{deepdrr:deepdrr.network_segmentation.ContBatchNorm3d.track_running_stats}}
\pysigstartsignatures
\pysigline{\sphinxbfcode{\sphinxupquote{track\_running\_stats}}\sphinxbfcode{\sphinxupquote{\DUrole{p,p}{:}\DUrole{w,w}{  }bool}}}
\pysigstopsignatures
\end{fulllineitems}


\end{fulllineitems}

\index{DownTransition (class in deepdrr.network\_segmentation)@\spxentry{DownTransition}\spxextra{class in deepdrr.network\_segmentation}}

\begin{fulllineitems}
\phantomsection\label{\detokenize{deepdrr:deepdrr.network_segmentation.DownTransition}}
\pysigstartsignatures
\pysiglinewithargsret{\sphinxbfcode{\sphinxupquote{class\DUrole{w,w}{  }}}\sphinxcode{\sphinxupquote{deepdrr.network\_segmentation.}}\sphinxbfcode{\sphinxupquote{DownTransition}}}{\sphinxparam{\DUrole{n,n}{inChans}}\sphinxparamcomma \sphinxparam{\DUrole{n,n}{nConvs}}\sphinxparamcomma \sphinxparam{\DUrole{n,n}{elu}}\sphinxparamcomma \sphinxparam{\DUrole{n,n}{dropout}\DUrole{o,o}{=}\DUrole{default_value}{False}}}{}
\pysigstopsignatures
\sphinxAtStartPar
Bases: \sphinxcode{\sphinxupquote{Module}}
\index{forward() (deepdrr.network\_segmentation.DownTransition method)@\spxentry{forward()}\spxextra{deepdrr.network\_segmentation.DownTransition method}}

\begin{fulllineitems}
\phantomsection\label{\detokenize{deepdrr:deepdrr.network_segmentation.DownTransition.forward}}
\pysigstartsignatures
\pysiglinewithargsret{\sphinxbfcode{\sphinxupquote{forward}}}{\sphinxparam{\DUrole{n,n}{x}}}{}
\pysigstopsignatures
\sphinxAtStartPar
Defines the computation performed at every call.

\sphinxAtStartPar
Should be overridden by all subclasses.

\begin{sphinxadmonition}{note}{Note:}
\sphinxAtStartPar
Although the recipe for forward pass needs to be defined within
this function, one should call the \sphinxcode{\sphinxupquote{Module}} instance afterwards
instead of this since the former takes care of running the
registered hooks while the latter silently ignores them.
\end{sphinxadmonition}

\end{fulllineitems}

\index{training (deepdrr.network\_segmentation.DownTransition attribute)@\spxentry{training}\spxextra{deepdrr.network\_segmentation.DownTransition attribute}}

\begin{fulllineitems}
\phantomsection\label{\detokenize{deepdrr:deepdrr.network_segmentation.DownTransition.training}}
\pysigstartsignatures
\pysigline{\sphinxbfcode{\sphinxupquote{training}}\sphinxbfcode{\sphinxupquote{\DUrole{p,p}{:}\DUrole{w,w}{  }bool}}}
\pysigstopsignatures
\end{fulllineitems}


\end{fulllineitems}

\index{ELUCons() (in module deepdrr.network\_segmentation)@\spxentry{ELUCons()}\spxextra{in module deepdrr.network\_segmentation}}

\begin{fulllineitems}
\phantomsection\label{\detokenize{deepdrr:deepdrr.network_segmentation.ELUCons}}
\pysigstartsignatures
\pysiglinewithargsret{\sphinxcode{\sphinxupquote{deepdrr.network\_segmentation.}}\sphinxbfcode{\sphinxupquote{ELUCons}}}{\sphinxparam{\DUrole{n,n}{elu}}\sphinxparamcomma \sphinxparam{\DUrole{n,n}{nchan}}}{}
\pysigstopsignatures
\end{fulllineitems}

\index{InputTransition (class in deepdrr.network\_segmentation)@\spxentry{InputTransition}\spxextra{class in deepdrr.network\_segmentation}}

\begin{fulllineitems}
\phantomsection\label{\detokenize{deepdrr:deepdrr.network_segmentation.InputTransition}}
\pysigstartsignatures
\pysiglinewithargsret{\sphinxbfcode{\sphinxupquote{class\DUrole{w,w}{  }}}\sphinxcode{\sphinxupquote{deepdrr.network\_segmentation.}}\sphinxbfcode{\sphinxupquote{InputTransition}}}{\sphinxparam{\DUrole{n,n}{outChans}}\sphinxparamcomma \sphinxparam{\DUrole{n,n}{elu}}}{}
\pysigstopsignatures
\sphinxAtStartPar
Bases: \sphinxcode{\sphinxupquote{Module}}
\index{forward() (deepdrr.network\_segmentation.InputTransition method)@\spxentry{forward()}\spxextra{deepdrr.network\_segmentation.InputTransition method}}

\begin{fulllineitems}
\phantomsection\label{\detokenize{deepdrr:deepdrr.network_segmentation.InputTransition.forward}}
\pysigstartsignatures
\pysiglinewithargsret{\sphinxbfcode{\sphinxupquote{forward}}}{\sphinxparam{\DUrole{n,n}{x}}}{}
\pysigstopsignatures
\sphinxAtStartPar
Defines the computation performed at every call.

\sphinxAtStartPar
Should be overridden by all subclasses.

\begin{sphinxadmonition}{note}{Note:}
\sphinxAtStartPar
Although the recipe for forward pass needs to be defined within
this function, one should call the \sphinxcode{\sphinxupquote{Module}} instance afterwards
instead of this since the former takes care of running the
registered hooks while the latter silently ignores them.
\end{sphinxadmonition}

\end{fulllineitems}

\index{training (deepdrr.network\_segmentation.InputTransition attribute)@\spxentry{training}\spxextra{deepdrr.network\_segmentation.InputTransition attribute}}

\begin{fulllineitems}
\phantomsection\label{\detokenize{deepdrr:deepdrr.network_segmentation.InputTransition.training}}
\pysigstartsignatures
\pysigline{\sphinxbfcode{\sphinxupquote{training}}\sphinxbfcode{\sphinxupquote{\DUrole{p,p}{:}\DUrole{w,w}{  }bool}}}
\pysigstopsignatures
\end{fulllineitems}


\end{fulllineitems}

\index{LUConv (class in deepdrr.network\_segmentation)@\spxentry{LUConv}\spxextra{class in deepdrr.network\_segmentation}}

\begin{fulllineitems}
\phantomsection\label{\detokenize{deepdrr:deepdrr.network_segmentation.LUConv}}
\pysigstartsignatures
\pysiglinewithargsret{\sphinxbfcode{\sphinxupquote{class\DUrole{w,w}{  }}}\sphinxcode{\sphinxupquote{deepdrr.network\_segmentation.}}\sphinxbfcode{\sphinxupquote{LUConv}}}{\sphinxparam{\DUrole{n,n}{nchan}}\sphinxparamcomma \sphinxparam{\DUrole{n,n}{elu}}}{}
\pysigstopsignatures
\sphinxAtStartPar
Bases: \sphinxcode{\sphinxupquote{Module}}
\index{forward() (deepdrr.network\_segmentation.LUConv method)@\spxentry{forward()}\spxextra{deepdrr.network\_segmentation.LUConv method}}

\begin{fulllineitems}
\phantomsection\label{\detokenize{deepdrr:deepdrr.network_segmentation.LUConv.forward}}
\pysigstartsignatures
\pysiglinewithargsret{\sphinxbfcode{\sphinxupquote{forward}}}{\sphinxparam{\DUrole{n,n}{x}}}{}
\pysigstopsignatures
\sphinxAtStartPar
Defines the computation performed at every call.

\sphinxAtStartPar
Should be overridden by all subclasses.

\begin{sphinxadmonition}{note}{Note:}
\sphinxAtStartPar
Although the recipe for forward pass needs to be defined within
this function, one should call the \sphinxcode{\sphinxupquote{Module}} instance afterwards
instead of this since the former takes care of running the
registered hooks while the latter silently ignores them.
\end{sphinxadmonition}

\end{fulllineitems}

\index{training (deepdrr.network\_segmentation.LUConv attribute)@\spxentry{training}\spxextra{deepdrr.network\_segmentation.LUConv attribute}}

\begin{fulllineitems}
\phantomsection\label{\detokenize{deepdrr:deepdrr.network_segmentation.LUConv.training}}
\pysigstartsignatures
\pysigline{\sphinxbfcode{\sphinxupquote{training}}\sphinxbfcode{\sphinxupquote{\DUrole{p,p}{:}\DUrole{w,w}{  }bool}}}
\pysigstopsignatures
\end{fulllineitems}


\end{fulllineitems}

\index{OutputTransition (class in deepdrr.network\_segmentation)@\spxentry{OutputTransition}\spxextra{class in deepdrr.network\_segmentation}}

\begin{fulllineitems}
\phantomsection\label{\detokenize{deepdrr:deepdrr.network_segmentation.OutputTransition}}
\pysigstartsignatures
\pysiglinewithargsret{\sphinxbfcode{\sphinxupquote{class\DUrole{w,w}{  }}}\sphinxcode{\sphinxupquote{deepdrr.network\_segmentation.}}\sphinxbfcode{\sphinxupquote{OutputTransition}}}{\sphinxparam{\DUrole{n,n}{inChans}}\sphinxparamcomma \sphinxparam{\DUrole{n,n}{outChans}}\sphinxparamcomma \sphinxparam{\DUrole{n,n}{elu}}\sphinxparamcomma \sphinxparam{\DUrole{n,n}{nll}}}{}
\pysigstopsignatures
\sphinxAtStartPar
Bases: \sphinxcode{\sphinxupquote{Module}}
\index{forward() (deepdrr.network\_segmentation.OutputTransition method)@\spxentry{forward()}\spxextra{deepdrr.network\_segmentation.OutputTransition method}}

\begin{fulllineitems}
\phantomsection\label{\detokenize{deepdrr:deepdrr.network_segmentation.OutputTransition.forward}}
\pysigstartsignatures
\pysiglinewithargsret{\sphinxbfcode{\sphinxupquote{forward}}}{\sphinxparam{\DUrole{n,n}{x}}}{}
\pysigstopsignatures
\sphinxAtStartPar
Defines the computation performed at every call.

\sphinxAtStartPar
Should be overridden by all subclasses.

\begin{sphinxadmonition}{note}{Note:}
\sphinxAtStartPar
Although the recipe for forward pass needs to be defined within
this function, one should call the \sphinxcode{\sphinxupquote{Module}} instance afterwards
instead of this since the former takes care of running the
registered hooks while the latter silently ignores them.
\end{sphinxadmonition}

\end{fulllineitems}

\index{training (deepdrr.network\_segmentation.OutputTransition attribute)@\spxentry{training}\spxextra{deepdrr.network\_segmentation.OutputTransition attribute}}

\begin{fulllineitems}
\phantomsection\label{\detokenize{deepdrr:deepdrr.network_segmentation.OutputTransition.training}}
\pysigstartsignatures
\pysigline{\sphinxbfcode{\sphinxupquote{training}}\sphinxbfcode{\sphinxupquote{\DUrole{p,p}{:}\DUrole{w,w}{  }bool}}}
\pysigstopsignatures
\end{fulllineitems}


\end{fulllineitems}

\index{UpTransition (class in deepdrr.network\_segmentation)@\spxentry{UpTransition}\spxextra{class in deepdrr.network\_segmentation}}

\begin{fulllineitems}
\phantomsection\label{\detokenize{deepdrr:deepdrr.network_segmentation.UpTransition}}
\pysigstartsignatures
\pysiglinewithargsret{\sphinxbfcode{\sphinxupquote{class\DUrole{w,w}{  }}}\sphinxcode{\sphinxupquote{deepdrr.network\_segmentation.}}\sphinxbfcode{\sphinxupquote{UpTransition}}}{\sphinxparam{\DUrole{n,n}{inChans}}\sphinxparamcomma \sphinxparam{\DUrole{n,n}{outChans}}\sphinxparamcomma \sphinxparam{\DUrole{n,n}{nConvs}}\sphinxparamcomma \sphinxparam{\DUrole{n,n}{elu}}\sphinxparamcomma \sphinxparam{\DUrole{n,n}{dropout}\DUrole{o,o}{=}\DUrole{default_value}{False}}}{}
\pysigstopsignatures
\sphinxAtStartPar
Bases: \sphinxcode{\sphinxupquote{Module}}
\index{forward() (deepdrr.network\_segmentation.UpTransition method)@\spxentry{forward()}\spxextra{deepdrr.network\_segmentation.UpTransition method}}

\begin{fulllineitems}
\phantomsection\label{\detokenize{deepdrr:deepdrr.network_segmentation.UpTransition.forward}}
\pysigstartsignatures
\pysiglinewithargsret{\sphinxbfcode{\sphinxupquote{forward}}}{\sphinxparam{\DUrole{n,n}{x}}\sphinxparamcomma \sphinxparam{\DUrole{n,n}{skipx}}}{}
\pysigstopsignatures
\sphinxAtStartPar
Defines the computation performed at every call.

\sphinxAtStartPar
Should be overridden by all subclasses.

\begin{sphinxadmonition}{note}{Note:}
\sphinxAtStartPar
Although the recipe for forward pass needs to be defined within
this function, one should call the \sphinxcode{\sphinxupquote{Module}} instance afterwards
instead of this since the former takes care of running the
registered hooks while the latter silently ignores them.
\end{sphinxadmonition}

\end{fulllineitems}

\index{training (deepdrr.network\_segmentation.UpTransition attribute)@\spxentry{training}\spxextra{deepdrr.network\_segmentation.UpTransition attribute}}

\begin{fulllineitems}
\phantomsection\label{\detokenize{deepdrr:deepdrr.network_segmentation.UpTransition.training}}
\pysigstartsignatures
\pysigline{\sphinxbfcode{\sphinxupquote{training}}\sphinxbfcode{\sphinxupquote{\DUrole{p,p}{:}\DUrole{w,w}{  }bool}}}
\pysigstopsignatures
\end{fulllineitems}


\end{fulllineitems}

\index{VNet (class in deepdrr.network\_segmentation)@\spxentry{VNet}\spxextra{class in deepdrr.network\_segmentation}}

\begin{fulllineitems}
\phantomsection\label{\detokenize{deepdrr:deepdrr.network_segmentation.VNet}}
\pysigstartsignatures
\pysiglinewithargsret{\sphinxbfcode{\sphinxupquote{class\DUrole{w,w}{  }}}\sphinxcode{\sphinxupquote{deepdrr.network\_segmentation.}}\sphinxbfcode{\sphinxupquote{VNet}}}{\sphinxparam{\DUrole{n,n}{elu}\DUrole{o,o}{=}\DUrole{default_value}{False}}\sphinxparamcomma \sphinxparam{\DUrole{n,n}{nll}\DUrole{o,o}{=}\DUrole{default_value}{False}}}{}
\pysigstopsignatures
\sphinxAtStartPar
Bases: \sphinxcode{\sphinxupquote{Module}}
\index{forward() (deepdrr.network\_segmentation.VNet method)@\spxentry{forward()}\spxextra{deepdrr.network\_segmentation.VNet method}}

\begin{fulllineitems}
\phantomsection\label{\detokenize{deepdrr:deepdrr.network_segmentation.VNet.forward}}
\pysigstartsignatures
\pysiglinewithargsret{\sphinxbfcode{\sphinxupquote{forward}}}{\sphinxparam{\DUrole{n,n}{x}}}{}
\pysigstopsignatures
\sphinxAtStartPar
Defines the computation performed at every call.

\sphinxAtStartPar
Should be overridden by all subclasses.

\begin{sphinxadmonition}{note}{Note:}
\sphinxAtStartPar
Although the recipe for forward pass needs to be defined within
this function, one should call the \sphinxcode{\sphinxupquote{Module}} instance afterwards
instead of this since the former takes care of running the
registered hooks while the latter silently ignores them.
\end{sphinxadmonition}

\end{fulllineitems}

\index{training (deepdrr.network\_segmentation.VNet attribute)@\spxentry{training}\spxextra{deepdrr.network\_segmentation.VNet attribute}}

\begin{fulllineitems}
\phantomsection\label{\detokenize{deepdrr:deepdrr.network_segmentation.VNet.training}}
\pysigstartsignatures
\pysigline{\sphinxbfcode{\sphinxupquote{training}}\sphinxbfcode{\sphinxupquote{\DUrole{p,p}{:}\DUrole{w,w}{  }bool}}}
\pysigstopsignatures
\end{fulllineitems}


\end{fulllineitems}

\index{passthrough() (in module deepdrr.network\_segmentation)@\spxentry{passthrough()}\spxextra{in module deepdrr.network\_segmentation}}

\begin{fulllineitems}
\phantomsection\label{\detokenize{deepdrr:deepdrr.network_segmentation.passthrough}}
\pysigstartsignatures
\pysiglinewithargsret{\sphinxcode{\sphinxupquote{deepdrr.network\_segmentation.}}\sphinxbfcode{\sphinxupquote{passthrough}}}{\sphinxparam{\DUrole{n,n}{x}}\sphinxparamcomma \sphinxparam{\DUrole{o,o}{**}\DUrole{n,n}{kwargs}}}{}
\pysigstopsignatures
\end{fulllineitems}



\section{deepdrr.segmentation module}
\label{\detokenize{deepdrr:module-deepdrr.segmentation}}\label{\detokenize{deepdrr:deepdrr-segmentation-module}}\index{module@\spxentry{module}!deepdrr.segmentation@\spxentry{deepdrr.segmentation}}\index{deepdrr.segmentation@\spxentry{deepdrr.segmentation}!module@\spxentry{module}}\index{SegmentationNet (class in deepdrr.segmentation)@\spxentry{SegmentationNet}\spxextra{class in deepdrr.segmentation}}

\begin{fulllineitems}
\phantomsection\label{\detokenize{deepdrr:deepdrr.segmentation.SegmentationNet}}
\pysigstartsignatures
\pysigline{\sphinxbfcode{\sphinxupquote{class\DUrole{w,w}{  }}}\sphinxcode{\sphinxupquote{deepdrr.segmentation.}}\sphinxbfcode{\sphinxupquote{SegmentationNet}}}
\pysigstopsignatures
\sphinxAtStartPar
Bases: \sphinxcode{\sphinxupquote{object}}
\index{download() (deepdrr.segmentation.SegmentationNet method)@\spxentry{download()}\spxextra{deepdrr.segmentation.SegmentationNet method}}

\begin{fulllineitems}
\phantomsection\label{\detokenize{deepdrr:deepdrr.segmentation.SegmentationNet.download}}
\pysigstartsignatures
\pysiglinewithargsret{\sphinxbfcode{\sphinxupquote{download}}}{}{{ $\rightarrow$ Path}}
\pysigstopsignatures
\end{fulllineitems}

\index{filename (deepdrr.segmentation.SegmentationNet attribute)@\spxentry{filename}\spxextra{deepdrr.segmentation.SegmentationNet attribute}}

\begin{fulllineitems}
\phantomsection\label{\detokenize{deepdrr:deepdrr.segmentation.SegmentationNet.filename}}
\pysigstartsignatures
\pysigline{\sphinxbfcode{\sphinxupquote{filename}}\sphinxbfcode{\sphinxupquote{\DUrole{w,w}{  }\DUrole{p,p}{=}\DUrole{w,w}{  }\textquotesingle{}model\_segmentation.pth.tar\textquotesingle{}}}}
\pysigstopsignatures
\end{fulllineitems}

\index{segment() (deepdrr.segmentation.SegmentationNet method)@\spxentry{segment()}\spxextra{deepdrr.segmentation.SegmentationNet method}}

\begin{fulllineitems}
\phantomsection\label{\detokenize{deepdrr:deepdrr.segmentation.SegmentationNet.segment}}
\pysigstartsignatures
\pysiglinewithargsret{\sphinxbfcode{\sphinxupquote{segment}}}{\sphinxparam{\DUrole{n,n}{input\_volume}}\sphinxparamcomma \sphinxparam{\DUrole{n,n}{show\_results}\DUrole{o,o}{=}\DUrole{default_value}{False}}}{}
\pysigstopsignatures
\end{fulllineitems}

\index{url (deepdrr.segmentation.SegmentationNet attribute)@\spxentry{url}\spxextra{deepdrr.segmentation.SegmentationNet attribute}}

\begin{fulllineitems}
\phantomsection\label{\detokenize{deepdrr:deepdrr.segmentation.SegmentationNet.url}}
\pysigstartsignatures
\pysigline{\sphinxbfcode{\sphinxupquote{url}}\sphinxbfcode{\sphinxupquote{\DUrole{w,w}{  }\DUrole{p,p}{=}\DUrole{w,w}{  }\textquotesingle{}https://www.dropbox.com/s/pn4aw4z2i01eoo4/model\_segmentation.pth.tar?dl=1\textquotesingle{}}}}
\pysigstopsignatures
\end{fulllineitems}


\end{fulllineitems}



\section{deepdrr.vis module}
\label{\detokenize{deepdrr:module-deepdrr.vis}}\label{\detokenize{deepdrr:deepdrr-vis-module}}\index{module@\spxentry{module}!deepdrr.vis@\spxentry{deepdrr.vis}}\index{deepdrr.vis@\spxentry{deepdrr.vis}!module@\spxentry{module}}
\sphinxAtStartPar
Visualization functions for DeepDRR.

\sphinxAtStartPar
DeepDRR uses pyvista to visualize the volumes and devices in 3D.
This is useful for debugging and verification purposes, but it
is not meant to replace a purpose built renderer. To view a scene
in detail, save the meshes to disk and open them in a suitable
viewer, such as 3D Slicer.

\sphinxAtStartPar
Note that these visualizations have the same limitations as PyVista.
They may not function properly in Jupyter notebooks.

\sphinxAtStartPar
Any object with the \sphinxtitleref{get\_mesh\_in\_world()} method can be visualized.

\sphinxAtStartPar
NOTE: often, PyVista will not render in an ssh window. To fix this, try some of the following:
\sphinxcode{\sphinxupquote{\textasciigrave{}bash
\#!/bin/bash
sudo apt\sphinxhyphen{}get install xvfb
export DISPLAY=:99.0
export PYVISTA\_OFF\_SCREEN=true
export PYVISTA\_USE\_IPYVTK=true
export MESA\_GL\_VERSION\_OVERRIDE=3.2
export MESA\_GLSL\_VERSION\_OVERRIDE=150
Xvfb :99 \sphinxhyphen{}screen 0 1024x768x24 \textgreater{} /dev/null 2\textgreater{}\&1 \&
sleep 3
\textasciigrave{}}}
\index{get\_frustum\_mesh() (in module deepdrr.vis)@\spxentry{get\_frustum\_mesh()}\spxextra{in module deepdrr.vis}}

\begin{fulllineitems}
\phantomsection\label{\detokenize{deepdrr:deepdrr.vis.get_frustum_mesh}}
\pysigstartsignatures
\pysiglinewithargsret{\sphinxcode{\sphinxupquote{deepdrr.vis.}}\sphinxbfcode{\sphinxupquote{get\_frustum\_mesh}}}{\sphinxparam{\DUrole{n,n}{camera\_projection}\DUrole{p,p}{:}\DUrole{w,w}{  }\DUrole{n,n}{{\hyperref[\detokenize{deepdrr.geo:deepdrr.geo.core.CameraProjection}]{\sphinxcrossref{CameraProjection}}}}}\sphinxparamcomma \sphinxparam{\DUrole{n,n}{pixel\_size}\DUrole{p,p}{:}\DUrole{w,w}{  }\DUrole{n,n}{float}}\sphinxparamcomma \sphinxparam{\DUrole{n,n}{image\_path}\DUrole{p,p}{:}\DUrole{w,w}{  }\DUrole{n,n}{str\DUrole{w,w}{  }\DUrole{p,p}{|}\DUrole{w,w}{  }None}\DUrole{w,w}{  }\DUrole{o,o}{=}\DUrole{w,w}{  }\DUrole{default_value}{None}}\sphinxparamcomma \sphinxparam{\DUrole{n,n}{image\_plane\_distance}\DUrole{p,p}{:}\DUrole{w,w}{  }\DUrole{n,n}{float\DUrole{w,w}{  }\DUrole{p,p}{|}\DUrole{w,w}{  }None}\DUrole{w,w}{  }\DUrole{o,o}{=}\DUrole{w,w}{  }\DUrole{default_value}{None}}\sphinxparamcomma \sphinxparam{\DUrole{n,n}{full\_frustum}\DUrole{p,p}{:}\DUrole{w,w}{  }\DUrole{n,n}{bool}\DUrole{w,w}{  }\DUrole{o,o}{=}\DUrole{w,w}{  }\DUrole{default_value}{True}}}{{ $\rightarrow$ PolyData}}
\pysigstopsignatures
\sphinxAtStartPar
Get a really simple camera mesh for the camera projections.
\begin{quote}\begin{description}
\sphinxlineitem{Parameters}\begin{itemize}
\item {} 
\sphinxAtStartPar
\sphinxstyleliteralstrong{\sphinxupquote{camera\_projection}} ({\hyperref[\detokenize{deepdrr.geo:deepdrr.geo.CameraProjection}]{\sphinxcrossref{\sphinxstyleliteralemphasis{\sphinxupquote{geo.CameraProjection}}}}}) \textendash{} The camera projection.

\item {} 
\sphinxAtStartPar
\sphinxstyleliteralstrong{\sphinxupquote{pixel\_size}} (\sphinxstyleliteralemphasis{\sphinxupquote{float}}) \textendash{} The pixel size in mm.

\item {} 
\sphinxAtStartPar
\sphinxstyleliteralstrong{\sphinxupquote{image\_path}} (\sphinxstyleliteralemphasis{\sphinxupquote{str}}\sphinxstyleliteralemphasis{\sphinxupquote{, }}\sphinxstyleliteralemphasis{\sphinxupquote{optional}}) \textendash{} The path to the image. Defaults to None.

\item {} 
\sphinxAtStartPar
\sphinxstyleliteralstrong{\sphinxupquote{image\_plane\_distance}} (\sphinxstyleliteralemphasis{\sphinxupquote{float}}\sphinxstyleliteralemphasis{\sphinxupquote{, }}\sphinxstyleliteralemphasis{\sphinxupquote{optional}}) \textendash{} The distance from the camera to the image plane visualization. Defaults to None,
which uses the distance from the camera to the image plane.

\item {} 
\sphinxAtStartPar
\sphinxstyleliteralstrong{\sphinxupquote{full\_frustum}} (\sphinxstyleliteralemphasis{\sphinxupquote{bool}}\sphinxstyleliteralemphasis{\sphinxupquote{, }}\sphinxstyleliteralemphasis{\sphinxupquote{optional}}) \textendash{} Whether to show the full frustum, or just the principle ray. Defaults to True.

\end{itemize}

\sphinxlineitem{Returns}
\sphinxAtStartPar
Mesh representing the C\sphinxhyphen{}arm frustum.

\sphinxlineitem{Return type}
\sphinxAtStartPar
pv.PolyData

\end{description}\end{quote}

\end{fulllineitems}

\index{show() (in module deepdrr.vis)@\spxentry{show()}\spxextra{in module deepdrr.vis}}

\begin{fulllineitems}
\phantomsection\label{\detokenize{deepdrr:deepdrr.vis.show}}
\pysigstartsignatures
\pysiglinewithargsret{\sphinxcode{\sphinxupquote{deepdrr.vis.}}\sphinxbfcode{\sphinxupquote{show}}}{\sphinxparam{\DUrole{o,o}{*}\DUrole{n,n}{item}\DUrole{p,p}{:}\DUrole{w,w}{  }\DUrole{n,n}{Any}}\sphinxparamcomma \sphinxparam{\DUrole{n,n}{full}\DUrole{p,p}{:}\DUrole{w,w}{  }\DUrole{n,n}{bool\DUrole{w,w}{  }\DUrole{p,p}{|}\DUrole{w,w}{  }List\DUrole{p,p}{{[}}bool\DUrole{p,p}{{]}}}\DUrole{w,w}{  }\DUrole{o,o}{=}\DUrole{w,w}{  }\DUrole{default_value}{False}}\sphinxparamcomma \sphinxparam{\DUrole{n,n}{colors}\DUrole{p,p}{:}\DUrole{w,w}{  }\DUrole{n,n}{List\DUrole{p,p}{{[}}str\DUrole{p,p}{{]}}}\DUrole{w,w}{  }\DUrole{o,o}{=}\DUrole{w,w}{  }\DUrole{default_value}{{[}\textquotesingle{}tan\textquotesingle{}, \textquotesingle{}cyan\textquotesingle{}, \textquotesingle{}green\textquotesingle{}, \textquotesingle{}red\textquotesingle{}{]}}}\sphinxparamcomma \sphinxparam{\DUrole{n,n}{background}\DUrole{p,p}{:}\DUrole{w,w}{  }\DUrole{n,n}{str}\DUrole{w,w}{  }\DUrole{o,o}{=}\DUrole{w,w}{  }\DUrole{default_value}{\textquotesingle{}white\textquotesingle{}}}\sphinxparamcomma \sphinxparam{\DUrole{n,n}{use\_cached}\DUrole{p,p}{:}\DUrole{w,w}{  }\DUrole{n,n}{bool\DUrole{w,w}{  }\DUrole{p,p}{|}\DUrole{w,w}{  }List\DUrole{p,p}{{[}}bool\DUrole{p,p}{{]}}}\DUrole{w,w}{  }\DUrole{o,o}{=}\DUrole{w,w}{  }\DUrole{default_value}{True}}\sphinxparamcomma \sphinxparam{\DUrole{n,n}{offscreen}\DUrole{p,p}{:}\DUrole{w,w}{  }\DUrole{n,n}{bool}\DUrole{w,w}{  }\DUrole{o,o}{=}\DUrole{w,w}{  }\DUrole{default_value}{False}}\sphinxparamcomma \sphinxparam{\DUrole{n,n}{mesh}\DUrole{p,p}{:}\DUrole{w,w}{  }\DUrole{n,n}{PolyData\DUrole{w,w}{  }\DUrole{p,p}{|}\DUrole{w,w}{  }None}\DUrole{w,w}{  }\DUrole{o,o}{=}\DUrole{w,w}{  }\DUrole{default_value}{None}}\sphinxparamcomma \sphinxparam{\DUrole{n,n}{mesh\_color}\DUrole{p,p}{:}\DUrole{w,w}{  }\DUrole{n,n}{str}\DUrole{w,w}{  }\DUrole{o,o}{=}\DUrole{w,w}{  }\DUrole{default_value}{\textquotesingle{}black\textquotesingle{}}}}{{ $\rightarrow$ ndarray\DUrole{w,w}{  }\DUrole{p,p}{|}\DUrole{w,w}{  }None}}
\pysigstopsignatures
\sphinxAtStartPar
Show the given items in a pyvista window.
\begin{quote}\begin{description}
\sphinxlineitem{Parameters}
\sphinxAtStartPar
\sphinxstyleliteralstrong{\sphinxupquote{full}} (\sphinxstyleliteralemphasis{\sphinxupquote{bool}}\sphinxstyleliteralemphasis{\sphinxupquote{, }}\sphinxstyleliteralemphasis{\sphinxupquote{optional}}) \textendash{} {[}description{]}. Defaults to True.

\end{description}\end{quote}

\end{fulllineitems}



\section{Module contents}
\label{\detokenize{deepdrr:module-deepdrr}}\label{\detokenize{deepdrr:module-contents}}\index{module@\spxentry{module}!deepdrr@\spxentry{deepdrr}}\index{deepdrr@\spxentry{deepdrr}!module@\spxentry{module}}\index{CArm (class in deepdrr)@\spxentry{CArm}\spxextra{class in deepdrr}}

\begin{fulllineitems}
\phantomsection\label{\detokenize{deepdrr:deepdrr.CArm}}
\pysigstartsignatures
\pysiglinewithargsret{\sphinxbfcode{\sphinxupquote{class\DUrole{w,w}{  }}}\sphinxcode{\sphinxupquote{deepdrr.}}\sphinxbfcode{\sphinxupquote{CArm}}}{\sphinxparam{\DUrole{n,n}{isocenter\_distance}\DUrole{p,p}{:}\DUrole{w,w}{  }\DUrole{n,n}{float}}\sphinxparamcomma \sphinxparam{\DUrole{n,n}{isocenter}\DUrole{p,p}{:}\DUrole{w,w}{  }\DUrole{n,n}{{\hyperref[\detokenize{deepdrr.geo:deepdrr.geo.core.Point3D}]{\sphinxcrossref{Point3D}}}\DUrole{w,w}{  }\DUrole{p,p}{|}\DUrole{w,w}{  }None}\DUrole{w,w}{  }\DUrole{o,o}{=}\DUrole{w,w}{  }\DUrole{default_value}{None}}\sphinxparamcomma \sphinxparam{\DUrole{n,n}{phi}\DUrole{p,p}{:}\DUrole{w,w}{  }\DUrole{n,n}{float}\DUrole{w,w}{  }\DUrole{o,o}{=}\DUrole{w,w}{  }\DUrole{default_value}{0}}\sphinxparamcomma \sphinxparam{\DUrole{n,n}{theta}\DUrole{p,p}{:}\DUrole{w,w}{  }\DUrole{n,n}{float}\DUrole{w,w}{  }\DUrole{o,o}{=}\DUrole{w,w}{  }\DUrole{default_value}{0}}\sphinxparamcomma \sphinxparam{\DUrole{n,n}{rho}\DUrole{p,p}{:}\DUrole{w,w}{  }\DUrole{n,n}{float}\DUrole{w,w}{  }\DUrole{o,o}{=}\DUrole{w,w}{  }\DUrole{default_value}{0}}\sphinxparamcomma \sphinxparam{\DUrole{n,n}{degrees}\DUrole{p,p}{:}\DUrole{w,w}{  }\DUrole{n,n}{bool}\DUrole{w,w}{  }\DUrole{o,o}{=}\DUrole{w,w}{  }\DUrole{default_value}{False}}}{}
\pysigstopsignatures
\sphinxAtStartPar
Bases: \sphinxcode{\sphinxupquote{object}}

\sphinxAtStartPar
C\sphinxhyphen{}arm device for positioning a camera in space.

\sphinxAtStartPar
It is suggested to use MobileCArm instead.
\index{camera3d\_from\_world (deepdrr.CArm property)@\spxentry{camera3d\_from\_world}\spxextra{deepdrr.CArm property}}

\begin{fulllineitems}
\phantomsection\label{\detokenize{deepdrr:deepdrr.CArm.camera3d_from_world}}
\pysigstartsignatures
\pysigline{\sphinxbfcode{\sphinxupquote{property\DUrole{w,w}{  }}}\sphinxbfcode{\sphinxupquote{camera3d\_from\_world}}\sphinxbfcode{\sphinxupquote{\DUrole{p,p}{:}\DUrole{w,w}{  }{\hyperref[\detokenize{deepdrr.geo:deepdrr.geo.core.FrameTransform}]{\sphinxcrossref{FrameTransform}}}}}}
\pysigstopsignatures
\end{fulllineitems}

\index{get\_camera3d\_from\_world() (deepdrr.CArm method)@\spxentry{get\_camera3d\_from\_world()}\spxextra{deepdrr.CArm method}}

\begin{fulllineitems}
\phantomsection\label{\detokenize{deepdrr:deepdrr.CArm.get_camera3d_from_world}}
\pysigstartsignatures
\pysiglinewithargsret{\sphinxbfcode{\sphinxupquote{get\_camera3d\_from\_world}}}{\sphinxparam{\DUrole{n,n}{isocenter}\DUrole{p,p}{:}\DUrole{w,w}{  }\DUrole{n,n}{{\hyperref[\detokenize{deepdrr.geo:deepdrr.geo.core.Point3D}]{\sphinxcrossref{Point3D}}}}}\sphinxparamcomma \sphinxparam{\DUrole{n,n}{phi}\DUrole{p,p}{:}\DUrole{w,w}{  }\DUrole{n,n}{float}}\sphinxparamcomma \sphinxparam{\DUrole{n,n}{theta}\DUrole{p,p}{:}\DUrole{w,w}{  }\DUrole{n,n}{float}}\sphinxparamcomma \sphinxparam{\DUrole{n,n}{rho}\DUrole{p,p}{:}\DUrole{w,w}{  }\DUrole{n,n}{float\DUrole{w,w}{  }\DUrole{p,p}{|}\DUrole{w,w}{  }None}\DUrole{w,w}{  }\DUrole{o,o}{=}\DUrole{w,w}{  }\DUrole{default_value}{0}}\sphinxparamcomma \sphinxparam{\DUrole{n,n}{degrees}\DUrole{p,p}{:}\DUrole{w,w}{  }\DUrole{n,n}{bool}\DUrole{w,w}{  }\DUrole{o,o}{=}\DUrole{w,w}{  }\DUrole{default_value}{False}}}{{ $\rightarrow$ {\hyperref[\detokenize{deepdrr.geo:deepdrr.geo.core.FrameTransform}]{\sphinxcrossref{FrameTransform}}}}}
\pysigstopsignatures
\sphinxAtStartPar
Get the FrameTransform for the C\sphinxhyphen{}Arm device at the given pose.

\sphinxAtStartPar
This ignores the internal state except for the isocenter\_distance.
\begin{quote}\begin{description}
\sphinxlineitem{Parameters}\begin{itemize}
\item {} 
\sphinxAtStartPar
\sphinxstyleliteralstrong{\sphinxupquote{isocenter}} ({\hyperref[\detokenize{deepdrr.geo:deepdrr.geo.Point3D}]{\sphinxcrossref{\sphinxstyleliteralemphasis{\sphinxupquote{geo.Point3D}}}}}) \textendash{} isocenter of the device.

\item {} 
\sphinxAtStartPar
\sphinxstyleliteralstrong{\sphinxupquote{phi}} (\sphinxstyleliteralemphasis{\sphinxupquote{float}}) \textendash{} CRAN/CAUD angle of the C\sphinxhyphen{}Arm (along the actual arc of the arm)

\item {} 
\sphinxAtStartPar
\sphinxstyleliteralstrong{\sphinxupquote{theta}} (\sphinxstyleliteralemphasis{\sphinxupquote{float}}) \textendash{} Lect/Right angulation of C\sphinxhyphen{}arm (rotation at the base)

\item {} 
\sphinxAtStartPar
\sphinxstyleliteralstrong{\sphinxupquote{rho}} (\sphinxstyleliteralemphasis{\sphinxupquote{Optional}}\sphinxstyleliteralemphasis{\sphinxupquote{{[}}}\sphinxstyleliteralemphasis{\sphinxupquote{float}}\sphinxstyleliteralemphasis{\sphinxupquote{{]}}}\sphinxstyleliteralemphasis{\sphinxupquote{, }}\sphinxstyleliteralemphasis{\sphinxupquote{optional}}) \textendash{} rotation about principle axis, after main rotation. Defaults to 0.

\item {} 
\sphinxAtStartPar
\sphinxstyleliteralstrong{\sphinxupquote{degrees}} (\sphinxstyleliteralemphasis{\sphinxupquote{bool}}\sphinxstyleliteralemphasis{\sphinxupquote{, }}\sphinxstyleliteralemphasis{\sphinxupquote{optional}}) \textendash{} Whether given angles are in degrees. Defaults to False.

\item {} 
\sphinxAtStartPar
\sphinxstyleliteralstrong{\sphinxupquote{offset}} (\sphinxstyleliteralemphasis{\sphinxupquote{Optional}}\sphinxstyleliteralemphasis{\sphinxupquote{{[}}}{\hyperref[\detokenize{deepdrr.geo:deepdrr.geo.core.Vector3D}]{\sphinxcrossref{\sphinxstyleliteralemphasis{\sphinxupquote{Vector3D}}}}}\sphinxstyleliteralemphasis{\sphinxupquote{{]}}}\sphinxstyleliteralemphasis{\sphinxupquote{, }}\sphinxstyleliteralemphasis{\sphinxupquote{optional}}) \textendash{} world\sphinxhyphen{}space offset to add to the initial C\sphinxhyphen{}arm isocenter. Defaults to None.

\end{itemize}

\sphinxlineitem{Returns}
\sphinxAtStartPar
the extrinsic matrix or “camera3d\_from\_world” frame transformation for the oriented C\sphinxhyphen{}Arm camera.

\sphinxlineitem{Return type}
\sphinxAtStartPar
{\hyperref[\detokenize{deepdrr.geo:deepdrr.geo.core.FrameTransform}]{\sphinxcrossref{FrameTransform}}}

\end{description}\end{quote}

\end{fulllineitems}

\index{move\_by() (deepdrr.CArm method)@\spxentry{move\_by()}\spxextra{deepdrr.CArm method}}

\begin{fulllineitems}
\phantomsection\label{\detokenize{deepdrr:deepdrr.CArm.move_by}}
\pysigstartsignatures
\pysiglinewithargsret{\sphinxbfcode{\sphinxupquote{move\_by}}}{\sphinxparam{\DUrole{n,n}{delta\_isocenter}\DUrole{p,p}{:}\DUrole{w,w}{  }\DUrole{n,n}{{\hyperref[\detokenize{deepdrr.geo:deepdrr.geo.core.Vector3D}]{\sphinxcrossref{Vector3D}}}\DUrole{w,w}{  }\DUrole{p,p}{|}\DUrole{w,w}{  }None}\DUrole{w,w}{  }\DUrole{o,o}{=}\DUrole{w,w}{  }\DUrole{default_value}{None}}\sphinxparamcomma \sphinxparam{\DUrole{n,n}{delta\_phi}\DUrole{p,p}{:}\DUrole{w,w}{  }\DUrole{n,n}{float\DUrole{w,w}{  }\DUrole{p,p}{|}\DUrole{w,w}{  }None}\DUrole{w,w}{  }\DUrole{o,o}{=}\DUrole{w,w}{  }\DUrole{default_value}{None}}\sphinxparamcomma \sphinxparam{\DUrole{n,n}{delta\_theta}\DUrole{p,p}{:}\DUrole{w,w}{  }\DUrole{n,n}{float\DUrole{w,w}{  }\DUrole{p,p}{|}\DUrole{w,w}{  }None}\DUrole{w,w}{  }\DUrole{o,o}{=}\DUrole{w,w}{  }\DUrole{default_value}{None}}\sphinxparamcomma \sphinxparam{\DUrole{n,n}{delta\_rho}\DUrole{p,p}{:}\DUrole{w,w}{  }\DUrole{n,n}{float\DUrole{w,w}{  }\DUrole{p,p}{|}\DUrole{w,w}{  }None}\DUrole{w,w}{  }\DUrole{o,o}{=}\DUrole{w,w}{  }\DUrole{default_value}{None}}\sphinxparamcomma \sphinxparam{\DUrole{n,n}{degrees}\DUrole{p,p}{:}\DUrole{w,w}{  }\DUrole{n,n}{bool}\DUrole{w,w}{  }\DUrole{o,o}{=}\DUrole{w,w}{  }\DUrole{default_value}{False}}\sphinxparamcomma \sphinxparam{\DUrole{n,n}{min\_isocenter}\DUrole{p,p}{:}\DUrole{w,w}{  }\DUrole{n,n}{{\hyperref[\detokenize{deepdrr.geo:deepdrr.geo.core.Point3D}]{\sphinxcrossref{Point3D}}}\DUrole{w,w}{  }\DUrole{p,p}{|}\DUrole{w,w}{  }None}\DUrole{w,w}{  }\DUrole{o,o}{=}\DUrole{w,w}{  }\DUrole{default_value}{None}}\sphinxparamcomma \sphinxparam{\DUrole{n,n}{max\_isocenter}\DUrole{p,p}{:}\DUrole{w,w}{  }\DUrole{n,n}{{\hyperref[\detokenize{deepdrr.geo:deepdrr.geo.core.Point3D}]{\sphinxcrossref{Point3D}}}\DUrole{w,w}{  }\DUrole{p,p}{|}\DUrole{w,w}{  }None}\DUrole{w,w}{  }\DUrole{o,o}{=}\DUrole{w,w}{  }\DUrole{default_value}{None}}\sphinxparamcomma \sphinxparam{\DUrole{n,n}{min\_phi}\DUrole{p,p}{:}\DUrole{w,w}{  }\DUrole{n,n}{float\DUrole{w,w}{  }\DUrole{p,p}{|}\DUrole{w,w}{  }None}\DUrole{w,w}{  }\DUrole{o,o}{=}\DUrole{w,w}{  }\DUrole{default_value}{None}}\sphinxparamcomma \sphinxparam{\DUrole{n,n}{max\_phi}\DUrole{p,p}{:}\DUrole{w,w}{  }\DUrole{n,n}{float\DUrole{w,w}{  }\DUrole{p,p}{|}\DUrole{w,w}{  }None}\DUrole{w,w}{  }\DUrole{o,o}{=}\DUrole{w,w}{  }\DUrole{default_value}{None}}\sphinxparamcomma \sphinxparam{\DUrole{n,n}{min\_theta}\DUrole{p,p}{:}\DUrole{w,w}{  }\DUrole{n,n}{float\DUrole{w,w}{  }\DUrole{p,p}{|}\DUrole{w,w}{  }None}\DUrole{w,w}{  }\DUrole{o,o}{=}\DUrole{w,w}{  }\DUrole{default_value}{None}}\sphinxparamcomma \sphinxparam{\DUrole{n,n}{max\_theta}\DUrole{p,p}{:}\DUrole{w,w}{  }\DUrole{n,n}{float\DUrole{w,w}{  }\DUrole{p,p}{|}\DUrole{w,w}{  }None}\DUrole{w,w}{  }\DUrole{o,o}{=}\DUrole{w,w}{  }\DUrole{default_value}{None}}}{{ $\rightarrow$ None}}
\pysigstopsignatures
\sphinxAtStartPar
Move the C\sphinxhyphen{}arm by the specified deltas.

\sphinxAtStartPar
Clips the internal state by the provided values if not None.
\begin{quote}\begin{description}
\sphinxlineitem{Parameters}\begin{itemize}
\item {} 
\sphinxAtStartPar
\sphinxstyleliteralstrong{\sphinxupquote{delta\_isocenter}} ({\hyperref[\detokenize{deepdrr.geo:deepdrr.geo.core.Vector3D}]{\sphinxcrossref{\sphinxstyleliteralemphasis{\sphinxupquote{Vector3D}}}}}) \textendash{} offset for the isocenter of the C\sphinxhyphen{}arm in world\sphinxhyphen{}space. This is the center about which rotations are performed.

\item {} 
\sphinxAtStartPar
\sphinxstyleliteralstrong{\sphinxupquote{phi}} (\sphinxstyleliteralemphasis{\sphinxupquote{float}}) \textendash{} CRAN/CAUD angle of the C\sphinxhyphen{}Arm (along the actual arc of the arm)

\item {} 
\sphinxAtStartPar
\sphinxstyleliteralstrong{\sphinxupquote{theta}} (\sphinxstyleliteralemphasis{\sphinxupquote{float}}) \textendash{} Lect/Right angulation of C\sphinxhyphen{}arm (rotation at the base)

\item {} 
\sphinxAtStartPar
\sphinxstyleliteralstrong{\sphinxupquote{rho}} (\sphinxstyleliteralemphasis{\sphinxupquote{float}}\sphinxstyleliteralemphasis{\sphinxupquote{, }}\sphinxstyleliteralemphasis{\sphinxupquote{optional}}) \textendash{} rotation about principle axis, after main rotation. Defaults to 0.

\item {} 
\sphinxAtStartPar
\sphinxstyleliteralstrong{\sphinxupquote{degrees}} (\sphinxstyleliteralemphasis{\sphinxupquote{bool}}\sphinxstyleliteralemphasis{\sphinxupquote{, }}\sphinxstyleliteralemphasis{\sphinxupquote{optional}}) \textendash{} Whether given angles are in degrees. Defaults to False.

\end{itemize}

\end{description}\end{quote}

\end{fulllineitems}

\index{move\_to() (deepdrr.CArm method)@\spxentry{move\_to()}\spxextra{deepdrr.CArm method}}

\begin{fulllineitems}
\phantomsection\label{\detokenize{deepdrr:deepdrr.CArm.move_to}}
\pysigstartsignatures
\pysiglinewithargsret{\sphinxbfcode{\sphinxupquote{move\_to}}}{\sphinxparam{\DUrole{n,n}{isocenter}\DUrole{p,p}{:}\DUrole{w,w}{  }\DUrole{n,n}{{\hyperref[\detokenize{deepdrr.geo:deepdrr.geo.core.Point3D}]{\sphinxcrossref{Point3D}}}\DUrole{w,w}{  }\DUrole{p,p}{|}\DUrole{w,w}{  }None}\DUrole{w,w}{  }\DUrole{o,o}{=}\DUrole{w,w}{  }\DUrole{default_value}{None}}\sphinxparamcomma \sphinxparam{\DUrole{n,n}{phi}\DUrole{p,p}{:}\DUrole{w,w}{  }\DUrole{n,n}{float\DUrole{w,w}{  }\DUrole{p,p}{|}\DUrole{w,w}{  }None}\DUrole{w,w}{  }\DUrole{o,o}{=}\DUrole{w,w}{  }\DUrole{default_value}{None}}\sphinxparamcomma \sphinxparam{\DUrole{n,n}{theta}\DUrole{p,p}{:}\DUrole{w,w}{  }\DUrole{n,n}{float\DUrole{w,w}{  }\DUrole{p,p}{|}\DUrole{w,w}{  }None}\DUrole{w,w}{  }\DUrole{o,o}{=}\DUrole{w,w}{  }\DUrole{default_value}{None}}\sphinxparamcomma \sphinxparam{\DUrole{n,n}{rho}\DUrole{p,p}{:}\DUrole{w,w}{  }\DUrole{n,n}{float\DUrole{w,w}{  }\DUrole{p,p}{|}\DUrole{w,w}{  }None}\DUrole{w,w}{  }\DUrole{o,o}{=}\DUrole{w,w}{  }\DUrole{default_value}{None}}\sphinxparamcomma \sphinxparam{\DUrole{n,n}{degrees}\DUrole{p,p}{:}\DUrole{w,w}{  }\DUrole{n,n}{bool}\DUrole{w,w}{  }\DUrole{o,o}{=}\DUrole{w,w}{  }\DUrole{default_value}{False}}}{{ $\rightarrow$ None}}
\pysigstopsignatures
\sphinxAtStartPar
Move the C\sphinxhyphen{}arm to the specified pose.
\begin{quote}\begin{description}
\sphinxlineitem{Parameters}\begin{itemize}
\item {} 
\sphinxAtStartPar
\sphinxstyleliteralstrong{\sphinxupquote{isocenter}} ({\hyperref[\detokenize{deepdrr.geo:deepdrr.geo.core.Point3D}]{\sphinxcrossref{\sphinxstyleliteralemphasis{\sphinxupquote{Point3D}}}}}) \textendash{} New isocenter of the C\sphinxhyphen{}arm in device space. This is the center about which rotations are performed.

\item {} 
\sphinxAtStartPar
\sphinxstyleliteralstrong{\sphinxupquote{phi}} (\sphinxstyleliteralemphasis{\sphinxupquote{float}}) \textendash{} CRAN/CAUD angle of the C\sphinxhyphen{}Arm (along the actual arc of the arm)

\item {} 
\sphinxAtStartPar
\sphinxstyleliteralstrong{\sphinxupquote{theta}} (\sphinxstyleliteralemphasis{\sphinxupquote{float}}) \textendash{} Lect/Right angulation of C\sphinxhyphen{}arm (rotation at the base)

\item {} 
\sphinxAtStartPar
\sphinxstyleliteralstrong{\sphinxupquote{rho}} (\sphinxstyleliteralemphasis{\sphinxupquote{float}}\sphinxstyleliteralemphasis{\sphinxupquote{, }}\sphinxstyleliteralemphasis{\sphinxupquote{optional}}) \textendash{} rotation about principle axis, after main rotation. Defaults to 0.

\item {} 
\sphinxAtStartPar
\sphinxstyleliteralstrong{\sphinxupquote{degrees}} (\sphinxstyleliteralemphasis{\sphinxupquote{bool}}\sphinxstyleliteralemphasis{\sphinxupquote{, }}\sphinxstyleliteralemphasis{\sphinxupquote{optional}}) \textendash{} Whether given angles are in degrees. Defaults to False.

\end{itemize}

\end{description}\end{quote}

\end{fulllineitems}


\end{fulllineitems}

\index{MobileCArm (class in deepdrr)@\spxentry{MobileCArm}\spxextra{class in deepdrr}}

\begin{fulllineitems}
\phantomsection\label{\detokenize{deepdrr:deepdrr.MobileCArm}}
\pysigstartsignatures
\pysiglinewithargsret{\sphinxbfcode{\sphinxupquote{class\DUrole{w,w}{  }}}\sphinxcode{\sphinxupquote{deepdrr.}}\sphinxbfcode{\sphinxupquote{MobileCArm}}}{\sphinxparam{\DUrole{n,n}{world\_from\_device}\DUrole{p,p}{:}\DUrole{w,w}{  }\DUrole{n,n}{{\hyperref[\detokenize{deepdrr.geo:deepdrr.geo.core.FrameTransform}]{\sphinxcrossref{FrameTransform}}}\DUrole{w,w}{  }\DUrole{p,p}{|}\DUrole{w,w}{  }None}\DUrole{w,w}{  }\DUrole{o,o}{=}\DUrole{w,w}{  }\DUrole{default_value}{None}}\sphinxparamcomma \sphinxparam{\DUrole{n,n}{isocenter}\DUrole{p,p}{:}\DUrole{w,w}{  }\DUrole{n,n}{{\hyperref[\detokenize{deepdrr.geo:deepdrr.geo.core.Point3D}]{\sphinxcrossref{Point3D}}}}\DUrole{w,w}{  }\DUrole{o,o}{=}\DUrole{w,w}{  }\DUrole{default_value}{{[}0, 0, 0{]}}}\sphinxparamcomma \sphinxparam{\DUrole{n,n}{alpha}\DUrole{p,p}{:}\DUrole{w,w}{  }\DUrole{n,n}{float}\DUrole{w,w}{  }\DUrole{o,o}{=}\DUrole{w,w}{  }\DUrole{default_value}{0}}\sphinxparamcomma \sphinxparam{\DUrole{n,n}{beta}\DUrole{p,p}{:}\DUrole{w,w}{  }\DUrole{n,n}{float}\DUrole{w,w}{  }\DUrole{o,o}{=}\DUrole{w,w}{  }\DUrole{default_value}{0}}\sphinxparamcomma \sphinxparam{\DUrole{n,n}{gamma}\DUrole{p,p}{:}\DUrole{w,w}{  }\DUrole{n,n}{float}\DUrole{w,w}{  }\DUrole{o,o}{=}\DUrole{w,w}{  }\DUrole{default_value}{0}}\sphinxparamcomma \sphinxparam{\DUrole{n,n}{degrees}\DUrole{p,p}{:}\DUrole{w,w}{  }\DUrole{n,n}{bool}\DUrole{w,w}{  }\DUrole{o,o}{=}\DUrole{w,w}{  }\DUrole{default_value}{True}}\sphinxparamcomma \sphinxparam{\DUrole{n,n}{horizontal\_movement}\DUrole{p,p}{:}\DUrole{w,w}{  }\DUrole{n,n}{float}\DUrole{w,w}{  }\DUrole{o,o}{=}\DUrole{w,w}{  }\DUrole{default_value}{200}}\sphinxparamcomma \sphinxparam{\DUrole{n,n}{vertical\_travel}\DUrole{p,p}{:}\DUrole{w,w}{  }\DUrole{n,n}{float}\DUrole{w,w}{  }\DUrole{o,o}{=}\DUrole{w,w}{  }\DUrole{default_value}{430}}\sphinxparamcomma \sphinxparam{\DUrole{n,n}{min\_alpha}\DUrole{p,p}{:}\DUrole{w,w}{  }\DUrole{n,n}{float}\DUrole{w,w}{  }\DUrole{o,o}{=}\DUrole{w,w}{  }\DUrole{default_value}{\sphinxhyphen{}40}}\sphinxparamcomma \sphinxparam{\DUrole{n,n}{max\_alpha}\DUrole{p,p}{:}\DUrole{w,w}{  }\DUrole{n,n}{float}\DUrole{w,w}{  }\DUrole{o,o}{=}\DUrole{w,w}{  }\DUrole{default_value}{110}}\sphinxparamcomma \sphinxparam{\DUrole{n,n}{min\_beta}\DUrole{p,p}{:}\DUrole{w,w}{  }\DUrole{n,n}{float}\DUrole{w,w}{  }\DUrole{o,o}{=}\DUrole{w,w}{  }\DUrole{default_value}{\sphinxhyphen{}225}}\sphinxparamcomma \sphinxparam{\DUrole{n,n}{max\_beta}\DUrole{p,p}{:}\DUrole{w,w}{  }\DUrole{n,n}{float}\DUrole{w,w}{  }\DUrole{o,o}{=}\DUrole{w,w}{  }\DUrole{default_value}{225}}\sphinxparamcomma \sphinxparam{\DUrole{n,n}{source\_to\_detector\_distance}\DUrole{p,p}{:}\DUrole{w,w}{  }\DUrole{n,n}{float}\DUrole{w,w}{  }\DUrole{o,o}{=}\DUrole{w,w}{  }\DUrole{default_value}{1020}}\sphinxparamcomma \sphinxparam{\DUrole{n,n}{source\_to\_isocenter\_vertical\_distance}\DUrole{p,p}{:}\DUrole{w,w}{  }\DUrole{n,n}{float}\DUrole{w,w}{  }\DUrole{o,o}{=}\DUrole{w,w}{  }\DUrole{default_value}{530}}\sphinxparamcomma \sphinxparam{\DUrole{n,n}{source\_to\_isocenter\_horizontal\_offset}\DUrole{p,p}{:}\DUrole{w,w}{  }\DUrole{n,n}{float}\DUrole{w,w}{  }\DUrole{o,o}{=}\DUrole{w,w}{  }\DUrole{default_value}{0}}\sphinxparamcomma \sphinxparam{\DUrole{n,n}{immersion\_depth}\DUrole{p,p}{:}\DUrole{w,w}{  }\DUrole{n,n}{float}\DUrole{w,w}{  }\DUrole{o,o}{=}\DUrole{w,w}{  }\DUrole{default_value}{730}}\sphinxparamcomma \sphinxparam{\DUrole{n,n}{free\_space}\DUrole{p,p}{:}\DUrole{w,w}{  }\DUrole{n,n}{float}\DUrole{w,w}{  }\DUrole{o,o}{=}\DUrole{w,w}{  }\DUrole{default_value}{820}}\sphinxparamcomma \sphinxparam{\DUrole{n,n}{sensor\_height}\DUrole{p,p}{:}\DUrole{w,w}{  }\DUrole{n,n}{int}\DUrole{w,w}{  }\DUrole{o,o}{=}\DUrole{w,w}{  }\DUrole{default_value}{1536}}\sphinxparamcomma \sphinxparam{\DUrole{n,n}{sensor\_width}\DUrole{p,p}{:}\DUrole{w,w}{  }\DUrole{n,n}{int}\DUrole{w,w}{  }\DUrole{o,o}{=}\DUrole{w,w}{  }\DUrole{default_value}{1536}}\sphinxparamcomma \sphinxparam{\DUrole{n,n}{pixel\_size}\DUrole{p,p}{:}\DUrole{w,w}{  }\DUrole{n,n}{float}\DUrole{w,w}{  }\DUrole{o,o}{=}\DUrole{w,w}{  }\DUrole{default_value}{0.194}}\sphinxparamcomma \sphinxparam{\DUrole{n,n}{rotate\_camera\_left}\DUrole{p,p}{:}\DUrole{w,w}{  }\DUrole{n,n}{bool}\DUrole{w,w}{  }\DUrole{o,o}{=}\DUrole{w,w}{  }\DUrole{default_value}{True}}\sphinxparamcomma \sphinxparam{\DUrole{n,n}{enforce\_isocenter\_bounds}\DUrole{p,p}{:}\DUrole{w,w}{  }\DUrole{n,n}{bool}\DUrole{w,w}{  }\DUrole{o,o}{=}\DUrole{w,w}{  }\DUrole{default_value}{False}}}{}
\pysigstopsignatures
\sphinxAtStartPar
Bases: {\hyperref[\detokenize{deepdrr.device:deepdrr.device.device.Device}]{\sphinxcrossref{\sphinxcode{\sphinxupquote{Device}}}}}

\sphinxAtStartPar
A C\sphinxhyphen{}arm imaging device with orbital movement (alpha, beta) and isocenter movement (x, y, z).

\sphinxAtStartPar
Default parameters are based on the Siemens CIOS Spin.
\index{alpha (deepdrr.MobileCArm attribute)@\spxentry{alpha}\spxextra{deepdrr.MobileCArm attribute}}

\begin{fulllineitems}
\phantomsection\label{\detokenize{deepdrr:deepdrr.MobileCArm.alpha}}
\pysigstartsignatures
\pysigline{\sphinxbfcode{\sphinxupquote{alpha}}\sphinxbfcode{\sphinxupquote{\DUrole{p,p}{:}\DUrole{w,w}{  }float}}}
\pysigstopsignatures
\end{fulllineitems}

\index{arm\_from\_device (deepdrr.MobileCArm property)@\spxentry{arm\_from\_device}\spxextra{deepdrr.MobileCArm property}}

\begin{fulllineitems}
\phantomsection\label{\detokenize{deepdrr:deepdrr.MobileCArm.arm_from_device}}
\pysigstartsignatures
\pysigline{\sphinxbfcode{\sphinxupquote{property\DUrole{w,w}{  }}}\sphinxbfcode{\sphinxupquote{arm\_from\_device}}\sphinxbfcode{\sphinxupquote{\DUrole{p,p}{:}\DUrole{w,w}{  }{\hyperref[\detokenize{deepdrr.geo:deepdrr.geo.core.FrameTransform}]{\sphinxcrossref{FrameTransform}}}}}}
\pysigstopsignatures
\sphinxAtStartPar
Transformation from the device frame (which doesn’t move) to the arm frame (which rotates and translates with the arm, origin at the isocenter).

\end{fulllineitems}

\index{arm\_width (deepdrr.MobileCArm attribute)@\spxentry{arm\_width}\spxextra{deepdrr.MobileCArm attribute}}

\begin{fulllineitems}
\phantomsection\label{\detokenize{deepdrr:deepdrr.MobileCArm.arm_width}}
\pysigstartsignatures
\pysigline{\sphinxbfcode{\sphinxupquote{arm\_width}}\sphinxbfcode{\sphinxupquote{\DUrole{w,w}{  }\DUrole{p,p}{=}\DUrole{w,w}{  }100}}}
\pysigstopsignatures
\end{fulllineitems}

\index{beta (deepdrr.MobileCArm attribute)@\spxentry{beta}\spxextra{deepdrr.MobileCArm attribute}}

\begin{fulllineitems}
\phantomsection\label{\detokenize{deepdrr:deepdrr.MobileCArm.beta}}
\pysigstartsignatures
\pysigline{\sphinxbfcode{\sphinxupquote{beta}}\sphinxbfcode{\sphinxupquote{\DUrole{p,p}{:}\DUrole{w,w}{  }float}}}
\pysigstopsignatures
\end{fulllineitems}

\index{camera3d\_from\_device (deepdrr.MobileCArm property)@\spxentry{camera3d\_from\_device}\spxextra{deepdrr.MobileCArm property}}

\begin{fulllineitems}
\phantomsection\label{\detokenize{deepdrr:deepdrr.MobileCArm.camera3d_from_device}}
\pysigstartsignatures
\pysigline{\sphinxbfcode{\sphinxupquote{property\DUrole{w,w}{  }}}\sphinxbfcode{\sphinxupquote{camera3d\_from\_device}}\sphinxbfcode{\sphinxupquote{\DUrole{p,p}{:}\DUrole{w,w}{  }{\hyperref[\detokenize{deepdrr.geo:deepdrr.geo.core.FrameTransform}]{\sphinxcrossref{FrameTransform}}}}}}
\pysigstopsignatures
\sphinxAtStartPar
Get the camera3d frame from device coordinates

\sphinxAtStartPar
The Z axis points from the source to the detector.

\end{fulllineitems}

\index{camera3d\_from\_world (deepdrr.MobileCArm property)@\spxentry{camera3d\_from\_world}\spxextra{deepdrr.MobileCArm property}}

\begin{fulllineitems}
\phantomsection\label{\detokenize{deepdrr:deepdrr.MobileCArm.camera3d_from_world}}
\pysigstartsignatures
\pysigline{\sphinxbfcode{\sphinxupquote{property\DUrole{w,w}{  }}}\sphinxbfcode{\sphinxupquote{camera3d\_from\_world}}\sphinxbfcode{\sphinxupquote{\DUrole{p,p}{:}\DUrole{w,w}{  }{\hyperref[\detokenize{deepdrr.geo:deepdrr.geo.core.FrameTransform}]{\sphinxcrossref{FrameTransform}}}}}}
\pysigstopsignatures
\sphinxAtStartPar
Rigid transformation of the C\sphinxhyphen{}arm camera pose.

\end{fulllineitems}

\index{detector\_height (deepdrr.MobileCArm attribute)@\spxentry{detector\_height}\spxextra{deepdrr.MobileCArm attribute}}

\begin{fulllineitems}
\phantomsection\label{\detokenize{deepdrr:deepdrr.MobileCArm.detector_height}}
\pysigstartsignatures
\pysigline{\sphinxbfcode{\sphinxupquote{detector\_height}}\sphinxbfcode{\sphinxupquote{\DUrole{w,w}{  }\DUrole{p,p}{=}\DUrole{w,w}{  }100}}}
\pysigstopsignatures
\end{fulllineitems}

\index{device\_from\_arm (deepdrr.MobileCArm property)@\spxentry{device\_from\_arm}\spxextra{deepdrr.MobileCArm property}}

\begin{fulllineitems}
\phantomsection\label{\detokenize{deepdrr:deepdrr.MobileCArm.device_from_arm}}
\pysigstartsignatures
\pysigline{\sphinxbfcode{\sphinxupquote{property\DUrole{w,w}{  }}}\sphinxbfcode{\sphinxupquote{device\_from\_arm}}\sphinxbfcode{\sphinxupquote{\DUrole{p,p}{:}\DUrole{w,w}{  }{\hyperref[\detokenize{deepdrr.geo:deepdrr.geo.core.FrameTransform}]{\sphinxcrossref{FrameTransform}}}}}}
\pysigstopsignatures
\end{fulllineitems}

\index{device\_from\_camera3d (deepdrr.MobileCArm property)@\spxentry{device\_from\_camera3d}\spxextra{deepdrr.MobileCArm property}}

\begin{fulllineitems}
\phantomsection\label{\detokenize{deepdrr:deepdrr.MobileCArm.device_from_camera3d}}
\pysigstartsignatures
\pysigline{\sphinxbfcode{\sphinxupquote{property\DUrole{w,w}{  }}}\sphinxbfcode{\sphinxupquote{device\_from\_camera3d}}\sphinxbfcode{\sphinxupquote{\DUrole{p,p}{:}\DUrole{w,w}{  }{\hyperref[\detokenize{deepdrr.geo:deepdrr.geo.core.FrameTransform}]{\sphinxcrossref{FrameTransform}}}}}}
\pysigstopsignatures
\sphinxAtStartPar
Get the FrameTransform for the device’s camera3d\_from\_device frame (in the current pose).
\begin{quote}\begin{description}
\sphinxlineitem{Parameters}
\sphinxAtStartPar
\sphinxstyleliteralstrong{\sphinxupquote{camera3d\_transform}} ({\hyperref[\detokenize{deepdrr.geo:deepdrr.geo.core.FrameTransform}]{\sphinxcrossref{\sphinxstyleliteralemphasis{\sphinxupquote{FrameTransform}}}}}) \textendash{} the “camera3d\_from\_device” frame transformation for the device.

\sphinxlineitem{Returns}
\sphinxAtStartPar
the “device\_from\_camera3d” frame transformation for the device.

\sphinxlineitem{Return type}
\sphinxAtStartPar
{\hyperref[\detokenize{deepdrr.geo:deepdrr.geo.core.FrameTransform}]{\sphinxcrossref{FrameTransform}}}

\end{description}\end{quote}

\end{fulllineitems}

\index{get\_camera3d\_from\_world() (deepdrr.MobileCArm method)@\spxentry{get\_camera3d\_from\_world()}\spxextra{deepdrr.MobileCArm method}}

\begin{fulllineitems}
\phantomsection\label{\detokenize{deepdrr:deepdrr.MobileCArm.get_camera3d_from_world}}
\pysigstartsignatures
\pysiglinewithargsret{\sphinxbfcode{\sphinxupquote{get\_camera3d\_from\_world}}}{}{{ $\rightarrow$ {\hyperref[\detokenize{deepdrr.geo:deepdrr.geo.core.FrameTransform}]{\sphinxcrossref{FrameTransform}}}}}
\pysigstopsignatures
\end{fulllineitems}

\index{get\_camera\_projection() (deepdrr.MobileCArm method)@\spxentry{get\_camera\_projection()}\spxextra{deepdrr.MobileCArm method}}

\begin{fulllineitems}
\phantomsection\label{\detokenize{deepdrr:deepdrr.MobileCArm.get_camera_projection}}
\pysigstartsignatures
\pysiglinewithargsret{\sphinxbfcode{\sphinxupquote{get\_camera\_projection}}}{}{{ $\rightarrow$ {\hyperref[\detokenize{deepdrr.geo:deepdrr.geo.core.CameraProjection}]{\sphinxcrossref{CameraProjection}}}}}
\pysigstopsignatures
\sphinxAtStartPar
Get the camera projection for the device in the current pose.
\begin{quote}\begin{description}
\sphinxlineitem{Returns}
\sphinxAtStartPar
the “index\_from\_world” camera projection for the device.

\sphinxlineitem{Return type}
\sphinxAtStartPar
{\hyperref[\detokenize{deepdrr.geo:deepdrr.geo.core.CameraProjection}]{\sphinxcrossref{CameraProjection}}}

\end{description}\end{quote}

\end{fulllineitems}

\index{get\_mesh\_in\_world() (deepdrr.MobileCArm method)@\spxentry{get\_mesh\_in\_world()}\spxextra{deepdrr.MobileCArm method}}

\begin{fulllineitems}
\phantomsection\label{\detokenize{deepdrr:deepdrr.MobileCArm.get_mesh_in_world}}
\pysigstartsignatures
\pysiglinewithargsret{\sphinxbfcode{\sphinxupquote{get\_mesh\_in\_world}}}{\sphinxparam{\DUrole{n,n}{full}\DUrole{o,o}{=}\DUrole{default_value}{False}}\sphinxparamcomma \sphinxparam{\DUrole{n,n}{use\_cached}\DUrole{o,o}{=}\DUrole{default_value}{True}}}{}
\pysigstopsignatures
\sphinxAtStartPar
Get the pyvista mesh for the C\sphinxhyphen{}arm, in its world\sphinxhyphen{}space orientation.
\begin{quote}\begin{description}
\sphinxlineitem{Raises}
\sphinxAtStartPar
\sphinxstyleliteralstrong{\sphinxupquote{RuntimeError}} \textendash{} if pyvista is not available.

\end{description}\end{quote}

\end{fulllineitems}

\index{isocenter (deepdrr.MobileCArm attribute)@\spxentry{isocenter}\spxextra{deepdrr.MobileCArm attribute}}

\begin{fulllineitems}
\phantomsection\label{\detokenize{deepdrr:deepdrr.MobileCArm.isocenter}}
\pysigstartsignatures
\pysigline{\sphinxbfcode{\sphinxupquote{isocenter}}\sphinxbfcode{\sphinxupquote{\DUrole{p,p}{:}\DUrole{w,w}{  }{\hyperref[\detokenize{deepdrr.geo:deepdrr.geo.core.Point3D}]{\sphinxcrossref{Point3D}}}}}}
\pysigstopsignatures
\end{fulllineitems}

\index{isocenter\_in\_world (deepdrr.MobileCArm property)@\spxentry{isocenter\_in\_world}\spxextra{deepdrr.MobileCArm property}}

\begin{fulllineitems}
\phantomsection\label{\detokenize{deepdrr:deepdrr.MobileCArm.isocenter_in_world}}
\pysigstartsignatures
\pysigline{\sphinxbfcode{\sphinxupquote{property\DUrole{w,w}{  }}}\sphinxbfcode{\sphinxupquote{isocenter\_in\_world}}\sphinxbfcode{\sphinxupquote{\DUrole{p,p}{:}\DUrole{w,w}{  }{\hyperref[\detokenize{deepdrr.geo:deepdrr.geo.core.Point3D}]{\sphinxcrossref{Point3D}}}}}}
\pysigstopsignatures
\end{fulllineitems}

\index{jitter() (deepdrr.MobileCArm method)@\spxentry{jitter()}\spxextra{deepdrr.MobileCArm method}}

\begin{fulllineitems}
\phantomsection\label{\detokenize{deepdrr:deepdrr.MobileCArm.jitter}}
\pysigstartsignatures
\pysiglinewithargsret{\sphinxbfcode{\sphinxupquote{jitter}}}{}{}
\pysigstopsignatures
\end{fulllineitems}

\index{max\_isocenter (deepdrr.MobileCArm property)@\spxentry{max\_isocenter}\spxextra{deepdrr.MobileCArm property}}

\begin{fulllineitems}
\phantomsection\label{\detokenize{deepdrr:deepdrr.MobileCArm.max_isocenter}}
\pysigstartsignatures
\pysigline{\sphinxbfcode{\sphinxupquote{property\DUrole{w,w}{  }}}\sphinxbfcode{\sphinxupquote{max\_isocenter}}\sphinxbfcode{\sphinxupquote{\DUrole{p,p}{:}\DUrole{w,w}{  }ndarray}}}
\pysigstopsignatures
\end{fulllineitems}

\index{min\_isocenter (deepdrr.MobileCArm property)@\spxentry{min\_isocenter}\spxextra{deepdrr.MobileCArm property}}

\begin{fulllineitems}
\phantomsection\label{\detokenize{deepdrr:deepdrr.MobileCArm.min_isocenter}}
\pysigstartsignatures
\pysigline{\sphinxbfcode{\sphinxupquote{property\DUrole{w,w}{  }}}\sphinxbfcode{\sphinxupquote{min\_isocenter}}\sphinxbfcode{\sphinxupquote{\DUrole{p,p}{:}\DUrole{w,w}{  }ndarray}}}
\pysigstopsignatures
\end{fulllineitems}

\index{move\_by() (deepdrr.MobileCArm method)@\spxentry{move\_by()}\spxextra{deepdrr.MobileCArm method}}

\begin{fulllineitems}
\phantomsection\label{\detokenize{deepdrr:deepdrr.MobileCArm.move_by}}
\pysigstartsignatures
\pysiglinewithargsret{\sphinxbfcode{\sphinxupquote{move\_by}}}{\sphinxparam{\DUrole{n,n}{delta\_isocenter}\DUrole{p,p}{:}\DUrole{w,w}{  }\DUrole{n,n}{{\hyperref[\detokenize{deepdrr.geo:deepdrr.geo.core.Vector3D}]{\sphinxcrossref{Vector3D}}}\DUrole{w,w}{  }\DUrole{p,p}{|}\DUrole{w,w}{  }None}\DUrole{w,w}{  }\DUrole{o,o}{=}\DUrole{w,w}{  }\DUrole{default_value}{None}}\sphinxparamcomma \sphinxparam{\DUrole{n,n}{delta\_alpha}\DUrole{p,p}{:}\DUrole{w,w}{  }\DUrole{n,n}{float\DUrole{w,w}{  }\DUrole{p,p}{|}\DUrole{w,w}{  }None}\DUrole{w,w}{  }\DUrole{o,o}{=}\DUrole{w,w}{  }\DUrole{default_value}{None}}\sphinxparamcomma \sphinxparam{\DUrole{n,n}{delta\_beta}\DUrole{p,p}{:}\DUrole{w,w}{  }\DUrole{n,n}{float\DUrole{w,w}{  }\DUrole{p,p}{|}\DUrole{w,w}{  }None}\DUrole{w,w}{  }\DUrole{o,o}{=}\DUrole{w,w}{  }\DUrole{default_value}{None}}\sphinxparamcomma \sphinxparam{\DUrole{n,n}{delta\_gamma}\DUrole{p,p}{:}\DUrole{w,w}{  }\DUrole{n,n}{float\DUrole{w,w}{  }\DUrole{p,p}{|}\DUrole{w,w}{  }None}\DUrole{w,w}{  }\DUrole{o,o}{=}\DUrole{w,w}{  }\DUrole{default_value}{None}}\sphinxparamcomma \sphinxparam{\DUrole{n,n}{degrees}\DUrole{p,p}{:}\DUrole{w,w}{  }\DUrole{n,n}{bool}\DUrole{w,w}{  }\DUrole{o,o}{=}\DUrole{w,w}{  }\DUrole{default_value}{True}}}{{ $\rightarrow$ None}}
\pysigstopsignatures
\sphinxAtStartPar
Move the C\sphinxhyphen{}arm to the specified pose.
\begin{quote}\begin{description}
\sphinxlineitem{Parameters}\begin{itemize}
\item {} 
\sphinxAtStartPar
\sphinxstyleliteralstrong{\sphinxupquote{delta\_isocenter}} (\sphinxstyleliteralemphasis{\sphinxupquote{Optional}}\sphinxstyleliteralemphasis{\sphinxupquote{{[}}}{\hyperref[\detokenize{deepdrr.geo:deepdrr.geo.Vector3D}]{\sphinxcrossref{\sphinxstyleliteralemphasis{\sphinxupquote{geo.Vector3D}}}}}\sphinxstyleliteralemphasis{\sphinxupquote{{]}}}\sphinxstyleliteralemphasis{\sphinxupquote{, }}\sphinxstyleliteralemphasis{\sphinxupquote{optional}}) \textendash{} change to the isocenter in DEVICE space
(as a vector, this only matters if the scaling/rotation is different).
This is the center about which rotations are performed. Defaults to None.

\item {} 
\sphinxAtStartPar
\sphinxstyleliteralstrong{\sphinxupquote{delta\_alpha}} (\sphinxstyleliteralemphasis{\sphinxupquote{Optional}}\sphinxstyleliteralemphasis{\sphinxupquote{{[}}}\sphinxstyleliteralemphasis{\sphinxupquote{float}}\sphinxstyleliteralemphasis{\sphinxupquote{{]}}}\sphinxstyleliteralemphasis{\sphinxupquote{, }}\sphinxstyleliteralemphasis{\sphinxupquote{optional}}) \textendash{} change in alpha. Defaults to None.

\item {} 
\sphinxAtStartPar
\sphinxstyleliteralstrong{\sphinxupquote{delta\_beta}} (\sphinxstyleliteralemphasis{\sphinxupquote{Optional}}\sphinxstyleliteralemphasis{\sphinxupquote{{[}}}\sphinxstyleliteralemphasis{\sphinxupquote{float}}\sphinxstyleliteralemphasis{\sphinxupquote{{]}}}\sphinxstyleliteralemphasis{\sphinxupquote{, }}\sphinxstyleliteralemphasis{\sphinxupquote{optional}}) \textendash{} change in beta. Defaults to None.

\item {} 
\sphinxAtStartPar
\sphinxstyleliteralstrong{\sphinxupquote{degrees}} (\sphinxstyleliteralemphasis{\sphinxupquote{bool}}\sphinxstyleliteralemphasis{\sphinxupquote{, }}\sphinxstyleliteralemphasis{\sphinxupquote{optional}}) \textendash{} whether the given angles are in degrees. Defaults to False.

\end{itemize}

\end{description}\end{quote}

\end{fulllineitems}

\index{move\_to() (deepdrr.MobileCArm method)@\spxentry{move\_to()}\spxextra{deepdrr.MobileCArm method}}

\begin{fulllineitems}
\phantomsection\label{\detokenize{deepdrr:deepdrr.MobileCArm.move_to}}
\pysigstartsignatures
\pysiglinewithargsret{\sphinxbfcode{\sphinxupquote{move\_to}}}{\sphinxparam{\DUrole{n,n}{isocenter}\DUrole{p,p}{:}\DUrole{w,w}{  }\DUrole{n,n}{{\hyperref[\detokenize{deepdrr.geo:deepdrr.geo.core.Point3D}]{\sphinxcrossref{Point3D}}}\DUrole{w,w}{  }\DUrole{p,p}{|}\DUrole{w,w}{  }None}\DUrole{w,w}{  }\DUrole{o,o}{=}\DUrole{w,w}{  }\DUrole{default_value}{None}}\sphinxparamcomma \sphinxparam{\DUrole{n,n}{isocenter\_in\_world}\DUrole{p,p}{:}\DUrole{w,w}{  }\DUrole{n,n}{{\hyperref[\detokenize{deepdrr.geo:deepdrr.geo.core.Point3D}]{\sphinxcrossref{Point3D}}}\DUrole{w,w}{  }\DUrole{p,p}{|}\DUrole{w,w}{  }None}\DUrole{w,w}{  }\DUrole{o,o}{=}\DUrole{w,w}{  }\DUrole{default_value}{None}}\sphinxparamcomma \sphinxparam{\DUrole{n,n}{alpha}\DUrole{p,p}{:}\DUrole{w,w}{  }\DUrole{n,n}{float\DUrole{w,w}{  }\DUrole{p,p}{|}\DUrole{w,w}{  }None}\DUrole{w,w}{  }\DUrole{o,o}{=}\DUrole{w,w}{  }\DUrole{default_value}{None}}\sphinxparamcomma \sphinxparam{\DUrole{n,n}{beta}\DUrole{p,p}{:}\DUrole{w,w}{  }\DUrole{n,n}{float\DUrole{w,w}{  }\DUrole{p,p}{|}\DUrole{w,w}{  }None}\DUrole{w,w}{  }\DUrole{o,o}{=}\DUrole{w,w}{  }\DUrole{default_value}{None}}\sphinxparamcomma \sphinxparam{\DUrole{n,n}{gamma}\DUrole{p,p}{:}\DUrole{w,w}{  }\DUrole{n,n}{float\DUrole{w,w}{  }\DUrole{p,p}{|}\DUrole{w,w}{  }None}\DUrole{w,w}{  }\DUrole{o,o}{=}\DUrole{w,w}{  }\DUrole{default_value}{None}}\sphinxparamcomma \sphinxparam{\DUrole{n,n}{degrees}\DUrole{p,p}{:}\DUrole{w,w}{  }\DUrole{n,n}{bool}\DUrole{w,w}{  }\DUrole{o,o}{=}\DUrole{w,w}{  }\DUrole{default_value}{True}}\sphinxparamcomma \sphinxparam{\DUrole{n,n}{interest\_point\_in\_world}\DUrole{p,p}{:}\DUrole{w,w}{  }\DUrole{n,n}{{\hyperref[\detokenize{deepdrr.geo:deepdrr.geo.core.Point3D}]{\sphinxcrossref{Point3D}}}\DUrole{w,w}{  }\DUrole{p,p}{|}\DUrole{w,w}{  }None}\DUrole{w,w}{  }\DUrole{o,o}{=}\DUrole{w,w}{  }\DUrole{default_value}{None}}\sphinxparamcomma \sphinxparam{\DUrole{n,n}{principle\_ray\_in\_world}\DUrole{p,p}{:}\DUrole{w,w}{  }\DUrole{n,n}{{\hyperref[\detokenize{deepdrr.geo:deepdrr.geo.core.Vector3D}]{\sphinxcrossref{Vector3D}}}\DUrole{w,w}{  }\DUrole{p,p}{|}\DUrole{w,w}{  }None}\DUrole{w,w}{  }\DUrole{o,o}{=}\DUrole{w,w}{  }\DUrole{default_value}{None}}}{{ $\rightarrow$ None}}
\pysigstopsignatures
\sphinxAtStartPar
Move to the specified point.
\begin{quote}\begin{description}
\sphinxlineitem{Parameters}\begin{itemize}
\item {} 
\sphinxAtStartPar
\sphinxstyleliteralstrong{\sphinxupquote{isocenter\_in\_world}} (\sphinxstyleliteralemphasis{\sphinxupquote{Optional}}\sphinxstyleliteralemphasis{\sphinxupquote{{[}}}{\hyperref[\detokenize{deepdrr.geo:deepdrr.geo.Point3D}]{\sphinxcrossref{\sphinxstyleliteralemphasis{\sphinxupquote{geo.Point3D}}}}}\sphinxstyleliteralemphasis{\sphinxupquote{{]}}}\sphinxstyleliteralemphasis{\sphinxupquote{, }}\sphinxstyleliteralemphasis{\sphinxupquote{optional}}) \textendash{} the desired isocenter in world coordinates.
Overrides \sphinxtitleref{isocenter} if provided. Defaults to None.

\item {} 
\sphinxAtStartPar
\sphinxstyleliteralstrong{\sphinxupquote{isocenter}} \textendash{} Desired isocenter in device coordinates.

\item {} 
\sphinxAtStartPar
\sphinxstyleliteralstrong{\sphinxupquote{alpha}} (\sphinxstyleliteralemphasis{\sphinxupquote{Optional}}\sphinxstyleliteralemphasis{\sphinxupquote{{[}}}\sphinxstyleliteralemphasis{\sphinxupquote{float}}\sphinxstyleliteralemphasis{\sphinxupquote{{]}}}\sphinxstyleliteralemphasis{\sphinxupquote{, }}\sphinxstyleliteralemphasis{\sphinxupquote{optional}}) \textendash{} the desired alpha angulation. Defaults to None.

\item {} 
\sphinxAtStartPar
\sphinxstyleliteralstrong{\sphinxupquote{beta}} (\sphinxstyleliteralemphasis{\sphinxupquote{Optional}}\sphinxstyleliteralemphasis{\sphinxupquote{{[}}}\sphinxstyleliteralemphasis{\sphinxupquote{float}}\sphinxstyleliteralemphasis{\sphinxupquote{{]}}}\sphinxstyleliteralemphasis{\sphinxupquote{, }}\sphinxstyleliteralemphasis{\sphinxupquote{optional}}) \textendash{} the desired secondary angulation. Defaults to None.

\item {} 
\sphinxAtStartPar
\sphinxstyleliteralstrong{\sphinxupquote{degrees}} (\sphinxstyleliteralemphasis{\sphinxupquote{bool}}\sphinxstyleliteralemphasis{\sphinxupquote{, }}\sphinxstyleliteralemphasis{\sphinxupquote{optional}}) \textendash{} whether angles are in degrees or radians. Defaults to False.

\item {} 
\sphinxAtStartPar
\sphinxstyleliteralstrong{\sphinxupquote{interest\_point}} ({\hyperref[\detokenize{deepdrr.geo:deepdrr.geo.core.Point3D}]{\sphinxcrossref{\sphinxstyleliteralemphasis{\sphinxupquote{Point3D}}}}}\sphinxstyleliteralemphasis{\sphinxupquote{, }}\sphinxstyleliteralemphasis{\sphinxupquote{optional}}) \textendash{} If this world\sphinxhyphen{}space point is provided, add a translation such that the rotation
maintains the camera\sphinxhyphen{}space position of this point. Overrides \sphinxtitleref{isocenter}. Defaults to None.

\item {} 
\sphinxAtStartPar
\sphinxstyleliteralstrong{\sphinxupquote{principle\_ray\_in\_world}} (\sphinxstyleliteralemphasis{\sphinxupquote{Optional}}\sphinxstyleliteralemphasis{\sphinxupquote{{[}}}{\hyperref[\detokenize{deepdrr.geo:deepdrr.geo.core.Vector3D}]{\sphinxcrossref{\sphinxstyleliteralemphasis{\sphinxupquote{Vector3D}}}}}\sphinxstyleliteralemphasis{\sphinxupquote{{]}}}\sphinxstyleliteralemphasis{\sphinxupquote{, }}\sphinxstyleliteralemphasis{\sphinxupquote{optional}}) \textendash{} If this world\sphinxhyphen{}space vector is provided, override alpha, beta so the C\sphinxhyphen{}arm points along this vector.

\end{itemize}

\end{description}\end{quote}

\end{fulllineitems}

\index{principle\_ray (deepdrr.MobileCArm property)@\spxentry{principle\_ray}\spxextra{deepdrr.MobileCArm property}}

\begin{fulllineitems}
\phantomsection\label{\detokenize{deepdrr:deepdrr.MobileCArm.principle_ray}}
\pysigstartsignatures
\pysigline{\sphinxbfcode{\sphinxupquote{property\DUrole{w,w}{  }}}\sphinxbfcode{\sphinxupquote{principle\_ray}}\sphinxbfcode{\sphinxupquote{\DUrole{p,p}{:}\DUrole{w,w}{  }{\hyperref[\detokenize{deepdrr.geo:deepdrr.geo.core.Vector3D}]{\sphinxcrossref{Vector3D}}}}}}
\pysigstopsignatures
\sphinxAtStartPar
Unit vector along principle ray.

\end{fulllineitems}

\index{reposition() (deepdrr.MobileCArm method)@\spxentry{reposition()}\spxextra{deepdrr.MobileCArm method}}

\begin{fulllineitems}
\phantomsection\label{\detokenize{deepdrr:deepdrr.MobileCArm.reposition}}
\pysigstartsignatures
\pysiglinewithargsret{\sphinxbfcode{\sphinxupquote{reposition}}}{\sphinxparam{\DUrole{n,n}{viewpoint\_in\_world}\DUrole{p,p}{:}\DUrole{w,w}{  }\DUrole{n,n}{{\hyperref[\detokenize{deepdrr.geo:deepdrr.geo.core.Point3D}]{\sphinxcrossref{Point3D}}}\DUrole{w,w}{  }\DUrole{p,p}{|}\DUrole{w,w}{  }None}\DUrole{w,w}{  }\DUrole{o,o}{=}\DUrole{w,w}{  }\DUrole{default_value}{None}}\sphinxparamcomma \sphinxparam{\DUrole{n,n}{device\_in\_world}\DUrole{p,p}{:}\DUrole{w,w}{  }\DUrole{n,n}{{\hyperref[\detokenize{deepdrr.geo:deepdrr.geo.core.Point3D}]{\sphinxcrossref{Point3D}}}\DUrole{w,w}{  }\DUrole{p,p}{|}\DUrole{w,w}{  }None}\DUrole{w,w}{  }\DUrole{o,o}{=}\DUrole{w,w}{  }\DUrole{default_value}{None}}}{{ $\rightarrow$ None}}
\pysigstopsignatures
\sphinxAtStartPar
Reposition the C\sphinxhyphen{}arm by resetting its internal pose to the parameters and adjusting the world\_from\_device transform.

\sphinxAtStartPar
TODO: currently, this eliminates any scaling/rotation of the device in world.

\sphinxAtStartPar
May provide either the isocenter location (device\_in\_world) or viewpoint (viewpoint\_in\_world)
\begin{quote}\begin{description}
\sphinxlineitem{Parameters}\begin{itemize}
\item {} 
\sphinxAtStartPar
\sphinxstyleliteralstrong{\sphinxupquote{viewpoint\_in\_world}} ({\hyperref[\detokenize{deepdrr.geo:deepdrr.geo.Point3D}]{\sphinxcrossref{\sphinxstyleliteralemphasis{\sphinxupquote{geo.Point3D}}}}}) \textendash{} the initial viewpoint the device should have.

\item {} 
\sphinxAtStartPar
\sphinxstyleliteralstrong{\sphinxupquote{(}}\sphinxstyleliteralstrong{\sphinxupquote{)}} (\sphinxstyleliteralemphasis{\sphinxupquote{device\_in\_world}}) \textendash{} initial isocenter.

\end{itemize}

\end{description}\end{quote}

\end{fulllineitems}

\index{source\_height (deepdrr.MobileCArm attribute)@\spxentry{source\_height}\spxextra{deepdrr.MobileCArm attribute}}

\begin{fulllineitems}
\phantomsection\label{\detokenize{deepdrr:deepdrr.MobileCArm.source_height}}
\pysigstartsignatures
\pysigline{\sphinxbfcode{\sphinxupquote{source\_height}}\sphinxbfcode{\sphinxupquote{\DUrole{w,w}{  }\DUrole{p,p}{=}\DUrole{w,w}{  }200}}}
\pysigstopsignatures
\end{fulllineitems}

\index{source\_in\_arm (deepdrr.MobileCArm property)@\spxentry{source\_in\_arm}\spxextra{deepdrr.MobileCArm property}}

\begin{fulllineitems}
\phantomsection\label{\detokenize{deepdrr:deepdrr.MobileCArm.source_in_arm}}
\pysigstartsignatures
\pysigline{\sphinxbfcode{\sphinxupquote{property\DUrole{w,w}{  }}}\sphinxbfcode{\sphinxupquote{source\_in\_arm}}\sphinxbfcode{\sphinxupquote{\DUrole{p,p}{:}\DUrole{w,w}{  }{\hyperref[\detokenize{deepdrr.geo:deepdrr.geo.core.Point3D}]{\sphinxcrossref{Point3D}}}}}}
\pysigstopsignatures
\end{fulllineitems}

\index{source\_in\_device (deepdrr.MobileCArm property)@\spxentry{source\_in\_device}\spxextra{deepdrr.MobileCArm property}}

\begin{fulllineitems}
\phantomsection\label{\detokenize{deepdrr:deepdrr.MobileCArm.source_in_device}}
\pysigstartsignatures
\pysigline{\sphinxbfcode{\sphinxupquote{property\DUrole{w,w}{  }}}\sphinxbfcode{\sphinxupquote{source\_in\_device}}\sphinxbfcode{\sphinxupquote{\DUrole{p,p}{:}\DUrole{w,w}{  }{\hyperref[\detokenize{deepdrr.geo:deepdrr.geo.core.Point3D}]{\sphinxcrossref{Point3D}}}}}}
\pysigstopsignatures
\end{fulllineitems}

\index{source\_radius (deepdrr.MobileCArm attribute)@\spxentry{source\_radius}\spxextra{deepdrr.MobileCArm attribute}}

\begin{fulllineitems}
\phantomsection\label{\detokenize{deepdrr:deepdrr.MobileCArm.source_radius}}
\pysigstartsignatures
\pysigline{\sphinxbfcode{\sphinxupquote{source\_radius}}\sphinxbfcode{\sphinxupquote{\DUrole{w,w}{  }\DUrole{p,p}{=}\DUrole{w,w}{  }100}}}
\pysigstopsignatures
\end{fulllineitems}

\index{to\_config() (deepdrr.MobileCArm method)@\spxentry{to\_config()}\spxextra{deepdrr.MobileCArm method}}

\begin{fulllineitems}
\phantomsection\label{\detokenize{deepdrr:deepdrr.MobileCArm.to_config}}
\pysigstartsignatures
\pysiglinewithargsret{\sphinxbfcode{\sphinxupquote{to\_config}}}{}{{ $\rightarrow$ Dict\DUrole{p,p}{{[}}str\DUrole{p,p}{,}\DUrole{w,w}{  }Any\DUrole{p,p}{{]}}}}
\pysigstopsignatures
\sphinxAtStartPar
Get a json\sphinxhyphen{}safe dictionary that can be used to initialize the C\sphinxhyphen{}arm in its current pose.

\end{fulllineitems}

\index{viewpoint (deepdrr.MobileCArm property)@\spxentry{viewpoint}\spxextra{deepdrr.MobileCArm property}}

\begin{fulllineitems}
\phantomsection\label{\detokenize{deepdrr:deepdrr.MobileCArm.viewpoint}}
\pysigstartsignatures
\pysigline{\sphinxbfcode{\sphinxupquote{property\DUrole{w,w}{  }}}\sphinxbfcode{\sphinxupquote{viewpoint}}\sphinxbfcode{\sphinxupquote{\DUrole{p,p}{:}\DUrole{w,w}{  }{\hyperref[\detokenize{deepdrr.geo:deepdrr.geo.core.Point3D}]{\sphinxcrossref{Point3D}}}}}}
\pysigstopsignatures
\sphinxAtStartPar
Get the point along the principle ray, where objects of interest should ideally be placed.
\begin{quote}\begin{description}
\sphinxlineitem{Returns}
\sphinxAtStartPar
the viewpoint in the device frame.

\sphinxlineitem{Return type}
\sphinxAtStartPar
{\hyperref[\detokenize{deepdrr.geo:deepdrr.geo.Point3D}]{\sphinxcrossref{geo.Point3D}}}

\end{description}\end{quote}

\end{fulllineitems}

\index{viewpoint\_in\_world (deepdrr.MobileCArm property)@\spxentry{viewpoint\_in\_world}\spxextra{deepdrr.MobileCArm property}}

\begin{fulllineitems}
\phantomsection\label{\detokenize{deepdrr:deepdrr.MobileCArm.viewpoint_in_world}}
\pysigstartsignatures
\pysigline{\sphinxbfcode{\sphinxupquote{property\DUrole{w,w}{  }}}\sphinxbfcode{\sphinxupquote{viewpoint\_in\_world}}\sphinxbfcode{\sphinxupquote{\DUrole{p,p}{:}\DUrole{w,w}{  }{\hyperref[\detokenize{deepdrr.geo:deepdrr.geo.core.Point3D}]{\sphinxcrossref{Point3D}}}}}}
\pysigstopsignatures
\end{fulllineitems}

\index{world\_from\_device (deepdrr.MobileCArm attribute)@\spxentry{world\_from\_device}\spxextra{deepdrr.MobileCArm attribute}}

\begin{fulllineitems}
\phantomsection\label{\detokenize{deepdrr:deepdrr.MobileCArm.world_from_device}}
\pysigstartsignatures
\pysigline{\sphinxbfcode{\sphinxupquote{world\_from\_device}}\sphinxbfcode{\sphinxupquote{\DUrole{p,p}{:}\DUrole{w,w}{  }{\hyperref[\detokenize{deepdrr.geo:deepdrr.geo.core.FrameTransform}]{\sphinxcrossref{FrameTransform}}}}}}
\pysigstopsignatures
\end{fulllineitems}


\end{fulllineitems}

\index{Projector (class in deepdrr)@\spxentry{Projector}\spxextra{class in deepdrr}}

\begin{fulllineitems}
\phantomsection\label{\detokenize{deepdrr:deepdrr.Projector}}
\pysigstartsignatures
\pysiglinewithargsret{\sphinxbfcode{\sphinxupquote{class\DUrole{w,w}{  }}}\sphinxcode{\sphinxupquote{deepdrr.}}\sphinxbfcode{\sphinxupquote{Projector}}}{\sphinxparam{\DUrole{n,n}{volume}\DUrole{p,p}{:}\DUrole{w,w}{  }\DUrole{n,n}{{\hyperref[\detokenize{deepdrr.vol:deepdrr.vol.volume.Volume}]{\sphinxcrossref{Volume}}}\DUrole{w,w}{  }\DUrole{p,p}{|}\DUrole{w,w}{  }List\DUrole{p,p}{{[}}{\hyperref[\detokenize{deepdrr.vol:deepdrr.vol.volume.Volume}]{\sphinxcrossref{Volume}}}\DUrole{p,p}{{]}}}}\sphinxparamcomma \sphinxparam{\DUrole{n,n}{priorities}\DUrole{p,p}{:}\DUrole{w,w}{  }\DUrole{n,n}{List\DUrole{p,p}{{[}}int\DUrole{p,p}{{]}}\DUrole{w,w}{  }\DUrole{p,p}{|}\DUrole{w,w}{  }None}\DUrole{w,w}{  }\DUrole{o,o}{=}\DUrole{w,w}{  }\DUrole{default_value}{None}}\sphinxparamcomma \sphinxparam{\DUrole{n,n}{camera\_intrinsics}\DUrole{p,p}{:}\DUrole{w,w}{  }\DUrole{n,n}{{\hyperref[\detokenize{deepdrr.geo:deepdrr.geo.core.CameraIntrinsicTransform}]{\sphinxcrossref{CameraIntrinsicTransform}}}\DUrole{w,w}{  }\DUrole{p,p}{|}\DUrole{w,w}{  }None}\DUrole{w,w}{  }\DUrole{o,o}{=}\DUrole{w,w}{  }\DUrole{default_value}{None}}\sphinxparamcomma \sphinxparam{\DUrole{n,n}{device}\DUrole{p,p}{:}\DUrole{w,w}{  }\DUrole{n,n}{{\hyperref[\detokenize{deepdrr.device:deepdrr.device.device.Device}]{\sphinxcrossref{Device}}}\DUrole{w,w}{  }\DUrole{p,p}{|}\DUrole{w,w}{  }None}\DUrole{w,w}{  }\DUrole{o,o}{=}\DUrole{w,w}{  }\DUrole{default_value}{None}}\sphinxparamcomma \sphinxparam{\DUrole{n,n}{step}\DUrole{p,p}{:}\DUrole{w,w}{  }\DUrole{n,n}{float}\DUrole{w,w}{  }\DUrole{o,o}{=}\DUrole{w,w}{  }\DUrole{default_value}{0.1}}\sphinxparamcomma \sphinxparam{\DUrole{n,n}{mode}\DUrole{p,p}{:}\DUrole{w,w}{  }\DUrole{n,n}{str}\DUrole{w,w}{  }\DUrole{o,o}{=}\DUrole{w,w}{  }\DUrole{default_value}{\textquotesingle{}linear\textquotesingle{}}}\sphinxparamcomma \sphinxparam{\DUrole{n,n}{spectrum}\DUrole{p,p}{:}\DUrole{w,w}{  }\DUrole{n,n}{ndarray\DUrole{w,w}{  }\DUrole{p,p}{|}\DUrole{w,w}{  }str}\DUrole{w,w}{  }\DUrole{o,o}{=}\DUrole{w,w}{  }\DUrole{default_value}{\textquotesingle{}90KV\_AL40\textquotesingle{}}}\sphinxparamcomma \sphinxparam{\DUrole{n,n}{add\_scatter}\DUrole{p,p}{:}\DUrole{w,w}{  }\DUrole{n,n}{bool\DUrole{w,w}{  }\DUrole{p,p}{|}\DUrole{w,w}{  }None}\DUrole{w,w}{  }\DUrole{o,o}{=}\DUrole{w,w}{  }\DUrole{default_value}{None}}\sphinxparamcomma \sphinxparam{\DUrole{n,n}{scatter\_num}\DUrole{p,p}{:}\DUrole{w,w}{  }\DUrole{n,n}{int}\DUrole{w,w}{  }\DUrole{o,o}{=}\DUrole{w,w}{  }\DUrole{default_value}{0}}\sphinxparamcomma \sphinxparam{\DUrole{n,n}{add\_noise}\DUrole{p,p}{:}\DUrole{w,w}{  }\DUrole{n,n}{bool}\DUrole{w,w}{  }\DUrole{o,o}{=}\DUrole{w,w}{  }\DUrole{default_value}{False}}\sphinxparamcomma \sphinxparam{\DUrole{n,n}{photon\_count}\DUrole{p,p}{:}\DUrole{w,w}{  }\DUrole{n,n}{int}\DUrole{w,w}{  }\DUrole{o,o}{=}\DUrole{w,w}{  }\DUrole{default_value}{10000}}\sphinxparamcomma \sphinxparam{\DUrole{n,n}{threads}\DUrole{p,p}{:}\DUrole{w,w}{  }\DUrole{n,n}{int}\DUrole{w,w}{  }\DUrole{o,o}{=}\DUrole{w,w}{  }\DUrole{default_value}{8}}\sphinxparamcomma \sphinxparam{\DUrole{n,n}{max\_block\_index}\DUrole{p,p}{:}\DUrole{w,w}{  }\DUrole{n,n}{int}\DUrole{w,w}{  }\DUrole{o,o}{=}\DUrole{w,w}{  }\DUrole{default_value}{1024}}\sphinxparamcomma \sphinxparam{\DUrole{n,n}{collected\_energy}\DUrole{p,p}{:}\DUrole{w,w}{  }\DUrole{n,n}{bool}\DUrole{w,w}{  }\DUrole{o,o}{=}\DUrole{w,w}{  }\DUrole{default_value}{False}}\sphinxparamcomma \sphinxparam{\DUrole{n,n}{neglog}\DUrole{p,p}{:}\DUrole{w,w}{  }\DUrole{n,n}{bool}\DUrole{w,w}{  }\DUrole{o,o}{=}\DUrole{w,w}{  }\DUrole{default_value}{True}}\sphinxparamcomma \sphinxparam{\DUrole{n,n}{intensity\_upper\_bound}\DUrole{p,p}{:}\DUrole{w,w}{  }\DUrole{n,n}{float\DUrole{w,w}{  }\DUrole{p,p}{|}\DUrole{w,w}{  }None}\DUrole{w,w}{  }\DUrole{o,o}{=}\DUrole{w,w}{  }\DUrole{default_value}{None}}\sphinxparamcomma \sphinxparam{\DUrole{n,n}{attenuate\_outside\_volume}\DUrole{p,p}{:}\DUrole{w,w}{  }\DUrole{n,n}{bool}\DUrole{w,w}{  }\DUrole{o,o}{=}\DUrole{w,w}{  }\DUrole{default_value}{False}}\sphinxparamcomma \sphinxparam{\DUrole{n,n}{source\_to\_detector\_distance}\DUrole{p,p}{:}\DUrole{w,w}{  }\DUrole{n,n}{float}\DUrole{w,w}{  }\DUrole{o,o}{=}\DUrole{w,w}{  }\DUrole{default_value}{\sphinxhyphen{}1}}\sphinxparamcomma \sphinxparam{\DUrole{n,n}{carm}\DUrole{p,p}{:}\DUrole{w,w}{  }\DUrole{n,n}{{\hyperref[\detokenize{deepdrr.device:deepdrr.device.device.Device}]{\sphinxcrossref{Device}}}\DUrole{w,w}{  }\DUrole{p,p}{|}\DUrole{w,w}{  }None}\DUrole{w,w}{  }\DUrole{o,o}{=}\DUrole{w,w}{  }\DUrole{default_value}{None}}}{}
\pysigstopsignatures
\sphinxAtStartPar
Bases: \sphinxcode{\sphinxupquote{object}}
\index{camera\_intrinsics (deepdrr.Projector property)@\spxentry{camera\_intrinsics}\spxextra{deepdrr.Projector property}}

\begin{fulllineitems}
\phantomsection\label{\detokenize{deepdrr:deepdrr.Projector.camera_intrinsics}}
\pysigstartsignatures
\pysigline{\sphinxbfcode{\sphinxupquote{property\DUrole{w,w}{  }}}\sphinxbfcode{\sphinxupquote{camera\_intrinsics}}\sphinxbfcode{\sphinxupquote{\DUrole{p,p}{:}\DUrole{w,w}{  }{\hyperref[\detokenize{deepdrr.geo:deepdrr.geo.core.CameraIntrinsicTransform}]{\sphinxcrossref{CameraIntrinsicTransform}}}}}}
\pysigstopsignatures
\end{fulllineitems}

\index{free() (deepdrr.Projector method)@\spxentry{free()}\spxextra{deepdrr.Projector method}}

\begin{fulllineitems}
\phantomsection\label{\detokenize{deepdrr:deepdrr.Projector.free}}
\pysigstartsignatures
\pysiglinewithargsret{\sphinxbfcode{\sphinxupquote{free}}}{}{}
\pysigstopsignatures
\sphinxAtStartPar
Free the allocated GPU memory.

\end{fulllineitems}

\index{initialize() (deepdrr.Projector method)@\spxentry{initialize()}\spxextra{deepdrr.Projector method}}

\begin{fulllineitems}
\phantomsection\label{\detokenize{deepdrr:deepdrr.Projector.initialize}}
\pysigstartsignatures
\pysiglinewithargsret{\sphinxbfcode{\sphinxupquote{initialize}}}{}{}
\pysigstopsignatures
\sphinxAtStartPar
Allocate GPU memory and transfer the volume, segmentations to GPU.

\end{fulllineitems}

\index{initialize\_output\_arrays() (deepdrr.Projector method)@\spxentry{initialize\_output\_arrays()}\spxextra{deepdrr.Projector method}}

\begin{fulllineitems}
\phantomsection\label{\detokenize{deepdrr:deepdrr.Projector.initialize_output_arrays}}
\pysigstartsignatures
\pysiglinewithargsret{\sphinxbfcode{\sphinxupquote{initialize\_output\_arrays}}}{\sphinxparam{\DUrole{n,n}{sensor\_size}\DUrole{p,p}{:}\DUrole{w,w}{  }\DUrole{n,n}{Tuple\DUrole{p,p}{{[}}int\DUrole{p,p}{,}\DUrole{w,w}{  }int\DUrole{p,p}{{]}}}}}{{ $\rightarrow$ None}}
\pysigstopsignatures
\sphinxAtStartPar
Allocate arrays dependent on the output size. Frees previously allocated arrays.

\sphinxAtStartPar
This may have to be called multiple times if the output size changes.

\end{fulllineitems}

\index{output\_size (deepdrr.Projector property)@\spxentry{output\_size}\spxextra{deepdrr.Projector property}}

\begin{fulllineitems}
\phantomsection\label{\detokenize{deepdrr:deepdrr.Projector.output_size}}
\pysigstartsignatures
\pysigline{\sphinxbfcode{\sphinxupquote{property\DUrole{w,w}{  }}}\sphinxbfcode{\sphinxupquote{output\_size}}\sphinxbfcode{\sphinxupquote{\DUrole{p,p}{:}\DUrole{w,w}{  }int}}}
\pysigstopsignatures
\end{fulllineitems}

\index{project() (deepdrr.Projector method)@\spxentry{project()}\spxextra{deepdrr.Projector method}}

\begin{fulllineitems}
\phantomsection\label{\detokenize{deepdrr:deepdrr.Projector.project}}
\pysigstartsignatures
\pysiglinewithargsret{\sphinxbfcode{\sphinxupquote{project}}}{\sphinxparam{\DUrole{o,o}{*}\DUrole{n,n}{camera\_projections}\DUrole{p,p}{:}\DUrole{w,w}{  }\DUrole{n,n}{{\hyperref[\detokenize{deepdrr.geo:deepdrr.geo.core.CameraProjection}]{\sphinxcrossref{CameraProjection}}}}}}{{ $\rightarrow$ ndarray}}
\pysigstopsignatures
\sphinxAtStartPar
Perform the projection.
\begin{quote}\begin{description}
\sphinxlineitem{Parameters}
\sphinxAtStartPar
\sphinxstyleliteralstrong{\sphinxupquote{camera\_projection}} \textendash{} any number of camera projections. If none are provided, the Projector uses the CArm device to obtain a camera projection.

\sphinxlineitem{Raises}
\sphinxAtStartPar
\sphinxstyleliteralstrong{\sphinxupquote{ValueError}} \textendash{} if no projections are provided and self.device is None.

\sphinxlineitem{Returns}
\sphinxAtStartPar
array of DRRs, after mass attenuation, etc.

\sphinxlineitem{Return type}
\sphinxAtStartPar
np.ndarray

\end{description}\end{quote}

\end{fulllineitems}

\index{project\_over\_carm\_range() (deepdrr.Projector method)@\spxentry{project\_over\_carm\_range()}\spxextra{deepdrr.Projector method}}

\begin{fulllineitems}
\phantomsection\label{\detokenize{deepdrr:deepdrr.Projector.project_over_carm_range}}
\pysigstartsignatures
\pysiglinewithargsret{\sphinxbfcode{\sphinxupquote{project\_over\_carm\_range}}}{\sphinxparam{\DUrole{n,n}{phi\_range}\DUrole{p,p}{:}\DUrole{w,w}{  }\DUrole{n,n}{Tuple\DUrole{p,p}{{[}}float\DUrole{p,p}{,}\DUrole{w,w}{  }float\DUrole{p,p}{,}\DUrole{w,w}{  }float\DUrole{p,p}{{]}}}}\sphinxparamcomma \sphinxparam{\DUrole{n,n}{theta\_range}\DUrole{p,p}{:}\DUrole{w,w}{  }\DUrole{n,n}{Tuple\DUrole{p,p}{{[}}float\DUrole{p,p}{,}\DUrole{w,w}{  }float\DUrole{p,p}{,}\DUrole{w,w}{  }float\DUrole{p,p}{{]}}}}\sphinxparamcomma \sphinxparam{\DUrole{n,n}{degrees}\DUrole{p,p}{:}\DUrole{w,w}{  }\DUrole{n,n}{bool}\DUrole{w,w}{  }\DUrole{o,o}{=}\DUrole{w,w}{  }\DUrole{default_value}{True}}}{{ $\rightarrow$ ndarray}}
\pysigstopsignatures
\sphinxAtStartPar
Project over a range of angles using the included CArm.

\sphinxAtStartPar
Ignores the CArm’s internal pose, except for its isocenter.

\end{fulllineitems}

\index{source\_to\_detector\_distance (deepdrr.Projector property)@\spxentry{source\_to\_detector\_distance}\spxextra{deepdrr.Projector property}}

\begin{fulllineitems}
\phantomsection\label{\detokenize{deepdrr:deepdrr.Projector.source_to_detector_distance}}
\pysigstartsignatures
\pysigline{\sphinxbfcode{\sphinxupquote{property\DUrole{w,w}{  }}}\sphinxbfcode{\sphinxupquote{source\_to\_detector\_distance}}\sphinxbfcode{\sphinxupquote{\DUrole{p,p}{:}\DUrole{w,w}{  }float}}}
\pysigstopsignatures
\end{fulllineitems}

\index{volume (deepdrr.Projector property)@\spxentry{volume}\spxextra{deepdrr.Projector property}}

\begin{fulllineitems}
\phantomsection\label{\detokenize{deepdrr:deepdrr.Projector.volume}}
\pysigstartsignatures
\pysigline{\sphinxbfcode{\sphinxupquote{property\DUrole{w,w}{  }}}\sphinxbfcode{\sphinxupquote{volume}}}
\pysigstopsignatures
\end{fulllineitems}

\index{volumes (deepdrr.Projector attribute)@\spxentry{volumes}\spxextra{deepdrr.Projector attribute}}

\begin{fulllineitems}
\phantomsection\label{\detokenize{deepdrr:deepdrr.Projector.volumes}}
\pysigstartsignatures
\pysigline{\sphinxbfcode{\sphinxupquote{volumes}}\sphinxbfcode{\sphinxupquote{\DUrole{p,p}{:}\DUrole{w,w}{  }List\DUrole{p,p}{{[}}{\hyperref[\detokenize{deepdrr.vol:deepdrr.vol.volume.Volume}]{\sphinxcrossref{Volume}}}\DUrole{p,p}{{]}}}}}
\pysigstopsignatures
\end{fulllineitems}


\end{fulllineitems}

\index{Volume (class in deepdrr)@\spxentry{Volume}\spxextra{class in deepdrr}}

\begin{fulllineitems}
\phantomsection\label{\detokenize{deepdrr:deepdrr.Volume}}
\pysigstartsignatures
\pysiglinewithargsret{\sphinxbfcode{\sphinxupquote{class\DUrole{w,w}{  }}}\sphinxcode{\sphinxupquote{deepdrr.}}\sphinxbfcode{\sphinxupquote{Volume}}}{\sphinxparam{\DUrole{n,n}{data}\DUrole{p,p}{:}\DUrole{w,w}{  }\DUrole{n,n}{ndarray}}\sphinxparamcomma \sphinxparam{\DUrole{n,n}{materials}\DUrole{p,p}{:}\DUrole{w,w}{  }\DUrole{n,n}{Dict\DUrole{p,p}{{[}}str\DUrole{p,p}{,}\DUrole{w,w}{  }ndarray\DUrole{p,p}{{]}}}}\sphinxparamcomma \sphinxparam{\DUrole{n,n}{anatomical\_from\_IJK}\DUrole{p,p}{:}\DUrole{w,w}{  }\DUrole{n,n}{{\hyperref[\detokenize{deepdrr.geo:deepdrr.geo.core.FrameTransform}]{\sphinxcrossref{FrameTransform}}}\DUrole{w,w}{  }\DUrole{p,p}{|}\DUrole{w,w}{  }None}\DUrole{w,w}{  }\DUrole{o,o}{=}\DUrole{w,w}{  }\DUrole{default_value}{None}}\sphinxparamcomma \sphinxparam{\DUrole{n,n}{world\_from\_anatomical}\DUrole{p,p}{:}\DUrole{w,w}{  }\DUrole{n,n}{{\hyperref[\detokenize{deepdrr.geo:deepdrr.geo.core.FrameTransform}]{\sphinxcrossref{FrameTransform}}}\DUrole{w,w}{  }\DUrole{p,p}{|}\DUrole{w,w}{  }None}\DUrole{w,w}{  }\DUrole{o,o}{=}\DUrole{w,w}{  }\DUrole{default_value}{None}}\sphinxparamcomma \sphinxparam{\DUrole{n,n}{anatomical\_coordinate\_system}\DUrole{p,p}{:}\DUrole{w,w}{  }\DUrole{n,n}{str\DUrole{w,w}{  }\DUrole{p,p}{|}\DUrole{w,w}{  }None}\DUrole{w,w}{  }\DUrole{o,o}{=}\DUrole{w,w}{  }\DUrole{default_value}{None}}\sphinxparamcomma \sphinxparam{\DUrole{n,n}{cache\_dir}\DUrole{p,p}{:}\DUrole{w,w}{  }\DUrole{n,n}{str\DUrole{w,w}{  }\DUrole{p,p}{|}\DUrole{w,w}{  }None}\DUrole{w,w}{  }\DUrole{o,o}{=}\DUrole{w,w}{  }\DUrole{default_value}{None}}\sphinxparamcomma \sphinxparam{\DUrole{n,n}{config}\DUrole{p,p}{:}\DUrole{w,w}{  }\DUrole{n,n}{Dict\DUrole{p,p}{{[}}str\DUrole{p,p}{,}\DUrole{w,w}{  }Any\DUrole{p,p}{{]}}}\DUrole{w,w}{  }\DUrole{o,o}{=}\DUrole{w,w}{  }\DUrole{default_value}{\{\}}}\sphinxparamcomma \sphinxparam{\DUrole{n,n}{anatomical\_from\_ijk}\DUrole{p,p}{:}\DUrole{w,w}{  }\DUrole{n,n}{{\hyperref[\detokenize{deepdrr.geo:deepdrr.geo.core.FrameTransform}]{\sphinxcrossref{FrameTransform}}}\DUrole{w,w}{  }\DUrole{p,p}{|}\DUrole{w,w}{  }None}\DUrole{w,w}{  }\DUrole{o,o}{=}\DUrole{w,w}{  }\DUrole{default_value}{None}}}{}
\pysigstopsignatures
\sphinxAtStartPar
Bases: \sphinxcode{\sphinxupquote{object}}
\index{IJK\_from\_LPS (deepdrr.Volume property)@\spxentry{IJK\_from\_LPS}\spxextra{deepdrr.Volume property}}

\begin{fulllineitems}
\phantomsection\label{\detokenize{deepdrr:deepdrr.Volume.IJK_from_LPS}}
\pysigstartsignatures
\pysigline{\sphinxbfcode{\sphinxupquote{property\DUrole{w,w}{  }}}\sphinxbfcode{\sphinxupquote{IJK\_from\_LPS}}\sphinxbfcode{\sphinxupquote{\DUrole{p,p}{:}\DUrole{w,w}{  }{\hyperref[\detokenize{deepdrr.geo:deepdrr.geo.core.FrameTransform}]{\sphinxcrossref{FrameTransform}}}}}}
\pysigstopsignatures
\end{fulllineitems}

\index{IJK\_from\_RAS (deepdrr.Volume property)@\spxentry{IJK\_from\_RAS}\spxextra{deepdrr.Volume property}}

\begin{fulllineitems}
\phantomsection\label{\detokenize{deepdrr:deepdrr.Volume.IJK_from_RAS}}
\pysigstartsignatures
\pysigline{\sphinxbfcode{\sphinxupquote{property\DUrole{w,w}{  }}}\sphinxbfcode{\sphinxupquote{IJK\_from\_RAS}}}
\pysigstopsignatures
\end{fulllineitems}

\index{IJK\_from\_anatomical (deepdrr.Volume property)@\spxentry{IJK\_from\_anatomical}\spxextra{deepdrr.Volume property}}

\begin{fulllineitems}
\phantomsection\label{\detokenize{deepdrr:deepdrr.Volume.IJK_from_anatomical}}
\pysigstartsignatures
\pysigline{\sphinxbfcode{\sphinxupquote{property\DUrole{w,w}{  }}}\sphinxbfcode{\sphinxupquote{IJK\_from\_anatomical}}}
\pysigstopsignatures
\end{fulllineitems}

\index{IJK\_from\_world (deepdrr.Volume property)@\spxentry{IJK\_from\_world}\spxextra{deepdrr.Volume property}}

\begin{fulllineitems}
\phantomsection\label{\detokenize{deepdrr:deepdrr.Volume.IJK_from_world}}
\pysigstartsignatures
\pysigline{\sphinxbfcode{\sphinxupquote{property\DUrole{w,w}{  }}}\sphinxbfcode{\sphinxupquote{IJK\_from\_world}}\sphinxbfcode{\sphinxupquote{\DUrole{p,p}{:}\DUrole{w,w}{  }{\hyperref[\detokenize{deepdrr.geo:deepdrr.geo.core.FrameTransform}]{\sphinxcrossref{FrameTransform}}}}}}
\pysigstopsignatures
\end{fulllineitems}

\index{LPS\_from\_IJK (deepdrr.Volume property)@\spxentry{LPS\_from\_IJK}\spxextra{deepdrr.Volume property}}

\begin{fulllineitems}
\phantomsection\label{\detokenize{deepdrr:deepdrr.Volume.LPS_from_IJK}}
\pysigstartsignatures
\pysigline{\sphinxbfcode{\sphinxupquote{property\DUrole{w,w}{  }}}\sphinxbfcode{\sphinxupquote{LPS\_from\_IJK}}\sphinxbfcode{\sphinxupquote{\DUrole{p,p}{:}\DUrole{w,w}{  }{\hyperref[\detokenize{deepdrr.geo:deepdrr.geo.core.FrameTransform}]{\sphinxcrossref{FrameTransform}}}}}}
\pysigstopsignatures
\sphinxAtStartPar
Get the LPS\_from\_IJK transform.

\end{fulllineitems}

\index{RAS\_from\_IJK (deepdrr.Volume property)@\spxentry{RAS\_from\_IJK}\spxextra{deepdrr.Volume property}}

\begin{fulllineitems}
\phantomsection\label{\detokenize{deepdrr:deepdrr.Volume.RAS_from_IJK}}
\pysigstartsignatures
\pysigline{\sphinxbfcode{\sphinxupquote{property\DUrole{w,w}{  }}}\sphinxbfcode{\sphinxupquote{RAS\_from\_IJK}}}
\pysigstopsignatures
\sphinxAtStartPar
Get the RAS\_from\_IJK transform.

\end{fulllineitems}

\index{anatomical\_coordinate\_system (deepdrr.Volume attribute)@\spxentry{anatomical\_coordinate\_system}\spxextra{deepdrr.Volume attribute}}

\begin{fulllineitems}
\phantomsection\label{\detokenize{deepdrr:deepdrr.Volume.anatomical_coordinate_system}}
\pysigstartsignatures
\pysigline{\sphinxbfcode{\sphinxupquote{anatomical\_coordinate\_system}}\sphinxbfcode{\sphinxupquote{\DUrole{p,p}{:}\DUrole{w,w}{  }str\DUrole{w,w}{  }\DUrole{p,p}{|}\DUrole{w,w}{  }None}}}
\pysigstopsignatures
\end{fulllineitems}

\index{anatomical\_from\_IJK (deepdrr.Volume attribute)@\spxentry{anatomical\_from\_IJK}\spxextra{deepdrr.Volume attribute}}

\begin{fulllineitems}
\phantomsection\label{\detokenize{deepdrr:deepdrr.Volume.anatomical_from_IJK}}
\pysigstartsignatures
\pysigline{\sphinxbfcode{\sphinxupquote{anatomical\_from\_IJK}}\sphinxbfcode{\sphinxupquote{\DUrole{p,p}{:}\DUrole{w,w}{  }{\hyperref[\detokenize{deepdrr.geo:deepdrr.geo.core.FrameTransform}]{\sphinxcrossref{FrameTransform}}}}}}
\pysigstopsignatures
\end{fulllineitems}

\index{anatomical\_from\_ijk (deepdrr.Volume property)@\spxentry{anatomical\_from\_ijk}\spxextra{deepdrr.Volume property}}

\begin{fulllineitems}
\phantomsection\label{\detokenize{deepdrr:deepdrr.Volume.anatomical_from_ijk}}
\pysigstartsignatures
\pysigline{\sphinxbfcode{\sphinxupquote{property\DUrole{w,w}{  }}}\sphinxbfcode{\sphinxupquote{anatomical\_from\_ijk}}\sphinxbfcode{\sphinxupquote{\DUrole{p,p}{:}\DUrole{w,w}{  }{\hyperref[\detokenize{deepdrr.geo:deepdrr.geo.core.FrameTransform}]{\sphinxcrossref{FrameTransform}}}}}}
\pysigstopsignatures
\end{fulllineitems}

\index{anatomical\_from\_world (deepdrr.Volume property)@\spxentry{anatomical\_from\_world}\spxextra{deepdrr.Volume property}}

\begin{fulllineitems}
\phantomsection\label{\detokenize{deepdrr:deepdrr.Volume.anatomical_from_world}}
\pysigstartsignatures
\pysigline{\sphinxbfcode{\sphinxupquote{property\DUrole{w,w}{  }}}\sphinxbfcode{\sphinxupquote{anatomical\_from\_world}}}
\pysigstopsignatures
\end{fulllineitems}

\index{cache\_dir (deepdrr.Volume attribute)@\spxentry{cache\_dir}\spxextra{deepdrr.Volume attribute}}

\begin{fulllineitems}
\phantomsection\label{\detokenize{deepdrr:deepdrr.Volume.cache_dir}}
\pysigstartsignatures
\pysigline{\sphinxbfcode{\sphinxupquote{cache\_dir}}\sphinxbfcode{\sphinxupquote{\DUrole{p,p}{:}\DUrole{w,w}{  }Path\DUrole{w,w}{  }\DUrole{p,p}{|}\DUrole{w,w}{  }None}}\sphinxbfcode{\sphinxupquote{\DUrole{w,w}{  }\DUrole{p,p}{=}\DUrole{w,w}{  }None}}}
\pysigstopsignatures
\end{fulllineitems}

\index{center\_in\_world (deepdrr.Volume property)@\spxentry{center\_in\_world}\spxextra{deepdrr.Volume property}}

\begin{fulllineitems}
\phantomsection\label{\detokenize{deepdrr:deepdrr.Volume.center_in_world}}
\pysigstartsignatures
\pysigline{\sphinxbfcode{\sphinxupquote{property\DUrole{w,w}{  }}}\sphinxbfcode{\sphinxupquote{center\_in\_world}}\sphinxbfcode{\sphinxupquote{\DUrole{p,p}{:}\DUrole{w,w}{  }{\hyperref[\detokenize{deepdrr.geo:deepdrr.geo.core.Point3D}]{\sphinxcrossref{Point3D}}}}}}
\pysigstopsignatures
\sphinxAtStartPar
The center of the volume in world coorindates. Useful for debugging.

\end{fulllineitems}

\index{copy\_pose() (deepdrr.Volume method)@\spxentry{copy\_pose()}\spxextra{deepdrr.Volume method}}

\begin{fulllineitems}
\phantomsection\label{\detokenize{deepdrr:deepdrr.Volume.copy_pose}}
\pysigstartsignatures
\pysiglinewithargsret{\sphinxbfcode{\sphinxupquote{copy\_pose}}}{\sphinxparam{\DUrole{n,n}{other}\DUrole{p,p}{:}\DUrole{w,w}{  }\DUrole{n,n}{{\hyperref[\detokenize{deepdrr.vol:deepdrr.vol.volume.Volume}]{\sphinxcrossref{Volume}}}}}}{{ $\rightarrow$ None}}
\pysigstopsignatures
\sphinxAtStartPar
Copy the pose of another volume.

\end{fulllineitems}

\index{crop() (deepdrr.Volume method)@\spxentry{crop()}\spxextra{deepdrr.Volume method}}

\begin{fulllineitems}
\phantomsection\label{\detokenize{deepdrr:deepdrr.Volume.crop}}
\pysigstartsignatures
\pysiglinewithargsret{\sphinxbfcode{\sphinxupquote{crop}}}{\sphinxparam{\DUrole{n,n}{crop\_box}\DUrole{p,p}{:}\DUrole{w,w}{  }\DUrole{n,n}{Tuple\DUrole{p,p}{{[}}Tuple\DUrole{p,p}{{[}}float\DUrole{p,p}{,}\DUrole{w,w}{  }float\DUrole{p,p}{{]}}\DUrole{p,p}{,}\DUrole{w,w}{  }\DUrole{p,p}{...}\DUrole{p,p}{{]}}}}}{{ $\rightarrow$ {\hyperref[\detokenize{deepdrr.vol:deepdrr.vol.volume.Volume}]{\sphinxcrossref{Volume}}}}}
\pysigstopsignatures
\sphinxAtStartPar
Crop the volume to a given bounding box.
\begin{quote}\begin{description}
\sphinxlineitem{Parameters}
\sphinxAtStartPar
\sphinxstyleliteralstrong{\sphinxupquote{crop\_box}} (\sphinxstyleliteralemphasis{\sphinxupquote{Tuple}}\sphinxstyleliteralemphasis{\sphinxupquote{{[}}}\sphinxstyleliteralemphasis{\sphinxupquote{Tuple}}\sphinxstyleliteralemphasis{\sphinxupquote{{[}}}\sphinxstyleliteralemphasis{\sphinxupquote{float}}\sphinxstyleliteralemphasis{\sphinxupquote{, }}\sphinxstyleliteralemphasis{\sphinxupquote{float}}\sphinxstyleliteralemphasis{\sphinxupquote{{]}}}\sphinxstyleliteralemphasis{\sphinxupquote{, }}\sphinxstyleliteralemphasis{\sphinxupquote{...}}\sphinxstyleliteralemphasis{\sphinxupquote{{]}}}) \textendash{} The bounding box to crop to, in IJK.

\sphinxlineitem{Returns}
\sphinxAtStartPar
The cropped volume.

\sphinxlineitem{Return type}
\sphinxAtStartPar
{\hyperref[\detokenize{deepdrr:deepdrr.Volume}]{\sphinxcrossref{Volume}}}

\end{description}\end{quote}

\end{fulllineitems}

\index{data (deepdrr.Volume attribute)@\spxentry{data}\spxextra{deepdrr.Volume attribute}}

\begin{fulllineitems}
\phantomsection\label{\detokenize{deepdrr:deepdrr.Volume.data}}
\pysigstartsignatures
\pysigline{\sphinxbfcode{\sphinxupquote{data}}\sphinxbfcode{\sphinxupquote{\DUrole{p,p}{:}\DUrole{w,w}{  }ndarray}}}
\pysigstopsignatures
\end{fulllineitems}

\index{facedown() (deepdrr.Volume method)@\spxentry{facedown()}\spxextra{deepdrr.Volume method}}

\begin{fulllineitems}
\phantomsection\label{\detokenize{deepdrr:deepdrr.Volume.facedown}}
\pysigstartsignatures
\pysiglinewithargsret{\sphinxbfcode{\sphinxupquote{facedown}}}{}{}
\pysigstopsignatures
\sphinxAtStartPar
Turns the volume to be face down.

\sphinxAtStartPar
This aligns the patient so that, in world space,
the posterior side is toward +Z, inferior is toward +X,
and right is toward +Y.
\begin{quote}\begin{description}
\sphinxlineitem{Raises}
\sphinxAtStartPar
\sphinxstyleliteralstrong{\sphinxupquote{NotImplementedError}} \textendash{} If the anatomical coordinate system is not “RAS”.

\end{description}\end{quote}

\end{fulllineitems}

\index{faceup() (deepdrr.Volume method)@\spxentry{faceup()}\spxextra{deepdrr.Volume method}}

\begin{fulllineitems}
\phantomsection\label{\detokenize{deepdrr:deepdrr.Volume.faceup}}
\pysigstartsignatures
\pysiglinewithargsret{\sphinxbfcode{\sphinxupquote{faceup}}}{}{}
\pysigstopsignatures
\sphinxAtStartPar
Turns the volume to be face up.

\sphinxAtStartPar
This aligns the patient so that, in world space,
the anterior side is toward +Z, inferior is toward +X,
and left is toward +Y.
\begin{quote}\begin{description}
\sphinxlineitem{Raises}
\sphinxAtStartPar
\sphinxstyleliteralstrong{\sphinxupquote{NotImplementedError}} \textendash{} If the anatomical coordinate system is not “RAS”.

\end{description}\end{quote}

\end{fulllineitems}

\index{from\_dicom() (deepdrr.Volume class method)@\spxentry{from\_dicom()}\spxextra{deepdrr.Volume class method}}

\begin{fulllineitems}
\phantomsection\label{\detokenize{deepdrr:deepdrr.Volume.from_dicom}}
\pysigstartsignatures
\pysiglinewithargsret{\sphinxbfcode{\sphinxupquote{classmethod\DUrole{w,w}{  }}}\sphinxbfcode{\sphinxupquote{from\_dicom}}}{\sphinxparam{\DUrole{n,n}{path}\DUrole{p,p}{:}\DUrole{w,w}{  }\DUrole{n,n}{Path}}\sphinxparamcomma \sphinxparam{\DUrole{n,n}{use\_thresholding}\DUrole{p,p}{:}\DUrole{w,w}{  }\DUrole{n,n}{bool}\DUrole{w,w}{  }\DUrole{o,o}{=}\DUrole{w,w}{  }\DUrole{default_value}{True}}\sphinxparamcomma \sphinxparam{\DUrole{n,n}{world\_from\_anatomical}\DUrole{p,p}{:}\DUrole{w,w}{  }\DUrole{n,n}{{\hyperref[\detokenize{deepdrr.geo:deepdrr.geo.core.FrameTransform}]{\sphinxcrossref{FrameTransform}}}\DUrole{w,w}{  }\DUrole{p,p}{|}\DUrole{w,w}{  }None}\DUrole{w,w}{  }\DUrole{o,o}{=}\DUrole{w,w}{  }\DUrole{default_value}{None}}\sphinxparamcomma \sphinxparam{\DUrole{n,n}{use\_cached}\DUrole{p,p}{:}\DUrole{w,w}{  }\DUrole{n,n}{bool}\DUrole{w,w}{  }\DUrole{o,o}{=}\DUrole{w,w}{  }\DUrole{default_value}{True}}\sphinxparamcomma \sphinxparam{\DUrole{n,n}{cache\_dir}\DUrole{p,p}{:}\DUrole{w,w}{  }\DUrole{n,n}{Path\DUrole{w,w}{  }\DUrole{p,p}{|}\DUrole{w,w}{  }None}\DUrole{w,w}{  }\DUrole{o,o}{=}\DUrole{w,w}{  }\DUrole{default_value}{None}}\sphinxparamcomma \sphinxparam{\DUrole{o,o}{**}\DUrole{n,n}{kwargs}}}{}
\pysigstopsignatures
\sphinxAtStartPar
load a volume from a dicom file and compute the anatomical\_from\_ijk transform from metadata
\sphinxurl{https://www.slicer.org/wiki/Coordinate\_systems}
:param path: path\sphinxhyphen{}like to a multi\sphinxhyphen{}frame dicom file. (Currently only Multi\sphinxhyphen{}Frame from Siemens supported)
:param use\_thresholding: segment the materials using thresholding (faster but less accurate). Defaults to True.
:type use\_thresholding: bool, optional
:param world\_from\_anatomical: position the volume in world space. If None, uses identity. Defaults to None.
:type world\_from\_anatomical: Optional{[}geo.FrameTransform{]}, optional
:param use\_cached: {[}description{]}. Use a cached segmentation if available. Defaults to True.
:type use\_cached: bool, optional
:param cache\_dir: Where to load/save the cached segmentation. If None, use the parent dir of \sphinxtitleref{path}. Defaults to None.
:type cache\_dir: Optional{[}Path{]}, optional
\begin{quote}\begin{description}
\sphinxlineitem{Returns}
\sphinxAtStartPar
an instance of a deepdrr volume

\sphinxlineitem{Return type}
\sphinxAtStartPar
{\hyperref[\detokenize{deepdrr:deepdrr.Volume}]{\sphinxcrossref{Volume}}}

\end{description}\end{quote}

\end{fulllineitems}

\index{from\_hu() (deepdrr.Volume class method)@\spxentry{from\_hu()}\spxextra{deepdrr.Volume class method}}

\begin{fulllineitems}
\phantomsection\label{\detokenize{deepdrr:deepdrr.Volume.from_hu}}
\pysigstartsignatures
\pysiglinewithargsret{\sphinxbfcode{\sphinxupquote{classmethod\DUrole{w,w}{  }}}\sphinxbfcode{\sphinxupquote{from\_hu}}}{\sphinxparam{\DUrole{n,n}{hu\_values}\DUrole{p,p}{:}\DUrole{w,w}{  }\DUrole{n,n}{ndarray}}\sphinxparamcomma \sphinxparam{\DUrole{n,n}{origin}\DUrole{p,p}{:}\DUrole{w,w}{  }\DUrole{n,n}{{\hyperref[\detokenize{deepdrr.geo:deepdrr.geo.core.Point3D}]{\sphinxcrossref{Point3D}}}}}\sphinxparamcomma \sphinxparam{\DUrole{n,n}{use\_thresholding}\DUrole{p,p}{:}\DUrole{w,w}{  }\DUrole{n,n}{bool}\DUrole{w,w}{  }\DUrole{o,o}{=}\DUrole{w,w}{  }\DUrole{default_value}{True}}\sphinxparamcomma \sphinxparam{\DUrole{n,n}{spacing}\DUrole{p,p}{:}\DUrole{w,w}{  }\DUrole{n,n}{{\hyperref[\detokenize{deepdrr.geo:deepdrr.geo.core.Vector3D}]{\sphinxcrossref{Vector3D}}}\DUrole{w,w}{  }\DUrole{p,p}{|}\DUrole{w,w}{  }None}\DUrole{w,w}{  }\DUrole{o,o}{=}\DUrole{w,w}{  }\DUrole{default_value}{(1, 1, 1)}}\sphinxparamcomma \sphinxparam{\DUrole{n,n}{anatomical\_coordinate\_system}\DUrole{p,p}{:}\DUrole{w,w}{  }\DUrole{n,n}{str\DUrole{w,w}{  }\DUrole{p,p}{|}\DUrole{w,w}{  }None}\DUrole{w,w}{  }\DUrole{o,o}{=}\DUrole{w,w}{  }\DUrole{default_value}{None}}\sphinxparamcomma \sphinxparam{\DUrole{n,n}{world\_from\_anatomical}\DUrole{p,p}{:}\DUrole{w,w}{  }\DUrole{n,n}{{\hyperref[\detokenize{deepdrr.geo:deepdrr.geo.core.FrameTransform}]{\sphinxcrossref{FrameTransform}}}\DUrole{w,w}{  }\DUrole{p,p}{|}\DUrole{w,w}{  }None}\DUrole{w,w}{  }\DUrole{o,o}{=}\DUrole{w,w}{  }\DUrole{default_value}{None}}\sphinxparamcomma \sphinxparam{\DUrole{o,o}{**}\DUrole{n,n}{kwargs}}}{{ $\rightarrow$ None}}
\pysigstopsignatures
\end{fulllineitems}

\index{from\_nifti() (deepdrr.Volume class method)@\spxentry{from\_nifti()}\spxextra{deepdrr.Volume class method}}

\begin{fulllineitems}
\phantomsection\label{\detokenize{deepdrr:deepdrr.Volume.from_nifti}}
\pysigstartsignatures
\pysiglinewithargsret{\sphinxbfcode{\sphinxupquote{classmethod\DUrole{w,w}{  }}}\sphinxbfcode{\sphinxupquote{from\_nifti}}}{\sphinxparam{\DUrole{n,n}{path}\DUrole{p,p}{:}\DUrole{w,w}{  }\DUrole{n,n}{Path}}\sphinxparamcomma \sphinxparam{\DUrole{n,n}{world\_from\_anatomical}\DUrole{p,p}{:}\DUrole{w,w}{  }\DUrole{n,n}{{\hyperref[\detokenize{deepdrr.geo:deepdrr.geo.core.FrameTransform}]{\sphinxcrossref{FrameTransform}}}\DUrole{w,w}{  }\DUrole{p,p}{|}\DUrole{w,w}{  }None}\DUrole{w,w}{  }\DUrole{o,o}{=}\DUrole{w,w}{  }\DUrole{default_value}{None}}\sphinxparamcomma \sphinxparam{\DUrole{n,n}{use\_thresholding}\DUrole{p,p}{:}\DUrole{w,w}{  }\DUrole{n,n}{bool}\DUrole{w,w}{  }\DUrole{o,o}{=}\DUrole{w,w}{  }\DUrole{default_value}{True}}\sphinxparamcomma \sphinxparam{\DUrole{n,n}{use\_cached}\DUrole{p,p}{:}\DUrole{w,w}{  }\DUrole{n,n}{bool}\DUrole{w,w}{  }\DUrole{o,o}{=}\DUrole{w,w}{  }\DUrole{default_value}{True}}\sphinxparamcomma \sphinxparam{\DUrole{n,n}{save\_cache}\DUrole{p,p}{:}\DUrole{w,w}{  }\DUrole{n,n}{bool}\DUrole{w,w}{  }\DUrole{o,o}{=}\DUrole{w,w}{  }\DUrole{default_value}{False}}\sphinxparamcomma \sphinxparam{\DUrole{n,n}{cache\_dir}\DUrole{p,p}{:}\DUrole{w,w}{  }\DUrole{n,n}{Path\DUrole{w,w}{  }\DUrole{p,p}{|}\DUrole{w,w}{  }None}\DUrole{w,w}{  }\DUrole{o,o}{=}\DUrole{w,w}{  }\DUrole{default_value}{None}}\sphinxparamcomma \sphinxparam{\DUrole{n,n}{materials}\DUrole{p,p}{:}\DUrole{w,w}{  }\DUrole{n,n}{Dict\DUrole{p,p}{{[}}str\DUrole{p,p}{,}\DUrole{w,w}{  }ndarray\DUrole{p,p}{{]}}\DUrole{w,w}{  }\DUrole{p,p}{|}\DUrole{w,w}{  }None}\DUrole{w,w}{  }\DUrole{o,o}{=}\DUrole{w,w}{  }\DUrole{default_value}{None}}\sphinxparamcomma \sphinxparam{\DUrole{n,n}{segmentation}\DUrole{p,p}{:}\DUrole{w,w}{  }\DUrole{n,n}{bool}\DUrole{w,w}{  }\DUrole{o,o}{=}\DUrole{w,w}{  }\DUrole{default_value}{False}}\sphinxparamcomma \sphinxparam{\DUrole{n,n}{label}\DUrole{p,p}{:}\DUrole{w,w}{  }\DUrole{n,n}{None\DUrole{w,w}{  }\DUrole{p,p}{|}\DUrole{w,w}{  }int\DUrole{w,w}{  }\DUrole{p,p}{|}\DUrole{w,w}{  }List\DUrole{p,p}{{[}}int\DUrole{p,p}{{]}}}\DUrole{w,w}{  }\DUrole{o,o}{=}\DUrole{w,w}{  }\DUrole{default_value}{None}}\sphinxparamcomma \sphinxparam{\DUrole{n,n}{density\_kwargs}\DUrole{p,p}{:}\DUrole{w,w}{  }\DUrole{n,n}{dict}\DUrole{w,w}{  }\DUrole{o,o}{=}\DUrole{w,w}{  }\DUrole{default_value}{\{\}}}\sphinxparamcomma \sphinxparam{\DUrole{o,o}{**}\DUrole{n,n}{kwargs}}}{}
\pysigstopsignatures
\sphinxAtStartPar
Load a volume from NiFti file.
\begin{quote}\begin{description}
\sphinxlineitem{Parameters}\begin{itemize}
\item {} 
\sphinxAtStartPar
\sphinxstyleliteralstrong{\sphinxupquote{path}} (\sphinxstyleliteralemphasis{\sphinxupquote{Path}}) \textendash{} path to the .nii.gz file.

\item {} 
\sphinxAtStartPar
\sphinxstyleliteralstrong{\sphinxupquote{use\_thresholding}} (\sphinxstyleliteralemphasis{\sphinxupquote{bool}}\sphinxstyleliteralemphasis{\sphinxupquote{, }}\sphinxstyleliteralemphasis{\sphinxupquote{optional}}) \textendash{} segment the materials using thresholding (faster but less accurate). Defaults to True.

\item {} 
\sphinxAtStartPar
\sphinxstyleliteralstrong{\sphinxupquote{world\_from\_anatomical}} (\sphinxstyleliteralemphasis{\sphinxupquote{Optional}}\sphinxstyleliteralemphasis{\sphinxupquote{{[}}}{\hyperref[\detokenize{deepdrr.geo:deepdrr.geo.FrameTransform}]{\sphinxcrossref{\sphinxstyleliteralemphasis{\sphinxupquote{geo.FrameTransform}}}}}\sphinxstyleliteralemphasis{\sphinxupquote{{]}}}\sphinxstyleliteralemphasis{\sphinxupquote{, }}\sphinxstyleliteralemphasis{\sphinxupquote{optional}}) \textendash{} position the volume in world space. If None, uses identity. Defaults to None.

\item {} 
\sphinxAtStartPar
\sphinxstyleliteralstrong{\sphinxupquote{use\_cached}} (\sphinxstyleliteralemphasis{\sphinxupquote{bool}}\sphinxstyleliteralemphasis{\sphinxupquote{, }}\sphinxstyleliteralemphasis{\sphinxupquote{optional}}) \textendash{} Use a cached segmentation if available. Defaults to True.

\item {} 
\sphinxAtStartPar
\sphinxstyleliteralstrong{\sphinxupquote{cache\_dir}} (\sphinxstyleliteralemphasis{\sphinxupquote{Optional}}\sphinxstyleliteralemphasis{\sphinxupquote{{[}}}\sphinxstyleliteralemphasis{\sphinxupquote{Path}}\sphinxstyleliteralemphasis{\sphinxupquote{{]}}}\sphinxstyleliteralemphasis{\sphinxupquote{, }}\sphinxstyleliteralemphasis{\sphinxupquote{optional}}) \textendash{} Where to load/save the cached segmentation. If None, use a “cache” directory
in the same location as the nifti file. Defaults to None.

\item {} 
\sphinxAtStartPar
\sphinxstyleliteralstrong{\sphinxupquote{materials}} \textendash{} Optional material segmentation, as a dictionary mapping material name to binary segmentation.
If not provided, materials are segmented from the CT. Defaults to None.
Can also provide a dictionary mapping material names to Nifti files containing the segmentations.

\item {} 
\sphinxAtStartPar
\sphinxstyleliteralstrong{\sphinxupquote{segmentation}} (\sphinxstyleliteralemphasis{\sphinxupquote{bool}}\sphinxstyleliteralemphasis{\sphinxupquote{, }}\sphinxstyleliteralemphasis{\sphinxupquote{optional}}\sphinxstyleliteralemphasis{\sphinxupquote{) }}\sphinxstyleliteralemphasis{\sphinxupquote{If the file is a segmentation file}}\sphinxstyleliteralemphasis{\sphinxupquote{, }}\sphinxstyleliteralemphasis{\sphinxupquote{then its "materials" correspond to a high density material}}\sphinxstyleliteralemphasis{\sphinxupquote{ (}}\sphinxstyleliteralemphasis{\sphinxupquote{bone}}) \textendash{} where the values are \textgreater{}0. Defaults to false. Overrides provided materials.

\item {} 
\sphinxAtStartPar
\sphinxstyleliteralstrong{\sphinxupquote{label}} \textendash{} which labels to treat as solid. If None, then all nonzero labels are treated as solid. Defaults to None.

\item {} 
\sphinxAtStartPar
\sphinxstyleliteralstrong{\sphinxupquote{density\_kwargs}} \textendash{} Additional kwargs passed to convert\_hounsfield\_to\_density.

\end{itemize}

\sphinxlineitem{Returns}
\sphinxAtStartPar
A new volume object.

\sphinxlineitem{Return type}
\sphinxAtStartPar
{\hyperref[\detokenize{deepdrr:deepdrr.Volume}]{\sphinxcrossref{Volume}}}

\end{description}\end{quote}

\end{fulllineitems}

\index{from\_nrrd() (deepdrr.Volume class method)@\spxentry{from\_nrrd()}\spxextra{deepdrr.Volume class method}}

\begin{fulllineitems}
\phantomsection\label{\detokenize{deepdrr:deepdrr.Volume.from_nrrd}}
\pysigstartsignatures
\pysiglinewithargsret{\sphinxbfcode{\sphinxupquote{classmethod\DUrole{w,w}{  }}}\sphinxbfcode{\sphinxupquote{from\_nrrd}}}{\sphinxparam{\DUrole{n,n}{path}\DUrole{p,p}{:}\DUrole{w,w}{  }\DUrole{n,n}{str}}\sphinxparamcomma \sphinxparam{\DUrole{n,n}{world\_from\_anatomical}\DUrole{p,p}{:}\DUrole{w,w}{  }\DUrole{n,n}{{\hyperref[\detokenize{deepdrr.geo:deepdrr.geo.core.FrameTransform}]{\sphinxcrossref{FrameTransform}}}\DUrole{w,w}{  }\DUrole{p,p}{|}\DUrole{w,w}{  }None}\DUrole{w,w}{  }\DUrole{o,o}{=}\DUrole{w,w}{  }\DUrole{default_value}{None}}\sphinxparamcomma \sphinxparam{\DUrole{n,n}{use\_thresholding}\DUrole{p,p}{:}\DUrole{w,w}{  }\DUrole{n,n}{bool}\DUrole{w,w}{  }\DUrole{o,o}{=}\DUrole{w,w}{  }\DUrole{default_value}{True}}\sphinxparamcomma \sphinxparam{\DUrole{n,n}{use\_cached}\DUrole{p,p}{:}\DUrole{w,w}{  }\DUrole{n,n}{bool}\DUrole{w,w}{  }\DUrole{o,o}{=}\DUrole{w,w}{  }\DUrole{default_value}{True}}\sphinxparamcomma \sphinxparam{\DUrole{n,n}{cache\_dir}\DUrole{p,p}{:}\DUrole{w,w}{  }\DUrole{n,n}{Path\DUrole{w,w}{  }\DUrole{p,p}{|}\DUrole{w,w}{  }None}\DUrole{w,w}{  }\DUrole{o,o}{=}\DUrole{w,w}{  }\DUrole{default_value}{None}}\sphinxparamcomma \sphinxparam{\DUrole{o,o}{**}\DUrole{n,n}{kwargs}}}{}
\pysigstopsignatures
\sphinxAtStartPar
Load a volume from a nrrd file.
\begin{quote}\begin{description}
\sphinxlineitem{Parameters}\begin{itemize}
\item {} 
\sphinxAtStartPar
\sphinxstyleliteralstrong{\sphinxupquote{path}} (\sphinxstyleliteralemphasis{\sphinxupquote{str}}) \textendash{} path to the file.

\item {} 
\sphinxAtStartPar
\sphinxstyleliteralstrong{\sphinxupquote{use\_thresholding}} (\sphinxstyleliteralemphasis{\sphinxupquote{bool}}\sphinxstyleliteralemphasis{\sphinxupquote{, }}\sphinxstyleliteralemphasis{\sphinxupquote{optional}}) \textendash{} segment the materials using thresholding (faster but less accurate). Defaults to True.

\item {} 
\sphinxAtStartPar
\sphinxstyleliteralstrong{\sphinxupquote{world\_from\_anatomical}} (\sphinxstyleliteralemphasis{\sphinxupquote{Optional}}\sphinxstyleliteralemphasis{\sphinxupquote{{[}}}{\hyperref[\detokenize{deepdrr.geo:deepdrr.geo.FrameTransform}]{\sphinxcrossref{\sphinxstyleliteralemphasis{\sphinxupquote{geo.FrameTransform}}}}}\sphinxstyleliteralemphasis{\sphinxupquote{{]}}}\sphinxstyleliteralemphasis{\sphinxupquote{, }}\sphinxstyleliteralemphasis{\sphinxupquote{optional}}) \textendash{} position the volume in world space. If None, uses identity. Defaults to None.

\item {} 
\sphinxAtStartPar
\sphinxstyleliteralstrong{\sphinxupquote{use\_cached}} (\sphinxstyleliteralemphasis{\sphinxupquote{bool}}\sphinxstyleliteralemphasis{\sphinxupquote{, }}\sphinxstyleliteralemphasis{\sphinxupquote{optional}}) \textendash{} Use a cached segmentation if available. Defaults to True.

\item {} 
\sphinxAtStartPar
\sphinxstyleliteralstrong{\sphinxupquote{cache\_dir}} (\sphinxstyleliteralemphasis{\sphinxupquote{Optional}}\sphinxstyleliteralemphasis{\sphinxupquote{{[}}}\sphinxstyleliteralemphasis{\sphinxupquote{Path}}\sphinxstyleliteralemphasis{\sphinxupquote{{]}}}\sphinxstyleliteralemphasis{\sphinxupquote{, }}\sphinxstyleliteralemphasis{\sphinxupquote{optional}}) \textendash{} Where to load/save the cached segmentation. If None, use the parent dir of \sphinxtitleref{path}. Defaults to None.

\end{itemize}

\sphinxlineitem{Returns}
\sphinxAtStartPar
A volume formed from the NRRD.

\sphinxlineitem{Return type}
\sphinxAtStartPar
{\hyperref[\detokenize{deepdrr:deepdrr.Volume}]{\sphinxcrossref{Volume}}}

\end{description}\end{quote}

\end{fulllineitems}

\index{from\_parameters() (deepdrr.Volume class method)@\spxentry{from\_parameters()}\spxextra{deepdrr.Volume class method}}

\begin{fulllineitems}
\phantomsection\label{\detokenize{deepdrr:deepdrr.Volume.from_parameters}}
\pysigstartsignatures
\pysiglinewithargsret{\sphinxbfcode{\sphinxupquote{classmethod\DUrole{w,w}{  }}}\sphinxbfcode{\sphinxupquote{from\_parameters}}}{\sphinxparam{\DUrole{n,n}{data}\DUrole{p,p}{:}\DUrole{w,w}{  }\DUrole{n,n}{ndarray}}\sphinxparamcomma \sphinxparam{\DUrole{n,n}{materials}\DUrole{p,p}{:}\DUrole{w,w}{  }\DUrole{n,n}{Dict\DUrole{p,p}{{[}}str\DUrole{p,p}{,}\DUrole{w,w}{  }ndarray\DUrole{p,p}{{]}}}}\sphinxparamcomma \sphinxparam{\DUrole{n,n}{origin}\DUrole{p,p}{:}\DUrole{w,w}{  }\DUrole{n,n}{{\hyperref[\detokenize{deepdrr.geo:deepdrr.geo.core.Point3D}]{\sphinxcrossref{Point3D}}}}}\sphinxparamcomma \sphinxparam{\DUrole{n,n}{spacing}\DUrole{p,p}{:}\DUrole{w,w}{  }\DUrole{n,n}{{\hyperref[\detokenize{deepdrr.geo:deepdrr.geo.core.Vector3D}]{\sphinxcrossref{Vector3D}}}\DUrole{w,w}{  }\DUrole{p,p}{|}\DUrole{w,w}{  }None}\DUrole{w,w}{  }\DUrole{o,o}{=}\DUrole{w,w}{  }\DUrole{default_value}{{[}1, 1, 1{]}}}\sphinxparamcomma \sphinxparam{\DUrole{n,n}{anatomical\_coordinate\_system}\DUrole{p,p}{:}\DUrole{w,w}{  }\DUrole{n,n}{str\DUrole{w,w}{  }\DUrole{p,p}{|}\DUrole{w,w}{  }None}\DUrole{w,w}{  }\DUrole{o,o}{=}\DUrole{w,w}{  }\DUrole{default_value}{None}}\sphinxparamcomma \sphinxparam{\DUrole{n,n}{world\_from\_anatomical}\DUrole{p,p}{:}\DUrole{w,w}{  }\DUrole{n,n}{{\hyperref[\detokenize{deepdrr.geo:deepdrr.geo.core.FrameTransform}]{\sphinxcrossref{FrameTransform}}}\DUrole{w,w}{  }\DUrole{p,p}{|}\DUrole{w,w}{  }None}\DUrole{w,w}{  }\DUrole{o,o}{=}\DUrole{w,w}{  }\DUrole{default_value}{None}}\sphinxparamcomma \sphinxparam{\DUrole{o,o}{**}\DUrole{n,n}{kwargs}}}{}
\pysigstopsignatures
\sphinxAtStartPar
Create a volume object with a segmentation of the materials, from parameters.

\sphinxAtStartPar
Note that the anatomical coordinate system is not the world coordinate system (which is cartesian).

\sphinxAtStartPar
Suggested anatomical coordinate space units is millimeters.
A helpful introduction to the geometry is can be found {[}here{]}(\sphinxurl{https://www.slicer.org/wiki/Coordinate\_systems}).
\begin{quote}\begin{description}
\sphinxlineitem{Parameters}\begin{itemize}
\item {} 
\sphinxAtStartPar
\sphinxstyleliteralstrong{\sphinxupquote{volume}} (\sphinxstyleliteralemphasis{\sphinxupquote{np.ndarray}}) \textendash{} the volume density data.

\item {} 
\sphinxAtStartPar
\sphinxstyleliteralstrong{\sphinxupquote{materials}} (\sphinxstyleliteralemphasis{\sphinxupquote{Dict}}\sphinxstyleliteralemphasis{\sphinxupquote{{[}}}\sphinxstyleliteralemphasis{\sphinxupquote{str}}\sphinxstyleliteralemphasis{\sphinxupquote{, }}\sphinxstyleliteralemphasis{\sphinxupquote{np.ndarray}}\sphinxstyleliteralemphasis{\sphinxupquote{{]}}}) \textendash{} mapping from material names to binary segmentation of that material.

\item {} 
\sphinxAtStartPar
\sphinxstyleliteralstrong{\sphinxupquote{origin}} ({\hyperref[\detokenize{deepdrr.geo:deepdrr.geo.core.Point3D}]{\sphinxcrossref{\sphinxstyleliteralemphasis{\sphinxupquote{Point3D}}}}}) \textendash{} Location of the volume’s origin in the anatomical coordinate system.

\item {} 
\sphinxAtStartPar
\sphinxstyleliteralstrong{\sphinxupquote{spacing}} (\sphinxstyleliteralemphasis{\sphinxupquote{Tuple}}\sphinxstyleliteralemphasis{\sphinxupquote{{[}}}\sphinxstyleliteralemphasis{\sphinxupquote{float}}\sphinxstyleliteralemphasis{\sphinxupquote{, }}\sphinxstyleliteralemphasis{\sphinxupquote{float}}\sphinxstyleliteralemphasis{\sphinxupquote{, }}\sphinxstyleliteralemphasis{\sphinxupquote{float}}\sphinxstyleliteralemphasis{\sphinxupquote{{]}}}\sphinxstyleliteralemphasis{\sphinxupquote{, }}\sphinxstyleliteralemphasis{\sphinxupquote{optional}}) \textendash{} Spacing of the volume in the anatomical coordinate system. Defaults to (1, 1, 1).

\item {} 
\sphinxAtStartPar
\sphinxstyleliteralstrong{\sphinxupquote{anatomical\_coordinate\_system}} (\sphinxstyleliteralemphasis{\sphinxupquote{Optional}}\sphinxstyleliteralemphasis{\sphinxupquote{{[}}}\sphinxstyleliteralemphasis{\sphinxupquote{str}}\sphinxstyleliteralemphasis{\sphinxupquote{{]}}}) \textendash{} anatomical coordinate system convention, either “RAS” or “LPS”. Defaults to None.

\item {} 
\sphinxAtStartPar
\sphinxstyleliteralstrong{\sphinxupquote{world\_from\_anatomical}} ({\hyperref[\detokenize{deepdrr.geo:deepdrr.geo.core.FrameTransform}]{\sphinxcrossref{\sphinxstyleliteralemphasis{\sphinxupquote{FrameTransform}}}}}\sphinxstyleliteralemphasis{\sphinxupquote{, }}\sphinxstyleliteralemphasis{\sphinxupquote{optional}}) \textendash{} Optional transformation from anatomical to world coordinates.
If None, then identity is used. Defaults to None.

\end{itemize}

\end{description}\end{quote}

\end{fulllineitems}

\index{get\_bbox\_IJK() (deepdrr.Volume method)@\spxentry{get\_bbox\_IJK()}\spxextra{deepdrr.Volume method}}

\begin{fulllineitems}
\phantomsection\label{\detokenize{deepdrr:deepdrr.Volume.get_bbox_IJK}}
\pysigstartsignatures
\pysiglinewithargsret{\sphinxbfcode{\sphinxupquote{get\_bbox\_IJK}}}{}{{ $\rightarrow$ ndarray\DUrole{w,w}{  }\DUrole{p,p}{|}\DUrole{w,w}{  }None}}
\pysigstopsignatures
\sphinxAtStartPar
Get the bounding box of the materials in IJK.
\begin{quote}\begin{description}
\sphinxlineitem{Returns}
\sphinxAtStartPar
The bounding box as a {[}3, 2{]} array.
None, if the volume is empty.

\sphinxlineitem{Return type}
\sphinxAtStartPar
np.ndarray

\end{description}\end{quote}

\end{fulllineitems}

\index{get\_bounding\_box\_in\_world() (deepdrr.Volume method)@\spxentry{get\_bounding\_box\_in\_world()}\spxextra{deepdrr.Volume method}}

\begin{fulllineitems}
\phantomsection\label{\detokenize{deepdrr:deepdrr.Volume.get_bounding_box_in_world}}
\pysigstartsignatures
\pysiglinewithargsret{\sphinxbfcode{\sphinxupquote{get\_bounding\_box\_in\_world}}}{}{{ $\rightarrow$ Tuple\DUrole{p,p}{{[}}{\hyperref[\detokenize{deepdrr.geo:deepdrr.geo.core.Point3D}]{\sphinxcrossref{Point3D}}}\DUrole{p,p}{,}\DUrole{w,w}{  }{\hyperref[\detokenize{deepdrr.geo:deepdrr.geo.core.Point3D}]{\sphinxcrossref{Point3D}}}\DUrole{p,p}{{]}}}}
\pysigstopsignatures
\sphinxAtStartPar
Get the corners of a bounding box enclosing the volume in world coordinates.

\sphinxAtStartPar
Assumes cell\sphinxhyphen{}centered sampling.
\begin{quote}\begin{description}
\sphinxlineitem{Returns}
\sphinxAtStartPar
The lower corner of the bounding box.
geo.Point3D: The upper corner of the bounding box.

\sphinxlineitem{Return type}
\sphinxAtStartPar
{\hyperref[\detokenize{deepdrr.geo:deepdrr.geo.Point3D}]{\sphinxcrossref{geo.Point3D}}}

\end{description}\end{quote}

\end{fulllineitems}

\index{get\_config() (deepdrr.Volume method)@\spxentry{get\_config()}\spxextra{deepdrr.Volume method}}

\begin{fulllineitems}
\phantomsection\label{\detokenize{deepdrr:deepdrr.Volume.get_config}}
\pysigstartsignatures
\pysiglinewithargsret{\sphinxbfcode{\sphinxupquote{get\_config}}}{}{{ $\rightarrow$ Dict\DUrole{p,p}{{[}}str\DUrole{p,p}{,}\DUrole{w,w}{  }Any\DUrole{p,p}{{]}}}}
\pysigstopsignatures
\sphinxAtStartPar
Get the configuration of the volume. Does not include volumetric data.

\sphinxAtStartPar
Includes any info passed into \sphinxtitleref{config}.
\begin{quote}\begin{description}
\sphinxlineitem{Returns}
\sphinxAtStartPar
The configuration of the volume.

\sphinxlineitem{Return type}
\sphinxAtStartPar
Dict{[}str, Any{]}

\end{description}\end{quote}

\end{fulllineitems}

\index{get\_mesh\_in\_world() (deepdrr.Volume method)@\spxentry{get\_mesh\_in\_world()}\spxextra{deepdrr.Volume method}}

\begin{fulllineitems}
\phantomsection\label{\detokenize{deepdrr:deepdrr.Volume.get_mesh_in_world}}
\pysigstartsignatures
\pysiglinewithargsret{\sphinxbfcode{\sphinxupquote{get\_mesh\_in\_world}}}{\sphinxparam{\DUrole{n,n}{full}\DUrole{p,p}{:}\DUrole{w,w}{  }\DUrole{n,n}{bool}\DUrole{w,w}{  }\DUrole{o,o}{=}\DUrole{w,w}{  }\DUrole{default_value}{False}}\sphinxparamcomma \sphinxparam{\DUrole{n,n}{use\_cached}\DUrole{p,p}{:}\DUrole{w,w}{  }\DUrole{n,n}{bool}\DUrole{w,w}{  }\DUrole{o,o}{=}\DUrole{w,w}{  }\DUrole{default_value}{True}}}{{ $\rightarrow$ PolyData}}
\pysigstopsignatures
\sphinxAtStartPar
Get a pyvista mesh of the outline in world\sphinxhyphen{}space.
\begin{quote}\begin{description}
\sphinxlineitem{Parameters}\begin{itemize}
\item {} 
\sphinxAtStartPar
\sphinxstyleliteralstrong{\sphinxupquote{full}} (\sphinxstyleliteralemphasis{\sphinxupquote{bool}}) \textendash{} Whether to render the full volume or just a wireframe. Defaults to False.

\item {} 
\sphinxAtStartPar
\sphinxstyleliteralstrong{\sphinxupquote{cache\_dir}} (\sphinxstyleliteralemphasis{\sphinxupquote{Optional}}\sphinxstyleliteralemphasis{\sphinxupquote{{[}}}\sphinxstyleliteralemphasis{\sphinxupquote{Path}}\sphinxstyleliteralemphasis{\sphinxupquote{{]}}}\sphinxstyleliteralemphasis{\sphinxupquote{, }}\sphinxstyleliteralemphasis{\sphinxupquote{optional}}) \textendash{} a location to cache the bone surface.

\item {} 
\sphinxAtStartPar
\sphinxstyleliteralstrong{\sphinxupquote{use\_cached}} (\sphinxstyleliteralemphasis{\sphinxupquote{bool}}) \textendash{} If False, don’t use the cached bone surface but re\sphinxhyphen{}create it (expensive). Defaults to True.

\end{itemize}

\sphinxlineitem{Returns}
\sphinxAtStartPar
pyvista mesh.

\sphinxlineitem{Return type}
\sphinxAtStartPar
pv.PolyData

\end{description}\end{quote}

\end{fulllineitems}

\index{get\_surface() (deepdrr.Volume method)@\spxentry{get\_surface()}\spxextra{deepdrr.Volume method}}

\begin{fulllineitems}
\phantomsection\label{\detokenize{deepdrr:deepdrr.Volume.get_surface}}
\pysigstartsignatures
\pysiglinewithargsret{\sphinxbfcode{\sphinxupquote{get\_surface}}}{\sphinxparam{\DUrole{n,n}{material}\DUrole{p,p}{:}\DUrole{w,w}{  }\DUrole{n,n}{str}\DUrole{w,w}{  }\DUrole{o,o}{=}\DUrole{w,w}{  }\DUrole{default_value}{\textquotesingle{}bone\textquotesingle{}}}\sphinxparamcomma \sphinxparam{\DUrole{n,n}{use\_cached}\DUrole{p,p}{:}\DUrole{w,w}{  }\DUrole{n,n}{bool}\DUrole{w,w}{  }\DUrole{o,o}{=}\DUrole{w,w}{  }\DUrole{default_value}{True}}}{}
\pysigstopsignatures
\end{fulllineitems}

\index{ijk\_from\_anatomical (deepdrr.Volume property)@\spxentry{ijk\_from\_anatomical}\spxextra{deepdrr.Volume property}}

\begin{fulllineitems}
\phantomsection\label{\detokenize{deepdrr:deepdrr.Volume.ijk_from_anatomical}}
\pysigstartsignatures
\pysigline{\sphinxbfcode{\sphinxupquote{property\DUrole{w,w}{  }}}\sphinxbfcode{\sphinxupquote{ijk\_from\_anatomical}}}
\pysigstopsignatures
\end{fulllineitems}

\index{ijk\_from\_world (deepdrr.Volume property)@\spxentry{ijk\_from\_world}\spxextra{deepdrr.Volume property}}

\begin{fulllineitems}
\phantomsection\label{\detokenize{deepdrr:deepdrr.Volume.ijk_from_world}}
\pysigstartsignatures
\pysigline{\sphinxbfcode{\sphinxupquote{property\DUrole{w,w}{  }}}\sphinxbfcode{\sphinxupquote{ijk\_from\_world}}\sphinxbfcode{\sphinxupquote{\DUrole{p,p}{:}\DUrole{w,w}{  }{\hyperref[\detokenize{deepdrr.geo:deepdrr.geo.core.FrameTransform}]{\sphinxcrossref{FrameTransform}}}}}}
\pysigstopsignatures
\end{fulllineitems}

\index{interpolate() (deepdrr.Volume method)@\spxentry{interpolate()}\spxextra{deepdrr.Volume method}}

\begin{fulllineitems}
\phantomsection\label{\detokenize{deepdrr:deepdrr.Volume.interpolate}}
\pysigstartsignatures
\pysiglinewithargsret{\sphinxbfcode{\sphinxupquote{interpolate}}}{\sphinxparam{\DUrole{o,o}{*}\DUrole{n,n}{x}\DUrole{p,p}{:}\DUrole{w,w}{  }\DUrole{n,n}{{\hyperref[\detokenize{deepdrr.geo:deepdrr.geo.core.Point3D}]{\sphinxcrossref{Point3D}}}}}\sphinxparamcomma \sphinxparam{\DUrole{n,n}{method}\DUrole{p,p}{:}\DUrole{w,w}{  }\DUrole{n,n}{str}\DUrole{w,w}{  }\DUrole{o,o}{=}\DUrole{w,w}{  }\DUrole{default_value}{\textquotesingle{}linear\textquotesingle{}}}}{{ $\rightarrow$ ndarray}}
\pysigstopsignatures
\sphinxAtStartPar
Interpolate the value of the volume at the point.

\sphinxAtStartPar
This is a \sphinxstyleemphasis{slow} version of interpolation, using scipy under the hood. DeepDRR uses cubic
spline interpolation on the GPU for rendering. This function is provided as a convenience.
\begin{quote}\begin{description}
\sphinxlineitem{Parameters}\begin{itemize}
\item {} 
\sphinxAtStartPar
\sphinxstyleliteralstrong{\sphinxupquote{x}} ({\hyperref[\detokenize{deepdrr.geo:deepdrr.geo.Point3D}]{\sphinxcrossref{\sphinxstyleliteralemphasis{\sphinxupquote{geo.Point3D}}}}}) \textendash{} The point or points in world\sphinxhyphen{}space.

\item {} 
\sphinxAtStartPar
\sphinxstyleliteralstrong{\sphinxupquote{method}} (\sphinxstyleliteralemphasis{\sphinxupquote{str}}) \textendash{} The interpolation method to be used.
Accepted values are “linear” and “nearest”.
Defaults to “linear.”

\end{itemize}

\sphinxlineitem{Returns}
\sphinxAtStartPar
\begin{description}
\sphinxlineitem{The interpolated value(s) of the point(s)}
\sphinxAtStartPar
in the Volume. If a point is outside the volume, the value is NaN.

\end{description}


\sphinxlineitem{Return type}
\sphinxAtStartPar
Union{[}float, np.ndarray{]}

\end{description}\end{quote}

\end{fulllineitems}

\index{isosurface() (deepdrr.Volume method)@\spxentry{isosurface()}\spxextra{deepdrr.Volume method}}

\begin{fulllineitems}
\phantomsection\label{\detokenize{deepdrr:deepdrr.Volume.isosurface}}
\pysigstartsignatures
\pysiglinewithargsret{\sphinxbfcode{\sphinxupquote{isosurface}}}{\sphinxparam{\DUrole{n,n}{value}\DUrole{p,p}{:}\DUrole{w,w}{  }\DUrole{n,n}{float}\DUrole{w,w}{  }\DUrole{o,o}{=}\DUrole{w,w}{  }\DUrole{default_value}{0.5}}\sphinxparamcomma \sphinxparam{\DUrole{n,n}{label}\DUrole{p,p}{:}\DUrole{w,w}{  }\DUrole{n,n}{int\DUrole{w,w}{  }\DUrole{p,p}{|}\DUrole{w,w}{  }None}\DUrole{w,w}{  }\DUrole{o,o}{=}\DUrole{w,w}{  }\DUrole{default_value}{None}}\sphinxparamcomma \sphinxparam{\DUrole{n,n}{node\_centered}\DUrole{p,p}{:}\DUrole{w,w}{  }\DUrole{n,n}{bool}\DUrole{w,w}{  }\DUrole{o,o}{=}\DUrole{w,w}{  }\DUrole{default_value}{True}}\sphinxparamcomma \sphinxparam{\DUrole{n,n}{smooth}\DUrole{p,p}{:}\DUrole{w,w}{  }\DUrole{n,n}{bool}\DUrole{w,w}{  }\DUrole{o,o}{=}\DUrole{w,w}{  }\DUrole{default_value}{True}}\sphinxparamcomma \sphinxparam{\DUrole{n,n}{decimation}\DUrole{p,p}{:}\DUrole{w,w}{  }\DUrole{n,n}{float}\DUrole{w,w}{  }\DUrole{o,o}{=}\DUrole{w,w}{  }\DUrole{default_value}{0.01}}\sphinxparamcomma \sphinxparam{\DUrole{n,n}{decimation\_points}\DUrole{p,p}{:}\DUrole{w,w}{  }\DUrole{n,n}{int\DUrole{w,w}{  }\DUrole{p,p}{|}\DUrole{w,w}{  }None}\DUrole{w,w}{  }\DUrole{o,o}{=}\DUrole{w,w}{  }\DUrole{default_value}{None}}\sphinxparamcomma \sphinxparam{\DUrole{n,n}{smooth\_iter}\DUrole{p,p}{:}\DUrole{w,w}{  }\DUrole{n,n}{int}\DUrole{w,w}{  }\DUrole{o,o}{=}\DUrole{w,w}{  }\DUrole{default_value}{200}}\sphinxparamcomma \sphinxparam{\DUrole{n,n}{relaxation\_factor}\DUrole{p,p}{:}\DUrole{w,w}{  }\DUrole{n,n}{float}\DUrole{w,w}{  }\DUrole{o,o}{=}\DUrole{w,w}{  }\DUrole{default_value}{0.25}}\sphinxparamcomma \sphinxparam{\DUrole{n,n}{convert\_to\_LPS}\DUrole{p,p}{:}\DUrole{w,w}{  }\DUrole{n,n}{bool}\DUrole{w,w}{  }\DUrole{o,o}{=}\DUrole{w,w}{  }\DUrole{default_value}{False}}}{{ $\rightarrow$ PolyData}}
\pysigstopsignatures
\sphinxAtStartPar
Make an isosurface from the volume’s data, transforming to anatomical\_coordinates.

\sphinxAtStartPar
Accepts arguments passed to {\hyperref[\detokenize{deepdrr.utils:deepdrr.utils.mesh_utils.isosurface}]{\sphinxcrossref{\sphinxcode{\sphinxupquote{deepdrr.utils.mesh\_utils.isosurface()}}}}}.
\begin{quote}\begin{description}
\sphinxlineitem{Parameters}\begin{itemize}
\item {} 
\sphinxAtStartPar
\sphinxstyleliteralstrong{\sphinxupquote{value}} (\sphinxstyleliteralemphasis{\sphinxupquote{float}}) \textendash{} The value at which to make the isosurface.

\item {} 
\sphinxAtStartPar
\sphinxstyleliteralstrong{\sphinxupquote{label}} (\sphinxstyleliteralemphasis{\sphinxupquote{int}}) \textendash{} The label of the isosurface.

\item {} 
\sphinxAtStartPar
\sphinxstyleliteralstrong{\sphinxupquote{node\_centered}} (\sphinxstyleliteralemphasis{\sphinxupquote{bool}}) \textendash{} If True, the isosurface is centered at the node.
If False, the isosurface is centered at the cell.

\item {} 
\sphinxAtStartPar
\sphinxstyleliteralstrong{\sphinxupquote{smooth}} (\sphinxstyleliteralemphasis{\sphinxupquote{bool}}) \textendash{} If True, the isosurface is smoothed.

\item {} 
\sphinxAtStartPar
\sphinxstyleliteralstrong{\sphinxupquote{decimation}} (\sphinxstyleliteralemphasis{\sphinxupquote{float}}) \textendash{} The decimation factor (how many points to remove).

\item {} 
\sphinxAtStartPar
\sphinxstyleliteralstrong{\sphinxupquote{smooth\_iter}} (\sphinxstyleliteralemphasis{\sphinxupquote{int}}) \textendash{} The number of smoothing iterations.

\item {} 
\sphinxAtStartPar
\sphinxstyleliteralstrong{\sphinxupquote{relaxation\_factor}} (\sphinxstyleliteralemphasis{\sphinxupquote{float}}) \textendash{} The relaxation factor.

\item {} 
\sphinxAtStartPar
\sphinxstyleliteralstrong{\sphinxupquote{convert\_to\_LPS}} (\sphinxstyleliteralemphasis{\sphinxupquote{bool}}) \textendash{} If True, the isosurface is converted to LPS coordinates. (Recommended)

\end{itemize}

\sphinxlineitem{Returns}
\sphinxAtStartPar
The surface mesh in anatomical coordinates.

\sphinxlineitem{Return type}
\sphinxAtStartPar
pv.PolyData

\end{description}\end{quote}

\end{fulllineitems}

\index{load() (deepdrr.Volume class method)@\spxentry{load()}\spxextra{deepdrr.Volume class method}}

\begin{fulllineitems}
\phantomsection\label{\detokenize{deepdrr:deepdrr.Volume.load}}
\pysigstartsignatures
\pysiglinewithargsret{\sphinxbfcode{\sphinxupquote{classmethod\DUrole{w,w}{  }}}\sphinxbfcode{\sphinxupquote{load}}}{\sphinxparam{\DUrole{n,n}{path}\DUrole{p,p}{:}\DUrole{w,w}{  }\DUrole{n,n}{Path}}\sphinxparamcomma \sphinxparam{\DUrole{n,n}{segmentation}\DUrole{p,p}{:}\DUrole{w,w}{  }\DUrole{n,n}{bool}\DUrole{w,w}{  }\DUrole{o,o}{=}\DUrole{w,w}{  }\DUrole{default_value}{False}}}{{ $\rightarrow$ {\hyperref[\detokenize{deepdrr.vol:deepdrr.vol.volume.Volume}]{\sphinxcrossref{Volume}}}}}
\pysigstopsignatures
\sphinxAtStartPar
Load a volume from disk.
\begin{quote}\begin{description}
\sphinxlineitem{Parameters}\begin{itemize}
\item {} 
\sphinxAtStartPar
\sphinxstyleliteralstrong{\sphinxupquote{path}} (\sphinxstyleliteralemphasis{\sphinxupquote{Path}}) \textendash{} a directory containing the volume and segmentations.

\item {} 
\sphinxAtStartPar
\sphinxstyleliteralstrong{\sphinxupquote{segmentation}} (\sphinxstyleliteralemphasis{\sphinxupquote{bool}}\sphinxstyleliteralemphasis{\sphinxupquote{, }}\sphinxstyleliteralemphasis{\sphinxupquote{optional}}) \textendash{} if the volume is a segmentation,
populate the materials from that.

\end{itemize}

\end{description}\end{quote}

\sphinxAtStartPar
Returns: Volume.

\end{fulllineitems}

\index{materials (deepdrr.Volume attribute)@\spxentry{materials}\spxextra{deepdrr.Volume attribute}}

\begin{fulllineitems}
\phantomsection\label{\detokenize{deepdrr:deepdrr.Volume.materials}}
\pysigstartsignatures
\pysigline{\sphinxbfcode{\sphinxupquote{materials}}\sphinxbfcode{\sphinxupquote{\DUrole{p,p}{:}\DUrole{w,w}{  }Dict\DUrole{p,p}{{[}}str\DUrole{p,p}{,}\DUrole{w,w}{  }ndarray\DUrole{p,p}{{]}}}}}
\pysigstopsignatures
\end{fulllineitems}

\index{orient\_patient() (deepdrr.Volume method)@\spxentry{orient\_patient()}\spxextra{deepdrr.Volume method}}

\begin{fulllineitems}
\phantomsection\label{\detokenize{deepdrr:deepdrr.Volume.orient_patient}}
\pysigstartsignatures
\pysiglinewithargsret{\sphinxbfcode{\sphinxupquote{orient\_patient}}}{\sphinxparam{\DUrole{n,n}{head\_first}\DUrole{p,p}{:}\DUrole{w,w}{  }\DUrole{n,n}{bool}\DUrole{w,w}{  }\DUrole{o,o}{=}\DUrole{w,w}{  }\DUrole{default_value}{True}}\sphinxparamcomma \sphinxparam{\DUrole{n,n}{supine}\DUrole{p,p}{:}\DUrole{w,w}{  }\DUrole{n,n}{bool}\DUrole{w,w}{  }\DUrole{o,o}{=}\DUrole{w,w}{  }\DUrole{default_value}{True}}\sphinxparamcomma \sphinxparam{\DUrole{n,n}{world\_from\_device}\DUrole{p,p}{:}\DUrole{w,w}{  }\DUrole{n,n}{{\hyperref[\detokenize{deepdrr.geo:deepdrr.geo.core.FrameTransform}]{\sphinxcrossref{FrameTransform}}}\DUrole{w,w}{  }\DUrole{p,p}{|}\DUrole{w,w}{  }None}\DUrole{w,w}{  }\DUrole{o,o}{=}\DUrole{w,w}{  }\DUrole{default_value}{None}}}{{ $\rightarrow$ None}}
\pysigstopsignatures
\sphinxAtStartPar
Orient the patient with the given orientation, aligning with the Loop\sphinxhyphen{}X coordinates.
\begin{quote}\begin{description}
\sphinxlineitem{Parameters}\begin{itemize}
\item {} 
\sphinxAtStartPar
\sphinxstyleliteralstrong{\sphinxupquote{head\_first}} \textendash{} If True, the patient is oriented with head (superior axis) pointing in the \sphinxhyphen{}Y direction. Defaults to True.

\item {} 
\sphinxAtStartPar
\sphinxstyleliteralstrong{\sphinxupquote{supine}} \textendash{} If True, the patient is oriented so that the anterior axis (stomach) points toward +Z. Defaults to True.

\end{itemize}

\end{description}\end{quote}

\end{fulllineitems}

\index{origin (deepdrr.Volume property)@\spxentry{origin}\spxextra{deepdrr.Volume property}}

\begin{fulllineitems}
\phantomsection\label{\detokenize{deepdrr:deepdrr.Volume.origin}}
\pysigstartsignatures
\pysigline{\sphinxbfcode{\sphinxupquote{property\DUrole{w,w}{  }}}\sphinxbfcode{\sphinxupquote{origin}}\sphinxbfcode{\sphinxupquote{\DUrole{p,p}{:}\DUrole{w,w}{  }{\hyperref[\detokenize{deepdrr.geo:deepdrr.geo.core.Point3D}]{\sphinxcrossref{Point3D}}}}}}
\pysigstopsignatures
\sphinxAtStartPar
The origin of the volume in anatomical space.

\end{fulllineitems}

\index{origin\_in\_anatomical (deepdrr.Volume property)@\spxentry{origin\_in\_anatomical}\spxextra{deepdrr.Volume property}}

\begin{fulllineitems}
\phantomsection\label{\detokenize{deepdrr:deepdrr.Volume.origin_in_anatomical}}
\pysigstartsignatures
\pysigline{\sphinxbfcode{\sphinxupquote{property\DUrole{w,w}{  }}}\sphinxbfcode{\sphinxupquote{origin\_in\_anatomical}}\sphinxbfcode{\sphinxupquote{\DUrole{p,p}{:}\DUrole{w,w}{  }{\hyperref[\detokenize{deepdrr.geo:deepdrr.geo.core.Point3D}]{\sphinxcrossref{Point3D}}}}}}
\pysigstopsignatures
\sphinxAtStartPar
The origin of the volume in anatomical space.

\end{fulllineitems}

\index{origin\_in\_world (deepdrr.Volume property)@\spxentry{origin\_in\_world}\spxextra{deepdrr.Volume property}}

\begin{fulllineitems}
\phantomsection\label{\detokenize{deepdrr:deepdrr.Volume.origin_in_world}}
\pysigstartsignatures
\pysigline{\sphinxbfcode{\sphinxupquote{property\DUrole{w,w}{  }}}\sphinxbfcode{\sphinxupquote{origin\_in\_world}}\sphinxbfcode{\sphinxupquote{\DUrole{p,p}{:}\DUrole{w,w}{  }{\hyperref[\detokenize{deepdrr.geo:deepdrr.geo.core.Point3D}]{\sphinxcrossref{Point3D}}}}}}
\pysigstopsignatures
\sphinxAtStartPar
The origin of the volume in world space.

\end{fulllineitems}

\index{place() (deepdrr.Volume method)@\spxentry{place()}\spxextra{deepdrr.Volume method}}

\begin{fulllineitems}
\phantomsection\label{\detokenize{deepdrr:deepdrr.Volume.place}}
\pysigstartsignatures
\pysiglinewithargsret{\sphinxbfcode{\sphinxupquote{place}}}{\sphinxparam{\DUrole{n,n}{point\_in\_anatomical}\DUrole{p,p}{:}\DUrole{w,w}{  }\DUrole{n,n}{{\hyperref[\detokenize{deepdrr.geo:deepdrr.geo.core.Point3D}]{\sphinxcrossref{Point3D}}}}}\sphinxparamcomma \sphinxparam{\DUrole{n,n}{desired\_point\_in\_world}\DUrole{p,p}{:}\DUrole{w,w}{  }\DUrole{n,n}{{\hyperref[\detokenize{deepdrr.geo:deepdrr.geo.core.Point3D}]{\sphinxcrossref{Point3D}}}}}}{{ $\rightarrow$ None}}
\pysigstopsignatures
\sphinxAtStartPar
Translate the volume so that x\_in\_anatomical corresponds to x\_in\_world.

\end{fulllineitems}

\index{place\_center() (deepdrr.Volume method)@\spxentry{place\_center()}\spxextra{deepdrr.Volume method}}

\begin{fulllineitems}
\phantomsection\label{\detokenize{deepdrr:deepdrr.Volume.place_center}}
\pysigstartsignatures
\pysiglinewithargsret{\sphinxbfcode{\sphinxupquote{place\_center}}}{\sphinxparam{\DUrole{n,n}{x}\DUrole{p,p}{:}\DUrole{w,w}{  }\DUrole{n,n}{{\hyperref[\detokenize{deepdrr.geo:deepdrr.geo.core.Point3D}]{\sphinxcrossref{Point3D}}}}}}{{ $\rightarrow$ None}}
\pysigstopsignatures
\sphinxAtStartPar
Translate the volume so that its center is located at world\sphinxhyphen{}space point x.

\sphinxAtStartPar
Only changes the translation elements of the world\_from\_anatomical transform. Preserves the current rotation of the
\begin{quote}\begin{description}
\sphinxlineitem{Parameters}
\sphinxAtStartPar
\sphinxstyleliteralstrong{\sphinxupquote{x}} ({\hyperref[\detokenize{deepdrr.geo:deepdrr.geo.Point3D}]{\sphinxcrossref{\sphinxstyleliteralemphasis{\sphinxupquote{geo.Point3D}}}}}) \textendash{} the world\sphinxhyphen{}space point.

\end{description}\end{quote}

\end{fulllineitems}

\index{prone() (deepdrr.Volume method)@\spxentry{prone()}\spxextra{deepdrr.Volume method}}

\begin{fulllineitems}
\phantomsection\label{\detokenize{deepdrr:deepdrr.Volume.prone}}
\pysigstartsignatures
\pysiglinewithargsret{\sphinxbfcode{\sphinxupquote{prone}}}{}{}
\pysigstopsignatures
\sphinxAtStartPar
Turns the volume to be face down.

\sphinxAtStartPar
This aligns the patient so that, in world space,
the posterior side is toward +Z, inferior is toward +X,
and right is toward +Y.
\begin{quote}\begin{description}
\sphinxlineitem{Raises}
\sphinxAtStartPar
\sphinxstyleliteralstrong{\sphinxupquote{NotImplementedError}} \textendash{} If the anatomical coordinate system is not “RAS”.

\end{description}\end{quote}

\end{fulllineitems}

\index{rotate() (deepdrr.Volume method)@\spxentry{rotate()}\spxextra{deepdrr.Volume method}}

\begin{fulllineitems}
\phantomsection\label{\detokenize{deepdrr:deepdrr.Volume.rotate}}
\pysigstartsignatures
\pysiglinewithargsret{\sphinxbfcode{\sphinxupquote{rotate}}}{\sphinxparam{\DUrole{n,n}{rotation}\DUrole{p,p}{:}\DUrole{w,w}{  }\DUrole{n,n}{{\hyperref[\detokenize{deepdrr.geo:deepdrr.geo.core.Vector3D}]{\sphinxcrossref{Vector3D}}}\DUrole{w,w}{  }\DUrole{p,p}{|}\DUrole{w,w}{  }{\hyperref[\detokenize{deepdrr.geo:deepdrr.geo.Rotation}]{\sphinxcrossref{Rotation}}}}}\sphinxparamcomma \sphinxparam{\DUrole{n,n}{center}\DUrole{p,p}{:}\DUrole{w,w}{  }\DUrole{n,n}{{\hyperref[\detokenize{deepdrr.geo:deepdrr.geo.core.Point3D}]{\sphinxcrossref{Point3D}}}\DUrole{w,w}{  }\DUrole{p,p}{|}\DUrole{w,w}{  }None}\DUrole{w,w}{  }\DUrole{o,o}{=}\DUrole{w,w}{  }\DUrole{default_value}{None}}}{{ $\rightarrow$ {\hyperref[\detokenize{deepdrr.vol:deepdrr.vol.volume.Volume}]{\sphinxcrossref{Volume}}}}}
\pysigstopsignatures
\sphinxAtStartPar
Rotate the volume by \sphinxtitleref{rotation} about \sphinxtitleref{center}.
\begin{quote}\begin{description}
\sphinxlineitem{Parameters}\begin{itemize}
\item {} 
\sphinxAtStartPar
\sphinxstyleliteralstrong{\sphinxupquote{rotation}} (\sphinxstyleliteralemphasis{\sphinxupquote{Union}}\sphinxstyleliteralemphasis{\sphinxupquote{{[}}}{\hyperref[\detokenize{deepdrr.geo:deepdrr.geo.Vector3D}]{\sphinxcrossref{\sphinxstyleliteralemphasis{\sphinxupquote{geo.Vector3D}}}}}\sphinxstyleliteralemphasis{\sphinxupquote{, }}{\hyperref[\detokenize{deepdrr.geo:deepdrr.geo.Rotation}]{\sphinxcrossref{\sphinxstyleliteralemphasis{\sphinxupquote{Rotation}}}}}\sphinxstyleliteralemphasis{\sphinxupquote{{]}}}) \textendash{} the rotation in world\sphinxhyphen{}space. If it is a vector, \sphinxtitleref{Rotation.from\_rotvec(rotation)} is used.

\item {} 
\sphinxAtStartPar
\sphinxstyleliteralstrong{\sphinxupquote{center}} ({\hyperref[\detokenize{deepdrr.geo:deepdrr.geo.Point3D}]{\sphinxcrossref{\sphinxstyleliteralemphasis{\sphinxupquote{geo.Point3D}}}}}\sphinxstyleliteralemphasis{\sphinxupquote{, }}\sphinxstyleliteralemphasis{\sphinxupquote{optional}}) \textendash{} the center of rotation in world space coordinates. If None, the center of the volume is used.

\end{itemize}

\end{description}\end{quote}

\end{fulllineitems}

\index{save() (deepdrr.Volume method)@\spxentry{save()}\spxextra{deepdrr.Volume method}}

\begin{fulllineitems}
\phantomsection\label{\detokenize{deepdrr:deepdrr.Volume.save}}
\pysigstartsignatures
\pysiglinewithargsret{\sphinxbfcode{\sphinxupquote{save}}}{\sphinxparam{\DUrole{n,n}{output\_dir}\DUrole{p,p}{:}\DUrole{w,w}{  }\DUrole{n,n}{Path}}\sphinxparamcomma \sphinxparam{\DUrole{n,n}{segmentation}\DUrole{p,p}{:}\DUrole{w,w}{  }\DUrole{n,n}{bool}\DUrole{w,w}{  }\DUrole{o,o}{=}\DUrole{w,w}{  }\DUrole{default_value}{False}}}{}
\pysigstopsignatures
\sphinxAtStartPar
Save the volume to disk as a nifti file.
\begin{quote}\begin{description}
\sphinxlineitem{Parameters}\begin{itemize}
\item {} 
\sphinxAtStartPar
\sphinxstyleliteralstrong{\sphinxupquote{output\_dir}} (\sphinxstyleliteralemphasis{\sphinxupquote{Path}}) \textendash{} a directory to save the volume and segmentations to.

\item {} 
\sphinxAtStartPar
\sphinxstyleliteralstrong{\sphinxupquote{segmentation}} (\sphinxstyleliteralemphasis{\sphinxupquote{bool}}\sphinxstyleliteralemphasis{\sphinxupquote{, }}\sphinxstyleliteralemphasis{\sphinxupquote{optional}}) \textendash{} if the volume is a segmentation, there’s
no need to save the materials.

\end{itemize}

\end{description}\end{quote}

\end{fulllineitems}

\index{segment\_materials() (deepdrr.Volume class method)@\spxentry{segment\_materials()}\spxextra{deepdrr.Volume class method}}

\begin{fulllineitems}
\phantomsection\label{\detokenize{deepdrr:deepdrr.Volume.segment_materials}}
\pysigstartsignatures
\pysiglinewithargsret{\sphinxbfcode{\sphinxupquote{classmethod\DUrole{w,w}{  }}}\sphinxbfcode{\sphinxupquote{segment\_materials}}}{\sphinxparam{\DUrole{n,n}{hu\_values}\DUrole{p,p}{:}\DUrole{w,w}{  }\DUrole{n,n}{ndarray}}\sphinxparamcomma \sphinxparam{\DUrole{n,n}{anatomical\_from\_ijk}\DUrole{p,p}{:}\DUrole{w,w}{  }\DUrole{n,n}{{\hyperref[\detokenize{deepdrr.geo:deepdrr.geo.core.FrameTransform}]{\sphinxcrossref{FrameTransform}}}}}\sphinxparamcomma \sphinxparam{\DUrole{n,n}{use\_thresholding}\DUrole{p,p}{:}\DUrole{w,w}{  }\DUrole{n,n}{bool}\DUrole{w,w}{  }\DUrole{o,o}{=}\DUrole{w,w}{  }\DUrole{default_value}{True}}\sphinxparamcomma \sphinxparam{\DUrole{n,n}{use\_cached}\DUrole{p,p}{:}\DUrole{w,w}{  }\DUrole{n,n}{bool}\DUrole{w,w}{  }\DUrole{o,o}{=}\DUrole{w,w}{  }\DUrole{default_value}{True}}\sphinxparamcomma \sphinxparam{\DUrole{n,n}{save\_cache}\DUrole{p,p}{:}\DUrole{w,w}{  }\DUrole{n,n}{bool}\DUrole{w,w}{  }\DUrole{o,o}{=}\DUrole{w,w}{  }\DUrole{default_value}{False}}\sphinxparamcomma \sphinxparam{\DUrole{n,n}{cache\_dir}\DUrole{p,p}{:}\DUrole{w,w}{  }\DUrole{n,n}{Path\DUrole{w,w}{  }\DUrole{p,p}{|}\DUrole{w,w}{  }None}\DUrole{w,w}{  }\DUrole{o,o}{=}\DUrole{w,w}{  }\DUrole{default_value}{None}}\sphinxparamcomma \sphinxparam{\DUrole{n,n}{cache\_name}\DUrole{p,p}{:}\DUrole{w,w}{  }\DUrole{n,n}{str\DUrole{w,w}{  }\DUrole{p,p}{|}\DUrole{w,w}{  }None}\DUrole{w,w}{  }\DUrole{o,o}{=}\DUrole{w,w}{  }\DUrole{default_value}{None}}}{{ $\rightarrow$ Dict\DUrole{p,p}{{[}}str\DUrole{p,p}{,}\DUrole{w,w}{  }ndarray\DUrole{p,p}{{]}}}}
\pysigstopsignatures
\sphinxAtStartPar
Segment the materials in a volume, potentially caching.

\sphinxAtStartPar
If cache\_dir is None, then
\begin{quote}\begin{description}
\sphinxlineitem{Parameters}\begin{itemize}
\item {} 
\sphinxAtStartPar
\sphinxstyleliteralstrong{\sphinxupquote{hu\_values}} (\sphinxstyleliteralemphasis{\sphinxupquote{np.ndarray}}) \textendash{} volume data in Hounsfield Units.

\item {} 
\sphinxAtStartPar
\sphinxstyleliteralstrong{\sphinxupquote{use\_thretholding}} (\sphinxstyleliteralemphasis{\sphinxupquote{bool}}\sphinxstyleliteralemphasis{\sphinxupquote{, }}\sphinxstyleliteralemphasis{\sphinxupquote{optional}}) \textendash{} whether to segment with thresholding (true) or a DNN. Defaults to True.

\item {} 
\sphinxAtStartPar
\sphinxstyleliteralstrong{\sphinxupquote{use\_cached}} (\sphinxstyleliteralemphasis{\sphinxupquote{bool}}\sphinxstyleliteralemphasis{\sphinxupquote{, }}\sphinxstyleliteralemphasis{\sphinxupquote{optional}}) \textendash{} use the cached segmentation, if it exists. Defaults to True.

\item {} 
\sphinxAtStartPar
\sphinxstyleliteralstrong{\sphinxupquote{save\_cache}} (\sphinxstyleliteralemphasis{\sphinxupquote{bool}}\sphinxstyleliteralemphasis{\sphinxupquote{, }}\sphinxstyleliteralemphasis{\sphinxupquote{optional}}) \textendash{} save the segmentation to cache\_dir. Defaults to True.

\item {} 
\sphinxAtStartPar
\sphinxstyleliteralstrong{\sphinxupquote{cache\_dir}} (\sphinxstyleliteralemphasis{\sphinxupquote{Optional}}\sphinxstyleliteralemphasis{\sphinxupquote{{[}}}\sphinxstyleliteralemphasis{\sphinxupquote{Path}}\sphinxstyleliteralemphasis{\sphinxupquote{{]}}}\sphinxstyleliteralemphasis{\sphinxupquote{, }}\sphinxstyleliteralemphasis{\sphinxupquote{optional}}) \textendash{} where to look for the segmentation cache. If None, no caching performed. Defaults to None.

\item {} 
\sphinxAtStartPar
\sphinxstyleliteralstrong{\sphinxupquote{cache\_name}} (\sphinxstyleliteralemphasis{\sphinxupquote{str}}\sphinxstyleliteralemphasis{\sphinxupquote{, }}\sphinxstyleliteralemphasis{\sphinxupquote{optional}}) \textendash{} Name of cache file. Must be provided if use\_cached or cache\_dir is True. Defaults to None.

\end{itemize}

\sphinxlineitem{Returns}
\sphinxAtStartPar
materials segmentation.

\sphinxlineitem{Return type}
\sphinxAtStartPar
Dict{[}str, np.ndarray{]}

\end{description}\end{quote}

\end{fulllineitems}

\index{shape (deepdrr.Volume property)@\spxentry{shape}\spxextra{deepdrr.Volume property}}

\begin{fulllineitems}
\phantomsection\label{\detokenize{deepdrr:deepdrr.Volume.shape}}
\pysigstartsignatures
\pysigline{\sphinxbfcode{\sphinxupquote{property\DUrole{w,w}{  }}}\sphinxbfcode{\sphinxupquote{shape}}\sphinxbfcode{\sphinxupquote{\DUrole{p,p}{:}\DUrole{w,w}{  }Tuple\DUrole{p,p}{{[}}int\DUrole{p,p}{,}\DUrole{w,w}{  }int\DUrole{p,p}{,}\DUrole{w,w}{  }int\DUrole{p,p}{{]}}}}}
\pysigstopsignatures
\end{fulllineitems}

\index{shrink() (deepdrr.Volume method)@\spxentry{shrink()}\spxextra{deepdrr.Volume method}}

\begin{fulllineitems}
\phantomsection\label{\detokenize{deepdrr:deepdrr.Volume.shrink}}
\pysigstartsignatures
\pysiglinewithargsret{\sphinxbfcode{\sphinxupquote{shrink}}}{}{{ $\rightarrow$ {\hyperref[\detokenize{deepdrr.vol:deepdrr.vol.volume.Volume}]{\sphinxcrossref{Volume}}}}}
\pysigstopsignatures
\sphinxAtStartPar
Crop the volume to remove empty space.
\begin{quote}\begin{description}
\sphinxlineitem{Returns}
\sphinxAtStartPar
The cropped volume.

\sphinxlineitem{Return type}
\sphinxAtStartPar
{\hyperref[\detokenize{deepdrr:deepdrr.Volume}]{\sphinxcrossref{Volume}}}

\end{description}\end{quote}

\end{fulllineitems}

\index{spacing (deepdrr.Volume property)@\spxentry{spacing}\spxextra{deepdrr.Volume property}}

\begin{fulllineitems}
\phantomsection\label{\detokenize{deepdrr:deepdrr.Volume.spacing}}
\pysigstartsignatures
\pysigline{\sphinxbfcode{\sphinxupquote{property\DUrole{w,w}{  }}}\sphinxbfcode{\sphinxupquote{spacing}}\sphinxbfcode{\sphinxupquote{\DUrole{p,p}{:}\DUrole{w,w}{  }{\hyperref[\detokenize{deepdrr.geo:deepdrr.geo.core.Vector3D}]{\sphinxcrossref{Vector3D}}}}}}
\pysigstopsignatures
\sphinxAtStartPar
The spacing of the voxels.

\end{fulllineitems}

\index{supine() (deepdrr.Volume method)@\spxentry{supine()}\spxextra{deepdrr.Volume method}}

\begin{fulllineitems}
\phantomsection\label{\detokenize{deepdrr:deepdrr.Volume.supine}}
\pysigstartsignatures
\pysiglinewithargsret{\sphinxbfcode{\sphinxupquote{supine}}}{}{}
\pysigstopsignatures
\sphinxAtStartPar
Turns the volume to be face up.

\sphinxAtStartPar
This aligns the patient so that, in world space,
the anterior side is toward +Z, inferior is toward +X,
and left is toward +Y.
\begin{quote}\begin{description}
\sphinxlineitem{Raises}
\sphinxAtStartPar
\sphinxstyleliteralstrong{\sphinxupquote{NotImplementedError}} \textendash{} If the anatomical coordinate system is not “RAS”.

\end{description}\end{quote}

\end{fulllineitems}

\index{translate() (deepdrr.Volume method)@\spxentry{translate()}\spxextra{deepdrr.Volume method}}

\begin{fulllineitems}
\phantomsection\label{\detokenize{deepdrr:deepdrr.Volume.translate}}
\pysigstartsignatures
\pysiglinewithargsret{\sphinxbfcode{\sphinxupquote{translate}}}{\sphinxparam{\DUrole{n,n}{t}\DUrole{p,p}{:}\DUrole{w,w}{  }\DUrole{n,n}{{\hyperref[\detokenize{deepdrr.geo:deepdrr.geo.core.Vector3D}]{\sphinxcrossref{Vector3D}}}}}}{{ $\rightarrow$ {\hyperref[\detokenize{deepdrr.vol:deepdrr.vol.volume.Volume}]{\sphinxcrossref{Volume}}}}}
\pysigstopsignatures
\sphinxAtStartPar
Translate the volume by \sphinxtitleref{t}.
\begin{quote}\begin{description}
\sphinxlineitem{Parameters}
\sphinxAtStartPar
\sphinxstyleliteralstrong{\sphinxupquote{t}} ({\hyperref[\detokenize{deepdrr.geo:deepdrr.geo.Vector3D}]{\sphinxcrossref{\sphinxstyleliteralemphasis{\sphinxupquote{geo.Vector3D}}}}}) \textendash{} The vector to translate by, in world space.

\end{description}\end{quote}

\end{fulllineitems}

\index{translate\_center\_to() (deepdrr.Volume method)@\spxentry{translate\_center\_to()}\spxextra{deepdrr.Volume method}}

\begin{fulllineitems}
\phantomsection\label{\detokenize{deepdrr:deepdrr.Volume.translate_center_to}}
\pysigstartsignatures
\pysiglinewithargsret{\sphinxbfcode{\sphinxupquote{translate\_center\_to}}}{\sphinxparam{\DUrole{n,n}{x}\DUrole{p,p}{:}\DUrole{w,w}{  }\DUrole{n,n}{{\hyperref[\detokenize{deepdrr.geo:deepdrr.geo.core.Point3D}]{\sphinxcrossref{Point3D}}}}}}{{ $\rightarrow$ None}}
\pysigstopsignatures
\sphinxAtStartPar
Translate the volume so that its center is located at world\sphinxhyphen{}space point x.

\sphinxAtStartPar
Only changes the translation elements of the world\_from\_anatomical transform. Preserves the current rotation of the
\begin{quote}\begin{description}
\sphinxlineitem{Parameters}
\sphinxAtStartPar
\sphinxstyleliteralstrong{\sphinxupquote{x}} ({\hyperref[\detokenize{deepdrr.geo:deepdrr.geo.Point3D}]{\sphinxcrossref{\sphinxstyleliteralemphasis{\sphinxupquote{geo.Point3D}}}}}) \textendash{} the world\sphinxhyphen{}space point.

\end{description}\end{quote}

\end{fulllineitems}

\index{world\_from\_IJK (deepdrr.Volume property)@\spxentry{world\_from\_IJK}\spxextra{deepdrr.Volume property}}

\begin{fulllineitems}
\phantomsection\label{\detokenize{deepdrr:deepdrr.Volume.world_from_IJK}}
\pysigstartsignatures
\pysigline{\sphinxbfcode{\sphinxupquote{property\DUrole{w,w}{  }}}\sphinxbfcode{\sphinxupquote{world\_from\_IJK}}\sphinxbfcode{\sphinxupquote{\DUrole{p,p}{:}\DUrole{w,w}{  }{\hyperref[\detokenize{deepdrr.geo:deepdrr.geo.core.FrameTransform}]{\sphinxcrossref{FrameTransform}}}}}}
\pysigstopsignatures
\end{fulllineitems}

\index{world\_from\_anatomical (deepdrr.Volume attribute)@\spxentry{world\_from\_anatomical}\spxextra{deepdrr.Volume attribute}}

\begin{fulllineitems}
\phantomsection\label{\detokenize{deepdrr:deepdrr.Volume.world_from_anatomical}}
\pysigstartsignatures
\pysigline{\sphinxbfcode{\sphinxupquote{world\_from\_anatomical}}\sphinxbfcode{\sphinxupquote{\DUrole{p,p}{:}\DUrole{w,w}{  }{\hyperref[\detokenize{deepdrr.geo:deepdrr.geo.core.FrameTransform}]{\sphinxcrossref{FrameTransform}}}}}}
\pysigstopsignatures
\end{fulllineitems}

\index{world\_from\_ijk (deepdrr.Volume property)@\spxentry{world\_from\_ijk}\spxextra{deepdrr.Volume property}}

\begin{fulllineitems}
\phantomsection\label{\detokenize{deepdrr:deepdrr.Volume.world_from_ijk}}
\pysigstartsignatures
\pysigline{\sphinxbfcode{\sphinxupquote{property\DUrole{w,w}{  }}}\sphinxbfcode{\sphinxupquote{world\_from\_ijk}}\sphinxbfcode{\sphinxupquote{\DUrole{p,p}{:}\DUrole{w,w}{  }{\hyperref[\detokenize{deepdrr.geo:deepdrr.geo.core.FrameTransform}]{\sphinxcrossref{FrameTransform}}}}}}
\pysigstopsignatures
\end{fulllineitems}


\end{fulllineitems}

\index{setup\_log() (in module deepdrr)@\spxentry{setup\_log()}\spxextra{in module deepdrr}}

\begin{fulllineitems}
\phantomsection\label{\detokenize{deepdrr:deepdrr.setup_log}}
\pysigstartsignatures
\pysiglinewithargsret{\sphinxcode{\sphinxupquote{deepdrr.}}\sphinxbfcode{\sphinxupquote{setup\_log}}}{}{}
\pysigstopsignatures
\end{fulllineitems}


\begin{sphinxthebibliography}{PointGro}
\bibitem[PointGroups]{deepdrr.geo:pointgroups}
\sphinxAtStartPar
\sphinxhref{https://en.wikipedia.org/wiki/Point\_groups\_in\_three\_dimensions}{Point groups}
on Wikipedia.
\end{sphinxthebibliography}


\renewcommand{\indexname}{Python Module Index}
\begin{sphinxtheindex}
\let\bigletter\sphinxstyleindexlettergroup
\bigletter{d}
\item\relax\sphinxstyleindexentry{deepdrr}\sphinxstyleindexpageref{deepdrr:\detokenize{module-deepdrr}}
\item\relax\sphinxstyleindexentry{deepdrr.annotations}\sphinxstyleindexpageref{deepdrr.annotations:\detokenize{module-deepdrr.annotations}}
\item\relax\sphinxstyleindexentry{deepdrr.annotations.fiducials}\sphinxstyleindexpageref{deepdrr.annotations:\detokenize{module-deepdrr.annotations.fiducials}}
\item\relax\sphinxstyleindexentry{deepdrr.annotations.line\_annotation}\sphinxstyleindexpageref{deepdrr.annotations:\detokenize{module-deepdrr.annotations.line_annotation}}
\item\relax\sphinxstyleindexentry{deepdrr.device}\sphinxstyleindexpageref{deepdrr.device:\detokenize{module-deepdrr.device}}
\item\relax\sphinxstyleindexentry{deepdrr.device.carm}\sphinxstyleindexpageref{deepdrr.device:\detokenize{module-deepdrr.device.carm}}
\item\relax\sphinxstyleindexentry{deepdrr.device.device}\sphinxstyleindexpageref{deepdrr.device:\detokenize{module-deepdrr.device.device}}
\item\relax\sphinxstyleindexentry{deepdrr.device.mobile\_carm}\sphinxstyleindexpageref{deepdrr.device:\detokenize{module-deepdrr.device.mobile_carm}}
\item\relax\sphinxstyleindexentry{deepdrr.device.simple\_device}\sphinxstyleindexpageref{deepdrr.device:\detokenize{module-deepdrr.device.simple_device}}
\item\relax\sphinxstyleindexentry{deepdrr.downsample\_tool}\sphinxstyleindexpageref{deepdrr:\detokenize{module-deepdrr.downsample_tool}}
\item\relax\sphinxstyleindexentry{deepdrr.geo}\sphinxstyleindexpageref{deepdrr.geo:\detokenize{module-deepdrr.geo}}
\item\relax\sphinxstyleindexentry{deepdrr.geo.core}\sphinxstyleindexpageref{deepdrr.geo:\detokenize{module-deepdrr.geo.core}}
\item\relax\sphinxstyleindexentry{deepdrr.geo.exceptions}\sphinxstyleindexpageref{deepdrr.geo:\detokenize{module-deepdrr.geo.exceptions}}
\item\relax\sphinxstyleindexentry{deepdrr.geo.hyperplane}\sphinxstyleindexpageref{deepdrr.geo:\detokenize{module-deepdrr.geo.hyperplane}}
\item\relax\sphinxstyleindexentry{deepdrr.geo.random}\sphinxstyleindexpageref{deepdrr.geo:\detokenize{module-deepdrr.geo.random}}
\item\relax\sphinxstyleindexentry{deepdrr.geo.ray}\sphinxstyleindexpageref{deepdrr.geo:\detokenize{module-deepdrr.geo.ray}}
\item\relax\sphinxstyleindexentry{deepdrr.geo.segment}\sphinxstyleindexpageref{deepdrr.geo:\detokenize{module-deepdrr.geo.segment}}
\item\relax\sphinxstyleindexentry{deepdrr.geo.typing}\sphinxstyleindexpageref{deepdrr.geo:\detokenize{module-deepdrr.geo.typing}}
\item\relax\sphinxstyleindexentry{deepdrr.geo.utils}\sphinxstyleindexpageref{deepdrr.geo:\detokenize{module-deepdrr.geo.utils}}
\item\relax\sphinxstyleindexentry{deepdrr.instruments}\sphinxstyleindexpageref{deepdrr.instruments:\detokenize{module-deepdrr.instruments}}
\item\relax\sphinxstyleindexentry{deepdrr.instruments.base}\sphinxstyleindexpageref{deepdrr.instruments:\detokenize{module-deepdrr.instruments.base}}
\item\relax\sphinxstyleindexentry{deepdrr.load\_dicom}\sphinxstyleindexpageref{deepdrr:\detokenize{module-deepdrr.load_dicom}}
\item\relax\sphinxstyleindexentry{deepdrr.load\_dicom\_tool}\sphinxstyleindexpageref{deepdrr:\detokenize{module-deepdrr.load_dicom_tool}}
\item\relax\sphinxstyleindexentry{deepdrr.logging}\sphinxstyleindexpageref{deepdrr:\detokenize{module-deepdrr.logging}}
\item\relax\sphinxstyleindexentry{deepdrr.network\_segmentation}\sphinxstyleindexpageref{deepdrr:\detokenize{module-deepdrr.network_segmentation}}
\item\relax\sphinxstyleindexentry{deepdrr.projector}\sphinxstyleindexpageref{deepdrr.projector:\detokenize{module-deepdrr.projector}}
\item\relax\sphinxstyleindexentry{deepdrr.projector.analytic\_generators}\sphinxstyleindexpageref{deepdrr.projector:\detokenize{module-deepdrr.projector.analytic_generators}}
\item\relax\sphinxstyleindexentry{deepdrr.projector.conv\_to\_mcgpu}\sphinxstyleindexpageref{deepdrr.projector:\detokenize{module-deepdrr.projector.conv_to_mcgpu}}
\item\relax\sphinxstyleindexentry{deepdrr.projector.cuda\_scatter\_structs}\sphinxstyleindexpageref{deepdrr.projector:\detokenize{module-deepdrr.projector.cuda_scatter_structs}}
\item\relax\sphinxstyleindexentry{deepdrr.projector.mass\_attenuation}\sphinxstyleindexpageref{deepdrr.projector:\detokenize{module-deepdrr.projector.mass_attenuation}}
\item\relax\sphinxstyleindexentry{deepdrr.projector.material\_coefficients}\sphinxstyleindexpageref{deepdrr.projector:\detokenize{module-deepdrr.projector.material_coefficients}}
\item\relax\sphinxstyleindexentry{deepdrr.projector.mcgpu\_compton\_data}\sphinxstyleindexpageref{deepdrr.projector:\detokenize{module-deepdrr.projector.mcgpu_compton_data}}
\item\relax\sphinxstyleindexentry{deepdrr.projector.mcgpu\_density}\sphinxstyleindexpageref{deepdrr.projector:\detokenize{module-deepdrr.projector.mcgpu_density}}
\item\relax\sphinxstyleindexentry{deepdrr.projector.mcgpu\_incoherent\_scatter\_data}\sphinxstyleindexpageref{deepdrr.projector.mcgpu_incoherent_scatter_data:\detokenize{module-deepdrr.projector.mcgpu_incoherent_scatter_data}}
\item\relax\sphinxstyleindexentry{deepdrr.projector.mcgpu\_incoherent\_scatter\_data.adipose\_compton\_data}\sphinxstyleindexpageref{deepdrr.projector.mcgpu_incoherent_scatter_data:\detokenize{module-deepdrr.projector.mcgpu_incoherent_scatter_data.adipose_compton_data}}
\item\relax\sphinxstyleindexentry{deepdrr.projector.mcgpu\_incoherent\_scatter\_data.air\_compton\_data}\sphinxstyleindexpageref{deepdrr.projector.mcgpu_incoherent_scatter_data:\detokenize{module-deepdrr.projector.mcgpu_incoherent_scatter_data.air_compton_data}}
\item\relax\sphinxstyleindexentry{deepdrr.projector.mcgpu\_incoherent\_scatter\_data.blood\_compton\_data}\sphinxstyleindexpageref{deepdrr.projector.mcgpu_incoherent_scatter_data:\detokenize{module-deepdrr.projector.mcgpu_incoherent_scatter_data.blood_compton_data}}
\item\relax\sphinxstyleindexentry{deepdrr.projector.mcgpu\_incoherent\_scatter\_data.bone\_compton\_data}\sphinxstyleindexpageref{deepdrr.projector.mcgpu_incoherent_scatter_data:\detokenize{module-deepdrr.projector.mcgpu_incoherent_scatter_data.bone_compton_data}}
\item\relax\sphinxstyleindexentry{deepdrr.projector.mcgpu\_incoherent\_scatter\_data.brain\_compton\_data}\sphinxstyleindexpageref{deepdrr.projector.mcgpu_incoherent_scatter_data:\detokenize{module-deepdrr.projector.mcgpu_incoherent_scatter_data.brain_compton_data}}
\item\relax\sphinxstyleindexentry{deepdrr.projector.mcgpu\_incoherent\_scatter\_data.breast\_compton\_data}\sphinxstyleindexpageref{deepdrr.projector.mcgpu_incoherent_scatter_data:\detokenize{module-deepdrr.projector.mcgpu_incoherent_scatter_data.breast_compton_data}}
\item\relax\sphinxstyleindexentry{deepdrr.projector.mcgpu\_incoherent\_scatter\_data.cartilage\_compton\_data}\sphinxstyleindexpageref{deepdrr.projector.mcgpu_incoherent_scatter_data:\detokenize{module-deepdrr.projector.mcgpu_incoherent_scatter_data.cartilage_compton_data}}
\item\relax\sphinxstyleindexentry{deepdrr.projector.mcgpu\_incoherent\_scatter\_data.connective\_compton\_data}\sphinxstyleindexpageref{deepdrr.projector.mcgpu_incoherent_scatter_data:\detokenize{module-deepdrr.projector.mcgpu_incoherent_scatter_data.connective_compton_data}}
\item\relax\sphinxstyleindexentry{deepdrr.projector.mcgpu\_incoherent\_scatter\_data.glands\_others\_compton\_data}\sphinxstyleindexpageref{deepdrr.projector.mcgpu_incoherent_scatter_data:\detokenize{module-deepdrr.projector.mcgpu_incoherent_scatter_data.glands_others_compton_data}}
\item\relax\sphinxstyleindexentry{deepdrr.projector.mcgpu\_incoherent\_scatter\_data.liver\_compton\_data}\sphinxstyleindexpageref{deepdrr.projector.mcgpu_incoherent_scatter_data:\detokenize{module-deepdrr.projector.mcgpu_incoherent_scatter_data.liver_compton_data}}
\item\relax\sphinxstyleindexentry{deepdrr.projector.mcgpu\_incoherent\_scatter\_data.lung\_compton\_data}\sphinxstyleindexpageref{deepdrr.projector.mcgpu_incoherent_scatter_data:\detokenize{module-deepdrr.projector.mcgpu_incoherent_scatter_data.lung_compton_data}}
\item\relax\sphinxstyleindexentry{deepdrr.projector.mcgpu\_incoherent\_scatter\_data.muscle\_compton\_data}\sphinxstyleindexpageref{deepdrr.projector.mcgpu_incoherent_scatter_data:\detokenize{module-deepdrr.projector.mcgpu_incoherent_scatter_data.muscle_compton_data}}
\item\relax\sphinxstyleindexentry{deepdrr.projector.mcgpu\_incoherent\_scatter\_data.PMMA\_compton\_data}\sphinxstyleindexpageref{deepdrr.projector.mcgpu_incoherent_scatter_data:\detokenize{module-deepdrr.projector.mcgpu_incoherent_scatter_data.PMMA_compton_data}}
\item\relax\sphinxstyleindexentry{deepdrr.projector.mcgpu\_incoherent\_scatter\_data.red\_marrow\_compton\_data}\sphinxstyleindexpageref{deepdrr.projector.mcgpu_incoherent_scatter_data:\detokenize{module-deepdrr.projector.mcgpu_incoherent_scatter_data.red_marrow_compton_data}}
\item\relax\sphinxstyleindexentry{deepdrr.projector.mcgpu\_incoherent\_scatter\_data.skin\_compton\_data}\sphinxstyleindexpageref{deepdrr.projector.mcgpu_incoherent_scatter_data:\detokenize{module-deepdrr.projector.mcgpu_incoherent_scatter_data.skin_compton_data}}
\item\relax\sphinxstyleindexentry{deepdrr.projector.mcgpu\_incoherent\_scatter\_data.soft\_tissue\_compton\_data}\sphinxstyleindexpageref{deepdrr.projector.mcgpu_incoherent_scatter_data:\detokenize{module-deepdrr.projector.mcgpu_incoherent_scatter_data.soft_tissue_compton_data}}
\item\relax\sphinxstyleindexentry{deepdrr.projector.mcgpu\_incoherent\_scatter\_data.stomach\_intestines\_compton\_data}\sphinxstyleindexpageref{deepdrr.projector.mcgpu_incoherent_scatter_data:\detokenize{module-deepdrr.projector.mcgpu_incoherent_scatter_data.stomach_intestines_compton_data}}
\item\relax\sphinxstyleindexentry{deepdrr.projector.mcgpu\_incoherent\_scatter\_data.titanium\_compton\_data}\sphinxstyleindexpageref{deepdrr.projector.mcgpu_incoherent_scatter_data:\detokenize{module-deepdrr.projector.mcgpu_incoherent_scatter_data.titanium_compton_data}}
\item\relax\sphinxstyleindexentry{deepdrr.projector.mcgpu\_incoherent\_scatter\_data.water\_compton\_data}\sphinxstyleindexpageref{deepdrr.projector.mcgpu_incoherent_scatter_data:\detokenize{module-deepdrr.projector.mcgpu_incoherent_scatter_data.water_compton_data}}
\item\relax\sphinxstyleindexentry{deepdrr.projector.mcgpu\_mean\_free\_path\_data}\sphinxstyleindexpageref{deepdrr.projector.mcgpu_mean_free_path_data:\detokenize{module-deepdrr.projector.mcgpu_mean_free_path_data}}
\item\relax\sphinxstyleindexentry{deepdrr.projector.mcgpu\_mean\_free\_path\_data.adipose\_mfp}\sphinxstyleindexpageref{deepdrr.projector.mcgpu_mean_free_path_data:\detokenize{module-deepdrr.projector.mcgpu_mean_free_path_data.adipose_mfp}}
\item\relax\sphinxstyleindexentry{deepdrr.projector.mcgpu\_mean\_free\_path\_data.air\_mfp}\sphinxstyleindexpageref{deepdrr.projector.mcgpu_mean_free_path_data:\detokenize{module-deepdrr.projector.mcgpu_mean_free_path_data.air_mfp}}
\item\relax\sphinxstyleindexentry{deepdrr.projector.mcgpu\_mean\_free\_path\_data.blood\_mfp}\sphinxstyleindexpageref{deepdrr.projector.mcgpu_mean_free_path_data:\detokenize{module-deepdrr.projector.mcgpu_mean_free_path_data.blood_mfp}}
\item\relax\sphinxstyleindexentry{deepdrr.projector.mcgpu\_mean\_free\_path\_data.bone\_mfp}\sphinxstyleindexpageref{deepdrr.projector.mcgpu_mean_free_path_data:\detokenize{module-deepdrr.projector.mcgpu_mean_free_path_data.bone_mfp}}
\item\relax\sphinxstyleindexentry{deepdrr.projector.mcgpu\_mean\_free\_path\_data.brain\_mfp}\sphinxstyleindexpageref{deepdrr.projector.mcgpu_mean_free_path_data:\detokenize{module-deepdrr.projector.mcgpu_mean_free_path_data.brain_mfp}}
\item\relax\sphinxstyleindexentry{deepdrr.projector.mcgpu\_mean\_free\_path\_data.breast\_mfp}\sphinxstyleindexpageref{deepdrr.projector.mcgpu_mean_free_path_data:\detokenize{module-deepdrr.projector.mcgpu_mean_free_path_data.breast_mfp}}
\item\relax\sphinxstyleindexentry{deepdrr.projector.mcgpu\_mean\_free\_path\_data.cartilage\_mfp}\sphinxstyleindexpageref{deepdrr.projector.mcgpu_mean_free_path_data:\detokenize{module-deepdrr.projector.mcgpu_mean_free_path_data.cartilage_mfp}}
\item\relax\sphinxstyleindexentry{deepdrr.projector.mcgpu\_mean\_free\_path\_data.connective\_mfp}\sphinxstyleindexpageref{deepdrr.projector.mcgpu_mean_free_path_data:\detokenize{module-deepdrr.projector.mcgpu_mean_free_path_data.connective_mfp}}
\item\relax\sphinxstyleindexentry{deepdrr.projector.mcgpu\_mean\_free\_path\_data.glands\_others\_mfp}\sphinxstyleindexpageref{deepdrr.projector.mcgpu_mean_free_path_data:\detokenize{module-deepdrr.projector.mcgpu_mean_free_path_data.glands_others_mfp}}
\item\relax\sphinxstyleindexentry{deepdrr.projector.mcgpu\_mean\_free\_path\_data.liver\_mfp}\sphinxstyleindexpageref{deepdrr.projector.mcgpu_mean_free_path_data:\detokenize{module-deepdrr.projector.mcgpu_mean_free_path_data.liver_mfp}}
\item\relax\sphinxstyleindexentry{deepdrr.projector.mcgpu\_mean\_free\_path\_data.lung\_mfp}\sphinxstyleindexpageref{deepdrr.projector.mcgpu_mean_free_path_data:\detokenize{module-deepdrr.projector.mcgpu_mean_free_path_data.lung_mfp}}
\item\relax\sphinxstyleindexentry{deepdrr.projector.mcgpu\_mean\_free\_path\_data.muscle\_mfp}\sphinxstyleindexpageref{deepdrr.projector.mcgpu_mean_free_path_data:\detokenize{module-deepdrr.projector.mcgpu_mean_free_path_data.muscle_mfp}}
\item\relax\sphinxstyleindexentry{deepdrr.projector.mcgpu\_mean\_free\_path\_data.PMMA\_mfp}\sphinxstyleindexpageref{deepdrr.projector.mcgpu_mean_free_path_data:\detokenize{module-deepdrr.projector.mcgpu_mean_free_path_data.PMMA_mfp}}
\item\relax\sphinxstyleindexentry{deepdrr.projector.mcgpu\_mean\_free\_path\_data.red\_marrow\_mfp}\sphinxstyleindexpageref{deepdrr.projector.mcgpu_mean_free_path_data:\detokenize{module-deepdrr.projector.mcgpu_mean_free_path_data.red_marrow_mfp}}
\item\relax\sphinxstyleindexentry{deepdrr.projector.mcgpu\_mean\_free\_path\_data.skin\_mfp}\sphinxstyleindexpageref{deepdrr.projector.mcgpu_mean_free_path_data:\detokenize{module-deepdrr.projector.mcgpu_mean_free_path_data.skin_mfp}}
\item\relax\sphinxstyleindexentry{deepdrr.projector.mcgpu\_mean\_free\_path\_data.soft\_tissue\_mfp}\sphinxstyleindexpageref{deepdrr.projector.mcgpu_mean_free_path_data:\detokenize{module-deepdrr.projector.mcgpu_mean_free_path_data.soft_tissue_mfp}}
\item\relax\sphinxstyleindexentry{deepdrr.projector.mcgpu\_mean\_free\_path\_data.stomach\_intestines\_mfp}\sphinxstyleindexpageref{deepdrr.projector.mcgpu_mean_free_path_data:\detokenize{module-deepdrr.projector.mcgpu_mean_free_path_data.stomach_intestines_mfp}}
\item\relax\sphinxstyleindexentry{deepdrr.projector.mcgpu\_mean\_free\_path\_data.titanium\_mfp}\sphinxstyleindexpageref{deepdrr.projector.mcgpu_mean_free_path_data:\detokenize{module-deepdrr.projector.mcgpu_mean_free_path_data.titanium_mfp}}
\item\relax\sphinxstyleindexentry{deepdrr.projector.mcgpu\_mean\_free\_path\_data.water\_mfp}\sphinxstyleindexpageref{deepdrr.projector.mcgpu_mean_free_path_data:\detokenize{module-deepdrr.projector.mcgpu_mean_free_path_data.water_mfp}}
\item\relax\sphinxstyleindexentry{deepdrr.projector.mcgpu\_mfp\_data}\sphinxstyleindexpageref{deepdrr.projector:\detokenize{module-deepdrr.projector.mcgpu_mfp_data}}
\item\relax\sphinxstyleindexentry{deepdrr.projector.mcgpu\_rita\_params}\sphinxstyleindexpageref{deepdrr.projector.mcgpu_rita_params:\detokenize{module-deepdrr.projector.mcgpu_rita_params}}
\item\relax\sphinxstyleindexentry{deepdrr.projector.mcgpu\_rita\_params.adipose\_rita\_params}\sphinxstyleindexpageref{deepdrr.projector.mcgpu_rita_params:\detokenize{module-deepdrr.projector.mcgpu_rita_params.adipose_rita_params}}
\item\relax\sphinxstyleindexentry{deepdrr.projector.mcgpu\_rita\_params.air\_rita\_params}\sphinxstyleindexpageref{deepdrr.projector.mcgpu_rita_params:\detokenize{module-deepdrr.projector.mcgpu_rita_params.air_rita_params}}
\item\relax\sphinxstyleindexentry{deepdrr.projector.mcgpu\_rita\_params.blood\_rita\_params}\sphinxstyleindexpageref{deepdrr.projector.mcgpu_rita_params:\detokenize{module-deepdrr.projector.mcgpu_rita_params.blood_rita_params}}
\item\relax\sphinxstyleindexentry{deepdrr.projector.mcgpu\_rita\_params.bone\_rita\_params}\sphinxstyleindexpageref{deepdrr.projector.mcgpu_rita_params:\detokenize{module-deepdrr.projector.mcgpu_rita_params.bone_rita_params}}
\item\relax\sphinxstyleindexentry{deepdrr.projector.mcgpu\_rita\_params.brain\_rita\_params}\sphinxstyleindexpageref{deepdrr.projector.mcgpu_rita_params:\detokenize{module-deepdrr.projector.mcgpu_rita_params.brain_rita_params}}
\item\relax\sphinxstyleindexentry{deepdrr.projector.mcgpu\_rita\_params.breast\_rita\_params}\sphinxstyleindexpageref{deepdrr.projector.mcgpu_rita_params:\detokenize{module-deepdrr.projector.mcgpu_rita_params.breast_rita_params}}
\item\relax\sphinxstyleindexentry{deepdrr.projector.mcgpu\_rita\_params.cartilage\_rita\_params}\sphinxstyleindexpageref{deepdrr.projector.mcgpu_rita_params:\detokenize{module-deepdrr.projector.mcgpu_rita_params.cartilage_rita_params}}
\item\relax\sphinxstyleindexentry{deepdrr.projector.mcgpu\_rita\_params.connective\_rita\_params}\sphinxstyleindexpageref{deepdrr.projector.mcgpu_rita_params:\detokenize{module-deepdrr.projector.mcgpu_rita_params.connective_rita_params}}
\item\relax\sphinxstyleindexentry{deepdrr.projector.mcgpu\_rita\_params.glands\_others\_rita\_params}\sphinxstyleindexpageref{deepdrr.projector.mcgpu_rita_params:\detokenize{module-deepdrr.projector.mcgpu_rita_params.glands_others_rita_params}}
\item\relax\sphinxstyleindexentry{deepdrr.projector.mcgpu\_rita\_params.liver\_rita\_params}\sphinxstyleindexpageref{deepdrr.projector.mcgpu_rita_params:\detokenize{module-deepdrr.projector.mcgpu_rita_params.liver_rita_params}}
\item\relax\sphinxstyleindexentry{deepdrr.projector.mcgpu\_rita\_params.lung\_rita\_params}\sphinxstyleindexpageref{deepdrr.projector.mcgpu_rita_params:\detokenize{module-deepdrr.projector.mcgpu_rita_params.lung_rita_params}}
\item\relax\sphinxstyleindexentry{deepdrr.projector.mcgpu\_rita\_params.muscle\_rita\_params}\sphinxstyleindexpageref{deepdrr.projector.mcgpu_rita_params:\detokenize{module-deepdrr.projector.mcgpu_rita_params.muscle_rita_params}}
\item\relax\sphinxstyleindexentry{deepdrr.projector.mcgpu\_rita\_params.PMMA\_rita\_params}\sphinxstyleindexpageref{deepdrr.projector.mcgpu_rita_params:\detokenize{module-deepdrr.projector.mcgpu_rita_params.PMMA_rita_params}}
\item\relax\sphinxstyleindexentry{deepdrr.projector.mcgpu\_rita\_params.red\_marrow\_rita\_params}\sphinxstyleindexpageref{deepdrr.projector.mcgpu_rita_params:\detokenize{module-deepdrr.projector.mcgpu_rita_params.red_marrow_rita_params}}
\item\relax\sphinxstyleindexentry{deepdrr.projector.mcgpu\_rita\_params.skin\_rita\_params}\sphinxstyleindexpageref{deepdrr.projector.mcgpu_rita_params:\detokenize{module-deepdrr.projector.mcgpu_rita_params.skin_rita_params}}
\item\relax\sphinxstyleindexentry{deepdrr.projector.mcgpu\_rita\_params.soft\_tissue\_rita\_params}\sphinxstyleindexpageref{deepdrr.projector.mcgpu_rita_params:\detokenize{module-deepdrr.projector.mcgpu_rita_params.soft_tissue_rita_params}}
\item\relax\sphinxstyleindexentry{deepdrr.projector.mcgpu\_rita\_params.stomach\_intestines\_rita\_params}\sphinxstyleindexpageref{deepdrr.projector.mcgpu_rita_params:\detokenize{module-deepdrr.projector.mcgpu_rita_params.stomach_intestines_rita_params}}
\item\relax\sphinxstyleindexentry{deepdrr.projector.mcgpu\_rita\_params.titanium\_rita\_params}\sphinxstyleindexpageref{deepdrr.projector.mcgpu_rita_params:\detokenize{module-deepdrr.projector.mcgpu_rita_params.titanium_rita_params}}
\item\relax\sphinxstyleindexentry{deepdrr.projector.mcgpu\_rita\_params.water\_rita\_params}\sphinxstyleindexpageref{deepdrr.projector.mcgpu_rita_params:\detokenize{module-deepdrr.projector.mcgpu_rita_params.water_rita_params}}
\item\relax\sphinxstyleindexentry{deepdrr.projector.mcgpu\_rita\_samplers}\sphinxstyleindexpageref{deepdrr.projector:\detokenize{module-deepdrr.projector.mcgpu_rita_samplers}}
\item\relax\sphinxstyleindexentry{deepdrr.projector.plane\_surface}\sphinxstyleindexpageref{deepdrr.projector:\detokenize{module-deepdrr.projector.plane_surface}}
\item\relax\sphinxstyleindexentry{deepdrr.projector.projector}\sphinxstyleindexpageref{deepdrr.projector:\detokenize{module-deepdrr.projector.projector}}
\item\relax\sphinxstyleindexentry{deepdrr.projector.rita}\sphinxstyleindexpageref{deepdrr.projector:\detokenize{module-deepdrr.projector.rita}}
\item\relax\sphinxstyleindexentry{deepdrr.projector.scatter}\sphinxstyleindexpageref{deepdrr.projector:\detokenize{module-deepdrr.projector.scatter}}
\item\relax\sphinxstyleindexentry{deepdrr.projector.spectral\_data}\sphinxstyleindexpageref{deepdrr.projector:\detokenize{module-deepdrr.projector.spectral_data}}
\item\relax\sphinxstyleindexentry{deepdrr.segmentation}\sphinxstyleindexpageref{deepdrr:\detokenize{module-deepdrr.segmentation}}
\item\relax\sphinxstyleindexentry{deepdrr.utils}\sphinxstyleindexpageref{deepdrr.utils:\detokenize{module-deepdrr.utils}}
\item\relax\sphinxstyleindexentry{deepdrr.utils.data\_utils}\sphinxstyleindexpageref{deepdrr.utils:\detokenize{module-deepdrr.utils.data_utils}}
\item\relax\sphinxstyleindexentry{deepdrr.utils.dicom\_utils}\sphinxstyleindexpageref{deepdrr.utils:\detokenize{module-deepdrr.utils.dicom_utils}}
\item\relax\sphinxstyleindexentry{deepdrr.utils.heatmap\_utils}\sphinxstyleindexpageref{deepdrr.utils:\detokenize{module-deepdrr.utils.heatmap_utils}}
\item\relax\sphinxstyleindexentry{deepdrr.utils.image\_utils}\sphinxstyleindexpageref{deepdrr.utils:\detokenize{module-deepdrr.utils.image_utils}}
\item\relax\sphinxstyleindexentry{deepdrr.utils.mesh\_utils}\sphinxstyleindexpageref{deepdrr.utils:\detokenize{module-deepdrr.utils.mesh_utils}}
\item\relax\sphinxstyleindexentry{deepdrr.utils.test\_utils}\sphinxstyleindexpageref{deepdrr.utils:\detokenize{module-deepdrr.utils.test_utils}}
\item\relax\sphinxstyleindexentry{deepdrr.vis}\sphinxstyleindexpageref{deepdrr:\detokenize{module-deepdrr.vis}}
\item\relax\sphinxstyleindexentry{deepdrr.vol}\sphinxstyleindexpageref{deepdrr.vol:\detokenize{module-deepdrr.vol}}
\item\relax\sphinxstyleindexentry{deepdrr.vol.kwire}\sphinxstyleindexpageref{deepdrr.vol:\detokenize{module-deepdrr.vol.kwire}}
\item\relax\sphinxstyleindexentry{deepdrr.vol.volume}\sphinxstyleindexpageref{deepdrr.vol:\detokenize{module-deepdrr.vol.volume}}
\end{sphinxtheindex}

\renewcommand{\indexname}{Index}
\printindex
\end{document}